\fancyhead[LO]{{\scriptsize 1950-1954: 由抗拒到认罪 · 糊纸盒}} %奇數頁眉的左邊
\fancyhead[RO]{} %奇數頁眉的右邊
\fancyhead[LE]{} %偶數頁眉的左邊
\fancyhead[RE]{{\scriptsize 1950-1954: 由抗拒到认罪 · 糊纸盒}} %偶數頁眉的右邊
\chapter*{糊纸盒}
\addcontentsline{toc}{chapter}{\hspace{1cm} 糊纸盒}
\thispagestyle{empty}
  一九五二年末,我们搬出了那所带铁栏杆的房子,住进房间宽敞的新居。这里有新板铺,有桌子、板凳,有明亮的窗户。我觉着所长说的“改造”,越发像是真的,加上我交代了那段历史之后,不但没受到惩办,反而受到了表扬,于是我便开始认真地学习起来。我当时的想法,认为改造就是念书;把书念会了,把书上的意思弄明白了,就算是改造成功了。我当时并没有想到,事情并不这么简单;改造并不能仅仅靠念书,书上的意思也并不单靠念一念就能明白。例如对于《什么叫封建社会》这本书,是我早在一九五零年底到一九五一年初念过的,但是如果我没有经过那一段劳动(生活和生产方面的劳动),我到现在也不会明白封建制度造了什么孽。什么叫封建社会?我在念了那本书的两年多之后,即一九五三年春天糊纸盒的时候,才真正找到了自己的答案。\\

一九五三年春,所方和哈尔滨一家铅笔厂联系好,由犯人们包糊一部分装铅笔的纸盒。从这时起,我们每天学习四个小时,劳动四个小时。所方说这是为了调剂一下我们的生活,又说,我们这些人从来没劳动过,干点活儿,会对我们有好处。这句话对我的特殊意义,是我当时完全意识不到的。\\

我从前不用说糊铅笔盒,就是削铅笔也没动过手。我对铅笔的有关知识至多是记得些商标图案——维纳斯牌是个缺胳臂的女人,施德楼牌是一只公鸡等等;我从来没留心它的盒子,更不知糊一个盒子要这么费事。我糊了不大功夫,起先感到的那点新鲜味全没有了,心里像也抹上浆糊似的,弄得胡里胡涂。别人糊出了好几个,我的一个仍拿不出手去,简直说不上是个盒子还是什么别的东西。\\

“你这是怎么糊的?”前伪满军医院长老宪把我的作品拿在手里端详着,“怎么打不开?这叫什么东西?”\\

老宪是肃亲王善耆的儿子,从小跟他的几个兄弟姊妹受日本浪人川岛浪速的教育。他在日本长大,学过医。金碧辉(日名川岛芳子)是他的妹妹,做过伪哈尔滨市长的金碧东是他的兄弟,一家满门都是亲日派汉奸。在苏联他跟我第一次见面,曾经跪在我面前哭着说:“奴才这可看见主人了!”现在跟我住在一起,却是最喜欢找我的碴儿。原因是他为人尖酸刻薄,又极容易跟人争执,却又争不过人,而我各方面都不如别人能干,向来没勇气和人争论,所以成了他的发泄对象。\\

我这时心里混合着妒嫉、失望和对于讥笑的担心,而老宪的多事偏又引起了人们的注意,纷纷过来围观那个作品,发出了讨厌的笑声。我走过去,一把从老宪手中夺下来,把它扔进了废料堆里。\\

“怎么?你这不是任意报废么?”老宪对我瞪起了眼。\\

“谁报废?我糊的差点,不见得就不能用。”我叽咕着,又从废料堆里把我的作品拣回来,把它放在成品堆里。这样一摆,就更显得不像样了。\\

“你放在哪里,也是个废品!”\\

听了他这句双关话,我气得几乎发抖。我一时控制不住,破例地回敬了一句:“你有本事对付我,真是欺软怕硬!”这句话碰了他的伤疤,他立刻红了脸,嚷道:“我欺谁?我怕谁?你还以为你是个皇上,别人都得捧着你才对吗?……”幸亏这时没有人理他,组长也出来阻止,他才没嚷下去。\\

可是事情并没有就此结束。老宪可不是个善罢甘休的人。\\

第二天糊纸盒的时候,老宪选了我旁边的一个位置坐下,从一开始糊起,总是用一种挑剔的眼光瞧我的活。我扭了一下身子,把后背给了他。\\

我这天的成绩,虽说比不上别人,总算有了些进步。到了晚上,所方用我们昨天生产所得的酬劳,买了些糖果发给我们。这是我头一次享受自己的劳动果实(虽然我的成绩是最次的),我觉得我分得的糖果,比过去任何一次吃到的都要甜。这时候,老宪说话了:\\

“溥仪今天成绩不坏吧?”\\

“还好,没有废品。”我顶撞地说。\\

“嘻,还是虚心些的好。”他的脸上皮笑肉不笑。\\

“说没有废品就算不虚心?”我心中直冒火,糖果也不觉着甜了。我最讨厌老宪的地方,就是他专爱挑人家高兴的时候找碴子。“如果再出废品,再随你扣帽子吧。”\\

我想堵他这一句就不再理他。不料他走到我那堆成品里顺手拿出了一个,当着众人举了起来说:\\

“请看!”\\

我抬头一看,几乎把嘴里的糖果吸到肺里去。原来我糊倒了标签。\\

我气极了,真想过去把那盒子抓过来扔到那张凹凸不平的脸上。我控制了自己,半晌只说了这么一句话:\\

“你想怎么就怎么吧!”\\

“喝,好大口气!还是臭皇帝架子。”他提高嗓门,“我对你批评,是对你好意。你不想一想。”他听见门外看守员的脚步声,嗓门更响了:“你还幻想将来当你的皇帝吧?”\\

“你简直胡说八道!”我激怒地回答,“我比你笨,不如你会说会做,我天生的不如你。这行了吧?”\\

别人都离开了座位,过来劝架。我们这时住的房间很大,一共有十八个人,除我之外,有三个伪大臣,十四个伪将官。组长是老韦,也是伪将官。张景惠是三名伪大臣之一,他老得糊涂,平时不学习、不劳动,也不爱说话。这天晚上除了张景惠之外,其余的都为了“纸盒事件”参与了议论。有人批评老宪说,既然是好意批评就不应大喊大叫地说话;有人批评我说,盒子糊坏了,就应承认,不该耍态度;蒙古族的老郭认为老宪的态度首先不好,不怪溥仪生气;向来和老宪要好的一个伪禁卫军团长则表示反对,说是老郭用“带色眼镜”看人;又有人说,这问题可以放到星期六的生活检讨会上去谈,一时七嘴八舌,彼此各不相让。正在闹得不可开交的时候,我看见“禁卫军团长”拉了吵得嘴角起沫的老宪衣襟一下,而且别人也都突然静了下来。我回头一看,原来管学习的李科员走了进来。\\

原先管学习的李科长,已经调走了,新来的这位又姓李,大家因为对从前那位叫惯了“学习主任”,所以现在对这位李科员也叫“学习主任”。他问组长大家吵什么,老韦说:\\

“报告主任,是由一个废纸盒引起的……”\\

李科员听完,把我糊倒标签的纸盒拿起来看了看,说道:\\

“这算是什么大事,值得争吵?标签倒了,在上面再糊个正的不就行了吗?”\\

李科员的这席话把大伙说得个个哑口无言。\\

事情这还不算完。\\

过了几天,负责分配纸盒材料的小瑞向我们转达,另外几组要发起一个劳动竞赛,问我们参加不参加。我们表示了响应。小瑞又告诉了一个消息,说小固在他们那个组里创造了一个用一道手续糊盒的“底盖一码成的快速糊盒法”,效率比以前提高了一倍还多。我们组里一听,觉得参加竞赛是不能用老办法了,得想个提高效率的新办法才行。那时我们常从报上看到关于技术革新创造的记载,如郝建秀工作法、流水作业法等等,有人从这方面得到了启发,提出了流水作业法,就是每人专搞一门专业,抹浆糊的专抹浆糊,粘盒帮的专粘盒帮,贴纸的专贴纸,糊标签的专糊标签,组成一道流水作业线。大家一致同意试试这办法,我也很高兴,因为这样分工序的办法,干的活儿比较简单,混在一起也容易遮丑。谁知道这样干了不久,问题就暴露出来了,在流水作业线里,东西到了我这儿很快地积压起来,水流不过去了。而且,这又是老宪发现的。\\

“由于个人的过失,影响了集体,这怎么办?”他故意表示很为难的样子。\\

这次我一句也没和他吵。我面对着一大叠等着糊亮光纸的半成品,像从前站在养心殿门外等着叫“起儿”的人们那样呆着。当我听到我下手工序的一个伙伴也说我的操作不合乎标准,废品率必然会提高的时候,我知道无论是公正的老郭,还是李科员出来,都不会反对老宪的挑剔了。结果是,我退出了流水作业线,另外去单干。\\

这是我和家里人分开之后,再一次感到了孤寂的滋味,而这次被排除出整体之外,好像脱光了身子站在众人面前,对比特别强烈,格外觉着难受。特别是老宪,那张橘皮脸上露出幸灾乐祸和报复的满足,走过我面前时还故意咳嗽一声,气得我的肺都要炸了。我很想找个同情者谈谈,但是组里每个人都是忙忙碌碌的,都没有谈话的兴趣。碰巧这时我又患了感冒,心里特别不痛快。\\

这天夜里,我做起了噩梦,梦见那张凹凸不平的橘皮脸直逼着我,恶狠狠地对我说:“你是个废物!你只能去当要饭花子!”接着我又梦见自己蹲在一座桥上,像童年时太监们向我描绘的“镇桥猴”那样。突然有个人伸出一只手压在我头上,把我惊醒过来。我在朦胧中看见一个穿白衣服的人立在我面前,用手摸我的脑门,说:“你发高烧,感冒加重了,不要紧,让我给你检查一下吧。”\\

我觉得头昏昏的,太阳穴的血管突突直跳,定了定神,才明白了是怎么回事。原来看守员发现我在说梦话,又说又闹,叫不醒我,就报告了看守长,看守长把军医温大夫找来了。大夫看过了体温计,护士给我注射了一针药。我渐渐睡着了,不知他们什么时候离去的。\\

我病了半个月,经过大夫、护士每天的治疗,渐渐恢复起来。在这半个月里,我每天大部时间睡在床上,不学习,不劳动,整天想心事。我在这半个月里想的比过去几年想的还多。我从纸盒一直回想到西太后那张吓得我大哭的脸。\\

我从前一回忆起那个模糊的印象,只觉得西太后很可怕,而现在,我觉得她可恨了。她为什么单单挑上我来当那皇帝呢?我本来是个无知的、纯洁的孩子,从任何方面来说,我至少不会比溥杰的天分还差,可是由于做了皇帝,在那密不通风的罐子中养大,连起码的生活知识也没有人教给我,我今天什么也不懂,什么也不会,我的知识、能力不但比不上溥杰,恐怕也比不上一个孩子。我受到人们的嘲笑,受到像老宪这样人的欺负,如果让我独自去生活,我真不知怎么能活下去。我今天弄成这样,不该西太后和那些王公大臣们负责吗?\\

我从前每逢听到别人笑我,或者由于被人指出自己无能,心里总是充满了怨恨,怨恨别人过于挑剔,甚至怨恨着把我关起来的人民政府,但我现在觉得这都不是应该怨恨的,事实证明我确实是可笑的、无能和无知的。从前我怨恨侄子们太不顾面子,把我的尊严竟全盘否定了,但我现在承认,实在没有什么可以给自己作脸的事。比如有一次吃包子,我觉得很香,王看守员问我:“你喜欢韭菜?”我说没吃过,不知道。别人都笑起来说:“你吃的不是韭菜吗?”既然我小到尝不出韭菜,大到迎“天照大神”代替自己的祖宗,我还有什么“圣明”?又如何能不让别人笑骂呢?蒙古人老正是民国初年发动蒙古叛乱的巴布扎布的儿子,有一天他对我说,当年他全家发过誓要为拥戴我复辟而死,他母亲简直拿我当神仙那么崇拜。他说:“真可惜,她已经死了,不然我一定要告诉她,宣统是个什么样的废物!”既然我本来不是神仙,我本来无能无知,又如何怪别人说这类话呢?\\

我只有怪西太后和那一伙人,只有怪我为什么生在那个圈子里。我对紫禁城发生了新的怨恨。我想到这里,觉得连老宪都算不上什么冤家了。\\

我差不多完全痊愈了,这天所长找我去谈话,问了我的身体情况,追问到我和老宪争吵的情形,问我是不是感到了什么刺激。我把经过简单地说了,最后说:\\

“我当时确实很受刺激,可是我现在倒不怎么气了,我只恨自己实在无能。我恨北京宫里的那些人。”\\

“很好,你已经认识到了自己的弱点,这是一个进步。无能,这不用发愁,只要你肯学,无能就会变成有能。你找到了无能的原因,这更重要。你还可以想想,从前的王公大臣那些人为什么那样教育你?”\\

“他们光为了他们自己。”我说,“不顾我,自私而已。”\\

“恐怕不完全如此,”所长笑着说,“你能说陈宝琛跟你父亲,是成心跟你过不去吗?是成心害你吗?”\\

我答不上来了。\\

“你可以慢慢想想这问题。如果明白了,那么你这场病就生得大有价值。”\\

从所长那里回来之后,我真的放不下这个问题了。到我参加病后的第一次生活检讨会时,我把过去的生活已经想了好几遍。我没有得到什么答案,怨气却越聚越多。\\

在这次生活检讨会上,有人批评了老宪,说他完全不是与人为善的态度,总是成心打击我。接着,差不多一半以上的人都对他发表了类似的意见,甚至有人把我生病的责任也放在他身上,并据以证明他在大家的改造中起了坏作用。老宪慌张了起来,脸色发灰,结结巴巴地做了检讨。我在会上一言没发,继续想着我的怨恨。有人提出,我应该发表一下意见。老宪的脸更加发灰了。\\

“我没什么意见,”我低声说,“我只恨我自己无能!”\\

大家一时都怔住了。老宪大大张开了嘴巴。我忽然放大了嗓音,像喊似地说:\\

“我恨!我恨我从小生长的地方!我恨那个鬼制度!什么叫封建社会?从小把人毁坏,这就是封建社会!”\\

我的嗓子突然被一阵痉挛埂住,说不下去了,别人唧唧哝哝地说什么,我也听不见了。……\\