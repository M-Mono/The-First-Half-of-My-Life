\fancyhead[LO]{{\scriptsize 1917-1924: 北京的“小朝廷” · 整顿内务府}} %奇數頁眉的左邊
\fancyhead[RO]{} %奇數頁眉的右邊
\fancyhead[LE]{} %偶數頁眉的左邊
\fancyhead[RE]{{\scriptsize 1917-1924: 北京的“小朝廷” · 整顿内务府}} %偶數頁眉的右邊
\chapter*{整顿内务府}
\addcontentsline{toc}{chapter}{\hspace{1cm}整顿内务府}
\thispagestyle{empty}
我遣散太监的举动,大受社会舆论的称赞和鼓励。在庄师傅的进一步指引下,我接着把“励精图治”的目标又转到内务府方面。\\

关于内务府,我想先抄一段内务府一位故人写给我的材料:\\

\begin{quote}
	内务府人多不读书\\

内务府人多不知书,且甚至以教子弟读书为播种灾祸者。察其出言则一意磨楞,观其接待则每多繁缛;视中饱如经逾格之恩,作舞弊如被特许之命。昌言无忌,自得洋洋。乃有“天棚鱼缸石榴树,地炕肥狗胖丫头”,以及“树小房新画不古,一看就知内务府”之讽,极形其鄙而多金,俗而无学也。余窃耻之,而苦不得采其源。追及民十七八之间,遍读东华录,在嘉庆朝某事故中(林清之变或成德之案,今不能清楚矣)发现有嘉庆之文字,略叙在清代中之背反者,其中有宗室有八旗有太监,而独无内务府人,足见内务府尚不辜负历代豢养之恩,较之他辈实为具有天良者。嘉庆之慨叹,实为内务府人之表彰。于是始得解惑焉。内务府人亦常有自谓“皇上家叫我们赚钱,就为的养活我们”,此语之来,必基于此矣。至其言语举动之不成文章者,正所以表其驯贴之愚,而绝无圭角之志;其畏读书,则为预避文祸之于触,与夫遗祸于后昆;其视舞弊及中饱如奉明言者,乃用符“不枉受历代优遇豢养之恩”也欤?……而内务府人之累代子孙亦为之贻误,乃至于此,曷胜叹哉!\\
\end{quote}

这位老先生当年由于家庭不许他升学深造,受过不少刺激,所以他对于内务府人不读书的感慨特别深。我那时对三旗世家所包办的内务府\footnote{在满清八旗中,镶黄、正黄、正自三个满军旗系皇室亲自率领的所谓亲军,内务府人均出自这最亲信的三旗,自堂郎中以下所有司员全不例外;堂郎中以上即内务府大臣,也有的是司员提上来的,也有的是从外调来的。总之,除个别大臣外,全被三旗包下来了。},最不满的还不是俗而无学,而是他们“视中饱舞弊,如奉明言”。\\

关于内务府中饱、舞弊的故事,在这里只举出两个例子就行了。一个是内务府每年的惊人开支,即使四百万元的优待费全部照付,也会人不敷出。民国十三年我出宫后,“清室善后委员会”在北京《京报》上揭露的当年收入抵押金银古玩款,达五百多万元,当年并无剩余,全部开支出去了。据前面那段文字的作者说,那几年每年开支都在三百六十万两上下,这是和《京报》上揭露的材料大体相符的。\\

另一个例子是我岳父荣源经手的一次抵押。抵押合同日期是民国十三年五月三十一日,签字人是内务府绍英、耆龄、荣源和北京盐业银行经理岳乾斋,抵押品是金编钟、金册、金宝和其他金器,抵押款数八十万元,期限一年,月息一分。合同内规定,四十万元由十六个金钟(共重十一万一千四百三十九两)做押品,另四十万元的押品则是:八个皇太后和五个皇后的金宝十个,金册十三个,以及金宝箱、金印池、金宝塔、金盘、金壶等,计重一万零九百六十九两七钱九分六厘,不足十成的金器三十六件,计重八百八十三两八钱,另加上嵌镶珍珠一千九百五十二颗,宝石一百八十四块,玛瑙碗等珍品四十五件。只这后一笔的四十万元抵押来说,就等于是把金宝金册等十成金的物件当做荒金折卖,其余的则完全白送。这样的抵押和变价,每年总要有好几宗,特别是逢年过节需要开销的时候。一到这时候,报上就会出现秘闻消息,也必有内务府辟谣或解释的声明。比如这一次抵押事先就有传闻,内务府和荣源本人也有声明,说所卖都是作废的东西,其中决没有传说中的慈禧的册宝云云\footnote{上面说的这个合同,见民国十四年二月十四日北京《京报》,关于事先的传闻和内务府与荣源的声明,见于十三年年底的《京报》。}。\\

我在出宫之前,虽然对内务府的中饱和舞弊拿不到像上面说的这样证据,但是,每年的“放过款项”的数字告诉了我一个事实:我的内务府的开支,竟超过了西太后的内务府的最高纪录。内务府给我写过一份叫做官统七年放过款项及近三年比较”的材料,是内务府为了应付清理财产的上谕而编造的(后面还要谈到这次清理),据他们自己的统计,除去了王公大臣的俸银不计,属于内务府开支的,民国四年是二百六十四万两,民国八、九、十年是二百三十八万两,一百八十九万两,一百七十一万两,而西太后时代的内务府,起先每年开支不过三十万两,到西太后过七十整寿时,也不过才加到七十万两,我这个人再不识数,也不能不觉得奇怪。同时我也注意到了这个事实:有些贵族、显宦之家已经坐吃山空,日趋潦倒,甚至于什么世子王孙倒毙城门洞,郡主、命妇坠入烟花等等新闻已出现在报纸社会栏内,而内务府人却开起了古玩店、票庄(钱庄)、当铺、木厂(营造业)等等大买卖。师傅们虽然帮助过内务府,反对我买汽车、安电话,可是一提起内务府这些事,谁也没有好感。伊克坦师傅在去世前(我结婚前一年)不久曾因为陈师傅不肯向我揭发内务府的弊端,说陈师傅犯了“欺君之罪”,不配当“太傅”。至于庄师傅就更不用说了,内务府在他看来就是“吸血鬼”的化身。他对内务府的看法促成了我整顿内务府的决心。\\

“从宫廷的内务府到每个王公的管家人,都是最有钱的。”他有一次说,“主人对自己的财产不知道,只有问这些管家的人,甚至于不得不求这些管家的人,否则就一个钱也拿不到。不必说恢复故物,就说手里的这点珍宝吧,如果不把管家的整顿好,也怕保不住!”\\

他又说:“内务府有个座右铭,这就是——维持现状!无论是一件小改革还是一个伟大的理想,碰到这个座右铭,全是——Stop(停车)!”\\

我的“车”早已由师傅们加足了油,而且开动了引擎。如果说以前是由别人替我驾驶着,那么现在则是我自己坐在司机座位上,向着一个理想目标开去。现在我刚刚胜利地开过“遣散太监”的路口,无论是谁叫我“停车”,也不行了。\\

我下了决心。我也找到了“力量”。\\

我在婚礼过去之后,最先运用我当家做主之权的,是从参加婚礼的遗老里,挑选了几个我认为最忠心的、最有才干的人,作为我的股肽之臣。被选中的又推荐了他们的好友,这样,紫禁城里一共增加了十二三条辫子。这就是:郑孝胥、罗振玉、景永昶、温肃、柯劭囗、杨锺羲、朱汝珍、王国维、商衍瀛等等。我分别给了他们“南书房(皇帝书房)行走”、“懋勤殿(管皇帝读书文具的地方)行走”的名衔。另外我还用了两名旗人,做过张学良老师的镶红旗蒙古副都统金梁和我的岳父荣源,派为内务府大臣。\\

他们那些动人的口头奏对都没留下纪录,他们写的条陈也一时找不全,现在把手头上一份金梁的条陈——日期是“宣统十六年正月”,即金梁当内务府大臣前两个月写的——抄下一段(原文中抬头和侧书都在此免了):\\

\begin{quote}
	臣意今日要事,以密图恢复为第一。恢复大计,旋乾转坤,经纬万端,当先保护宫廷,以团根本;其次清理财产,以维财政。盖必有以自养,然后有以自保,能自养自保,然后可密图恢复,三者相连,本为一事,不能分也。今请次第陈之:\\
	
	\begin{itemize}
		\item 筹清理。清理办法当分地产、宝物二类。\\
		\item 清地产,从北京及东三省入手,北京如内务府之官地、官房,西山之园地,二陵之余地、林地;东三省如奉天之盐滩、鱼池、果园,三陵庄地,内务府庄地,官山林地,吉林黑龙江之贡品各产地,旺清、楧木囗林,汤原鹏棚地,其中包有煤铁宝石等矿,但得其一,已足富国。是皆皇室财产,得人而理,皆可收回,或派专员放地招垦,或设公司合资兴业,酌看情形,随时拟办。……\\
		\item 清宝物,各殿所藏,分别清检,佳者永保,次者变价,既免零星典售之损,亦杜盗窃散失之虞。筹有巨款,预算用途,或存内库,或兴实业,当谋持久,勿任消耗。……此清理财产之大略也。\\
		\item 曰重保护。保护办法当分旧殿、古物二类。一、保古物,拟将宝物清理后,即请设皇室博览馆,移置尊藏,任人观览,并约东西各国博物馆,借赠古物,联络办理,中外一家,古物公有,自可绝人干涉。\\
		\item 保旧殿,拟即设博览馆于三殿,收回自办,三殿今成古迹,合保存古物古迹为一事,名正言顺,谁得觊觎。且此事既与友邦联络合办,遇有缓急,互相援助,即内廷安危,亦未尝不可倚以为重。……此保护官廷之大略也。\\
		\item 图恢复。恢复办法,务从缜密,当内自振奋而外示韬晦。求贤才、收人心、联友邦,以不动声色为主。求贤才,在勤延揽,则守旧维新不妨并用;收人心,在广宣传,则国间外论皆宜注意;联友邦,在通情谊,则赠聘酬答不必避嫌。至于恢复大计,心腹之臣运筹于内,忠贞之士效命于外。成则国家蒙其利,不成则一二人任其害。机事唯密,不能尽言……\\
	\end{itemize}

此密图恢复之大略也。\\
\end{quote}

金梁当了内务府大臣之后,又有奏折提出了所谓“自保自养二策,”他说“自养以理财为主,当从裁减人手,自保以得人为主,当从延揽人手”。“裁减之法,有应裁弊者,有应裁人者,有应裁款者”,总之,是先从内务府整顿着手。这是我完全赞同的做法。\\

除了这些最积极于“密图恢复”的人之外,就是那些态度消极悲观的遗老们,大多数也不反对“保护宫廷,清理财产”和裁人裁款裁弊。其中只有很小的一部分人,可以我的陈师傅为代表,一提到改革内务府的各种制度总是摇头的。这些人大抵认为内务府积弊已深,冰冻三尺,非一日之寒,从乾隆时代起,随着宫廷生活的日趋奢靡,即已造成这种局势,嘉庆和道光时代未尝不想整顿,但都办不到,现在更谈何容易?在陈师傅们看来,内务府不整顿还好,若整起来必然越整越坏;与其弄得小朝廷内部不安,不如暂且捺下,等到时来运转再说。但是像陈师傅这样的遗老,尽管不赞成整顿,却也并不说内务府的好话,甚至还可以守中立。\\

我在婚前不久,干过一次清理财产的傻事。那时根据庄土敦的建议,我决定组织一个机构,专门进行这项工作。我邀请庄士敦的好朋友、老洋务派李经迈来主持这件事,李不肯来,推荐了他一位姓刘的亲戚代替他。内务府并没有直接表示反对,曾搬出了我的父亲来拦阻。我没有理睬父亲的劝阻,坚持要委派李经迈的亲戚进行这件事,他们让了步,请刘上任。可是他干了不过三个月,就请了长假,回上海去了。\\

经过那次失败,我还没有看出内务府的神通。我把失败原因放在用人失当和我自己尚未“亲政”上面;那时正值政局急变,我几乎要逃到英使馆去,也无暇顾及此事。现在,我认为情形与前已大不相同,一则我已当家成人,任何人拦阻不了我,再则我身边有了一批人,力量强大了。我兴致勃勃地从这批人才里面,选出了郑孝胥来担当这件整顿重任。\\

郑孝胥是陈宝琛的同乡,在清朝做过驻日本神户的领事,做过一任广西边务督办。陈宝琛和庄士敦两位师傅过去都向我推崇过他,尤其是庄师傅的推崇最力,说郑孝胥是他在中国二十多年来最佩服的人,道德文章,全中国找不出第二位来,说到办事才干和魄力,没有比他更好的。陈师傅还告诉过我,郑孝胥曾多次拒绝民国总统的邀请,不肯做民国的官,不拿民国的钱。我从报纸上也看到过颂扬他的文字,说他十几年来以诗酒自娱,“持节不阿”,捧他为同光派诗人的后起之秀。他的书法我早看过,据说他鬻书笔润收入,日达千金。他既然放弃了功名利禄前来效力,可见是个难得的忠臣。\\

我和郑孝胥第一次见面是在民国十二年夏天。他从盘古开天辟地一直谈到未来的大清中兴,谈到高兴处,眉飞色舞,唾星乱飞,说到激昂慷慨处,声泪俱下,让我大为倾倒。我立时决定让他留下,请他施展他的抱负。我当时怎么说的已记不清了,只记得当时他听我谈完后大为感动,很快做出了一首“纪思诗”:\\

\begin{quote}
	君臣各辟世,世难谁能平?\\

天心有默启,惊人方一鸣。\\

落落数百言,肝脑输微诚。\\

使之尽所怀,日月悬殿楹。\\

进言何足异,知言乃圣明。\\

自意转沟壑,岂知复冠缨。\\

独抱忠义气,未免流俗轻。\\

须臾愿无死,终见德化成。\\
\end{quote}

郑孝胥成了“懋勤殿行走”之后,几次和我讲过要成大业,必先整顿内务府,并提出了比金梁的条陈更具体的整顿计划。按照这个计划,整个内务府的机构只要四个科就够了,大批的人要裁去,大批的开支要减去,不仅能杜绝流失,更有开源之策。总之,他的整顿计划如果能够实现,复辟首先就有了财务上的保证。因此我破格授这位汉大臣为总理内务府大臣,并且“掌管印钥”,为内务府大臣之首席。郑孝胥得到了我这破格提拔,又洋洋自得地做了两首诗:\\

\begin{quote}
	三月初十日夜值\\

大王事獯鬻,勾践亦事吴。\\

以此慰吾主,能屈诚丈夫。\\

一惭之不忍,而终身惭乎。\\

勿云情难堪,且复安须臾。\\

天命将安归,要观人所与。\\

苟能得一士,岂不胜多许。\\

狸首虽写形,聊以辟群鼠。\\

持危谁同心,相倚譬蛩驱。\\
\end{quote}

但是,如果认为俗而无学的内务府会败在郑孝胥的手里,那就把这有二百多年历史的宫廷管家衙门估计得太低了。尽管郑孝胥吹得天花乱坠,而且有我的支持和信赖,他的命运还是和李经迈的亲戚一样,也只干了三个月。\\

那些俗而无学的内务府人,究竟是谁把郑孝胥挤走的,我始终没有完全弄清楚。是绍英捣乱吗?可是绍英是出名的胆小怕事的人。是耆龄吗?耆龄是个不熟悉内务府差使的外行,一向不多问事。至于宝熙,来的时间很短,未必有那样大的神通。如果说一切都是下面的人自作主张,竟敢和郑大臣捣乱,也不全像。郑孝胥上任之后,遇见的第一件事,是面前出现了辛亥以来成堆的积案。郑孝胥对付的办法是先来个下马威,把原任堂郎中开除,把这个重要的位置抓过来,由他的亲信佟济煦接任。可是没想到,从此内务府就像瘫痪了一样,要钱,根本没钱——真的没有,账上是明明的这样记着:要东西,东西总是找不到存放的地方,账上也是这样记着……\\

郑孝胥为了拉拢下级司员,表示虚怀若谷,倾听下情,他规定每星期和司员们座谈一次,请司员们为改革出些主意。有一位司员建议说,宫中各处祭祀供品向例需用大批果品糕点,所费实在太大,其实只不过是个意思,不如用泥土和木雕的代替,一样的庄重。郑对这个主意大为赏识,下令执行,并且对出主意的人摆升一级。可是那些把供品作为自己合法收入的太监(裁减后还剩下百名左右),个个都把郑孝胥恨之入骨。郑孝胥上任没有几天,就成了紫禁城中最不得人心的人。\\

郑孝胥不想收兵,于是便接到了恐吓信。信上说:你正在绝人之路,你要当心脑袋。与此同时,被我派去整顿颐和园的庄士敦也接到了恐吓信。信上说:你如果敢去上任,路上就有人等着杀你。后来庄士敦很自得地对我说:“我也没坐车,偏骑马去,看他们敢不敢杀我,结果我活着到任了。我早看透了那些人!”他指的那些人就是内务府的人。他和郑孝胥对恐吓信都表示不在乎。\\

事情最后的收场,还是在我这里。\\

我刚刚任命了郑的差使,就得到了一个很头痛的消息:民国国会里又有一批议员提出了议案,要废止优待条件,由民国接收紫禁城。早在两年前,在国会里就有过这类提案,理由根据是清室在民国六年闹过复辟,现在又不断向民国官吏赐官赐爵赐谥,俨然驾于民国之上,显然图谋复辟。现在旧案重提,说我不但给复辟犯张勋谥法,更非法的是赏给汉人郑孝胥紫禁城骑马和援内务府大臣。\\

报纸上登出了这个消息,这个消息就像信号一样,攻击内务府的举动接二连三地出现了。如内务府出售古玩给日本商人,内务府大臣荣源把历代帝后册宝押进四大银行等等,这些过去本来不足为奇的事情,也引起了社会上啧啧烦言。\\

同时,在清点字画中,那些被我召集到身边的股肱之臣,特别是罗振玉,也遭到了物议。这些新增加的辫子们来到紫禁城里,本来没有别的事,除了左一个条陈,右一个密奏,陈说复兴大计之外,就是清点字画古玩,替我在清点过的字画上面盖上一个“宣统御览之宝”,登记上账。谁知这一清点,引起了满城风雨。当时我却不知道,不点还好,东西越点越少,而且给遗老们增辟了各种生财之道。罗振玉的散氏盘、毛公鼎的古铜器拓片,佟济煦的月罗版的宫中藏画集都卖了大价钱,轰动了中外。顶伤脑筋的是,民国的内务部突然颁布了针对清宫贩卖古物出口而定的“古籍、古物及古迹保存法草案”。\\

不久,郑孝胥的开源之策——想把四库全书运到上海商务印书馆出版,遭受当局的阻止,把书全部扣下了。\\

我父亲这时找到我,婉婉转转地,更加结结巴巴地向我说,郑孝胥的办法值得斟酌,如果连民国当局也不满意,以后可就更不好办了。\\

原来的那些内务府大臣绍英、耆龄、宝熙,还是那么恭顺,没有说出一句关于郑、金、荣三人的坏话。不过荣源因为卖册宝出了事,不露头了,金梁因为上的条陈里有劝我让醇亲王退休的话,被我父亲大骂一顿,也不知哪里去了。\\

这一天,绍英带着一副胆小怕事的样子出现在我面前,说现在的步军统领王怀庆对郑孝胥的做法很不满意,王怀庆说如果再叫郑孝胥闹下去,民国如果有什么举动,他就再没办法帮我的忙。一听这话,我才真怵了头。这时,郑孝胥“恳请开去差事”的奏折到了。结果是,郑孝胥回到“懋勤殿行走”,绍英依然又掌管了内务府印钥。\\