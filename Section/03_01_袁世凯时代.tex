\fancyhead[LO]{{\scriptsize 1917-1924: 北京的“小朝廷” · 袁世凯时代}} %奇數頁眉的左邊
\fancyhead[RO]{} %奇數頁眉的右邊
\fancyhead[LE]{} %偶數頁眉的左邊
\fancyhead[RE]{{\scriptsize 1917-1924: 北京的“小朝廷” · 袁世凯时代}} %偶數頁眉的右邊
\chapter*{袁世凯时代}
\addcontentsline{toc}{chapter}{\hspace{1cm}袁世凯时代}
\thispagestyle{empty}
紫禁城中的早晨,有时可以遇到一种奇异的现象,处于深宫但能听到远远的市声。有很清晰的小贩叫卖声,有木轮大车的隆隆声,有时还听到大兵的唱歌声。太监们把这现象叫做“响城”。离开紫禁城以后,我常常回忆起这个引起我不少奇怪想象的响城。响城给我印象最深的,是有几次听到中南海的军乐演奏。\\

“袁世凯吃饭了。”总管太监张谦和有一次告诉我,“袁世凯吃饭的时候还奏乐,简直是‘钟鸣鼎食’,比皇上还神气!”\\

张谦和的光嘴巴抿得扁扁的,脸上带着忿忿然的神色。我这时不过九岁上下,可是已经能够从他的声色中感到类似悲凉的滋味。军乐声把我引进到耻辱难忍的幻象中:袁世凯面前摆着比太后还要多的菜肴,有成群的人伺候他,给他奏乐,扇着扇子……\\

但也有另外一种形式的响城,逐渐使我发生了浓厚的兴趣。这种“响城”是我在毓庆宫里从老师们的嘴里听到的。这就是种种关于复辟的传说。\\

复辟——用紫禁城里的话说,也叫做“恢复祖业”,用遗老和旧臣们的话说,这是“光复故物”,“还政于清”——这种活动并不始于尽人皆知的“丁巳事件”,也并不终于民国十三年被揭发过的“甲子阴谋”。可以说从颁布退位诏起到“满洲帝国”成立止\footnote{严格地说,复辟活动到此时尚未停止。伪满改帝制后,虽然我的活动告一段落,但关内有些人仍不死心,后来日本发动了全面侵略,占领了平津,这些人在建立“后清”的幻想下,曾有一度活动。因为日本主子不同意,才没闹起来。},没有一天停顿过。起初是我被大人指导着去扮演我的角色,后来便是凭着自己的阶级本能去活动。在我少年时期,给我直接指导的是师傅们,在他们的背后,自然还有内务府大臣们,以及内务府大臣世续商得民国总统同意,请来照料皇室的“王爷”(他们这样称呼我的父亲)。这些人的内心热情,并不弱于任何紫禁城外的人,但是后来我逐渐地明白,实现复辟理想的实际力量并不在他们身上。连他们自己也明白这一点。说起来滑稽,但的确是事实:紫禁城的希望是放在取代大清而统治天下的新贵们身上的。第一个被寄托这样幻想的人,却是引起紫禁城忿忿之声的袁世凯大总统。\\

我到现在还记得很清楚,紫禁城里是怎样从绝望中感到了希望,由恐惧而变为喜悦的。在那短暂的时间里,宫中气氛变化如此剧烈,以致连我这八岁的孩子也很诧异。\\

我记得太后在世时,宫里很难看到一个笑脸,太监们个个是唉声叹气的,好像祸事随时会降临的样子。那时我还没搬到养心殿,住在太后的长春宫,我给太后请安时,常看见她在擦眼泪。有一次我在西二长街散步,看见成群的太监在搬动体元殿的自鸣钟和大瓶之类的陈设。张谦和愁眉苦脸地念叨着:\\

“这是太后叫往颐和园搬的。到了颐和园,还不知怎么样呢!”\\

这时太监逃亡的事经常发生。太监们纷纷传说,到了颐和园之后,大伙全都活不成。张谦和成天地念叨这些事,每念叨一遍,必然又安慰我说:“万岁爷到哪儿,奴才跟哪儿保驾,决不像那些胆小鬼!”我还记得,那些天早晨,他在我的“龙床”旁替我念书的声音,总是有气无力的。\\

民国二年的新年,气氛开始有了变化。阳历除夕这天,陈师傅在毓庆宫里落了座,一反常态,不去拿朱笔圈书,却微笑着瞅了我一会,然后说:\\

“明天阳历元旦,民国要来人给皇上拜年。是他们那个大总统派来的。”\\

这是不是他第一次向我进行政务指导,我不记得了,他那少有的得意之色,大概是我第一次的发现。他告诉我,这次接见民国礼官,采用的是召见外臣之礼,我用不着说话,到时候有内务府大臣绍英照料一切,我只要坐在龙书案后头看着就行了。\\

到了元旦这天,我被打扮了一下,穿上金龙袍褂,戴上珠顶冠,挂上朝珠,稳坐在乾清宫的宝座上。在我两侧立着御前大臣、御前行走和带刀的御前侍卫们。总统派来的礼官朱启铃走进殿门,遥遥地向我鞠了一个躬,向前几步立定,再鞠一躬,走到我的宝座台前,又深深地鞠了第三躬,然后向我致贺词。贺毕,绍英走上台,在我面前跪下。我从面前龙书案上的黄绢封面的木匣子里,取出事先写好的答辞交给他。他站起身来向朱启铃念了一遍,念完了又交还给我。朱启铃这时再鞠躬,后退,出殿,于是礼成。\\

第二天早晨,气氛便发生了进一步的变化,首先是我的床帐子外边张谦和的书声朗朗,其次是在毓庆宫里,陈师傅微笑着捻那乱成一团的白胡须,摇头晃脑地说:\\

“优待条件,载在盟府,为各国所公认,连他总统也不能等闲视之!”\\

过了新年不久,临到我的生日,阴历正月十四这天,大总统袁世凯又派来礼官,向我祝贺如仪。经过袁世凯这样连续的捧场,民国元年间一度销声匿迹的王公大臣们,又穿戴起蟒袍补褂、红顶花翎,甚至于连顶马开路、从骑簇拥的仗列也有恢复起来的。神武门前和紫禁城中一时熙熙攘攘。在民国元年,这些人到紫禁城来大多数是穿着便衣,进城再换上朝服袍褂,从民国二年起,又敢于翎翎顶顶、袍袍褂褂地走在大街上了。\\

完全恢复了旧日城中繁荣气象的,是隆裕的寿日和丧日那些天。隆裕寿日是在三月十五,过了七天她就去世了。在寿日那天,袁世凯派了秘书长梁士诒前来致贺,国书上赫然写着:“大中华民国大总统致书大清隆裕皇太后陛下”。梁士诒走后,国务总理赵秉钧率领了全体国务员,前来行礼。隆裕去世后,袁世凯的举动更加动人:他亲自在衣袖上缠了黑纱,并通令全国下半旗一天,文武官员服丧二十七天,还派全体国务员前来致祭。接着,在太和殿举行了所谓国民哀悼大会,由参议长吴景濂主祭;军界也举行了所谓全国陆军哀悼大会,领衔的是袁的另一心腹,上将军段祺瑞。在紫禁城内,在太监乾嚎的举哀声中,清朝的玄色袍褂和民国的西式大礼服并肩进出。被赏穿孝服百日的亲贵们,这时脸上洋溢着得意的神色。最让他们感到兴奋的是徐世昌也从青岛赶到,接受了清室赏戴的双眼花翎。这位清室太傅在颁布退位后,拖着辫子跑到德国人盘踞的青岛当了寓公,起了一个有双关含意的别号“东海”。他在北京出现的意义,我在后面还要谈到。\\

隆裕的丧事未办完,南方发起了讨袁运动,即所谓“二次革命”。不多天,这次战争以袁世凯的胜利而告终。接着,袁世凯用军警包围国会,强迫国会选他为正式大总统。这时他给我写了一个报告:\\

\begin{quote}
	大清皇帝陛下:中华民国大总统谨致书大清皇帝陛下:前于宣统三年十二月二十五日奉大清隆裕皇太后懿旨,将统治权公诸全国,定为共和立宪国体,命袁世凯以全权组织临时共和政府,合满汉蒙回藏五族,完全领土为一大中华民国。旋经国民公举,为中华民国临时大总统。受任以来,两稳于兹,深虞险越。今幸内乱已平,大局安定,于中华民国二年十月六日经国民公举为正式大总统。国权实行统一,友邦皆已承认,于是年十月十日受任。凡我五族人民皆有进于文明、跻于太平之希望。此皆仰荷大清隆裕皇太后暨大清皇帝天下为公,唐虞揖让之盛轨,乃克臻此。我五族人民感戴兹德,如日月之照临,山河之涵育,久而弥昭,远而弥挚。维有董督国民,幸新治化,烙守优待条件,使民国巩固,五族协和,庶有以慰大清隆裕皇太后在天之灵。用特报告,并祝万福。\\

\begin{flushright}
	中华民国二年十月十九日\\
	
	袁世凯
\end{flushright}


\end{quote}

由于这一连串的新闻,遗老中间便起了多种议论。\\

“袁世凯究竟是不是曹操?”\\

“项城当年和徐、冯、段说过,对民军只可智取不可力敌,徐。冯、段才答应办共和。也许这就是智取?”\\

“我早说过,那个优待条件里的辞位的辞字有意思。为什么不用退位、逊位,袁宫保单要写成个辞位呢?辞者,暂别之谓也。”\\

“大总统常说‘办共和’办的怎样。既然是办,就是试行的意思。”\\

这年冬天,光绪和隆裕“奉安”,在梁格庄的灵棚里演出了一幕活剧。主演者是那位最善表情的梁鼎芬,那时他还未到宫中当我的师傅,配角是另一位自命孤臣的劳乃宣,是宣统三年的学部副大臣兼京师大学堂总监督,辛亥后曾躲到青岛,在德国人专为收藏这流人物而设的“尊孔文社”主持社事。在这出戏里被当做小丑来捉弄的是前清朝山东巡抚、袁政府里的国务员孙宝琦,这时他刚当上外交总长(孙宝琦的父亲孙诒经被遗老们视为同光时代的名臣之一)。那一天,这一批国务员由赵秉钧率领前来。在致祭前赵秉钧先脱下大礼服,换上清朝素袍褂,行了三跪九叩礼。孤臣孽子梁鼎芬一时大为兴奋,也不知怎么回事,在那些没穿清朝袍褂来的国务员之中,叫他一眼看中了孙宝琦。他直奔这位国务员面前,指着鼻子问:\\

“你是谁?你是哪国人?”\\

孙宝琦给这位老朋友问得怔住了,旁边的人也都给弄得莫名其妙。梁鼎芬的手指头哆嗦着,指点着孙宝倚,嗓门越说越响:\\

“你忘了你是孙诒经的儿子!你做过大清的官,你今天穿着这身衣服,行这样的礼,来见先帝先后,你有廉耻吗?你——是个什么东西!”\\

“问得好!你是个什么东西?!”劳乃宣跟了过来。这一唱一帮,引过来一大群人,把这三个人围在中心。孙宝琦面无人色,低下头连忙说:\\

“不错,不错,我不是东西!我不是东西!”\\

后来梁师傅一谈起这幕活剧时,就描述得有声有色。这个故事和后来的“结庐守松”、“凛然退刺客”,可算是他一生中最得意的事迹。他和我讲了不知多少次,而且越讲情节越完整,越富于传奇性。\\

到民国三年,就有人称这年为复辟年了。孤臣孽子感到兴奋的事情越来越多:袁世凯祀孔,采用三卿士大夫的官秩,设立清史馆,擢用前清旧臣。尤其令人眼花缭乱的,是前东三省总督赵尔巽被任为清史馆馆长。陈师傅等人视他为贰臣,他却自己宣称:“我是清朝官,我编清朝史,我吃清朝饭,我做清朝事。”那位给梁鼎芬在梁格庄配戏的劳乃宣,在青岛写出了正续《共和解》,公然宣传应该“还政于清”,并写信给徐世昌,请他劝说袁世凯。这时徐世昌既是清室太傅同时又是民国政府的国务卿,他把劳的文章给袁看了。袁叫人带信给劳乃宣,请他到北京做参议。前京师大学堂的刘廷琛,也写了一篇《复礼制馆书》,还有一位在国史馆当协修的宋育仁,发表了还政于清的演讲,都一时传遍各地。据说在这个复辟年里,连四川一个绰号叫十三哥的土匪,也穿上清朝袍褂,坐上绿呢大轿,俨然以遗老自居,准备分享复辟果实了。\\

在紫禁城里,这时再没有人提起搬家的事。谨慎稳健的内务府大臣世续为了把事情弄牢靠些,还特地找了他的把兄弟袁世凯一次。他带回的消息更加令人兴奋,因为袁世凯是这样对他说的:“大哥你还不明白,那些条条不是应付南边的吗?太庙在城里,皇上怎么好搬?再说皇宫除了皇上,还能叫谁住?”这都是很久以后,在内务府做过事的一位遗少告诉我的。当时世续和王爷根本不和我谈这类事情,要谈的也要经过陈师傅。师傅当时的说法是:“看样子,他们总统,倒像是优待大清的。优待条件本是载在盟府……”\\

师傅的话,好像总没有说完全。现在回想起来,这正是颇有见地的“慎重”态度。和紫禁城外那些遗老比起来,紫禁城里在这段时期所表现的乐观,确实是谨慎而有保留的。袁世凯的种种举动——从公开的不忘隆裕“在天之灵”,到私下认定“皇上”不能离开皇宫和太庙,这固然给了紫禁城里的人不少幻想,但是紫禁城从“袁宫保”这里所能看到的也只限于此。因此,紫禁城里的人就不能表现出太多的兴奋。到了复辟年的年底,北京开始变风头的时候,证明了这种“审慎”确实颇有见地。\\

风头之变换,始于一个肃政史提出要追查复辟传闻。袁世凯把这一案批交内务部“查明办理”,接着,演讲过还政于清的朱育仁被步军统领衙门递解回籍。这个消息一经传出,不少人便恐慌了,劝进文章和还政于清的言论都不见了,在青岛正准备进京赴任的劳乃宣也不敢来了。不过人们还有些惶惑不解,因为袁世凯在查办复辟的民政部呈文上,批上了“严禁复辟谣言,既往不咎”这样奇怪的话,而宋育仁被递解回籍时,袁世凯送了他三千块大洋,一路上又大受各衙门的酒宴迎送,叫人弄不清他到底是受罚还是受奖。直到民国四年,总统府的美国顾问古德诺发表了一篇文章,说共和制不适中国国情\footnote{古德诺原为美国政治大学教授。他发表的这篇文章的题目叫做《共和与君主论》,胡说什么“中国如用君主制较共和制为宜”,作为袁世凯实行帝制的理论根据。},继而又有“筹安会”\footnote{筹安会是袁世凯实行帝制的御用机关,由杨度建议,吸收孙毓筠、严复、刘师培、李燮和、胡瑛等组成,为袁称帝进行鼓吹和筹备工作。}出现,主张推袁世凯为中华帝国的皇帝,这才扫清了满天疑云,使人们明白了袁世凯要复的是什么辟。风头所向弄明白了,紫禁城里的气氛也变了。\\

我从响城中听见中南海的军乐声,就是在这时候。那时,三大殿正进行油缮工程,在养心殿的台阶上,可以清清楚楚地望见脚手架上油工们的活动。张谦和告诉我,那是为袁世凯登极做准备。后来,“伦贝子”(溥伦)代表皇室和八旗向袁世凯上劝进表,袁世凯许给他亲王双俸,接着他又到宫里来向太妃索要仪仗和玉玺。这些消息使我感到心酸、悲忿,也引起了我的恐惧。虽然陈师傅不肯明讲,我也懂得“天无二日,国无二君”这句老话。袁世凯自己做了皇帝,还能让我这多余的皇帝存在吗?历史上的例子可太多了,太史公就统计过“春秋之中,弑君三十六”哩!\\

在那些日子里,乾清门外的三大殿的动静,牵连着宫中每个人的每根神经。不论谁在院子里行走,都要关心地向那边张望一下,看看关系着自己命运的油缮工程,是否已经完工。太妃们每天都要烧香拜佛,求大清的护国神“协天大帝关圣帝君”给以保佑。仪仗是忙不迭地让溥伦搬走了,玉玺因为是满汉合壁的,并不合乎袁世凯的要求,所以没有拿去。\\

这时毓庆宫里最显著的变化,是师傅们对毓崇特别和气,没有人再拿他当伯禽来看待。他在太妃那里竟成了红人,常常被叫进去赏赐些鼻烟壶、搬指之类的玩艺儿。每逢我说话提到袁世凯的时候,师傅就向我递眼色,暗示我住嘴,以免让毓崇听见,传到他父亲溥伦耳朵里去。\\

有一天,毓崇应召到太妃那里去了,陈宝琛看见窗外已经没有了他的影子,从怀里拿出一张纸条,神秘地对我说:\\

“臣昨天卜得的易卦,皇上看看。”\\

我拿过来,看见这一行字:\\

“我仇有疾,不我能即,吉!”\\

他解释说,这是说我的仇人袁世凯前途凶恶,不能危害于我,是个吉卦。他还烧了龟背,弄过蓍草,一切都是吉利的,告诉我可以大大放心。这位老夫子为了我的命运,把原始社会的一切算命办法都使用过了。因此,他乐观地做出结论:\\

“天作孽,犹可违,自作孽,不可活。元凶大憝的袁世凯作孽如此,必不得善终!‘我仇有疾,终无尤也!’何况优待条件藏在盟府,为各国所公认,袁世凯焉能为害于我乎?”\\

为了“不我能即”和保住优待条件,师傅、王爷和内务府大臣们在算卦之外的活动,他们虽没有告诉我,我也多少知道一些。他们和袁世凯进行了一种交易,简单地说,就是由清室表示拥护袁皇帝,袁皇帝承认优待条件。内务府给了袁一个正式公文,说:“现由全国国民代表决定君主立宪国体,并推戴大总统为中华帝国大皇帝,为除旧更新之计,作长治久安之谋,凡我皇室极表赞成。”这个公文换得了袁世凯亲笔写在优待条件上的一段跋语:\\

先朝政权,未能保全,仅留尊号,至今耿耿。所有优待条件各节,无论何时断乎不许变更,容当列入宪法。袁世凯志,乙卯孟冬。\\

这两个文件的内容后来都见于民国四年十二月十六日的“大总统令”中。这个“令”发表之前不多天,我父亲日记里就有了这样一段记载:\\

\begin{quote}
	十月初十日(即阳历十一月十六日)上门。偕世太傅公见四皇贵妃,禀商皇室与袁大总统结亲事宜,均承认可,命即妥行筹办一切云。在内观秘件,甚妥,一切如恒云云。\\
\end{quote}

所谓秘件,就是袁的手书跋语。所谓亲事,就是袁世凯叫步兵统领江朝宗向我父亲同世续提出的让他女儿当皇后。太妃们心里虽不愿意,也不得不从。其结果是,优待条件既没列入宪法,我也没跟袁家女儿结婚,因为袁世凯只做了八十三天的皇帝,就在一片反袁声中气死了。\\