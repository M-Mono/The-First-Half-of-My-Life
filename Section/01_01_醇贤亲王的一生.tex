\fancyhead[LO]{{\scriptsize 1859-1908: 我的家世 · 醇贤亲王的一生}} %奇數頁眉的左邊
\fancyhead[RO]{} %奇數頁眉的右邊
\fancyhead[LE]{} %偶數頁眉的左邊
\fancyhead[RE]{{\scriptsize 1859-1908: 我的家世 · 醇贤亲王的一生}} %偶數頁眉的右邊
\chapter*{醇贤亲王的一生}
\addcontentsline{toc}{chapter}{\hspace{1cm}醇贤亲王的一生}
\thispagestyle{empty}
  公元一九零六年,即清朝光绪三十二年的旧历正月十四,我出生于北京的醇王府。我的祖父奕譞,是道光皇帝的第七子,初封郡王,后晋亲王,死后谥法“贤”,所以后来称做醇贤亲王。我的父亲载沣,是祖父的第五子,因为第一和第三、四子早殇,第二子载湉被姨母慈禧太后接进宫里,当了皇帝(即光绪皇帝),所以祖父死后,由父亲袭了王爵。我是第二代醇王的长子。在我三岁那年的旧历十月二十日,慈禧太后和光绪皇帝病笃,慈禧突然决定立我为嗣皇帝,承继同治(载淳,是慈禧亲生子,载湉的堂兄弟),兼祧光绪。在我入宫后的两天内,光绪与慈禧相继去世。十一月初九日,我便登极为皇帝——清朝的第十代,也是最末一代的皇帝,年号宣统。不到三年,辛亥革命爆发,我退了位。\\

  我的记忆是从退位时开始的。但是叙述我的前半生,如果先从我的祖父和我的老家醇王府说起,事情就会更清楚些。\\

  醇王府,在北京曾占据过三处地方。咸丰十年,十九岁的醇郡王奕譞奉旨与懿贵妃叶赫那拉氏的妹妹成婚,依例先行分府出宫,他受赐的府邸坐落在宣武门内的太平湖东岸,即现在中央音乐学院所在地。这就是第一座醇王府。后来,载湉做了皇帝,根据雍正朝的成例,“皇帝发祥地”(又称为“潜龙邸”)须升为宫殿,或者空闲出来,或者仿雍王府(雍正皇帝即位前住的)升为雍和宫的办法,改成庙宇,供奉菩萨。为了腾出这座“潜龙邸”,慈禧太后把什刹后海的一座贝子府\footnote{宗室爵位分为亲王、郡王、贝勒、贝子、公、将军各等。贝子府即是贝子的府第。}赏给了祖父,拨出了十六万两银子重加修缮。这是第二座醇王府,也就是被一些人惯称为“北府”的那个地方。我做了皇帝之后,我父亲做了监国摄政王,这比以前又加了一层搬家的理由,因此隆裕太后(光绪的皇后,慈禧太后和我祖母的侄女)决定给我父亲建造一座全新的王府,这第三座府邸地址选定在西苑三海集灵囿紫光阁一带。正在大兴土木之际,武昌起义掀起了革命风暴,于是醇王府的三修府邸、两度“潜龙”、一朝摄政的家世,就随着清朝的历史一起告终了。\\

  在清朝最后的最黑暗的年代里,醇王一家给慈禧太后做了半世纪的忠仆。我的祖父更为她效忠了一生。\\

  我祖父为道光皇帝的庄顺皇贵妃乌雅氏所出,生于道光二十二年,死于光绪十六年。翻开皇室家谱“玉牒”来看,醇贤亲王奕譞在他哥哥咸丰帝在位的十一年间,除了他十岁时因咸丰登极而按例封为醇郡王之外,没有得到过什么“恩典”,可是在咸丰帝死后那半年间,也就是慈禧太后的尊号刚出现的那几个月间,他忽然接二连三地得到了一大堆头衔:正黄旗汉军都统、正黄旗领侍卫内大臣、御前大臣、后扈大臣、管理善扑营事务、署理奉宸苑事务、管理正黄旗新旧营房事务、管理火枪营事务、管理神机营事务……。这一年,他只有二十一岁。一个二十一岁的青年,能出这样大的风头,当然是由于妻子的姐姐当上了皇太后。但是事情也并非完全如此。我很小的时候曾听说过这样一个故事。有一天王府里演戏,演到“铡美案”最后一场,年幼的六叔载洵看见陈士美被包龙图铡得鲜血淋漓,吓得坐地大哭,我祖父立即声色俱厉地当众喝道:“太不像话!想我二十一岁时就亲手拿过肃顺,像你这样,将来还能担当起国家大事吗?”原来,拿肃顺这件事才是他飞黄腾达的真正起点。\\

  这事发生在一八六一年。第二次鸦片战争以屈辱的和议宣告结束,逃到热河卧病不起的咸丰皇帝,临终之前,召集了随他逃亡的三个御前大臣和五个军机大臣,立了六岁的儿子载淳为皇太子,并且任命这八位大臣为赞襄政务大臣。第二天,咸丰帝“驾崩”,八位“顾命王大臣”按照遗命,扶载淳就位,定年号为“棋祥”,同时把朝政抓在手里。\\

  这八位顾命王大臣是恰亲王载垣、郑亲王端华、协办大学士户部尚书肃顺和军机大臣景寿、穆荫、匡源、杜翰、焦佑瀛,其中掌握实权的是两位亲王和一位协办大学士,而肃顺更是其中的主宰。肃顺在咸丰朝很受器重,据说他善于擢用“人才”,后来替清廷出力镇压太平天国革命的汉族大地主曾国藩、左宗棠之流,就是由他推荐提拔的。因为他重用汉人,贵族们对他极其嫉恨。有人说他在太平军声势最盛的时期,连纳贿勒索也仅以旗人\footnote{满族统治阶级对满族人民实行的统治制度是军事、行政、生产合一的八旗制度。这个制度是由“牛录”制(汉译作“位领”,是满族早期的一种生产和军事合一的组织形式)发展而来的,明万历二十九年(1601)努尔哈赤建黄。白、红、蓝四旗,万历四十四年(1615)增设镶黄、镶白、镶红、镶蓝田旗,共为八旗。凡满族成员都被编入旗,叫做旗人,平时生产战时出征。皇太极时又建立了蒙古八旗与汉军八旗。}为对象。又说他为人凶狠残暴,专权跋扈,对待异己手腕狠毒,以致结怨内外,种下祸根。其实,肃顺遭到杀身之祸,最根本的原因,是他这个集团与当时新形成的一派势力水火不能相容,换句话说,是他们没弄清楚在北京正和洋人拉上关系的恭亲王,这时已经有了什么力量。\\

  恭亲王奕訢\footnote{奕訢(1832—1898)是道光的第六子,道光三十年封为恭亲王。他因为这次与英法联军谈判之机缘,得到了帝国主义的信任与支持,顺利地实行了政变。此后即开办近代军事工业和同文馆,进行洋务活动,成为洋务派的首领。但是后来他因有野心,慈禧与他发生了矛盾,而帝国主义也物色到了更好的鹰犬,即把他抛弃,洋务派首领位置便由李鸿章等所代替。},在咸丰朝本来不是个得意的人物。咸丰把奕訢丢在北京去办议和,这件苦差事却给奕訢造成了机运,奕訢代表清廷和英法联军办了和议,接受了空前丧权辱国的北京条约,颇受到洋人的赏识。这位得到洋人支持的“皇叔”,自然不甘居于肃顺这班人之下,再加上素来嫉恨肃顺的王公大臣的怂恿,恭亲王于是跃跃欲试了。正在这时,忽然有人秘密地从热河“离宫”带来了两位太后的懿旨。\\

  这两位太后一位是咸丰的皇后钮祜录氏,后来尊号叫慈安,又称东太后,另一位就是慈禧,又称西太后。西太后原是一个宫女,由于怀孕,升为贵妃,儿子载淳是咸丰的独子,后来当了皇帝,母以子贵,她立时成了太后。不知是怎么安排的,她刚当上太后,便有一个御史奏请两太后垂帘听政。这主意遭到肃顺等人的狠狠驳斥,说是本朝根本无此前例。这件事对没有什么野心的慈安太后说来,倒无所谓,在慈禧心里却结下了深仇。她首先让慈安太后相信了那些顾命大臣心怀叵测,图谋不轨,然后又获得慈安的同意,秘密传信给恭亲王,召他来热河离宫商议对策。当时肃顺等人为了巩固既得势力,曾多方设法来防范北京的恭亲王和离宫里的太后。关于太后们如何避过肃顺等人的耳目和恭亲王取得联系的事,有种种不同的传说。有人说太后的懿旨是由一个厨役秘密带到北京的,又有人说是慈禧先把心腹太监安德海公开责打一顿,然后下令送他到北京内廷处理,懿旨就这样叫安德海带到了北京。总之,懿旨是到了恭亲王手里。恭亲王得信后,立即送来奏折,请求觐见皇帝。肃顺等人用“留守责任重大”的“上谕”堵他,没能堵住。肃顺又用叔嫂不通问的礼法,阻他和太后们会见,依然没有成功。关于恭亲王与太后的会见,后来有许多传说,有的说是恭亲王化妆成“萨满”\footnote{据说满族早期有一种原始宗教,叫做“萨满教”。以天堂为上界,诸神所居,地面为中界,人类所居,地狱为下界,恶魔所居。男巫叫“萨满”,女巫叫“乌答有”。他们为人治病、驱邪时,口念咒语,手舞足蹈,作神鬼附身状。满族进关后,此教仍然保存,但只限女巫(称萨满太太)经常进宫。}进去的,有的说是恭亲王直接将了肃顺一军,说既然叔嫂见面不妥。就请你在场监视好了,肃顺一时脸上下不来,只好不再阻拦。还有一个说法是恭亲王祭拜咸丰灵位时,慈禧太后让安德海送一碗面赏给恭亲王吃,碗底下藏着慈禧写给奕訢的懿旨。总之,不管哪个传说可靠,反正恭亲王和太后们把一切都商议好了。结果是,太后们回到北京,封奕訢为议政王,八个顾命王大臣全部被捕,两个亲王赐自尽,肃顺砍了头,其余的充军的充军,监禁的监禁。载淳的年号也改为“同治”,意思是两太后一同治政。从此开始了西太后在同光两代四十七年垂帘听政的历史。我的祖父在这场政变中的功勋,是为慈禧在半壁店捉拿了护送“梓宫”\footnote{皇帝的棺材是梓木做的,皇帝生时居住的是宫殿,故死后躺的棺材亦叫做“梓宫”。}返京的肃顺。我祖父于是获得了前面所说的那一串头衔。\\

  此后,同治三年,奕譞又被赐以“加亲王衔”的荣誉,同治十一年正式晋封为亲王。同治十三年,同治皇帝去世,光绪皇帝即位,他更被加封亲王“世袭罔替”,意思是子孙世代承袭王爵,而不必按例降袭。在光绪朝,恭亲王曾几度失宠,但醇亲王受到的恩典却是有增无已,极尽人世之显赫。\\

  我在醇王府里看见过祖父留下的不少亲笔写的格言家训,有对联,有条幅,挂在各个儿孙的房中。有一副对联是:“福禄重重增福禄,恩光辈辈受思光”。当时我觉得祖父似乎是心满意足的。但我现在却另有一种看法,甚至觉得前面说到的那个看戏训子的举动,祖父都是另有用意。\\

  如果说二十一岁的醇郡王缺乏阅历,那么经历了同治朝十三年的醇亲王,就该有足够的见识了。特别是关于同治帝后之死,醇亲王身为宗室亲贵,是比外人知之尤详,感之尤深的。\\

  在野史和演义里,同治是因得花柳病不治而死的,据我听说,同治是死于天花(翁同和的日记也有记载)。按理说天花并非必死之症,但同治在病中受到了刺激,因此发生“痘内陷”的病变,以致抢救无术而死。据说经过是这样:有一天同治的皇后去养心殿探病,在同治床前说起了婆婆又为了什么事责骂了她,失声哭泣。同治劝她忍受着,说将来会有出头的日子。慈禧本来就不喜欢这个儿媳,对儿子和媳妇早设下了监视的耳目。这天她听说皇后去探视同治,就亲自来到养心殿东暖阁外,偷听儿子和媳妇的谈话。这对小夫妻万没料到几句私房话竟闯下滔天大祸,只见慈禧怒气冲冲地闯了进来,一把抓住皇后的头发,举手痛打,并且叫内廷准备棍杖伺候。同治吓得昏厥过去了,慈禧因此没有对皇后用刑。同治一死,慈禧把责任全部安到皇后的头上,下令限制皇后的饮食。两个月后,皇后也就被折磨死了。皇后死后,慈禧的怒气还不消,又革掉了皇后的父亲崇绔的侍郎职位。第二年,有个多事的御史上了一个奏折,说外边传说很多,有说皇后死于悲痛过度,有说死于绝粟,总之,节烈如此,应当表彰,赐以美谥云云。结果皇后的谥法没有争到,这位御史把自己的官也丢了。\\

  在同治死前,慈禧同治母子不和已是一件公开的秘密。我在故宫时就听到老太监说过,同治给东太后请安,还留下说一会话,在自己亲生母亲那里,简直连一句话也说不出来。同治亲政时,慈禧在朝中的亲信羽翼早已形成,东太后又一向不大问事;皇帝办起事来如果不先问问西太后,根本行不通。这就是母子不和的真正原因。慈禧是个权势欲非常强烈的人,绝不愿丢开到手的任何权力。对她说来,所谓三纲五常、祖宗法制只能用来适应自己,决不能让它束缚自己。为了保持住自己的权威和尊严,什么至亲骨肉、外戚内臣,一律顺我者昌,逆我者亡。同治帝后之死,可以说是慈禧面目的进一步暴露。我祖父如果不是看得很清楚,他决不会一听说叫儿子去当皇帝就吓得魂不附体。参加了那次御前会议的翁同和在日记里写过,当慈禧宣布立载湉为嗣的话一出口,我祖父立即“碰头痛哭,昏迷伏地,掖之不能起……”\\

  按照祖制,皇帝无嗣就该从近支晚辈里选立皇太子。载淳死后,自然要选一个溥字辈的,但是那样一来,慈禧成了太皇太后,再去垂帘听政就不成了。因此她不给儿子立嗣,却把外甥载湉要去做儿子。当时有个叫吴可读的御史,以“尸谏”为同治争嗣,也没能使她改变主意。她只不过许了一个愿,说新皇帝得了儿子,就过继给同治。有一位侍读学士的后人,也是我家一位世交,给我转述过那次御前会议情形时说,那天东太后没在场,只有西太后一人,她对那些跪着的王公大臣们说:“我们姐儿俩已商议好了,挑个年岁大点儿的,我们姐儿俩也不愿意。”连惟一能控制她一点的东太后也没出来表示意见,别人自然明白,无论是“尸谏”还是痛哭昏迷,都是无用的了。\\

  从那以后,在我祖父的经历上,就出现了很有趣的记载。一方面是慈禧屡赐恩荣,一方面是祖父屡次的辞谢。光绪入宫的那年,他把一切官职都辞掉了。“亲王世袭罔替”的恩典是力辞不准才接受的。这以后几年,他的惟一差使是照料皇帝读书。他于得兢兢业业,诚惶诚恐,于是慈禧又赏了他“亲王双俸”、“紫禁城内乘坐四人轿”。后来恭亲王失宠,革掉了议政王大臣,慈禧太后又命军机大臣们,今后凡有重大政务要先和醇亲王商议,这等于给了他更高的职务。按例,男子结婚便算成年。光绪如果结了婚,太后理应归政。这是慈禧极不情愿的事,于是就在光绪婚前,由奕譞带头向太后叩请继续“训政”。清朝创建新式海军,奕譞接受了这个重任,海军初步建成之后,他须代表太后去检阅,偏要拉着一位太监同去,因为这位李莲英大总管是慈禧的心腹人。慈禧赐他夫妇坐杏黄轿,他一次没敢坐进去。这种诚惶诚恐的心理,不仅表现在他的一切言行之中,连家里的陈设上也带着痕迹。他命名自己住的正房为“思谦堂”,命名书斋为“退省斋”。书斋里条几上摆着“欹器”\footnote{欹器亦叫做敧器,苟子《宥生篇》云:“孔子观于鲁桓公之庙,有敧器焉,孔子问于守庙者曰:‘此为何器?’守庙者日:‘此盖为宥坐之器。’(宥与右同,言人君可置于坐右,以为戒,或曰宥与侑同,即劝。)孔子曰:‘吾闻宥坐之器者,虚则敧,中则正,满则复。’孔子顾谓弟子曰:‘注水焉。’弟子挹水而注之,中而正,满而复,虚而奇攴。孔子喟然而叹曰:‘吁!恶有满而不复者哉!’”},刻着“满招损,谦受益”的铭言。子女的房中,到处挂着格言家训,里面有这样一段话:“财也大,产也大,后来子孙祸也大,若问此理是若何?子孙钱多胆也大,天样大事都不怕,不丧身家不肯罢。”其实问题不在钱财,而是怕招灾惹祸。最有意思的是,他在光绪二年写了一个奏折,控告一个没有具体对象的被告,说是将来可能有人由于他的身份,要援引明朝的某些例子,想给他加上什么尊崇;如果有这样的事,就该把倡议人视为小人。他还要求把这奏折存在宫里,以便对付未来的那种小人。过了十几年之后,果然发生了他预料到的事情。光绪十五年,河道总督吴大澄上疏请尊崇皇帝本生父以称号。慈禧见疏大怒,吓得吴大澄忙借母丧为由,在家里呆了三年没敢出来。\\

  毫无疑问,自从光绪入宫以后,我祖父对于他那位姻姊的性格一定有更多的了解。在光绪年间,她的脾气更加喜怒无常。有一个太监陪她下棋,说了一句“奴才杀老祖宗的这只马”,她立刻大怒道:“我杀你一家子!”就叫人把这太监拉了出去活活打死了。慈禧很爱惜自己的头发,给她梳头的某太监有一次在梳子上找到一根头发,不由得心里发慌,想悄悄把这根头发藏起来,不料被慈禧从镜子里看到了,这太监因此挨了一顿板子。伺候过慈禧的太监都说过,除了李莲英之外,谁轮着在慈禧的跟前站班,谁就提心吊胆。慈禧年岁渐老,有了颜面肌抽搐的毛病,她最不愿意人家看见。有个太监大概是多瞧了一眼,她立刻问:“你瞧什么?”太监没答上来,就挨了几十大板。别的太监知道了,站班时老是不敢抬头,她又火了:“你低头干什么?”这太监无法回答,于是也挨了几十大板。还有一回,慈禧问一个太监天气怎样,这个乡音未变的太监说:“今儿个天气生冷生冷的。”慈禧对这个“生冷生冷”听着不顺耳,也叫人把这太监打了一顿。除了太监,宫女也常挨打。\\

  奴仆挨打以至杖毙,在北京王府里不算什么稀奇事,也许这类事情并不足以刺激醇亲王。如果这都不算,那么光绪七年的关于东太后的暴卒,对醇亲王来说,就不能是一件平常事了。据说咸丰去世前就担心懿贵妃将来母以子贵做了太后,会恃尊跋扈,那时皇后必不是她的对手,因此特意留下一道朱谕,授权皇后,可在必要时制裁她。生于侯门而毫无社会阅历的慈安,有一次无意中把这件事向慈禧泄露出来。慈禧从此下尽功夫向慈安讨好,慈安竟被她哄弄得终于当她的面前烧掉了咸丰的遗诏。过了不久,东太后就暴卒宫中。有的说是吃了慈禧送去的点心,有的说喝了慈禧给慈安亲手做的什么汤。这件事对醇亲王说来无疑地是个很大刺激,因为后来的事实就是如此:他更加谨小慎微,兢兢业业,把取信讨好慈禧,看做是他惟一的本分。他负责建设海军的时候(李鸿章是会办大臣),为了让太后有个玩的地方,便将很大一部分海军经费挪出来修建了颐和园。这座颐和园修建工程最紧张的阶段,正值直隶省和京师遭受特大水灾,御史吴兆泰因为怕激起灾民闹事,建议暂时停工,因此夺官,“交部议处”。而醇亲王却一言不发,鞠躬尽瘁地完成了修建任务。一八九零年颐和园完工,他也与世长辞了。四年后,他手创的所谓海军惨败于甲午之役。花了几千万两白银所建造的船只,除了颐和园的那个石舫,大概没有再剩下什么了。\\
