\fancyhead[LO]{{\scriptsize 1950-1954: 由抗拒到认罪 · 东北人民的灾难和仇恨}} %奇數頁眉的左邊
\fancyhead[RO]{} %奇數頁眉的右邊
\fancyhead[LE]{} %偶數頁眉的左邊
\fancyhead[RE]{{\scriptsize 1950-1954: 由抗拒到认罪 · 东北人民的灾难和仇恨}} %偶數頁眉的右邊
\chapter*{东北人民的灾难和仇恨}
\addcontentsline{toc}{chapter}{\hspace{1cm} 东北人民的灾难和仇恨}
\thispagestyle{empty}
关于日本侵略者在东北造下的灾难,我过去从来没听人具体地谈过,也从来没有在这方面用过心。我多少知道一些东北人民的怨恨,但是我只想到那是东北人与日本人之间的事,与我无关。历史过去了十来年,到今天我才如梦初醒,才感觉到真正的严重性。\\

工作团的人员给我们专门讲过一次,关于日本侵略者在东北罪行的部分调查结果。我当时听了还有点疑惑。他列举了一些不完全的统计数字,例如惨案数字,某些惨案中的集体屠杀的数字,种植鸦片面积、吸鸦片的烟民及从鸦片贩卖中获得利润的数字,等等,都是骇人听闻的。那些屠杀、惨案的情节更是令人发指。我听的时候一面感到毛骨悚然,一面却在想:“果真是如此吗?如果是真的,我不知道,怎么我的弟弟、妹夫、侄子和随侍他们也没有人向我说过呢?”\\

一直到后来参加了日本战犯的学习大会,我才不再怀疑这些血淋淋的事实。\\

我们这是第一次看见日本战犯。后来从报上才知道,抚顺的日本战犯是在中国羁押的日本战犯的一部分。根据这次大会和后来日本战犯的释放、宣判以及以后陆续得到的消息,我们发现这些罪犯在学习中发生了意想不到的变化。关于这点,我后面还要说到。现在说一说这个大会。这个大会虽然有所方和工作团的人员在场,事实上是由他们自己的“学委会”组织起来的。“学委会”是在大多数日本战犯思想有了觉悟后,自己选出来管理自己的生活和学习的组织。在这次大会上,有几个日本战犯讲了自己的学习体会,坦白交代了许多罪行,有的人则对别人进行了检举。他们用事实回答了一个学习的中心问题:日本帝国主义是不是在中国犯了罪。我们全体伪满战犯参加大会旁听。在那些坦白与检举中,给我们印象最深、使我们感到震动最大的是前伪满总务厅次长古海忠之和一个伪满宪兵队长的坦白。\\

古海忠之是日本军部跟前的红人,他和武部六藏(总务厅长官)秉承关东军的意旨,以伪满政权的实际统治者的地位,策划和执行了对全东北的掠夺和统治。他具体地谈出了强占东北农民土地的移民开拓政策,掠夺东北资源的“产业开发五年计划”,毒害东北人民的鸦片政策,以及如何榨取东北的粮食和其他物资以准备太平洋战争等等的内幕。他谈出了许多秘密会议的内情,谈出了许多令人咋舌的数字;他所谈到的那些政策的后果,每个例子都是一个惨案。例如一九四四年从各县征用了一万五千多名劳工,在兴安岭王爷庙修建军事工程,由于劳动与生活条件恶劣,在严寒中缺吃少穿,死掉了六千多人。又例如为了准备对苏作战,修改流人兴凯湖的穆棱河河道,工人由于同样原因致死的有一千七百多人。\\

我记得最清楚的是他谈的鸦片政策。\\

一九三三年初,日军在热河发动军事行动之前,为了筹办军费,决定采用鸦片政策。当时尚未控制东北的鸦片生产情况,手中现货不足,乃向国外贩进二百多万两,同时用飞机在热河广散传单,鼓励种植鸦片。后来,大约是一九三六年,在伪满七省扩大种植面积,大力生产,以后又以法律形式确定了鸦片的专卖垄断。为了鼓励吸毒,各地广设“禁烟协会”、鸦片馆,并设“女招待”,大事吸引青年。一九四二年,日本“兴亚院”召开了“支那鸦片需给会议”,做出了“由满洲国和蒙疆供应大东亚共荣圈内的鸦片需要”的决议,据此又在伪满扩大种植面积到三千公顷。据古海估计,至伪满垮台止,伪满共生产了鸦片约达三亿两之多。鸦片利润在一九三八年占伪满财政收入的六分之一,一九四四年利润增至三亿元,为伪满初期的一百倍,是日本侵略战争的军费重要来源之一。吸毒的烟民,仅热河一省就达三十万人左右,全东北平均一百个居民里就有五个中烟毒的人。\\

那个宪兵队长所坦白的,都是非常具体的事例。他交代出的每件事,都是一幅血腥的图画。\\

他做过伪满西南地区宪兵队队长。为了镇压人民,宪兵队采取了各种恐怖的手段。杀人,往往是集体屠杀,杀后还召集群众去参观尸体。有时把一些他们认为可疑的人抓了来,站成一排,从中随便挑出一个来,当众用刀劈死。他自己用这种办法就杀了三十多个。抓来的人,要受到各种刑罚的折磨:棍子打,鼻孔里倒灌冷水、辣椒水、煤油,用香火烧,红铁烙,倒挂起来,等等。\\

在许多日本战犯的检举中,惊心动魄的惨剧是数不胜数的。这些惨剧的主演者实在比野兽还要残暴。有一段故事我记得是这样:一个日本兵闯进一户人家,一个年轻的母亲,正坐在锅台边上抱着孩子喂奶,这个兵一把抢走孩子,顺手扔进开水锅里,然后强奸了那母亲,最后用棍子插进阴道,活活弄死。这类的故事当年普遍发生于东北各地和日军的各个占领区内。原来这就是“圣战”的内容,这些“皇军勇士”正是我当年祝福、遥拜、拥护的对象,正是我当年的依靠。\\

后来,检察人员不断地送来调查材料、统计材料和东北人民的控诉检举材料。当年东北地区的地狱景象,在我面前越来越清晰。我终于明白了在我屈从、谄媚日本关东军的同时,在我力求保存我的“尊号”的同时,有多少善良无辜的人死于非命;同时也明白了在我恬然事敌的时候,正有无数爱国志士抛头颅、洒热血,向敌人进行着抗争。\\

东北人民所遭受的残害,如果不算直接在日本统治者手里受到的那些,只算经过伪政权和汉奸们那里间接受到的,就可以不费事地举出很多例子和数字来。例如在种种有关粮食的法令、政策,即所谓“粮谷出荷”的规定下,东北人民每年收获的粮食被大批掠走,特别是在伪满后期,东北人民只能靠配给的玉米穰、豆饼、椽子面等等掺成的“混合面”过日子。被掠去的粮食除了充做军用,大部运往日本。输日数量逐年增加,据伪满官方资料,在一九四四年一年内,即输往日本三百万吨。在伪满的最后六年间,粮食输往日本共计一千一百一十多万吨。\\

在统制粮谷、棉布、金属等等物资的法令下,人民动不动就成了“经济犯”。例如,大米是绝对不准老百姓吃的,即使从呕吐中被发现是吃了大米,也要算“经济犯”而被加以治罪。仅仅一九四四到一九四五年的一年间,被当做“经济犯”治罪的就有三十一万七千一百人。当然,被抓去挨了一顿痛打之后放出来的,并不在此数之内。\\

东北农民在粮食被强征的同时,耕地也不断地被侵占着。根据“日满拓植条约”,日本计划于二十年内从日本移民五百万人到东北来。这个计划没有全部实现,日本就垮台了,但是在最后两年内移人的三十九万人,就经过伪满政权从东北农民手中夺去了土地三千六百五十万公顷。此外,借口应付抗日联军而实行的“集家并屯”政策,又使东北人民丧失了大量土地,这尚未计算在内。\\

又例如,日本统治者为了榨取东北的资源,为了把东北建设成它的后方基地,通过伪满政权,巧立了各种名目,残酷地奴役着东北人民,实行了野蛮的奴隶劳动制度,造成了惊人的死亡。自一九三八年用我的名义颁行了“劳动统制法”后,每年强征劳工二百五十万人(不算从关内征集的),强迫进行无偿劳动。大都是在矿山和军事工程中进行劳动,条件十分恶劣,造成了成批死亡。像一九四四年辽阳市的“防水作业”中,二千名青年劳工因劳动过度不到一年就被折磨死的,竟有一百七十人。吉林省蛟河县靠山屯农民王盛才写来一份控诉书,他说:\\

\begin{quote}
	我哥哥王盛有在伪满康德十年旧历一月间,被拉法村公所抓去到东安省当劳工,他在那里吃橡子面,还不让吃饱,夜晚睡在湖地上,还挨打受骂,共去七个月,折磨成病,回来后九个月死去。嫂子改嫁,我父亲终日忧愁,不久死去了。我全家四口,只剩下我一个人,使我家破人亡。\\
\end{quote}

这样的家庭,在当时的东北是非常普遍的。不仅是农民,普通的职工、学生,以及因检查体格不合乎当兵条件的,即所谓“国兵漏”的青年,都要定期从事这种奴隶劳动,即所谓“勤劳奉仕”。蛟河县拉法屯的陈承财控诉说:\\

\begin{quote}
	伪满康德十年的旧历五月初一,伪蚊河县公署把我和我乡“国兵”检查不合格的其他青年共一九八名,编成“勤劳奉仕队”,集中县城。第三日由日本兵押着我们,到东安省勃河县小王站屯做苦工。让我们在野地里挖了一米宽四十米长的沟渠,一栋挨一栋的搭起草席棚子。里边铺些野草,非常潮湿,让我们住在这里。吃的简直不能说了,每天只有橡子面饭团,也不给吃饱。在吃饭前还得排成队,双手举饭“默祷”三分钟。每天重劳动超过十二小时,不管天气炎热与寒冷,叫我们全脱光衣服进行劳动。冬天把我们冻得起疙瘩,夏天晒成脓疤直流水。就在这样劳累苦难的环境下,为伪满洲国修所谓“国境道”。我乡富太河屯刘继生家,一家只父子二人,刘继生就是于同年七月十七日死在工地上的。父亲在家听说儿子死了,也上吊自杀了。挨打是经常的事。在同年五月初四逃跑了五名,不幸被鬼子抓回一名,当场把抓回的青年用绳子拴在马脖子上,人骑着马在地里磨,一直把这个人的肚子磨破,肠子流出而死。\\
\end{quote}

处境最惨的是“矫正辅导院”里的人。在伪满后期,日本的统治,已经残酷到接近疯狂的程度。为了解决劳动力不足和镇压人民越来越大的反抗,一九四三年颁布了“思想矫正法”和“保安矫正法”,在全东北各地普遍设立了集中营,名为“矫正辅导院”,以所谓“思想不良”或“社会浮浪”为名,绑架贫苦无业者或被认为有不满情绪的人,从事最苦的劳役。有时候,连任何询问都用不着,把行路人突然拦截起来,统统加上“浮浪者”的罪名,送进矫正辅导院。进去之后,就没有出来的日子。那些熬到伪满垮台的人,今天怀着刻骨的仇恨,向人民政府控诉了伪满政权。鹤岗市翻身街的一个农民,伪满时原在鹤岗“新开基满洲土木”做工,一九四四年被以反满抗日名义抓到伪警察署。同他一起的有十七个人。他们被毒打之后,被送到鹤岗矫正辅导院,强迫到东山煤矿挖煤,每天十二小时,每顿饭只有一个小高粱饭团,没衣服穿,没被子盖,经常受毒打。他说:\\

\begin{quote}
	我母亲听说我在辅导院押着,就到我做活的地方隔着刺网看我,被辅导警看见,当时把我母亲揪着头发,脚踢拳打了一顿,打得我母亲躺在地下爬不起来。后来又用洋镐打我,打得我浑身是伤,昏迷不醒,七天人事不知。有一次我们因为吃饭不给菜,同押的宋开通拿我的钱向过路人买些葱,被辅导科的汉奸王科长看见,把我和宋开通叫去,在我身上搜出五元钱。他们就打我,把嘴和鼻子打得都流出血,又把我装在麻袋里,不蹲下就敲脑袋,装在麻袋里举起来摔,摔了三下我就昏过去了。每天都死人,每隔三四天就抬出七八个死人,我一同被抓的十七个人就死了九个。我得了肺病,到现在不能做活。那时我母亲也得了疯魔,我三个弟弟那时最大的十一岁,他们每天讨饭过活。\\
\end{quote}

当时在鹤岗矫正辅导院用度科当用度员的尹影,在检举书上写道:\\

\begin{quote}
	伪满鹤岗矫正辅导院从一九四四年成立至一九四五年八月九号,囚禁人数达一千一百九十人。被囚禁之人员大部是由佳木斯、牡丹江、富锦等地区监狱里押送来的。其中有一人叫陈永福,是我认识的。他在街上行走,无故被警察抓来的。在矫正辅导院里的犯人,每天做工十二小时,每人每天只给六两粗粮,穿更生布衣。吃不饱穿不暖,做工时间又长,坑内通风不良,空气非常恶劣。有了病不但不给营养的东西吃,反而将粮食减到四两至三两半,有的人怕减粮就带病上班挖煤。就这样造成大批死亡。在病室里有的死了很长时间才被发现,死后当时并不给抬走,经一二日才抬出去放在停尸场中,用小木牌写上号码拴在手腕上,按井字样堆成垛。一九四五年三月二十号我亲眼看见使用黄毯子卷尸体三十四具,叫患病的人两人抬一个,送到鹤岗东山“万人坑”埋掉,将毯子拿回,再发给别人使用。\\

为防止“浮浪者”(被押人)的逃跑,施行恐怖镇压手段,经常由监房提出被押人扒去衣服吊起毒打,打得人浑身发紫,还强迫劳动。我现在还记得有一次富锦县监狱押送来的所谓“浮浪者”刘永才,被打在小便上,提回监房即死。……\\
\end{quote}

伪满的军队、警察、法院、监狱对东北人民的镇压,更是充满了血腥气,造成的惨案更是数不胜数。据检察人员从残余的伪满官方档案里找到的部分材料,就统计出了被伪满军杀害的抗日军民有六万余人,屠杀的居民八千八百余人,烧毁的民房有三千一百余处所。伪满警察、特务机关所杀害的善良人民,那数目是无法计算了。仅据三十六起有案可查的统计,在被逮捕的五千零九十八名爱国人士和无辜群众中,只有三人经不起诉释放,档案中声明判死刑者四百二十一人,未判刑即死于狱中者二百十三人,判徒刑者二千一百七十七人,其余二千二百八十四名则无下落。伪满时期,东北是警察的世界,几乎村村都有警察。一个县的警察署,就等于是个阎王殿。这种地方制造的惨剧,在地狱里也不过如此。肇源县八家子有位六十一岁的农民黄永洪,当年因为给抗日联军送过信,被伪警察署提了去,他经历了一场集体屠杀。他说:\\

\begin{quote}
	这年阴历二月二十六,伪警察提出我们被押的三十多人,让拿着洋镐到肇源西门外挖坑,天黑又回到监狱。二十七日又提出我和王亚民、高寿三、刘成发四个人,另一批又提二十人,到了西门外,把那二十人枪毙了,又提来二十二个人,又把他们枪毙了。枪毙以后,警察在他们身上倒汽油,点着了烧,在烧的时候,有一个人未死,被火一烧,就出来逃跑,又被警察用枪打死了。烧完之后,叫我们四个人将他们四十二人用土都埋了。现在肇源西门外还有那个大坑,我还能找到那个地方……\\
\end{quote}

这座活地狱,在“执政”、“康德皇帝”、“王道乐土”等等幌子底下存在了十四年!所有的残酷暴行,都是在我这个“执政”和“皇帝”的标签下进行的。每个受难者都被迫向“御真影”叩拜,背诵“诏书”,感谢“亲邦”和“皇帝”的恩赐。因此,今天每份控诉书后面都有这类的呼声:\\

“要求人民政府给我们申冤报仇!我们要向日寇和汉奸讨还血债!”\\

“给我们死去的亲人报仇!惩办日寇和汉奸!”\\