\fancyhead[LO]{{\scriptsize 1950-1954: 由抗拒到认罪 · 我只想到死}} %奇數頁眉的左邊
\fancyhead[RO]{} %奇數頁眉的右邊
\fancyhead[LE]{} %偶數頁眉的左邊
\fancyhead[RE]{{\scriptsize 1950-1954: 由抗拒到认罪 · 我只想到死}} %偶數頁眉的右邊
\chapter*{我只想到死}
\addcontentsline{toc}{chapter}{\hspace{1cm} 我只想到死}
\thispagestyle{empty}
押送伪满战犯的苏联列车,于一九五零年七月三十一日到达了中苏边境的绥芬河车站。负责押送的阿斯尼斯大尉告诉我,向中国政府的移交,要等到明天早晨才能办。他劝我安心地睡一觉。\\

从伯力上车时,我和家里的人分开了,被安置在苏联军官们的车厢里。他们给我准备了啤酒、糖果,一路上说了不少逗趣的话。尽管如此,我仍然觉得他们是在送我去死。我相信只要我一踏上中国的土地,便没有命了。\\

在对面卧铺上,阿斯尼斯大尉发出了均匀的呼吸声。我睁着眼睛,被死亡的恐惧搅得不能入睡。我坐起来,默诵了几遍《般若波罗蜜多心经》,刚要躺下,站台上传来了越来越近的脚步声,好像走来了一队士兵。我凑近车窗,向外张望,却看不见人影。皮靴步伐声渐渐远去了,只剩下远处的灯光在不祥地闪烁着。我叹了口气,缩身回到卧铺的犄角上,望着窗桌上的空酒杯出神。我记起了阿斯尼斯喝酒时说的几句话:“天亮就看见你的祖国了,回祖国总是一件值得庆贺的事。你放心,共产党的政权是世界上最文明的,中国的党和人民气量是最大的。”\\

“欺骗!”我恶狠狠地瞅了躺在对面卧铺上的阿斯尼斯一眼,他已经打起鼾来了。“你的话,你的酒,你的糖果,全是欺骗!我的性命跟窗外的露水一样,太阳一出来便全消失了!你倒睡得瓷实!”\\

那时在我的脑子里,只有祖宗而无祖国,共产党只能与“洪水猛兽”联系着,决谈不上什么文明。我认为苏联虽也是共产党国家,对我并无非人道待遇,但苏联是“盟国”之一,要受到国际协议的约束,不能乱来。至于中国,情况就不同了。中国共产党打倒了蒋介石,不承认任何“正统”,对于我自然可以为所欲为,毫无顾忌。我在北京、天津、长春几十年间听到的宣传,所谓“共产党”不过全是“残酷”、“凶恶”等等字眼的化身,而且比蒋介石对我还仇恨百倍。我到了这种人手里,还有活路吗?“好死不如赖活”的思想曾支配了我十来年,现在我认为“赖活”固然是幻想,“好死”也是奢望。\\

我在各种各样恐怖的设想中度过了一夜。当天明之后,阿斯尼斯大尉让我跟他去见中国政府代表的时候,我只想着一件事:我临死时有没有勇气喊一声“太祖高皇帝万岁”?\\

我昏头胀脑地随阿斯尼斯走进一间厢房。这里坐着两个中国人,一位穿中山装,一位穿草绿色的没有衔级的军装,胸前符号上写着“中国人民解放军”七个字。他们俩站起身跟阿大尉说了几句话,其中穿中山装的转过身对我打量了一下,然后说:\\

“我奉周恩来总理的命令来接收你们。现在,你们回到了祖国。……”\\

我低头等着那军人给我上手铐。可是那军人对我瞅着,一动不动。\\

“他知道我跑不了的。”一个多小时之后,我这样想着,跟阿斯尼斯走出车厢,上了站台。站台上站着两排持枪的兵,一边是苏联军队,一边是个个都佩戴着那种符号的中国军队。我们从中间走过,上了对面的列车。在这短暂的片刻时间内,我想起了蒋介石的八百万军队,就是由戴这种符号的人消灭的。我现在在他们眼里,大概连个虫子也不如吧?\\

进了车厢,我看见了伪满那一伙人,看见了我家里的人。他们规规矩矩地坐着,身上都没有镣铐和绳索。我被领到靠尽头不远的一个座位上,有个兵把我的皮箱放上行李架。我坐下来,想看看窗外的大兵们在干什么,这时我才发现,原来车窗玻璃都被报纸糊上了;再看看车厢两头,一头各站着一个端冲锋枪的大兵。我的心凉下来了。气氛如此严重,这不是送我们上刑场又是干什么呢?我看了看左近的犯人,每个人的脸上都呈现出死灰般的颜色。\\

过了不大功夫,有个不带任何武器的人,看样子是个军官,走到车厢中央。\\

“好,现在你们回到祖国了。”他环视着犯人们说,“中央人民政府对你们已经做好安排,你们可以放心。……车上有医务人员,有病的就来报名看病……”\\

这是什么意思呢?祖国,安排,放心,有病的看病?呵,我明白了,这是为了稳定我们的心,免得路上出事故。后来,几个大兵拿来一大筐碗筷,发给每人一副,一面发一面说:“自己保存好,不要打了,路上不好补充。”我想,看来这条通往刑场的路还不短,不然为什么要说这个呢。\\

早餐是酱菜、咸蛋和大米稀饭。这久别的家乡风味勾起了大家的食欲,片刻间一大桶稀饭全光了。大兵们发现后,把他们自己正要吃的一桶让给了我们。我知道车上没有炊事设备,他们要到下一个车站才能重新做饭,因此对大兵们的这个举动,简直是百思不得一解,最后只能得出这样一个结论:反正他们对我们不会有什么好意。\\

吃过这顿早饭之后,不少人脸上的愁容舒展了一些。后来有人谈起,他们从大兵们让出自己的早饭这件事上,觉出了押送人员很有修养、很有纪律,至少在旅途中不会虐待我们。我当时却没有这种想法,我想的正相反,认为共产党人对我是最仇恨的,说不定在半路上就会对我下手,施行报复。就像中了魔一样,我往这上头一想,就觉得事情好像非发生不可,而且就像是出不了这天夜里似的。有的人吃过早饭打起盹来,我却坐立不安,觉得非找人谈谈不可。我要向押送人员尽早地表白一下,我是不该死的。\\

坐在我对面的是个很年轻的公安战士。这是我面前最现成的谈话对象。我仔细地打量了他一番,最后从他的胸章上找到了话题。我就从“中国人民解放军”这几个字谈起。\\

“您是中国人民解放军(我这是头一次使用“您”字),解放,这两个字意思好极了。我是念佛的人,佛经里就有这意思。我佛慈悲,发愿解放一切生灵……”\\

年轻的战士瞪起两只大眼,一声不响地听着我叨叨。当我说到我一向不杀生,连苍蝇都没打过的时候,他脸上的表情,是令人捉摸不透的。我不由得气馁下来,说不下去了。我哪里知道,这位年轻的战士对我也是同样的摸不着头脑呢!\\

我的绝望心情加重了。我听着车轮轧着铁轨的闹声,觉着死亡越来越近了。我离开了坐位,漫无目的地在通道上走着,走到车的另一头,在厕所门边站了几秒钟,又转身往回走。我走到中途,听见旁边的侄子小秀在和什么人低声说话,好像说什么“君主”、“民主”。我忽然站住向他嚷道:\\

“这时候还讲什么君主?谁要说民主不好,我可要跟他决斗!”\\

人们全给我弄呆了。我继续歇斯底里地说:“你们看我干什么?反正枪毙的不过是我,你们不用怕!”\\

一位战士过来拉我回去,劝我说:“你该好好休息一下。”我像鬼迷了似地拉住这位战士,悄悄对他说:“那个是我的侄子,思想很坏,反对民主。还有一个姓赵的,从前是个将官,在苏联说了不少坏话……”\\

我回到座位上,继续絮叨着。那战士要我躺下来,我不得已,躺在椅子上,闭上眼,嘴里仍停不下来。后来,大概是几夜没睡好的缘故吧,不知道是从什么时候起,我竟睡着了。\\

一觉醒来,已是第二天的清晨。我想起了昨天的事,很想知道被我检举的那两个人命运如何。我站起来寻找了一下,看见小秀和姓赵的还都坐在原来的位子上,小秀神色如常,姓赵的却似乎有点异样。我走近他,越看越觉得他的神色凄惨;他正端详着自己的两手,翻来覆去地看。我断定他自知将死,正在怜惜自己。这时我竟又想起了死鬼报冤的故事,生怕他死后找我算账。想到这里,我身不由己走到他面前,跪下来给他磕了一个头。行过这个“攘灾”礼,我一面往回走,一面嘟嘟囔囔念起“往生神咒”。\\

列车速度降低下来,终于停了。不知是谁低低说了一声:“长春!”我像弹簧似地一下子跳起,扑向糊着报纸的窗户,恨不得能钻个窟窿看看。我什么也看不见,只听到不远的地方有许多人唱歌的声音。我想,这就是我死的地方了。这里曾是我做皇帝的地方,人们已经到齐,在等着公审我了。我在苏联曾从《实话报》上看到过关于斗争恶霸的描写,知道公审的程序,首先是民兵夹着被审者上场。这时正好车门那边来了两个大兵,让我受了一场虚惊。原来他们是来送早餐稀饭的。与此同时,列车又开动了。\\

列车到了沈阳。我想这回不会再走了,我一定是死在祖宗发祥的地方。车停下不久,车厢里进来一位陌生的人,他拿着一张字条,当众宣布说:“天气太热,年纪大些的现在随我去休息一下。”然后念起名单来。我听到那名单里不仅有我,而且里面还有我的侄子小秀,我奇怪了。我今年四十四岁,如果勉强可以算是年纪大的,可是三十几岁的小秀是怎么算进去的呢?我断定,这必是一个骗局。我是皇帝,其他的都是大臣,小秀则是叫我检举连累的,全都完了。我同名单上的人们一起坐进了一部大轿车,随车的也是端冲锋枪的大兵。我对小秀说:“完啦!我带你见祖宗去吧!”小秀脸色一下子变得煞白。拿名单的那个人却笑道:“你怕什么呀?不是告诉过你这是休息吗?”我没有理他,心里只顾说:“骗局!骗局!骗局!”\\

汽车在一座大楼门前停下了,门口又是端着冲锋枪的大兵。一个不带武器的军人迎着我们,领我们进了大门,说了一声:“上楼!”我已经是豁出去了,既然得死,那就快点吧。我把上衣一团,夹在胳臂下就上了楼。我越走越快,竟超过了带头的那位,弄得他不得不赶紧抢到我前面去。到了楼上,他快步走到一个屋门口,示意叫我进去。这是间很大的屋子,当中摆着长桌、椅子,桌上是些水果、纸烟、点心。我把衣服往桌上一扔,随手拿起一个苹果,咬了一口,心里说,这是“送命宴”,快吃快走。我咬了一半苹果,后面的人才陆续到达。片刻间,屋里坐满了人,除了点名来的我们十几个之外,还来了不少穿中山服和军装的人。\\

在离我身边不远的地方,出现了一位穿中山装的中年人,开始讲话了。我费劲地咽着嘴里的东西,他的话竟一句也没听见。我好容易吃完那个苹果,便站起来打断了他的话:\\

“别说了,快走吧!”\\

有些穿中山装的笑了起来。那讲话的人也笑道:\\

“你太紧张了。不用怕。到了抚顺,好好休息一下,老老实实地学习……”\\

听清了这几句话,我怔在那里了。难道是不叫我死吗?这是怎么回事?这时正好带我们来的那人走了过来,手里拿着那张点名的名单,向刚才讲话的那人汇报说,除熙治因病未到外,其余需要休息的都来了。我一听,这更不是瞎猎了。为了证实这一点,我不顾一切地,上前一把将那个名单抢了过来。这个举动虽然引起了一阵哄堂的笑声,但是我却弄明白了那确实是个名单,不是什么死刑判决书之类的东西。正在这时,张景惠的儿子小张也来了。他是跟另一批伪满战犯首先回国的,他把那一批人的现状告诉了我们,又把一些人的家属情况说了。大家听说先来的一批人都活着,而且家里情况很好,子女们读书的读书,工作的工作,每个人的脸上都放了光。这时我的眼泪有如泉水,汹涌而至……\\

固然,我所得到的这种轻松感,历时并没有多久,只不过是从沈阳到抚顺这段路上的一个小时,但它毕竟是起了松弛神经的作用,否则我真会发起疯来的。因为从伯力上火车以后,五天来我想到的只是死。\\