\fancyhead[LO]{{\scriptsize 1932-1945: 伪满十四年 · 几个“诏书”的由来}} %奇數頁眉的左邊
\fancyhead[RO]{} %奇數頁眉的右邊
\fancyhead[LE]{} %偶數頁眉的左邊
\fancyhead[RE]{{\scriptsize 1932-1945: 伪满十四年 · 几个“诏书”的由来}} %偶數頁眉的右邊
\chapter*{几个“诏书”的由来}
\addcontentsline{toc}{chapter}{\hspace{1cm} 几个“诏书”的由来}
\thispagestyle{empty}
在伪满学校读过书的人,都被迫背过我的“诏书”。在学校、机关、军队里,每逢颁布一种诏书的日子,都要由主管人在集会上把那种诏书念一遍。听人讲,学校里的仪式是这样的:仪式进行时,穿“协和服”\footnote{协和服是伪满公教人员统一的制服,墨绿色,荐任官以上还有一根黄色的绳子套在颈间,称为“协和带”。学校里的校长和训育主任,一般都有这根所谓“协和带”。}的师生们在会场的高台前列队肃立,教职员在前,学生在后。戴着白手套的训育主任双手捧着一个黄布包,高举过顶,从房里出来。黄布包一出现,全场立即低下头。训育主任把它捧上台,放在桌上,打开包袱和里面的黄木匣,取出卷着的诏书,双手递给戴白手套的校长,校长双手接过,面向全体展开,然后宣读。如果这天是五月二日,就念一九三五年我第一次访日回来在这天颁布的“回銮训民诏书”(原无标点):\\

\begin{quote}
	朕自登极以来,亟思躬访日本皇室,修睦联欢,以伸积慕。今次东渡,宿愿克遂。日本皇室,恳切相待,备极优隆,其臣民热诚迎送,亦无不殚竭礼敬。衷怀铭刻,殊不能忘。深维我国建立,以达今兹,皆赖友邦之仗义尽力,以奠丕基。兹幸致诚佃,复加意观察,知其政本所立,在乎仁爱,教本所重,在乎忠孝;民心之尊君亲上,如天如地,莫不忠勇奉公,诚意为国,故能安内攘外,讲信恤邻,以维持万世一系之皇统。朕今躬接其上下,咸以至诚相结,气同道合,依赖不渝。朕与日本天皇陛下,精神如一体。尔众庶等,更当仰体此意,与友邦一心一德,以奠定两国永久之基础,发扬东方道德之真义。则大局和平,人类福祉,必可致也。凡我臣民,务遵朕旨,以垂万囗。钦此!\\
\end{quote}

诏书共有六种,即:\\

一九三四年三月一日的“即位诏书”;\\

一九三五年五月二日的“回銮训民诏书”;\\

一九四零年七月十五日的“国本奠定诏书”;\\

一九四一年十二月八日的“时局诏书”;\\

一九四二年三月一日的“建国十周年诏书”;\\

一九四五年八月十五日的“退位诏书”。\\

“即位诏书”后来为第五个即“建国十周年诏书”所代替。一九四五年八月十五日的“退位诏书”,那是没有人念的。所以主要的是四个诏书。学生、士兵都必须背诵如流,背不来或背错的要受一定惩罚。这不但是日本在东北进行奴化的宣传材料,也是用以镇压任何反抗的最高司法根据。东北老百姓如果流露出对殖民统治有一丝不满,都可能被借口违背诏书的某一句话而加以治罪。\\

从每一种诏书的由来上,可以看出一个人的灵魂如何在堕落。前两个我在前面说过了,现在说一下第三个,即“国本奠定诏书”是怎么出世的。\\

有一天,我在缉熙楼和吉冈呆坐着。他要谈的话早已谈完,仍赖在那里不走。我料想他必定还有什么事情要办。果然,他站起了身,走到摆佛像的地方站住了,鼻子发过了一阵嗯嗯之声后,回头向我说:\\

“佛,这是外国传进来的。嗯,外国宗教!日满精神如一体,信仰应该相同,哈?”\\

然后他向我解释说日本天皇是天照大神的神裔,每代天皇都是“现人神”,即大神的化身,日本人民凡是为天皇而死的,死后即成神。\\

我凭着经验,知道这又是关东军正在通过这条高压线送电。但是他说了这些,就没电了。我对他的这些神话,费了好几天功夫,也没思索出个结果来。\\

事实是,关东军又想出了一件事要叫我做,但由于关东军司令官植田谦吉正因发动的张鼓峰和诺门坎两次战事不利,弄得心神不宁,一时还来不及办。后来植田指挥的这两次战役都失败了,终于被调回国卸职。临走,他大概想起了这件事,于是在辞行时向我做了进一步的表示:日满亲善,精神如一体,因此满洲国在宗教上也该与日本一致。他希望我把这件事考虑一下。\\

“太上皇”每次嘱咐我办的事,我都顺从地加以执行,惟有这一次,简直叫我啼笑皆非,不知所措。这时,胡嗣瑗已经被挤走,陈曾寿已经告退回家,万绳栻已经病故,佟济煦自护军出事以后胆小如鼠,其他的人则无法靠近我。被视为亲信并能见我的,只有几个妹夫和在“内延”念书的几个侄子。那时,在身边给我出谋献策的人没有了,那些年轻的妹夫和侄子们又没阅历,商量不出个名堂来,我无可奈何地独自把植田的话想了几遍。还没想出个结果,新继任的司令官兼第五任大使梅津美治郎来了。他通过吉冈向我摊了牌,说日本的宗教就是满洲的宗教,我应当把日本皇族的祖先“天照大神”迎过来立为国教。又说,现在正值日本神武天皇纪元二千六百年大庆,是迎接大神的大好时机,我应该亲自去日本祝贺,同时把这件事办好。\\

后来我才听说,在日本军部里早就酝酿过此事,由于意见不一,未做出决定。据说,有些比较懂得中国人心理的日本人,如本庄繁之流,曾认为这个举动可能在东北人民中间引起强烈的反感,导致日本更形孤立,故搁了下来。后来由于主谋者断定,只要经过一段时间,在下一代的思想中就会扎下根,在中年以上的人中间,也会习以为常,于是便做出了这个最不得人心的决定。他们都没有想到,这件事不但引起了东北人民更大的仇恨,就是在一般汉奸心里,也是很不受用的。以我自己来说,这件事就完全违背了我的“敬天法祖”思想,所以我的心情比发生“东陵事件”时更加难受。\\

我当了皇帝以后,曾因为祭拜祖陵的问题跟吉冈发生过争执。登极即位祭祖拜陵,这在我是天经地义之事,但是吉冈说,我不是清朝皇帝而是满蒙汉日朝五民族的皇帝,祭清朝祖陵将引起误会,这是不可以的。我说我是爱新觉罗的子孙,自然可以祭爱新觉罗的祖先陵墓。他说那可以派个爱新觉罗的其他子孙去办。争论结果,当然是我屈服,打消了北陵之行,然而我却一面派人去代祭,一面关上门在家里自己祭。现在事情竟然发展到不但祭不了祖宗,而且还要换个祖宗,我自然更加不好受了。\\

自从我在旅顺屈服于板垣的压力以来,尽管我每一件举动都是对民族祖先的公开背叛,但那时我尚有自己的纲常伦理,还有一套自我宽解的哲学:我先是把自己的一切举动看做是恢复祖业、对祖宗尽责的孝行,以后又把种种屈服举动解释成“屈蠖求伸之计”,相信祖宗在天之灵必能谅解,且能暗中予以保佑。可是现在,日本人逼着我抛弃祖宗,调换祖宗,这是怎么也解释不过去的。\\

然而,一种潜于灵魂深处的真正属我所有的哲学,即以自己的利害为行为最高准则的思想提醒了我:如果想保证安全、保住性命,只得答应下来。当然,在这同时我又找到了自我宽解的办法,即私下保留祖先灵位,一面公开承认新祖宗,一面在家里祭祀原先的祖宗。因此,我向祖宗灵位预先告祭了一番,就动身去日本了。\\

这是我第二次访问日本,时间在一九四○年五月,呆了一共只有八天。\\

在会见裕仁的时候,我拿出了吉冈安直给我写好的台词,照着念了一遍,大意是:为了体现“日满一德一心、不可分割”的关系,我希望,迎接日本天照大神,到“满洲国”奉祀。他的答词简单得很,只有这一句:\\

“既然是陛下愿意如此,我只好从命!”\\

接着,裕仁站起来,指着桌子上的三样东西,即一把剑、一面铜镜和一块勾玉,所谓代表天照大神的三件神器,向我讲解了一遍。我心里想:听说在北京琉璃厂,这种玩艺很多,太监从紫禁城里偷出去的零碎,哪一件也比这个值钱,这就是神圣不可侵犯的大神吗?这就是祖宗吗?\\

在归途的车上,我突然忍不住哭了起来。\\

我回到长春之后,便在“帝宫”旁修起了一所用白木头筑的“建国神庙”,专门成立了“祭祀府”,由做过日本近卫师团长、关东军参谋长和宪兵司令官的桥本虎之助任祭祀府总裁,沈瑞麟任副总裁。从此,就按关东军的规定,每逢初一、十五,由我带头,连同关东军司令和“满洲国”的官员们,前去祭祀一次。以后东北各地也都按照规定建起这种“神庙”,按时祭祀,并规定无论何人走过神庙,都要行九十度鞠躬礼,否则就按“不敬处罚法”加以惩治。由于人们都厌恶它,不肯向它行礼,因此凡是神庙所在,都成了门可罗雀的地方。据说有一个充当神庙的“神官”(即管祭祀的官员),因为行祭礼时要穿上一套特制的官服,样子十分难看,常常受到亲友们的耻笑,有一次他的妻子的女友对他妻子说:“你瞧你们当家的,穿上那身神官服,不是活像《小上坟》里的柳录景吗?”这对夫妻羞愧难当,悄悄丢下了这份差事,跑到关内谋生去了。\\

关东军叫祭祀府也给我做了一套怪模怪样的祭祀服,我觉着穿着实在难看,便找到一个借口说,现值战争时期,理应穿戎服以示支援日本盟邦的决心,我还说穿军服可以戴上日本天皇赠的勋章,以表示“日满一德一心”。关东军听我说得振振有词,也没再勉强我。我每逢动身去神庙之前,先在家里对自己的祖宗磕一回头,到了神庙,面向天照大神的神龛行礼时,心里念叨着:“我这不是给它行礼,这是对着北京坤宁宫行礼。”\\

我在全东北人民的耻笑、暗骂中,发布了那个定天照大神为祖宗和宗教的“国本奠定诏书”。这次不是郑孝胥的手笔(郑孝胥那时已死了两年),而是“国务院总务厅”嘱托一位叫佐藤知恭的日本汉学家的作品。其原文如下:\\

\begin{quote}
	朕兹为敬立\\

建国神庙,以奠国本于悠久,张国纲于无疆,诏尔众庶曰:我国自建国以来,邦基益固,邦运益兴,烝烝日跻隆治。仰厥渊源,念斯丕绩,莫不皆赖天照大神之种麻,天皇陛下之保佑。是以朕向躬访日本皇室,诚烟致谢,感戴弥重,诏尔众庶,训以一德一心之义,其旨深矣。今兹东渡,恭祝纪元二千六百年庆典,亲拜皇大神宫,回銮之吉,敬立建国神庙,奉祀天照大神,尽厥崇敬,以身祷国民福祉,式为永典,令朕子孙万世祗承,有孚无穷。庶几国本奠于惟神之道,国纲张于忠孝之教。仁爱所安,协和所化,四海清明,笃保神麻。尔众庶其克体朕意,培本振纲,力行弗懈,自强勿息。钦此!\\
\end{quote}

诏书中的“天照大神之神麻,天皇陛下之保佑”,以后便成了每次诏书不可少的谀词。\\

为了让我和伪大臣们接受“神道”思想,日本关东军不怕麻烦,特地把著名神道家览克彦(据说是日本皇太后的神道讲师)请来,给我们讲课。这位神道家讲课时,总有不少奇奇怪怪的教材。比如有一幅挂图,上面画着一棵树,据他讲,这棵树的树根,等于日本的神道,上面的枝,是各国各教,所谓八纶—宇,意思就是一切根源于日本这个祖宗。又一张纸上,画着一碗清水,旁边立着若干酱油瓶子、醋瓶子,说清水是日本神道,酱油醋则是世界各宗教,如佛教、儒教、道教、基督教、回教等等。日本神道如同纯净的水,别的宗教均发源于日本的神道。还有不少奇谭,详细的已记不清了。总之,和我后来听到的关于一贯道的说法,颇有点相像。我不知日本人在听课时,都有什么想法,我只知道我自己和伪大臣们,听课时总忍不住要笑,有的就索性睡起觉来。绰号叫于大头的伪军政部大臣于深囗,每逢听“道”就歪着大头打呼噜。但这并不妨害他在自己的故乡照样设大神庙,以示对新祖宗的虔诚。\\

一九四一年十二月八日,日本对美英宣战,在关东军的指示下,伪满又颁布了“时局诏书”。以前每次颁发诏书都是由国务院办的,但这次专门召开了“御前会议”,吉冈让我亲自宣读。这是十二月八日傍晚的事。这诏书也是佐藤的手笔。\\

奉天承运大满洲帝国皇帝诏尔众庶曰:\\

\begin{quote}
	盟邦大日本帝国天皇陛下兹以本日宣战美英两国,明诏煌煌,悬在天日,朕与日本天皇陛下,精神如一体,尔众庶亦与其臣民咸有一德之心,夙将不可分离关系,团结共同防卫之义,死生存亡,断弗分携。尔众庶咸宜克体朕意,官民一心,万方一志,举国人而尽奉公之诚,举国力而援盟邦之战,以辅东亚戡定之功,贡献世界之和平,钦此!
\end{quote}

这些恭维谄媚的词令,和“天照大神之神麻,天皇陛下之保佑”一样,以后都成了我的口头禅。\\

我每逢见来访我的关东军司令官,一张嘴便流利地说出:\\

“日本与满洲国乃是一体不可分的关系,生死存亡的关系,我一定举国力为大东亚圣战的最后胜利,为以日本为首的大东亚共荣圈奋斗到底。”\\

一九四二年,做了日本首相的前关东军参谋长东条英机,到伪满作闪电式的访问。我见了他,曾忙不迭地说:\\

“请首相阁下放心,我当举满洲国之全力,支援亲邦日本的圣战!”\\

这时已经把“盟邦”改称为“亲邦”。这是伪满“建国十周年”所带来的新屈辱,是写在“建国十周年诏书”里的。\\

在这个“十周年”(一九四二年)的前夕,吉冈曾和我说:\\

“没有日本,便不会有满洲国,嗯,所以应该把日本看成是满洲国的父亲。所以,嗯,满洲国就不能和别的国家①一样,称日本国为盟邦友邦,应称做亲邦。”\\

①伪满于一九三九年,参加了日德意三国于一九三一年订的“防共协定”,这就是所谓盟邦。太平洋战争爆发后,又增添了与伪满建交的日本统治的南洋各傀儡国家。\\

与此同时,国务院最末一任总务厅长官武部六藏,把张景惠和各部伪大臣召到他的办公室里,讲了一番称日本为亲邦的道理。接着“建国十周年诏书”就出来了:\\

\begin{quote}
	我国自肇兴以来,历兹十载,仰赖天照大神之神庥,天皇陛下之保佑,国本奠于惟神之道,政教明于四海之民崇本敬始之典,万世维尊。奉天承运之作,垂统无穷。明明之鉴如亲,穆穆之爱如子。夙夜乾惕,惟念昭德,励精自懋,弗敢豫逸。尔有司众庶,亦成以朕心为心,忠诚任事,勤勉治业,上下相和,万方相协。自创业以至今日,始终一贯,奉公不懈,深堪嘉慰。宜益砥其所心,励其所志,献身大东亚圣战,奉翼亲邦之天业,以尽报本之至诚,努力国本之培养,振张神人合一之纲纪,以奉答建国之明命。钦此!
\end{quote}

从此“亲邦”二字便成了“日本”的代名词。\\

我自认是它的儿子还嫌不够,武部六藏和吉冈安直竟又决定,要我写一封“亲书”,由总理张景惠代表我到日本去“谢恩”。我在这里把“谢恩”二字加引号,并非是杜撰,而是真正引用原文的。张景惠的正式身分,乃是“满洲帝国特派赴日本帝国谢恩大使”,这也是写在“亲书”里的。\\

到了一九四四年,日本的败象越来越清楚,连我也能察觉出来,日本军队要倒楣了。有一次吉冈跑来,转弯抹角地先说了一通“圣战正在紧要关头,日本皇军为了东亚共荣圈各国的共存共荣,作奋不顾身的战争,大家自应尽量供应物资,特别是金属……”最后绕到正题上,“陛下可以率先垂范,亲自表现出日满一体的伟大精神……”\\

这回他没有嗯、哈,可见其急不可待,连装腔作势也忘了。而我是浑身毫无一根硬骨头,立即遵命,命令首先把伪宫中的铜铁器具,连门窗上的铜环、铁挂钩等等,一齐卸下来,交给吉冈,以支持“亲邦圣战”。过了两天,我又自动地拿出许多白金、钻石首饰和银器交给吉冈,送关东军。不久吉冈从关东军司令部回来,说起关东军司令部里连地毯都捐献了,我连忙又命把伪宫中所有地毯一律卷起来送去。后来我去关东军司令部,见他们的地毯还好好地铺在那里,究竟吉冈为什么要卷我的地毯,我自然不敢过问。\\

以后我又自动地拿出几百件衣服,让他送给山田乙三,即最末一任的关东军司令长官。\\

当然,经我这一番带头,报纸上一宣扬,于是便给日伪官吏开了大肆搜刮的方便之门。听说当时在层层逼迫之下,小学生都要回家去搜敛一切可搜敛的东西。\\

吉冈后来对溥杰和我的几个妹夫都说过这样的话:“皇帝陛下,在日满亲善如一体方面,乃是最高的模范。”然而,这位“最高模范”在无关紧要之处,也曾叫他上过当。例如捐献白金的这次,我不舍得全给他们,但又要装出“模范”的样儿,于是我便想出这样一个办法,把白金手表收藏起来,另买了一块廉价表带在手腕上。有一天,我故意当着他的面看表,说:“这只表又慢了一分钟。”他瞅瞅我这只不值钱的表,奇怪起来:“陛下的表,换了的,这个不好……”“换了的,”我说,“原来那只是白金的,献了献了的!”\\

一九四五年,东北人民经过十几年的搜刮,已经衣不蔽体。食无粒米,再加上几次的“粮谷出荷”、“报恩出荷”的掠夺,弄得农民们已是求死无门。这时,为了慰问日本帝国主义,又进行了一次搜刮,挤出食盐三千担,大米三十万吨,送到日本国内去。\\

本来这次关东军是打算让我亲自带到“亲邦”进行慰问的。日本这时已开始遭受空袭,我怕在日本遇见炸弹,只得推说:“值此局势之下,北方镇护的重任,十分重大,我岂可以在这时离开国土一步?”不知道关东军是怎么考虑的,后来决定派一个慰问大使来代替我。张景惠又轮上这个差使,去了日本一趟。他此去死活,我自然就不管了。\\