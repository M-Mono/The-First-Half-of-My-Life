\fancyhead[LO]{{\scriptsize 1955-1959: 接受改造 · 会见亲属}} %奇數頁眉的左邊
\fancyhead[RO]{} %奇數頁眉的右邊
\fancyhead[LE]{} %偶數頁眉的左邊
\fancyhead[RE]{{\scriptsize 1955-1959: 接受改造 · 会见亲属}} %偶數頁眉的右邊
\chapter*{会见亲属}
\addcontentsline{toc}{chapter}{\hspace{1cm}会见亲属}
\thispagestyle{empty}
人民可以宽恕,问题在于自己能否“做个正经人”——我从这次参观中明白了这个道理,并且还不只是这一个道理。从前,就是在开始参观的那天,我还用旧的眼光看待今天的政府同群众的关系,认为任何政府同人民之间都没有书上所说的那种一致、那样互相信赖。我总以为共产党之所以有那样强大的军队和有力的政府,是由于“手段”高明和善于“笼络人心”的结果。我所以担心在群众激愤时会牺牲了我,就是由于这种看法。现在我明白了,人民所以拥护党,相信党,实在是由于共产党给人民做了无数好事,这些好事是历史上任何朝代都不可能也不肯于去做的。为矿工——从前被称做“煤黑子”的——做出营养设计,为矿工的安全拿出整个党组织的精力向瓦斯宣战,让“大官旅馆”的命运变成下棋、赏花的晚景,让百分之八十的单身汉从“大房子”搬进新房,让存在了若干世纪的妓院、赌馆、鸦片馆从社会上消失……在过去,哪个政府能够和肯于去做这些事呢? \\

从前,我有时还这样想:也许在新社会里只有穷人得到好处,那些有钱的人,旧社会里有点地位的人,跟我们这类人有瓜葛的人,以及汉族之外的少数民族,恐怕都说不上满意。参观后不久,我亲眼看到了我的亲属,我才明白了这还是过时了的旧眼光。原来满意这个新社会的,在新社会里找到自己前途出路的,竟包括了那么广泛的阶层,实在是历史上空前的。\\

我们跟亲属之间的通信,从一九五五年夏天就开始了。人们从家信里知道了亲属并未因自己是罪犯而受到歧视,知道了子女们有的在上学,有的在工作,有的成了专家,有的参加了共青团,甚至还有的加入了共产党。许多人从家信上受到了很大鼓舞,进一步觉出了社会变化对自己的意义。但是也还有某些多疑的人仍然疑信参半,甚至于还有人全凭偏见而加以穿凿附会、妄加曲解。前伪满将官老张,接到儿子第一次来信。这封信头一句是这样写的:“张先生:对不住,我只能这样称呼你,不能用别的……”老张看完信大为悲\xpinyin*{恸},几乎得了精神病。许多人都为他不平,有人暗地里说:“这不是新社会教育出来的青年吗?新社会里父亲坐牢,儿子就不要他了。”我不由得想起\xpinyin*{陈宝琛}说过的“共产党无情无义”之类的话。跟\xpinyin*{溥杰}同组的前伪满将官老刘,向来对新社会什么都不相信。他非常想念自己的女儿,很怕她受到社会上的歧视。女儿来信告诉他,她的生活很好,人了团,得到组织的关怀,有许多好朋友,她现在夙愿得偿,国家已按她的升学第一志愿分配她到艺术学院。他看了信,摇晃着满头白发说:“说得千真万确,不叫我亲眼看一看我还是不相信。”这些问题,从一九五六年起,都得到了解决,而在我看来,解决的还不只是一家一户的问题,而是整个民族,整个下一代的问题。\\

三月十日,即参观后的第三天,看守员通知我和\xpinyin*{溥杰},还有三妹夫、五妹夫和三个侄子,一齐到所长那里去。我们走进了所长的接待室,在这里出乎意料地看见了别离了十多年的七叔\xpinyin*{载涛}和三妹五妹。\\

看着健壮如昔的胞叔和穿着棉制服的妹妹们,我好象走进了梦境。\\

\xpinyin*{载涛}是我的嫡亲长辈中仅存的一个人。在一九五四年选举中,他作为二百多万满族的代表被选人全国人民代表大会。他同时是人民政协全国委员会的委员。他告诉我,在来看我的前几天,在全国人民代表大会第二次会议上,他看见了毛主席。\xpinyin*{周恩来}总理把他介绍给主席,说这是\xpinyin*{载涛}先生,\xpinyin*{溥仪}的叔叔。主席和他握过手,说:听说\xpinyin*{溥仪}学习的还不错,你可以去看看他们……\\

七叔说到这里,颤抖的语音淹没在哽咽声中,我的眼泪早已无法止住了。一家人都抹着泪,瑞侄竟至哭出了声音……\\

从这次和家族会见中,我明白了不但是我自己得到了挽救,我们整个的满族和满族中的\xpinyin*{爱新觉罗}氏族也得到了挽救。\\

七叔告诉我,解放前满族人口登记是八万人,而今天是这个数目的三十倍。\\

我是明白这个数目变化的意义的。我知道\xpinyin*{辛亥}革命之后,在北洋政府和国民党统治下的旗人是什么处境。那时满族人如果不冒称汉族,找职业都很困难。从那时起,\xpinyin*{爱新觉罗}的子孙纷纷姓了金、赵、罗,我父亲在天津的家,就姓了金。解放后,承认自己是少数民族的一年比一年多。宪法公布之后,满族全都登记了,于是才有了二百四十万这个连满族人自己也出乎意外的数目。\\

我还记得发生“东陵事件”时的悲忿心情,还记得向祖宗灵牌发过的报仇誓愿。我这个自认的\xpinyin*{佛库伦}后裔和复兴满族的代表人,对自己的种族步步走向消亡的命运,我不但未曾加以扭转,而且只能加速着这种命运的到来。只是在声称扶持满族的日本人和我这个以恢复祖业为天职的集团垮台之后,满族和\xpinyin*{爱新觉罗}氏的后人才有了可靠的前途。由八万变成二百四十万,这就是一个证据!\\

这个历史性的变化,包含有\xpinyin*{爱新觉罗}的后人,包含有过去的“涛贝勒”和过去的“三格格”、“五格格”。\\

七叔这年是六十九岁,身体健壮,精神旺盛,几乎使我看不出他有什么老态来。我甚至觉得他和我说话的习惯都没有变。解放以后,他以将近古稀之年参加了解放军的马政工作,兴致勃勃地在西北高原上工作了一段时间。在谈到这些活动的时候,他的脸上露出了愉快之色。他又告诉我,他正打算到外地去视察少数民族的工作,以尽他的人大代表的责任。提到这些,他脸上更发出了光彩。\\

在那数目降到八万的时候,哪个满族的老人的脸上能发出光彩来呢?\\

解放军刚刚进入北京城的时候,有许多满族的遗老是不安的,特别是\xpinyin*{爱新觉罗}氏的后人,看了约法八章之后还是惴惴然,惶惶然。住在北京的这些老人,大多不曾在“满洲国”和\xpinyin*{汪精卫}政府当过“新贵”,但也有人并非能够忘掉自己“天潢贵胄”的身份,放弃掉对我的迷信,所以在我当了囚犯之后,他们比旧时代更感到不安,加上每况愈下的满族人口的凋落和自身景况的潦倒,他们的生活是黯淡无光的,对解放军是不曾抱什么“幻想”的。最先出乎他们意料的,是听到东北人民政府给满族子弟专门办了学校,后来又看见有满族代表也走进了怀仁堂,和各界人士一同坐在全国人民政治协商会议的会场上,参加了共同纲领的讨论。接着,他们中间不少人的家里来了人民政府的干部,向他们访问,邀请他们做地方政协的代表,请他们为满族也为他们自己表示意见,请他们为新社会的建设提供自己的才能。在北京,我曾祖父(\xpinyin*{道光}帝)的后人以及\xpinyin*{惇亲王}、恭亲王和\xpinyin*{醇亲王}这三支的子弟,溥字辈的除了七叔家的几个弟弟比较年轻之外,其余都已是六十以上的老人。我的堂兄\xpinyin*{溥伒}\footnote{\xpinyin*{溥伒}:字雪斋,停亲王\xpinyin*{奕淙}之孙、多罗贝勒\ruby{载瀛}{zǎi yíng}之子。},擅长绘画、书法和古琴,这时已六十多岁,他没想到又能从墙上摘下原已面临绝响厄运的古琴,他不但自己每星期有一天在北海之滨,能和新朋旧友们沉醉在心爱的古老艺术的享受中,而且也从年轻的弟子身上看见了民族古乐的青春。他当选为古琴研究会的副会长、书法研究会的会长,被邀进了一个区的政协,又是中国画院的画师。\xpinyin*{溥伒}的胞兄弟溥\xpinyin*{僴}也是一位老画家,这时也被聘为北京中国画院的画师,这位年近古稀的老人又挥笔向青年一代传授着中国画。他的亲叔伯兄弟溥修\footnote{溥修:\ruby{载濂}{zǎi lián}的次子。},是瑞侄的胞叔,他曾做过“乾清门行走”,我在长春时曾委托他在天津看管过房产,后来双目失明,丧失了一切活动能力,生活潦倒无依。解放后,他的经历以及他肚子里的活史料被新社会所重视,聘他为文史馆员。这种文史馆全国各地都普遍设立着,里面有前清的举人、秀才,也有从北洋政府到\xpinyin*{蒋介石}朝代各个时期各个事件的见证人,有\xpinyin*{辛亥}革命以及更早的同盟会举事的参加者,也有最末一个封建宫廷内幕的目击人。经过他们取得了大量的近代珍贵史料,在他们的晚年,也为新社会贡献了自己的力量。双目失明的修二哥对生活有了信心,心满意足地回忆着清代史料,想好一段,口述一段,由别人代为记录下来。\\

这些已经被新社会视为正常的现象,到了我的心目里却是非常新鲜、印象强烈的新闻。而印象更强烈的,更新鲜的,是我亲眼看到的妹妹们身上的变化。\\

半年前,我和北京的弟弟妹妹们通了信,从来信中我就感觉到了我的家族正在发生变化,但是我从未对这种变化认真思索过。在伪满时代,除了四弟和六妹七妹外,其余的弟弟妹妹都住在长春,大崩溃时都随我逃到通化。我做了俘虏之后,曾担心过这些妹妹会因汉奸家属的身份而受到歧视。二妹的丈夫是\xpinyin*{郑孝胥}的孙子,三妹五妹的丈夫一个是“皇后”的弟弟,一个是\xpinyin*{张勋}的参谋长的儿子,全是伪满中校。四妹夫的父亲是清末因杀\xpinyin*{秋瑾}而出名的绍兴知府。这几个妹夫不是伪满的军官,就是伪政权的官吏,只有六妹夫和七妹夫是两个规规矩矩的读书人,不过她们会不会被汉奸头子的哥哥牵累上呢?我心里也没有底。这类的顾虑是同犯们共有的,我的顾虑比他们更大。后来在通信里,才知道这种顾虑完全是多余。弟弟和妹妹同别人一样有就业机会,孩子们和别人的孩子一样可以入学、升学以及享受助学金的待遇,四弟和七妹还是照旧当着小学教师,六妹是个自由职业者——画家,五妹做了缝纫工人,三妹还是个社会活动家,被街道邻居们选做治安保卫委员。尽管她们自己做饭、照顾孩子,但是她们在信中流露出的情绪总是满意的、愉快的。我放了心。现在,我看到了她们,听着她们和自己的丈夫谈起别后经过,使我联想起了过去。\\

我还记得五妹夫老万睁着他那双大眼睛问五妹:“你真会骑车了?你还会缝纫?”这是在他接到她的来信后就感到十分惊讶的问题,他现在又拿出来问她了。他的惊讶是有根据的。谁料得到从小连跑也不敢跑,长大了有多少仆妇和使女伺候,没进过厨房没摸过剪刀的“五格格”,居然今天能骑上自行车去上班,能拿起剪刀裁制衣服,成了一名自食其力的女缝纫工人呢?\\

更令我们这位学委会主任惊异的,是他的妻子回答得那么自然:“那有什么稀奇?这不比什么都不会好吗?”\\

要知道,假如过去的“五格格”说这样的话,不但亲戚朋友会嘲笑她,就连她自己也认为是羞耻的。那时候她只应该会打扮。会打麻将,会按着标准行礼如仪,而现在,她拿起了剪刀,像个男子一样骑上自行车,过自食其力的生活了。\\

三妹的经历比五妹更多一些。日本投降以后,她没有立刻回到北京,因为孩子生病,她和两个保姆一起留在通化。财产是没有了,她恐怕留下的细软财物和自己的身份引人注意,就在通化摆香烟摊,卖旧衣。在这个期间,她几乎被国民党特务骗走,她上过商人的当,把划不着的火柴批发给她。她经过这些不平常的生活,到一九四九年才回到北京。解放后,街道上开会,她不断去参加,因为在东北接触过解放军和人民政府,她知道些政府的政策,得到了邻居们的信任,被推选出来做街道工作。她谈起来最高兴的一段工作,是宣传新婚\xpinyin*{姻}法……\\

这个经历,在别人看来也许平淡无奇,在我可是不小的惊异。她过去的生活比五妹还要“娇贵”,每天只知道玩,向我撒娇,每逢听说我送了别人东西,总要向我打听,讨“赏”,谁料得到,这个娇慵懒散、只知道谢恩讨赏的“三格格”竟会成了一名社会活动家?乍一听来,真是不可思议。但这个变化是可以理解的。我理解她后来为什么那么积极地宣传新婚\xpinyin*{姻}法,为什么她会在向邻居们读报时哭出来,因为我相信她说的这句话:“我从前是什么?是个摆设!”\\

从前,她虽然有着一定文化水平,名义上是个“贵”妇,而实际上生活是空虚的,贫乏的。她和三妹夫在日本住着的时候,我曾去信叫她把日常生活告诉我,她回信说:“我现在坐在屋里,下女在旁用熨斗烫衣服,老仆在窗外浇花,小狗瞪着眼珠蹲着,看着一匣糖果……实在没有词儿了。”现在,生活给她打开了眼界,丰富了思想,当邻居那样殷切地等她读报时,她才觉出自己有了存在的意义。\\

她后来谈过这样一段经历:“在通化,有一天民兵找了我去,说老百姓在开会,要我去交代一下。我吓坏了,我以为斗争会斗汉奸是很可怕的。我说,你饶了我吧,叫我干什么都行。后来见了干部,他们说不用怕,老百姓是最讲理的。我没法,到了群众会上吓得直哆嗦,我向人们讲了自己的经历。那次会上人多极了,也有人听说看皇姑,都来了。听我讲完,人们嘁嘁喳喳议论开了,后来有人站起来说:‘她自己没干过什么坏事,我们没意见了。’大伙听了都赞成,就散会了。我这才知道,老百姓真是最讲理的。”\\

她这最后一句话,是我刚刚才懂得的。而她在十年前就懂得了。\\

在会见的第二天,正巧接到了二妹来的信,信中说,她的大女儿,一个体育学院的二年级生,已经成了业余的优秀汽车教练员,最近驾驶着摩托车完成了天津到汉口的长途训练。她以幸福的语气告诉我,不但这个十二年前小姐式的女儿成了运动健将,其他的几个孩子也都成了优秀生。当我把这些告诉了三妹。五妹,她们又抹了眼泪,并且把自己的孩子的情况讲了一遍。在这里,我发现这才是\xpinyin*{爱新觉罗}的命运的真正变化。\\

我曾根据一九三七年修订的“玉牒”和妹妹弟弟们提供的材料,做过一个统计。\xpinyin*{爱新觉罗}氏\xpinyin*{醇王}这一支从载字辈算起,婴儿夭折和不成年的死亡率,在清末时是百分之三十四,民国时代是百分之十,解放后十年则是个零。如果把\xpinyin*{爱新觉罗}全家的未成年的死亡率算一下,那就更令人触目惊心。只算我曾祖父的后代,载字与溥字辈未成年的死亡率,男孩是百分之四十强,女孩是百分之五十弱,合计是百分之四十五。在夭亡人口中不足两岁以下的又占百分之五十八强。这就是说\xpinyin*{道光}皇帝的后人每出生十个就有四个半夭折,其中大半又是不到两岁就死了的。\\

我同七叔和妹妹们会见的时候,还没有做这个统计,但是一听到妹妹们屈起手指讲述每个孩子迥异往昔的现况时,我不由得想起了因被我祖母疼爱以至于活活饿死的伯父,十七岁时就死了的大胞妹,不到两岁就死了的三胞弟,以及我在玉牒上看到的那一连串“未有名”字样(来不及起名就死了)。问题还不仅仅在于死亡与成长的数字上,即使每个孩子都长大,除了提鸟笼什么都不会,或者除了失学、失业就看不见什么别的前途,那比起短命来也没什么更多的意思。在民国时代,八旗子弟的命运大部分正是如此。长一辈的每天除了提着鸟笼溜后门,就是一清早坐着喝茶,喝到中午吃饭时,十个八个碟儿的萝卜条豆腐干摆谱,吃完饭和家里人发威风,此外再也不知道有什么好于;晚一辈的除了请安、服侍长辈、照长辈的样子去仿效之外,也很少有知道再要学些什么的。到后来坐吃山空,就业无能,或者有些才能的却又就业无门,结果还是个走投无路。这类事情我知道的不少,现在是全变了!我从这次会见中,深刻地感受到我们下一代的命运,与前一代是如何的不同,他们受到的待遇,实在是我从前所不敢企望的。在北京的一个弟弟和六个妹妹,共有二十七个孩子,除了未达学龄的以外,都在学校里念书,最大的已进了大学。我七叔那边有十六个孙儿孙女和重孙儿重孙女:二十八岁的长孙是水电站技术员;一个孙女是军医大学学生;一个孙女参加过志愿军,立过三等功,已从朝鲜复员回来,转入大学念书;一个孙女是解放军的文艺工作者;其他的除了幼儿或在校。或就业,没有一个游手好闲的。过去的走马放鹰、提笼逛街的上代人生活,在这一代人眼中成了笑话。\\

下一代人也有例外的命运,那是生活在另外一个社会里的\xpinyin*{溥杰}的女儿。他有两个女儿,那时跟她们的母亲住在日本,最大的十八岁。在我们这次跟亲属会见的九个月后,\xpinyin*{溥杰}的妻子从日本寄来一个悲痛的消息,这个大女儿因为恋爱问题跟一个男朋友一起自杀了。后来我听到种种传说,不管怎么传说,我相信那男孩子跟我的侄女一样都是不幸的。在不同的时代和不同的社会里,青年们的命运就是如此不同。\\

从这年起,管理所就不断来人探亲。值得一说的是,顽固的“怀疑派”老刘,看见了他的学艺术的女儿,并且看见了女儿带来的女婿。\\

女儿对他说:“你还不相信,爸爸?我在艺术学院!这就是我的朋友!”他说:“我信了。”\\

女儿说:“你明白不明白,如果不是毛主席的领导,我能进艺术学院吗?我能有今天的幸福吗?”他说:“这也明白了!”\\

女儿说:“明白了,你就要好好地学习,好好地改造!”\\

老刘明白了的事情,老张也明白了。他因为儿子叫他先生,几乎发了疯。这时他女儿来看他,带来了儿子的一封信,他把这封信几乎给每个人都看了:\\

\begin{quote}
	“爸爸:我现在明白了,我有过‘左’的情绪。团组织给我的教育,同志们给我的批评,完全是对的,我不应该对您那样……您学习中有什么困难?我想您学习中一定用得上金笔,我买了一支,托姐姐带上……”\\
\end{quote}
