\fancyhead[LO]{{\scriptsize 1924-1930: 天津的“行在” · 郑孝胥的理想}} %奇數頁眉的左邊
\fancyhead[RO]{} %奇數頁眉的右邊
\fancyhead[LE]{} %偶數頁眉的左邊
\fancyhead[RE]{{\scriptsize 1924-1930: 天津的“行在” · 郑孝胥的理想}} %偶數頁眉的右邊
\chapter*{郑孝胥的理想}
\addcontentsline{toc}{chapter}{\hspace{1cm}郑孝胥的理想}
\thispagestyle{empty}
\xpinyin*{郑孝胥}在北京被\xpinyin*{罗振玉}气跑之后,转年春天回到了我的身边。这时\xpinyin*{罗振玉}逐渐遭到怀疑和冷淡,敌对的人逐渐增多,而\xpinyin*{郑孝胥}却受到了我的欢迎和日益增长的信赖。\xpinyin*{陈宝琛}和\xpinyin*{胡嗣瑗}跟他的关系也相当融洽。一九二五年,我派他总管总务处,一九二八年,又派他总管外务,派他的儿子\xpinyin*{郑垂}承办外务,一同做我对外联络活动的代表。后来他与我之间的关系,可以说是到了\xpinyin*{荣禄}与\xpinyin*{慈禧}之间的那种程度。\\

他比\xpinyin*{陈宝琛}更随和我。那次我会见\xpinyin*{张作霖},事前他和\xpinyin*{陈宝琛}都表示反对,事后,\xpinyin*{陈宝琛}鼓着嘴不说话,他却说:“\xpinyin*{张作霖}有此诚意表示,见之亦善。”他和\xpinyin*{胡嗣瑗}都是善于争辩的,但是\xpinyin*{胡嗣瑗}出口或成文,只用些老古典,而他却能用一些洋知识,如\xpinyin*{墨索里尼}创了什么法西斯主义,日本怎么有个明治维新,英国《泰晤士报》上如何评论了中国局势等等,这是\xpinyin*{胡嗣瑗}望尘莫及的。\xpinyin*{陈宝琛}是我认为最忠心的人,然而讲到我的未来,绝没有\xpinyin*{郑孝胥}那种令我心醉的慷慨激昂,那种满腔热情,动辄声泪俱下。有一次他在给我讲《通鉴》时,话题忽然转到了我未来的“帝国”:\\

\begin{quote}
	“帝国的版图,将超越圣祖仁皇帝一朝的规模,那时京都将有三座,一在北京,一在南京,一在帕米尔高原之上……”\\
\end{quote}

他说话时是秃头摇晃,唾星四溅,终至四肢颤动,老泪横流。\\

有时,在同一件事上说的几句话,也让我觉出\xpinyin*{陈宝琛}和\xpinyin*{郑孝胥}的不同。在\xpinyin*{康有为}赐谥问题上,他两人都是反对的,\xpinyin*{陈宝琛}在反对之余,还表示以后少赐谥为妥,而他在发表反对意见之后,又添了这么一句:“\xpinyin*{戊戌}之狱,将来自然要拿到朝议上去定。”好像我不久就可以回紫禁城似的。\\

\xpinyin*{郑孝胥}和\xpinyin*{罗振玉}都积极为复辟而奔走活动,但\xpinyin*{郑孝胥}的主张更使我动心。虽然他也是屡次反对我出洋和移居旅顺、大连的计划的。\\

\xpinyin*{郑孝胥}反对我离开天津到任何地方去,是七年来一贯的。甚至到“九·一八”事变发生,\xpinyin*{罗振玉}带着关东军的策划来找我的时候,他仍然不赞成我动身。这除了由于他和\xpinyin*{罗振玉}的对立,不愿我被罗垄断居奇,以及他比罗略多一点慎重之外,还有一条被人们忽视了的原因,这就是:他当时并不把日本当做唯一的依靠;他所追求的东西,是“列强共管”。\\

在天津时代,\xpinyin*{郑孝胥}有个著名的“三共论”。他常说:“大清亡于共和,共和将亡于共产,共产则必然亡于共管。”他把北伐战争是看做要实行“共产”的。这次革命战争失败后,他还是念不绝口。他说:“又闹罢工了,罢课了,外国人的商业受到了损失,怎能不出头来管?”他的“三共论”表面上看,好像是他的感慨,其实是他的理想,他的愿望。\\

如果考查一下郑、罗二人与日本人的结交历史,郑到日本做中国使馆的书记宫是一八九一年,罗卖古玩字画、办上海《农报》,由此结识了给《农报》译书的日人藤田剑峰是在一八九六年,郑结交日人比罗要早五年。但是\xpinyin*{罗振玉}自从认识了日本方面的朋友,眼睛里就只有日本人,\xpinyin*{辛亥}后,他把复辟希望全放到日本人的身上,而\xpinyin*{郑孝胥}却在日本看见了“列强”,从那时起他就认为中国老百姓不用说,连做官的也都无能,没出息,中国这块地方理应让“列强”来开发,来经营。他比张之洞的“中学为体,西学为用”更发展了一步,不但要西洋技术,西洋资本,而且主张要西人来做官,连皇家的禁卫军也要由客卿训练、统领。不然的话,中国永远是乱得一团糟,中国的资源白白藏在地里,“我主江山”迟早被“乱党”、“乱民”抢走,以至毁灭。\xpinyin*{辛亥}革命以后,他认为要想复辟成功,决不能没有列强的帮忙。这种帮忙如何才能实现呢?他把希望寄托在“共管”上。\\

那时关于“列强”共管中国的主张,经常可以从天津外文报纸上看到。\xpinyin*{郑孝胥}对这类言论极为留意,曾认真地抄进他的日记、\xpinyin*{札}记,同时还叫他的儿子\xpinyin*{郑垂}译呈给我。这是一九二七年六月九日登在日文报纸《天津日日新闻》上的一篇:\\

\begin{quote}
	英人提倡共管中国\\

联合社英京特约通信。据政界某要人表示意见谓:中国现局,日形纷乱,旅华外国观察家曾留心考察,以为中国人民须候长久时期,方能解决内部纠纷,外国如欲作军事的或外交的干涉,以解决中国时局问题,乃不可能之事。其唯一方法,只有组织国际共管中国委员会,由英美法日德意六国各派代表一名为该会委员,以完全管理中国境内之军事。各委员之任期为三年,期内担任完全责任,首先由各国代筹二百五十兆元以为行政经费,外交家或政客不得充当委员,委员人才须与美国商(务)部长胡佛氏相仿佛。此外,又组织对该委员会负责之中外混合委员会,使中国人得在上述之会内受训练。\\
\end{quote}

\xpinyin*{郑孝胥}认为,这类的计划如果能实现,我的复位的时机便到了。\\

那年夏天我听了\xpinyin*{罗振玉}的劝说,打算到日本去,\xpinyin*{郑孝胥}就根据那篇文章勾起的幻想,向我提出了“留津不动,静候共管”的劝告。这是他记在日记里的一段:\\

\begin{quote}
	五月戊子二十四日(六月二十三日)。\xpinyin*{诣}行在。召见,询日领事约谈情形(即去日事)。因奏曰:今乘\xpinyin*{舆}狩于天津,皇帝与天下犹未离也。中原士大夫与列国人士犹得常接,气脉未寒,若去津一步,则形势大变,是为去国亡命,自绝于天下。若寄居日本,则必为日本所留,兴复之望绝矣!自古中兴之主,必借兵力。今则海内大乱,日久莫能安戢,列国通不得已,乃遣兵自保其商业。他日非为中国置一贤主,则将启争端,其祸益大。故今日皇上欲图中兴,不必待兵力也,但使圣德令名彰于中外,必有人人欲以为君之日。\\
\end{quote}

他提出过不少使“圣德令名彰于中外”的办法,如用我的名义捐款助赈,用我的名义编纂《清朝历代政要》,用我的名义倡议召开世界各国弭兵会议等等。有的我照办了,有的无法办,我也表示了赞许和同意。\\

我委任奥国亡命贵族\xpinyin*{阿克第}男爵到欧洲为我进行游说宣传,临行时,\xpinyin*{郑孝胥}亲自向他说明,将来如蒙各国支持“复国”,立刻先实行这四条政策:“一、设责任内阁,阁员参用客卿;二、禁卫军以客将统帅教练;三、速办张家口-伊犁铁路,用借款包工之策;四、国内设立之官办商办事业,限五年内一体成立。”\\

\xpinyin*{郑孝胥}的想法,以后日益体系化了。有一次,他说:“帝国铁路,将四通八达,矿山无处不开,学校教育以孔教为基础……。”我问他:“列强真的会投资吗?”他说:“他们要赚钱,一定争先恐后。臣当年承办瑷珲铁路,投资承包的就是如此,可惜朝廷给压下了,有些守旧大臣竟看不出这事大有便宜。”那时我还不知道,作为\xpinyin*{辛亥}革命风暴导火线的铁路国有化政策,原来就是\xpinyin*{郑孝胥}给盛宣怀做幕府时出的主意。假若我当时知道这事,就准不会再那样相信他。当时听他说起办铁路,只想到这样的问题:“可是\xpinyin*{辛亥}国变,不就是川、湘各地路矿的事闹起来的吗?”他附和说:“是的,所以臣的方策中有官办有商办。不过中国人穷,钱少少办,外国人富,投资多多办,这很公平合理。”我又曾问过他:“那些外国人肯来当差吗?”他说:“待如上宾,许以优待,享以特权,绝无不来之理。”我又问他:“许多外国人都来投资,如果他们争起来怎么办?”他很有把握地说:“唯因如此,他们更非尊重皇上不可。”\\

这就是由共管论引申出来的日益体系化的\xpinyin*{郑孝胥}的政策,也是我所赞许的政策。我和他共同认为,只有这样,才能取回我的宝座,继续大清的气脉,恢复宗室觉罗、文武臣僚、士大夫等等的旧日光景。\\

\xpinyin*{郑孝胥}在我出宫后,曾向\xpinyin*{段祺瑞}活动“复原还宫”,在我到天津后,曾支持我拉拢军阀、政客的活动,但是,在他心里始终没忘掉这个理想。特别是在其他活动屡不见效的情况下,他在这方面的愿望尤其显得热烈。这在使用\xpinyin*{谢米诺夫}这位客卿的问题上,分外地可以看出来。\\

当我把接见\xpinyin*{谢米诺夫}的问题提出来时,\xpinyin*{陈宝琛}担心的是这件事会引起外界的责难,\xpinyin*{郑孝胥}着急的却是怕我背着他和\xpinyin*{罗振玉}进行这件事。他对\xpinyin*{陈宝琛}说:“反对召见,反而使皇上避不咨询,不如为皇上筹一妥善谨密之策,召见一次。”结果,\xpinyin*{谢米诺夫}这个关系便叫他拉到手上了。\\

使他对\xpinyin*{谢米诺夫}最感到兴趣的,是谢和列强的关系。当\xpinyin*{谢米诺夫}吹嘘列强如何支持他,而各国干涉中国的政局之声又甚嚣尘上的时候,\xpinyin*{郑孝胥}认为时机来了,兴高采烈地给\xpinyin*{张宗昌}和\xpinyin*{谢米诺夫}撮合,让\xpinyin*{谢米诺夫}的党羽\xpinyin*{多布端}到蒙古举兵起事,并且亲自跑上海,跑青岛。他进行了些什么具体活动,我现在已记忆不清了,只记得他十分得意地写了不少诗。他的日记里有这样自我欣赏的描写:“晨起,忽念近事,此后剥极而复,乃乾旋坤转之会,非能创能改之才,不足以应之也。”“如\xpinyin*{袁世凯}之谋篡,\xpinyin*{张勋}之复辟,皆已成而旋败,何者?无改创之识则\xpinyin*{枘}凿而不合矣!”(一九二五年十一月)“诸人本极畏事,固宜如此!”“夜与\xpinyin*{谢米诺夫}。包文渊、\xpinyin*{毕瀚章}、\xpinyin*{刘凤池}同至国民饭店,……皆大欢畅,约为同志,而推余为大哥。”(一九二六年五月)\\

英国骗子\xpinyin*{罗斯},以办报纸助我复辟为名,骗了我一笔钱,后来又托\xpinyin*{郑孝胥}介绍银行贷款,\xpinyin*{郑孝胥}因罗是\xpinyin*{谢米诺夫}和\xpinyin*{多布端}的朋友,就用自己的存折作押,给他从银行借了四千元。\xpinyin*{郑垂}觉得\xpinyin*{罗斯}不可靠,来信请他父亲留心,他回信教训儿子说:“不能冒险,焉能举事?”后来果然不出他儿子所料,\xpinyin*{罗斯}这笔钱到期不还,银行扣了郑的存款抵了账。尽管如此,当\xpinyin*{罗斯}底下的人又来向郑借钱的时候,由于\xpinyin*{谢米诺夫}的关系,经\xpinyin*{多布端}的说情,他又掏出一千元给了那个骗子。当然,我的钱经他手送出去的,那就更多。被他讥笑为“本极畏事,固宜如此”的\xpinyin*{陈宝琛},后来在叹息“\xpinyin*{苏龛}(郑宇),\xpinyin*{苏龛},真乃疏忽不堪!”之外更加了一句:“慷慨,慷慨,岂非慷他人之慨!”\\

后来,他由期待各国支持\xpinyin*{谢米诺夫},转而渴望日本多对\xpinyin*{谢米诺夫}加点劲,他又由期待各国共管,转而渴望日本首先加速对中国的干涉。当他的路线转而步\xpinyin*{罗振玉}后尘的时候,他的眼光远比\xpinyin*{罗振玉}高得多,什么三野公馆以及天津日军司令部和领事馆,都不在他眼里;他活动的对象是直接找东京。不过他仍然没忘了共管,他不是把日本看做唯一的外援,而是第一个外援,是求得外援的起点,也可以说是为了吸引共管的第一步,为“开放门户”请的第一位“客人”。\\

他提出了到东京活动的建议,得到了我的赞许,得到了\xpinyin*{芳泽}公使的同意。和他同去的,有一个在日本朝野间颇有“路子”的日本人\xpinyin*{太田}外世雄。他经过这个浪人的安排,和军部以及黑龙会方面都发生了接触,后来,他很满意地告诉我,日本朝野大多数都对我的复辟表示了“关心”和“同情”,对我们的未来的开放政策感到了兴趣。总之,只要时机一到,我们就可以提出请求支援的要求来。\\

关于他在日本活动的详细情形,我已记不清了。我把他的日记摘录几段如后,也可以从中看出一些他在日本广泛活动的蛛丝马迹:\\

\begin{quote}
	八月\xpinyin*{乙丑}初九日(阴历,下同)。八点抵神户。\xpinyin*{福田}与其友来迎。每日新闻记者携具来摄影。偕\xpinyin*{太田}、\xpinyin*{福田}步至西村旅馆小憩,忽有\xpinyin*{岩日爱之助}者,投刺云:兵库县得\xpinyin*{芳泽}公使来电嘱招待,兵库县在东京未回,今备汽车唯公所用。遂同出至中华会馆。又至捕公庙,复归西村馆,即赴汽车站买票,至西京,入京都大旅馆。来访者有:大阪时事报社\xpinyin*{守田耕治}、\xpinyin*{大田}之友僧足利净回,\xpinyin*{岩田}之友\xpinyin*{小山内大六},为国杂社干事。与\xpinyin*{岩田}、\xpinyin*{福田}、\xpinyin*{太田}同至山东馆午饭。夜付\xpinyin*{本多吉}来访,谈久之。去云:十点将复来,候至十二点,竟不至。\\

\xpinyin*{丙寅}初十日。……将访\xpinyin*{竹本},遇于门外,遂同往。\xpinyin*{内藤虎}来谈久之。\xpinyin*{太田}之友\xpinyin*{松尾八百藏}来访,密谈奉天事。\\

\xpinyin*{丁卯}十三日。\xpinyin*{福田}以电话告:\xpinyin*{长尾}昨日已归,即与\xpinyin*{太田}、\xpinyin*{大七}走访之。\xpinyin*{长尾}犹卧,告其夫人今日匆来,遂乘电车赴大限。……\xpinyin*{岩田爱之助}与\xpinyin*{肃邸}四子俱来访。宪立(定之)密语余奉天事,消息颇急,欲余至东京日往访藤田正实、\xpinyin*{宇垣一成}。朝日、每日二社皆摄影,复与肃四子共摄一影,乃访住友经理\xpinyin*{小仓}君。……\\

\xpinyin*{庚午}十四日。\xpinyin*{长尾}来谈,劝取奉天为恢复之基。……\\

\xpinyin*{壬申}十六日。\xpinyin*{长尾雨山}以电话约勿出,当即来访,遂以汽车同游天满官金阁寺而至岚山。高峰峭立,水色甚碧,密林到顶,若无路可入者。入酒家,亦在林中,隐约见岩\xpinyin*{岫}压檐而已,饮酒食鱼,谈至三时乃去。\\

\xpinyin*{癸酉}十七日。……\xpinyin*{长尾}来赠画扇,送至圆山公园,\xpinyin*{左阿}、\xpinyin*{娿家}、\xpinyin*{狩野}、\xpinyin*{内藤}、\xpinyin*{近重}、\xpinyin*{铃木}皆至,顷之\xpinyin*{高濑}亦至,唯\xpinyin*{荒木}、\xpinyin*{内村}在东京未归。……\\

\xpinyin*{丙子}二十日。作字。雨。\xpinyin*{诣}\xpinyin*{长尾}辞行。……\xpinyin*{太田}来云,东京备欢迎者甚众,将先往约期。\\

\xpinyin*{辛巳}廿五日。十一时至东京下火车。至车站投刺者数十人。\xpinyin*{小田切}、\xpinyin*{高田丰村}、\xpinyin*{冈野}皆来帝国旅馆。雨甚大。\xpinyin*{岩田}、\xpinyin*{水野}梅晓亦来。\xpinyin*{冈野}自\xpinyin*{吴佩孚}败后\xpinyin*{遯}而为僧。夜宿于此。\\

\xpinyin*{壬午}二十六日。……\xpinyin*{水野}谈日政府近状颇详,谓如床次、\xpinyin*{后藤}、细川侯、近卫公,皆可与谈。\\

\xpinyin*{癸未}二十七日。……送过\xpinyin*{水野},复同访床次。床次脱离民主党而立昭和俱乐部,将为第三党之魁。\xpinyin*{岩田}来。\xpinyin*{小田切}来。\xpinyin*{大田}、\xpinyin*{白井}、\xpinyin*{水野}、\xpinyin*{佃信夫}来。\xpinyin*{山田}来。\xpinyin*{汪荣宝}来。……夜赴近卫公之约,坐客十余人,\xpinyin*{小田切}、\xpinyin*{津田}、\xpinyin*{水野}、\xpinyin*{太田}皆在坐。近卫询上近状,且极致殷勤。……\\

\xpinyin*{甲申}二十九日。……\xpinyin*{川田瑞穗}者称,\xpinyin*{长尾雨山}之代理人,与\xpinyin*{松本洪}同来约九月初八日会宴,坐客为:\xpinyin*{平沼骐一郎},枢密院副议长;\xpinyin*{桦山资英},前内阁秘书长;\xpinyin*{牧野谦次郎},\xpinyin*{能文},早稻田教授;\xpinyin*{松平康国},早稻田教授;\xpinyin*{国分青崖},诗人;\xpinyin*{田边碧堂},诗人;\xpinyin*{内田周平},\xpinyin*{能汉文}。此外尚十余人。……\xpinyin*{岩田}与\xpinyin*{肃邸}第十八子宪开来访,今在士官学校。……\xpinyin*{津田静枝}海军大佐邀至麻布区日本料理馆,为海军军令部公宴。主席者为\xpinyin*{米内}少将,坐客为:\xpinyin*{有田八郎},\xpinyin*{水野梅晓},\xpinyin*{中岛}少将,\xpinyin*{园田}男爵(东乡之婿),\xpinyin*{久保田久晴}海军中佐等。……九月\xpinyin*{丙戌}朔。\xpinyin*{太田}来。参谋本部总长\xpinyin*{铃木},次长南,以电话约十时会晤。与\xpinyin*{大七}、\xpinyin*{大田}同往。\xpinyin*{铃木}询上近状,且云:有恢复之志否?南次长云:如有所求,可以见语。对曰:正究将来开放全国之策,时机苟至,必将来求。\xpinyin*{吉田茂}外务次官约午饭,座中有:清浦子爵\xpinyin*{奎吾},\xpinyin*{冈部长景}子爵,\xpinyin*{高田}中将,\xpinyin*{池田}男爵,\xpinyin*{有田},\xpinyin*{岩村},\xpinyin*{水野},\xpinyin*{太田}等。……\\

\xpinyin*{丁亥}初二日。……\xpinyin*{岩田偕宪开}、\xpinyin*{李宝琏}、\xpinyin*{刘牧蟾}来访。李刘皆在士官学校。……\\

\xpinyin*{庚寅}初五日。……\xpinyin*{水野}、\xpinyin*{太田}来。与\xpinyin*{水野}同访\xpinyin*{后藤新平},谈俄事良久。……\\

\xpinyin*{癸巳}初八日。……\xpinyin*{工藤}邀同至\xpinyin*{白井新太郎}宅,\xpinyin*{晤高山}中将,\xpinyin*{野中}、\xpinyin*{多贺二}少将,\xpinyin*{田锅}、\xpinyin*{松平}皆在座,颇询行在情形。\\

\xpinyin*{戊戌}十三日。\xpinyin*{太田}送至神户登长崎九,\xpinyin*{长尾雨山}自西京来别。\xpinyin*{富冈}、\xpinyin*{福田}皆来。十一点半展轮。……\\
\end{quote}

他在日本,被当做我的代表,受到各种热心于恢复清朝的人物的接待。其中有不少原是我的旧交,例如\xpinyin*{高田丰村}是前天津驻屯军司令官,\xpinyin*{有田八郎}和\xpinyin*{吉田茂}做过天津总领事,\xpinyin*{白井}是副领事,\xpinyin*{竹本多吉}是在北京时把我接进日本兵营的那位大任。\xpinyin*{岩田爱之助}就是在我窗外放枪的那位黑龙会会员,\xpinyin*{佃信夫}则是不肯在总领事有田面前谈“机密”的那位黑龙会重要人物。不管他们在中国时怎样不和,这时却彼此融洽无间地共同接待着“\xpinyin*{郑大臣}”。除了这些过去曾直接出头露面的以外,那些原居于幕后的大人物,如后来做过首相、陆相等要职的近卫(\xpinyin*{文麿})、\xpinyin*{宇垣一成}、\xpinyin*{米内光政}、\xpinyin*{平沼骐一郎}、\xpinyin*{铃木}(贯太郎)、南(次郎),以及在第二次世界大战后上台的\xpinyin*{吉田茂}等人,还有一些出名的政客、财阀,此时全都出了面。也许\xpinyin*{郑孝胥}和这些人会谈时,他的“开放全国之策”引起的反应使他太高兴了,所以在伪满成立以后,第一批“客人”已经走进了打开的“门户”,他仍然没有忘记共管的理想,一有机会便向外面宣传“门户开放,机会均等”。这犹如给强盗做底线的仆人,打开了主人家的大门,放进了一帮强盗,当了一帮强盗的大管事,尤感不足,一定还要向所有各帮强盗发请帖,以广招徕。这自然就惹恼了已经进了门的强盗,一脚把他踢到一边。
