\fancyhead[LO]{{\scriptsize 1931-1932: 到东北去 · 日本人意见分歧}} %奇數頁眉的左邊
\fancyhead[RO]{} %奇數頁眉的右邊
\fancyhead[LE]{} %偶數頁眉的左邊
\fancyhead[RE]{{\scriptsize 1931-1932: 到东北去 · 日本人意见分歧}} %偶數頁眉的右邊
\chapter*{日本人意见分歧}
\addcontentsline{toc}{chapter}{\hspace{1cm} 日本人意见分歧}
\thispagestyle{empty}
还不等静园里商量出一致意见来,日本驻津总领事馆的后藤副领事,第二天便找上了门。他们对我去日本兵营的事全知道了。总领事馆表示,他们对我的心情和处境是完全理解的,但我最好是慎重从事,现在不要离开天津;他们负有保护的责任,不得不作这个劝告。\\

从这天起,这位后藤副领事不是直接来见我,就是找\xpinyin*{陈宝琛}舅甥或是\xpinyin*{郑孝胥}父子,进行劝阻。另方面,日本驻屯军的通译官吉田,却一再向我宣传,说日本军方决心支持我上台,我最好立刻动身出行。\\

这时我对于日本军政双方有了新的看法,和\xpinyin*{陈宝琛}那一伙人的看法有了分歧。\xpinyin*{陈宝琛}一向认为文人主政是天经地义,所以他只肯联络日本芳泽公使,他的外甥只肯和领事馆以及东京的政友会人物来往。这时他坚决主张,如果东京方面没有表示,千万别听军人们的话。我的看法则不同,认为现在能决定我的命运的不是日本政客,而是军人。我并没有什么高深的见解和情报,我是从当前摆着的事实上看出来的。我看到日本人一方面在外交上宣称,准备和南京政府通过和平途径解决“中日纠纷”,另方面关东军却一路不停地前进,攻击退却着的中国军队。我那时虽然还不太明白,这和蒋介石、汪精卫们一边嚷着抵抗,一边把国土让给敌人,原都是用以欺世的两面手法,但我能看出决定问题的还是日本军人。\xpinyin*{陈宝琛}指出国际列强的暧昧态度可虑,这也和我的感觉不同。我去过日本兵营后不多天,英国驻津军队司令官牛湛德准将忽然来到静园访问。他对“九·一八”事变给我造成的机会,表示了“私人的祝贺”,并且说:“如果陛下能在伟大的满洲重新登极,陛下的仆人牛湛德,愿意充当龙旗下的一名士兵。”这话使我更加相信\xpinyin*{郑孝胥}说的英方袒日的消息。牛湛德来访之后,庄士敦也突然和我久别重逢,据他说这回是代表英国外交部,来办理废款和归还威海卫的余留问题,顺便前来看望看望我。他为我的“前途”表示高兴,同时请我为他的著作《紫禁城的黄昏》书稿作一篇序文,他说,他将在这书的最末添上一章,叫做“龙归故里”。\\

\xpinyin*{刘骧业}和\xpinyin*{佟济煦}先后从东北带来的消息,对我也是一种鼓舞。\xpinyin*{佟济煦}先回来说,他和沈阳的遗老袁金销等人见了面,都认为时机已至,不必迟疑。接着\xpinyin*{刘骧业}也来了,虽然他没有能见到\xpinyin*{内田康哉}和本庄繁,这有点令人失望,但他见到了板垣和金梁,证实了罗振玉和上角利一并不是骗人的。金梁对他表示的尤其乐观:“奉天一切完备,惟候乘\xpinyin*{舆}临幸。”他也去过吉林,证实罗振玉说的不错,日本军队已控制了全省,熙洽等人随时准备响应复辟。\\

除了这些之外,当时出现的一些谣言也在促使我急于动身。那时天津的新闻界消息非常灵通,我去日本兵营的事,很快就传到了社会上,有的报纸甚至报道了我已乘轮到了东北。与此同时,不知从哪里传来谣言,说中国人要对我有不利的举动。因此我更觉得不能在天津呆下去了。\\

我派郑垂去拜会日本总领事桑岛,说既然时机不至,我就不一定一直去奉天,不妨先到旅顺暂住,这总比在天津安全一些。桑岛立刻表示,到旅顺去也不必要。他叫郑垂转告我,满铁总裁\xpinyin*{内田康哉}也不同意我现在动身,内田是日本政界的老前辈,日本军部对他也是尊重的,因此还是慎重从事的好,至于安全,他愿负完全责任。最后说,他要和驻屯军司令官香椎交换一下意见。第二天,副领事来找郑垂说,桑岛和香椎商量过了,意见一致,都不主张我现在离开天津。\\

我听了这消息觉得非常胡涂,为了弄清真相,不得不把那位司令部的通译官请来。不料吉田的回答却是,所谓总领事和司令官的会商,根本没这么回事,香椎司令官主张我立刻随上角利一走。他给我出了个主意,由我亲笔写信给司令部,把坚决要走的态度告诉他。我在胡里胡涂中写了这封信。可是不知怎么弄的,日本总领事又知道了,连忙来找\xpinyin*{陈宝琛}、\xpinyin*{郑孝胥}探听有没有这回事,那封信是真的还是假的?……\\

我对日本军政两界的这种摩擦非常生气,可是又没什么办法可想。这时二次去东北的\xpinyin*{刘骧业}来了信,说是探得了关东军司令官本庄的真正意思:现在东北三省尚未全部控制,\xpinyin*{俟}“三省团结稳固,当由内田请上临幸沈阳”。既然决定命运的最高权威有了这样的表示,我只好遵命静候。\\

从那以后,我多少明白了一点,不仅天津的领事馆与驻屯军之间意见分歧,就连关东军内部步调也不太一致。我对某些现象不由得有些担心:前恭亲王溥伟在日本人的保护下祭祀沈阳北陵,辽宁省出现了“东北地方维持会”的组织,旧东北系重要人物减式毅在受着关东军的“优待”,前民国执政段祺瑞的行踪消息,又出现于报端,传闻日本人要用他组织北方政权。假如我当时知道日本人曾一度想用段祺瑞,又一度要用“东北行政委员会”的空架子,又一度要用溥伟搞“明光帝国”(这是很快就知道的),以及其他的一些可怕的主意,我的心情就更加难受了。\\

我给了罗振玉和上角利一“暂不出行”的答复之后,度日如年地等着消息。在等待中,我连续发出“谕旨”,让两个刚从日本士官学校毕业的侄子宪原、宪基到东北宣抚某些蒙古王公,赏赐首先投靠日本占领军的张海鹏、贵福等人以美玉。我根据日本武官森纠的请求,写信给正和张海鹏对抗的马占山和具有民族气节的另一些蒙古工公,劝他们归降。我封张海鹏为满蒙独立军司令官,马占山为北路总司令,贵福为西路总司令,赐宪原、宪基等以大佐军衔。我预备了大批写着各种官衔的空白封官谕旨,以备随时填上姓名……\\

特别应当提到的一件事,是我按照\xpinyin*{郑孝胥}的意见,直接派人到日本去进行活动。自从罗振玉遭到我的拒绝,快快离去之后,\xpinyin*{郑孝胥}一变表面上的慎重态度,由主张观望变成反对观望,主张积极行动了。这时他认为在日本和铃木、南次郎以及黑龙会方面所谈的那个时机已经到来,是提出要求的时候了,同时,他大概也看出了有人在和我竞争着,所以主张派人到东京去活动。我对这种突然的变化不但不惊异,反而十分高兴。我背着\xpinyin*{陈宝琛},采纳了\xpinyin*{郑孝胥}的意见,派了日本人远山猛雄去日本,找刚上台的陆相南次郎和“黑龙会”首领头山满进行联络。我根据\xpinyin*{郑孝胥}起的草,用黄绢亲笔给这两个大人物各写了一封信。后来,一九四六年在东京国际法庭上南次郎拿出了这封信,给律师作为替他辩护的证据。我因为害怕将来回到祖国会受到审判,否认了这封信,引起了一场轩然大波。可惜此信的原文现在没有得到,只好暂时从日本书籍上转译如下:\\

\begin{quote}
	此次东省事变,民国政府处措失当,开衅友邦,涂炭生灵,予甚悯之。\\

兹遣皇室家庭教师远山猛雄赴日,慰视陆军大臣南大将,转达予意。我朝以不忍目睹万民之疾苦,将政权让之汉族,愈趋愈紊,实非我朝之初怀。\\

今者欲谋东亚之强固,有赖于中日两国提携,否则无以完成。如不彻底解决前途之障碍,则殷忧四伏,永无宁日,必有赤党横行,灾难无穷矣。\\

\begin{flushright}
	辛未九月一日(十月十一日)\\

宣统御玺\\

今上御笔       \xpinyin*{郑孝胥}(签字)\\
\end{flushright}
\end{quote}

我就这样地一边等待,一边活动着。这封信由远山猛雄带走了三个多星期之后,我终于等到了\xpinyin*{郑孝胥}在自己的日记里所写的这一天:\\

\begin{quote}
	九月辛百二十三日(十一月二日)。\xpinyin*{诣}行在。召对。上云:“\xpinyin*{商衍瀛}来见,言奉天吉林皆望这幸;吉田来言,土肥原至津,与司令部秘商,谓宜速往。”对曰:“土肥原为本庄之参谋,乃关东军中之要人,果来迎幸,则不宜迟。”明日以告领事馆。夜召土肥原。……\\
\end{quote}
