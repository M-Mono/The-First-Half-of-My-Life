\fancyhead[LO]{{\scriptsize 1960: 新的一章}} %奇數頁眉的左邊
\fancyhead[RO]{} %奇數頁眉的右邊
\fancyhead[LE]{} %偶數頁眉的左邊
\fancyhead[RE]{{\scriptsize 1960: 新的一章}} %偶數頁眉的右邊
\chapter*{新的一章}
\addcontentsline{toc}{chapter}{\hspace{1cm}新的一章}
\thispagestyle{empty}
列车奔驰着。外面,是白雪覆盖着的平原,光明、辽阔,正如展现在我面前的生活前程。车内,我身前身后都是普通的劳动人民。这是我平生第一次和他们坐在一起,坐在同一列客车上。我将同他们共同生活,共同建设,我将成为他们中间的一个,不,我现在就是他们中间的一个了。\\

在抚顺上车不久,身旁的一件事情立刻说明我是迈进了一个什么样的社会,我是列身于什么样的人之间。列车员和一位女乘客搀扶着一个小姑娘,走进我们这个车厢找座位。在我身后有个空座,空座旁的一位乘客把自己的位子一齐让了出来,给她们坐。那妇女让孩子在座位上躺好,自己侧身偎靠着她,神情十分焦灼。邻座有人向她探问,孩子是不是有病?有病为什么还要出门?那妇女的回答是出乎人们意料的。原来她是车站附近的一个小学的教师,小姑娘是她的学生,刚才在课堂上小姑娘突然腹痛难忍,学校卫生人员怀疑是阑尾炎,主张立即送医院。小姑娘的父母都在很远的矿上工作,通知他们来带孩子去看病怕来不及,直接把孩子送到能动手术的矿上医院,也费时间,于是女教师毅然做出决定,立刻带孩子搭这趟客车去沈阳。站台人员让她先上车后补票,同时叫她放心,他们将打电话告诉沈阳照应她们。这个简单的插曲叫我想起了\xpinyin*{陶渊明}说的“落地为兄弟,何必骨肉亲”的胸怀,在今天已不是少数人的胸怀。我更想起\xpinyin*{孟子}说的“老吾老以及人之老,幼吾幼以及人之幼”,这在今天也已成为现实。我现在所迈进的社会,我现在所参加的行列,原来比我所想象的更高更美呵!\\

十二月九日,我来到祖国的首都。这是我别离了三十五年的故乡。在辉煌壮丽的北京站台上,我看见了三年多不见的五妹和二十多年不见的四弟。我和他们紧紧地握着手,听到叫喊“大哥”的声音。这是前半生中,妹妹弟弟们从来没对我叫过的称呼。我从这一声称呼中,感到了在自己家族中也开始了新的生命。\\

我告别了伴随我们来的李科员,也告别了同行的老孟。老孟是我们同所的\xpinyin*{蒋介石}集团的八名蒙受特赦的战犯之一(连我和前伪满将官\xpinyin*{郭文林},抚顺管理所赦出共十人)。他和前来迎接的妻子走了。四弟给我提起那只黑皮箱,五妹和老万走在我两旁,我们一起走出站台。在站外,我对着站台大钟,掏出我的怀表。离开抚顺前,所长从我献交给政府的那堆东西里面拣出这只表,叫我收下,我说这是我从前用剥削的钱买的,我不能要。所长说现在是人民给你的,你收下吧!这是我在一九二四年从父亲家里逃人东交民巷那天,为了摆脱\xpinyin*{张文治},在乌利文洋行想主意时买的那只法国金表。从那一刻起,开始了我的可耻的历史。如今,我让它也开始一个新生命,用北京的时间,拨正了它的指针。\\

所长发给我这只表的那天,对我们蒙赦的十个人说,你们回到了家,见了乡亲们和家里的人,应该给他们道个歉,因为过去对不起他们。他说:“我相信,家乡的人会原谅你们,只要你们好好地做人,勤勤恳恳地为人民服务。”我到了五妹和老万的家里,所长的话完全证实了,同院的每个人对我都是和蔼可亲的。第二天早晨,我很想和这些邻居们一起做点什么,我看到有人拿着笤帚去扫胡同,就参加了打扫。我一直扫到胡同口,回来的时候,找不着家门了,结果走进一个陌生的人家。这家人明白了是怎么回事,便十分热情地把我送了回来,并且告诉我用不着道谢,说“咱们还是街坊,就不是街坊,新社会里帮这点忙又算什么呀!”\\

我还见到了七叔七婶,堂兄堂弟和妹妹、妹夫们。我从七叔这里知道了家族最新的兴旺史,知道了他在人民代表大会上关于少数民族地区视察情况的发言。我听到了伒大哥的古琴,看到了他给我写的字。他的书法造\xpinyin*{诣}确实又达到了新的水平。我还欣赏了\xpinyin*{僴}五哥的花鸟新作。我去看了二妹,这时她已办起了街道托儿所,据现任邮电部门工程师的二妹夫说,二妹现在忙得连头晕的老病也没有了。三妹夫妇,四妹、六妹夫妇,七妹夫妇,我也都看到了,三妹夫妇在区政协正参加学习,四妹在故宫档案部门工作,六妹夫妇是一对画家,七妹夫妇是教育工作者。更激动人心的则是第二代。过春节的那天,数不清的红领巾拥满了七叔的屋里屋外。在已经成为青年的第二代中,我见到了立过功勋的前志愿军战士、北京女子摩托车冠军、登山队队长、医生。护士、教师、汽车司机。更多的是正在学着各种专业的和读着中学的学生。这里面有共产党员,共青团员,而其余的人无一例外都在争取获得这个光荣的称号。这些成长起来的青年,又是那些带红领巾的弟弟妹妹们心目中的榜样。\\

我还会见了许多旧时代的老朋友。\xpinyin*{商衍瀛}在卧榻上和我见了面。他是文史馆的馆员,因为老病,说话已不清楚。他见了我,面容似乎还有点拘谨的严肃,挣扎着要起来。我拉着他的手说:“你是老人而且有病,应该躺着休息。我们是新社会的人,现在的关系才是最正常的关系。等你好了,一块为人民服务。”他脸上拘谨的神色消失了,向我点头微笑,说:“我跟着你走。”我说:“我跟着共产党走。”他说:“我也跟着共产党走。”我更见到了当过太监的老朋友,知道了他们许多人的近况。他们正在民政局为他们专办的养老院中安度晚年。\\

我第一天见到的人差不多都说:“你回来了,要到各处去看看,你还没逛过北京呢!”我说:“我先去天安门!”\\

天安门广场,我是早已从电影、报刊以及家信中熟悉了的。我从银幕上看到过举着各项建设成绩标牌的游行队伍,在这里接受毛主席的检阅;我还看到过这里节日的狂欢活动。我从报刊上还看到过交通民警在这里领着幼儿院的孩子们过马路,看到过停在这里的“红旗”牌和“东风”牌小轿车。我知道了人民大会堂的巨大工程,是在十个月之内完工的,知道了来自世界各国的外宾在这里受到了什么样的感动。今天,我来到了这个朝思暮想的地方。\\

在我面前,巍峨的天安门是祖国从苦难到幸福的历史见证,也是旧\ruby{溥仪}{Pu I}变成新\ruby{溥仪}{Pu I}的见证。在我左面,是庄严壮丽的人民大会堂,祖国大家庭的重大家务在那里做出决定,其中也有使我获得了新生的决定。在我右面是革命博物馆,在我后面矗立着革命英雄纪念碑,它们告诉人们,一个多世纪以来有多少英雄烈士进行了什么样的艰巨斗争,才给我们争得了今天的果实,而我也成了其中的一个分享者!\\

在天安门广场上,我平生第一次满怀自由、安全、幸福和自豪地散着步。\\

我和五妹、俭六弟缓步西行。走到白身蓝顶的民族文化宫的时候,五妹关心地说:“大哥累不累?这是头一回走这么多路吧?”我说:“不累,正因为是头一回,特别不累。”\\

“头一回”这三个字充满了刚开始的新生活中。“头一回”是很不方便的,但我只觉得兴奋,并不因此有什么不安。\\

我头一回到理发店去理发——严格地说,这是第二次,因为三十多年前我在天津中原公司理过一次,但是这一次在理发店遇到的事还是头一回。我一坐上座位,就发现了在哈尔滨百货店里看到过的叫不出名称的东西。我问理发员,旁位上呜呜响的是什么,他说:“吹风。”我问:“先吹风还是先理发?”他一听,怔住了:“你没理过发吗?”他还以为我开玩笑哩!后来弄明白了,我们都不禁大笑起来。等到我头上也响起了那呜呜之声时,我心里更乐了。\\

我头一回坐公共汽车,给俭六弟造成了一场虚惊。我排队上车,看到人们让老人小孩先上,我也把身旁一位妇女让了上去,却不知这是位售票员。她看我不上,就跨上了车,车门随着关上,车子也开走了。过了一会儿,俭六弟从下一站下了车跑来,我们俩离着还老远就彼此相对大笑起来。笑过之后,我信心十足地对他说:“不用担心,决出不了事!”在这样多人的关怀下,我有什么可担心的呢?就在这天上午,我从三妹家附近一个商店里,刚找回来昨天丢在那里的一个皮夹子。难道我这个人还会丢了吗?\\

北京市民政局为了帮助我们了解北京,熟悉生活,组织了特赦后住北京的一些人,包括从前的国民党将军\xpinyin*{杜聿明}、\xpinyin*{王耀武}、\xpinyin*{宋希濂}等人,进行了一系列的参观。我们看了一些新建的工厂。扩建的各种公用事业以及城市的人民公社等单位,历时约两个月。最后,经同伴们的请求,游了故宫,由我临时充当了一次解说员。\\

令我惊异的是,我临离开故宫时的那副陈旧、衰败的景象不见了。到处都油缮得焕然一新,连门帘、窗帘以及床慢、褥垫、桌围等等都是新的。打听了之后才知道,这都是故宫的自设工厂仿照原样重新织造的。故宫的玉器、瓷器、字画等等古文物,历经北洋政府和国民党政府以及包括我在内的监守自盗,残剩下来的是很少了,但是,我在这里发现了不少解放后又经博物院买回来或是收藏家献出来的东西。例如,\xpinyin*{张择端}的《清明上河图》,是经我和\ruby{溥杰}{Pu Giye}盗运出去的,现在又买回来了。\\

在御花园里,我看到那些在阳光下嬉戏的孩子,在茶座上品茗的老人。我嗅到了古柏喷放出来的青春的香气,感到了这里的阳光也比从前明亮了。我相信故宫也获得了新生。\\

一九六零年三月,我被分配到中国科学院植物研究所的北京植物园,开始了每天半日劳动、半日学习的生活。这是我走上为人民服务岗位前的准备阶段。在技术员的指导下,我在温室里学习下种、育苗、移植等等工作。其余的半天有时学习,有时进行这本书的写作。\\

我在前半生中,不知“家”为何物。在抚顺的最后几年里,我开始有了“家”的感觉。到了植物园不久,我觉得又有了第二个“家”。我处在自上而下互助友爱的气氛中。有一次我从外面游逛之后回来,发现那只表不见了,不免十分惋惜,觉得在这么长的路线上,无法去寻找,只好作罢。同屋的总务员老刘知道了这件事,他本来正该休息,连休息也忘了,问清了我游逛的路线,立刻就出去了。许多人都知道了这件事,休息着的人都去找表,我被弄得很不好意思。后来老刘从四季青人民公社一个大队的食堂前找到了它,非常高兴地拿了回来。这时,我觉得我接过来的不是一只表,而是一颗火热的心。\\

这年的夏季,植物园里建立了民兵,每天进行操练。我报名参加,别人都说我年岁超过了标准。我说:“作为祖国大家庭的一员,我也应当站在保卫祖国的岗位上。”后来人们被我说服了,我参加了操练,当上了一名超龄民兵。\\

我站在植物园的民兵排里,心里想着我们的队伍有多大。这里是几十个人,但整个民兵队伍却是若干若干万。我想,我什么时候才可以列入那个像洪流的队伍中,在天安门前经过呢?\\

这个愿望很快得到了实现。我参加了支援日本人民反对“日美安全条约”斗争的百万人游行的队伍。这不仅是一百万,而是全世界参加共同斗争的千百万人的一部分。我们高声呼着口号,走过了天安门。在天安门上刻着我们的心声:“中华人民共和国万岁!”“全世界人民大团结万岁!”\\

从此我开始了社会活动。从这些活动中,我感觉到自己同全国人民,全世界争取和平、民主、民族独立和社会主义的人民连在一起了。\\

一九六零年十一月二十六日,我拿到了那张写着“\ruby{爱新觉罗}{Aisin Gioro}·\ruby{溥仪}{Pu I}”的选民证,我觉得把我有生以来所知道的一切珍宝加起来,也没有它贵重。我把选票投进了那个红色票箱,从那一刹那间起,我觉得自己是世界上最富有的人。我和我国六亿五千万同胞一起,成了这块九百六十万平方公里上地上的主人。从这块土地上伸向世界各地被压迫人民和被压迫民族的手,是一只巨大的可靠的手。\\

一九六一年三月,我结束了准备阶段,走上了为人民服务的正式岗位,在全国政协文史资料研究委员会担任专员职务。我做了一名文史工作者。\\

我参加的这部分工作是处理清末和北洋政府时代的文史资料。我在自己的工作中,经常遇到我所熟悉的名字,有时还遇到与我的往事牵连着的历史事件。资料的作者大多是历史事件的亲历或目击者。对这段时期的历史来说,我和他们都是见证人。我从丰富的资料中,从我的工作中,更清楚地看出时代的变化。那些被历史所抛弃的人物——\ruby{叶赫那拉}{Yehe Nara}氏(\xpinyin*{慈禧})、\xpinyin*{袁世凯}、\xpinyin*{段祺瑞}、\xpinyin*{张作霖}等等在当时似乎是不可一世的,被他们宰割压榨的人民似乎是无能为力的。像\xpinyin*{胡适}之流的文人们曾为他们捧场,遗老、遗少们曾把复辟幻想寄托在他们身上,而他们自己更自吹为强大,认为他们背后的列强是永远可恃的依靠。但是,他们都是纸老虎,终于被历史烧掉了。历史,这就是人民。“看起来,反动派的样子是可怕的,但是实际上并没有什么了不起的力量。从长远的观点看问题,真正强大的力量不属于反动派,而是属于人民。”我的经验使我接受了这项真理,我的工作使我更加相信了这项真理。我还要通过我的工作和我的见证人的身份,向人民宣扬这项真理。\\

我在工作之余,继续写我的前半生。\\

为了写作,我看了不少资料。我的工作单位给了我种种便利,供给我许多宝贵的文史资料,我在许多外界朋友的热情帮助下看到了许多图书、档案部门的宝贵材料,得到了许多专门调查材料。有的材料是不相识的朋友给我从珍贵的原件中一字一字抄下来的,有的材料是出版界的同志给我到远地调查核实的,有的材料是几位老先生根据自己的亲历目击认真回忆记录下来的。有不少难得的资料则是档案、图书部门提供的。特别要提到的是国家档案馆、历史博物馆、北京图书馆和首都图书馆的同志特意为我设法寻找,以及专门汇集的。我受到这样多的关怀和支援很感不安。其实,这在我们的国家里,早已是平常的现象了。\\

在我们的国家里,只要做的是有益于人民的事情,只要是宣扬真理,就会得到普遍的关心和支持,更不用说到党和政府了。\\

我的写作也引起了许多外国朋友的兴趣。曾有许多外国记者和外国客人访问过我,问我前半生的经历,特别注意打听我十年来的改造情形。一位拉丁美洲的朋友对我说:“从这件事情上,我又一次感到了\xpinyin*{毛泽东}思想的伟大。把你的事情快些写成书吧!”一位亚洲的朋友说:“希望你这本书的英文版出版后立刻送我一部,我要把它译成我国文字,让我国人民都看到这个奇迹。”\\

一九六二年,我们对来自国内外的困难所进行的艰巨斗争,取得了辉煌的成就。这一年对我来说,还有更多的喜事临门。四月间,我被邀列席人民政协全国委员会,并旁听了全国人民代表大会关于祖国建设的报告。五月一日,我和我的妻子\xpinyin*{李淑贤}\footnote{\xpinyin*{李淑贤}(1925-1997),是清朝末代皇帝\ruby{溥仪}{Pu I}的第五任也是最后一任妻子。汉族人,曾经担任医院工友。在\ruby{溥仪}{Pu I}坐牢十年受到“革命教育”与“思想改造”,获得特赦令予以释放后,李淑贤经人介绍遇见溥仪,与\ruby{溥仪}{Pu I}在1962年结婚,而她自己之前曾经结婚两次。他们没有子女。在其夫\ruby{溥仪}{Pu I}于1967年逝世后,\xpinyin*{李淑贤}从公众视野引退。}建立了我们自己的小家庭。这是一个普通的而对我却是不平凡的真正的家庭。\\

这就是我的新的一章。我的新生就是这样开始的。看看我的家,看看我的选民证,面对着无限广阔的未来,我永远不能忘记我的新生是怎样得到的。\\

我在这里要补充一段,关于给我新生命的伟大的改造罪犯政策的故事。用我瑞侄的话说:“这不写到书里是不行的。”\\

一九六零年夏天,我和小瑞游香山公园,谈起了我们每个人最初的思想变化,哪一件事引起了思想的最初震动。\\

小瑞先谈了小固和小秀。据他知道,小固在绥芬河车站上发现中国列车由中国司机驾驶,感到了第一次的震动,小秀则是从沈阳车站上群众对一个失掉一只手的女工的迎接,感到过去自己生活的无味。说到他自己,他说:“我难忘的事情很多。第一件事是我刚干活不久,有一回揩窗子打破了一块玻璃。玻璃掉在地上,看守员听声就跑来了,我吓得要死,谁知他过来问我:你伤着了没有?我说人没伤,可是玻璃破了。他说,玻璃破了不要紧,下次留神别伤着人。”\\

“这类事我也遇到过。”我说,“可是对我说来,最初我最关心的是生死问题,是宽大政策对我有无效验的问题。让我最初看到生机的,从此一步步看到希望的,是交出了那箱子夹底的东西后,受到了出乎意料的宽免。说起这个,还不能不感激你的帮助。”\\

“我的帮助?”小瑞睁大了眼睛,“你还不知道那是怎么回事吗?难道所长没跟你说吗?”\\

“说过这件事。检举认罪时,因为小固质问,我在大会上交代之后,向所长做了检讨,过了年,我又向所长说,我交东西时候没敢说收过你的纸条,是因为怕你受处分。所长说,这件事他全知道,是他让你写那纸条来帮助,以便促使我主动交代。这是所长的苦心,但也有你的帮助啊!”\\

“这么说你还不知道其中详细情形。你根本不知道,写那纸条本来是不合我心意的,我的意思是搜查你,没收东西,好好惩罚你一下。可是……这件事我得告诉你,这不写到你的书里是不行的!”\\

这件事的详细过程,我这时才明白。原来小瑞早就向所方谈出了我的箱底的秘密,要求所方搜查没收。所长却不这么干,他说:“搜查是很容易的,但这并不见得利于他的改造。等等吧,搜查不如他自动交代,要他自觉才好。”以后等了好久,小瑞又找所长谈,要求搜查。所长说,每个人思想发展速度不同,不能急。共产党人相信,在人民掌握的政权下,大多数罪犯是可以改造的,但每人有每人的过程。问题不在于珠宝和监规,而是要看怎样更利于对一个人的改造。所长说:“你要知道,由于他的特殊身份,他很难立即相信政府的坦白从宽的政策。如果我们去搜查了他,这就是让他失去了一次体验政策的机会。还是把主动让给他吧!你着急搜查,不如用用脑子设法来促进他自觉。”结果,就想出了由他写纸条给我的办法。纸条递出之后,多日不见动静,小瑞又急了,对所长说:“\ruby{溥仪}{Pu I}这人至死不悟,既然毫不自觉,为什么不搜他?”所长说:“原来不能着急,现在更不能着急。”后来的情形就是那样,结果是我着急了,交出了那批东西。我从那时开始看到了一条新的出路。\\

“从那时起,我就明白了,政府是坚定地相信多数人可以改造的。”小瑞激动地说,“你自己知道,那时你还一个劲儿地对抗。欺骗,可是所方早知道了你那些事。我们几个人还在检察人员到来之前,全都告诉了政府!可是从那时起,所方就相信你是可以改造的,就为你的学习、改造操着心了。”\\

我站在香山的山腰上,遥望太阳照耀着的北京城,我心中十年来的往事又被一件件勾起。我想起老所长的花白头发,年轻的副所长的爽朗语音,我想起了每位看守员,每位大夫、护士,每位所方人员。在我欺骗他们的时候,在我用各种可耻的方法进行对抗的时候,在我完全暴露出自己的无知、无能、愚蠢的时候,在我对自己都已感到绝望到极点、不能活下去的时候,他们,这些共产党人,始终坚定地相信我可以改造,耐心地引导我重新做人。\\

“人”,这是我在开蒙读本《三字经》上认识的第一个字,可是在我前半生中一直没有懂得它。有了共产党人,有了改造罪犯的政策,我今天才明白了这个庄严字眼的含义,才做了真正的人。\\

\begin{center}
	- 全书完 -
\end{center}
