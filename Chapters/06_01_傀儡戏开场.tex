\fancyhead[LO]{{\scriptsize 1932-1945: 伪满十四年 · 傀儡戏开场}} %奇數頁眉的左邊
\fancyhead[RO]{} %奇數頁眉的右邊
\fancyhead[LE]{} %偶數頁眉的左邊
\fancyhead[RE]{{\scriptsize 1932-1945: 伪满十四年 · 傀儡戏开场}} %偶數頁眉的右邊
\chapter*{傀儡戏开场}
\addcontentsline{toc}{chapter}{\hspace{1cm} 傀儡戏开场}
\thispagestyle{empty}
  在板垣的宴会上,我的思想是紊乱而又矛盾的。我不知道对自己的命运是应该高兴,还是应该忧愁。那天晚上,板垣召来了一大批日本妓女,给每个赴宴者配上一名,佰酒取乐。他自己左拥右抱,把斯文正经丢得一干二净。他时而举杯豪饮,时而纵声大笑,毫不掩饰其得意的心情。起初,在他还能矜持的时候,曾十分恭敬地向我祝酒,脸上带着暗示的笑容,祝我“前途顺利,达成宿愿”,这时,我觉得似乎可以高兴一点。到后来,随着钦量的增加,他的脸色越来越发青,情形就不对了。有个日本妓女用生硬的中国话问了我一句:“你是做买卖的干活!”板垣听见了,突然怪声大笑起来。这时我又想,我实在没有什么值得高兴的。\\

我这种忧喜不定、前途茫茫的心情,一直保持到胡嗣瑗、陈曾寿等人回到我身边的时候。这些老头子得到关东军的准许,能回到我的身边来,都是很高兴的。这种高兴与其说是由于君臣重聚,倒不如说是出于官爵财禄的热衷。他们一面因我纤尊降贵屈为执政而表示悲愤,一面向我列举历史故事,说明创业的君王每每有暂寄篱下,以求凭借之必要。有了这些教导,加上商衍瀛拿来的“老祖降坛训戒”,我的心情居然逐渐稳定下来。二月二十六日,我命随侍们给我准备香案,对祖宗祭告了一番,祭文如下:\\

\begin{quote}
	二十年来,视民水火,莫由拯救,不胜付托,丛疚滋深。今以东三省人民之拥戴,邻邦之援助,情势交迫,不得不出任维持之责。事属创举,成败利钝,非所逆睹。惟念自昔创业之君,若晋文之于秦穆,汉光武之于更始,蜀先主之于刘表、袁绍,明太祖之于韩林儿,当其经纶未展,不能不有所凭借,以图大举。兹本忍辱负重之心,为屈蠖求仲之计,降。心迁就,志切救民;兢兢业业,若履虎尾。敢诉愚诚,昭告于我列祖列宗之灵,伏祈默佑。\\
\end{quote}

二月的最末一天,在关东军第四课的导演下,沈阳的所谓“全满洲会议”通过决议,宣告东北独立,拥我出任“新国家执政”。上角利一和郑孝胥告诉我,这个会议的“代表”们就要来旅顺向我请愿,须先准备一下答词。答词要准备两个,第一个是表示拒绝,等“代表”们二次恳请,再拿出第二个来表示接受。三月一日,张燕卿、谢介石等九人到达旅顺。郑孝胥先代我接见,拿出了第一个答词:\\

\begin{quote}
	予自经播越,退处民间,闭户读书,罕间外事。虽宗国之砧危,时轸于私念,而拯救之方略未讲。平时忧患余生,才微德鲜。今某某等前来,猥以藐藐之躬,当兹重任,五中惊震,倍切惭惶。事未更则阅历之途浅,学未裕则经国之术疏,加以世变日新,多逾常轨,际遇艰屯,百倍畴昔。\\

人民之疾苦已臻其极,风俗之邪诐未知所届。既不可以陈方医变症,又不可以推助徇末流。所谓危急存亡之秋,一发千钧之会,苟非通达中外,融贯古今,天生圣哲,殆难宏济,断非薄德所能胜任。所望另举贤能,造福桑梓,勿以负疚之身,更滋罪戾。\\
\end{quote}

然后由我接见。彼此说了一通全是事先别人已嘱咐好的话,无非是一方“恳请”,一方“婉辞”。历时不过二十分钟,各自退场。三月五日,按关东军第四课的计划,“代表”人数增到二十九名,二次出场“恳请”。这次“代表”们完成了任务。我的答词最后是这样的:\\

\begin{quote}
	承以大义相责,岂敢以暇逸自宽,审度再三,重违群望。……勉竭愚昧,暂任执政一年;一年之后,如多陨越,敬避贤路。傥一年之内,宪法成立,国体决定,若与素志相合,再当审慎,度德量力,以定去就。\\
\end{quote}

走完“过场”,我于次日和婉容以及郑孝胥等人回到汤岗子。张景惠、赵欣伯等人早已在此等候,表示“恭迎”。我们在此过了一夜,次日一同前往长春。\\

三月八日下午三时,火车到达长春站。车还未停,就听见站台上响起军乐声和人们的呼叫声。我在张景惠、熙洽、甘粕、上角等一帮人的簇拥下走上站台,看见到处是日本宪兵队和各色服装的队列。在队列里,有袍子马褂,有西服和日本和服,人人手中都有一面小旗。我不禁激动起来,心想我在营口码头上没盼到的场面,今日到底盼来了。我在队列前走着,熙洽忽然指着一队夹在太阳旗之间的黄龙旗给我看,并且说:“这都是旗人,他们盼皇上盼了二十年。”听了这话,我不禁热泪盈眶,越发觉得我是大有希望的。\\

我坐上了汽车,脑子里只顾想我的紫禁城,想我当年被冯玉祥的国民军赶出城的情形,也想到“东陵事件”和我发过的誓言,我的心又被仇恨和欲望燃烧着,全然没有注意到长春街道的景色是什么样子,被恐怖与另一种仇恨弄得沉默的市民们,在用什么样的眼色看我们。过了不多时间,车子驶进了一个古旧的院落。这就是我的“执政府”。\\

这所房子从前是道尹衙门,在长春算不上是最宽敞的地方,而且破旧不堪,据说因为时间过于仓猝,只好暂时将就着。第二天,在匆忙收拾起的一间大厅里,举行了我的就职典礼。东北的日本“满铁”总裁内田康哉、关东军司令官本庄繁、关东军参谋长三宅光治、参谋板垣等等重要人物都来了。参加典礼的“旧臣”除了郑、罗、胡、陈等人外;还有前盛京副都统三多,做过绍兴知府以杀害秋瑾出名的赵景祺,蒙古王公贵福和他的儿子凌升以及蒙古王公齐默特色木丕勒等等。此外还有旧奉系人物张景惠、臧式毅、熙洽、张海鹏,在天津给我办过离婚案件的律师林迁琛、林囗。曾给张宗昌做过参谋的金卓这时也跑来做了我的侍从武官。\\

那天我穿的是西式大礼服,行的是鞠躬礼。在日本要人的旁观下,众“元勋”们向我行了三鞠躬,我以一躬答之。臧式毅和张景惠二人代表“满洲民众”献上了用黄绫包裹着的“执政印”。郑孝胥代念了“执政宣言”,其文曰:\\

\begin{quote}
	人类必重道德,然有种族之见,则抑人扬己,而道德薄矣。人类必重仁爱,然有国际之争,则报人利己,而仁爱薄矣。今立吾国,以道德仁爱为主,除去种族之见,国际之争,王道乐土,当可见诸实事。凡我国人,望其勉之。\\
\end{quote}

典礼完毕,接见外宾时,内田康哉致了“祝词”,罗振玉代读我的“答词”。然后到院子里升旗、照相。最后举行庆祝宴会。\\

当天下午,在“执政办公室”里,郑孝胥送上一件“公事”:\\

“本庄司令官已经推荐臣出任国务总理,组织内阁,”他微弓着身子,秃头发光,语音柔和,“这是特任状和各部总长名单\footnote{伪满大汉奸及其职务:国务总理郑孝胥,民政部总长臧式毅,外交部总长谢介石,军政部总长张景惠,财政部总长熙洽,实业部总长张燕卿,交通部总长丁鉴修,司法部总长冯涵清,文教部总长郑孝胥(兼),奉天省长臧式毅(兼),吉林省长熙洽(兼),黑龙江省长程志远(兼),立法院院长赵欣伯,监察院院长于冲汉,最高法院院长林棨,最高检查厅厅长李槃,参议府议长张景惠(兼),参议府副议长杨玉麟,参议府参议张海鹏、袁金铠、罗振玉、贵福,执政府秘书处处长胡嗣瑗,执政府秘书处秘书万绳木式、商行瀛、罗福葆。许宝衡、林廷琛,内务处处长宝熙,内务官特任张燕卿、金壁东、王季烈、佟济煦、王大忠、商衍瀛,警备处处长佟济煦,侍从武官长张海鹏,国务院秘书官郑垂,国务院秘书官郑禹。},请签上御名。”\\

这原是在旅顺时日本人甘粕正彦早跟我说好了的。我默默地拿起笔,办了就职后的第一件公事。\\

我走出办公室,遇上了胡嗣瑗和陈曾寿。这两个老头脸色都不好看,因为知道了特任官名单里,根本没有他们的名字。我对他们说:我要把他们放在身边,让胡嗣瑗当我的秘书处长,陈曾寿当秘书。胡嗣瑗叹着气谢了恩,陈曾寿却说他天津家里有事,求我务必准他回去。\\

第二天,罗振玉来了。他在封官中得的官职是一名“参议”,他是来辞这个不称心的官职的。我表示了挽留,他却说:“皇上屈就执政,按说君辱就该臣死,臣万不能就参议之职。”后来他做了一任“监察院院长”,又跑回大连继续卖他的假古董,一直到死。\\

但是我的思想反而跟他们不同了。长春车站上的龙旗和军乐,就职典礼时的仪节、以及外宾接见时的颂词,给我留下了深刻的印象,使我不禁有些飘飘然。另方面,我已公开露了头,上了台,退路是绝对没有了。即使板垣今天对我说,你不干就请便吧,我也回不去了。既然如此,就只好“降心迁就”到底。再说,如果对日本人应付得好,或许会支持我恢复皇帝尊号的。我现在既然是一国的元首,今后有了资本,就更好同日本人商量了。由于我专往称心如意的方面想,所以不仅不再觉着当“执政”是受委屈的事,而且把“执政”的位置看成了通往“皇帝宝座”的阶梯。\\

在这样自我安慰和充满幻想的思想支配下,如何好好地利用这个“阶梯”,顺利地登上“宝座”,就成了我进一步思索的中心问题。我想了几天之后,有一天晚上,把我思索的结果告诉了陈曾寿和胡嗣瑗:\\

“我现在有三个誓愿,告诉你们:第一,我要改掉过去的一切毛病,陈宝琛十多年前就说过我懒惰轻佻,我发誓从今永不再犯;第二,我将忍耐一切困苦,兢兢业业,发誓恢复祖业,百折不挠,不达目的誓不甘休;第三,求上天降一皇子,以承继大清基业。此三愿实现,我死亦瞑目。”\\

典礼后一个月左右,“执政府”迁到新修缮的前吉黑榷运局的房子。为表示决心,我亲自为每所建筑命名。我把居住楼命名为“缉熙”,系取自《诗经:大雅·文王》“于缉熙敬止”句。我更根据祖训“敬天法祖、勤政爱民”,以“勤民”命名我的办公楼。我从此真的每天早早起来,进办公室“办公”,一直到天晚,才从“勤民楼”回到“缉熙楼”来。为了誓愿,为了复辟,我一面听从着关东军的指挥,以求凭借,一面“宵衣旰食”,想把“元首”的职权使用起来。\\

然而,我的“宵衣旰食”没有维持多久,因为首先是无公可办,接着我便发现,“执政”的职权只是写在纸上的,并不在我手里。\\