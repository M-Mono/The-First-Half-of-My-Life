\fancyhead[LO]{{\scriptsize 1924-1930: 天津的“行在” · 东陵事件}} %奇數頁眉的左邊
\fancyhead[RO]{\thepage} %奇數頁眉的右邊
\fancyhead[LE]{\thepage} %偶數頁眉的左邊
\fancyhead[RE]{{\scriptsize 1924-1930: 天津的“行在” · 东陵事件}} %偶數頁眉的右邊
\chapter*{东陵事件}
\addcontentsline{toc}{chapter}{\hspace{11mm}东陵事件}
%\thispagestyle{empty}
一九二八年,对我是充满了刺激的一年,也是使我忧喜不定的一年。在这一年里,一方面日本的\ruby{田中}{\textcolor{PinYinColor}{たなか}}内阁发表了满蒙不容中国军队进入的声明,并且出兵济南,拦阻南方的军队前进,另方面\xpinyin*{张作霖}、\xpinyin*{吴佩孚}、\xpinyin*{张宗昌}这些和我有瓜葛的军队,由节节败退而溃不成军,为我联络军阀们的活动家刚报来了动人的好消息,我马上又读到那些向我效忠的军人逃亡和被枪毙的新闻。我听说中国的南北政府都和苏联绝交了,英苏也绝交了,国民党大肆清党,\xpinyin*{郑孝胥}、\xpinyin*{陈宝琛}以及日本人和我谈的那个“洪水猛兽”,似乎对我减少了威胁,但又据这些人说,危险正逼近到我的身边,到处有仇恨我的人在活动。我看到了报纸上关于广东有暴动的消息,同时,一直被我看成“过激”、“赤化”分子的\xpinyin*{冯玉祥},已和\xpinyin*{蒋介石}合作,正从京汉线上打过来。一九二八年下半年,使人灰心丧气的消息越来越多,\xpinyin*{张作霖}死了,美国的公使在给\xpinyin*{张学良}和\xpinyin*{蒋介石}撮合,……除了这些上面已说过的事件之外,这年还发生了最富刺激性的\xpinyin*{孙殿英}东陵盗墓事件。\\

东陵在河北省遵化县的马兰峪,是\ruby{乾隆}{\textcolor{PinYinColor}{\Man ᠠᠪᡴᠠᡳ ᠸᡝᡥᡳᠶᡝᡥᡝ}}和西太后的陵寝。\xpinyin*{孙殿英}是一个赌棍和贩毒犯出身的流氓军人,在\xpinyin*{张宗昌}部当过师长、军长。一九二七年孙受\xpinyin*{蒋介石}的改编,任四十一军军长。一九二八年,孙率部到蓟县、马兰峪一带,进行了有计划的盗墓。他预先贴出布告,说是要举行军事演习,封锁了附近的交通,然后由他的工兵营营长\xpinyin*{颛孙子瑜}带兵挖掘,用三个夜晚的时间,把\ruby{乾隆}{\textcolor{PinYinColor}{\Man ᠠᠪᡴᠠᡳ ᠸᡝᡥᡳᠶᡝᡥᡝ}}和\xpinyin*{慈禧}的殉葬财宝,搜罗一空。\\

\ruby{乾隆}{\textcolor{PinYinColor}{\Man ᠠᠪᡴᠠᡳ ᠸᡝᡥᡳᠶᡝᡥᡝ}}和\xpinyin*{慈禧}是清朝历代帝后中生活最奢侈的。我从一份文史资料中,看到过一段关于他们的陵墓的描述:\\

\begin{quote}
	墓中隧道全用汉白玉砌成,有石门四进,亦全系汉白玉雕制,寝宫为八角形,上覆圆顶,雕塑着九条金龙,闪闪发光。寝宫面积约与故官的中和殿相等。\ruby{乾隆}{\textcolor{PinYinColor}{\Man ᠠᠪᡴᠠᡳ ᠸᡝᡥᡳᠶᡝᡥᡝ}}的棺\xpinyin*{梓}是用阴沉木制成的,安放在一个八角井的上边。两座坟墓中的殉葬器物,除金银元宝和明器外,都是些罕见的珍宝。\xpinyin*{慈禧}的殉葬物品,多是一些珠宝翠钻之类,她的凤冠是用很大的珍珠以金线穿制而成的;\xpinyin*{衾}被上有大朵的牡丹花,亦全用珍珠堆制;手镯系用大小钻石镶成一大朵菊花和六小朵梅花,澄彻晶莹,光彩夺目;手里握着一柄降魔杵,长约三寸余,为翡翠制;她的脚上还穿着一双珠鞋。另外,在棺中还放置着十七串用珠宝缀成的念珠和几双翠质手镯。\ruby{乾隆}{\textcolor{PinYinColor}{\Man ᠠᠪᡴᠠᡳ ᠸᡝᡥᡳᠶᡝᡥᡝ}}的殉葬品都是一些字画、书剑和玉石、象牙、珊瑚雕刻的文玩及金质佛像等物,其中绢、丝制品都已腐朽,不可辨认。\\
\end{quote}

我听到东陵守护大臣报告了\xpinyin*{孙殿英}盗掘东陵的消息,当时所受到的刺激,比我自己被驱逐出宫时还严重。宗室和遗老们全激动起来了。\xpinyin*{陈宝琛}、\xpinyin*{朱益藩}、\xpinyin*{郑孝胥}、\xpinyin*{罗振玉}、\xpinyin*{胡嗣瑗}、\xpinyin*{万绳栻}、\xpinyin*{景方昶}、\xpinyin*{袁励准}、\xpinyin*{杨锺羲}、\ruby{铁良}{\textcolor{PinYinColor}{\Man ᠲᡞᠶᡝᡞ ᠯᡞᠶᠠᠩ}}、\xpinyin*{袁大化}、\xpinyin*{升允}……不论是哪一派的,不论已经消沉的和没有消沉的,纷纷赶到我这里,表示了对\xpinyin*{蒋介石}军队的愤慨。各地遗老也纷纷寄来重修祖陵的费用。在这些人的建议和安排下,张园里摆上了\ruby{乾隆}{\textcolor{PinYinColor}{\Man ᠠᠪᡴᠠᡳ ᠸᡝᡥᡳᠶᡝᡥᡝ}}、\xpinyin*{慈禧}的灵位和香案祭席,就像办丧事一样,每天举行三次祭奠,遗老遗少们络绎不绝地来行礼叩拜,痛哭流涕。清室和遗老们分别向\xpinyin*{蒋介石}和平津卫戍司令\xpinyin*{阎锡山}以及各报馆发出通电,要求惩办\xpinyin*{孙殿英},要求当局赔修陵墓。张园的灵堂决定要摆到陵墓修复为止。\\

起初,\xpinyin*{蒋介石}政府的反应还好,下令给\xpinyin*{阎锡山}查办此事。\xpinyin*{孙殿英}派到北平来的一个师长被\xpinyin*{阎锡山}扣下了。随后不久,消息传来,说被扣的师长被释放,\xpinyin*{蒋介石}决定不追究了。又传说\xpinyin*{孙殿英}给\xpinyin*{蒋介石}新婚的夫人\xpinyin*{宋美龄}送去了一批赃品,\xpinyin*{慈禧}凤冠上的珠子成了\xpinyin*{宋美龄}鞋子上的饰物。我心里燃起了无比的仇恨怒火,走到阴阴森森的灵堂前,当着满脸鼻涕眼泪的宗室人等,向着空中发了誓言:\\

“不报此仇,便不是\ruby{爱新觉罗}{\textcolor{PinYinColor}{\Man ᠠᡳᠰᡳᠨ ᡤᡳᠣᡵᠣ}}的子孙!”\\

我此时想起\ruby{溥}{\textcolor{PinYinColor}{Pu}}\ruby{伟}{\textcolor{PinYinColor}{Wei}}到天津和我第一次见面时说的:“有\ruby{溥}{\textcolor{PinYinColor}{Pu}}\ruby{伟}{\textcolor{PinYinColor}{Wei}}在,大清就一定不会亡!”我也发誓说:\\

“有我在,大清就不会亡!”\\

我的复辟、复仇的思想,这时达到了一个新的顶峰。\\

在那些日子里,\xpinyin*{郑孝胥}和\xpinyin*{罗振玉}是我最接近的人,他们所谈的每个历史典故和当代新闻,都使我感到激动和愤慨不已,都增强着我的复辟和复仇的决心。和国民党的国民政府斗争到底,把灵堂摆到修复原墓为止,就是他们想出的主意。但是后来形势越来越不利,盗墓的人不追究了,北京天津一带面目全非,当权的新贵中再没有像\xpinyin*{段祺瑞}、\xpinyin*{王怀庆}这类老朋友,我父亲也不敢再住在北京,全家都搬到天津租界里来了。于是我的心情也由激愤转成忧郁。蒋宋两家的结亲,就使张园里明白了英美买办世家和安清帮兼交易所经纪人的这种结合,说明\xpinyin*{蒋介石}有了比\xpinyin*{段祺瑞}、\xpinyin*{张作霖}、\xpinyin*{孙传芳}、\xpinyin*{吴佩孚}这些倒台的军人更硬的后台。这年年末,\xpinyin*{蒋介石}的国民政府得到了包括日本在内的各国的承认,他的势力和地位已超过了以往的任何一个军阀。我觉得自己的前途已十分黯淡,认为在这样一个野心人物的统治下,不用说复辟,连能否在他的势力范围内占一席地,恐怕全成问题。\\

我在心里发出了狠毒的诅咒,怀着深刻的忧虑,为\xpinyin*{蒋介石}的政府和自己的命运,一次又一次地卜过卦,扶过比我曾卜占“国民政府能长久否?”得“天大同人变离,主申年化冲而散”的一个卦文,其意思是:\xpinyin*{蒋介石}政府将众叛亲离,在一九三二年灭亡。当然,\xpinyin*{蒋介石}的政府如果垮台,可以发泄我的仇恨,使我痛快。但是,我更关心的是我自己的命运。我屡次叫\ruby{荣}{\textcolor{PinYinColor}{Žung}}\ruby{源}{\textcolor{PinYinColor}{Yuwan}}扶\xpinyin*{乩},有一次他得到这样一个\xpinyin*{乩}文:\\

\begin{quote}
	今上乃重兴之主,清仍有天下,然子(按指\ruby{荣}{\textcolor{PinYinColor}{Žung}}\ruby{源}{\textcolor{PinYinColor}{Yuwan}})乃朝廷勋戚大臣,必须直谏君,于致光武,务必劝戒奢华,弥问世事,晦迹韬光,暗成事业,亲君子,远小人,去伪忠,此皆要图,子忠实君子,吾所夙知,故愿直言,将来再兴,务必改元,\ruby{宣统}{\textcolor{PinYinColor}{\Man ᡤᡝᡥᡠᠩᡤᡝ ᠶᠣᠰᠣ}}二字,乃宁日一乱丝充满天下尽,贼犯紫微,务用隆武,隆若不用,可改兴武,此天机也,国事且不泄。\\
\end{quote}

但是任何一个欲望强烈和报仇心切的人,都不会只记得“成事在天”而忘了“求事在人”这句话。我自己几年来的阅历,特别是\xpinyin*{蒋介石}的发家史,给了我一条重要的信念,这就是若求成事必须手握兵权,有了兵权实力,洋人自然会来帮助。像我这样一个正统的“大清皇帝”,倘若有了军队,自然要比一个红胡子或者一个流氓出身的将帅更会受到洋人的重视。因此,我决定派我身边最亲信的亲族子弟去日本学陆军。我觉得这比我自己出洋更有必要。\\

促成我这个想法的,还有一个原因,就是\ruby{溥杰}{\textcolor{PinYinColor}{\Man ᡦᡠ ᡤᡳᠶᡝ}}正为了要投笔从戎,在家里闹得马仰人翻。他从军的动机本来也颇可笑,与其说是受到母亲遗嘱的影响,立志要恢复清朝,还不如说是由于他羡慕那些手握虎符的青年将帅,自己也想当军官,出出风头。\xpinyin*{张学良}在\xpinyin*{张作霖}死后,临国奉天之前对\ruby{溥杰}{\textcolor{PinYinColor}{\Man ᡦᡠ ᡤᡳᠶᡝ}}说过:“你要当军官,我送你进讲武堂(奉军的军官学校)。”于是他便和\xpinyin*{张学良}的家眷乘船离了天津。我父亲看到了他留下的信,急得要命,要我无论如何想个办法把他追回来。天津日本总领事答应了我的请求,发了电报给大连。在大连码头上,\ruby{溥杰}{\textcolor{PinYinColor}{\Man ᡦᡠ ᡤᡳᠶᡝ}}刚从船上走下来,就给日本警察截住了。他被我派去的人接回到天津,见了我就诉说他投军的志向,是为了恢复祖业。他的话触动了我送他去日本学陆军的心思。\\

我决定了派\ruby{溥杰}{\textcolor{PinYinColor}{\Man ᡦᡠ ᡤᡳᠶᡝ}}和我的三妹夫\xpinyin*{润麟}一同到日本去学陆军。为了准备他们的留学,我请天津日本总领事介绍了一位家庭教师,教他们日文。日本总领事推荐了一位叫\ruby{远山}{\textcolor{PinYinColor}{とーやま}}\ruby{猛雄}{\textcolor{PinYinColor}{たけお}}的日本人,后来知道,这是一个日本黑龙会的会员,认识不少日本政客。这个人后来也为了我的复辟理想,替我到日本奔走过。我到东北以后,因为他不是军部系统的,受到排挤,离开了我。这位远山教师教了\ruby{溥杰}{\textcolor{PinYinColor}{\Man ᡦᡠ ᡤᡳᠶᡝ}}和\xpinyin*{润麟}不多日子的日文,就为他们的留学问题回到日本去活动了一趟,据说是暂时还不能人日本士官学校,但是可以先进专供日本贵族子弟读书的学习院,并且还得到了日本的大财阀\ruby{大仓}{\textcolor{PinYinColor}{おおくら}}\ruby{喜八郎}{\textcolor{PinYinColor}{きはちろう}}的帮助。一九二九年三月,即“东陵事件”发生后七个月,我这两个未来的武将就和\ruby{远山}{\textcolor{PinYinColor}{とーやま}}一起到日本去了。
