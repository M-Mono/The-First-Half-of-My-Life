\fancyhead[LO]{{\scriptsize 1932-1945: 伪满十四年 · 大崩溃}} %奇數頁眉的左邊
\fancyhead[RO]{} %奇數頁眉的右邊
\fancyhead[LE]{} %偶數頁眉的左邊
\fancyhead[RE]{{\scriptsize 1932-1945: 伪满十四年 · 大崩溃}} %偶數頁眉的右邊
\chapter*{大崩溃}
\addcontentsline{toc}{chapter}{\hspace{1cm}大崩溃}
\thispagestyle{empty}
在战犯管理所的时候,有个前伪满军的旅长对我说过一个故事。太平洋战争发生的那一年冬天,他在关东军的指挥下,率伪满军前去袭击抗联部队。他的队伍在森林里扑了一个空,只找到了一个藏在地下小屋里的生病的抗联战士。这个人衣服破烂,头发、胡子挺长,就像关了很久的囚犯似的。他望见这俘虏的外貌,不禁嘲弄地说:\\

“看你们苦成这副模样,还有什么干头!你知道不知道,大日本皇军把新加坡、香港都占领啦……”\\

“俘虏”突然笑起来。这位“满洲国”少将拍着桌子制止道:“笑什么?你知道你这是受审判吗?”那战士对他的回答,叫他大吃一惊:\\

“谁审判谁?你们的末日不远了,要不了多长时候,你们这群人,都要受人民的审判!”\\

伪满的文武官员,一般说来都知道东北人民仇恨日寇和汉奸,但却不理解他们何以有这么大的胆量,何以那么相信自己的力量,同时又确信强大的统治者必败无疑。我从前一直把日本帝国主义的力量看做强大无比,不可动摇。在我心里,能拿来和日本做比较的,连大清帝国、北洋政府和国民党的中华民国都够不上,至于“老百姓”,我连想也没想过。\\

究竟是谁强大无比,是谁软弱无力?其实早有无数的事实告诉过我,但是我极不敏感,一直到从\ruby{吉冈}{\textcolor{PinYinColor}{よしおか}}嘴里透露了出来的时候,我还是模模糊糊。\\

有一次,关东军安排我外出“巡幸”(一年有一次),去的地点是延吉朝鲜族地区。我的专车到达那里,发现大批的日本宪兵和六个团的伪军,把那里层层围了起来。我问\ruby{吉冈}{\textcolor{PinYinColor}{よしおか}}这是什么意思,他说是“防土匪”。“防土匪何用这么多兵力?”“这土匪可不是从前那种土匪,这是共产军哪!”“怎么满洲国也有共产军?共产军不是在中华民国吗?”“有的,有的,小小的有的,……”\ruby{吉冈}{\textcolor{PinYinColor}{よしおか}}含含混混回答着,转移了话题。\\

又一次,关东军参谋在例行的军事形势报告之外,特地专门向我报告了一次“胜利”。在这次战役中,抗联的领袖\xpinyin*{杨靖宇}将军牺牲了。他兴高采烈地说,杨将军之死,消除了“满洲国的一个大患”。我一听“大患”二字,忙问他:“土匪有多少?”他也是这么说:“小小的,小小的有。”\\

一九四二年,华北和华中的日本军队发动了“大扫荡”,到处实行三光政策,制造无人区。有一次,\ruby{吉冈}{\textcolor{PinYinColor}{よしおか}}和我谈到日军对华北“共产军”的种种战术,如“铁壁合围”、“梳蓖扫荡”等等,说这给“大日本皇军战史上,增添了无数资料”。我听他说的天花乱坠,便凑趣说:“共产军小小的,何犯上用这许多新奇战术?”不料这话引起了他的嘲弄:\\

“皇帝陛下倘若有实战体验,必不会说这话。”\\

我逢迎道:“愿闻其详。”\\

“共产军,这和国民党军不一样。军民不分,嗯,军民不分,举例说,嗯,就像赤豆混在红砂土里,……”他看我茫然无知的样子,又举出中国的“鱼目混珠”的成语来做比喻,说日本军队和八路军、新四军作战时,常常陷入四面受敌的困境。后来,他竟不怕麻烦,边说边在纸上涂抹着解释:“共产军”不管到哪里,百姓都不怕他;当兵一年就不想逃亡,这实在是大陆上从来没有的军队;这样队伍越打越多,将来不得了。“可怕!这是可怕的!”他不由自主地摇头感叹起来。看见这位“大日本皇军”将官居然如此评论“小小的”敌人,我惶惑得不知说什么才合适,拼命地搜索枯肠,想起了这么两句:\\

“杀人放火,共产共妻,真是可怕!”\\

“只有鬼才相信这个!”他粗暴地打断了我的话。过了一会儿,他又用嘲弄的眼神看着我说:\\

“我这并不是正式评论,还是请陛下听关东军参谋长的报告吧。”\\

说着,他把刚才涂抹过的纸片都收了起来,放进口袋。\\

我逐渐地觉出了\ruby{吉冈}{\textcolor{PinYinColor}{よしおか}}的“非正式评论”,比关东军司令官和参谋长的“正式评论”比较近乎事实。\ruby{植田}{\textcolor{PinYinColor}{うえだ}}\ruby{谦吉}{\textcolor{PinYinColor}{けんきち}}发动诺门坎战役时,为了证实他的“正式评论”,曾把我和\xpinyin*{张景惠}等都请了去,参观日本飞机超过苏联飞机的速度表演。事实上,那次日军被打得落花流水,损失了五万多人,\ruby{植田}{\textcolor{PinYinColor}{うえだ}}也因之撤职。\ruby{吉冈}{\textcolor{PinYinColor}{よしおか}}在非正式评论时说:“苏军的大炮比皇军的射程远多了!”\\

藏在\ruby{吉冈}{\textcolor{PinYinColor}{よしおか}}心底的隐忧,我渐渐地从收音机里,越听越明白。日军在各个战场失利的消息越来越多,报纸上的“赫赫战果”、“堂堂入城”的协和语标题,逐渐被“玉碎”字样代替。物资匾乏情况严重,我在封锁重重中也能觉察出来。不但是搜刮门环、痰桶等废铜烂铁的活动,伸进“帝宫”里来,而且“内延”官员家属因缺乏食物,也纷纷来向我求助了。“强大无比”的日本统治者开始露馅,“无畏的皇军”变成样样畏惧。因为怕我知道军队供应质量低劣,关东军司令官特地展览了一次军用口粮请我去参观;因为怕我相信从收音机里听到的海外广播,送来宣传日军战绩的影片给我放映……。不用说我不相信这些,就连我最小的侄子也不相信。\\

给我印象最深的,是日本军人流露出来的恐惧。\\

占领了新加坡之后到东北来任关东军某一方面军司令长官的山下奉文,当时趾高气扬不可一世的狂态还留在我的记忆里,可是到了一九四五年,当他再次奉调南洋,临行向我告别时,却对我捂着鼻子哭了起来,说:“这是最后的永别,此一去是不能再回来了!”\\

在一次给“肉弹”举行饯行式时,我又看到了更多的眼泪。肉弹是从日本军队中挑选出来的士兵,他们受了“武士道”和“忠君”的毒素教育,被挑出来用肉体去和飞机坦克碰命,日本话叫做“体挡”。\ruby{吉冈}{\textcolor{PinYinColor}{よしおか}}从前每次提到这种体挡,都表示无限崇敬。听那些事迹,我确实很吃惊。这回是关东军叫我对这批中选的肉弹鼓励一下,为他们祝福。那天正好是阴天,风沙大作。饯行地点在同德殿的院里,院里到处是一堆堆的防空沙袋,更显得气象颓丧。肉弹一共有十几个人,排成一列站在我面前,我按\ruby{吉冈}{\textcolor{PinYinColor}{よしおか}}写好的祝词向他们念了,然后向他们举杯。这时我才看见,这些肉弹个个满脸灰暗,泪流双颊,有的竟硬咽出声。\\

仪式在风沙中草草结束了,我心慌意乱,又急着要回屋里去洗脸,\ruby{吉冈}{\textcolor{PinYinColor}{よしおか}}却不离开,紧跟在我身后不去。我知道他一定又有话说,只好等着他。他清了清嗓子,嗯了几声,然后说:\\

“陛下的祝词很好,嗯,所以他们很感动,嗯,所以才流下了日本男子的眼泪……”\\

听了这几句多余的话,我心说:“你这也是害怕呵!你怕我看出了肉弹的马脚!你害怕,我更害怕啦!”\\

一九四五年五月,德国战败后,日本四面受敌的形势就更明显了,苏联的出兵不过是个时间上的问题。日本过去给我的印象不管如何强大,我也明白了它的孤立劣势。\\

最后崩溃的日子终于来了。\\

一九四五年八月九日的早晨,最末一任的关东军司令官\ruby{山田}{\textcolor{PinYinColor}{やまだ}}\ruby{乙三}{\textcolor{PinYinColor}{おとぞう}}同他的参谋长\ruby{秦}{\textcolor{PinYinColor}{はた}}\ruby{彦三郎}{\textcolor{PinYinColor}{ひこさぶろう}}来到了同德殿。向我报告说,苏联已向日本宣战了。\\

\ruby{山田}{\textcolor{PinYinColor}{やまだ}}\ruby{乙三}{\textcolor{PinYinColor}{おとぞう}}是个矮瘦的小老头,平时举止沉稳,说话缓慢。这天他的情形全变了,他急促地向我讲述日本军队如何早有十足准备,如何具有必胜之信心。他那越说越快的话音,十足的证明连他自己也没有十足的准备和信心。他的话没说完,忽然响起了空袭警报。我们一齐躲进了同德殿外的防空洞,进去不久,就听见不很远的地方响起了爆炸声。我暗诵佛号,他默不作声。一直到警报解除,我们分手时为止,他再没提到什么信心问题。\\

从这天夜里起,我再没有脱衣服睡觉。我的袋里总放着一支手枪,并亲自规定了内廷的戒严口令。\\

次日,\ruby{山田}{\textcolor{PinYinColor}{やまだ}}\ruby{乙三}{\textcolor{PinYinColor}{おとぞう}}和\ruby{秦}{\textcolor{PinYinColor}{はた}}\ruby{彦三郎}{\textcolor{PinYinColor}{ひこさぶろう}}又来了,宣布日军要退守南满。“国都”要迁到通化去,并告诉我必须当天动身。我想到我的财物和人口太多,无论如何当天也搬不了。经我再三要求,总算给了三天的宽限。\\

从这天起,我开始受到了一种新的精神折磨。这一半是由于\ruby{吉冈}{\textcolor{PinYinColor}{よしおか}}态度上有了进一步的变化,一半是由于我自己大大地犯了疑心病,自作自受。我觉出了\ruby{吉冈}{\textcolor{PinYinColor}{よしおか}}的变化,是由于他在\ruby{山田}{\textcolor{PinYinColor}{やまだ}}\ruby{乙三}{\textcolor{PinYinColor}{おとぞう}}走后,向我说了这么一句话:\\

“陛下如果不走,必定首先遭受苏联军的杀害!”\\

他说这句话的时候,样子是恶狠狠的。但是让我更害怕的,是我从他的话里猜测到,日本人正疑心我不想走,疑心我对他们怀有贰心。\\

“他们怕我这个人证落在盟军手里,会不会杀我灭口?”这个问题一冒头,我的汗毛都竖起来了。\\

我想起了十多年的故技,我得设法在\ruby{青冈}{\textcolor{PinYinColor}{せいこう}}面前表现“忠诚”。我灵机一动,叫人把国务院总理\xpinyin*{张景惠}和总务厅长官\ruby{武部}{\textcolor{PinYinColor}{たけべ}}\ruby{六藏}{\textcolor{PinYinColor}{ろくぞう}}找来。我向他们命令道:\\

“要竭尽全力支援亲邦进行圣战,要抗拒苏联军到底,到底……”\\

说完,我回头去看\ruby{吉冈}{\textcolor{PinYinColor}{よしおか}}的脸色。但这个形影不离的“御用挂”,却不知道什么时候出去了。\\

我莫名其妙地起了不祥的预感,在屋子里转来转去。这样过了一会儿,我忽然看见窗外有几个日本兵端着枪,向同德殿这边走来。我的魂简直飞出了窍,以为是下毒手来了。我觉着反正没处可躲,索性走到楼梯口,迎上了他们。这几个日本兵看见了我,却又转身走了。\\

我认为这是来查看我,是不是跑了。我越想越怕,就拿起电话找\ruby{吉冈}{\textcolor{PinYinColor}{よしおか}},电话怎么也叫不通。我以为日本人已经扔下我走了,这叫我同样的害怕。\\

后来我给\ruby{吉冈}{\textcolor{PinYinColor}{よしおか}}打电话,电话通了,\ruby{吉冈}{\textcolor{PinYinColor}{よしおか}}的声音很微弱,说他病了。我连忙表示对他的关怀,说了一堆好话,听他说了“谢谢陛下”,我放了电话,松了一口气。这时我感到肚子很饿,原来一天没吃一点东西了。我叫剩下来的随侍大李给我“传膳”,大李说厨师全走了。我只好胡乱吃点饼于。\\

十一日晚上九点多,\ruby{吉冈}{\textcolor{PinYinColor}{よしおか}}来了。这时我的弟弟、妹妹、妹夫和侄子们都已先去了火车站,家里只剩下我和两个妻子。\ruby{吉冈}{\textcolor{PinYinColor}{よしおか}}对我和随行的一些随侍们用命令口气说:\\

“无论是步行,或是上下车辆,由\ruby{桥本}{\textcolor{PinYinColor}{はしもと}}\ruby{虎之助}{\textcolor{PinYinColor}{とらのすけ}}恭捧‘神器’走在前面。无论是谁,经过‘神器’,都须行九十度鞠躬礼。”\\

我知道这真到了出发的时候了。我恭恭敬敬地站着,看祭祀长\ruby{桥本}{\textcolor{PinYinColor}{はしもと}}\ruby{虎之助}{\textcolor{PinYinColor}{とらのすけ}}捧着那个盛着‘神器’的包袱,上了头辆汽车,然后自己进了第二辆。汽车开出了“帝宫”,我回头看了一眼,在“建国神庙”上空,升起了一股火苗。\\

在通往通化大栗子沟的路上,火车走了三夜两天。本来应从沈阳走,为了躲避空袭,改走了吉林——梅河口的路线。两天里只吃了两顿饭和一些饼干。沿途到处是日本兵车,队伍不像队伍,难民不像难民。在梅河口,车停下来,关东军司令官\ruby{山田}{\textcolor{PinYinColor}{やまだ}} 来到了车上。他向我报告日军打了胜仗,击毁了多少苏军飞机和坦克。但是在吉林站上,我却看到一幅相反的景象:成批的日本妇女和孩子叫嚷着拥向火车,向拦阻她们的宪兵哀求着,哭号着……在站台尽头处,日本士兵和宪兵厮打着……\\

大栗子沟是一座煤矿,在一个山弯里,与朝鲜一江之隔,清晨,白雾弥漫着群山,太阳升起之后,青山翠谷,鸟语花香,景色极美,在当时,这一切在我的眼里却都是灰暗的。我住的地方是日本矿长的住宅,有七八间房,这种日本式的房间隔音不好,所以成天闹哄哄的。\\

八月十三日到了这里,过了两天惊惶不安的生活,八月十五日日本就宣布投降了。\\

当\ruby{吉冈}{\textcolor{PinYinColor}{よしおか}}告诉了我“天皇陛下宣布了投降,美国政府已表示对天皇陛下的地位和安全给以保证”,我立即双膝跪下,向苍天磕了几个头,念诵道:“我感谢上天保佑天皇陛下平安!”\ruby{吉冈}{\textcolor{PinYinColor}{よしおか}}也随我跪了下来,磕了一阵头。\\

磕完头,\ruby{吉冈}{\textcolor{PinYinColor}{よしおか}}愁眉苦脸地说,日本关东军已和东京联系好,决定送我到日本去。“不过,”他又说,“天皇陛下也不能绝对担保陛下的安全。这一节要听盟军的了。”\\

我认为死亡已经向我招手了。\\

\xpinyin*{张景惠}、\ruby{武部}{\textcolor{PinYinColor}{たけべ}}\ruby{六藏}{\textcolor{PinYinColor}{ろくぞう}}和那一群“大臣”、“参议”找我来了。原来还有一场戏要演,他们拿来了那位汉学家的新手笔——我的“退位诏书”。我站在犹如一群丧家犬的大臣、参议面前,照着念了一遍。这个第六件诏书的字句已不记得了,只记得这件事:这篇诏书原稿上本来还有那少不了的“仰赖天照大神之神庥,天皇陛下之保佑”,可是叫\ruby{桥本}{\textcolor{PinYinColor}{はしもと}}\ruby{虎之助}{\textcolor{PinYinColor}{とらのすけ}}苦笑着给划掉了。\ruby{桥本}{\textcolor{PinYinColor}{はしもと}}担任过守护天皇的近卫师团长,后来又做了守护天照大神的祭祀长,可算是最了解天皇和天照大神的人了。\\

我假如知道,我这时的身价早已降在\xpinyin*{张景惠}那一批人之下,心情一定更糟。日本人在决定我去东京的同时,布置了\xpinyin*{张景惠}和\ruby{武部}{\textcolor{PinYinColor}{たけべ}}\ruby{六藏}{\textcolor{PinYinColor}{ろくぞう}}回到长春,安排后事。他们到了长春,由\xpinyin*{张景惠}出面,通过广播电台和重庆的\xpinyin*{蒋介石}取得了联系,同时宣布成立“治安维持会”,准备迎接\xpinyin*{蒋介石}的军队接收。他们打算在苏军到达之前,尽快变成“中华民国”的代表。但没有料到苏军来得如此神速,而共产党领导的抗联军队也排除了日军的抵抗,逼近了城市。苏军到了长春,苏联指挥官对他们说了一句:“等候吩咐吧。”\xpinyin*{张景惠}他们以为维持会被承认了,不禁对苏联又生了幻想,\xpinyin*{张景惠}回家对他老婆说:“行啦,这又捞着啦!”第二天,伪大臣们应邀到达了苏军司令部,等着苏军司令的委派,不料苏联军官宣布道:“都到齐啦,好,用飞机送你们到苏联去!”\\

八月十六日,日本人听说在长春的禁卫军已和日军发生了冲突,就把随我来的一连禁卫军缴了械。这时\ruby{吉冈}{\textcolor{PinYinColor}{よしおか}}通知我,明天就动身去日本,我当然连忙点头称是,装出高兴的样子。\\

\ruby{吉冈}{\textcolor{PinYinColor}{よしおか}}叫我挑选几个随行的人。因为飞机小,不能多带,我挑了\ruby{溥杰}{\textcolor{PinYinColor}{Pu Giye}}、两个妹夫、三个侄子、一个医生和随侍大李。“福贵人”哭哭啼啼地问我:“我可怎么办呢?”我说:“飞机太小,你们坐火车去吧。”“火车能到日本吗?”我不假思索地说:“火车能到。顶多过三天,你和皇后他们就见着我了。”“火车要是不来接呢?我在这里一个亲人也没有呀!”“过两天就见着了,行了行了!”\\

我心乱如麻,反复思索着如何能逃脱死亡,哪还有心顾什么火车不火车呢?\\

飞机飞行的第一个目标是沈阳,我们要在那里换乘大型飞机。从通化出发,和我在一起的是\ruby{吉冈}{\textcolor{PinYinColor}{よしおか}}、\ruby{桥本}{\textcolor{PinYinColor}{はしもと}}、\ruby{溥杰}{\textcolor{PinYinColor}{Pu Giye}}和一名日本神官(随\ruby{桥本}{\textcolor{PinYinColor}{はしもと}}捧“神器”的),其他人和一名日本宪兵在另一架飞机上。这天上午十一时,我先到了沈阳机场,在机场休息室里,等候着那另一架飞机。\\

等候了不久,忽然响起了一片震耳的飞机马达声。原来是苏军飞机来着陆了。一队队手持冲锋枪的苏联士兵,走下飞机,立即将机场上的日本军队缴了械。不大的时间,机场上到处是苏联的军人。这是苏军受降的军使来到了。\\

由于这个变化,我没有能够到日本去。第二天,便被苏联飞机载往苏联去了。
