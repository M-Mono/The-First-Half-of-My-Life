\fancyhead[LO]{{\scriptsize 1950-1954: 由抗拒到认罪 · 检察人员来了}} %奇數頁眉的左邊
\fancyhead[RO]{} %奇數頁眉的右邊
\fancyhead[LE]{} %偶數頁眉的左邊
\fancyhead[RE]{{\scriptsize 1950-1954: 由抗拒到认罪 · 检察人员来了}} %偶數頁眉的右邊
\chapter*{检察人员来了}
\addcontentsline{toc}{chapter}{\hspace{1cm}检察人员来了}
\thispagestyle{empty}
 从一九五三年末起,我们连着学习了三个月的《帝国主义论》。一九五四年三月,学习结束后,管理所迁回抚顺。过了不久,检察机关的工作团来到管理所,开始了对战犯的调查。\\

后来才知道,政府为了这次调查日本战犯和伪满战犯的罪行,做了很周密的准备,组织了庞大的力量。一大批日本战犯调到抚顺来了。几年前政府人员就准备了大量材料。大约二百名左右的检察工作人员集中起来,事先受到了政策和业务的专门训练。\\

日本战犯住在“三所”、“四所”和“七所”里,那边的情形不清楚,我们一所伪满战犯这边三月末开过了一个大会,开始了调查。调查工作(从犯人这方面说是检举与认罪)一直进行到年底,才基本结束。\\

在大会上,工作团的负责人员讲了话。他说,你们经过了这几年的学习和反省,现在已经到了认罪的时候了,政府有必要来查清你们的罪行,你们也应该对过去有个正确的认识,交代自己的罪行,并且检举日本帝国主义战犯和其他汉奸的罪行;无论是坦白交代和检举他人,都要老老实实,不扩大、不缩小;政府对你们最后的处理,一方面要根据罪行,一方面要根据你们的态度;政府的政策是坦白从宽、抗拒从严。\\

所长同时宣布了监规:不准交换案情,不准跟别的监房传递字条信件,等等。从这天起,每日休息时间各组轮流到院子里去,想跟别组的人会面也办不到了。\\

开过大会,各组回到各自的屋子开讨论会,每个人都表示了要彻底坦白、检举,低头认罪,争取宽大。有人说:“我一直在盼这天,只要能审判,就有期限了。”也有的人,比如老宪说了他相信宽大政策,却又神色不安,显然是言不由衷。\\

看到老宪面色发灰,我并没什么幸灾乐祸的想法,反而被他传染上了不安的情绪。自从在学习心得里交代了历史关键问题之后,当时我对宽大政策有了信任,现在又觉得政策还没兑现,不知将来处理的时候,是不是仍如所长说过的,对我并不例外。如果像老宪这样一个“军医院长”也值得担心,我这“皇帝”又该如何呢?\\

但是,无论如何,最大的问题我都已经交代出来了。我的情形可能跟老宪不同,他也许在考虑是不是交代,而我的问题只能是如何让检察人员相信,我早已就是认了罪的。\\

为了取得检察人员的信任,我决定详细而系统地把自己的历史重写一遍,同时把自己知道的日本战犯的罪行尽量写出来。我在小组会上做了这样的保证。\\

完全实现这个保证,却不是那么容易。\\

我写到伪满末期,写到苏联对日本宣战那一段,想起了一件事。那时我担心日本人在这紧张时机对我怀疑,把我踢开,总想着法儿取宠关东军。在得到苏军宣战消息后的一天夜里,我没经任何人的指点,把\xpinyin*{张景惠}和总务厅长官\xpinyin*{武部六藏}叫了来,给他们下了一道口头“\xpinyin*{敕}令”,命他们紧急动员,全力支持日本皇军抵抗苏军的进攻。这件事情我该如何写?不写,这件事难保别人不知,写吧,这件并非日本人授意的举动(那时\ruby{吉冈}{よしおか}正称病不露面),是否会引起检察人员的怀疑,不相信我是处处受着\ruby{吉冈安直}{よしおか やすなお}摆布的呢?如果检察人员发生了误会,我所交代的全部历史就变成不可信的了。\\

我最后决定,不能写的太多,坏事少写一件不算什么,把这件事也算到\ruby{吉冈安直}{よしおか やすなお}的账上去吧。\\

写完了,我又考虑写得太少也不好。于是我把能写的尽量写详细。写完了坦白材料,我又尽量地写检举材料。\\

材料都交上去了。我等待着检察人员的传讯。\\

在等待中,我不住地猜想着审问时候的场面。检察人员跟所方人员一样不一样?凶不凶?是不是要动刑?\\

在我脑子里,审问犯人是不可能不厉害的。我在紫禁城和宫内府里对待犯过失的太监、仆役,就向来离不开刑具。\\

我怕死,更怕受刑。不用说皮肉受苦,即使有人像我从前对待别人那样打我一顿耳光,也不如死了的好。我曾经认为,住共产党的监狱如果受不到野蛮的虐待是不可能的。进了管理所之后受到的待遇,是出乎意料的。这里不打人、不骂人,人格受到尊重。三年多来,一贯如此,按说我不该再有什么怀疑,可是一想到审问,总还是不放心,因为我认为审问就是审问,犯人不可能跟问官一致,问官不可能相信犯人,结果自然会僵住,自然是有权威的问官要打人,这本是无可非议的。\\

我在这些念头的折磨下,过了十多天寝食不安的日子。终于等到了这一天,看守员来通知我去谈话。\\

我被领进中央甬道里的一间屋子。这间屋子大约有两丈见方。当中有一张大书桌,桌前有个茶几,放着茶碗茶壶和烟灰碟。一位中年人和一位青年坐在桌后。他们示意,让我在茶几旁的椅子上坐下。\\

“你叫什么名字?”那中年人问。\\

“\xpinyin*{爱新觉罗}·\xpinyin*{溥仪}。”\\

他问了年龄、籍贯和性别。那个青年的笔尖,随着我们的谈话“嚓、嚓”地在纸上动着。\\

“你写的坦白材料我们看了,”那中年人说,“想听你当面谈谈。你可以抽烟。”\\

就这样开始了。中年的检察员从我幼时问起,问到我被捕。我都说完了,他对我点点头,样子好像还满意。\\

“好吧,就谈到这里。以后赵讯问员可能有问题问你。”\\

总之,这种讯问的气氛是颇出乎意料的。我心里少了一个问题。\\

第二次讯问,当我发现屋里只有赵讯问员一个人的时候,不禁有点失望。我坐在这位讯问员面前,注视着他的年轻的面庞,心中不住地想:他行吗?他弄得清楚吗?他能明白我说的话是真的?他正当血气方刚之年,有没有脾气?如果别人瞎检举我,他信谁的?……\\

“有个问题要问你一下,”他打断了我的思路,问起我在伪满时颁布\xpinyin*{敕}令和诏书的手续问题。我照着事实做了回答。在谈到一项\xpinyin*{敕}令时,他问我在颁布前几天看到的,我想不起来了。\\

“大概是一两天前,也许,三天,不,四天吧?”\\

“不用立刻回答,”他说,“你想想,几时想起几时说。现在谈另一个问题……”\\

在这另一个问题上,我又记不起来,僵在那里了。我心里不免暗暗着急:“我又想不起来啦,好像我不肯说似的,他该火了吧?”但是他并没发火,还是那句话:“这且放一边,你想起来再说。”\\

后来,我终于对这个年轻人完全服了。\\

已不记得那是第几次讯问了。他拿出一份我写的检举材料,放在我面前,问我:\\

“你写的这个检举材料上说,在日本战犯、前伪满总务厅次长\ruby{古海忠之}{ふるみ ただゆき}的策划下,日本侵略者在一年中掠去东北粮食一千六百万吨。这件事说的太不具体。是一年吗?是哪一年?一千六百万吨的数字怎么知道的?你再详细说说。”\\

我怎么能知道呢?这不过是我从同屋的两个伪大臣谈天中无意中听来的,我自然不敢把这件事说出来,只有学一下\xpinyin*{苏东坡}的“想当然耳”,说日寇对东北财富,无不尽力搜刮,粮食是产多少要多少。说到这里,讯问员拦住了我:\\

“东北年产粮食多少,你知道吗?”\\

我张口结舌,半晌说不出话来。\\

“你这条检举的根据是什么?”\\

我看是混不下去了,只好说出了这条马路情报的来源。\\

“那么,你相信不相信这个材料?”\\

“我,……没什么把握。”\\

“哦,连你自己也不信!”讯问员睁大了眼,“那么你为什么还要写?”\\

我正在呐呐然,不知说什么是好,他却把自来水笔的笔帽套好,收拾着桌上的纸张和书本,有厚厚的伪满的《年鉴》、《政府公报》,显然是不再需要我的答案。这次讯问是他用这句话结束的:\\

“无论对人对己,都要实事求是。”\\

我望着这个比我年龄小十几岁的人,没有话说。我从心底承认了他的话。因为我就害怕着别人给我编造和夸大呀。\\

我走出讯问室,心底\xpinyin*{蓦}地冒出一个问题:“是不是每个讯问员都是像这小伙子似的认真呢?倘若有一个不是这样,而正巧收到了诬赖我的检举材料,那怎么办呢?”\\

这个问题很快就得到了答案。同屋的老元后来告诉我们一件同样的经历。他曾按估算写了日本从东北掠夺钢铁的数字,讯问员不相信,给他一支铅笔,叫他算一算生产这些钢铁需要多少矿石,东北各矿年产多少矿石……。“他带着东北资源档案哩!”老元最后这样说。\\

因此我也明白了为什么赵讯问员的桌子上放着那些《年鉴》、《公报》之类的材料。不过工作团为了查证每件材料,使用了几百名调查人员,花了一年多的时间,跑遍了各地城乡,翻遍了数以吨计的档案,这还是到了我在检察员的总结意见书上签字时才知道的。\\

我在年轻的讯问员那里碰了一个钉子,由于他的实事求是的精神感到高兴,又因自己的愚蠢而担心他把我看做不老实的人。因此我赶紧写了一个自我检讨书给他送去。\\

“情形不像很严重。”交出了检讨书,我这样的想。
