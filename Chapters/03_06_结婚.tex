\fancyhead[LO]{{\scriptsize 1917-1924: 北京的“小朝廷” · 结婚}} %奇數頁眉的左邊
\fancyhead[RO]{} %奇數頁眉的右邊
\fancyhead[LE]{} %偶數頁眉的左邊
\fancyhead[RE]{{\scriptsize 1917-1924: 北京的“小朝廷” · 结婚}} %偶數頁眉的右邊
\chapter*{结婚}
\addcontentsline{toc}{chapter}{\hspace{1cm}结婚}
\thispagestyle{empty}
当王公大臣们奉了太妃们之命,向我提出我已经到了“大婚”年龄的时候,如果说我对这件事还有点兴趣的话,那因为结婚是个成人的标志,经过这道手续,别人就不能把我像个孩子似地管束了。\\

对这类事情最操心的是老太太们。民国十年年初,即我刚过了十五周岁的时候,太妃们把我父亲找去商议了几次,接着,召集了十位王公,讨论这件事。从议婚到成婚,经历了将近两年的时间。在这中间,由于\xpinyin*{庄和}太妃和我母亲的先后去世,师傅们因时局不宁谏劝从缓,特别是发生了情形颇为复杂的争执,婚事曾有过几起几落,不能定案。\\

这时\xpinyin*{庄和}太妃刚去世,\xpinyin*{荣惠}太妃没什么主见,剩下的两个太妃,对未来“皇后”人选,发生了争执,都想找一个跟自己亲近些的当皇后。这不单是由于老太太的偏爱,而是由于和将来的地位大有关系。\xpinyin*{敬懿}太妃原是\xpinyin*{同治}妃,她总忘不了\xpinyin*{慈禧}在遗嘱上把我定为承继\xpinyin*{同治}、兼桃\xpinyin*{光绪}的这句话。\xpinyin*{隆裕}太后在世时满不睬这一套,不但没有因为这句话而对\xpinyin*{同治}的妃有什么尊重的表示,反而把\xpinyin*{同治}的妃打人了冷宫。\xpinyin*{隆裕}死后,虽然太妃被我一律以皇额娘相称,但\xpinyin*{袁世凯}又来干涉“内政”,指定\xpinyin*{端康}主持宫中一切事务,因此\xpinyin*{敬懿}依然不能因“正宗”而受到重视。她的素志未偿,对\xpinyin*{端康}很不服气。所以在议婚过程中,这两个太妃各自提出了自己中意的候选人,互不相让。\\

最有趣的是我的两位叔父,就像从前一个强调海军,一个强调陆军,在摄政王面前各不相让的情形一样,也各为一位太妃奔走。“海军”主张选\xpinyin*{端恭}的女儿,“陆军”主张选\ruby{荣源}{\textcolor{PinYinColor}{Žung Yuwan}}的女儿。为了做好这个媒,前清的这两位统帅连日仆仆风尘于京津道上,匆匆忙忙出入于永和宫和太极殿。\\

究竟选谁,当然要“皇帝”说话,“钦定”一下。\xpinyin*{同治}和\xpinyin*{光绪}时代的办法,是叫候选的姑娘们,站成一排,由未来的新郎当面挑拣,挑中了的当面做出个记号来——我听到的有两个说法,一说是递玉如意给中意的姑娘,一说是把一个荷包系在姑娘的扣子上。到我的时代,经过王公大臣们的商议,认为把人家闺女摆成一排挑来挑去,不大妥当,于是改为挑照片的办法:我看着谁好,就用铅笔在照片上做个记号。\\

照片送到了养心殿,一共四张。在我看来,四个人都是一个模样,身段都像纸糊的桶子。每张照片的脸部都很小,实在分不出丑俊来,如果一定要比较,只能比一比旗袍的花色,谁的特别些。我那时想不到什么终身大事之类的问题,也没有个什么标准,便不假思索地在一张似乎顺眼一些的相片上,用铅笔画了一个圈儿。\\

这是满洲\ruby{额尔德特}{\textcolor{PinYinColor}{Erdet}}氏\xpinyin*{端恭}的女儿,名叫\xpinyin*{文绣},又名\xpinyin*{惠心},比我小三岁,看照片的那年是十二岁。这是\xpinyin*{敬懿}太妃所中意的姑娘。这个挑选结果送到太妃那里,\xpinyin*{端康}太妃不满意了,她不顾\xpinyin*{敬懿}的反对,硬叫王公们来劝我重选她中意的那个,理由是\xpinyin*{文绣}家境贫寒,长的不好,而她推荐的这个是个富户,又长的很美。她推荐的这个是满洲正白旗\ruby{郭布罗}{\textcolor{PinYinColor}{Gobeir}}氏\ruby{荣源}{\textcolor{PinYinColor}{Žung Yuwan}}家的女儿,名\xpinyin*{婉容},字\xpinyin*{慕鸿}(后来在天津有个驻张园的日本警察写了一本关于我的书,把\xpinyin*{慕鸿}写成\xpinyin*{秋鸿},以后以讹传讹,又成了\xpinyin*{鸿秋}),和我同岁,看照片那年是十五岁。我听了王公们的劝告,心里想你们何不早说,好在用铅笔画圈不费什么事,于是我又在\xpinyin*{婉容}的相片上画了一下。\\

可是\xpinyin*{敬懿}和\xpinyin*{荣惠}两太妃又不愿意了。不知太妃们和王公们是怎么争辩的,结果\xpinyin*{荣惠}太妃出面说:“既然皇上圈过\xpinyin*{文绣},她是不能再嫁给臣民了,因此可以纳为妃。”我想,一个老婆我还不觉得有多大的必要,怎么一下子还要两个呢?我不大想接受这个意见。可是禁不住王公大臣根据祖制说出“皇帝必须有后有妃”的道理,我想既然这是皇帝的特点,我当然要具备,于是答应了他们。\\

这个选后妃的过程,说得简单,其实是用了一年的时间才这样定下来的。定下来之后,发生了直奉战争,婚礼拖下来了,一直拖到了民国十一年十二月一日,这时\xpinyin*{徐世昌}已经下台,而大规模的婚礼筹备工作已经收不住\xpinyin*{辔}头,只得举行。王公们对二次上台的\xpinyin*{黎元洪}总统不像对\xpinyin*{徐世昌}那么信赖,生怕他对婚礼排场横加干涉,但是事情的结果,\xpinyin*{黎元洪}政府答应给的支持,出乎意料的好;即使\xpinyin*{徐世昌}在台上,也不过如此。民国的财政部写来一封颇含歉意的信给内务府,说经费实在困难,以致优待岁费不能发足,现在为助大婚,特意从关税款内拨出十万元来,其中两万,算民国贺礼。同时,民国政府军、宪、警各机关还主动送来特派官兵担任警卫的计划。其中计开:\\

\begin{quote}
	淑妃妆\xpinyin*{奁}进宫。步军统领衙门派在神武门、东安门等处及妆\xpinyin*{奁}经过沿途站哨官员三十名,士兵三百名。皇后妆\xpinyin*{奁}进官。步军统领衙门派在神武门、皇后宅等处及随行护送妆奋经过沿途站哨官员三十一名,士兵四百十六名(其中有号兵六名)。\\

行册立(皇后)礼。派在神武门、皇后宅等处及随行护送经过沿途站哨步军统领衙门官员三十四名(其中有军乐队官员三人),士兵四百五十八名(其中有军乐队士兵四十二人,号兵六人)。宪兵司令部除官员九名、士兵四十名外还派二个整营沿途站哨。\\

淑妃进宫。派在神武门、淑妃宅等处及随行护送经过沿途站哨步军统领衙门官员三十一名、士兵四百十六名。宪兵司令部官员三名,士兵十四名。警察厅官兵二百八十名。\\

行奉迎(皇后)\xpinyin*{札}。派在东华门、皇后宅等处及随行护送经过沿途站哨步军统领衙门官兵六百十名,另有军乐队一队。宪兵司令部除官兵八十四名外,并于第一、二、五营中各抽大部分官兵担任沿途站哨。警察厅官兵七百四十七名。\\

在神武门、东华门、皇后宅、淑妃宅等处及经过地区警察厅所属各该管区,加派警察保护。\\

本来按民国的规定,只有神武门属于清宫,这次破例,特准“凤\xpinyin*{舆}”从东华门进宫。\\
\end{quote}

婚礼全部仪程是五天:\\

\begin{quote}
	十一月二十九日已刻,淑妃妆\xpinyin*{奁}入宫。\\

十一月三十日午刻,皇后妆\xpinyin*{奁}入宫。\xpinyin*{巳}刻,皇后行册立礼。丑刻,淑妃入宫。\\

十二月一日子刻,举行大婚典礼。寅刻,迎皇后入宫。\\

十二月二日帝后在景山寿皇殿向列祖列宗行礼。\\

十二月三日帝在乾清宫受贺。\\
\end{quote}

在这个仪程之外,还从婚后次日起连演三天戏。在这个礼仪之前,即十一月十日,还有几件事预先做的,即纳采礼,晋封四个太妃(四太妃从这天起才称太妃)。事后,又有一番封赏荣典给王公大臣,不必细说了。\\

这次举动最引起社会上反感的,是小朝廷在一度复辟之后,又公然到紫禁城外边摆起了威风。在民国的大批军警放哨布岗和恭敬护卫之下,清宫仪仗耀武扬威地在北京街道上摆来摆去。正式婚礼举行那天,在民国的两班军乐队后面,是一对穿着蟒袍补褂的册封正副使(庆亲王和郑亲王)骑在马上,手中执节(像苏武牧羊时手里拿的那个鞭子),在他们后面跟随着民国的军乐队和陆军马队、警察马队、保安队马队。再后面则是龙凤旗伞、鸾驾仪仗七十二副,黄亭(内有皇后的金宝礼服)四架,宫灯三十对,浩浩荡荡,向“后邸”进发。在张灯结彩的后邸门前,又是一大片军警,保卫着\xpinyin*{婉容}的父亲\ruby{荣源}{\textcolor{PinYinColor}{Žung Yuwan}}和她的兄弟们——都跪在那里迎接正副使带来的“圣旨”……\\

民国的头面人物的厚礼,也颇引人注目。大总统\xpinyin*{黎元洪}在红帖子上写着“中华民国大总统\xpinyin*{黎元洪}赠\xpinyin*{宣统}大皇帝”,礼物八件,计:珐琅器四件,绸缎二种,帐一件,联一副,其联文云:“汉瓦当文,延年益寿,周铜盘铭,富贵吉祥”。前总统\xpinyin*{徐世昌}送了贺礼二万元和许多贵重的礼物,包括二十八件瓷器和一张富丽堂皇的龙凤中国地毯。\xpinyin*{张作霖}、\xpinyin*{吴佩孚}、\xpinyin*{张勋}、\xpinyin*{曹锟}等军阀、政客都赠送了现款和许多别的礼物。\\

民国派来总统府侍从武官长\xpinyin*{荫昌},以对外国君主之礼正式祝贺。他向我鞠躬以后,忽然宣布:“刚才那是代表民国的,现在奴才自己给皇上行礼。”说罢,跪在地下磕起头来。\\

当时许多报纸对这些怪事发出了严正的评论,这也挡不住王公大臣们的兴高采烈,许多地方的遗老们更如惊蛰后的虫子,成群飞向北京,带来他们自己的和别人的现金、古玩等等贺礼。重要的还不是财物,而是声势,这个声势大得连他们自己也出乎意外,以致又觉得事情像是大有可为的样子。\\

最令王公大臣、遗老遗少以及太妃们大大兴奋的,是东交民巷来的客人们。这是\xpinyin*{辛亥}以后紫禁城中第一次出现外国官方人员。虽然说他们是以私人身分来的,但毕竟是外国官员。\\

为了表示对外国客人观礼的重视和感谢,按\ruby{庄士敦}{\textcolor{PinYinColor}{Johnston}}的意思,在乾清宫特意安排了一个招待酒会,由\xpinyin*{张勋}复辟时的“外务部大臣”\xpinyin*{梁敦彦}给我拟了一个英文谢词,我按词向外宾念了一遍。这个谢词如下:\\

\begin{quote}
	今天在这里,见到来自世界各地的高贵客人,朕感到不胜荣幸。谢谢诸位光临,并祝诸位身体的健康,万事如意。\\
\end{quote}

在这闹哄哄之中,我从第一天起,一遍又一遍地想着一个问题:“我有了一后一妃,成了家了,这和以前的区别何在呢?”我一遍又一遍地回答自己:“我成年了。如果不是闹革命,是我‘亲政’的时候开始了!”\\

除了这个想法之外,对于夫妻、家庭,我几乎连想也没想它。只是当头上蒙着一块绣着龙凤的大红缎子的皇后进入我眼帘的时候,我才由于好奇心,想知道她长的什么模样。\\

按着传统,皇帝和皇后新婚第一夜,要在坤宁宫里的一间不过十米见方的喜房里渡过。这间屋子的特色是:没有什么陈设,炕占去了四分之一,除了地皮,全涂上了红色。行过“合\xpinyin*{卺}礼”,吃过了“子孙饽饽”,进入这间一片暗红色的屋子里,我觉得很憋气。新娘子坐在炕上,低着头,我在旁边看了一会,只觉着眼前一片红:红帐子、红褥子、红衣、红裙、红花朵、红脸蛋……好像一摊溶化了的红蜡烛。我感到很不自在,坐也不是,站也不是。我觉得还是养心殿好,便开开门,回来了。\\

我回到养心殿,一眼看见了裱在墙壁上的\xpinyin*{宣统}朝全国各地大臣的名单,那个问题又来了:\\

“我有了一后一妃,成了人了,和以前有什么不同呢?”\\

被孤零零地扔在坤宁宫的\xpinyin*{婉容}是什么心情?那个不满十四岁的\xpinyin*{文绣}在想些什么?我连想也没有想到这些。我想的只是:\\

“如果不是革命,我就开始亲政了……我要恢复我的祖业!”
