\fancyhead[LO]{{\scriptsize 1917-1924: 北京的“小朝廷” · 不绝的希望}} %奇數頁眉的左邊
\fancyhead[RO]{\thepage} %奇數頁眉的右邊
\fancyhead[LE]{\thepage} %偶數頁眉的左邊
\fancyhead[RE]{{\scriptsize 1917-1924: 北京的“小朝廷” · 不绝的希望}} %偶數頁眉的右邊
\chapter*{不绝的希望}
\addcontentsline{toc}{chapter}{\hspace{11mm}不绝的希望}
%\thispagestyle{empty}
有一天,我在御花园里骑自行车玩,骑到拐角的地方,几乎撞着一个人。在宫里发生这样的事情,应该算这个人犯了君前失礼的过失,不过我倒没有理会。我的车子在那里打了个圈子,准备绕过去了,不料这个人却跪下来不走,嘴里还说:\\

“小的给万岁爷请安!”\\

这人身上的紫色坎肩,和太监穿的一样。我瞅了他一眼,看见他嘴上还有一抹胡茬子,知道他并不是太监。我骑着车打着圈子问他:\\

“干什么的?”\\

“小的是管电灯的。”\\

“噢,你是干那玩艺的。刚才没摔着,算你运气。干么你老跪着?”\\

“小的运气好,今天见着了真龙天子。请万岁爷开开天恩,赏给小的个爵儿吧!”\\

我一听这傻话就乐了。我想起了太监们告诉我的,北京街上给蹲桥头的乞丐起的诨名,就说:\\

“行,封你一个‘镇桥侯(猴)’吧!哈哈……”\\

我开完了这个玩笑,万没有想到,这个中了官迷的人后来果真找内务府要“官\xpinyin*{诰}”去了。内务府的人说:“这是一句笑话,你干么认真?”他急了:“皇上是金口玉言,你们倒敢说是笑话,不行!……”这件事后来怎么了结的,我就不知道了。\\

那时我常常听到师傅们和太监们说,内地乡下总有人间:“\ruby{宣统}{\textcolor{PinYinColor}{\Man ᡤᡝᡥᡠᠩᡤᡝ ᠶᠣᠰᠣ}}皇帝怎么样了?”“现在坐朝廷的是谁?”“真龙天子坐上了宝座,天下就该太平了吧?”我的英国师傅根据一本刊物上的文章说,连最反对帝制的人也对共和感到了失望,可见反对帝制的人也变了主意。其实人们念叨一下“前清”,不过是表示对军阀灾难的痛恨而已。我的师傅们却把这些诅咒的语言抬了来,作为人心思旧的证据,也成了对我使用的教材。\\

不过中了迷的人,在\xpinyin*{徐世昌}时代的末期,倒也时时可以遇到。有个叫\xpinyin*{王九成}的商人,给直系军队做军装发了财,他为了想得一个穿黄马褂的赏赐,曾花过不少功夫,费了不少钞票。太监们背后给他起了一个绰号,叫散财童子。不知他通过什么关节,每逢年节就混到遗老中间来磕头进贡,来时带上大批钞票,走到哪里散到哪里。太监们最喜欢他来,因为不管是给他引路的,传见的,打帘子的,倒茶的,以及没事儿走过来和他说句话儿的,都能得到成卷儿的钞票。至于在各个真正的关节地方花的钱,就更不用说了。最后他真的达到了目的,得到了赏穿黄马褂的“荣誉”。\\

为了一件黄马褂,为了将来续家谱时写上个清朝的官衔,为了死后一个谥法,那时每天都有人往紫禁城跑,或者从遥远的地方寄来奏折。绰号叫梁疯子的\xpinyin*{梁巨川},不惜投到北京积水潭的水坑里,用一条性命和泡过水的“遗折”,换了一个“贞端”的谥法。后来伸手要谥法的太多了,未免有损小朝廷的尊严,所以规定三品京堂\footnote{清制凡都察院、通政司、詹事府以及其他诸卿寺的堂官均称为京堂,除左都御史外都是三、四品官。后来京堂便兼用为三、四品京官的虚衔,因此,三品京堂即是指三品京官。}以下的不予赐谥,以为限制。至于赏紫禁城骑马,赏乘坐二人肩\xpinyin*{舆},赐写春条、福寿字、对联等等,限制就更严些。那时不但是王公大臣,就是一些民国的将领们如果获得其中的一种,也会认为是难得的“殊荣”。那些官职较低或者在前清没有“前程”,又没有\xpinyin*{王九成}那种本钱,走不进紫禁城的人,如当时各地的“商绅”之类,他们也有追求的目标,这便是等而下之,求遗老们给死了的长辈灵牌上“点主”,写个墓志铭,在儿女婚礼上做个证婚人。上海地皮大王英籍犹太人\ruby{哈同}{\textcolor{PinYinColor}{Hardoon}}的满族籍夫人\ruby{罗迦陵}{\textcolor{PinYinColor}{Lisa Roos}},曾把清朝最末一位状元\xpinyin*{刘春霖},以重礼聘到上海,为他准备了特制的八人绿呢大轿,请他穿上清朝官服,为她的亡夫灵牌点主。当时某些所谓新文人如\xpinyin*{胡适}、\xpinyin*{江亢虎}等人也有类似的举动。我十五岁时从\ruby{庄士敦}{\textcolor{PinYinColor}{Johnston}}师傅的谈话中,知道了有位提倡白话文的\xpinyin*{胡适}博士。\ruby{庄士敦}{\textcolor{PinYinColor}{Johnston}}一边嘲笑他的中英合壁的“\ruby{匹克尼克}{\textcolor{PinYinColor}{Picnic}}来江边”的诗句,一边又说“不妨看看他写的东西,也算一种知识”。我因此动了瞧一瞧这个新人物的念头。有一天,在好奇心发作之下打了个电话给他,没想到一叫他就来了。这次会面的情形预备后面再谈,这里我要提一下在这短暂的而无聊的会面之后,我从\xpinyin*{胡适}给\ruby{庄士敦}{\textcolor{PinYinColor}{Johnston}}写的一封信上发现,原来洋博士也有着那种遗老似的心理。他的信中有一段说:\\

\begin{quote}
	我不得不承认,我很为这次召见所感动。我当时竟能在我国最末一代皇帝——历代伟大的君主的最后一位代表的面前,占一席位!\\
\end{quote}

更重要的是,紫禁城从外国人的议论上也受到了鼓舞。\ruby{庄士敦}{\textcolor{PinYinColor}{Johnston}}曾告诉我不少这方面的消息。据他说,很多外国人认为复辟是一般中国人的愿望。他有时拿来外文报纸讲给我听,他后来抄进了他的著作《紫禁城的黄昏》中的一段,是他曾讲过的。这是刊在一九一九年九月十九日天津《华北每日邮电》上的一篇题为《另一次复辟是不是在眼前?》的社论中的一段:\\

\begin{quote}
	共和政府的经历一直是惨痛的。今天我们看到,南北都在剑拔弩张,这种情形只能引出这样结论:在中国,共和政体经过了试验并发现有缺点。\\

这个国家的中坚分子——商人阶层和士绅,很厌恶种种互相残杀的战争。\\

我们深信,他们一定会衷心拥护任何形式的政府,只要它能确保十八省的太平就行。\\

不要忘记,保皇党是有坚强阵容的。他们对共和体从来不满,但由于某种原因,他们近几年保持着缄默。显然,他们同情着军阀的行动,他们有些知名之士奔走于军人集会的处所,并非没有意义。\\

那些暗地赞同和希望前皇帝复辟成功的人的论点是,共和主义者正在破坏这个国家,因而必然采取措施——甚至是断然措施——来恢复旧日的欣欣向荣、歌舞升平的气象。\\

复辟帝制绝不会受到多方面的欢迎,相反,还会受到外交上的相当大的反对,反对的公使馆也不只一个。可是,只要政变成功,这种反对就必然消失,因为我们知道:成者为王败者寇。\\
\end{quote}

当然,尽管在外国人的报纸上有了那么多的鼓励性的话,直接决定小朝廷的安危和前途祸福的,还是那些拿枪杆子的军人。正如《华北每日邮电》所说,“奔走于军人集会的处所,并非没有意义”。我记得这年(1919年)的下半年,紫禁城里的小朝廷和老北洋系以外的军人便有了较亲密的交往。第一个对象是奉系的首领\xpinyin*{张作霖}巡间使。\\

起初,紫禁城收到了奉天汇来的一笔代售皇产庄园的款子,是由我父亲收转的。我父亲去函致谢,随后内务府选出两件古物,一件是《御制题咏\xpinyin*{董邦达}淡月寒林图》画轴,另一件是一对\ruby{乾隆}{\textcolor{PinYinColor}{\Man ᠠᠪᡴᠠᡳ ᠸᡝᡥᡳᠶᡝᡥᡝ}}款的瓷瓶,用我父亲的名义赠馈\xpinyin*{张作霖},并由一位三品专差\xpinyin*{唐铭盛}直接送到奉天。\xpinyin*{张作霖}派了他的把兄弟,当时奉军的副总司令,也就是后来当了伪满国务总理的\xpinyin*{张景惠},随\xpinyin*{唐铭盛}一起回到北京,答谢了我的父亲。从此,醇王府代表小朝廷和奉军方面有了深一层的往来。在\xpinyin*{张勋}复辟时,曾有三个奉军的将领(\xpinyin*{张海鹏}、\xpinyin*{冯德麟}、\xpinyin*{汤玉麟})亲身在北京参加了复辟,现在又有了\xpinyin*{张景惠}、\xpinyin*{张宗昌}被赐紫禁城骑马。\xpinyin*{张宗昌}当时是奉军的师长,他父亲在北京做八十岁大寿时,我父亲曾亲往祝贺。民国九年,直皖战争中直系联合了奉系打败了皖系,直系首领(\xpinyin*{冯国璋}已死)\xpinyin*{曹锟}和奉系首领\xpinyin*{张作霖}进北京之后,小朝廷派了内务府大臣\xpinyin*{绍英}亲往迎接。醇王府更忙于交际。因为一度听说\xpinyin*{张作霖}要进宫请安,内务府大臣为了准备赐品,特意到醇王府聚议一番。结果决定,在预定的一般品目之外,加上一把古刀。我记得\xpinyin*{张作霖}没有来,又回奉天去了。两个月后,醇王身边最年轻的一位贝勒得了\xpinyin*{张作霖}顾问之衔,跟着就到奉天去了一趟。皖系失败,直奉合作期间,北京的奉天会馆成了奉系的将领们聚会的地方,也是某些王公们奔走的地方。连醇王府的总管\xpinyin*{张文治}也成了这里的常客,并在这里和\xpinyin*{张景惠}拜了把兄弟。\\

这两年,和\xpinyin*{张勋}复辟前的情况差不多,复辟的“谣传”弄得满城风雨。下面是登在民国八年十二月二十七日(也就是醇亲王派人到奉天送礼品、和\xpinyin*{张景惠}来北京之后的两个月)英文《导报》上的发自奉天的消息:\\

\begin{quote}
	最近几天以来,在沈阳的各阶层人士中间,尤其是\xpinyin*{张作霖}将军部下中间盛传一种谣言,说将在北京恢复满清帝制以代替民国政府。根据目前的种种断言,这次帝制将由张将军发动,合作的则有西北的皇族的军事领导人,前将军\xpinyin*{张勋}也将起重要作用。……说是甚至于徐总统和前冯总统,鉴于目前国家局势以及外来危险,也都同意恢复帝制,至于\xpinyin*{曹锟}、\xpinyin*{李纯}以及其他次要的军人,让他们保持现有地位再当上王公,就会很满足了。……\\
\end{quote}

我从\ruby{庄士敦}{\textcolor{PinYinColor}{Johnston}}那里得知这段新闻,是比较靠后一些时间。我还记得,他同时还讲过许多其他关于\xpinyin*{张作霖}活动复辟的传说。大概这类消息一直传播到民国十一年,即\xpinyin*{张作霖}又败回东北时为止。我对上面这条消息印象特别深刻,它使我从心底感到了欣喜,我从而也明白了为什么奉军首领们对紫禁城那样热诚,为什么\xpinyin*{端康}“千秋”时\xpinyin*{张景惠}夹在王公大臣中间去磕头,为什么人们说奉天会馆特别热闹,某些王公们那样兴致勃勃。但是我们的高兴没有维持多久,扫兴的事就来了:直奉两系的合作突然宣告破裂,双方开起火来了。结果奉军失利,跑出了山海关。\\

令人不安的消息接连而至:\xpinyin*{徐世昌}忽然下台;直军统治了北京;在\xpinyin*{张勋}复辟时被赶下台的\xpinyin*{黎元洪},二次当了总统。紫禁城里发生了新的惊慌,王公大臣们请求\ruby{庄士敦}{\textcolor{PinYinColor}{Johnston}}带我到英国使馆去避难。\ruby{庄士敦}{\textcolor{PinYinColor}{Johnston}}和英国公使\ruby{贝尔利·阿尔斯顿}{Belize Aston}勋爵商议好,英国公使馆可以拨出一些房间,必要时我可以作为\ruby{庄士敦}{\textcolor{PinYinColor}{Johnston}}的私人客人,住到里面去。同时还和葡萄牙和荷兰公使馆商议好,可以容纳皇室其他人前去避难。我的想法和他们不同,我认为与其躲到外国使馆,还不如索性到外国去。我向\ruby{庄士敦}{\textcolor{PinYinColor}{Johnston}}提出,请他立即带我出洋。因为我是突然之间把他找来提出的,所以这位英国师傅怔住了,他几乎是来不及思索就回答我:“这是不合时宜的,陛下要冷静考虑,徐总统刚逃出北京,皇帝陛下立刻从紫禁城失踪,这会引起联想,说\xpinyin*{徐世昌}和清室有什么阴谋。再说,在这种情形下,英国也不会接待陛下……”\\

当时我却没有这种联想的本领,因为人们不曾告诉我,张。徐之间以及张、徐与小朝廷之间暗中发生的事情,当然更想不到直奉战争之发生以及这一场胜负和东交民巷的关系。我当时一听这个要求办不到,只好作罢。后来时局稳定了下来,没有人再提出洋,就连避难问题也不提了。\\

这是民国十一年春夏间的事。第二年,直系的首领\xpinyin*{曹锟}用五千元买一张选票的办法,贿赂议员选他当上了总统。小朝廷对这位直系首领的恐惧刚刚消失,又对另一位声望日高的直系首领\xpinyin*{吴佩孚}\footnote{\xpinyin*{吳佩孚}(1874年4月22日-1939年12月4日),字\xpinyin*{子玉},山东省蓬莱县人。晚清秀才,北洋军阀中曾經為實力最雄厚的軍閥之一,並担任直系军阀的首领,官至直鲁豫巡阅使。}发生了兴趣。后来到我身边来的\xpinyin*{郑孝胥}\footnote{\xpinyin*{郑孝胥}(1860-1938),字\xpinyin*{苏龛},号\xpinyin*{海藏},中国福建省闽县(今福州)人。他是清朝的改革派政治家,亦是满洲国建国的参与者之一,后出任满洲国国务总理。},此时向我献过策,说\xpinyin*{吴佩孚}是个最有希望的军人,他素来以关羽自居,心存大清社稷,大可前去游说。这年\xpinyin*{吴佩孚}在洛阳做五十整寿,在我同意之下,\xpinyin*{郑孝胥}带了一份厚礼前去拜寿。但\xpinyin*{吴佩孚}的态度若即若离,没有明白的表示。后来\xpinyin*{康有为}又去游说他,也没得到肯定的答复。事实上,吴的得意时代也太短促了,就在他做寿的第二年,直奉两系之间发生战争,\xpinyin*{吴佩孚}部下的\xpinyin*{冯玉祥}“倒戈”,宣布和平,结果\xpinyin*{吴佩孚}一败涂地,我也在紫禁城坐不住,被\xpinyin*{冯玉祥}的国民军赶了出来。\\

在我结婚前这几年沧海白云之间,小朝廷里王公大臣们的心情变化,并不完全一样。表现最为消极的是内务府领衔大臣\xpinyin*{世续}。他从\xpinyin*{丁巳}复辟起,越来越泄气,后来成了完全灰心悲观的人。他甚至和人这样说过,就算复辟成功,对我也没有什么好处,因为那些不知好歹的年轻王公,必定会闹出一场比\xpinyin*{辛亥}年更大的乱子。他又说:“就算王公出不了乱子,这位皇帝自己也保不了险,说不定给自己会弄个什么结局。”他的主张,是让我和蒙古王公结亲,以便必要时跑到老丈人家里去过日子。\xpinyin*{世续}死于我结婚前一年左右,去世前一年即因病不再问事了,代替他的是\xpinyin*{绍英}。\xpinyin*{绍英}的见识远不如他的前任,谨慎小心、胆小怕事则有过之。在\xpinyin*{绍英}心里,只有退保,决无进取打算。他要保守的与其说是我这个皇上,倒不如说是“优待条件”。因为保住了这个东西,就等于保住了他的一切——从财产生命到他的头衔。他是首先从\ruby{庄士敦}{\textcolor{PinYinColor}{Johnston}}身上看到这种保险作用的。他宁愿把自己的空房子白给外国人住,也不收肯出高租的中国人为房客。\ruby{庄士敦}{\textcolor{PinYinColor}{Johnston}}自己不愿意领他这份情,帮忙给找了一个外国人做了他的邻居,在他的屋顶上挂上了外国的国旗,因此他对\ruby{庄士敦}{\textcolor{PinYinColor}{Johnston}}是感思不尽的。\\

处于最年轻的王公和最年老的内务府大臣之间的是\xpinyin*{陈宝琛}师傅。他不像\xpinyin*{世续}那样悲观,不像\xpinyin*{绍英}那样除了保守优待条件以外,别的事连想也不想,也不像年轻的王公们对军人们那么感到兴趣。他并不反对和军人们联络,他甚至自己亲自出马去慰劳过\xpinyin*{冯玉祥},在商议给军人送礼时,出主意也有他一份,不过他一向对军人不抱希望。他所希望的,是军人火并到最后,民国垮了台,出现“天与人归”的局势。因此,在\xpinyin*{张勋}失败后,他总是翻来覆去地给我讲《\xpinyin*{孟子}》里的这一段:\\

\begin{quote}
	故天将降大任于斯人也,必先苦其心志,劳其筋骨,饿其体肤,空乏其身,行拂乱其所为,所以动心忍性,曾益其所不能。\\
\end{quote}

\xpinyin*{陈宝琛}本来是我惟一的灵魂。不过自从来了\ruby{庄士敦}{\textcolor{PinYinColor}{Johnston}},我又多了一个灵魂。
