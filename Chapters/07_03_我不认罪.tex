\fancyhead[LO]{{\scriptsize 1945-1950: 在苏联的五年 · 我不认罪}} %奇數頁眉的左邊
\fancyhead[RO]{\thepage} %奇數頁眉的右邊
\fancyhead[LE]{\thepage} %偶數頁眉的左邊
\fancyhead[RE]{{\scriptsize 1945-1950: 在苏联的五年 · 我不认罪}} %偶數頁眉的右邊
\chapter*{我不认罪}
\addcontentsline{toc}{chapter}{\hspace{11mm}我不认罪}
%\thispagestyle{empty}
既然放不下架子,又不肯学习,我的思想根本不起变化,认罪自然更谈不到。\\

我知道,在法律面前,我是犯有叛国罪的。但我对这件事,只看做是命运的偶然安排。“强权就是公理”和“胜者王侯败者寇”,这就是我那时的思想。我根本不去想自己该负什么责任,当然更想不到支配我犯罪的是什么思想,也从来没有听说过什么思想必须改造。\\

为了争取摆脱受惩办的厄运,我采取的办法仍然是老一套。既然在眼前决定我命运的是苏联,那么就向苏联讨好吧。于是我便以支援战后苏联的经济建设为词,向苏联献出了我的珠宝首饰。\\

我并没有献出它的全部,我把其中最好的一部分留了下来,并让我的侄子把留下的那部分,藏进一个黑色皮箱的箱底夹层里。因为夹层小,不能全装进去,就又往一切我认为可以塞的地方塞,以致连肥皂里都塞满了,还是装不下,最后只好把未装下的扔掉。\\

有一天,苏联的翻译和一个军官走进大厅,手里举着一个亮晃晃的东西向大家问道:\\

“这是谁的?谁放在院子里的废暖气炉片里的?”\\

大厅里的抑留者们都围了过去,看出军官手里的东西是一些首饰。有人说:“这上面还有北京银楼的印记呢,奇怪,这是谁搁的呢?”\\

我立刻认出来,这是我叫侄子们扔掉的。这时他们都在另一个收容所里,我也就不去认账,连忙摇头道:\\

“奇怪,奇怪,这是谁搁的呢?……”\\

不料那翻译手里还有一把旧木梳,他拿着它走到我跟前说:\\

“在一块的还有这个东西。我记得,这木梳可是你的呢!”\\

我慌张起来,连忙否认说:“不是不是!木梳也不是我的!”\\

弄得这两个苏联人没办法,怔了一阵,最后只好走了。他们可能到现在还没弄清楚,我这个人到底是什么心理。其实我只有一个心理,这就是怕承认了这件事会引起他们对我的猜疑,所以我采取了一推二赖的办法。我推得竟这样笨,不由得不使他们发怔了。\\

我不但扔了一些首饰,还放在炉子里烧了一批珍珠。在临离开苏联之前,我叫我的佣人大李把最后剩下的一些,扔进了房顶上的烟囱里。\\

我对日本人是怨恨的。苏联向我调查日寇在东北的罪行时,我以很大的积极性提供了材料。后来我被召到东京的“远东国际军事法庭”去作证,我痛快淋漓地控诉了日本战犯。但我每次谈起那段历史,从来都不谈我自己的罪过,而且尽力使自己从中摆脱出来。因为我怕自己受审判。\\

我到东京“远东国际军事法庭”去作证,是在一九四六年的八月间。我共计出庭了八天,据说这是这个法庭中作证时间最长的一次。那些天的法庭新闻,成了世界各地某些以猎奇为能事的报纸上的头等消息。\\

证实日本侵略中国的真相,说明日本如何利用我这个清朝末代皇帝为傀儡,以进行侵略和统治东北四省,这是对我作证的要求。\\

今天回想起那一次作证来,我感到很遗憾。由于那时我害怕将来会受到祖国的惩罚,心中顾虑重重,虽然说出了日本侵略者的一部分罪恶事实,但是为了给自己开脱,我在掩饰自己的罪行的同时,也掩盖了一部分与自己的罪行有关的历史真相,以致没有将日本帝国主义的罪行,予以充分的、彻底的揭露。\\

日本帝国主义者和以我为首的那个集团的秘密勾结,这本是在“九·一八”以前就开始了的。日本人对我们这伙人的\xpinyin*{豢}养。培植,本来也是公开的秘密。“九·一八”事变后我们这伙人的公开投敌,就是与日本人长期勾结的结果。我为了开脱自己,却回避了这个问题,只顾谈了我怎么被逼和受害。\\

外边的帝国主义和里边的反动势力的勾结,跟任何黑帮搭伙一样,内部摩擦是不可避免的,而我却把这类事说成好像是善与恶的冲突。\\

我在法庭上曾有几次表现了激动。谈到了迎接“天照大神”那回事时,一个日本律师向我提出,我攻击了日本天皇的祖宗,这很不合乎东方的道德。我激昂地大声咆哮:“我可是并没有强迫他们,把我的祖先当他们的祖先!”这引起了哄堂大笑,而我犹忿忿不已。提起了\xpinyin*{谭玉龄}之死,我把自己的怀疑也当做了已肯定了的事实,并且悲忿地说:“连她,也遭到了日本人的杀害!”固然,这时我的心情是激动的,但同时,我更愿意人人把我看成是一个被迫害者。\\

被告的辩护人为了减轻被告的罪,曾使用了许多办法来对付我,企图降低我的证言价值,甚至想否定我的证人资格。当然,他们是失败了;即使他们真把我全否定了,也无法改变被告者的命运。但是如果他们是在利用我的畏惧惩罚的心理,使我少谈真相,那么他们是达到了部分目的。我还记得在我历数日本战犯的罪行之后,一个美国律师对我大嚷大叫:“你把一切罪行都推到日本人身上,可是你也是罪犯,你终究要受中国政府的裁判的!”他这话确实打中了我的要害,说到了我最害怕的地方。我就是出于这种心理,才把投敌叛国说成是被绑架的结果的。我把我与日本的勾结,一律否认,甚至在法庭上拿出了我给\ruby{南}{\textcolor{PinYinColor}{みなみ}}\ruby{次郎}{\textcolor{PinYinColor}{じろう}}写的信时,我也坚决否认,说成是日本人伪造的。我掩盖了这件事,也掩盖了日本军国主义的种种阴谋手段,所以到头来还是便宜了日本军国主义者。
