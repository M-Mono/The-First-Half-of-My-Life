\fancyhead[LO]{{\scriptsize 1945-1950: 在苏联的五年 · 疑惧和幻想}} %奇數頁眉的左邊
\fancyhead[RO]{\thepage} %奇數頁眉的右邊
\fancyhead[LE]{\thepage} %偶數頁眉的左邊
\fancyhead[RE]{{\scriptsize 1945-1950: 在苏联的五年 · 疑惧和幻想}} %偶數頁眉的右邊
\chapter*{疑惧和幻想}
\addcontentsline{toc}{chapter}{\hspace{11mm}疑惧和幻想}
%\thispagestyle{empty}
飞机飞到赤塔,天差不多快黑了。我们是第一批到苏联的伪满战犯,和我同来的有\ruby{溥杰}{\textcolor{PinYinColor}{\Man ᡦᡠ ᡤᡳᠶᡝ}}、两个妹夫、三个侄子、一个医生和一个佣人。我们这一家人乘坐苏军预备好的小汽车,离开了机场。从车中向外瞭望,好像是走在原野里,两边黑忽忽的看不到尽头。走了一阵,穿过几座树林,爬过几道山坡,道路变得崎岖狭厌,车子速度也降低下来。忽然间车停了,车外传来一句中国话:\\

“想要解手的,可以下来!”\\

我不觉大吃一惊,以为是中国人接我们回去的。其实说话的是一位中国血统的苏联军官。在我前半生中,我的疑心病可把自己害苦了,总随时随地无谓地折磨自己。明明是刚刚坐着苏联飞机从中国飞到苏联来,怎么会在这里向中国人移交呢!这时我最怕的就是落在中国人手里。我认为落在外国人手里,尚有活命的一线希望,若到了中国人手里,则是准死无疑。\\

我们解完手,上了汽车,继续走了大约两小时,进入一个山峡间,停在一座灯火辉煌的楼房面前。我们这一家人下了车,看着这座漂亮的建筑,有人小声嘀咕说:“这是一家饭店呵!”大家都高兴起来了。\\

走进了这座“饭店”,迎面走过来一位四十多岁穿便服的人,后面跟着一群苏联军官。他庄严地向我们宣布道:\\

“苏联政府命令:从现在起对你们实行拘留。”\\

原来这是赤塔市的卫戍司令,一位苏联陆军少将。他宣布完了命令,很和气地告诉我们说,可以安心地住下,等候处理。说罢,指着桌上一个盛满了清水的瓶子说:\\

“这里是有名的矿泉,矿泉水是很有益于身体健康的饮料。”\\

这种矿泉水乍喝有点不大受用,后来却成了我非常喜欢的东西。我们就在这个疗养所里开始了颇受优待的拘留生活。每日有三顿丰盛的俄餐,一次俄式午茶。有服务员照顾着,有医生、护士经常检查身体,治疗疾病,有收音机,有书报,有各种文娱器材,还经常有人陪着散步。对这种生活,我立刻感到了满意。\\

住了不久,我便生出一个幻想:既然苏联和英美是盟邦,我也许还可以从这里迁到英美去做寓公。这时我还带着大批的珠宝首饰,是足够我后半生花用的。要想达到这个目的,首先必须确定我能在苏联住下来。因此,我在苏联的五年间,除了口头以外,共三次上书给苏联当局,申请准许我永远留居苏联。三次上书,一次是在赤塔,两次是在两个月以后迁到离中国不远的伯力。这三次申请,全无下文。\\

伪满的其他“抑留者”\footnote{拘留中的伪满文官身分是抑留者,武官是战犯。},在这个问题上,自始至终与我采取了完全相反的态度。\\

我到赤塔后不几天,\xpinyin*{张景惠}、\xpinyin*{臧式毅}、\xpinyin*{照治}等这批伪大臣便到了。大约是第二天,张、臧、熙等人到我住的这边来看我。我以为他们来给我请安的,不料却是向我请愿。\xpinyin*{张景惠}先开的口:\\

“听说您愿意留在苏联,可是我们这些人家口在东北,都得自己照料,再说,还有些公事没办完。请您跟苏联人说一说,让我们早些回东北去,您瞧行不行?”\\

他们有什么“公事”没办完,我不知道,也不关心,因此对于他们的请求,毫无兴趣。\\

“我怎么办得到呢?连我是留是去,还要看人家苏联的决定。”\\

这些家伙一听我不管,就苦苦哀求起来:“您说说吧,您一定做得到。”“这是大伙儿的意思,大伙推我们做代表来请求溥大爷的。”“大伙的事,不求您老人家,还能求谁呢?”\\

他们现在不能再叫我“皇上”、“陛下”,就没口地乱叫起来。我被缠的没法,只好找负责管理我们的苏联中校\ruby{渥罗阔夫}{\textcolor{PinYinColor}{\Rus Вороков}}。\\

中校\ruby{渥罗阔夫}{\textcolor{PinYinColor}{\Rus Вороков}}听了我告诉他的伪大臣们的要求,便说:“好吧,我代为转达。”\\

在我提出要求留苏申请的时候,他也是这样回答的。以后的情况也相同,没有下文。\\

但是这些大臣和我一样的不死心,迁到怕力市郊之后,我申请留在苏联,他们就申请回到东北,还是逼着我替他们说话。\\

那时我还不明白,他们比我了解国民党的政治内幕,知道国民党那些人对他们的特殊需要,因此相信回去不仅保险,还能捞一把。也许这个诱惑太大了,便有人想回去想得几乎发了疯。在伯力市郊的时候,有一次,一个充当打扫职责的伪满俘虏,大约是发羊角疯之类的病,倒在地下胡说八道。有一位崇信\xpinyin*{乩}坛的伪大臣,认定这是大神附体,便立刻跪在这个俘虏面前大叩其头,并且嘴里还念念叨叨,恭请“大神”示知,他什么时候能离苏回家。\\

在苏联,除了苏联翻译人员经常给大家讲新闻,我们还可以经常看到旅顺苏军发行的中文《实话报》,听到国内的战事消息。我对这些很不关心,认为无论谁胜谁败对我反正是一样,都会要我的命。我唯一的希望就是永远不回国。那些伪大臣们却很留心国内的形势。他们把希望放在\xpinyin*{蒋介石}的统治上,他们相信,有美国的帮助,\xpinyin*{蒋介石}是可以打败人民解放军的,所以起初听到人民解放军的胜利消息,谁也不相信。到后来,事实越来越真,于是他们又发起慌来。新中国宣告成立时,有个自认为经验丰富的人,提出打个贺电的意见,这个意见得到了广泛的响应。
