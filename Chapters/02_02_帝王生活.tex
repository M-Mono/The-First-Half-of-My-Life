\fancyhead[LO]{{\scriptsize 1908-1917: 我的童年 · 帝王生活}} %奇數頁眉的左邊
\fancyhead[RO]{} %奇數頁眉的右邊
\fancyhead[LE]{} %偶數頁眉的左邊
\fancyhead[RE]{{\scriptsize 1908-1917: 我的童年 · 帝王生活}} %偶數頁眉的右邊
\chapter*{帝王生活}
\addcontentsline{toc}{chapter}{\hspace{1cm}帝王生活}
\thispagestyle{empty}
“优待条件”里所说的“暂居宫禁”,没规定具体期限。紫禁城里除了三大殿划归民国之外,其余地方全属“宫禁”范围。我在这块小天地里一直住到民国十三年被民国军驱逐的时候,渡过了人世间最荒谬的少年时代。其所以荒谬,就在于中华号称为民国,人类进入了二十世纪,而我仍然过着原封未动的帝王生活,呼吸着十九世纪遗下的灰尘。\\

每当回想起自己的童年,我脑子里便浮起一层黄色:琉璃瓦顶是黄的,轿子是黄的,椅垫子是黄的,衣服帽子的里面、腰上系的带子、吃饭喝茶的瓷制碗碟、包盖稀饭锅子的棉套、裹书的包袱皮、窗帘、马缰……无一不是黄的。这种独家占有的所谓明黄色,从小把惟我独尊的自我意识埋进了我的心底,给了我与众不同的“天性”。\\

我十一岁的那年,根据太妃们的决定,祖母和母亲开始进宫“会亲”,杰二弟和大妹也跟着进宫来陪我玩几天。他们第一次来的那天,开头非常无味。我和祖母坐在炕上,祖母看着我在炕桌上摆骨牌,二弟和大妹规规矩矩地站在地上,一动不动地瞅着,就像衙门里站班的一样。后来,我想起个办法,把弟弟和妹妹带到我住的养心殿,我就问\ruby{溥杰}{Pu Giye}:“你们在家里玩什么?”\\

“\ruby{溥杰}{Pu Giye}会玩捉迷藏。”小我一岁的二弟恭恭敬敬地说。\\

“你们也玩捉迷藏呀?那太好玩了!”我很高兴。我和太监们玩过,还没跟比我小的孩子玩过呢。于是我们就在养心殿玩起捉迷藏来。越玩越高兴,二弟和大妹也忘掉了拘束。后来我们索性把外面的帘子都放下来,把屋子弄得很暗。比我小两岁的大妹又乐又害怕,我和二弟就吓唬她,高兴得我们又笑又嚷。捉迷藏玩得累了,我们就爬到炕上来喘气,我又叫他们想个新鲜游戏。\ruby{溥杰}{Pu Giye}想了一阵,没说话,光瞅着我傻笑。\\

“你想什么?”\\

他还是便笑。\\

“说,说!”我着急地催促他,以为他一定想出新鲜的游戏了,谁知他说:\\

“我想的,噢,\ruby{溥杰}{Pu Giye}想的是,皇上一定很不一样,就像戏台上那样有老长的胡子,……”\\

说着,他抬手做了一个持胡子的动作。谁知这个动作给他惹了祸,因为我一眼看见他的袖口里的衣里,很像那个熟悉的颜色。我立刻沉下脸来:\\

“\ruby{溥杰}{Pu Giye},这是什么颜色,你也能使?”\\

“这,这这是杏黄的吧?”\\

“瞎说!这不是明黄吗?”\\

“嗻,嗻……”\ruby{溥杰}{Pu Giye}忙垂手立在一边。大妹溜到他身后,吓得快要哭出来了。我还没完:\\

“这是明黄!不该你使的!”\\

“嗻!”\\

在嗻嗻声中,我的兄弟又恢复了臣仆的身份。……\\

嗻嗻之声早已成了绝响。现在想起来,那调儿很使人发笑。但是我从小便习惯了它,如果别人不以这个声调回答我,反而是不能容忍的。对于跪地磕头,也是这样。我从小就看惯了人家给我磕头,大都是年岁比我大十几倍的,有清朝遗老,也有我亲族中的长辈,有穿清朝袍褂的,也有穿西式大礼服的民国官员。\\

见怪不怪习以为常的,还有每日的排场。\\

据说曾有一位青年,读《红楼梦》时大为惊奇,他不明白为什么在贾母、王凤姐这样人身后和周围总有那么一大群人,即使他们从这间屋走到隔壁那间屋去,也会有一窝蜂似的人跟在后面,好像一条尾巴似的。其实《红楼梦》里的尾巴比宫里的尾巴小多了。《红楼梦》里的排场犹如宫里的排场的缩影,这尾巴也颇相似。我每天到\xpinyin*{毓}庆宫读书、给太妃请安,或游御花园,后面都有一条尾巴。我每逢去游颐和园,不但要有几十辆汽车组成的尾巴,还要请民国的警察们沿途警戒,一次要花去几千块大洋。我到宫中的御花园去玩一次,也要组成这样的行列:最前面是一名敬事房的太监,他起的作用犹如汽车喇叭,嘴里不时地发出“吃——吃——”的响声,警告人们早早回避,在他们后面二三十步远是两名总管太监,靠路两侧,鸭行鹅步地行进;再后十步左右即行列的中心(我或太后)。如果是坐轿,两边各有一名御前小太监扶着轿杆随行,以便随时照料应呼;如果是步行,就由他们搀扶而行。在这后面,还有一名太监举着一把大罗伞,伞后几步,是一大群拿着各样物件和徒手的太监:有捧马扎以便随时休息的,有捧衣服以便随时换用的,有拿着雨伞旱伞的;在这些御前太监后面是御茶房太监,捧着装着各样点心茶食的若干食盒,当然还有热水壶、茶具等等;更后面是御药房的太监,挑着担子,内装各类常备小药和急救药,不可少的是灯心水、菊花水、芦根水、竹叶水、竹茹水,夏天必有蕾香正气丸、六合定中丸、金衣祛暑丸、香薷丸、万应锭、痧药、避瘟散,不分四季都要有消食的三仙饮,等等;在最后面,是带大小便器的太监。如果没坐轿,轿子就在最后面跟随。轿子按季节有暖轿凉轿之分。这个杂七夹八的好几十人的尾巴,走起来倒也肃静安详,井然有序。\\

然而这个尾巴也常被我搅乱。我年岁小的时候,也和一般的孩子一样,高兴起来撒腿便跑。起初他们还亦步亦趋地跟着跑,跑得丢盔曳甲,喘吁不止。我大些以后,懂得了发号施令,想跑的时候,叫他们站在一边等着,于是除了御前小太监以外,那些捧盒挑担的便到一边静立,等我跑够了再重新贴在我后边。后来我学会了骑自行车,下令把宫门的门槛一律锯掉,这样出入无阻地到处骑,尾巴自然更无法跟随,只好暂时免掉。除此以外,每天凡到太妃处请安和去\xpinyin*{毓}庆宫上学等等日常行动,仍然要有一定的尾巴跟随。假如那时身后没有那个尾巴,例会觉得不自然。我从前听人家讲明朝\xpinyin*{崇祯}皇帝的故事,听到最后,说\xpinyin*{崇祯}身边只剩下一个太监,我就觉着特别不是滋味。\\

耗费人力物力财力最大的排场,莫过于吃饭。关于皇帝吃饭,另有一套术语,是绝对不准别人说错的。饭不叫饭而叫“膳”,吃饭叫“进膳”,开饭叫“传膳”,厨房叫“御膳房”。到了吃饭的时间——并无固定时间,完全由皇帝自己决定——我吩咐一声“传膳!”跟前的御前小太监便照样向守在养心殿的明殿上的殿上太监说一声“传膳!”殿上太监又把这话传给鹄立在养心门外的太监,他再传给候在西长街的御膳房太监……这样一直传进了御膳房里面。不等回声消失,一个犹如过嫁妆的行列已经走出了御膳房。这是由几十名穿戴齐整的太监们组成的队伍,抬着大小七张膳桌,捧着几十个绘有金龙的朱漆盒,浩浩荡荡地直奔养心殿而来。进到明殿里,由套上白袖头的小太监接过,在东暖阁摆好。平日菜肴两桌,冬天另设一桌火锅,此外有各种点心、米膳、粥品三桌,咸菜一小桌。食具是绘着龙纹和写着“万寿无疆”字样的明黄色的瓷器,冬天则是银器,下托以盛有热水的瓷罐。每个菜碟或菜碗都有一个银牌,这是为了戒备下毒而设的,并且为了同样原因,菜送来之前都要经过一个太监尝过,叫做“尝膳”。在这些尝过的东西摆好之后,我人座之前,一个小太监叫了一声“打碗盖!”其余四五个小太监便动手把每个菜上的银盖取下,放到一个大盒子里拿走。于是我就开始“用膳”了。\\

所谓食前方丈都是些什么东西呢?\xpinyin*{隆裕}太后每餐的菜肴有百样左右,要用六张膳桌陈放,这是她从\xpinyin*{慈禧}那里继承下来的排场,我的比她少,按例也有三十种上下。我现在找到了一份“\xpinyin*{宣统}四年二月糙卷单”(即民国元年三月的一份菜单草稿),上面记载的一次“早膳”\footnote{宫中只吃两餐:“早膳即午饭。早晨或午后有时吃一顿点心。}的内容如下:\\

口蘑肥鸡、三鲜鸭子、五\xpinyin*{绺}鸡丝、炖肉、炖肚肺、肉片炖白菜、黄焖羊肉、羊肉炖菠菜豆腐、樱桃肉山药、炉肉炖白菜、羊肉片汆小萝卜、鸭条溜海参、鸭丁溜葛仙米、烧茨菇、肉片焖玉兰片、羊肉丝焖跑跶丝、炸春卷、黄韭菜炒肉、熏肘花小肚、卤煮豆腐、熏干丝、烹掐菜、花椒油炒白菜丝、五香干、祭神肉片汤、白煮塞勒烹白肉\\

这些菜肴经过种种手续摆上来之后,除了表示排场之外,并无任何用处。它之所以能够在一声传膳之下,迅速摆在桌子上,是因为御膳房早在半天或一天以前就已做好,根在火上等候着的。他们也知道,反正从\xpinyin*{光绪}起,皇帝并不靠这些早已过了火候的东西充饥。我每餐实际吃的是太后送的菜肴,太后死后由四位太妃接着送。因为太后或太妃们都有各自的膳房,而且用的都是高级厨师,做的菜肴味美可口,每餐总有二十来样。这是放在我面前的菜,御膳房做的都远远摆在一边,不过做个样子而已。\\

太妃们为了表示对我的疼爱和关心,除了每餐送菜之外,还规定在我每餐之后,要有一名领班太监去禀报一次我的进膳情况。这同样是公式文章。不管我吃了什么,领班太监到了太妃那里双膝跪倒,说的总是这一套:\\

“奴才禀老主子:万岁爷进了一碗老米膳(或者白米膳),一个馒头(或者一个烧饼)和一碗粥。进得香!”\\

每逢年节或太妃的生日(这叫做“千秋”),为了表示应有的孝顺,我的膳房也要做出一批菜肴送给太妃。这些菜肴可用这四句话给以鉴定:华而不实,费而不惠,营而不养,淡而无味。\\

这种吃法,一个月要花多少钱呢?我找到了一本《\xpinyin*{宣统}二年九月初一至三十日内外膳房及各等处每日分例肉斤鸡鸭清册》,那上面的记载如下:\\

\begin{quote}
	皇上前分例菜肉二十二斤计,三十日分例共六百六十斤\\

汤肉五斤:共一百五十斤\\

猪油一斤:共三十斤\\

肥鸡二只:共六十只\\

肥鸭三只:共九十只\\

蒸鸡三只:共九十只\\
\end{quote}

下面还有太后和几位妃的分例,为省目力,现在把它并成一个统计表(皆全月分例)如下:\\

\begin{center}
	\begin{tabular}{ l r r  r}
		& 肉& 鸡& 鸭\\
				&&&\\
		太后 &1860斤&   30只&  30只\\
		瑾贵妃&285斤&7只&   7只\\
		\xpinyin*{瑜}皇贵妃&360斤&15只&  15只\\
		\xpinyin*{珣}皇贵妃&360斤&15只&  15只\\
		\xpinyin*{瑨}贵妃&285斤&7只&   7只\\
				&&&\\
		合计& 3150斤& 74只& 74只\\
		&&&\\
	\end{tabular}
\end{center}

我这一家六口,总计一个月要用三千九百六十斤肉,三百八十八只鸡鸭,其中八百一十斤肉和二百四十只鸡鸭是我这五岁孩子用的。此外,宫中每天还有大批为这六口之家效劳的军机大臣、御前侍卫、师傅、翰林、画师、勾字匠、有身份的太监,以及每天来祭神的萨满等等,也各有分例。连我们六口之家共吃猪肉一万四千六百四十二斤,合计用银二千三百四十二两七钱二分。除此之外,每日还要添菜,添的比分例还要多得多。这个月添的肉是三万一千八百四十四斤,猪油八百十四斤,鸡鸭四千七百八十六只,连什么鱼虾蛋品,用银一万一千六百四十一两七分,加上杂费支出三百四十八两,连同分例一共是一万四千七百九十四两一钱九分。显而易见,这些银子除了贪污中饱之外,差不多全为了表示帝王之尊而糟蹋了。这还不算一年到头不断的点心、果品、糖食、饮料这些消耗。\\

饭菜是大量的做而不吃,衣服则是大量的做而不穿。这方面我记得的不多,只知道后妃有分例,皇帝却毫无限制,而且一年到头都在做衣服,做了些什么,我也不知道,反正总是穿新的。我手头有一份改用银元以后的报账单子,没有记明年代,题为“十月初六日至十一月初五日承做上用衣服用过物料复实价目”,据这个单子所载,这个月给我做了:皮袄十一件,皮袍褂六件,皮紧身二件,棉衣裤和紧身三十件。不算正式工料,仅贴边。兜布、子母扣和线这些小零碎,就开支了银元二千一百三十七元六角三分三厘五毫。\\

至于后妃们的分例,也是相当可观的。在我结婚后的一本账上,有后妃们每年使用衣料的定例,现在把它统计如下:\\

\begin{center}
	\begin{tabular}{ l r r rr}
		&  皇后 & 淑妃 &  四太妃&  合计\\
%		&   &  &  太妃&  \\
		&&&&\\
		各种缎 &29匹&    15匹&      92匹&    136匹\\
		各种绸 &40匹&    21匹&     108匹&    169匹\\
		各种纱 &16匹&    5匹&      60匹&     81匹\\
		各种绫 &8匹&    5匹&      28匹&     41匹\\
		各种布 &60匹&    30匹&     144匹&    234匹\\
		绒和线 & 16斤&     8斤&      76斤&    100斤\\
		棉花    & 40斤&    20斤&     120斤&    180斤\\
		金线    &   20\xpinyin*{绺}&    10\xpinyin*{绺}&      76\xpinyin*{绺}&    106\xpinyin*{绺}\\
		貂皮    &     90张&    30张&     280张&    400张\\
		&&&&\\
	\end{tabular}
\end{center}

我更换衣服,也有明文规定,由“四执事库”太监负责为我取换。单单一项平常穿的袍褂,一年要照单子更换二十八种,从正月十九的青白嵌皮袍褂,换到十一月初一的貂皮褂。至于节日大典,服饰之复杂就更不用说了。\\

既然有这些穷奢极侈的排场,就要有一套相应的机构和人马。给皇帝管家的是内务府,它统辖着广储、都虞、掌礼、会计。庆丰、慎刑、营造等七个司(每司各有一套库房、作坊等单位,如广储司有银、皮、瓷、缎、衣、茶等六个库)和宫内四十八个处。据\xpinyin*{宣统}元年秋季《爵秩全览》所载,内务府官员共计一千零二十三人(不算禁卫军、太监和苏拉\footnote{苏拉,执役人的满语称呼。清时内延苏拉隶属于太监。内务府、军机处皆有之。雍和官的执役喇嘛,称苏拉喇嘛。}),民国初年曾减到六百多人,到我离开那里,还有三百多人。机构之大,用人之多,一般人还可以想象,其差使之无聊,就不大为人所知了。举个例子说,四十八处之一的如意馆,是专伺候帝后妃们画画写字的,如果太后想画个什么东西,就由如意馆的人员先给她描出稿子,然后由她着色题词。写大字匾额则是由\xpinyin*{懋}勤殿的勾字匠描出稿,或南书房翰林代笔。什么太后御笔或御制之宝,在清代末季大都是这样产生的。\\

除了这些排场之外,周围的建筑和宫殿陈设也对我起着教育作用。黄琉璃瓦惟有帝王才能使用,这不用说了,建筑的高度也是帝王特有的,这让我从小就确认,不但地面上的一切,所谓“普天之下莫非王土”,就连头上的一块天空也不属于任何别人。每一件陈设品都是我的直观教材。据说\xpinyin*{乾隆}皇帝曾经这样规定过:宫中的一切物件,哪怕是一寸草都不准丢失。为了让这句话变成事实,他拿了几根草放在宫中的案几上,叫人每天检查一次,少一根都不行,这叫做“寸草为标”。我在宫里十几年间,这东西一直摆在养心殿里,是一个景泰蓝的小罐,里面盛着三十六根一寸长的干草棍。这堆小干草棍儿曾引起我对那位祖先的无限崇敬,也曾引起我对\xpinyin*{辛亥}革命的无限忿慨。但是我并没想到,\xpinyin*{乾隆}留下的干草棍虽然一根不曾短少,而\xpinyin*{乾隆}留下的长满青草的土地,被儿孙们送给“与国”的,却要以成千方里计。\\

帝王生活所造成的浪费,已无法准确统计。据内务府编的材料,《\xpinyin*{宣统}七年放过款项及近三年比较》记载:民国四年的开支竟达二百七十九万余两,以后民国八、九、十各年数字逐年缩减,最低数仍达一百八十九万余两。总之,在民国当局的纵容下,以我为首的一伙人,照旧摆着排场,按原来标准过着寄生生活,大量地耗费着人民的血汗。\\

宫里有些规矩,当初并非完全出于摆排场,比如菜肴里放银牌和尝膳制度,出门一次要兴师动众地布警戒,这本是为了防止暗害的。据说皇帝没有厕所,就因为有一代皇帝外出如厕遇上了刺客。但这些故事和那些排场给我的影响全是一样:使我从任何方面都确认自己是尊贵的,统治一切和占有一切的人上之人。
