\fancyhead[LO]{{\scriptsize 1917-1924: 北京的“小朝廷” · 紫禁城的末日}} %奇數頁眉的左邊
\fancyhead[RO]{} %奇數頁眉的右邊
\fancyhead[LE]{} %偶數頁眉的左邊
\fancyhead[RE]{{\scriptsize 1917-1924: 北京的“小朝廷” · 紫禁城的末日}} %偶數頁眉的右邊
\chapter*{紫禁城的末日}
\addcontentsline{toc}{chapter}{\hspace{1cm}紫禁城的末日}
\thispagestyle{empty}
这次整顿内务府宣告失败,并不能使我就此“停车”。车没有停,不过拐个弯儿。我自从上了车,就不断有人给我加油打气,或者指点路标方向。\\

遗老们向我密陈恢复“大计”,前面说的只不过是其中的一例。在我婚后,像那样想为我效力的人,到处都有。例如\xpinyin*{康有为}和他的徒弟徐勤、徐良两父子,打着“中华帝国宪政党”的招牌,在国内国外活动。他们的活动情况,继续地通过\xpinyin*{庄士敦}传到宫中。徐勤写来奏折吹牛说,这个党在海外拥有十万党员和五家报纸。在我出宫前两年,徐良曾到广西找军阀\xpinyin*{林俊廷}去活动复辟,他给\xpinyin*{庄士敦}来信说,广西的三派军人首领\xpinyin*{陆荣廷}、\xpinyin*{林俊廷}和沈鸿英“三人皆与我党同宗旨,他日有事必可相助对待反对党也”\footnote{民十三年我出宫后,接收清宫的清室善后委员会在养心殿搜出了康有力和徐良给\xpinyin*{庄士敦}的信共二封,连同\xpinyin*{金梁}的条陈和\xpinyin*{江亢虎}请觐见的信都发表了出来,但当时却没发表这一封,也没发表\xpinyin*{康有为}向\xpinyin*{吴佩孚}进行活动的往来信件。}。民国十三年春节后,\xpinyin*{康有为}给\xpinyin*{庄士敦}的信中说:“经年奔走,至除夕乃归,幸所至游说,皆能见听,亦由各方厌乱,人有同心。”据他说陕西、湖北、湖南、江苏。安徽、江西、贵州、云南全都说好了,或者到时一说就行。他最寄予希望的是\xpinyin*{吴佩孚},说“洛(指吴,吴当时在洛阳)忠于孟德(指\xpinyin*{曹锟}),然闻已重病,如一有它,则传电可以旋转”。又说湖北萧耀南说过“一电可来”的话,到他生日,“可一赏之”。现在看起来,\xpinyin*{康有为}信中说了不少梦话,后来更成了没有实效的招摇行径。但当时我和\xpinyin*{庄士敦}对他的话不仅没有怀疑,而且大为欢欣鼓舞,并按他的指点送寿礼、赏福寿字。我在他们指点之下,开始懂得为自己的“理想”去动用财富了。\\


同样的例子还有“慈善捐款”。这是由哪位师傅的指点,不记得了,但动机是很清楚的,因为我这时懂得了社会\xpinyin*{舆}论的价值。那时在北京报纸的社会版上,差不多天天都有“\xpinyin*{宣统}帝施助善款待领”的消息。我的“施助”活动大致有两种,一种是根据报纸登载的贫民消息,把款送请报社代发,另一种是派人直接送到贫户家里。无论哪一种做法,过一两天报上总有这样的新闻:“本报前登某某求助一事,荷清帝遣人送去X元……”既表彰了我,又宣传了“本报”的作用。为了后者,几乎无报不登吸引我注意的贫民消息,我也乐得让各种报纸都给我做宣传。以至有的报居然登出这样的文章来:\\

\begin{quote}
	时事小言 皇恩浩荡\\

皇恩浩荡,乃君主时恭维皇帝的一句普通话,不意改建民国后,又闻有皇恩浩荡之声浪也。今岁入冬以来,京师贫民日众,凡经本报披露者,皆得有清帝之助款,贫民取款时,无不口诉皇恩之浩荡也。即本报代为介绍,同人帮同忙碌,然尽报纸之天职,一方替贫民之呼吁,一方代清帝之布恩,同人等亦无不忻忻然而云皇恩浩荡也。或日清帝退位深官,坐拥巨款,既无若何消耗,只好救济贫民,此不足为奇也。惟民国之政客军阀无不坐拥巨款,且并不见有一救济慈善者,于此更可见\xpinyin*{宣统}帝之皇恩浩荡也\footnote{见民国十二年十二月十五日《平报》,作者:\xpinyin*{秋隐}。}。
\end{quote}

像这样的文章,对我的价值自然比十块八块的助款大得太多了。\\

我付出最大的一笔赈款,是对民国十二年九月发生的日本“震灾”。那次日本地震的损失惊动了世界,我想让全世界知道“\xpinyin*{宣统}帝”的“善心”,决定拿出一笔巨款助赈。我的陈师傅看的比我更远,他在称赞了“皇恩浩荡,天心仁慈”之后,告诉我说:“此举之影响,必不仅限于此。”后来因为现款困难,便送去了据估价在美金三十万元上下的古玩字画珍宝。日本\xpinyin*{芳泽}公使陪同日本国会代表团来向我致谢时,宫中出现的兴奋气氛,竟和外国使节来观大婚礼时相像。\\

在这个时期,我的生活更加荒唐,干了不少自相矛盾的事。比如我一面责怪内务府开支太大,一面又挥霍无度。我从外国画报上看到洋狗的照片,就叫内务府向国外买来,连同狗食也要由国外定购。狗生了病请兽医,比给人治病用的钱还多。北京警察学校有位姓钱的兽医,大概看准了我的性格,极力巴结,给我写了好几个关于养狗知识的奏折,于是得到了绿玉手串、金戒指、鼻烟壶等十件珍品的赏赐。我有时从报上看见什么新鲜玩艺,如四岁孩子能读《孟子》,某人发现一只异样的蜘蛛,就会叫进宫里看看,当然也要赏钱。我一下子喜欢上了石头子儿,便有人买了各式各样的石头子儿送来,我都给以巨额赏赐。\\

我一面叫内务府裁人,把各司处从七百人戴到三百人,“御膳房”的二百厨师减到三十七个人,另方面又叫他们添设做西餐的“番菜膳房”,这两处“膳房”每月要开支一千三百多元菜钱。\\

关于我的每年开支数目,据我婚前一年(即民国十年)内务府给我编造的那个被缩小了数字的材料,不算我的吃穿用度,不算内务府各处司的开销,只算内务府的“交进”和“奉旨”支出的“恩赏”等款,共计年支八十七万零五百九十七两。\\

这种昏天黑地的生活,一直到民国十三年十一月五日,\xpinyin*{冯玉祥}的国民军把我驱逐出紫禁城,才起了变化。\\

这年九月由朝阳之战开始的第二次直奉战争,\xpinyin*{吴佩孚}的直军起初尚处于优势,十月间,吴部正向山海关的\xpinyin*{张作霖}的奉军发动总攻之际,吴部的\xpinyin*{冯玉祥}突然倒戈回师北京,发出和平通电。在冯、张合作之下,\xpinyin*{吴佩孚}的山海关前线军队一败涂地,\xpinyin*{吴佩孚}自己逃回洛阳。后来吴在河南没站住脚,又带着残兵败将逃到岳州,直到两年后和\xpinyin*{孙传芳}联合,才又回来,不过这已是后话。吴军在山海关败绩消息还未到,占领北京的\xpinyin*{冯玉祥}国民军已经把贿选总统\xpinyin*{曹锟}软禁了起来,接着解散了“猪仔国会”,\xpinyin*{颜惠庆}的内阁宣告辞职,国民军支持\xpinyin*{黄郛}\footnote{\xpinyin*{黄郛}字\xpinyin*{膺白},浙江人,反动的投机政客,后来北伐战争时帮助\xpinyin*{蒋介石}策划反革命政变,成为国民党亲日派,也是新政学系首领之一。}组成了摄政内阁。\\

政变消息刚传到宫里来,我立刻觉出了情形不对。紫禁城的内城守卫队被国民军缴械,调出了北京城,国民军接替了他们的营地,神武门换上了国民军的岗哨。我在御花园里用望远镜观察景山,看见了那边上上下下都是和守卫队服装不同的士兵们。内务府派去了人,送去茶水吃食,国民军收下了,没有什么异样态度,但是紫禁城里的人谁也放不下心。我们都记得,\xpinyin*{张勋}复辟那次,\xpinyin*{冯玉祥}参加了“讨逆军”,如果不是\xpinyin*{段祺瑞}及时地把他调出北京城,他是要一直打进紫禁城里来的。\xpinyin*{段祺瑞}上台之后,\xpinyin*{冯玉祥}和一些别的将领曾通电要求把小朝廷赶出紫禁城。凭着这点经验,我们对这次政变和守卫队的改编有了不祥的预感。接着,听说监狱里的政治犯都放出来了,又听说什么“过激党”都出来活动了,\xpinyin*{庄士敦}和陈师傅他们给我的种种关于“过激”“恐怖”的教育——最主要的一条是说他们要杀掉每一个贵族——这时发生了作用。我把\xpinyin*{庄士敦}找来,请他到东交民巷给我打听消息,要他设法给我安排避难的地方。\\

王公们陷入惶惶不安,有些人已在东交民巷的“六国饭店”定了房间,但是一听说我要出城,却都认为目前尚无必要。他们的根据还是那一条:有各国公认的优待条件在,是不会发生什么事情的。\\

然而必须发生的事,终归是发生了。\\

那天上午,大约是九点多钟,我正在储秀宫和婉客吃着水果聊天,内务府大臣们突然踉踉跄跄地跑了进来。为首的\xpinyin*{绍英}手里拿着一件公文,气喘吁吁地说:\\

“皇上,皇上,……\xpinyin*{冯玉祥}派了军队来了!还有李鸿藻的后人李石曾,说民国要废止优待条件,拿来这个叫,叫签字,……”\\

我一下子跳了起来,刚咬了一口的苹果滚到地上去了。我夺过他手里的公文,看见上面写着:\\

\begin{quote}
	大总统指令\\

派鹿钟麟、\xpinyin*{张璧}交涉清室优待条件修正事宜,此令。\\

中华民国十三年十一月五日\\

国务院代行国务总理\xpinyin*{黄郛}……\\

修正清室优待条件\\

今因大清皇帝欲贯彻五族共和之精神,不愿违反民国之各种制度仍存于今日,特将清室优待条件修正如左:\\

第一条、大清\xpinyin*{宣统}帝即日起永远废除皇帝尊号,与中华民国国民在法律上享有同等一切之权利;\\

第二条、自本条件修正后,民国政府每年补助清室家用五十万元,并特支出二百万元开办北京贫民工厂,尽先收容旗籍贫民;\\

第三条、清室应按照原优待条件第三条,即日移出官禁,以后得自由选择住居,但民国政府仍负保护责任;\\

第四条、清室之宗庙陵寝永远奉祀,由民国酌设卫兵妥为保护;\\

第五条、清室私产归清室完全享有,民国政府当为特别保护,其一切公产应归民国政府所有。\\

\begin{flushright}
	中华民国十三年十一月五日
\end{flushright}
\end{quote}

老实说,这个新修正条件并没有我原先想象的那么可怕。但是\xpinyin*{绍英}说了一句话,立即让我跳了起来:“他们说限三小时内全部搬出去!”\\

“那怎么办?我的财产呢?太妃呢?”我急得直转,“打电话找庄师傅!”\\

“电话线断,断,断了!”\xpinyin*{荣源}回答说。\\

“去人找王爷来!我早说要出事的!偏不叫我出去!找王爷!找王爷!”\\

“出不去了,”宝熙说,“外面把上了人。不放人出去了!”\\

“给我交涉去!”\\

“嗻!”\\

这时\xpinyin*{端康}太妃刚刚去世不多天,官里只剩下\xpinyin*{敬懿}和\xpinyin*{荣惠}两个太妃,这两位老太太说什么也不肯走。\xpinyin*{绍英}拿这个作理由,去和鹿钟麟商量,结果允许延到下午三点。过了中午,经过交涉,父亲进了宫,朱、陈两师傅被放了进来,只有\xpinyin*{庄士敦}被挡在外面。\\

听说王爷进来了,我马上走出屋子去迎他,看见他走进了宫门口,我立即叫道:\\

“王爷,这怎么办哪?”\\

他听见我的叫声,像挨了定身法似的,粘在那里了,既不走近前来,也不回答我的问题,嘴唇哆嗦了好半天,才进出一句没用的话:\\

“听,听旨意,听旨意……”\\

我又急又气,一扭身自己进了屋子。后来据太监告诉我,他听说我在修正条件上签了字,立刻把自己头上的花翎一把揪下来,连帽子一起摔在地上,嘴里嘟囔着说:“完了!完了!这个也甭要了!”\\

我回到屋里,过了不大功夫,\xpinyin*{绍英}回来了,脸色比刚才更加难看,哆哆嗦嗦地说:“鹿钟麟催啦,说,说再限二十分钟,不然的话,不然的话……景山上就要开炮啦……”\\

其实鹿钟麟只带了二十名手枪队,可是他这句吓唬人的话非常生效。首先是我岳父\xpinyin*{荣源}吓得跑到御花园,东钻西藏,找了个躲炮弹的地方,再也不肯出来。我看见王公大臣都吓成这副模样,只好赶快答应鹿的要求,决定先到我父亲的家里去。\\

这时国民军已给我准备好汽车,一共五辆,鹿钟麟坐头辆,我坐了第二辆,\xpinyin*{婉容}和\xpinyin*{文绣}、\xpinyin*{张璧}、\xpinyin*{绍英}等人依次上了后面的车。\\

车到北府门口,我下车的时候,鹿钟麟走了过来,这时我才和他见了面。鹿和我握了手,问我:\\

“\xpinyin*{溥仪}先生,你今后是还打算做皇帝,还是要当个平民?”\\

“我愿意从今天起就当个平民。”\\

“好!”鹿钟麟笑了,说:“那么我就保护你。”又说,现在既是中华民国,同时又有个皇帝称号是不合理的,今后应该以公民的身分好好为国效力。\xpinyin*{张璧}还说:\\

“既是个公民,就有了选举权和被选举权,将来也可能被选做大总统呢!”\\

一听大总统三个字,我心里特别不自在。这时我早已懂得“韬光养晦”的意义了,便说:\\

“我本来早就想不要那个优待条件,这回把它废止了,正合我的意思,所以我完全赞成你们的话。当皇帝并不自由,现在我可得到自由了。”\\

这段话说完,周围的国民军士兵都鼓起掌来。\\

我最后的一句话也并非完全是假话。我确实厌恶王公大臣们对我的限制和阻碍。我要“自由”,我要自由地按我自己的想法去实现我的理想——重新坐在我失掉的“宝座”上。
