\fancyhead[LO]{{\scriptsize 1917-1924: 北京的“小朝廷” · 北洋元老}} %奇數頁眉的左邊
\fancyhead[RO]{} %奇數頁眉的右邊
\fancyhead[LE]{} %偶數頁眉的左邊
\fancyhead[RE]{{\scriptsize 1917-1924: 北京的“小朝廷” · 北洋元老}} %偶數頁眉的右邊
\chapter*{北洋元老}
\addcontentsline{toc}{chapter}{\hspace{1cm}北洋元老}
\thispagestyle{empty}
这个退位诏并没有发出去,当时公布的只有裹夹在大总统命令中的一个内务府的声明。\\

\begin{quote}
	大总统令\\

据内务部呈称:准清室内务府函称:本日内务府奉谕:前于\ruby{宣统}{\textcolor{PinYinColor}{\Man ᡤᡝᡥᡠᠩᡤᡝ ᠶᠣᠰᠣ}}三年十二月二十五日钦奉\xpinyin*{隆裕}皇太后\xpinyin*{懿旨},因全国人民倾心共和,特率皇帝将统治权公诸全国,定为民国共和,并议定优待皇室条件,永资遵守,等因;\\

六载以来,备极优待,本无私政之心,岂有食言之理。不意七月一号\xpinyin*{张勋}率领军队,入宫盘踞,矫发谕旨,擅更国体,违背先朝懿训。冲入深居官禁,莫可如何。此中情形,当为天下所共谅。著内务府咨请民国政府,宣布中外,一体闻知,等因。函知到部,理合据情转呈等情。此次\xpinyin*{张勋}叛国矫挟,肇乱天下,本共有见闻,兹据呈明咨达各情,合\xpinyin*{亟}明白布告,咸使闻知。\\

\begin{flushright}
	此令!\\

中华民国六年七月十七日\\

国务总理\xpinyin*{段祺瑞}
\end{flushright}
\end{quote}

由自认“临朝听政”的退位诏,一变为“\xpinyin*{张勋}盘踞,冲人莫可如何”的内务府声明,这是北洋系三位元老与紫禁城合作的结果。想出这个妙计的是\xpinyin*{徐世昌}太傅,而执行的则是\xpinyin*{冯国璋}总统和\xpinyin*{段祺瑞}总理。\\

紫禁城在这次复辟中的行为,被轻轻掩盖过去了。紫禁城从复辟败局既定那天所展开的新活动,不再为外界所注意了。\\

下面是醇亲王在这段时间中所记的日记(括弧内是我注的):\\

\begin{quote}
	二十日。上门。\xpinyin*{张绍轩}(\xpinyin*{勋})辞职,\xpinyin*{王士珍}代之。不久,\xpinyin*{徐菊人}(\xpinyin*{世昌})往见皇帝,告知外边情形。……\\

\xpinyin*{廿}一日。上门。现拟采用虚下渐停之法。回府。已有表示密电出发,以明态度云云。荫兄(\ruby{载泽}{\textcolor{PinYinColor}{\Man ᡯᠠᡞ ᠵᡝ}})来谈。\\

\xpinyin*{廿}二日。上门住宿。近日七弟屡来电话、信和及晤谈云云。\xpinyin*{张绍轩}来函强硬云云。\\

\xpinyin*{廿}三日。上门。回府。……闻冯(\xpinyin*{国璋})已于南京继任(代理大总统)云云。\xpinyin*{张绍轩}遣傅民杰来谒。六弟来函。……\\

\xpinyin*{廿}四日。由\xpinyin*{寅正}余起,南河沿张宅一带开战,枪炮互放,至未正余始止射击。\xpinyin*{张绍轩}已往使馆避居。\\

\xpinyin*{廿}五日。\xpinyin*{丙辰}。上门。始明白(这三个字是后加的)宣布取销五月十三日以后办法(指宣布退位)。\\

\xpinyin*{廿}八日。上门。差片代候徐太傅、段总理两处。\\

\xpinyin*{廿}九日。初伏。差人赠于徐太傅洗尘\xpinyin*{肴馔}。大雨。世相(续)来谈,据云已晤徐太傅,竭力维持关于优待条件。惟二十五日所宣布之件(指“退位诏”)须另缮改正,今日送交云。徐太傅差人来谒。申刻亲往访问徐太傅晤谈刻许。\\

六月初一日。\xpinyin*{壬戌}。朔。上门。偕\xpinyin*{诣}长春宫(\xpinyin*{敬懿}太妃)行千秋贺祝(这后面贴着大总统令,将内务府的卸复辟之责的公函布告周知)。\\

初四日。徐太傅来答拜,晤谈甚详,并代段总理致意阻\xpinyin*{舆}云。\\

十二日。小雨。民国于六月以来,关于应筹皇室经费及旗\xpinyin*{饷}仍如例拨给云云。\\

十四日。遣派皇室代表润贝勒往迎冯总统,甚妥洽。……\\

十五日。差人持片代候冯总统,并赠\xpinyin*{肴馔}。\\

十六日。上门。绍官保(\xpinyin*{绍英})来谈。……\\

十七日。上门。民国代表汤总\xpinyin*{长化龙}\xpinyin*{觐}见,答礼毕,仍旧例周旋之。……\\

十八日。亲往访徐太傅,晤谭甚详,尚无大碍。\\

\xpinyin*{廿}一日。上门。……收六弟自津寓今早所发来函,略同十八日所晤徐太傅之意,尚好尚好。……\\

\xpinyin*{廿}七日。七弟自津回京来谈。阅报民国竟于今日与德奥两国宣战了。由绍官保送来五月二十二之强硬函件,存以备考。\\

\xpinyin*{廿}九日。亲访世太傅致嘱托之意。\\

七月初一日。\xpinyin*{壬辰}。朔。上门偕见四官皇贵妃前云云。……接七弟电语,畅谈许久。\\

初四日。七弟来谈,已见冯总统,意思尚好。……\\
\end{quote}

紫禁城用金蝉脱壳之计躲开了社会上的视线,紫禁城外的那些失败者则成了揭露和抨击的目标。我从报上的文章和师傅们的议论中,很快地得到了互相印证的消息,明白了这次复辟的内情真相。\\

复辟的酝酿,早发生在洪宪帝制失败的时候。当时,\xpinyin*{袁世凯}的北洋系陷于四面楚歌,一度出任国务卿后又因反对\xpinyin*{袁世凯}“\xpinyin*{僭}越”称帝而引退的\xpinyin*{徐世昌},曾经用密电和\xpinyin*{张勋}、\xpinyin*{倪嗣}冲商议过,说“民党煎追至此,不如以大政归还清室,项城仍居总理大臣之职,领握军权”。这个主意得到早有此心的张、倪二人的同意,但因后来没有得到各国公使方面的支持,所以未敢行动。袁死后,他们又继续活动,在徐州、南京先后召开了北洋系军人首脑会议。并在袁的\xpinyin*{舆}棕移到彰德时,乘北洋系的首脑、督军们齐往致祭的机会,在\xpinyin*{徐世昌}的主持下,做出了一致同意复辟的决议。\\

取得一致意见之后,复辟的活动便分成了两个中心。一个是徐州的\xpinyin*{张勋},另一个是天津的\xpinyin*{徐世昌}。\xpinyin*{张勋}由彰德回到徐州,把督军们邀集在一起开会(即所谓第二次徐州会议),决议先找外国人支持,首先是日本的支持。张通过天津的\xpinyin*{朱家宝}(直隶省长)和天津日本驻屯军的一个少将发生了接触,得到赞助后,又通过日本少将的关系,和活动在满蒙的\xpinyin*{善耆}、蒙古匪首\ruby{巴布扎布}{\textcolor{PinYinColor}{\Meng ᠪᠠᠪᠦᠽᠠᠪ}},徐蚌的张、倪,天津的\xpinyin*{雷震春}、\xpinyin*{朱家宝}等联络上,共同约定,\xpinyin*{俟}\ruby{巴布扎布}{\textcolor{PinYinColor}{\Meng ᠪᠠᠪᠦᠽᠠᠪ}}的军队打到张家口,\xpinyin*{雷震春}即策动张家口方面响应,张、倪更借口防卫京师发兵北上,如此便一举而成复辟之“大业”。这个计划后来因为\ruby{巴布扎布}{\textcolor{PinYinColor}{\Meng ᠪᠠᠪᠦᠽᠠᠪ}}的军队被奉军抵住,以\ruby{巴布扎布}{\textcolor{PinYinColor}{\Meng ᠪᠠᠪᠦᠽᠠᠪ}}被部下刺杀而流于失败。\xpinyin*{徐世昌}回到天津后,他派了\xpinyin*{陆宗舆}东渡日本,试探日本政界的态度。日本当时的内阁与军部意见并不完全一致,内阁对天津驻屯军少将的活动,不表示兴趣。\xpinyin*{陆宗舆}的失败,曾引起津沪两地遗老普遍的埋怨,怪\xpinyin*{徐世昌}用人失当。\xpinyin*{陆宗舆}不但外交无功,内交弄得也很糟。他东渡之前先到徐州访问了\xpinyin*{张勋},把\xpinyin*{徐世昌}和日方协商的条件拿给\xpinyin*{张勋}看,想先取得张的首肯。张对于徐答应日本方面的条件倒不觉得怎样,唯有\xpinyin*{徐世昌}要日方谅解和支持他当议政王这一条,把\xpinyin*{张勋}惹恼了。他对陆说:“原来复辟只为成全徐某?难道我张某就不配做这个议政王吗?”从此张徐之间有了猜忌,两个复辟中心的活动开始分道扬镰。\\

不久,协约国拉段内阁参加已打了三年的欧战。\xpinyin*{徐世昌}看出是一步好棋,认为以参战换得协约国的支持,大可巩固北洋系的地位,便怂恿\xpinyin*{段祺瑞}去进行。段一心想武力统一全国,参战即可换得日本贷款,以充其内战经费,于是提交国会讨论。但国会中多数反对参战,这时想夺取实权的\xpinyin*{黎元洪}总统乃和国会联合起来反对\xpinyin*{段祺瑞}。所谓府院之争逐步发展到白热化,结果,国务总理被免职,跑到天津。段到天津暗地策动北洋系的督军,向\xpinyin*{黎元洪}的中央闹独立,要求解散国会,同时发兵威胁京师。\xpinyin*{张勋}看到这是个好机会,加之在第四次徐州会议上又取得了各省督军和北洋系冯、段代表的一致支持,认为自己确实做了督军们的盟主和复辟的领袖,于是骗得\xpinyin*{黎元洪}把他认做和事老,请他到北京担任调解。当年的六月下旬,他率领军队北上,在天津先和北洋系的首领们接触后,再迫\xpinyin*{黎元洪}以解散国会为条件,然后进京,七月一日就演出了复辟那一幕。\\

许多报纸分析\xpinyin*{张勋}的失败,是由于独揽大权,犯了两大错误,造成了自己的孤立。一个错误是只给了\xpinyin*{徐世昌}一个弼德院长的空街头,这就注定了败局;另一个是他不该忽略了既有野心又拥有“研究系”谋士的\xpinyin*{段祺瑞}。早在徐州开会时,冯、段都有代表附议过复辟计划,\xpinyin*{张勋}后来入京过津见过段,段也没表示过任何不赞成的意思,因此他心里认为北洋系的元老徐、冯、段已无问题,只差一个\xpinyin*{王士珍}态度不明。最后在北京他把\xpinyin*{王士珍}也拉到了手,即认为任何问题都没有了。不料他刚发动了复辟,天津的\xpinyin*{段祺瑞}就在马厂誓师讨逆,各地的督军们也变了卦,由拥护复辟一变而为“保卫共和”。这一场复辟结果成全了\xpinyin*{段祺瑞}和\xpinyin*{冯国璋},一个重新当上了国务总理,一个当上了总统,而\xpinyin*{张勋}则成了元凶大憝。\\

\xpinyin*{张勋}为此曾经气得暴跳如雷。他警告\xpinyin*{段祺瑞}和那些督军们说:“你们不要逼人太甚,把一切都推到我一个人身上,必要时我会把有关的信电和会议纪录公布出来的。”\footnote{据说\xpinyin*{张勋}原来保存了一整箱子关于这方面的文件,可是后来竟不知被什么人偷去,并且运往法国去了。}我父亲日记里说的“来函强硬”就是指这件事。\xpinyin*{张勋}这一手很有效。冯、段知道\xpinyin*{张勋}这句危词的份量,因此也就没敢逼他。冯、段政府公布命令为清室开脱的那天,同时发布过一项通缉\xpinyin*{康有为}、\xpinyin*{万绳栻}等五名复辟犯的命令。但被讨逆军\xpinyin*{冯玉祥}部队捕获的复辟要犯\xpinyin*{张镇芳}。\xpinyin*{雷震春}等人,立刻被\xpinyin*{段祺瑞}要了去,随即释放。过了半年,总统明令宣布免除对一切帝制犯(从洪宪到\xpinyin*{丁巳}复辟)的追究,虽然把\xpinyin*{张勋}除外,但实际上他已经自由自在地走出了荷兰使馆,住在新买的漂亮公馆里。第二年,\xpinyin*{徐世昌}就任总统后不到两个星期,更明令对\xpinyin*{张勋}免予追究,后来\xpinyin*{张勋}被委为林垦督办,他还嫌官小不干呢。\\

这些内幕新闻最引起我注意的,是民国的大人物,特别是当权的北洋系的元老们,都曾经是热心于复辟的人。这次他们都把\xpinyin*{张勋}当做靶子来打,对我却无一不是尽力维护的。\\

\xpinyin*{段祺瑞}在讨逆的电报里说:“该逆\xpinyin*{张勋},忽集其凶党,勒召都中军警长官三十余人,列戟会议,复叱咤命令,迫众雷同。旋即挈\xpinyin*{康有为}闯入宫禁,强为推戴,世中堂续叩头力争,血流灭鼻,谨、\xpinyin*{瑜}两太妃痛哭求免,几不欲生,清帝于身冲龄,岂能御此强暴?竟遭诬胁,实可哀怜!”\xpinyin*{冯国璋}在通电里也说:\xpinyin*{张勋}“玩冲人于股掌,遗清室以至危”,又说:“国璋在前清时代,本非主张革命之人,遇\xpinyin*{辛亥}事起,大势所趋,造成民国”。他们为什么这样为紫禁城开脱呢?又何以情不自禁地抒发了自己的感情呢?我得到的惟一结论是:这些人并非真正反对复辟,问题不过是由谁来带头罢了。\\

在紫禁城看来,只要能捉老鼠,花猫白猫全是好猫,无论姓张姓段,只要能把复辟办成,全是好人。\\

所以在冯、段上台之后,孤臣孽子们的目光曾一度集中到这两位新的当权者身上。在\xpinyin*{张勋}的内阁中当过阁丞的\xpinyin*{胡嗣瑗},曾做过\xpinyin*{冯国璋}的幕府,在了巨复辟中是他一度说动了冯的,现在又活动\xpinyin*{冯国璋}去了。后来\xpinyin*{段祺瑞}也和\xpinyin*{世续}有过接洽。但在冯、段这一年任期中,事情都没有结果。因为冯、段上台之后闹了一年摩擦,北洋系由此开始分裂为直系(冯)和皖系(段)。在忙于摩擦中,冯没有给\xpinyin*{胡嗣瑗}什么答复就下了台。段虽然也找过\xpinyin*{世续},透露出复辟也无不可的意思,但经过了\xpinyin*{巳}事件变得更加谨慎的\xpinyin*{世续},摸不透这位靠讨伐复辟而上台的总理是什么意思,所以没敢接过话头。\\

冯下台后,\xpinyin*{徐世昌}出任总统,情形就不同了。在复辟刚失败之后,《上海新闻报》有篇评论文章,其中有一段是最能打动紫禁城里的人心的:\\

使徐东海为之,决不卤莽如是,故此次复辟而不出于\xpinyin*{张勋},则北洋诸帅早已俯首称臣……\\

不但我这个刚过了几天皇帝瘾的人为之动心,就是紫禁城内外的孤臣孽子们也普遍有此想法,至少在\xpinyin*{徐世昌}上任初期是如此。\\

有位六十多岁的满族老北京人和我说:“民国七年,\xpinyin*{徐世昌}一当上了大总统,北京街上的旗人的大马车、两把头又多起来了。贵族家里又大张旗鼓地做寺、唱戏、摆宴,热闹起来了。并办起了什么‘贵族票友团’、什么‘俱乐部’……”\\

有位汉族的老先生说:“民国以来北京街上一共有三次‘跑祖宗’\footnote{意思是穿着清朝袍褂的人在马路上出现,这种服装当时是只有从祖宗画像上才看得到的。},一次是\xpinyin*{隆裕}死后那些天,一次是\xpinyin*{张勋}复辟那几天,最后一次是从\xpinyin*{徐世昌}当大总统起,一直到‘大婚’。最后这次算闹到了顶点……”\\

\xpinyin*{徐世昌}是\xpinyin*{袁世凯}发迹前的好友,发迹后的“军师”。\xpinyin*{袁世凯}一生中的重大举动,几乎没有一件不是与这位军师合计的。据说袁逼劝\xpinyin*{隆裕}“逊国”之前,他和军师邀集了冯、段等人一起商议过,认为对民军只可智取不可力敌,先答应民军条件,建立共和,等离间了民军,再让“辞位”的皇帝复位。后来\xpinyin*{袁世凯}自己称帝,\xpinyin*{徐世昌}颇为不满。我的一位亲戚听\xpinyin*{徐世昌}一个外甥说过,“洪宪”撤销的那天他在徐家,恰好\xpinyin*{袁世凯}来找徐。袁进了客厅,他被堵在里边的烟室里没敢出来。从断断续续的谈话里,他听见\xpinyin*{徐世昌}在劝说\xpinyin*{袁世凯}“仍旧维持原议”,\xpinyin*{袁世凯}最后怎样说的他没有听清。后来的事实说明,\xpinyin*{袁世凯}没有照他的意见办,或者想办而没来得及办就死了,\xpinyin*{徐世昌}自己从来没有放弃过复辟的念头,这几乎是当时人所共知的事实。\\

民国七年九月,\xpinyin*{徐世昌}就任了大总统,要公开宣称他不能进占中南海,在正式总统府建成之前,他要在自己家里办公。他就任后立即赦免了\xpinyin*{张勋},提倡读经、尊孔,举行郊天典礼。根据他的安排,皇室王公有的(\ruby{毓}{\textcolor{PinYinColor}{Yū}}\ruby{朗}{\textcolor{PinYinColor}{Lang}})当上了议员,有的(\ruby{载}{\textcolor{PinYinColor}{Dzai}}\ruby{涛}{\textcolor{PinYinColor}{Tao}})被授为“将军”。他无论在人前人后都把前清称为“本朝”,把我称做“上边”\\

与此同时,紫禁城和徐太傅更进行着不可告人的活动。\xpinyin*{冯国璋}任总统时,内务府大臣\xpinyin*{世续}让\xpinyin*{徐世昌}拿走了票面总额值三百六十万元的优字爱国公债券(这是\xpinyin*{袁世凯}当总理大臣时,要去了\xpinyin*{隆裕}太后全部内\xpinyin*{帑}之后交内务府的,据内务府的人估计,实际数目比票面还要多)。\xpinyin*{徐世昌}能当上总统,这笔活动费起了一定作用。徐当选总统已成定局的时候,由内务府三位现任大臣\xpinyin*{世续}、\xpinyin*{绍英}、\ruby{耆龄}{\textcolor{PinYinColor}{Ci Ling}}作主,两位前任大臣\xpinyin*{增崇}、\xpinyin*{继禄}作陪,宴请了\xpinyin*{徐世昌},在什刹海水滨的会贤堂饭庄楼上,酒过三巡,\xpinyin*{世续}问道:“大哥这次出山,有何抱负?”徐太傅慨然道:“慰亭(\xpinyin*{袁世凯})先不该错过\xpinyin*{癸丑}年的时机(指民国二年袁扑灭“二次革命”),后不该闹什么洪宪。\xpinyin*{张绍轩}在\xpinyin*{丁巳}又太鲁莽灭裂,不得人心。”然后举杯,谦逊地说:“咱们这次出来,不过为幼主摄政而已。”后来\xpinyin*{徐世昌}送了\xpinyin*{世续}一副对联:“捧日立身超世界,拨云屈指数山川。上联是恭维\xpinyin*{世续};下联则是自况其“拨云见日”之志。\\

这些千真万确的故事,当时我身边的人并不肯直接告诉我。我只知道人们一提起徐太傅,总要流露出很有希望的神情。我记得从徐上台起,紫禁城又门庭若市,紫禁城里的谥法、朝马似乎又增了行情,各地真假遗老一时趋之若鹜。至于和\xpinyin*{徐世昌}的来往进展,师傅们则一概语焉不详。有一回,\xpinyin*{陈宝琛}在发议论中间,以鄙夷的神色说:“\xpinyin*{徐世昌}还想当议政王,未免过分。一个‘公’也就够了。”又有一次说:“当初主张以汉大臣之女为皇后,是何居心?其实以清太傅而出仕民国,早已可见其人!”\\

从\xpinyin*{陈宝琛}说了这些话后,紫禁城里再提起\xpinyin*{徐世昌},就没有过去的那股热情了。其实,\xpinyin*{徐世昌}上台一年后,他自己的情形就很不如意。自从北洋系分裂为直系皖系后,徐已不能凭其北洋元老资格驾驭各方,何况从他一上台,\xpinyin*{段祺瑞}就和他摩擦,次年又发生震动全国的“五四”学生运动,更使他们自顾不暇。徐太傅即使复辟心愿有多么高,对清室的忠顺多么让陈师傅满意,他也是无能为力的了。\\

尽管徐太傅那里的消息沉寂下去了,然而紫禁城里的小朝廷对前途并没有绝望……
