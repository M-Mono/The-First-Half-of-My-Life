\fancyhead[LO]{{\scriptsize 1917-1924: 北京的“小朝廷” · 北洋元老}} %奇數頁眉的左邊
\fancyhead[RO]{} %奇數頁眉的右邊
\fancyhead[LE]{} %偶數頁眉的左邊
\fancyhead[RE]{{\scriptsize 1917-1924: 北京的“小朝廷” · 北洋元老}} %偶數頁眉的右邊
\chapter*{北洋元老}
\addcontentsline{toc}{chapter}{\hspace{1cm}北洋元老}
\thispagestyle{empty}
这个退位诏并没有发出去,当时公布的只有裹夹在大总统命令中的一个内务府的声明。\\

\begin{quote}
	大总统令\\

据内务部呈称:准清室内务府函称:本日内务府奉谕:前于宣统三年十二月二十五日钦奉隆裕皇太后\xpinyin*{懿旨},因全国人民倾心共和,特率皇帝将统治权公诸全国,定为民国共和,并议定优待皇室条件,永资遵守,等因;\\

六载以来,备极优待,本无私政之心,岂有食言之理。不意七月一号张勋率领军队,入宫盘踞,矫发谕旨,擅更国体,违背先朝懿训。冲入深居官禁,莫可如何。此中情形,当为天下所共谅。著内务府咨请民国政府,宣布中外,一体闻知,等因。函知到部,理合据情转呈等情。此次张勋叛国矫挟,肇乱天下,本共有见闻,兹据呈明咨达各情,合\xpinyin*{亟}明白布告,咸使闻知。\\

\begin{flushright}
	此令!\\

中华民国六年七月十七日\\

国务总理段祺瑞
\end{flushright}
\end{quote}

由自认“临朝听政”的退位诏,一变为“张勋盘踞,冲人莫可如何”的内务府声明,这是北洋系三位元老与紫禁城合作的结果。想出这个妙计的是徐世昌太傅,而执行的则是冯国璋总统和段祺瑞总理。\\

紫禁城在这次复辟中的行为,被轻轻掩盖过去了。紫禁城从复辟败局既定那天所展开的新活动,不再为外界所注意了。\\

下面是\xpinyin*{醇亲王}在这段时间中所记的日记(括弧内是我注的):\\

\begin{quote}
	二十日。上门。张绍轩(勋)辞职,王士珍代之。不久,徐菊人(世昌)往见皇帝,告知外边情形。……\\

廿一日。上门。现拟采用虚下渐停之法。回府。已有表示密电出发,以明态度云云。荫兄(\ruby{载泽}{zǎi zé})来谈。\\

廿二日。上门住宿。近日七弟屡来电话、信和及晤谈云云。张绍轩来函强硬云云。\\

廿三日。上门。回府。……闻冯(国璋)已于南京继任(代理大总统)云云。张绍轩遣傅民杰来谒。六弟来函。……\\

廿四日。由寅正余起,南河沿张宅一带开战,枪炮互放,至未正余始止射击。张绍轩已往使馆避居。\\

廿五日。丙辰。上门。始明白(这三个字是后加的)宣布取销五月十三日以后办法(指宣布退位)。\\

廿八日。上门。差片代候徐太傅、段总理两处。\\

廿九日。初伏。差人赠于徐大傅洗尘\xpinyin*{肴馔}。大雨。世相(续)来谈,据云已晤徐太傅,竭力维持关于优待条件。惟二十五日所宣布之件(指“退位诏”)须另缮改正,今日送交云。徐太傅差人来谒。申刻亲往访问徐大傅晤谈刻许。\\

六月初一日。壬戌。朔。上门。偕\xpinyin*{诣}长春宫(敬懿太妃)行千秋贺祝(这后面贴着大总统令,将内务府的卸复辟之责的公函布告周知)。\\

初四日。徐太傅来答拜,晤谈甚详,并代段总理致意阻\xpinyin*{舆}云。\\

十二日。小雨。民国于六月以来,关于应筹皇室经费及旗\xpinyin*{饷}仍如例拨给云云。\\

十四日。遣派皇室代表润贝勒往迎冯总统,甚妥洽。……\\

十五日。差人持片代候冯总统,并赠\xpinyin*{肴馔}。\\

十六日。上门。绍宫保(英)来谈。……\\

十七日。上门。民国代表汤总长化龙觐见,答礼毕,仍旧例周旋之。……\\

十八日。亲往访徐太傅,晤谭甚详,尚无大碍。\\

廿一日。上门。……收六弟自津寓今早所发来函,\\

略同十八日所晤徐太傅之意,尚好尚好。……\\

廿七日。七弟自津回京来谈。阅报民国竟于今日与德奥两国宣战了。由绍官保送来五月二十二之强硬函件,存以备考。\\

廿九日。亲访世太傅致嘱托之意。\\

七月初一日。壬辰。朔。上门偕见四官皇贵妃前云云。……接七弟电语,畅谈许久。\\

初四日。七弟来谈,已见冯总统,意思尚好。……\\
\end{quote}

紫禁城用金蝉脱壳之计躲开了社会上的视线,紫禁城外的那些失败者则成了揭露和抨击的目标。我从报上的文章和师傅们的议论中,很快地得到了互相印证的消息,明白了这次复辟的内情真相。\\

复辟的酝酿,早发生在洪宪帝制失败的时候。当时,袁世凯的北洋系陷于四面楚歌,一度出任国务卿后又因反对袁世凯“僭越”称帝而引退的徐世昌,曾经用密电和张勋、倪嗣冲商议过,说“民党煎追至此,不如以大政归还清室,项城仍居总理大臣之职,领握军权”。这个主意得到早有此心的张、倪二人的同意,但因后来没有得到各国公使方面的支持,所以未敢行动。袁死后,他们又继续活动,在徐州、南京先后召开了北洋系军人首脑会议。并在袁的\xpinyin*{舆}棕移到彰德时,乘北洋系的首脑、督军们齐往致祭的机会,在徐世昌的主持下,做出了一致同意复辟的决议。\\

取得一致意见之后,复辟的活动便分成了两个中心。一个是徐州的张勋,另一个是天津的徐世昌。张勋由彰德回到徐州,把督军们邀集在一起开会(即所谓第二次徐州会议),决议先找外国人支持,首先是日本的支持。张通过天津的朱家宝(直隶省长)和天津日本驻屯军的一个少将发生了接触,得到赞助后,又通过日本少将的关系,和活动在满蒙的善耆、蒙古匪首巴布扎布,徐蚌的张、倪,天津的雷震春、朱家宝等联络上,共同约定,\xpinyin*{俟}巴布扎布的军队打到张家口,雷震春即策动张家口方面响应,张、倪更借口防卫京师发兵北上,如此便一举而成复辟之“大业”。这个计划后来因为巴布扎布的军队被奉军抵住,以巴布扎布被部下刺杀而流于失败。徐世昌回到天津后,他派了陆宗\xpinyin*{舆}东渡日本,试探日本政界的态度。日本当时的内阁与军部意见并不完全一致,内阁对天津驻屯军少将的活动,不表示兴趣。陆宗\xpinyin*{舆}的失败,曾引起津沪两地遗老普遍的埋怨,怪徐世昌用人失当。陆宗\xpinyin*{舆}不但外交无功,内交弄得也很糟。他东渡之前先到徐州访问了张勋,把徐世昌和日方协商的条件拿给张勋看,想先取得张的首肯。张对于徐答应日本方面的条件倒不觉得怎样,唯有徐世昌要日方谅解和支持他当议政王这一条,把张勋惹恼了。他对陆说:“原来复辟只为成全徐某?难道我张某就不配做这个议政王吗?”从此张徐之间有了猜忌,两个复辟中心的活动开始分道扬镰。\\

不久,协约国拉段内阁参加已打了三年的欧战。徐世昌看出是一步好棋,认为以参战换得协约国的支持,大可巩固北洋系的地位,便怂恿段祺瑞去进行。段一心想武力统一全国,参战即可换得日本贷款,以充其内战经费,于是提交国会讨论。但国会中多数反对参战,这时想夺取实权的黎元洪总统乃和国会联合起来反对段祺瑞。所谓府院之争逐步发展到白热化,结果,国务总理被免职,跑到天津。段到天津暗地策动北洋系的督军,向黎元洪的中央闹独立,要求解散国会,同时发兵威胁京师。张勋看到这是个好机会,加之在第四次徐州会议上又取得了各省督军和北洋系冯、段代表的一致支持,认为自己确实做了督军们的盟主和复辟的领袖,于是骗得黎元洪把他认做和事老,请他到北京担任调解。当年的六月下旬,他率领军队北上,在天津先和北洋系的首领们接触后,再迫黎元洪以解散国会为条件,然后进京,七月一日就演出了复辟那一幕。\\

许多报纸分析张勋的失败,是由于独揽大权,犯了两大错误,造成了自己的孤立。一个错误是只给了徐世昌一个弼德院长的空街头,这就注定了败局;另一个是他不该忽略了既有野心又拥有“研究系”谋士的段祺瑞。早在徐州开会时,冯、段都有代表附议过复辟计划,张勋后来入京过津见过段,段也没表示过任何不赞成的意思,因此他心里认为北洋系的元老徐、冯、段已无问题,只差一个王士珍态度不明。最后在北京他把王士珍也拉到了手,即认为任何问题都没有了。不料他刚发动了复辟,天津的段祺瑞就在马厂誓师讨逆,各地的督军们也变了卦,由拥护复辟一变而为“保卫共和”。这一场复辟结果成全了段祺瑞和冯国璋,一个重新当上了国务总理,一个当上了总统,而张勋则成了元凶大憝。\\

张勋为此曾经气得暴跳如雷。他警告段祺瑞和那些督军们说:“你们不要逼人太甚,把一切都推到我一个人身上,必要时我会把有关的信电和会议纪录公布出来的。”\footnote{据说张勋原来保存了一整箱子关于这方面的文件,可是后来竟不知被什么人偷去,并且运往法国去了。}我父亲日记里说的“来函强硬”就是指这件事。张勋这一手很有效。冯、段知道张勋这句危词的份量,因此也就没敢逼他。冯、段政府公布命令为清室开脱的那天,同时发布过一项通缉康有为、万绳栻等五名复辟犯的命令。但被讨逆军冯玉祥部队捕获的复辟要犯张镇芳。雷震春等人,立刻被段祺瑞要了去,随即释放。过了半年,总统明令宣布免除对一切帝制犯(从洪宪到丁巳复辟)的追究,虽然把张勋除外,但实际上他已经自由自在地走出了荷兰使馆,住在新买的漂亮公馆里。第二年,徐世昌就任总统后不到两个星期,更明令对张勋免予追究,后来张勋被委为林垦督办,他还嫌官小不干呢。\\

这些内幕新闻最引起我注意的,是民国的大人物,特别是当权的北洋系的元老们,都曾经是热心于复辟的人。这次他们都把张勋当做靶子来打,对我却无一不是尽力维护的。\\

段祺瑞在讨逆的电报里说:“该逆张勋,忽集其凶党,勒召都中军警长官三十余人,列戟会议,复叱咤命令,迫众雷同。旋即挈康有为闯入宫禁,强为推戴,世中堂续叩头力争,血流灭鼻,谨、\xpinyin*{瑜}两太妃痛哭求免,几不欲生,清帝于身冲龄,岂能御此强暴?竟遭诬胁,实可哀怜!”冯国璋在通电里也说:张勋“玩冲人于股掌,遗清室以至危”,又说:“国璋在前清时代,本非主张革命之人,遇辛亥事起,大势所趋,造成民国”。他们为什么这样为紫禁城开脱呢?又何以情不自禁地抒发了自己的感情呢?我得到的惟一结论是:这些人并非真正反对复辟,问题不过是由谁来带头罢了。\\

在紫禁城看来,只要能捉老鼠,花猫白猫全是好猫,无论姓张姓段,只要能把复辟办成,全是好人。\\

所以在冯、段上台之后,孤臣孽子们的目光曾一度集中到这两位新的当权者身上。在张勋的内阁中当过阁丞的\xpinyin*{胡嗣瑗},曾做过冯国璋的幕府,在了巨复辟中是他一度说动了冯的,现在又活动冯国璋去了。后来段祺瑞也和世续有过接洽。但在冯、段这一年任期中,事情都没有结果。因为冯、段上台之后闹了一年摩擦,北洋系由此开始分裂为直系(冯)和皖系(段)。在忙于摩擦中,冯没有给\xpinyin*{胡嗣瑗}什么答复就下了台。段虽然也找过世续,透露出复辟也无不可的意思,但经过了巳事件变得更加谨慎的世续,摸不透这位靠讨伐复辟而上台的总理是什么意思,所以没敢接过话头。\\

冯下台后,徐世昌出任总统,情形就不同了。在复辟刚失败之后,《上海新闻报》有篇评论文章,其中有一段是最能打动紫禁城里的人心的:\\

使徐东海为之,决不卤莽如是,故此次复辟而不出于张勋,则北洋诸帅早已俯首称臣……\\

不但我这个刚过了几天皇帝瘾的人为之动心,就是紫禁城内外的孤臣孽子们也普遍有此想法,至少在徐世昌上任初期是如此。\\

有位六十多岁的满族老北京人和我说:“民国七年,徐世昌一当上了大总统,北京街上的旗人的大马车、两把头又多起来了。贵族家里又大张旗鼓地做寺、唱戏、摆宴,热闹起来了。并办起了什么‘贵族票友团’、什么‘俱乐部’……”\\

有位汉族的老先生说:“民国以来北京街上一共有三次‘跑祖宗’\footnote{意思是穿着清朝袍褂的人在马路上出现,这种服装当时是只有从祖宗画像上才看得到的。},一次是隆裕死后那些天,一次是张勋复辟那几天,最后一次是从徐世昌当大总统起,一直到‘大婚’。最后这次算闹到了顶点……”\\

徐世昌是袁世凯发迹前的好友,发迹后的“军师”。袁世凯一生中的重大举动,几乎没有一件不是与这位军师合计的。据说袁逼劝隆裕“逊国”之前,他和军师邀集了冯、段等人一起商议过,认为对民军只可智取不可力敌,先答应民军条件,建立共和,等离间了民军,再让“辞位”的皇帝复位。后来袁世凯自己称帝,徐世昌颇为不满。我的一位亲戚听徐世昌一个外甥说过,“洪宪”撤销的那天他在徐家,恰好袁世凯来找徐。袁进了客厅,他被堵在里边的烟室里没敢出来。从断断续续的谈话里,他听见徐世昌在劝说袁世凯“仍旧维持原议”,袁世凯最后怎样说的他没有听清。后来的事实说明,袁世凯没有照他的意见办,或者想办而没来得及办就死了,徐世昌自己从来没有放弃过复辟的念头,这几乎是当时人所共知的事实。\\

民国七年九月,徐世昌就任了大总统,要公开宣称他不能进占中南海,在正式总统府建成之前,他要在自己家里办公。他就任后立即赦免了张勋,提倡读经、尊孔,举行郊天典礼。根据他的安排,皇室王公有的(\xpinyin*{毓朗})当上了议员,有的(载涛)被授为“将军”。他无论在人前人后都把前清称为“本朝”,把我称做“上边”\\

与此同时,紫禁城和徐太傅更进行着不可告人的活动。冯国璋任总统时,内务府大臣世续让徐世昌拿走了票面总额值三百六十万元的优字爱国公债券(这是袁世凯当总理大臣时,要去了隆裕太后全部内帑之后交内务府的,据内务府的人估计,实际数目比票面还要多)。徐世昌能当上总统,这笔活动费起了一定作用。徐当选总统已成定局的时候,由内务府三位现任大臣世续、绍英、耆龄作主,两位前任大臣增崇、继禄作陪,宴请了徐世昌,在什刹海水滨的会贤堂饭庄楼上,酒过三巡,世续问道:“大哥这次出山,有何抱负?”徐太傅慨然道:“慰亭(袁世凯)先不该错过癸丑年的时机(指民国二年袁扑灭“二次革命”),后不该闹什么洪宪。张绍轩在丁巳又太鲁莽灭裂,不得人心。”然后举杯,谦逊地说:“咱们这次出来,不过为幼主摄政而已。”后来徐世昌送了世续一副对联:“捧日立身超世界,拨云屈指数山川。上联是恭维世续;下联则是自况其“拨云见日”之志。\\

这些千真万确的故事,当时我身边的人并不肯直接告诉我。我只知道人们一提起徐太傅,总要流露出很有希望的神情。我记得从徐上台起,紫禁城又门庭若市,紫禁城里的\xpinyin*{谥法}、朝马似乎又增了行情,各地真假遗老一时趋之若鹜。至于和徐世昌的来往进展,师傅们则一概语焉不详。有一回,\xpinyin*{陈宝琛}在发议论中间,以鄙夷的神色说:“徐世昌还想当议政王,未免过分。一个‘公’也就够了。”又有一次说:“当初主张以汉大臣之女为皇后,是何居心?其实以清太傅而出仕民国,早已可见其人!”\\

从\xpinyin*{陈宝琛}说了这些话后,紫禁城里再提起徐世昌,就没有过去的那股热情了。其实,徐世昌上台一年后,他自己的情形就很不如意。自从北洋系分裂为直系皖系后,徐已不能凭其北洋元老资格驾驭各方,何况从他一上台,段祺瑞就和他摩擦,次年又发生震动全国的“五四”学生运动,更使他们自顾不暇。徐太傅即使复辟心愿有多么高,对清室的忠顺多么让陈师傅满意,他也是无能为力的了。\\

尽管徐太傅那里的消息沉寂下去了,然而紫禁城里的小朝廷对前途并没有绝望……