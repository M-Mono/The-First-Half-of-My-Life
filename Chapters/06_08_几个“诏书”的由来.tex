\fancyhead[LO]{{\scriptsize 1932-1945: 伪满十四年 · 几个“诏书”的由来}} %奇數頁眉的左邊
\fancyhead[RO]{\thepage} %奇數頁眉的右邊
\fancyhead[LE]{\thepage} %偶數頁眉的左邊
\fancyhead[RE]{{\scriptsize 1932-1945: 伪满十四年 · 几个“诏书”的由来}} %偶數頁眉的右邊
\chapter*{几个“诏书”的由来}
\addcontentsline{toc}{chapter}{\hspace{11mm}几个“诏书”的由来}
%\thispagestyle{empty}
在伪满学校读过书的人,都被迫背过我的“诏书”。在学校、机关、军队里,每逢颁布一种诏书的日子,都要由主管人在集会上把那种诏书念一遍。听人讲,学校里的仪式是这样的:仪式进行时,穿“协和服”\footnote{协和服是伪满公教人员统一的制服,墨绿色,荐任官以上还有一根黄色的绳子套在颈间,称为“协和带”。学校里的校长和训育主任,一般都有这根所谓“协和带”。}的师生们在会场的高台前列队肃立,教职员在前,学生在后。戴着白手套的训育主任双手捧着一个黄布包,高举过顶,从房里出来。黄布包一出现,全场立即低下头。训育主任把它捧上台,放在桌上,打开包袱和里面的黄木匣,取出卷着的诏书,双手递给戴白手套的校长,校长双手接过,面向全体展开,然后宣读。如果这天是五月二日,就念一九三五年我第一次访日回来在这天颁布的“回銮训民诏书”(原无标点):\\

\begin{quote}
	朕自登极以来,\xpinyin*{亟}思躬访日本皇室,修睦联欢,以伸积慕。今次东渡,宿愿克遂。日本皇室,恳切相待,备极优隆,其臣民热诚迎送,亦无不殚竭礼敬。衷怀铭刻,殊不能忘。深维我国建立,以达今兹,皆赖友邦之仗义尽力,以奠\xpinyin*{丕}基。兹幸致诚佃,复加意观察,知其政本所立,在乎仁爱,教本所重,在乎忠孝;民心之尊君亲上,如天如地,莫不忠勇奉公,诚意为国,故能安内攘外,讲信恤邻,以维持万世一系之皇统。朕今躬接其上下,咸以至诚相结,气同道合,依赖不渝。朕与日本天皇陛下,精神如一体。尔众庶等,更当仰体此意,与友邦一心一德,以奠定两国永久之基础,发扬东方道德之真义。则大局和平,人类福祉,必可致也。凡我臣民,务遵朕旨,以垂万禩。钦此!\\
\end{quote}

诏书共有六种,即:\\

一九三四年三月一日的“即位诏书”;\\

一九三五年五月二日的“回銮训民诏书”;\\

一九四零年七月十五日的“国本奠定诏书”;\\

一九四一年十二月八日的“时局诏书”;\\

一九四二年三月一日的“建国十周年诏书”;\\

一九四五年八月十五日的“退位诏书”。\\

“即位诏书”后来为第五个即“建国十周年诏书”所代替。一九四五年八月十五日的“退位诏书”,那是没有人念的。所以主要的是四个诏书。学生、士兵都必须背诵如流,背不来或背错的要受一定惩罚。这不但是日本在东北进行奴化的宣传材料,也是用以镇压任何反抗的最高司法根据。东北老百姓如果流露出对殖民统治有一丝不满,都可能被借口违背诏书的某一句话而加以治罪。\\

从每一种诏书的由来上,可以看出一个人的灵魂如何在堕落。前两个我在前面说过了,现在说一下第三个,即“国本奠定诏书”是怎么出世的。\\

有一天,我在\xpinyin*{缉熙}楼和\ruby{吉冈}{\textcolor{PinYinColor}{よしおか}}呆坐着。他要谈的话早已谈完,仍赖在那里不走。我料想他必定还有什么事情要办。果然,他站起了身,走到摆佛像的地方站住了,鼻子发过了一阵嗯嗯之声后,回头向我说:\\

“佛,这是外国传进来的。嗯,外国宗教!日满精神如一体,信仰应该相同,哈?”\\

然后他向我解释说日本天皇是天照大神的神裔,每代天皇都是“现人神”,即大神的化身,日本人民凡是为天皇而死的,死后即成神。\\

我凭着经验,知道这又是关东军正在通过这条高压线送电。但是他说了这些,就没电了。我对他的这些神话,费了好几天功夫,也没思索出个结果来。\\

事实是,关东军又想出了一件事要叫我做,但由于关东军司令官\ruby{植田}{\textcolor{PinYinColor}{うえだ}}\ruby{谦吉}{\textcolor{PinYinColor}{けんきち}}正因发动的张鼓峰和诺门坎两次战事不利,弄得心神不宁,一时还来不及办。后来\ruby{植田}{\textcolor{PinYinColor}{うえだ}}指挥的这两次战役都失败了,终于被调回国卸职。临走,他大概想起了这件事,于是在辞行时向我做了进一步的表示:日满亲善,精神如一体,因此满洲国在宗教上也该与日本一致。他希望我把这件事考虑一下。\\

“太上皇”每次嘱咐我办的事,我都顺从地加以执行,惟有这一次,简直叫我啼笑皆非,不知所措。这时,\xpinyin*{胡嗣瑗}已经被挤走,\xpinyin*{陈曾寿}已经告退回家,\xpinyin*{万绳栻}已经病故,\xpinyin*{佟济煦}自护军出事以后胆小如鼠,其他的人则无法靠近我。被视为亲信并能见我的,只有几个妹夫和在“内延”念书的几个侄子。那时,在身边给我出谋献策的人没有了,那些年轻的妹夫和侄子们又没阅历,商量不出个名堂来,我无可奈何地独自把\ruby{植田}{\textcolor{PinYinColor}{うえだ}}的话想了几遍。还没想出个结果,新继任的司令官兼第五任大使\ruby{梅津}{\textcolor{PinYinColor}{うめづ}}\ruby{美治郎}{\textcolor{PinYinColor}{よしじろう}}来了。他通过\ruby{吉冈}{\textcolor{PinYinColor}{よしおか}}向我摊了牌,说日本的宗教就是满洲的宗教,我应当把日本皇族的祖先“天照大神”迎过来立为国教。又说,现在正值日本神武天皇纪元二千六百年大庆,是迎接大神的大好时机,我应该亲自去日本祝贺,同时把这件事办好。\\

后来我才听说,在日本军部里早就酝酿过此事,由于意见不一,未做出决定。据说,有些比较懂得中国人心理的日本人,如\ruby{本庄}{\textcolor{PinYinColor}{ほんじょう}}之流,曾认为这个举动可能在东北人民中间引起强烈的反感,导致日本更形孤立,故搁了下来。后来由于主谋者断定,只要经过一段时间,在下一代的思想中就会扎下根,在中年以上的人中间,也会习以为常,于是便做出了这个最不得人心的决定。他们都没有想到,这件事不但引起了东北人民更大的仇恨,就是在一般汉奸心里,也是很不受用的。以我自己来说,这件事就完全违背了我的“敬天法祖”思想,所以我的心情比发生“东陵事件”时更加难受。\\

我当了皇帝以后,曾因为祭拜祖陵的问题跟\ruby{吉冈}{\textcolor{PinYinColor}{よしおか}}发生过争执。登极即位祭祖拜陵,这在我是天经地义之事,但是\ruby{吉冈}{\textcolor{PinYinColor}{よしおか}}说,我不是清朝皇帝而是满蒙汉日朝五民族的皇帝,祭清朝祖陵将引起误会,这是不可以的。我说我是\ruby{爱新觉罗}{\textcolor{PinYinColor}{\Man ᠠᡳᠰᡳᠨ ᡤᡳᠣᡵᠣ}}的子孙,自然可以祭\ruby{爱新觉罗}{\textcolor{PinYinColor}{\Man ᠠᡳᠰᡳᠨ ᡤᡳᠣᡵᠣ}}的祖先陵墓。他说那可以派个\ruby{爱新觉罗}{\textcolor{PinYinColor}{\Man ᠠᡳᠰᡳᠨ ᡤᡳᠣᡵᠣ}}的其他子孙去办。争论结果,当然是我屈服,打消了北陵之行,然而我却一面派人去代祭,一面关上门在家里自己祭。现在事情竟然发展到不但祭不了祖宗,而且还要换个祖宗,我自然更加不好受了。\\

自从我在旅顺屈服于\ruby{板垣}{\textcolor{PinYinColor}{いたがき}}的压力以来,尽管我每一件举动都是对民族祖先的公开背叛,但那时我尚有自己的纲常伦理,还有一套自我宽解的哲学:我先是把自己的一切举动看做是恢复祖业、对祖宗尽责的孝行,以后又把种种屈服举动解释成“屈蠖求伸之计”,相信祖宗在天之灵必能谅解,且能暗中予以保佑。可是现在,日本人逼着我抛弃祖宗,调换祖宗,这是怎么也解释不过去的。\\

然而,一种潜于灵魂深处的真正属我所有的哲学,即以自己的利害为行为最高准则的思想提醒了我:如果想保证安全、保住性命,只得答应下来。当然,在这同时我又找到了自我宽解的办法,即私下保留祖先灵位,一面公开承认新祖宗,一面在家里祭祀原先的祖宗。因此,我向祖宗灵位预先告祭了一番,就动身去日本了。\\

这是我第二次访问日本,时间在一九四零年五月,呆了一共只有八天。\\

在会见\ruby{裕仁}{\textcolor{PinYinColor}{ひろひと}}的时候,我拿出了\ruby{吉冈}{\textcolor{PinYinColor}{よしおか}}\ruby{安直}{\textcolor{PinYinColor}{やすなお}}给我写好的台词,照着念了一遍,大意是:为了体现“日满一德一心、不可分割”的关系,我希望,迎接日本天照大神,到“满洲国”奉祀。他的答词简单得很,只有这一句:\\

“既然是陛下愿意如此,我只好从命!”\\

接着,\ruby{裕仁}{\textcolor{PinYinColor}{ひろひと}}站起来,指着桌子上的三样东西,即一把剑、一面铜镜和一块勾玉,所谓代表天照大神的三件神器,向我讲解了一遍。我心里想:听说在北京琉璃厂,这种玩艺很多,太监从紫禁城里偷出去的零碎,哪一件也比这个值钱,这就是神圣不可侵犯的大神吗?这就是祖宗吗?\\

在归途的车上,我突然忍不住哭了起来。\\

我回到长春之后,便在“帝宫”旁修起了一所用白木头筑的“建国神庙”,专门成立了“祭祀府”,由做过日本近卫师团长、关东军参谋长和宪兵司令官的\ruby{桥本}{\textcolor{PinYinColor}{はしもと}}\ruby{虎之助}{\textcolor{PinYinColor}{とらのすけ}}任祭祀府总裁,\xpinyin*{沈瑞麟}任副总裁。从此,就按关东军的规定,每逢初一、十五,由我带头,连同关东军司令和“满洲国”的官员们,前去祭祀一次。以后东北各地也都按照规定建起这种“神庙”,按时祭祀,并规定无论何人走过神庙,都要行九十度鞠躬礼,否则就按“不敬处罚法”加以惩治。由于人们都厌恶它,不肯向它行礼,因此凡是神庙所在,都成了门可罗雀的地方。据说有一个充当神庙的“神官”(即管祭祀的官员),因为行祭礼时要穿上一套特制的官服,样子十分难看,常常受到亲友们的耻笑,有一次他的妻子的女友对他妻子说:“你瞧你们当家的,穿上那身神官服,不是活像《小上坟》里的\xpinyin*{柳录景}吗?”这对夫妻羞愧难当,悄悄丢下了这份差事,跑到关内谋生去了。\\

关东军叫祭祀府也给我做了一套怪模怪样的祭祀服,我觉着穿着实在难看,便找到一个借口说,现值战争时期,理应穿戎服以示支援日本盟邦的决心,我还说穿军服可以戴上日本天皇赠的勋章,以表示“日满一德一心”。关东军听我说得振振有词,也没再勉强我。我每逢动身去神庙之前,先在家里对自己的祖宗磕一回头,到了神庙,面向天照大神的神\xpinyin*{龛}行礼时,心里念叨着:“我这不是给它行礼,这是对着北京坤宁宫行礼。”\\

我在全东北人民的耻笑、暗骂中,发布了那个定天照大神为祖宗和宗教的“国本奠定诏书”。这次不是\xpinyin*{郑孝胥}的手笔(\xpinyin*{郑孝胥}那时已死了两年),而是“国务院总务厅”嘱托一位叫\ruby{佐藤}{\textcolor{PinYinColor}{さとう}}\ruby{知恭}{\textcolor{PinYinColor}{ちきょう}}的日本汉学家的作品。其原文如下:\\

\begin{quote}
	朕兹为敬立\\

建国神庙,以奠国本于悠久,张国纲于无疆,诏尔众庶曰:我国自建国以来,邦基益固,邦运益兴,\xpinyin*{烝烝}日跻隆治。仰厥渊源,念斯\xpinyin*{丕}绩,莫不皆赖天照大神之神庥,天皇陛下之保佑。是以朕向躬访日本皇室,诚烟致谢,感戴弥重,诏尔众庶,训以一德一心之义,其旨深矣。今兹东渡,恭祝纪元二千六百年庆典,亲拜皇大神宫,回銮之吉,敬立建国神庙,奉祀天照大神,尽厥崇敬,以身祷国民福祉,式为永典,令朕子孙万世祗承,有孚无穷。庶几国本奠于惟神之道,国纲张于忠孝之教。仁爱所安,协和所化,四海清明,笃保神庥。尔众庶其克体朕意,培本振纲,力行弗懈,自强勿息。钦此!\\
\end{quote}

诏书中的“天照大神之神庥,天皇陛下之保佑”,以后便成了每次诏书不可少的谀词。\\

为了让我和伪大臣们接受“神道”思想,日本关东军不怕麻烦,特地把著名神道家\ruby{览}{\textcolor{PinYinColor}{らん}}\ruby{克彦}{\textcolor{PinYinColor}{かつひこ}}(据说是日本皇太后的神道讲师)请来,给我们讲课。这位神道家讲课时,总有不少奇奇怪怪的教材。比如有一幅挂图,上面画着一棵树,据他讲,这棵树的树根,等于日本的神道,上面的枝,是各国各教,所谓八\xpinyin*{纮}一宇,意思就是一切根源于日本这个祖宗。又一张纸上,画着一碗清水,旁边立着若干酱油瓶子、醋瓶子,说清水是日本神道,酱油醋则是世界各宗教,如佛教、儒教、道教、基督教、回教等等。日本神道如同纯净的水,别的宗教均发源于日本的神道。还有不少奇谭,详细的已记不清了。总之,和我后来听到的关于一贯道的说法,颇有点相像。我不知日本人在听课时,都有什么想法,我只知道我自己和伪大臣们,听课时总忍不住要笑,有的就索性睡起觉来。绰号叫于大头的伪军政部大臣\xpinyin*{于琛澂},每逢听“道”就歪着大头打呼噜。但这并不妨害他在自己的故乡照样设大神庙,以示对新祖宗的虔诚。\\

一九四一年十二月八日,日本对美英宣战,在关东军的指示下,伪满又颁布了“时局诏书”。以前每次颁发诏书都是由国务院办的,但这次专门召开了“御前会议”,\ruby{吉冈}{\textcolor{PinYinColor}{よしおか}}让我亲自宣读。这是十二月八日傍晚的事。这诏书也是\ruby{佐藤}{\textcolor{PinYinColor}{さとう}}的手笔。\\

奉天承运大满洲帝国皇帝诏尔众庶曰:\\

\begin{quote}
	盟邦大日本帝国天皇陛下兹以本日宣战美英两国,明诏煌煌,悬在天日,朕与日本天皇陛下,精神如一体,尔众庶亦与其臣民咸有一德之心,夙将不可分离关系,团结共同防卫之义,死生存亡,断弗分携。尔众庶咸宜克体朕意,官民一心,万方一志,举国人而尽奉公之诚,举国力而援盟邦之战,以辅东亚\xpinyin*{戡}定之功,贡献世界之和平,钦此!
\end{quote}

这些恭维谄媚的词令,和“天照大神之神庥,天皇陛下之保佑”一样,以后都成了我的口头禅。\\

我每逢见来访我的关东军司令官,一张嘴便流利地说出:\\

“日本与满洲国乃是一体不可分的关系,生死存亡的关系,我一定举国力为大东亚圣战的最后胜利,为以日本为首的大东亚共荣圈奋斗到底。”\\

一九四二年,做了日本首相的前关东军参谋长\ruby{东条}{\textcolor{PinYinColor}{とうじょう}}\ruby{英机}{\textcolor{PinYinColor}{ひでき}},到伪满作闪电式的访问。我见了他,曾忙不迭地说:\\

“请首相阁下放心,我当举满洲国之全力,支援亲邦日本的圣战!”\\

这时已经把“盟邦”改称为“亲邦”。这是伪满“建国十周年”所带来的新屈辱,是写在“建国十周年诏书”里的。\\

在这个“十周年”(一九四二年)的前夕,\ruby{吉冈}{\textcolor{PinYinColor}{よしおか}}曾和我说:\\

“没有日本,便不会有满洲国,嗯,所以应该把日本看成是满洲国的父亲。所以,嗯,满洲国就不能和别的国家\footnote{伪满于一九三九年,参加了日德意三国于一九三一年订的“防共协定”,这就是所谓盟邦。太平洋战争爆发后,又增添了与伪满建交的日本统治的南洋各傀儡国家。}一样,称日本国为盟邦友邦,应称做亲邦。”\\

与此同时,国务院最末一任总务厅长官\ruby{武部}{\textcolor{PinYinColor}{たけべ}}\ruby{六藏}{\textcolor{PinYinColor}{ろくぞう}},把\xpinyin*{张景惠}和各部伪大臣召到他的办公室里,讲了一番称日本为亲邦的道理。接着“建国十周年诏书”就出来了:\\

\begin{quote}
	我国自肇兴以来,历兹十载,仰赖天照大神之神庥,天皇陛下之保佑,国本奠于惟神之道,政教明于四海之民崇本敬始之典,万世维尊。奉天承运之作,垂统无穷。明明之鉴如亲,穆穆之爱如子。夙夜乾惕,惟念昭德,励精自懋,弗敢豫逸。尔有司众庶,亦成以朕心为心,忠诚任事,勤勉治业,上下相和,万方相协。自创业以至今日,始终一贯,奉公不懈,深堪嘉慰。宜益砥其所心,励其所志,献身大东亚圣战,奉翼亲邦之天业,以尽报本之至诚,努力国本之培养,振张神人合一之纲纪,以奉答建国之明命。钦此!
\end{quote}

从此“亲邦”二字便成了“日本”的代名词。\\

我自认是它的儿子还嫌不够,\ruby{武部}{\textcolor{PinYinColor}{たけべ}}\ruby{六藏}{\textcolor{PinYinColor}{ろくぞう}}和\ruby{吉冈}{\textcolor{PinYinColor}{よしおか}}\ruby{安直}{\textcolor{PinYinColor}{やすなお}}竟又决定,要我写一封“亲书”,由总理\xpinyin*{张景惠}代表我到日本去“谢恩”。我在这里把“谢恩”二字加引号,并非是杜撰,而是真正引用原文的。\xpinyin*{张景惠}的正式身分,乃是“满洲帝国特派赴日本帝国谢恩大使”,这也是写在“亲书”里的。\\

到了一九四四年,日本的败象越来越清楚,连我也能察觉出来,日本军队要倒楣了。有一次\ruby{吉冈}{\textcolor{PinYinColor}{よしおか}}跑来,转弯抹角地先说了一通“圣战正在紧要关头,日本皇军为了东亚共荣圈各国的共存共荣,作奋不顾身的战争,大家自应尽量供应物资,特别是金属……”最后绕到正题上,“陛下可以率先垂范,亲自表现出日满一体的伟大精神……”\\

这回他没有嗯、哈,可见其急不可待,连装腔作势也忘了。而我是浑身毫无一根硬骨头,立即遵命,命令首先把伪宫中的铜铁器具,连门窗上的铜环、铁挂钩等等,一齐卸下来,交给\ruby{吉冈}{\textcolor{PinYinColor}{よしおか}},以支持“亲邦圣战”。过了两天,我又自动地拿出许多白金、钻石首饰和银器交给\ruby{吉冈}{\textcolor{PinYinColor}{よしおか}},送关东军。不久\ruby{吉冈}{\textcolor{PinYinColor}{よしおか}}从关东军司令部回来,说起关东军司令部里连地毯都捐献了,我连忙又命把伪宫中所有地毯一律卷起来送去。后来我去关东军司令部,见他们的地毯还好好地铺在那里,究竟\ruby{吉冈}{\textcolor{PinYinColor}{よしおか}}为什么要卷我的地毯,我自然不敢过问。\\

以后我又自动地拿出几百件衣服,让他送给\ruby{山田}{\textcolor{PinYinColor}{やまだ}}\ruby{乙三}{\textcolor{PinYinColor}{おとぞう}},即最末一任的关东军司令长官。\\

当然,经我这一番带头,报纸上一宣扬,于是便给日伪官吏开了大肆搜刮的方便之门。听说当时在层层逼迫之下,小学生都要回家去搜敛一切可搜敛的东西。\\

\ruby{吉冈}{\textcolor{PinYinColor}{よしおか}}后来对\ruby{溥杰}{\textcolor{PinYinColor}{\Man ᡦᡠ ᡤᡳᠶᡝ}}和我的几个妹夫都说过这样的话:“皇帝陛下,在日满亲善如一体方面,乃是最高的模范。”然而,这位“最高模范”在无关紧要之处,也曾叫他上过当。例如捐献白金的这次,我不舍得全给他们,但又要装出“模范”的样儿,于是我便想出这样一个办法,把白金手表收藏起来,另买了一块廉价表带在手腕上。有一天,我故意当着他的面看表,说:“这只表又慢了一分钟。”他瞅瞅我这只不值钱的表,奇怪起来:“陛下的表,换了的,这个不好……”“换了的,”我说,“原来那只是白金的,献了献了的!”\\

一九四五年,东北人民经过十几年的搜刮,已经衣不蔽体。食无粒米,再加上几次的“粮谷出荷”、“报恩出荷”的掠夺,弄得农民们已是求死无门。这时,为了慰问日本帝国主义,又进行了一次搜刮,挤出食盐三千担,大米三十万吨,送到日本国内去。\\

本来这次关东军是打算让我亲自带到“亲邦”进行慰问的。日本这时已开始遭受空袭,我怕在日本遇见炸弹,只得推说:“值此局势之下,北方镇护的重任,十分重大,我岂可以在这时离开国土一步?”不知道关东军是怎么考虑的,后来决定派一个慰问大使来代替我。\xpinyin*{张景惠}又轮上这个差使,去了日本一趟。他此去死活,我自然就不管了。
