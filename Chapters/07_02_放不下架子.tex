\fancyhead[LO]{{\scriptsize 1945-1950: 在苏联的五年 · 放不下架子}} %奇數頁眉的左邊
\fancyhead[RO]{} %奇數頁眉的右邊
\fancyhead[LE]{} %偶數頁眉的左邊
\fancyhead[RE]{{\scriptsize 1945-1950: 在苏联的五年 · 放不下架子}} %偶數頁眉的右邊
\chapter*{放不下架子}
\addcontentsline{toc}{chapter}{\hspace{1cm}放不下架子}
\thispagestyle{empty}
在苏联的五年拘留生活中,我始终没有放下架子。我们后来移到伯力收容所,这里虽然没有服务员,我照样有人服侍。家里人给我叠被、收拾屋子、端饭和洗衣服。他们不敢明目张胆地叫我“皇上”,便改称我为“上边”。每天早晨,他们进我的屋子,照例先向我请安。\\

刚到伯力郊外的时候,有一天,我想散散步,从楼上下来。楼梯底下椅子上坐着一个从前的“大臣”,他见了我,眼皮也没抬一下。我心里很生气,从此就不想下楼了。每天呆在楼上,大部分时间都花在念经上。不过一般说起来,那些伪大臣大多数对我还是保持尊敬的。举例说,在苏联的五年,每逢过旧历年,大家包饺子吃,第一碗总要先盛给我。\\

我自己不干活,还不愿意我家里这些人给别人干活。有一次吃饭,我的弟弟和妹夫给大家摆台子,就叫我给禁止住了。我的家里人怎么可以去伺候别人!\\

一九四七——四八年间,我家里的人一度被送到同一城市的另一个收容所里,这是我第一次跟家里人分开,感到了很大的不方便。苏联当局很照顾我,容许我单独吃饭。可是谁给我端饭呢?幸而我的岳父自告奋勇,他不仅给我端饭,连洗衣服都愿替我代劳。\\

为了使我们这批寄生虫,做些轻微的劳动,收容所给我们在院子里划出了一些地块,让我们种菜。我和家里人们分得一小块,种了青椒、西红柿、茄子、扁豆等等。看到青苗一天天在生长,我很觉得新奇,于是每天提个水壶接自来水去浇,而且浇得很有趣味。这是以前从来没有过的。但主要的兴趣,还是在于我很爱吃西红柿和青椒。当然,我常常想到,这到底不如从菜铺里买起来方便。\\

为了我们学习,收容所当局发给了我们一些中文书籍,并且有一个时期,叫我的弟弟和妹夫给大家照着本子讲《\ruby{列宁}{\textcolor{PinYinColor}{Ле́нин}}主义问题》和《联共党史》。讲的人莫名其妙,听的人也胡里胡涂。我自己心里只是纳闷,这和我有什么关系?假如不让我留在苏联,还要把我送回去,我就是能背下这两本书,又有什么用?\\

“学习”这两个字,那时对我说起来,还不如青椒、西红柿现实一些。每次学习,我坐在讲桌旁边一个特殊的座位上,总是一边听“教员”结结巴巴地讲我不懂而且也不想懂的“孟什维克”。“国家杜马”,一边胡思乱想:“如果能住在莫斯科,或者伦敦,这些珠宝首饰够我用几年?”“苏联人不吃茄子,这回收下的茄子,怎么个吃法?”……\\

不过,我还能装出很像用心听的样子,可有的人就不同了,他们索性打起鼾来。晚饭后,是自由活动时间,却另是一个样:走廊的一头是几桌麻将;另一头靠窗的地方,有人向窗外天空合掌,大声念着“南无阿弥陀佛!观世音菩萨!”楼上日本战犯那里传来“乌乌乌”的日本戏调子;更稀奇的是有人摆起测字摊,四面围着一群人,讯问什么时候可以回家,家里发生什么事没有。还有些人在卧室里偷着扶\xpinyin*{乩},问的全是有关回家的问题。最初几天,门外的苏联哨兵被吵声惊动,曾经十分惊奇地瞅着这群人,直摇脑袋,后来连他们也习惯了。\\

在这种时候,我多半是在自己的屋子里,摇我的金钱课,念我的金刚经。……
