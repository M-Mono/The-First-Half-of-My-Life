\fancyhead[LO]{{\scriptsize 1950-1954: 由抗拒到认罪 · 初到抚顺}} %奇數頁眉的左邊
\fancyhead[RO]{} %奇數頁眉的右邊
\fancyhead[LE]{} %偶數頁眉的左邊
\fancyhead[RE]{{\scriptsize 1950-1954: 由抗拒到认罪 · 初到抚顺}} %偶數頁眉的右邊
\chapter*{初到抚顺}
\addcontentsline{toc}{chapter}{\hspace{1cm} 初到抚顺}
\thispagestyle{empty}
火车到达抚顺以前,一路上可以听到各式各样关于美妙前景的估计。车上的气氛全变了,大家抽着从沈阳带来的纸烟,谈得兴高采烈。有人说他到过抚顺最豪华的俱乐部,他相信那里必定是接待我们的地方;有人说我们在抚顺不会住很久,休息几天,看几天共产党的书,就会回家;有人说,他到了抚顺首先给家里拍个平安电报,叫家里给准备一下;还有人说,可能在抚顺的温泉洗个澡就走。形形色色的幻想,不一而足。说起原来的恐惧——原来大家都跟我一样——又不禁哈哈大笑。可是,当到了抚顺,下了火车,看见了四面的武装哨兵时,谁的嘴角也不再向上翘了。\\

下了车,我们在武装哨兵的监视戒备下,被领上了几辆大卡车。从这时起,我的头又发起昏来。在胡里胡涂中,不知道过了多少时间,只知道后来车停下时,我已置身在一座深灰色大砖墙的里面。又是大墙!而且是上面装着铁丝网、角上矗立着岗楼。我下了车,随着人们列队走了一小段路,停在一排平房的面前。这排房子的每个窗口,都装着铁栏。我明白了,这是监狱。\\

我们被大兵领进了平房的入口,经过一条狭长的甬道,进了一间大屋子。我们在这里经过检查,然后由不带武器的军人分批领出去。我和另外几个人跟着一个军人在南道里走了一大段,进了一间屋子。我还没看清楚屋里的形势,身后就响起了门外拉铁闩的刺耳声。这间屋子里有一条长长的板炕,一条长桌和两条长凳。跟我一起进来的是伪满的几名将官,当时还不熟悉。我不想跟他们说话,不知道他们是同我一样的恐慌,还是由于在我面前感到拘谨,也一律一声不响,低着脑袋站在一边。这样怔了一阵,忽然那刺耳的铁闩声又响了,房门被拉开,一位看守人员走进来,让我跟他到另一间屋子去。我没想到在这间屋子里又看见了我的三个侄子、二弟溥杰和我的岳父荣源。原来还是让我们住在一起的。他们刚刚领到新被新褥和洗漱用具,而且给我也带了一套来。\\

最先使我受到安慰的,是荣源凭着他的阅历做出的一番分析。\\

“这是一所军事监狱,”他摸着窗栏说,“全是穿军装的,没有错。不像马上……出危险,不然何必发牙刷、毛巾呢。刚才检查的时候,留下了金银财物,给了存条,这也不像是对……,这是对待普遍犯人的。再说伙食也不错。”\\

“伙食不错,别是什么催命宴吧?”侄子小固毫无顾忌地说。\\

“不,那种饭有酒,可是这里并没有酒。”他很有把握地说,“我们看下顿,如果下顿仍是这么好,就不是了。没听说连吃几顿那个的。”\\

第二天,我开始有点相信岳父的话了,倒不是因为伙食和昨天不相上下,而是因为军医们给我们进行了身体检查。检查非常仔细,连过去生过什么病,平常吃什么、忌什么都问到了。同时还发了新的黑裤褂和白内衣,令人更惊异的是还给了纸烟。显然,这不像是对待死囚的。\\

过不多天,一个粗短身材、年在四十上下的人走进我们的屋子。他问了我们每个人的名字,在苏联都看过什么书,这几夜睡的好不好。听了我们的回答之后,他点点头,说:“好,马上就发给你们书籍、报纸,你们好好学习吧。”几个钟头之后,我们便收到了书籍、报纸,还有各类的棋和纸牌。从这天起,我们每天听两次广播,广播器就设在甬道里,一次是新闻,一次是音乐或戏曲节目。除此之外,每天下午还有一个半小时的院中散步。就在第一次外出散步时,侄子小固打听出这个叫我们“好好学习”的人是这个战犯管理所的所长。\\

给我们送书来的那人姓李,后来知道是位科长。\\

那时我们除了对所长之外,管所方人员一律叫“先生”(因为那时不知道别的称呼)。这位李先生给我拿来了三本书——《新民主主义论》、《中国近百年史》和《新民主主义革命史》。他说现在书还不够,大家可以轮流看,或者一人念大家听。这些书里有许多名词,我们感到很新鲜,然而更新鲜的则是叫我们这伙犯人念书。\\

对这些书最先发生兴趣的是小固,他看的比谁都快,而且立刻提出了疑难问题要别人解答。别人答不上来,他就去找管理所的人问。荣源讥笑了他,说:“你别以为这是学校,这可是监狱。”小固说:“所长不是说要我们学习吗?”荣源说:“学习,也是监狱。昨天放风时我听人说,这地方从前就是监狱。从前是,现在有书有报还是。”溥杰跟着说,日本监狱据说也给书看,不过还没听说过中国有这么“文明的监狱”。荣源仍是摇头晃脑地说:“监狱就是监狱,文明也是监狱。学那行子,还不如念念佛。”小固要和他争辩,他索兴闭上眼低声念起佛来。\\

这天我们从院子里散步回来,小固传播了刚听来的一条新闻:前伪满总务厅次长老谷拿一块表送给看守员(这时我们还不知道这个职务名称,我们当面称先生,背后叫“管人的”),结果挨了一顿训。这条新闻引起了几个年轻人的议论。小秀说,上次洗澡的热水,并不是热水管子里的;锅炉还没修好,那水是“管人的”先生们用水桶一担一担挑来的。“给犯人挑水,还没听说过。”小瑞也认为这里“管人的”跟传说中的“狱卒”不同,不骂人。不打人。荣源这时正为吃晚饭做准备月捻完“往生神咒”,冷笑了一下,低声说:\\

“你们年轻人太没阅历,大惊小怪!那送表的一定送的不是时候,叫别人看见了,当人面他怎么能要?不打、不骂,你就当他心里跟咱没仇?瞧着吧,受罪在后头!”\\

“挑水又怎么说?”小固顶撞地说,“给咱挑水洗澡,就是叫咱受罪?”\\

“不管怎么说,”荣源的声音压得更低了,“共产党,不会喜欢咱这种人!”\\

说着,他摸了一阵口袋,忽然懊恼地说:“我把烟忘在外边窗台上了。真可惜,从沈阳带回来的只剩这一包了。”他不情愿地打开一包所里发给的低级烟,还嘟囔着,“这里‘管人的’大都吸烟,我那包算白送礼了!”\\

真像戏里所说的,“无巧不成书”,他的话刚说完,房门被人拉开了,一个姓王的看守员手里举着一样东西问道:“这屋里有人丢了烟没有?”大家看得清楚,他手里的东西正是荣源那包沈阳烟。\\

荣源接过了烟,连声地说:“谢谢王先生,谢谢王先生!”听看守员的脚步声远了,小固先禁不住笑起来,问他刚才念的是什么咒,怎么一念就把烟给念回来了。荣源点上了烟,默默地喷了一阵,恍然大悟似地拍了一下大腿:\\

“这些‘管人的’准是专门挑选来的!为了跟咱们斗心眼儿,自然要挑些文明点儿的!”\\

小固不笑了,溥杰连忙点头,另外两个侄子也被荣源的“阅历”镇住了。我和溥杰一样,完全同意荣源的解释。\\

过了不多天,发生了一件事,使荣源的解释大为逊色。这天我们从院子里散步回来,溥杰一面急急忙忙地找报纸,一面兴奋地说,他刚听见别的屋子里的人都在议论今天报上登的一篇文章,这篇文章使他们猜透了新中国叫我们学习的意思。大家一听,都拥到了他身边,看他找的是什么文章。文章找着了,我忘了那文章的题目,只记得当溥杰念到其中新中国迫切需要各项人材,必须大量培养、大胆提拔干部的一段时,除了荣源之外,所有的脑袋都挤到了报纸上面。据溥杰听到别的屋子里的人判断,政府让我们学习,给我们优待,就是由于新国家缺少人材,要使用我们这些人。今天想起来,这个判断要多可笑有多可笑,可是在当时它确实是多数人的想法。在我们这间屋子里,尽管荣源表示了怀疑,其他人却越想越觉着像是这么回事。\\

我记得从那天起,屋里有了一个显著的变化,大家都认真地学习起来。从前,除了小固之外,别人对那些充满新名词的小册子都不感兴趣,每天半天的读书,主要是为了给甬道里的看守人员看。现在,不管看守人员在不在,学习都在进行着。那时还没有所方于部给讲解,所谓学习也只不过是抠抠名词而已。当然,荣源仍旧不参加,在别人学习的时候,他闭着眼念他的经。\\

这种盲目的乐观,并没有持续多久,当所方宣布调整住屋,把我和家族分开时,它就像昙花一现似地消失了。