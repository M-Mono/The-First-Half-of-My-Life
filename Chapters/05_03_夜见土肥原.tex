\fancyhead[LO]{{\scriptsize 1931-1932: 到东北去 · 夜见土肥原}} %奇數頁眉的左邊
\fancyhead[RO]{} %奇數頁眉的右邊
\fancyhead[LE]{} %偶數頁眉的左邊
\fancyhead[RE]{{\scriptsize 1931-1932: 到东北去 · 夜见土肥原}} %偶數頁眉的右邊
\chapter*{夜见土肥原}
\addcontentsline{toc}{chapter}{\hspace{1cm}夜见土肥原}
\thispagestyle{empty}
\begin{quote}
	在这里所处理的时期之初,\ruby{土肥原}{どいはら}是日本陆军大佐,一九四一年四月升到将官阶级,在“九·一八”事变前约十八年间居住中国,被视为陆军部内的中国通。他对于在满洲所进行的对华侵略战争的发动和进展,以及嗣后受日本支配的伪满洲国之设立,都具有密切关系。日本军部派对中国其他地区所采取的侵略政策,\ruby{土肥原}{どいはら}借着政治的谋略、武力的威胁、武力的行使,在促使事态的进展上担任了显著的任务。\\

\ruby{土肥原}{どいはら}当军部派其他指导者设计、准备和实行将东亚及东南亚置于日本支配之下时,曾和他们保持密切联络而行动。正当他的对华的特殊知识和他的在华行使阴谋的能力已无需要时,他就以现地将官的地位来担当实现他本人曾经参预的阴谋目的。他不但曾参加对中国的侵略战争的实行,并且也参加了对苏联以及对各国,即一九四一年至一九四五年日本曾对其实行侵略战争的各国,除法国以外的侵略战争的实行。\\

\begin{flushright}
	——《远东国际军事法庭判决书》\\
\end{flushright}
\end{quote}

\ruby{土肥原}{どいはら}和\ruby{板垣}{いたがき},在“远东国际军事法庭”审判的二十五名战犯中,是被判定犯罪条款最多的两人。他们两人罪状相同,都犯了七条“破坏和平罪”\footnote{这七条是:十八年间一贯为控制东南亚及太平洋的阴谋、对华实行侵略战争、对美实行侵略战争、对英实行侵略战争、对荷兰实行侵略战争、对法实行侵略战争、制造张鼓峰事件、制造诺门坎事件。},犯了“违反战争法规惯例及违反人道之犯罪”中最重的一条,即“命令准许违约行为”之罪。远东国际军事法庭对这批战犯拖到一九四八年十一月才判决,\ruby{土肥原}{どいはら}与\ruby{板垣}{いたがき}和其他五名战犯都被判处了绞刑。\\

\ruby{土肥原}{どいはら},是个完全靠侵略中国起家的日本军人。他在陆军士官学校十六期步兵科和陆军大学毕业后,做过日本参谋本部部员,第十三步兵联队长,一九一三年起他来到中国,在关东军中服务,给东北军阀的顾问\ruby{坂西}{ばんざい}\ruby{利八郎}{りはちろう}中将当了十多年的副官。他和\xpinyin*{张作霖}的关系特别深,一九二四年直奉战争中,他策动关东军帮助过\xpinyin*{张作霖}。一九二八年关东军决定消灭\xpinyin*{张作霖},在皇姑屯炸死\xpinyin*{张作霖}的阴谋,也有他参加。不久,他即因功晋级大佐,担任了沈阳特务机关长的职务,从此开始了判决书上所述的那些罪行,开始了飞黄腾达。其实\ruby{土肥原}{どいはら}的许多“杰作”《判决书》里都没有提到,例如一九三一年十一月的天津骚动事件、一九三二年热河战争的爆发、一九三五年五月的丰台事变和冀东伪组织的成立、十一月香河流氓暴动和冀察的特殊政权的出现,都离不开\ruby{土肥原}{どいはら}的策划活动。可以说,在那段时间里,\ruby{土肥原}{どいはら}走到哪里,灾难就降临哪里。大约他的失败只有过一次,即在他拉拢之下叛国的\xpinyin*{马占山},后来反正抗日。但是这并没有影响他后来的升迁,他被调去当旅团长的时间不长,又调回任关东军的特务机关长。一直到“七·七”事变,日本人要成立的伪组织都成立起来了,骚乱、暴动等等手段也被武装进攻代替了,\ruby{土肥原}{どいはら}才脱去了白手套,拿起了指挥刀,以师团长、军团长、方面军总司令等身分,统帅着日兵在中国大陆和东南亚进行屠杀和掠夺。就这样,在尸骨和血泊中,他从“九·一八”事变起不过十年间,由大佐升到大将。\\

那时关于他有种种充满了神秘色彩的传说,西方报纸称他为“东方的\ruby{劳伦斯}{Lawrence}”\footnote{著名的英国老特务。},中国报纸上说他惯穿中国服装,擅长中国方言。根据我的了解,他在中国的活动如果都像鼓动我出关那样做法,他并不需要传说中的\ruby{劳伦斯}{Lawrence}的诡诈和心机,只要有一副赌案上的面孔,能把谎话当真话说就行了。那次他和我会见也没有穿中国服装,只不过一套日本式的西服;他的中国话似乎并不十分高明,为了不致把话说错和听错,他还用了\ruby{吉田}{よしだ}\ruby{忠太郎}{ちゅうたろう}充当我们的翻译。\\

他那年四十八岁,眼睛附近的肌肉现出了松弛的迹象,鼻子底下有一撮小胡子,脸上自始至终带着温和恭顺的笑意。这种笑意给人的惟一感觉,就是这个人说出来的话,不会有一句是靠不住的。\\

他向我问候了健康,就转入正题,先解释日军行动,说是只对付\xpinyin*{张学良}一个人,说什么\xpinyin*{张学良}“把满洲闹得民不聊生,日本人的权益和生命财产得不到任何保证,这样日本才不得已而出兵”。他说关东军对满洲绝无领土野心,只是“诚心诚意地,要帮助满洲人民,建立自己的新国家”,希望我不要错过这个时机,很快回到我的祖先发祥地,亲自领导这个国家旧本将和这个国家订立攻守同盟,它的主权领土将受到日本的全力保护;作为这个国家的元首,我一切可以自主。\\

他的诚恳的语调,恭顺的笑容和他的名气、身分完全不容我用对待\xpinyin*{罗振玉}和\ruby{上角}{うえすみ}\ruby{利一}{としかず}的态度来对待他。\xpinyin*{陈宝琛}所担心的——\ruby{伯罗}{はくら}和\ruby{上角}{うえすみ}不能代表关东军,怕关东军不能代表日本政府——那两个问题,我认为更不存在了。\ruby{土肥原}{どいはら}本人就是个关东军的举足轻重的人物,况且他又斩钉截铁地说:“天皇陛下是相信关东军的!”\\

我心里还有一个极重要的问题,我问道:\\

“这个新国家是个什么样的国家?”\\

“我已经说过,是独立自主的,是由\xpinyin*{宣统}帝完全做主的。”\\

“我问的不是这个,我要知道这个国家是共和,还是帝制?是不是帝国?”\\

“这些问题,到了沈阳都可以解决。”\\

“不,”我坚持地说,“如果是复辟,我就去,不然的话我就不去。”\\

他微笑了,声调不变地说:\\

“当然是帝国,这是没有问题的。”\\

“如果是帝国,我可以去!”我表示了满意。\\

“那么就请\xpinyin*{宣统}帝早日动身,无论如何要在十六日以前到达满洲。详细办法到了沈阳再谈。动身的办法由\ruby{吉田}{よしだ}安排吧。”\\

他像来时那样恭敬地向我祝贺一路平安,行了礼,就告辞了。\ruby{土肥原}{どいはら}走后,我接见了和\ruby{土肥原}{どいはら}一齐来的\xpinyin*{金梁},他带来了以\xpinyin*{袁金铠}为首的东北遗老们的消息,说他们可以号召东北军旧部归服。总之,我认为完全没问题了。\\

\ruby{土肥原}{どいはら}去后,\ruby{吉田}{よしだ}告诉我,不必把这件事告诉总领事馆;关于动身去大连的事,自有他给我妥善安排。我当时决定,除了\xpinyin*{郑孝胥}之外,再不找别人商量。\\

但是,这回消息比上次我去日本兵营传得还快,第二天报上登出了\ruby{土肥原}{どいはら}和我见面的新闻,而且揭露出了\ruby{土肥原}{どいはら}此行的目的。\xpinyin*{陈宝琛}那几天本来不在天津,得到了消息,匆忙地从北京跑回来,一下火车直奔\xpinyin*{郑孝胥}家里,打探了消息,然后奔向静园。这时正好\xpinyin*{刘骧业}从日本东京发来一封电报,说日本军部方面认为我出山的时机仍然未至。看了这个电报,我不得不把会见\ruby{土肥原}{どいはら}的情形告诉了他,并且答应和大伙再商量一下。\\

这天是十一月五日,静园里开了一个别开生面的“御前会议”。记得被我召来的除\xpinyin*{陈宝琛}、\xpinyin*{郑孝胥}、\xpinyin*{胡嗣瑗}之外,还有在天津当寓公的\xpinyin*{袁大化}和\xpinyin*{铁良}(\xpinyin*{升允}此时刚刚去世)。在这次会议上,\xpinyin*{陈宝琛}和\xpinyin*{郑孝胥}两人展开了激烈的辩论。\\

“当前大局未定,轻举妄动有损无益。\xpinyin*{罗振玉}迎驾之举是躁进,现在启驾的主意何尝不是躁进!”\xpinyin*{陈宝琛}瞅着\xpinyin*{郑孝胥}说。\\

“彼一时,此一时。时机错过,外失友邦之热心,内失国人之欢心,不识时务,并非持重!”\xpinyin*{郑孝胥}瞅着\xpinyin*{陈宝琛}说。\\

“日本军部即使热心,可是日本内阁还无此意。事情不是儿戏,还请皇上三思而定。”\\

“日本内阁不足道,日本军部有帷幄上奏之权。三思再思,如此而已!”\\

“我说的请皇上三思,不是请你三思!”\\

“三思!三思!等日本人把\xpinyin*{溥伟}扶上去,我们为臣子的将陷皇上于何地?”\\

“\xpinyin*{溥伟}弄好弄坏,左不过还是个\xpinyin*{溥伟}。皇上出来只能成,不能败。倘若不成,更陷皇上于何地?更何以对得起列祖列宗?”\\

“眼看已经山穷水尽了!到了关外,又恢复了祖业,又不再愁生活,有什么对不起祖宗的?”\\

在\xpinyin*{郑孝胥}的飞溅的唾星下,\xpinyin*{陈宝琛}脸色苍白,颤巍巍地扶着桌子,探出上身,接近对面的秃头顶,冷笑道:\\

“你,有你的打算,你的热衷。你,有何成败,那是毫无价值可言!……”\\

一言不发的\xpinyin*{袁大化},低头不语的\xpinyin*{铁良},以及由于身分够不上说话只能在旁喘粗气的\xpinyin*{胡嗣瑗},觉着不能再沉默了,于是出来打圆场。\xpinyin*{铁良}说了些“从长计议”的话,透出他是支持\xpinyin*{陈宝琛}的,\xpinyin*{袁大化}嘟囔了几句,连意思都不清楚。\xpinyin*{胡嗣瑗}想支持\xpinyin*{陈宝琛},可是说不明白。我在会上没有表示态度,但心里认为\xpinyin*{陈宝琛}是“忠心可嘉,迂腐不堪”。\\

我觉得最好的办法,还是不要表示自己的想法,不透露自己的意图。对身边的人如此,对社会上更要如此。在这里我要插叙一下,大约是\ruby{土肥原}{どいはら}会见后的两三天,我接见\xpinyin*{高友唐}的一段事。\\

那几天要求见我的人非常多,我认为全部加以拒绝,只能证实报纸上的推测,那对我会更加不利。至于这个\xpinyin*{高友唐},更有接见的必要。他以前也是张园的客人,张园把他看做遗老,因为他是清朝仕学馆出身,做过清朝的官,后来办过几种报纸,当了国民党的监察院委员,曾自动为我向南京要求过“岁费”(没有结果)。我想他可能透点什么消息给我,所以接见了他。没想到他是给\xpinyin*{蒋介石}来做说客的。他说国民党政府给他来了电报,叫他告诉我,国民政府愿意恢复优待条件,每年照付优待费,或者一次付给我一笔整数也可以,请我提出数目;至于住的地方,希望我选择上海,我如果要出洋,或者要到除了东北和日本以外的任何地方,都可以。\\

听了他的话,我冷笑说:\\

“国民政府早干什么去了?优待条件废了多少年,\xpinyin*{孙殿英}读犯了我的祖陵,连管也没有管,现在是怕我出去丢\xpinyin*{蒋介石}他们的人吧,这才想起来优待。我这个人是不受什么优待的,我也不打算到哪儿去。你还是个大清的旧臣,何必替他们说话!”\\

\xpinyin*{高友唐}是用遗老身分,以完全为我设想的口气,向我说话的。他说国民政府的条件对我很有利,当然,他们常常说话不算数,但是,如果我认为有必要,可以由外国银行做保。他说:“如果有外国人做保,\xpinyin*{蒋介石}这回是决不敢骗人的。”他似乎颇能懂得我的心理,说优待条件恢复了,当然也恢复帝号,假使想回北京,也可以商量。\\

我对他的话并不相信。我早听说\xpinyin*{蒋介石}的手腕厉害,有人说他为了和英美拉拢而娶\xpinyin*{宋美龄},连他的发妻都不要了,根本不讲信义,这种人是专门欺软怕硬的。因为他怕日本人,现在看见日本人和我接近,就什么条件都答应下来,等我离开了日本人,大概就该收拾我了。就算他说的都算数,他给了我一个帝号,又哪比得上\ruby{土肥原}{どいはら}答应的帝位呢?他能给我的款子,又怎么比得上整个的东北呢?\xpinyin*{蒋介石}再对我好,他能把江山让给我吗?想到这里,我就不打算再跟\xpinyin*{高友唐}说下去了。\\

“好吧,你的话我都知道了,这次谈话可以告一段落。”\\

\xpinyin*{高友唐}看我沉思之后说了这么一句,却误认为事情有希望,连忙说:“好,好,您再想想,等过几天我再来。”\\

“嗯,再来吧。”\\

他满怀希望地走了。后来听说他向我七叔活动之后从北京回来,遇上了“天津事变”,被截在租界外边。等他设法进了日租界,我已经不在静园了。\\

那两天里陆陆续续还来了些探听消息的或提出忠告的人,我也收到了不少的来信。人们对我有忠告,有警告,甚至有姓\xpinyin*{爱新觉罗}的劝我不要认贼作父,要顾惜中国人的尊严。我已经被复辟的美梦完全迷了心窍,任何劝告都没有生效。我决定对外不说任何真心话。有个天津小报的记者,叫\xpinyin*{刘冉公}的,也是张园和静园常来的客人,时常在他的报上写文章恭维我,这时跑来打听我有没有出关的意思。他见我极力否认,于是又替我尽了辟谣的义务。他却没想到,就在他的报上登出了为我辟谣新闻的同一天,我登上了去营口的日本轮船。\\

在我离津前两天发生的一件事,不可不说。那天我正在唾星喷射之下听着进讲:\\

“勿失友邦之热心,勿拒国人之欢心……此乃英雄事业,决非书生文士所能理解……”\\

“不好了!”我的随侍\xpinyin*{祁继忠},忽然慌慌张张地跑了进来,“炸弹!两个炸弹!……”\\

我坐在沙发上,吓得连站也站不起来了。在混乱中,好容易才弄明白,刚才有个陌生人送来一份礼品,附着一张原东北保安总司令部顾问\xpinyin*{赵欣伯}的名片。来人放下了礼品,扬长而去。\xpinyin*{祁继忠}按例检视了礼品,竟在水果筐子里发现了两颗炸弹。\\

静园上下惊魂未定,日本警察和日军司令部的军官来了,拿走了炸弹。第二天,\ruby{吉田}{よしだ}翻译官向我报告说,那两颗炸弹经过检验,证明是\xpinyin*{张学良}的兵工厂制造的。\\

“\xpinyin*{宣统}帝不要再接见外人了。”\ruby{吉田}{よしだ}忠告我,“还是早些动身的好。”\\

“好!请你快些安排吧。”\\

“遵命!请陛下不要对不相干的人说。”\\

“不说。我这回只带\xpinyin*{郑孝胥}父子和一两个随侍。”\\

那两天我接到了不少恐吓信。有的信文很短,而措词却很吓人。有一封只有这么一句话:“如果你不离开这里,当心你的脑袋!”更惊人的,是\xpinyin*{祁继忠}接到了一个电话。据\xpinyin*{祁继忠}说,对方是维多利亚餐厅的一个茶房,他警告我这几天不要去那里吃饭,因为有些“形迹可疑的人”到那里打听我。这个关心我的朋友还说,他见那些形迹可疑的人,好像衣服里面藏有电刀。更奇的是,他居然能认出那些人都是\xpinyin*{张学良}派来的。\\

那个茶房是怎样的人,我已说不清了,关于\xpinyin*{祁继忠}这人,我却永远忘不了他。他是我从北京带到天津的男仆,宫里遣散太监后,他来到宫里,那时候还是个少年,很受我的宠信。在天津时代,他是我最喜欢的随侍之一,在伪满时,我送他到日本士官学校培养。可是后来,我发现了他竟是“内廷秽闻”中的人物,那时正巧听说他在日本和同学吵架,我就借了个破坏日满邦交的题目,请日本人把他开除出了学校。后来他经日本人介绍到华北当上伪军军官,以后又摇身一变成了华北伪军少将,解放后因反革命案被镇压。我离开天津去东北,他是随我同去的三个随侍之一,我的举动他无一不知。我到很晚才明白过来,日本人和\xpinyin*{郑孝胥}对我当时的动静那么清楚,对我的心情掌握的那么准确及时,而演给我看的那出戏——虽然演员们演的相当笨拙——效果又是那么好,\xpinyin*{祁继忠}实在是个很有关系的人。\\

紧接着炸弹、黑信、电话而至的,是“天津事件”的发生。日本人组织的汉奸便衣队对华界大肆骚扰(这也是\ruby{土肥原}{どいはら}导演的“杰作”),日租界宣布戒严,断绝了与华界的交通。静园门外开来担任“保护”之责的铁甲车。于是静园和外界也隔绝了。能拿到通行证的,只有郑氏父子二人。\\

后来我回想起来,\ruby{土肥原}{どいはら}这样急于弄我到东北去,如果不是关东军少壮派为了急于对付他们内部的反对派,或其他别的原因,而仅仅是怕我再变了主意的话,那就把外界对我的影响估计得太高了。事实上,不但我这时下定决心,就连\xpinyin*{陈宝琛}影响下的\xpinyin*{胡嗣瑗}、\xpinyin*{陈曾寿}等人,态度上也起了变化。他们不再坚持观望,开始打算主动和日本进行接触。不过他们仍怕军人靠不住,认为还是找日本政府的好。这些人的变化,和我一样是既怕错过机会,又怕羊肉没吃成反而惹上一身膻。对于和日本人交涉的条件,他们关心的是能不能当上大官,因此主张“用人权”必须在我,至于什么民族荣誉、经济利权等等,是完全可以当做换取自己地位的代价送出去的。\xpinyin*{陈曾寿}在我会见\ruby{土肥原}{どいはら}后立刻递上了这样一个奏折:\\

\begin{quote}
	奏为速赴机宜,以策万全,恭折仰祈圣鉴事。今日本因列强反对而成僵局,不得不变动东三省局面以自解于列强,乃有此劝进之举,诚千载一时之机会。遇此机会而无以赴之,则以后更有何机之可待?惟赴机若不得其宜,则其害有甚于失机者。今我所以自处之道,可两言而决:能与日本订约,酌让路、矿、商务之利,而用人行政之权,完全自主,则可以即动,否则万不可动,如是而已。现报纸喧腾,敌人疑忌,天津已有不能安处之势。欲动则恐受赚于日本,欲静又失此良机,进退两难,惟有请皇上密派重臣径赴日本,与其政府及元老西园寺等商洽,直接订约后再赴沈阳,则万全而无失矣。臣愚昧之见,是否有当,伏祈圣鉴。\\
\end{quote}
