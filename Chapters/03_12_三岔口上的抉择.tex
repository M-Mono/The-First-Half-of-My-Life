\fancyhead[LO]{{\scriptsize 1917-1924: 北京的“小朝廷” · 三岔口上的抉择}} %奇數頁眉的左邊
\fancyhead[RO]{} %奇數頁眉的右邊
\fancyhead[LE]{} %偶數頁眉的左邊
\fancyhead[RE]{{\scriptsize 1917-1924: 北京的“小朝廷” · 三岔口上的抉择}} %偶數頁眉的右邊
\chapter*{三岔口上的抉择}
\addcontentsline{toc}{chapter}{\hspace{1cm}三岔口上的抉择}
\thispagestyle{empty}
北府里的人虽然有共同的兴奋,却没有共同的想法。\xpinyin*{金梁}后来在他补写的《遇变日记》里说:“盖自段、张到京后,皆空言示好,实无办法。众为所欺,以为恢复即在目前,于是事实未见,而意见已生。有主张原订条件一字不能动者,有主必还宫复号者,有主改号逊帝者,有主岁费可减,必有外人保证者,有主移住颐和园者,有主在东城购屋者。实则主权在人,无异梦想,皆不知何所见而云然也。”这段话说的的确是实情。\\

一九二四年十一月五日的这场旋风,把我一下子抛出了紫禁城,落到一个三岔口上。我面前摆着三条路:一条是新“条件”给我指出的,放弃帝王尊号,放弃原来的野心,做个仍然拥有大量财宝和田庄的“平民”;另一条,是争取“同情者”的支援,取消国民军的新条件,全部恢复\xpinyin*{袁世凯}时代的!日条件,或者“复号还宫”,让我回到紫禁城,依然过着从前那样的生活;还有一条,是最曲折的道路,它通向海外,然后又指向紫禁城,不过那时的紫禁城必须是\xpinyin*{辛亥}以前的紫禁城。这条路当时的说法则是“借外力谋恢复”。\\

我站在这个三岔路口上,受着各种人的包围,听尽了他们的无穷无尽的争吵。他们对于第一条路,都认为不屑一顾,而在其他两条路线的选择上,则又互不相让。即使是同一条路线的拥护者,也各有不同的具体主张和详细计划。他们每个人都争先恐后地给我出主意,抢着给我带路。\\

在刚进北府的那几天,争论的中心是“留在北府呢,还是设法溜出去,躲进东交民巷”?前面已说过,主张溜走的一方是处于孤势的\xpinyin*{郑孝胥}和不公开表态的\xpinyin*{庄士敦},另一方则是以我父亲为首的王公大臣以及师傅们。这场冲突是以\xpinyin*{郑孝胥}的失败而告终。门禁开始放松以后,则以“出洋不出洋,争取不争取恢复原优待条件”为中心展开了第二次交锋。主张立即出洋的一方是\xpinyin*{金梁}和\xpinyin*{罗振玉}(\xpinyin*{庄士敦}仍是不公开表态的一个),另一方仍以我父亲为首,有师傅们参加。他们这次的矛头主要对着“急先锋”\xpinyin*{金梁},也取得了胜利。不过,这是一个表面的胜利。到第三个回合,即郑、罗、庄联合了起来,并争得了\xpinyin*{陈宝琛}的参与,问题重心转到了“我的当前处境危险不危险,要不要先跑进东交民巷”的时候,那些王公大臣便惨败了。\\

以我父亲为首的王公大臣们,一心一意地想恢复原状,争取复号还宫。他们对国民军怀着仇恨,却希望我加以忍受和等待。国民军取消了我的皇帝尊号,他们认为我还可以在家里做皇帝,反正他们不取消我的尊号。国民军的统治刚露出了不稳征兆(张、冯不和,黄内阁被拒于使团),他们的幻想就抬头了。他们一面劝我静待佳音,一面对于一切主张出洋以及出府的人,大肆攻击。他们在第一个回合上取得了胜利,让我去不成东交民巷,在第二个回合上,又让\xpinyin*{金梁}败得很狼狈。\xpinyin*{金梁}从报上看到了我对鹿锺麟的谈话以后,门禁刚一松动,便带着一份奏折和替我拟好的“宣言书”来了。他大大地夸奖了我的谈话,请我对外宣布“敝\xpinyin*{屣}一切,还我自由,余怀此志久矢”!叫我放弃帝号和优待费,把钱拿出来办图书馆和学校,以“收人心,抗\xpinyin*{舆}论”,同时要“托内事于忠贞之士,而先出洋留学,图其远者大者,尽人事以待天命,一旦有机可乘,立即归国”。他的论点是:“盖必敝\xpinyin*{屣}今日之假皇帝,始可希望将来之真皇帝”。他说过之后,又写成一个《请速发宣言疏》。这一番话,尽管令我动心,但是我父亲知道之后,对他大怒,把他称做“疯子”,请他以后不要再上门来。\\

其实,\xpinyin*{金梁}并不是坚决的“出洋派”。他的主张曾让我一时摸不着头脑。\xpinyin*{段祺瑞}上台后,还原的呼声甚嚣尘上之际,他托人递折子给我,再不提“敝\xpinyin*{屣}一切”和放弃优待条件、帝号的话,说如果能争回帝号,我亦不可放弃。他同时上书\xpinyin*{张作霖}说:“优待条件事关国信,效等约法,非可轻易修改。”他对别人解释说,他原并不是主张放弃帝号的,不过此事不宜由我去争而已。他的解释没有得到我父亲的谅解,也引不起我的兴趣,北府的大门也进不来了。\\

我父亲赶走\xpinyin*{金梁}之后,为了防范别人对我的影响,每逢有他认为靠不住的人来访我,他不是加以拦阻,就是立在一边看守着,因此另一个主张出洋的\xpinyin*{罗振玉}被他弄得无法跟我说话。我父亲的“王爷”威风只有对\xpinyin*{庄士敦}不敢使用,但是门口上的大兵无形中帮了父亲的忙,\xpinyin*{庄士敦}从第二天起就进不来了。所以我父亲这一次在对付出洋派上,又成了胜利者。\\

我父亲这一派人接连得到的两次胜利,却是十分不巩固的胜利。他的封锁首先弓!起我心中更大的反感。尽管我对自己的前途还没有个明确的打算,但这一点是从进了北府大门就明确了的:无论如何我得离开这个地方。我不能出了一座大紫禁城,又钻进一座小紫禁城,何况这里并不安全。\\

后来,我向父亲表示了不满,我不希望在我接见人的时候总有他在场,更不希望想见我的人受到阻拦。父亲让了步,于是情况有了变化,各种带路人都带着最好的主意来了。这时又出现了一个新的出洋派。我的老朋友\xpinyin*{胡适}博士来了。\\

不久以前,我刚在报上看到\xpinyin*{胡适}一封致王正廷的公开信,大骂国民军,表示了对于“以武力胁迫”修改优待条件这种行为的“义愤”。虽然\xpinyin*{陈宝琛}仍然把他视同蛇蝎,但\xpinyin*{郑孝胥}已经和他交上了朋友,有些遗老也认为他究竟比革命党和国民军好。他走进北府,没有受到阻拦,我见到他,表示了欢迎,并且称赞他在报上发表的文章。他又把国民军骂了一通,说:“这在欧美国家看来,全是东方的野蛮!”\\

\xpinyin*{胡适}这次见我,并不是单纯的慰问,而是出于他的“关心”。他问我今后有什么打算。我说王公大臣们都在活动恢复原状,我对那些毫无兴趣,我希望能独立生活,求些学问。\\

“皇上很有志气!”他点头称赞,“上次我从宫里回来,就对朋友说过,皇上很有志气。”\\

“我想出洋留学,可是很困难。”\\

“有困难,也不太困难。如果到英国,\xpinyin*{庄士敦}先生可以照料。如果想去美国,也不难找到帮忙的人。”\\

“王公大臣们不放我,特别是王爷。”\\

“上次在宫里,皇上也这样说过。我看,还是要果断。”\\

“民国当局也不一定让我走。”\\

“那倒好说,要紧的还是皇上自己下决心。”\\

尽管我对这位“新人物”本能地怀着戒心,但他的话确实给了我一种鼓励。我从他身上觉察出,我的出洋计划,一定可以得到社会上不少人的同情。因此,我越发讨厌那些反对我出洋的王公大臣们了。\\

我认为,那些主张恢复原状的,是因为只有这样,才好保住他们的名衔。他们的衣食父母不是皇上,而是优待条件。有了优待条件,\xpinyin*{绍英}就丢不了“总管内务府印钥”,\xpinyin*{荣源}就维持住乐在其中的抵押、变价生涯,\xpinyin*{醇王}府就每年可以照支四万二千四百八十两的岁费,这是不管民国政府拖欠与否,内务府到时都要凑足送齐的。除了这些人以外,下面的那些喽罗,不断地递折子、上条陈,也各有其小算盘。我六叔\ruby{载洵}{zǎi xún}有个叫吴锡宝的门客,写了一个“奏为陈善后大计”的折子,一上来就抱怨说,他早主张要聘用各国法学家研究法律,以备应付民国违法毁约的举动,因为没听他的主意,所以今天手忙脚乱,驳辩无力。接着他提出五条大计,说来说去都没离了用法律和法学家,其原因,他自己就是一名律师。还有一个名叫多济的旗人,是挂名的内务府员外郎,他坚决主张无论如何不可放弃帝号,不但如此,我将来有了儿子还要叫做“\xpinyin*{宣统}第二”。他又主张今后我应该把侍奉左右的人都换上八旗子弟。看来他也打好主意,让他的儿子做“多济第二”,来继承员外郎这份俸银。\\

我见过了\xpinyin*{胡适},\xpinyin*{庄士敦}也回到我身边,向我转达了\xpinyin*{张作霖}的关怀。我觉得\xpinyin*{胡适}说的不错,出洋的问题不致于受到当局的阻拦。我和\xpinyin*{庄士敦}计议如何筹备出洋的事,\xpinyin*{张作霖}又做了表示,欢迎我到东北去住。我想先到东北住一下也好,我到了东北,就随时可以出洋了。我刚拿定了主意,这时又出了新问题。\\

国民军的警卫从大门撤走之后,形势本来已经缓和,我已敢放胆向记者骂国民军了,忽然\xpinyin*{郑孝胥}面容严肃地出现在我面前,问我看过报没有。\\

“看了,没有什么呀!”\\

“皇上看看《顺天时报》。”他拿出报来,指着一条“赤化运动之平民自治歌”标题给我看。这条消息说,冯军入京以后,“赤化主义”乘机活动,最近竟出现数万张传单,主张“不要政府真自治、不要法律大自由”云云。那时我从郑、陈、庄诸人的嘴里和《顺天时报》上,常听到和看到什么共产党是过激主义、赤化主义,赤化、过激就是洪水猛兽、共产共妻,\xpinyin*{冯玉祥}的军队就和赤化过激有关,等等的鬼话。现在根据\xpinyin*{郑孝胥}的解释,那是马上要天下大乱的,“赤化主义”对我下毒手,则更无疑问。\\

我被\xpinyin*{郑孝胥}的话正闹得心惊胆战,愁容满面的\xpinyin*{罗振玉}出现了。我一向很重视\xpinyin*{罗振玉}从日本方面得来的消息。他这次报告我说,日本人得到情报,\xpinyin*{冯玉祥}和“过激主义”分子将对我有不利行动。“现在冯军占了颐和园,”他说,“出事可能就在这一两天。皇上要趁早离开这里,到东交民巷躲避一下才好。”\\

这时\xpinyin*{庄士敦}也来了,带来了外国报上的消息,说\xpinyin*{冯玉祥}要第三次对北京采取行动。\\

这样一来,我沉不住气了,连\xpinyin*{陈宝琛}也着了慌。\xpinyin*{陈宝琛}同意了这个意见:应该趁\xpinyin*{冯玉祥}的军队不在的时候,抓机会躲到东交民巷去,先住进德国医院,因为那位德国大夫是认识我的。我和陈、庄二师傅悄悄地商议了一个计策,这个计策不但要避免民国当局知道,也要防备着我的父亲。\\

我们按照密议的计划进行。第一步,我和陈师傅同出,去探望比我晚几天出宫的住在麒麟碑胡同的\xpinyin*{敬懿}、\xpinyin*{荣惠}两太妃,探望完了,依旧回北府,给北府上下一个守信用的印象。这一步我们做到了。第二天,我们打算再进行第二步,即借口去裱褙胡同看一所准备租用的住房,然后从那里绕一下奔东交民巷,先住进德国医院。第三步则是住进使馆。只要到了东交民巷,第三步以及让\xpinyin*{婉容}她们搬来的第四步,就全好办了。但是在执行这第二步计划的时候,我们刚上了汽车,我父亲便派了他的大管家\xpinyin*{张文治},偏要陪我们一起去。我和\xpinyin*{庄士敦}坐在第一辆汽车上,\xpinyin*{张文治}跟在\xpinyin*{陈宝琛}后边,上了另一辆车。\\

“事情有点麻烦。”\xpinyin*{庄士敦}坐进了汽车,皱着眉头,用英文对我说。\\

“不理他!”我满肚子的气,让司机开车。车子开出了北府。我真想一辈子再不进这个门呢。\\

\xpinyin*{庄士敦}认为,不理这个\xpinyin*{张文治}是不行的,总得设法摆脱他。在路上,他想出了个办法:我们先到乌利文洋行停一停,装作买东西,打发\xpinyin*{张文治}口去。\\

乌利文洋行开设在东交民巷西头一入口的地方,是外国人开的出售钟表、相机的铺子。我们到了乌利文,我和\xpinyin*{庄士敦}进了铺子。我看了一样又一样的商品,最后挑了一只法国金怀表,蘑菇了一阵,可是\xpinyin*{张文治}一直等在外面,没有离开的意思。到了这时,\xpinyin*{庄士敦}只好拿出最后一招,对\xpinyin*{张文治}说,我觉得不舒服,要去德国医院看看。\xpinyin*{张文治}狐疑不安地跟我们到了德国医院。到了医院,我们便把他甩在一边。\xpinyin*{庄士敦}向医院的棣柏大夫说明了来意,把我让到一间空病房里休息,\xpinyin*{张文治}一看不是门道,赶紧溜走了。我们知道他必是回北府向我父亲报信去了,\xpinyin*{庄士敦}不敢放松时间,立刻去英国使馆办交涉。谁知他这一去就古无音信,等得我好不心焦。我生怕这时\xpinyin*{张文治}把我父亲引了来,正在焦躁不安的功夫,\xpinyin*{陈宝琛}和\xpinyin*{郑孝胥}相继到了。\\

\xpinyin*{郑孝胥}的日记里,有这样一段记载:\\

\begin{quote}
	壬子初三日。\xpinyin*{韬庵}(\xpinyin*{陈宝琛})、叔言来。昨报载:李煜瀛见\xpinyin*{段祺瑞},争皇室事,李念言:“法国路易十四,英国杀君主,事尤数见,外交于涉必无可虑。”张继出告人曰:“非斩草除根,不了此事。”平民自治歌有曰:“留\xpinyin*{宣统},真怪异,惟一污点尚未去。”余语\xpinyin*{韬庵}曰:“事急矣!”\\

乃定德国医院之策。午后,\xpinyin*{诣}北府,至鼓楼,逢\xpinyin*{韬庵}之马车,曰:“已往苏州胡同矣!”驰至苏州胡同,无所见,余命往德国医院。登楼,唯见上(\xpinyin*{溥仪})及\xpinyin*{韬庵},云\xpinyin*{庄士敦}已往荷兰、英吉利使馆。余定议奉上幸日本使馆,上命余先告日人。即访竹本,告以皇帝已来。竹本白其公使\xpinyin*{芳泽},乃语余:“请皇帝速来。”于是大风暴作,黄沙蔽天,数步外不相见。余至医院,虑汽车或不听命,议以上乘马车;又虑院前门人甚众,乃引马车至后门,一德医持钥从,一看护引上下楼,开后门,登马车,余及一僮骖乘。\\

德医院至日使馆有二道,约里许:一自东交民巷转北,一自长安街转南。\\

余叱御者曰:“再赴日使馆。”御者利北道稍近,驱车过长安街。上惊叫曰:“街有华警,何为出此!”然车已迅驰,余曰:“咫尺即至!马车中安有皇帝?请上勿恐。”既转南至河岸,复奏上曰:“此为使馆界矣!”\\

送入日使馆。竹本、中平迎上入兵营。\xpinyin*{韬庵}亦至。方车行长安街,风沙悍怒,几不能前,昏晦中入室小憩。上曰:“北府人知我至医院耳,\xpinyin*{庄士敦}、\xpinyin*{张文治}必复往寻,宜告之。”余复至医院,摄政王、涛贝勒皆至。因与同来日馆,廷臣奔视者数人。上命余往告\xpinyin*{段祺瑞},命\xpinyin*{张文治}往告\xpinyin*{张作霖}。……\\
\end{quote}

关于\xpinyin*{庄士敦},\xpinyin*{郑孝胥}在日记里只简单地提了一句,原因是他在德国医院没有看见\xpinyin*{庄士敦},\xpinyin*{庄士敦}那时已经带着忿懑到日本使馆去了。我在日本使馆里和这位一去不回的庄师傅相见时,很觉奇怪。他对我解释说:“我到英国公使那里去了,麻克类说那里地方很小,不便招待……既然陛下受到日本公使先生的接待,那是太好了,总之,现在一切平安了。”在那匆匆忙忙之中,我没再细问——既然我保险了,过去的事情我也就没有兴趣再去知道了。后来我才弄明白,引起他忿懑的,并非像他那天和我解释的“麻克类说,那里地方很小,不便招待”,以致有失面子,更不像后来在自己的著作《紫禁城的黄昏》一书中所说,只有日本公使馆才愿意给我以有效保护(也许英国公使馆有这个看法——他在书中是这样说的),而他在这次争夺战中成了败北者,才是使他忿懑的根本原因。\\

\xpinyin*{郑孝胥}对自己在这次出逃中所起的作用,得意极了。这可以从他写的两首七言诗中看出来:\\

\begin{quote}
	十一月初三日奉乘\xpinyin*{舆}幸日本使馆\\

\xpinyin*{陈宝琛}、\xpinyin*{庄士敦}从幸德国医院,孝臀踵至,遂入日本使馆。\\

乘日风兮载云旗,纵横无人神鬼驰,\\

手持帝子出虎穴,青史茫茫无此奇!\\

是日何来蒙古风?天倾地拆见共工,\\

休嗟猛士不可得,犹有人问一秃翁\footnote{见刘邦《大风歌》:“大风起兮云飞扬,威加海内兮归故乡,安得猛士兮守四方?”}。\\
\end{quote}

这位俨然以“猛士”自居的人后来藏了一幅画:在角楼的上空云雾中,有一条张牙舞爪的龙。\xpinyin*{陈宝琛}虔诚地在画上题了“风异”二字,并作诗一首恭维他:“风沙叫啸日西垂,投止何门正此时;写作昌黎诗意读,天昏地黑扈龙移。”\xpinyin*{庄士敦}颇知凑趣,也用英文把事件经过写在上面。\\

让\xpinyin*{郑孝胥}如此得意忘形的原因之一,是他在这场争夺垄断的战斗中,胜过了他的暗中对手\xpinyin*{罗振玉}。罗不但没有赶上这个机会,而且\xpinyin*{竹本大住}这个值钱的关系,也被郑轻轻拿在手里,成了郑的本钱。郑、罗二人之间的冲突,原来是掩盖在他们与王公们的争夺战后面。而从这时起,开始了他们之间的争夺战了。\\

不过\xpinyin*{庄士敦}却在旁不免暗笑。在他的一九三二年出版的书里,他肯定了\xpinyin*{郑孝胥}的日记所叙述的正确性之后说:“不过有一点除外,那就是\xpinyin*{郑孝胥}错误地认为,\xpinyin*{竹本大住}在同意用他自己的住处接待皇帝之前,已经和日本公使商量过了。日本使馆内文武官员之间的关系,并不像其他使馆文武官员之间的关系那么亲近和友好,竹本大佐是否认为自己应当听从日本公使的命令,是大可怀疑的。因此,他并不认为必须把他和\xpinyin*{郑孝胥}先生谈的话向\xpinyin*{芳泽}谦吉先生汇报,而且他也没有这样做。事实上,他本人急于要接待皇帝,不希望日本公使把他的贵客夺走。……”\\

事实上,后来是夺走了。这刚开始不久的争夺战,不仅展开在王公大臣和郑、罗之间,也不仅在郑与罗之间,原来还发生在日本人之间。这一场争夺战中的真正胜利者,有一段谈话刊在第二天的《顺天时报》上:\\

\begin{quote}
	日使对容留逊帝之谈话\\

日本\xpinyin*{芳泽}公使,昨日对于往访记者所谈逊帝\xpinyin*{溥仪}迁入日本使馆之经过,并公使所持之态度如下:\\

上星期六午后三时,忽有某氏(公使不欲宣布其姓名)来访余(公使自称,下同),告以逊帝现已入德国医院,并谓此不过暂时办法,万难期其久居,且于某某方面亦曾恳谈逊帝迁居事,咸以迁居日本使馆为宜,故逊帝遣某来为之先容,万希俯允所请等语。余当时在大体上因无可推辞,然以事出突然,故答以容暂考虑,再为答复等语。某氏辞去约二十分钟,余即接得报告,谓逊帝已至日本兵营,要求与余面会。余当即亲赴兵营迎近,一面为之准备房屋。午后五点迎入本馆后,即派池部书记官赴外交部谒沈次长,说明逊帝突然来馆之始末,并请转达段执政,以免有所误会。当蒙其答复,极为谅解。……\\
\end{quote}
