\fancyhead[LO]{{\scriptsize 1859-1908: 我的家世 · 摄政王监国}} %奇數頁眉的左邊
\fancyhead[RO]{} %奇數頁眉的右邊
\fancyhead[LE]{} %偶數頁眉的左邊
\fancyhead[RE]{{\scriptsize 1859-1908: 我的家世 · 摄政王监国}} %偶數頁眉的右邊
\chapter*{摄政王监国}
\addcontentsline{toc}{chapter}{\hspace{1cm}摄政王监国}
\thispagestyle{empty}
我做皇帝、我父亲做摄政王的这三年间,我是在最后一年才认识自己的父亲的。那是我刚在\xpinyin*{毓}庆宫读书不久,他第一次照章来查看功课的时候。有个太监进来禀报说:“王爷来了。”老师立刻紧张起来,赶忙把书桌整理一下,并且把见王爷时该做什么,指点了给我,然后告诉我站立等候。过了一会,一个头戴花翎、嘴上没胡须的陌生人出现在书房门口,挺直地立在我的面前,这就是我的父亲。我按家礼给他请了安,然后一同落坐。坐好,我拿起书按老师的指示念起来:\\

“\xpinyin*{孟子}见梁惠王,王立于沼上,王立于沼上……”\\

不知怎的,我心慌得很,再也念不下去。梁惠王立于沼上是下不来了。幸好我的父亲原来比我还慌张,他连忙点头,声音含混地说:\\

“好,好,皇帝好,好好地念,念书吧!”说完,又点了一阵头,然后站起来走了。他在我这里一共呆了不过两分钟。\\

从这天起,我知道了自己的父亲是什么样:不像老师,他没胡子,脸上没皱纹,他脑后的花翎子总是跳动。以后他每隔一个月来一次,每次呆的时间也都不过两分钟。我又知道了他说话有点结巴,明白了他的花翎子之所以跳动,是由于他一说话就点头。他说话很少,除了几个“好,好,好”以外,别的话也很难听清楚。\\

我的弟弟曾听母亲说过,\xpinyin*{辛亥}那年父亲辞了摄政王位,从宫里一回来便对母亲说:“从今天起我可以回家抱孩子了!”母亲被他那副轻松神气气得痛哭了一场,后来告诫弟弟:“长大了万不可学阿玛(满族语父亲)那样!”这段故事和父亲自书的对联“有书真富贵,无事小神仙”,虽都不足以证明什么真正的“退隐”之志,但也可以看出他对那三年监国是够伤脑筋的。那三年可以说是他一生最失败的三年。\\

对他说来,最根本的失败是没有能除掉\xpinyin*{袁世凯}。有一个传说,\xpinyin*{光绪}临终时向摄政王托付过心事,并且留下了“杀\xpinyin*{袁世凯}”四字朱谕。据我所知,这场兄弟会见是没有的。摄政工要杀\xpinyin*{袁世凯}为兄报仇,虽确有其事,但是被\ruby{奕劻}{\textcolor{PinYinColor}{I Kuwang}}为首的一班军机大臣给拦阻住了。详情无从得知,只知道最让父亲泄气的是\ruby{奕劻}{\textcolor{PinYinColor}{I Kuwang}}的一番话:“杀\xpinyin*{袁世凯}不难,不过北洋军如果造起反来怎么办?”结果是\xpinyin*{隆裕}太后听从了\xpinyin*{张之洞}等人的主意,叫\xpinyin*{袁世凯}回家去养“足疾”,把他放走了。\\

有位在内务府干过差使的“遗少”给我说过,当时摄政王为了杀\xpinyin*{袁世凯},曾想照学一下\xpinyin*{康熙}皇帝杀大臣\ruby{鳌拜}{\textcolor{PinYinColor}{Oboi}}的办法。\xpinyin*{康熙}的办法是把\ruby{鳌拜}{\textcolor{PinYinColor}{Oboi}}召来,赐给他一个座位,那座位是一个只有三条好腿的椅子,\ruby{鳌拜}{\textcolor{PinYinColor}{Oboi}}坐在上面不提防给闪了一下,因此构成了“君前失礼”的死罪。和摄政王一起制定这个计划的是小恭亲王\ruby{溥伟}{\textcolor{PinYinColor}{Pu Wei}}\footnote{\ruby{溥伟}{\textcolor{PinYinColor}{Pu Wei}}(1880-1937),恭亲王\ruby{奕訢}{\textcolor{PinYinColor}{I Hin}}之孙,\xpinyin*{光绪}二十四年袭王爵,\xpinyin*{辛亥}革命前为禁烟大臣,\xpinyin*{辛亥}后在德帝国主义庇护下寓居青岛,青岛被日本占领后又投靠日本。在此期间与\xpinyin*{升允}等组织宗社党,不断进行复辟活动,“九·一八”事变后出任沈阳四民维持会会长,企图在日本支持下组织“明光帝国”,但不久即被抛弃,拿了日本人赏的一笔钱老死于旅顺。}。\ruby{溥伟}{\textcolor{PinYinColor}{Pu Wei}}有一柄\xpinyin*{咸丰}皇帝赐给他祖父\ruby{奕訢}{\textcolor{PinYinColor}{I Hin}}的白虹刀,他们把它看成太上宝剑一样的圣物,决定由\ruby{溥伟}{\textcolor{PinYinColor}{Pu Wei}}带着这把刀,做杀袁之用。一切计议停当了,结果被\xpinyin*{张之洞}等人拦住了。这件未可置信的故事至少有一点是真的,这就是那时有人极力保护\xpinyin*{袁世凯},也有人企图消灭\xpinyin*{袁世凯},给我父亲出谋划策的也大有人在。\xpinyin*{袁世凯}在\xpinyin*{戊戌}后虽然用大量银子到处送礼拉拢,但毕竟还有用银子消除不了的敌对势力。这些敌对势力,并不全是过去的维新派和帝党人物,其中有和\ruby{奕劻}{\textcolor{PinYinColor}{I Kuwang}}争地位的,有不把所有兵权拿到手誓不甘休的,也有为了其他目的而把希望寄托在倒袁上面的。因此杀\xpinyin*{袁世凯}和保\xpinyin*{袁世凯}的问题,早已不是什么维新与守旧、帝党与后党之争,也不是什么满汉显贵之争了,而是这一伙亲贵显要和那一伙亲贵显要间的夺权之争。以当时的亲贵内阁来说,就分成庆亲王\ruby{奕劻}{\textcolor{PinYinColor}{I Kuwang}}等人的一伙和公爵\ruby{载泽}{\textcolor{PinYinColor}{Dzai Je}}等人的一伙。给我父亲出谋划策以及要权力地位的,主要是后面这一伙。\\

无论是哪一伙,都有一群宗室觉罗、八旗世家、汉族大臣、南北谋士;这些人之间又都互有分歧,各有打算。比如载字辈的泽公,一心一意想把堂叔庆王的总\xpinyin*{揆}夺过来,而醇王府的兄弟们首先所瞩目的,则是\xpinyin*{袁世凯}等汉人的军权。就是向英国学海军的兄弟和向德国学陆军的兄弟,所好也各有不同。摄政王处于各伙人句心斗角之间,一会儿听这边的话,一会儿又信另一边的主意,一会对两边全说“好,好”,过一会又全办不了。弄得各伙人都不满意他。\\

其中最难对付的是\ruby{奕劻}{\textcolor{PinYinColor}{I Kuwang}}和\ruby{载泽}{\textcolor{PinYinColor}{Dzai Je}}。\ruby{奕劻}{\textcolor{PinYinColor}{I Kuwang}}在西太后死前是领衔军机,太后死后改革内阁官制,他又当了内阁总理大臣,这是叫度支部尚书\ruby{载泽}{\textcolor{PinYinColor}{Dzai Je}}最为忿忿不平的。\ruby{载泽}{\textcolor{PinYinColor}{Dzai Je}}一有机会就找摄政王,天天向摄政王揭\ruby{奕劻}{\textcolor{PinYinColor}{I Kuwang}}的短。西太后既搬不倒\ruby{奕劻}{\textcolor{PinYinColor}{I Kuwang}},摄政王又怎能搬得倒他?如果摄政王支持了\ruby{载泽}{\textcolor{PinYinColor}{Dzai Je}},或者摄政王自己采取了和\ruby{奕劻}{\textcolor{PinYinColor}{I Kuwang}}相对立的态度,\ruby{奕劻}{\textcolor{PinYinColor}{I Kuwang}}只要称老辞职,躲在家里不出来,摄政王立刻就慌了手脚。所以在泽公和庆王间的争吵,失败的总是\ruby{载泽}{\textcolor{PinYinColor}{Dzai Je}}。醇王府的人经常可以听见他和摄政王嚷:“老大哥这是为你打算,再不听我老大哥的,老庆就把大清断送啦!”摄政王总是半晌不出声,最后说了一句:“好,好,明儿跟老庆再说……”到第二天,还是老样子:\ruby{奕劻}{\textcolor{PinYinColor}{I Kuwang}}照他自己的主意去办事,\ruby{载泽}{\textcolor{PinYinColor}{Dzai Je}}又算白费一次力气。\\

\ruby{载泽}{\textcolor{PinYinColor}{Dzai Je}}的失败,往往就是\ruby{载沣}{\textcolor{PinYinColor}{Dzai Feng}}的失败,\ruby{奕劻}{\textcolor{PinYinColor}{I Kuwang}}的胜利,则意味着洹上垂钓\footnote{一九零九年\xpinyin*{袁世凯}被清廷罢斥后,息影于彰德迈水(安阳河),表面上不谈政治,曾经著蓑衣竹笠,作渔翁状,驾扁舟一叶,垂竿\xpinyin*{洹}水滨,以示志在山水之间,其实仍与旧部来往不断,尤其是有“军师”\xpinyin*{徐世昌}经常秘密向他报告国事政局,朝廷动向,并得到他暗中部署,因此,武昌事起,就有了\xpinyin*{徐世昌}等联名保举及袁讨价还价的故事。}的\xpinyin*{袁世凯}的胜利。摄政王明白这个道理,也未尝不想加以抵制,可是他毫无办法。\\

后来武昌起义的风暴袭来了,前去讨伐的清军,在满族陆军大臣\xpinyin*{荫昌}的统率下,作战不利,告急文书纷纷飞来。\xpinyin*{袁世凯}的“军师”\xpinyin*{徐世昌}看出了时机已至,就运动\ruby{奕劻}{\textcolor{PinYinColor}{I Kuwang}}、\xpinyin*{那桐}几个军机一齐向摄政王保举\xpinyin*{袁世凯}。这回摄政王自己拿主意了,向“愿以身家性命”为袁做担保的\xpinyin*{那桐}发了脾气,严肃地申斥了一顿。但他忘了\xpinyin*{那桐}既然敢出头保\xpinyin*{袁世凯},必然有恃无恐。摄政王发完了威风,\xpinyin*{那桐}便告老辞职,\ruby{奕劻}{\textcolor{PinYinColor}{I Kuwang}}不上朝应班,前线紧急军情电报一封接一封送到摄政工面前,摄政王没了主意,只好赶紧赏\xpinyin*{那桐}“乘坐二人肩\xpinyin*{舆}”,挽请\ruby{奕劻}{\textcolor{PinYinColor}{I Kuwang}}“体念时艰”,最后乖乖地签发了谕旨:授\xpinyin*{袁世凯}钦差大臣节制各军并委袁的亲信\xpinyin*{冯国璋}\footnote{\xpinyin*{冯国璋}(1857-1919),字华南,河北河间人,在清末亦是协助\xpinyin*{袁世凯}创办北洋军的得力将领。在\xpinyin*{辛亥}革命后成为北洋军阀的直系首领之一,是英美帝国主义的走狗。}、\xpinyin*{段祺瑞}为两军统领。他垂头丧气地回到府邸后,另一伙王公们包围了他,埋怨他先是放虎归山,这回又引狼入室。他后悔起来,就请这一伙王公们出主意。这伙人说,让\xpinyin*{袁世凯}出来也还可以,但要限制他的兵权,不能委派他的旧部\xpinyin*{冯国璋}、\xpinyin*{段祺瑞}为前线军统。经过一番争论之后,有人认为\xpinyin*{冯国璋}还有交情,可以保留,于是\ruby{载洵}{\textcolor{PinYinColor}{Dzai Xun}}贝勒也要求,用跟他有交情的姜桂题来顶替\xpinyin*{段祺瑞}。王公们给摄政王重新拟了电报,摄政王派人连夜把电报送到庆王府,叫\ruby{奕劻}{\textcolor{PinYinColor}{I Kuwang}}换发一下。庆王府回答说,庆王正歇觉,公事等明天上朝再说。第二天摄政王上朝,不等他拿出这一个上谕,\ruby{奕劻}{\textcolor{PinYinColor}{I Kuwang}}就告诉他,头一个上谕当夜就发出去了。\\

我父亲并非是个完全没有主意的人。他的主意便是为了维持皇族的统治,首先把兵权抓过来。这是他那次出使德国从德国皇室学到的一条:军队一定要放在皇室手里,皇族子弟要当军官。他做得更彻底,不但抓到皇室手里,而且还必须抓在自己家里。在我即位后不多天,他就派自己的兄弟\ruby{载涛}{\textcolor{PinYinColor}{Dzai Tao}}做专司训练禁卫军大臣,建立皇家军队。\xpinyin*{袁世凯}开缺后,他代替皇帝为大元帅,统率全国军队,派兄弟\ruby{载洵}{\textcolor{PinYinColor}{Dzai Xun}}为筹办海军大臣,另一个兄弟\ruby{载涛}{\textcolor{PinYinColor}{Dzai Tao}}管军\xpinyin*{谘}处(等于参谋总部的机构),后来我这两位叔叔就成了正式的海军部大臣和军\xpinyin*{谘}府大臣。\\

据说,当时我父亲曾跟王公们计议过,无论\xpinyin*{袁世凯}镇压革命成功与失败,最后都要消灭掉他。如果他失败了,就借口失败诛杀之,如果把革命镇压下去了,也要找借口解除他的军权,然后设法除掉他。总之,军队决不留在汉人手里,尤其不能留在\xpinyin*{袁世凯}手里。措施的背后还有一套实际掌握全国军队的打算。假定这些打算是我父亲自己想得出的,不说外界阻力,只说他实现它的才能,也和他的打算太不相称了。因此,不但跟着\xpinyin*{袁世凯}跑的人不满意他,就连自己的兄弟也常为他摇头叹息。\\

\xpinyin*{李鸿章}的儿子\xpinyin*{李经迈}出使德国赴任之前,到摄政王这里请示机宜,我七叔\ruby{载涛}{\textcolor{PinYinColor}{Dzai Tao}}陪他进宫,托付他在摄政王面前替他说一件关于禁卫军的事,大概他怕自己说还没用,所以要借重一下\xpinyin*{李经迈}的面子。\xpinyin*{李经迈}答应了他,进殿去了。过了不大功夫,在外边等候着的\ruby{载涛}{\textcolor{PinYinColor}{Dzai Tao}}看见\xpinyin*{李经迈}又出来了,大为奇怪,料想他托付的事必定没办,就问\xpinyin*{李经迈}是怎么回事。\xpinyin*{李经迈}苦笑着说:“王爷见了我一共就说了三句话:‘你哪天来的?’我说了,他接着就问:‘你哪天走?’我刚答完,不等说下去,王爷就说:‘好好,好好地干,下去吧!’——连我自己的事情都没说,怎么还能说得上你的事?”\\

我祖母患乳疮时,请中医总不见好,父亲听从了叔叔们的意见,请来了一位法国医生。医生打算开刀,遭到了醇王全家的反对,只好采取敷药的办法。敷药之前,医生点上了酒精灯准备给用具消毒,父亲吓坏了,忙问翻译道:\\

“这这这干么?烧老太太?”\\

我六叔看他这样外行,在他身后对翻译直摇头咧嘴,不让翻给洋医生听。\\

医生留下药走了。后来医生发现老太太病情毫无好转,觉得十分奇怪,就叫把用过的药膏盒子拿来看看。父亲亲自把药盒都拿来了,一看,原来一律原封未动。叔叔们又不禁摇头叹息一番。\\

醇王府的大管事\xpinyin*{张文治}是最爱议论“王爷”的。有一回他说,在王府附近有一座小庙,供着一口井,传说那里住着一位“仙家”。“银锭桥案件”\footnote{银锭桥在北京地安门附近,是\color{NOTEXTColor}\ruby{载沣}{\textcolor{PinYinColor}{Dzai Feng}}每天上朝必经之地。一九一零年\xpinyin*{汪精卫}、\xpinyin*{黄复生}为刺杀\ruby{载沣}{\textcolor{PinYinColor}{Dzai Feng}}秘密埋藏自制炸弹于桥下,因被军警识破,计划未遂。汪、黄被捕后,清廷慑于当时民气,未敢处以极刑,南北议和时即予释放。当时把这案件叫做银锭桥案件。}败露后,王爷有一次经过那个小庙,要拜一拜仙家,感谢对他的庇佑。他刚跪下去,忽然从供桌后面跳出个黄鼠狼来。这件事叫巡警知道了,报了上去,于是大臣们就传说王爷命大,连仙家都受不了他这一拜。\xpinyin*{张文治}说完了故事就揭穿了底细,原来这是王爷叫庙里人准备好的。\\

醇王府的人在\xpinyin*{慈禧}死后都喜欢自称是维新派,我父亲也不例外。提起父亲的生活琐事,颇有不少反对迷信和趋向时新风气的举动。我还听人说过,“老佛爷并不是反对维新的,\xpinyin*{戊戌}以后办的那些事不都是\xpinyin*{光绪}要办的吗?醇亲王也是位时新人物,老佛爷后来不是也让他当了军机吗?”\xpinyin*{慈禧}的维新和洋务,办的是什么,不必说了。关于父亲的维新,我略知一些。他对那些曾被“老臣”们称为奇技淫巧的东西,倒是不采取排斥的态度。醇王府是清朝第一个备汽车、装电话的王府,他们的辫子剪得最早,在王公中首先穿上西服的也有他一个。但是他对于西洋事物真正的了解,就以穿西服为例,可见一斑。他穿了许多天西服后,有一次很纳闷地问我杰二弟:“为什么你们的衬衫那么合适,我的衬衫总是比外衣长一块呢?”经杰二弟一检查,原来他一直是把衬衫放在裤子外面的,已经忍着这股别扭劲好些日子了。\\

此外,他曾经把给祖母治病的巫婆赶出了大门,曾经把仆役们不敢碰的刺猬一脚踢到沟里去,不过踢完之后,脸上却一阵煞白。他反对敬神念佛,但是逢年过节烧香上供却非常认真。他的生日是正月初五,北京的风俗把这天叫做“破五”,他不许人说这两个字,并在日历的这一页上贴上红条,写上寿宇,把坚笔拉得很长。杰二弟问他这是什么意思,他说:“这叫长寿嘛!”\\

为了了解摄政王监国三年的情况,我曾看过父亲那个时候的日记。在日记里没找到多少材料,却发现过两类很有趣的记载。一类是属于例行事项的,如每逢立夏,必“依例剪平头”,每逢立秋,则“依例因分发”;此外还有依例换什么衣服,吃什么时鲜,等等。另一类,是关于天象观察的详细记载和报上登载的这类消息的摘要,有时还有很用心画下的示意图。可以看出,一方面是内容十分贫乏的生活,一方面又有一种对天文的热烈爱好。如果他生在今天,说不定他可以学成一名天文学家。但可惜的是他生在那样的社会和那样的家庭,而且从九岁起便成了皇族中的一位亲王。
