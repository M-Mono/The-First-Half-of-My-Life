\fancyhead[LO]{{\scriptsize 1932-1945: 伪满十四年 · 订立密约以后}} %奇數頁眉的左邊
\fancyhead[RO]{} %奇數頁眉的右邊
\fancyhead[LE]{} %偶數頁眉的左邊
\fancyhead[RE]{{\scriptsize 1932-1945: 伪满十四年 · 订立密约以后}} %偶數頁眉的右邊
\chapter*{订立密约以后}
\addcontentsline{toc}{chapter}{\hspace{1cm}订立密约以后}
\thispagestyle{empty}
早在旅顺的时候,\xpinyin*{郑孝胥}就跟\xpinyin*{本庄}繁谈妥了由我出任执政和他出任国务总理的条件。这件事情,\xpinyin*{郑孝胥}直到\xpinyin*{本庄}繁卸任前夕才让我知道。\\

一九三二年八月十八日,\xpinyin*{郑孝胥}来到勤民楼,拿出一堆文件来对我说:\\

“这是臣跟本应司令官办的一项协定,请上头认可。”\\

我一看这个协定,就火了。\\

“这是谁叫你签订的?”\\

“这都是\xpinyin*{板垣}在旅顺谈好的条件,”他冷冷地回答,“\xpinyin*{板垣}跟上头也早说过。”\\

“\xpinyin*{板垣}跟谁说过?我就没听他说过。就算他说过,你签字之先也要告诉我呀!”\\

“这也是\xpinyin*{板垣}嘱咐的,说恐怕\xpinyin*{胡嗣瑗}他们不识大局,早拿来反而添麻烦。”\\

“究竟是谁当家?是你,是我?”\\

“臣岂敢。这些协定实在是权宜之计,皇上欲求凭借,岂能不许以条件?这原本是既成事实,将来还可以另订条约,规定几年将权益收回。”\\

他说的其实不错,日本在协定中所要的权利,本来是它已到了手的东西。这个协定共有十二条款,另有附则、附表、附属协定,主要内容是:“满洲国”的“国防、治安”全部委托日本旧本管理“满洲国”的铁路、港湾、水路、空路,并可增加修筑旧本军队所需各种物资、设备由“满洲国”负责供应旧本有权开发矿山。资源旧本人得充任“满洲国”官吏旧本有权向“满洲国”移民等等。在这协定中最后规定它将为日后两国间正式条约的基础。\xpinyin*{郑孝胥}说的道理也不错,既然要“凭借”,岂可不付代价?但是尽管事情是如此明白,我却不能不感到气恼。我恼的是\xpinyin*{郑孝胥}过于擅自专断,竟敢任意拿“我的”江山去跟日本人做交易,我也恼日本人的过分讹诈,“皇帝宝座”没给我,反而要去了这么多的东西。\\

我在气恼而又无可奈何之下,追认了既成的事实。\xpinyin*{郑孝胥}拿了我签过字的密约去了,\xpinyin*{胡嗣瑗}照例就跟着走了进来。我把这件事告诉了他,他立刻气忿地说:\\

“\xpinyin*{郑孝胥}真不像话!\xpinyin*{陈宝琛}早说过他惯于慷他人之慨!他如今竟敢如此擅断!”\\

“现在木已成舟!”我颓丧地说。\\

“或许并不尽然,且看东京方面的消息吧。”\\

许多天以前,我们便知道了关东军司令官将要换人和日本要承认“满洲国”的消息。\xpinyin*{胡嗣瑗}非常重视这件事,照他的看法,日本调换关东军司令官,很可能要改变一点态度,应该乘此机会派人到日本去活动一下。他说,不给日本好处是不行的,像矿山、铁路、资源以及国防都可以叫日本经管,但是在官制方面,任免权必须在我。我采纳了他的主意,并且按他的推荐派出了当过律师的林迁琛和台湾人蔡法平,到东京找他的台湾籍朋友许丙,通过许丙找军部上层人物去活动。林、蔡二人在东京见到了陆军总参谋长真崎甚三郎、前天津日本驻屯军司令\xpinyin*{香椎浩平},还有即将继任关东军司令官的武藤信义等人,向他们提出了我的具体要求:\\

\begin{quote}
	一、执政府依组织法行使职权;\\

二、改组国务院,由执政另提任命名单;\\

三、改组各部官制,主权归各部总长,取消总务厅长官制度;\\

四、练新兵,扩编军队;\\

五、立法院克期召集议会,定国体。\\
\end{quote}

这也是\xpinyin*{胡嗣瑗}为我拟定的。照他的意思,并不指望日本全部接受,只要它同意定国体和由我决定官吏的任免,便算达到了目的。但是条件还是多提一些,以备对方还价。\\

过了两天,\xpinyin*{胡嗣瑗}兴致勃勃地告诉我,东京来了好消息。据林、蔡二人的来信说,东京元老派和军部中某些人都同情于我,不满意\xpinyin*{本庄}对我的态度;表示愿意支持我的各项要求。\xpinyin*{胡嗣瑗}说,由此看来,继任的司令官到任后,情形会有变化,我将按规定行使自己的职权,治理自己的国家。但要治理好,非有个听话的总理不能办事。我听他说的有理,便决定把\xpinyin*{郑孝胥}换掉。我和他研究了一下,觉得\xpinyin*{臧式毅}比较合适,如果任命他为总理,他必定会感恩报德,听我指挥的。商量已定,便命\xpinyin*{胡嗣瑗}与许宝衡去找\xpinyin*{臧式毅}谈。\\

\xpinyin*{臧式毅}的态度尚在犹豫,\xpinyin*{郑孝胥}的儿子\xpinyin*{郑垂}来了。\\

“听说上头派人到东京找武藤信义去了。”他站在我面前,没头没脑地来了这么一句。说罢,盯着我,看我的反应。不用说,他是看出了我不想承认这件事的,于是跟着又说下去:“东京在传说着这件事,说上头打算改组国务院。臣听了,不得不跟上头说说。但愿是个谣传。”\\

“你怎么但愿是谣传?”\\

“但愿如此。这个打算是办不到的。即使办到了,一切由满人作主,各部长官也驾驭不了。不管是\xpinyin*{臧式毅}还是谁,全办不了。”\\

“你要说的就是这个吗?”\\

“臣说的是实情……”\\

“说完了你就去吧!”\\

“是”\\

\xpinyin*{郑垂}走了,我独自一人在办公室里生气。过了一会,\xpinyin*{胡嗣瑗}知道了,又翘起了胡子。\\

“郑氏父子,真乃一狼一狈。\xpinyin*{郑垂}尤其可恨。上回\xpinyin*{熙洽}送来红木家具,他劝上头节俭,无非是嫉妒,怕熙治独邀天眷,这次他又提防起\xpinyin*{臧式毅}来了!”\\

“真不是人!”我越听越恨,决心也更大了,便问\xpinyin*{胡嗣瑗},\xpinyin*{臧式毅}那边说好了没有。\\

“他不肯。”\\

事实上,\xpinyin*{臧式毅}比我和\xpinyin*{胡嗣瑗}都明白,没有关东军说话,他答应了只有找麻烦。\\

\xpinyin*{郑孝胥}知道了\xpinyin*{臧式毅}不敢,就更有恃无恐,居然对我使起当年\xpinyin*{奕劻}对付我父亲的办法,以退为进,向我称病请假了。不过他没料到,我有了东京的好消息,也是有恃无恐的。我看他请假,就看做是个机会,毫不挽留地说:\\

“你也到了养老的时候了。我不勉强你,你推荐个人吧。”\\

他的秃头一下子黯然无光了。\\

“臣的意思,是养几天病。”\\

“那,也好。”\\

\xpinyin*{郑孝胥}一下去,我立即命\xpinyin*{胡嗣瑗}去找减式毅,让他先代理总理职务,以后再找机会去掉\xpinyin*{郑孝胥}。可是过了五天,不等减式毅表示态度,\xpinyin*{郑孝胥}就销假办公了。\\

\xpinyin*{胡嗣瑗}知道了\xpinyin*{郑孝胥}已回到国务院,对我叹气说:“他用密约换的国务总理大印,自然是舍不得丢了。”言下颇为辛酸。\\

我也有辛酸处,这当然不为总理的那颗印,而是我这执政的权威无论对谁都使不上。这次失败给了我很重要的教训。这是由\xpinyin*{胡嗣瑗}的那句辛酸话启发的。\\

“\xpinyin*{郑孝胥}用密约换得总理大印,密约白白地变成了他的本钱,这真太岂有此理了。密约为什么不能是我的本钱,向日本人换得我的所需呢?”\\

我决定等新的关东军司令官到任时,再亲自提出那五项要求。\xpinyin*{胡嗣瑗}拥护这办法,并且提醒我别忘了请日本人撤换\xpinyin*{郑孝胥}。他是自从\xpinyin*{郑孝胥}上台当总理,就耿耿于怀地打了这个主意的。\\

这是九月上旬的事。九月中旬,日本新任关东军司令官兼第一任驻“满”大使武藤信义来到了长春。十五日这天,在勤民楼内,武藤与\xpinyin*{郑孝胥}签订了《日满议定书》,这就是以那个密约为基础的公开协议。\\

因日本国确认满洲国根据其住民之意旨,自由成立而成一独立国家之事实,因满洲国宣言中华民国所有之国际约款,其应得适用于满洲国者为限,即应尊重之。满洲政府及日本政府为永远巩固满日两国间善邻之关系,互相尊重其领土权,且确保东洋之和平起见,为协定如左:\\

\begin{quote}
	(一)满洲国将来满日两国间,未另订约款之前,在满洲国领土内,日本国或日本国臣民依据既存之日中两国间之条约协定,其他约款及公私契约所有之一切权利利益,即应确认尊重之。\\

(二)满洲国及日本国确认对于缔约国一方之领土,及治安之一切之威胁,同时亦为对于缔约国他方之安宁及存立之威胁,相约两国协同当防卫国家之任,为此所要之日本国军驻扎于满洲国内。\footnote{我手头无原件,这是引用《东方杂志》第29卷第4号上的。}\\
\end{quote}

举行完了仪式,喝过了香摈酒,我就急不可待地跟武藤单独进行了会谈。我这时是信心十足的。因为林廷琛和蔡法平不多天前刚从日本回来,他们告诉我,武藤在东京不但已经同意了我的要求,而且连恢复我的尊号都答应予以考虑哩。\\

武藤是日本大正时代晋升的陆军大将,做过参谋本部次长。教育总监、军事参议官,第一次世界大战率日军占领过苏联的西伯利亚。他这次以大将资格来东北,身兼三职——关东军司令长官(从前都是中将衔)、关东厅长官(“九·一八”事变前日本设在辽东半岛的殖民总督)和“驻满洲国大使”,到任不久就晋升为元帅,是这块土地上的事实上的最高统治者,“满洲国”的太上皇。日本报纸称他为“满洲的守护神”。在我的眼里,这个六十五岁的白发老头,确实像一个神似的那么具有威灵。当他十分有礼貌地向我鞠躬致敬时,我就有了一种得天独厚的感觉。等我把话说完,他很礼貌地回答道:\\

“对于阁下的意见,我必带回去认真地加以研究。”\\

他带走了\xpinyin*{胡嗣瑗}写的那几条要求。可是一天一天过去,不见他的研究结果。\\

按规定,我每月有三次和关东军司令兼大使会见。十天后,我和他第二次会见时,催问他研究的结果,他仍是说:“研究研究。”\\

他每次跟我见面,礼貌总是周到的,向我深深鞠躬,微笑,一口一个“阁下”,并且用一种崇敬神情谈起我的每位祖先,不过就是对我的各项要求绝口不提。如果我把话题转到这方面来,他则顾左右而言他。我被这样置之不理的应付了两次,就再没有勇气问他了。\\

一直到一九三三年七月武藤去世时为止,我和他每次见面只能谈佛学,谈儒学,谈“亲善”。在这期间,我的权威在任何人眼里都没增加,而他的权威在我心里则是日增一日,有增无已。