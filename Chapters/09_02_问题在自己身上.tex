\fancyhead[LO]{{\scriptsize 1955-1959: 接受改造 · 问题在自己身上}} %奇數頁眉的左邊
\fancyhead[RO]{} %奇數頁眉的右邊
\fancyhead[LE]{} %偶數頁眉的左邊
\fancyhead[RE]{{\scriptsize 1955-1959: 接受改造 · 问题在自己身上}} %偶數頁眉的右邊
\chapter*{问题在自己身上}
\addcontentsline{toc}{chapter}{\hspace{1cm}问题在自己身上}
\thispagestyle{empty}
星期日,我们照例洗衣服。我洗完衣服,正好是文体活动开始的时间,我没有心情去玩,就到小图书室,想独自看看书。刚坐下来,就听见外面有人说话:\\

“……你们都不打网球?”\\

“我不会打。你找\ruby{溥仪}{\textcolor{PinYinColor}{Pu I}},他会打。”\\

“他会打可是打不了,他的衣服还不知哪辈子洗完呢!”\\

“近来他洗得快多了。”\\

“我才不信呢!”\\

这可是太气人了。我明明洗完了衣服,而且洗的不比他们少,却还有人不信,好像我天生不能进步一点似的。\\

我找到了球拍,走进院子。我倒不是真想打球,而是要让人看看我是不是洗完衣服了。\\

我走到球场上,没找到刚才说话的人,正好另外有人要打网球,我跟他玩了一场。场外聚了一些人观看。我打得很高兴,出了一身汗。\\

打完球,在自来水管旁洗手时,遇见了所长。星期日遇见所长不是稀有的事,他常常在星期日到所里来。\\

“\ruby{溥仪}{\textcolor{PinYinColor}{Pu I}},你今天有了进步。”\\

“很久没打了。”我有点得意。\\

“我说的是这个,”他指着晒衣绳上的衣服,“由于你有了进步,洗衣服花费的时间不比人多了,所以你能跟别人一样的享受休息,享受文体活动的快乐。”\\

我连忙点头,陪他在院子里走着。\\

“从前,别人都休息,都参加文娱活动去了,你还忙个不了,你跟别人不能平等,心里很委屈,现在你会洗衣服了,这才在这方面有了平等的地位,心里痛快了。这样看来,问题的关键还是在自己身上。用不着担心别人对自己怎样。”\\

他过了一会儿,又笑着说:\\

“第二次世界大战,把你这个‘皇帝’变成了一个囚犯。现在,你的思想上又遇到一场大战。这场大战是要把‘皇帝’变成一个普通劳动者。你已经认识到一些皇帝的本质了,不过,这场战争还没有结束,你心里还没有跟别人平等。应该明白自己呵!”\\

所长走后,我想了许久许久。我心里承认前一半的话:看来问题确实是在我自己身上;我对后一半话却难于承认,难道我还在端皇帝架子吗?\\

可是只要承认了前一半,后一半也就慢慢明白了,因为生活回答了这个问题。正如所长所说,这是一场未结束的“战争”。\\

这一天,我们这一组清除完垃圾(这类的劳动已经比较经常了),回到屋里,生活委员向我们提出批评:\\

“你们洗完手,水门不关,一直在流。这样太不负责任了,下次可要注意。”\\

大李听了,立刻问我:\\

“\ruby{溥仪}{\textcolor{PinYinColor}{Pu I}},是你最后一个洗手的吧?”\\

我想了一想,果然不错。\\

“我大概是忘了关水门了。”\\

“你多喒不忘?”\\

“也有不忘的时候。”\\

有人立刻咯咯地乐起来了。其中一个是老元,他问:\\

“那么说,你还有忘的时候,还有几回没关水门。”\\

我没理他。大李却忿忿地对我说:\\

“你不害臊,还不知道这个习惯是哪儿来的。你这是从前的皇帝习惯,你从前从来也没自己关过水门。连门轴儿你也没摸过,都是别人给你开门,给你关门。你现在进出房门,只是开,从不随手带门。这是皇帝架子仍没放下!”\\

“我想起来了,”老元说,“有时看见你开门推门板,有时用报纸垫着门柄,是什么意思?”\\

“你这是怕脏,是不是?”大李抢着说。\\

“那地方人人摸,不脏吗?”\\

谁知这一句话,引起了好几位伙伴的不满。这个说:“怎么别人不嫌脏,单你嫌脏?”那个说:“应该你讲卫生,别人活该?”这个说:“你是嫌门脏,还是嫌别人脏?”那个说:“你这是不是高人一等?心里把别人都看低了?”……\\

我不得不竭力分辩说,决没有嫌恶别人的意思,但心里不由得挺纳闷,我这是怎么搞的呢?我到底是怎么想的呢?为什么我就跟别人不同?后来又有人提起每次洗澡,我总是首先跳进池子,等别人下去,我就出来了。又有人提起在苏联过年,我总要先吃第一碗饺子。听了这些从来没注意过的琐事,我心中不能不承认大李的分析:\\

“一句话,心里还没放下架子来。”\\

今天想起来,大李实在是我那时的一位严肃的教师。不管当时他是怎么想的,他的话总让我想起许多平常想不到的道理。我终于不得不承认,我遇到的苦恼大半要怪我自己。\\

有一天早晨漱洗的时候,大李关照大家注意,刷牙水别滴在地上,滴了就别忘了擦。因为今天各组联合查卫生,这是竞赛,有一点不干净都扣分。\\

我低头看看脚下,我的牙粉水滴了不少。我觉得并不显眼,未必算什么污点。大李过来看见了,叫我擦掉。我用鞋底蹭了蹭,就算了。\\

到了联合检查卫生的时间,各组的生活组长和学委会的生活委员小瑞逐屋进行了检查,按照会议规定的标准,给各组评定分数。检查到我们这间屋,发现了我没蹭干净的牙粉点,认为是个污点,照章扣了分数。最后比较各组总分,我们这个组成绩还不坏,可是大李并不因此忘掉了那个污点,他带来了一把墩布,进了屋先问我:\\

“你怎么不用墩布擦呢?”\\

“没想到。”\\

“没想到?”他粗声说,“你想到了什么呢?你除了自己,根本不想别的!你根本想不到集体!你脑袋里只有权利,没有义务!”\\

他怒气冲冲地拿起墩布,正待要擦,又改了主意,放下墩布对我说:\\

“你应当自觉一点!你擦!”\\

我顺从地执行了他的命令。\\

自从朝鲜和东北发现了美国的细菌弹,全国展开了爱国卫生运动以来,监狱里每年定期地要搞几次除四害、讲卫生的大规模活动。这种活动给我留下了许多深刻的印象,其中之一,是我和大李在打苍蝇上发生的一件事。\\

他从外面拿来几个新蝇拍。蝇拍不够分配,许多人都争着要分一把。我没有主动去要,但是大李先给了我一把。这是我头一次拿这东西,似乎有点特殊的感觉,老实说,我还没打死过一个苍蝇哩!\\

那时,监狱里的苍蝇已经不多,如果用“新京”的标准来说,就算是已经绝迹了。我找了一阵,在窗户框上发现了一个,那窗户是打开了的,我用蝇拍一挥,把它赶出去了。\\

“你这是干什么?”大李在我身后喊,“你是除四害还是放生?”\\

别人也许以为他是说笑话,其实我是明白他的意思的。我不禁涨红了脸,不自然地说:“谁还放生?”但是心里却也奇怪,我为什么把它赶走了呢?\\

“你不杀生!你怕报应,是吧?”他瞪着眼问我。我自感心虚,嘴上却强硬:\\

“什么报应?苍蝇自己跑啦!”\\

“你自己想想吧!”\\

这天晚上开检讨会的时候,起初没人理会这件事,后来经过大李的介绍,人们知道了我在长春时不准打苍蝇以及指挥众人从猫嘴里抢耗子的故事,全乐开了。乐完了,一齐批评我的迷信思想。我心里不得不接受,嘴里却不由自主地说:\\

“我为什么还迷信?我去年不是打了?”\\

“我想起来了!”老元忍不住笑起来,“你不说去年,我还想不起来。我记得去年你就把蝇拍推让给别人,自己拿张报纸扇呼,苍蝇全给你放走啦!”\\

在哄笑中只有大李板着脸,用十分厌恶的声调说:\\

“别人放生是什么意思,我不敢说,你放生我可明白,这完全是自私,为了取得代价,叫佛爷保佑你。别人都可以死光,惟独要保护你一个人。因为你把自己看得最贵重。”\\

“你说的太过分了。”我抗议说。\\

“\ruby{溥仪}{\textcolor{PinYinColor}{Pu I}}有时倒是很自卑。”老元说。\\

“是呀!”我接口说,“我从哪一点看自己也不比别人高。”\\

“也许,有时自卑,”大李表示了同意,可是接着又说,“有时你又把自己看得比别人高,比别人重要。你这是怎么搞的,我也不明白。”\\

我后来终于逐渐明白了。因为我是高高在上地活了四十年,一下子掉在地平线上的,所以总是不服气、生气、委屈的慌;又因为许多事实告诉我,我确实不如人,所以又泄气、恼恨、自卑和悲哀。总之,架子被打掉了,标尺还留着。我所以能明白这个道理,是因为后来发现了不能用我的标尺去衡量的人。在明白这一点之前,在跟大李相处的这段时间中,我只懂得了所长的话,渐渐明白了自己在与别人的关系上,是不平等的,就因为如此,我才引起别人的反感,得不到别人平等的看待或尊重,总之,问题是在自己身上。而当我亲眼看到了那些不可衡量的人,并且得到了他们的恩惠,我就更明白自己是什么样的人了。