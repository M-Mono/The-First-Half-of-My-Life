\fancyhead[LO]{{\scriptsize 1908-1917: 我的童年 · 母子之间}} %奇數頁眉的左邊
\fancyhead[RO]{} %奇數頁眉的右邊
\fancyhead[LE]{} %偶數頁眉的左邊
\fancyhead[RE]{{\scriptsize 1908-1917: 我的童年 · 母子之间}} %偶數頁眉的右邊
\chapter*{母子之间}
\addcontentsline{toc}{chapter}{\hspace{1cm}母子之间}
\thispagestyle{empty}
我入宫过继给\xpinyin*{同治}和\xpinyin*{光绪}为子,\xpinyin*{同治}和\xpinyin*{光绪}的妻子都成了我的母亲。我继承\xpinyin*{同治}兼祧\xpinyin*{光绪},按说正统是在\xpinyin*{同治}这边,但是\xpinyin*{光绪}的皇后——\xpinyin*{隆裕}太后不管这一套。她使用太后权威,把敢于和她争论这个问题的\xpinyin*{同治}的\xpinyin*{瑜}、\xpinyin*{珣}、\xpinyin*{瑨}三妃,打入了冷宫,根本不把她们算做我的母亲之数。\xpinyin*{光绪}的瑾妃也得不到庶母的待遇。遇到一家人同座吃饭的时候,\xpinyin*{隆裕}和我都坐着,她却要站着。直到\xpinyin*{隆裕}去世那天,\xpinyin*{同治}的三个妃和瑾妃联合起来找王公们说理,这才给她们明确了太妃的身份。从那天起,我才管她们一律叫“皇额娘”。\\

我虽然有过这么多的母亲,但并没有得过真正的母爱。今天回想起来,她们对我表现出的最大关怀,也就是前面说过的每餐送菜和听太监们汇报我“进得香”之类。\\

事实上我小时候并不能“进得香”。我从小就有胃病,得病的原因也许正和“母爱”有关。我六岁时有一次栗子吃多了,撑着了,有一个多月的时间\xpinyin*{隆裕}太后只许我吃糊米粥,尽管我天天嚷肚子饿,也没有人管。我记得有一天游中南海,太后叫人拿来干馒头,让我喂鱼玩。我一时情不自禁,就把馒头塞到自己嘴里去了。我这副饿相不但没有让\xpinyin*{隆裕}悔悟过来,反而让她布置了更严厉的戒备。他们越戒备,便越刺激了我抢吃抢喝的欲望。有一天,各王府给太后送来贡品\footnote{每月初一、十五各王府按例都要送食品给太后。},停在西长街,被我看见了。我凭着一种本能,直奔其中的一个食盒,打开盖子一看,食盒里是满满的酱肘子,我抓起一只就咬。跟随的太监大惊失色,连忙来抢。我虽然拼命抵抗,终于因为人小力弱,好香的一只肘子,刚到嘴又被抢跑了。\\

我恢复了正常饮食之后,也常免不了受罪。有一次我一连吃了六个春饼,被一个领班太监知道了。他怕我被春饼撑着,竟异想天开地发明了一个消食的办法,叫两个太监左右提起我的双臂,像砸夯似的在砖地上蹾了我一阵。过后他们很满意,说是我没叫春饼撑着,都亏那个治疗方法。\\

这或许被人认为是不通情理的事情,不过还有比这更不通情理的哩。我在八九岁以前,每逢心情急躁,发脾气折磨人的时候,我的总管太监张谦和或者阮进寿就会做出这样的诊断和治疗:“万岁爷心里有火,唱一唱败败火吧。”说着,就把我推进一间小屋里——多数是\xpinyin*{毓庆宫}里面的那间放“毛凳儿”的屋子,然后倒插上门。我被单独禁闭在里面,无论怎么叫骂,踢门,央求,哭喊,也没有人理我,直到我哭喊够了,用他们的话说是“唱”完了,“败了火”,才把我释放出来。这种奇怪的诊疗,并不是太监们的擅自专断,也不是\xpinyin*{隆裕}太后的个人发明,而是皇族家庭的一种传统,我的弟弟妹妹们在王府里,都受过这样的待遇。\\

\xpinyin*{隆裕}太后在我八岁时去世。我对她的“慈爱”只能记得起以上这些。\\

和我相处较久的是四位太妃。我和四位太妃平常很少见面。坐在一起谈谈,像普通人家那样亲热一会,根本没有过。每天早晨,我要到每位太妃面前请安。每到一处,太监给我放下黄缎子跪垫,我跪了一下,然后站在一边,等着太妃那几句例行公事的话。这时候太妃正让太监梳着头,一边梳着一边问着:“皇帝歇得好?”“天冷了,要多穿衣服。”“书念到哪儿啦?”全是千篇一律的枯燥话,有时给我一些泥人之类的玩意儿,最后都少不了一句:“皇帝玩去吧!”一天的会面就此结束,这一天就再也不见面了。\\

太后太妃都叫我皇帝,我的本生父母和祖母也这样称呼我。其他人都叫我皇上。虽然我也有名字,也有乳名,不管是哪位母亲也没有叫过。我听人说过,每个人一想起自己的乳名,便会联想起幼年和母爱来。我就没有这种联想。有人告诉我,他离家出外求学时,每逢生病,就怀念母亲,想起幼年病中在母亲怀里受到的爱抚。我在成年以后生病倒是常事,也想起过幼年每逢生病必有太妃的探望,却丝毫引不起我任何怀念之情。\\

我在幼时,一到冷天,经常伤风感冒。这时候,太妃们便分批出现了。每一位来了都是那几句话:“皇帝好些了?出汗没有?”不过两三分钟,就走了。印象比较深的,倒是那一群跟随来的太监,每次必挤满了我的小卧室。在这几分钟之内,一出一进必使屋里的气流发生一次变化。这位太妃刚走,第二位就来了,又是挤满一屋子。一天之内就四进四出,气流变化四次。好在我的病总是第二天见好,卧室里也就风平浪静。\\

我每次生病,都由永和宫的药房煎药。永和宫是\xpinyin*{端康}太妃住的地方,她的药房比其他太妃宫里的药房设备都好,是继承了\xpinyin*{隆裕}太后的。\xpinyin*{端康}太妃对我的管束也比较多,俨然代替了\xpinyin*{隆裕}原先的地位。这种不符清室先例的现象,是出于\xpinyin*{袁世凯}的干预。\xpinyin*{隆裕}去世后,\xpinyin*{袁世凯}向清室内务府提出,应该给同、光的四妃加以晋封和尊号,并且表示承认瑾妃列四妃之首。\xpinyin*{袁世凯}为什么管这种闲事,我不知道。有人说这是由于瑾妃娘家兄弟志钅奇的活动,也不知确否。我只知我父亲\ruby{载沣}{zǎi fēng}和其他王公们都接受了这种干预,给\xpinyin*{瑜}、\xpinyin*{珣}皇贵妃上了尊号(\xpinyin*{敬懿}、\xpinyin*{庄和})\xpinyin*{瑨}、瑾二贵妃也晋封为皇贵妃(尊号为\xpinyin*{荣惠}、\xpinyin*{端康});\xpinyin*{端康}成了我的首席母亲,从此,她对我越管越严,直到发生了一次大冲突为止。\\

我在四位母亲的那种“关怀”下长到十三四岁,也像别的孩子那样,很喜欢新鲜玩意。有些太监为了讨我高兴,不时从外面买些有趣的东西给我。有一次,一个太监给我制了一套民国将领穿的大礼服,帽子上还有个像白鸡毛掸子似的翎子,另外还有军刀和皮带。我穿戴起来,洋洋得意。谁知叫\xpinyin*{端康}知道了,她大为震怒,经过一阵检查,知道我还穿了太监从外面买来的洋袜子,认为这都是不得了的事,立刻把买军服和洋袜子给我的太监李长安、李延年二人叫到永和宫,每人责打了二百大板,发落到打扫处去充当苦役。发落完了太监,又把我叫了去,对我大加训斥:“大清皇帝穿民国的衣裳,还穿洋袜子,这还像话吗?”我不得已,收拾起了心爱的军服、洋刀,脱下洋袜,换上裤褂和绣着龙纹的布袜。\\

如果\xpinyin*{端康}对我的管教仅限于军服和洋袜子,我并不一定会有后来的不敬行为。因为这类的管教,只能让我更觉得自己与常人不同,更能和\xpinyin*{毓庆宫}的教育合上拍。我相信她让太监挨一顿板子和对我的训斥,正是出于这个教育目的。但这位一心一意想模仿\xpinyin*{慈禧}太后的瑾妃,虽然她的亲姐姐珍妃死于\xpinyin*{慈禧}之手,\xpinyin*{慈禧}仍然被她看做榜样。她不仅学会了毒打太监,还学了派太监监视皇帝的办法。她发落了我身边的李长安、李延年这些人之后,又把她身边的太监派到我的养心殿来伺候我。这个太监每天要到她那里报告我的一举一动,就和西太后对待\xpinyin*{光绪}一样。不管她是什么目的,这大大伤害了皇帝的自尊心。我的老师\xpinyin*{陈宝琛}为此忿忿不平,对我讲了一套嫡庶之分的理论,更加激起了我憋在心里的怒气。\\

过了不久,大医院里一个叫范一梅的大夫被\xpinyin*{端康}辞退,便成了爆发的导火线。范大夫是给\xpinyin*{端康}治病的大夫之一,这事本与我不相干,可是这时我耳边又出现了不少鼓动性的议论。陈老师说:“身为太妃,专擅未免过甚。”总管太监张谦和本来是买军服和洋袜子的告发人,这时也变成了“帝党”,发出同样的不平之论:“万岁爷这不又成了\xpinyin*{光绪}了吗?再说大医院的事,也要万岁爷说了算哪!连奴才也看不过去。”听了这些话,我的激动立刻升到顶点,气冲冲地跑到永和宫,一见\xpinyin*{端康}就嚷道:\\

“你凭什么辞掉范一梅?你太专擅了!我是不是皇帝?谁说了话算数?真是专擅已极!……”\\

我大嚷了一通,不顾气得脸色发白的\xpinyin*{端康}说什么,一甩袖子跑了出来。回到\xpinyin*{毓庆宫},师傅们都把我夸了一阵。\\

气急败坏的\xpinyin*{端康}太妃没有找我,却叫人把我的父亲和别的几位王公找了去,向他们大哭大叫,叫他们给拿主意。这些王公们谁也没敢出主意。我听到了这消息,便把他们叫到上书房\footnote{上书房是皇子念书的地方,在乾清宫左边。}里,慷慨激昂地说:\\

“她是什么人?不过是个妃。本朝历代从来没有皇帝管妃叫额娘的!嫡庶之分要不要?如果不要,怎么\xpinyin*{溥杰}不管王爷的侧福晋叫一声呢?凭什么我就得叫她,还要听她的呢?……”\\

这几位王公听我嚷了一阵,仍然是什么话也没说。\\

\xpinyin*{敬懿}太妃是跟\xpinyin*{端康}不和的。这时她特意来告诉我:“听说永和宫要请太太、奶奶\footnote{满族称祖母为太太,母亲为奶奶。}来,皇帝可要留神!”\\

果然,我的祖母和母亲都被\xpinyin*{端康}叫去了。她对王公们没办法,对我祖母和母亲一阵叫嚷可发生了作用,特别是祖母吓得厉害,最后和我母亲一齐跪下来恳求她息怒,答应了劝我赔不是。我到了永和宫配殿里见到了祖母和母亲,听到正殿里\xpinyin*{端康}还在叫嚷,我本来还要去吵,可是禁不住祖母和母亲流着泪苦苦哀劝,结果软了下来,答应了她们,去向\xpinyin*{端康}赔了不是。\\

这个不是赔得我很堵心。我走到\xpinyin*{端康}面前,看也没看她一眼,请了个安,含含糊糊地说了一句“皇额娘,我错了”,就又出来了。\xpinyin*{端康}有了面子,停止了哭喊。过了两天,我便听到了母亲自杀的消息。\\

据说,我母亲从小没受别人申斥过一句。她的个性极强,受不了这个刺激。她从宫里回去,就吞了鸦片烟。后来\xpinyin*{端康}担心我对她追究,从此便对我一变过去态度,不但不再加以管束,而且变得十分随和。于是紫禁城里的家庭恢复了往日的宁静,我和太妃们之间也恢复了母子关系。然而,却牺牲了我的亲生母亲。
