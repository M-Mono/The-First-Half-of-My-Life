\fancyhead[LO]{{\scriptsize 1932-1945: 伪满十四年 · 家门以内}} %奇數頁眉的左邊
\fancyhead[RO]{\thepage} %奇數頁眉的右邊
\fancyhead[LE]{\thepage} %偶數頁眉的左邊
\fancyhead[RE]{{\scriptsize 1932-1945: 伪满十四年 · 家门以内}} %偶數頁眉的右邊
\chapter*{家门以内}
\addcontentsline{toc}{chapter}{\hspace{11mm}家门以内}
%\thispagestyle{empty}
我不能过问政事,不能随便外出走走,不能找个“大臣”谈谈,所以当关东军那边的电流通不过来的时候,我就无事可干。我发展了迟眠晏起的习惯,晚上总要在后半夜,甚至过三点才睡,早晨要十一点才起。每日两餐,早餐在中午十二点至一两点,晚饭在九至十一点,有时是十二点。四点到五六点睡中觉。我的日常生活,除了吃睡之外,用这八个字就可以概括了,即:打骂、算挂、吃药、害怕。\\

这四样东西是相互有着关联的。随着日本崩溃的迹象越来越明显,我越是恐怖,就怕日本在垮台之前,会杀我灭口。在这种心理支配下,我对日本人是伺候颜色、谄媚逢迎,对家门以内则是脾气日趋暴躁,动辄打人骂人。我的迷信思想也更加发展,终日吃素念经,占卜打卦,求神佛保佑。在这种精神不宁和不正常的生活习惯下,本来就糟踏坏了的身体,这时越发虚弱,因此又挤命打针吃药。总而言之,这四样东西构成了我昏天昏地、神神颠颠的生活。\\

我的残暴多疑,早在紫禁城时代就种下了根子,到了天津,向前发展了一步。在天津,我给佣人们立下了这样的“家规”:\\

\begin{quote}
	\begin{itemize}
		\item 不准彼此随便说话,以防结党营私。\\
		\item 不准互相包庇袒护。\\
		\item 不准舞弊赚钱。\\
		\item 当同事犯有过错时须立即报告。\\
		\item 上级对下级犯过的人,须在发现之后立即加以责打。\\
		\item 如果放松看管。罪加一等。\\
	\end{itemize}
\end{quote}

到东北后,又附加了一项誓词:\\

\begin{quote}
	“如有违背,甘心承受天罚,遭受天打雷轰。”\\
\end{quote}

在我的大门内,我的残忍暴虐行为,后来发展到经常打人,甚至于使用刑具。打人的花样很多,都是叫别人替我执行。受到这种委派的人往往不是一个两个,而是全体在场的人。他们在动手的时候,必须打得很重,否则便可能引起我的疑心,认为他们朋比为奸,因此临时转移目标,改打不肯使劲打人的人。\\

我的打骂对象除了我的妻子、弟弟和妹夫之外,几乎包括家里的一切人。那时我有几个侄子,在宫里念书,同时又是陪我说话、伺候我的人,是我培养的亲信,可是我一样地打骂他们。他们那时最怕我说的一句话,就是:“叫他下去!”意思就是到楼下去挨打。\\

我这些举动,除了说明我的蛮横、狂妄、暴虐和喜怒无常的可耻性格之外,实在不能说明别的问题。有一次,一个童仆在我的椅子上坐了一下,别人根据我订立的家规,把他告发了。我认为这是冒犯了我,立即命人重重责打了他一顿。其实这个宝座,不是我也坐得心惊肉跳吗?\\

在长春,我因患痔疮,买了不少坐药。有个小侄子见到这种药很稀奇,无意中说了一句,“很像个枪弹”,立刻触了我的忌讳,“这不是咒我吃枪弹吗!”在我的授意之下,其他的侄子们给了他一顿板子。\\

在我这种统治下,境遇最惨的是一批重仆。这是从长春的一个所谓慈善团体要来的孤儿,大约有十几个,他们大都是父母被日本人杀害之后遗下来的。日本人怕这些后代记仇,便叫汉奸政权用慈善团体名义收养起来,并给他们改了姓名,进行奴化教育,用奴役劳动摧残他们。当他们听说被送到我这里来的时候,有的还抱过很大希望,认为生活一定比在慈善会里好些,事实上不但没有什么改善,反而更糟。他们在这里,吃的是最坏的高粱米,穿的是破烂不堪的衣服,每天要干十五六小时的活,晚上还要坐更守夜。冬天,因为又冷又饿,又累又困,有的在打扫工作中,不知不觉地伏在暖气上睡去,以致烤得皮焦肉烂。他们挨打更是经常的。干活睡觉要挨打,扫地不干净要挨打,说话大声要挨打。心里不高兴的随侍,还常拿他们出气。为了处罚他们,负责管理他们的随侍,特地设了禁闭室。这些孤儿在种种折磨下,长到十七八岁,还矮小得像十来岁的孩子。\\

有一个叫\xpinyin*{孙博元}的童仆,就是被生生折磨死的。这孩子在伪宫里实在受不了,他幻想着外面世界也许好些,屡次想找机会逃走。第一次逃走被发觉抓回来,挨了一顿毒打。第二次又逃走,他以为通暖气管的地道通到外面,便钻了进去,可是在里面转来转去,转了两天两夜也没找到出口。他又渴又饿,不得不出来找水喝,因此被人发现又抓住了。我听到了随侍的报告,便命令:“让他先吃点东西,然后再管教他!”可是这时他早被随侍们管教得\xpinyin*{奄奄}一息了。我听说他快死了,吓得要命,怕他死了变成冤鬼前来索命,便命令把医生叫来抢救,可是已经来不及了。这孩子终于在我的“家规”下,丧失了幼小的生命!\\

这件事发生后,我并没有受到良心的责备,只是由于害怕因果报应,花了几天功夫在佛坛前磕头念经,超度亡魂,同时责令打过他的随侍们,在半年以内,每天要用竹板打自己的手心,以示仟悔。好像这样措置之后,我就可以摆脱一切干系似的。\\

我对仆人们的苛刻待遇,后来竟因神经过敏而发展到极无聊的地步。我经常像防贼似地防备厨子买菜时嫌我几角钱。我甚至于派人秘密跟踪,看他是怎么买的,或者向我的妹妹们调查,肉多少钱一斤,鸡多少钱一只。有时候认为菜做的不好,或者发现有点什么脏东西,立刻下令罚钱。当然有时因为做的好,也赏钱。我在自己的屋子外面无权无力,只能在日本人决定的法令上划可,在自己的屋子里面,却作威作福,我行我法。\\

到了伪满末期,日本的败象越来越明显。无论是无线电中的盟国电台消息,还是\ruby{吉冈}{\textcolor{PinYinColor}{よしおか}}\ruby{安直}{\textcolor{PinYinColor}{やすなお}}流露出的颓丧心情,都逐日加深着我的末日情绪。我的脾气变得更坏了,在家门里发的威风也更凶了。一九四四年初,一位按例来给我祝寿的长辈,竟平白无辜地成了我发威风的对象。\\

那天为了庆祝我的生日,宫内府弄了一个滑冰晚会,找了些会滑冰的人来表演。在大家看滑冰的时候,这位关内来的长辈看见了\ruby{吉冈}{\textcolor{PinYinColor}{よしおか}}\ruby{安直}{\textcolor{PinYinColor}{やすなお}}和日本官吏们,为了表示礼貌,在我的面前跟他们招呼为礼。这样的事在一般人看来本是极为平常的,可是在当时我那一群人眼中却成了“大不敬”的失仪行为。因为“天子”乃是“至尊”,在“天子”面前没有谁更尊贵的,所以任何人不能有互相致敬、受礼的表示。家里的人都知道我是绝对不容许有这类事发生的,而且按照我的教诲,如有人发现任何不敬行为,不向我报告就要算做不忠。因此,这件当时并未被我发现的“不敬”行为,过了不大功夫,即在滑冰表演结束后举行家宴的时候,就有个侄子在宴席上报告了我。我这时正在高兴,加以想到他是个老人,不想深究,便示意叫这忠心的侄子退下。却不料那位刚犯了“大不敬”的老人,现在又犯了好奇心,想知道那个侄子俯在我耳边说什么,便探过头去问那个侄子,又一次犯了“大不敬”。我不禁勃然大怒,猛地拍了一下桌子,喝道:“给你脸,不作脸,你还有个够吗?”这位老人这才明白了他的“过失”,吓得面如土色,身不由己地向我双膝跪倒,诚惶诚恐地低下头来。而我却越想越气,索性离了席,对他嚷叫起来:“你的眼里还有我吗?你眼里没有我,就是没有德宗景皇帝,就是没有穆宗毅皇帝!……”弄得全场鸦雀无声,可谓大煞风景。\\

我所以如此气恼,说穿了不过是因为被伤害了虚荣心。我甚至觉得这个老人竟不如日本人。连日本人对我使颜色都是背着人进行的,可他倒当着人的面冒犯我!\\

到长春之后,我看了大量的迷信鬼神书,看得入了迷。我在书上看了什么六道轮回,说一切生物都有佛性,我就生怕吃的肉是死去的亲人变的,所以除了每天早晚念两次经外,每顿饭又加念一遍“往生咒”,给吃的肉主超生。开头是在开饭的时候,当着人面,我自己默默地念,后来我索性让人先出去,等我一个人嘟嘟囔囔地念完,再让他们进来。所以后来每逢吃饭,他们便自动等在外面,听我嘟囔完了才进来。记得有一次,我正在同德殿的地下防空洞里吃饭,忽然响起了空袭警报,我念了咒还不算,还把要吃的一个鸡蛋拿起来,对它磕三个头,才敢把这个“佛性”吃下肚去。这时,我已经索性吃素,除鸡蛋外,荤腥一概不动。我不许人们打苍蝇,只许向外轰。我知道苍蝇会带病菌传病给人,苍蝇落过的饭菜,我一律不吃,如果在我的嘴唇上落一下,我就拿酒精棉花擦一下(我身上总带着一个盛酒精药棉的小铁盒),如果发现菜里有苍蝇腿要罚厨师的钱,尽管如此,我却不准任何人打死一只苍蝇。有一次我看见一只猫抓住了一只老鼠,为了救这只老鼠,我就下令全体家人一齐出动去追猫。\\

我越看佛书越迷,有时做梦,梦见游了地狱,就越发相信。有一次,我从书上看到,念经多日之后,佛就会来,还要吃东西。我便布置出一间屋子,预备了东西。在念过经之后,对众人宣布道:佛来了!我便跪着爬进屋去。当然里面是空的,可是因为我自己也相信了自己的胡说人道,所以战战兢兢地向空中碰起头来。\\

我家里的人都叫我弄得神神颠颠的。在我的影响下,家中终日佛声四起,木鱼铜磬响声不绝,像居身于庙里一样。\\

我还常常给自己问卜算卦,而且算起来就没完,不得上古之卦,决不罢休。后来我日益害怕关东军害我,发展到每逢\ruby{吉冈}{\textcolor{PinYinColor}{よしおか}}找我一次,我要打卦卜一次吉凶。避凶趋吉,几乎成了支配我一举一动的中心思想。弄得行路、穿衣、吃饭,脑子里也是想着哪样吉,哪样不吉。至于吉凶的标准,也无一定之规,往往是见景生情,临时自定。比如走路时,前面有块砖头,心里便规定道:“从左面走过去,吉祥,从右边,不吉祥。”然后便从左面走过去。什么迈门坎用左腿右腿,夹菜是先夹白先夹绿,真是无穷无尽。\xpinyin*{婉容}也随我入了迷,她给自己规定,对于认为不吉的,就眨巴眨巴眼,或是吐吐唾沫。后来弄成了习惯,时常无缘无故地眨巴一阵眼,或者是嘴里“啐啐啐”连着出声,就像患了精神病似的。\\

在我的教育管制之下,我的侄子们——二十左右岁的一群青年,个个像苦修的隐士,有的每天“人定”,有的新婚之后不回家,有的在床头上悬挂“白骨图”,有的终日掐诀念咒,活像见了鬼似的。\\

我还每天“打坐”。“打坐”时,不准有一点声音。这时所有的人连大气都不敢出。我的院子里养了一只大鹤,它不管这套,高起兴来就要叫一下子。我交代给仆人负责,如果鹤叫一声,就罚他五角钱。仆人们被罚了不少钱之后,研究出一个办法:鹤一伸脖子他就打它脖子一下,这样就不叫了。\\

因为怕死,所以最怕病。我嗜药成瘾,给了我的家人和仆人不少罪受,也给自己找了不少罪受。我嗜药不仅是吃,而且还包括收藏。中药有药库,西药有药房。我有时为了菜的口味差一些,硬叫扣出厨子几角钱来,但为买些用不着的药品,可以拿出几千元、几万元去向国外订购。我的一些侄子,上学之外要为我管药房、药库。他们和我专雇的医生每天为我打补针,总要忙上几小时。\\

从前我在紫禁城里时常“疑病”,现在用不到疑心,我真的浑身是病了。记得有一次例行“巡幸”,到安东去看日本人新建的水力发电站。到了那里,由于穿着军服,还要在鬼子面前撑着架子,走了不多远,我就喘得透不过气来,回来的时候,眼看就要昏过去了,随行的侄子们和医生赶快抢着给我打强心剂和葡萄糖,这才把我抢救过来。\\

这种虚弱的身体,加上紧张的心情,让我总觉得死亡迫在眉睫。\\

有一天,我到院子里去打网球,走到院墙边,忽然看到墙上有一行粉笔写的字:\\

“日本人的气,还没受够吗?”\\

看到这行粉笔字,我连网球也忘记打了,赶紧叫人擦了去。我急忙口到我的卧室,心里砰砰跳个不停,觉得虚弱得支持不住了。\\

我怕日本人发现这行粉笔字之后,会不分青红皂白地在我这“内廷”来个“大检举”,那不定会闹成什么样子。令我更惊慌的是,显然在我这内延之中,有了“反满抗日分子”。他敢于在大庭广众之下写字,就不敢杀我吗?\\

由于我整天昏天黑地、神神颠颠,对家庭生活更没有一点兴趣。我先后有过四个妻子,按当时的说法,就是一个皇后,一个妃,两个贵人。如果从实质上说,她们谁也不是我的妻子,我根本就没有一个妻子,有的只是摆设。虽然她们每人的具体遭遇不同,但她们都是同样的牺牲品。\\

长时期受着冷淡的\xpinyin*{婉容},她的经历也许是现代新中国的青年最不能理解的。她如果不是在一出生时就被决定了命运,也是从一结婚就被安排好了下场。我后来时常想到,她如果在天津时能像\xpinyin*{文绣}那样和我离了婚,很可能不会有那样的结局。当然,她究竟和\xpinyin*{文绣}不同。在\xpinyin*{文绣}的思想里,有一个比封建的身分和礼教更被看重的东西,这就是要求有一个普通人的家庭生活。而\xpinyin*{婉容},却看重了自己的“皇后”身分,所以宁愿做个挂名的妻子,也不肯丢掉这块招牌。\\

自从她把\xpinyin*{文绣}挤走之后,我对她便有了反感,很少和她说话,也不大留心她的事情,所以也没有从她嘴里听说过她自己的心情、苦闷和愿望。只知道后来她染上了吸毒(鸦片)的嗜好,有了我所不能容忍的行为。\\

“八·一五”后她和我分手时,烟瘾已经很大,又加病弱不堪,第二年就病死在吉林了。\\

一九三七年,我为了表示对\xpinyin*{婉容}的惩罚,也为了有个必不可少的摆设,我另选了一名牺牲品——\xpinyin*{谭玉龄},她经北京一个亲戚的介绍,成了我的新“贵人”。她原姓\ruby{他他拉}{\textcolor{PinYinColor}{Tatara}}氏,是北京一个初中的学生,和我结婚时是十七岁。她也是一名挂名的妻子,我像养一只鸟儿似地把她养在“宫”里,一直养到一九四二年死去为止。\\

她的死因,对我至今还是一个谜。她的病,据中医诊断说是伤寒,但并不认为是个绝症。后来,我的医生\xpinyin*{黄子正}介绍市立医院的日本医生来诊治。\ruby{吉冈}{\textcolor{PinYinColor}{よしおか}}这时说是要“照料”,破例地搬到宫内府的勤民楼来了。就这样,在\ruby{吉冈}{\textcolor{PinYinColor}{よしおか}}的监督下,日本医生给\xpinyin*{谭玉龄}进行了医治,不料在进行治疗的第二天,她便突然死去了。\\

令我奇怪的是,日本医生开始治疗时,表现非常热心,在她身边守候着,给她打针,让护士给她输血,一刻不停地忙碌着。但是在\ruby{吉冈}{\textcolor{PinYinColor}{よしおか}}把他叫到另外一间屋子里,关上门谈了很长时间的话之后,再不那么热情了,他没有再忙着注射、输血,变成了沉默而悄悄的。住在勤民楼里的\ruby{吉冈}{\textcolor{PinYinColor}{よしおか}},这天整夜不住地叫日本宪兵给病室的护士打电话,讯问病况。这样过了一夜,次日一清早,\xpinyin*{谭玉龄}便死了。不由我不奇怪,为什么\ruby{吉冈}{\textcolor{PinYinColor}{よしおか}}在治疗的时候,找医生谈那么长时间的话呢?为什么谈过话之后,医生的态度便变了呢?\\

我刚听到了她的死讯,\ruby{吉冈}{\textcolor{PinYinColor}{よしおか}}就来了,说他代表关东军司令官向我吊唁,并且立即拿来了关东军司令官的花圈。我心里越发奇怪,他们怎么预备的这样快呢?\\

由于我犯了疑心,就不由得回想起\xpinyin*{谭玉龄}的生前。在生前她是时常和我谈论日本人的。她在北京念过书,知道不少关于日本人在关内横行霸道的事。自从德王那件事发生后,我有时疑心德王乱说,有时疑心日本人偷听了我们的谈话。\xpinyin*{谭玉龄}的死,我不由得又想起了这些。\\

\ruby{吉冈}{\textcolor{PinYinColor}{よしおか}}在\xpinyin*{谭玉龄}死后不久的一个举动,更叫我联想到,即使不是\ruby{吉冈}{\textcolor{PinYinColor}{よしおか}}使了什么坏,她的死还是和关东军有关的。\xpinyin*{谭玉龄}刚死,\ruby{吉冈}{\textcolor{PinYinColor}{よしおか}}就给我拿来了一堆日本姑娘的相片,让我挑选。\\

我拒绝了。我说\xpinyin*{谭玉龄}遗体未寒,无心谈这类事。他却说,正是因为要解除我的悲痛,所以他要早日为我办好这件大事。我只得又说,这确是一件大事,但总得要合乎自己的理想,不能草率从事,况且语言不通,也是个问题。\\

“语言通的,嗯,这是会满洲语言的,哈!”\\

我怕他看出我的心思,忙说:“民族是不成问题的,但习惯上、兴趣上总要合适才好。”\\

我拿定了主意,决不要日本妻子,因为这就等于在我床上安上了个耳目。但这话不好明说,只得推三阻四,找各式借题来抵挡。\\

不想这个“御用挂”,真像挂在我身上一样,死皮赖脸,天天纠缠。我怕惹恼他,又不好完全封口。后来,也许是他明白我一定不要日本人,也许关东军有了别的想法,又拿来了一些旅顺日本学校的中国女学生的相片。我二妹提醒我说,这是日本人训练好的,跟日本人一样。可是我觉得这样总拖也不是个办法,因为如果关东军硬给我指定一个,我还是得认可。我最后决定挑一个年岁幼小的,文化程度低些的。在我看来,这样的对象,即使日本人训练过,也还好对付;而且只要我功夫作好,也会把她训练回来。决定后,我向\ruby{吉冈}{\textcolor{PinYinColor}{よしおか}}说了。\\

就这样,一个后来被称做“福贵人”的十五岁的孩子,便成了我的第四名牺牲品。她来了不到两年,也就是她还不到成年的年岁,伪满就垮了台。在大崩溃中,我成了俘虏,她被遣送回长春老家去了。
