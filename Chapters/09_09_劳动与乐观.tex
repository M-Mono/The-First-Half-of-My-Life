\fancyhead[LO]{{\scriptsize 1955-1959: 接受改造 · 劳动与乐观}} %奇數頁眉的左邊
\fancyhead[RO]{} %奇數頁眉的右邊
\fancyhead[LE]{} %偶數頁眉的左邊
\fancyhead[RE]{{\scriptsize 1955-1959: 接受改造 · 劳动与乐观}} %偶數頁眉的右邊
\chapter*{劳动与乐观}
\addcontentsline{toc}{chapter}{\hspace{1cm}劳动与乐观}
\thispagestyle{empty}
经过这次参观,我深信新社会的大门对我是敞开着的,问题就看我自己了。\\

我满怀希望地迈进了一九五八年。这时我已经有了乐观情绪。这种情绪最早的出现,是在一九五七年秋季抬煤的时候。\\

每年秋季,管理所就大量地运来煤炭,一部分准备冬季取暖,一部分制成煤砖供蔬菜温室使用。我们冬季吃的青菜都是自己暖房生产的。\\

从前每次搬运煤炭和制作煤砖都用不着我们,我们从这年起才开始参加这项劳动。这时我的体质与往年大不相同了。在本组里我和老王、蒙古族老正与一个伪将官年岁较小,凡是重活大都由我们四个人做,我因此得到了锻炼,体质有了显著的增强,从前的毛病已全部消失。在制作煤砖的劳动中,我担任的是比较费力气的抬煤工作。这天因为所长和一些干部都来参加制作煤砖,大伙干得特别起劲。临完工,我和老宪又多抬了三满筐。\\

交工具的时候,我听见王看守员对一个同伴说:\\

“我看\xpinyin*{溥仪}干活是实在的。他不挑显眼的干。”\\

我和老宪放下煤筐,到树权上拿衣服穿,所长笑着问我:\\

“\xpinyin*{溥仪},你的肩膀行不行?”\\

我看看肩膀,回答说:“不痛不肿,只略有点红。”\\

“你现在的饭量怎样?”\\

“干饭三大碗,大饺子可以吃三十多个。”\\

“不失眠了?”\\

“躺下就睡着,什么病也没有了。”\\

在场的人不论是所方人员还是伙伴们,全冲我乐起来。显然,这是和从前完全不同的笑声。我觉得受讥笑的日子已成为过去了。\\

我这时在其他方面,也有了进步,例如学习《政治经济学》和《历史唯物主义》,并不像从前那样吃力了,在自己的衣物整洁方面,跟别人的距离也大大缩小了。不过,我最有信心的还是劳动。只要不叫我做那些像扎纸花之类的细巧活,我的成绩总是第一流的。即使是理论学习成绩最好的人,都不免在这方面对我表示羡慕。\\

伙伴们的羡慕和我的信心的增长,与其说是由于劳动观点的树立,还不如说是由于社会上新出现的劳动风气的启示。从一九五七年末开始,我们就从报纸、家信以及所内人员的各种新动态上觉出了一种新风气,好像人人都在争着参加体力劳动,把体力劳动看做是最光荣的事。数以万计的干部上山下乡了,学校里增加了劳动课,出现了各式各样的短期义务劳动的队伍。在所里,我们不但看到了干部们做煤砖,而且看到所长和科长们在厨房里洗菜、烧火,以及在南道里挑送饭菜。每天清晨,我们还没起床,院子里就传来了木制车轮声和车上的镐、锨撞击声。这种声音告诉我们,所长和干部们已经出门到后山开荒去了。这一切都在启示我们说:在新社会里,劳动是衡量人的一项标准,当然,在改造中更不能例外。\\

我忘记了是谁告诉过我,许多人都错误地把劳动看做是上帝对人类的惩罚,只有共产党人才正确地把劳动看做是人类自己的权利。我当时对任何神佛都已丧失了兴趣,看不出劳动和上帝有什么关系。我们每个人都能看出,劳动对于共产党人来说,确实是一件很自然的事。记得有一次我们清除一堆垃圾,文质彬彬的李科员从这里走过,顺手拿起一把铁锨就干起来,干得比我们既轻快又麻利,而且一点不觉得多余。\\

一九五八年,劳动之受到重视,劳动之成为热潮,给我们的感受就更深了。我从北京的来信中,知道了许多新鲜事。从来闷在家里不问外事的二妹,参加了街道上的活动,兴高采烈地筹备着街道托儿所,准备帮助参加劳动的母亲们看管孩子。在故宫里工作的四妹参加了德胜门外修湖的义务劳动,被评为“五好”积极分子。三妹夫和三妹都参加了区政协的学习。老润和区政协的老头们参加了十三陵水库工程的劳动,这些人的年龄加起来有七百六十六岁,工地上就称他们为“七六六黄忠队”,他因为一件先进经验的创造而得到了表扬。五妹夫老万和五妹,以自豪的口吻报道大儿子的消息,这个学地质的大学生参加了关于利用冰雪问题的科学研究工作,作为向自然进军的尖兵,正在向祖国西北一座雪峰探险攀登。几个侄子和大李都有了工作,在市郊农场做了生产小队长。到处是劳动,到处是欢腾,到处是向自然进军的战鼓声。人人都为了改变祖国的落后面貌的伟大历史运动,贡献出自己的一份力量。伙伴们收到的家信中反映的气氛全是如此。后来,大家知道了毛主席和周总理以及部长们都参加了十三陵水库的劳动,简直就安静不下来了,一致向所方和学委会提出,要求组织生产劳动。\\

所方满足了大家的要求,先试办了一个电动机工厂,制造小型电动机。后来因为这种生产很有前途,而我们一所的人力既弱又少,又转交给三所、四所的\xpinyin*{蒋介石}集团战犯去办,另给我们安排其它的劳动。这次的安排,是按照各人的体质和知识等条件,并且是从培养生产技能着眼的。我们共编成五个专业组,即畜牧组、食品加工组、园艺组、蔬菜与温室组和医务组。我和老元、老宪、老曲(伪满四平省长)、老罗(伪满驻外使节)五人被编入医务组。我们的工作是每天扫除医务室,承担全部杂务和一部分医务助理工作,边做边学,另外每天有两小时的医学课程,在医务室温大夫的辅导下,自己读书和集体讨论。我的四个同学都当过医生,三人复习西医,老罗和我学的是中医。此外,针灸是五人的共同课。分组劳动了一段时间,我又有了新的信心。\\

我初到医务组时,医务助理业务远不如那四位同学。我制作外科用的棉球时,做得活像从旧棉絮里拣出来的;我量血压时,注意了看表就忘了听听诊器,或者顾了听又忘了看;我学习操纵血压电疗器械时,起先老是手忙脚乱,总弄不好。只有在干杂活、用体力时,我比他们每人都强。后来,我下定决心非学好业务不可。大夫或护士教过了我,我再找同学们请教,同学们教过了,我独自一人又不停地练习。这样学了一段时间,医务助理业务慢慢地弄会了。那时每天有个日本战犯来电疗,每次完毕之后,他总是向我深深一躬到地,并且说:“谢谢大夫先生。”我不禁高兴地想,固然我的白罩衣和眼镜可能引起了误会,但是这也说明我的操作技术得到了患者的信任。第一个学程终了,温大夫对我们进行了测验,结果我和别人一样地得了个满分。\\

在试制电动机的时候,我曾遇到过很堵心的事。电动机的生产分组名单,是学委会提出的。自老万、小瑞等人释放后,大家新选了前伪满总务厅次长老韦、\xpinyin*{溥杰}、老王和两个伪将官为委员,老韦为主委。凡是带技术性的工作,这个学委会都不给我做,带危险性的也不给我做,缠线圈怕我缠坏,铸铁怕我出事故,结果只把一项最简单的工作交给我,让我跟几个老头捣焦炭——把大块焦炭捣成小块。我把这看做是对我的轻视,交涉几次都没结果。现在,我把医务助理业务学得跟别人一样了,连那个治高血压的日本人都把我误认成大夫,第一次测验又得了个满分,相信自己并不十分笨,这样地学下去,自信必能学得一技之长,没有四百六十八件珍宝,自信照样能生活。\\

有一天,我要求见所长。这时老所长已经调了工作,这里成了他兼管的单位,不常来上班,接见我的是一位姓金的副所长。这位年轻的副所长精通日文,原是专管日本战犯的,日本战犯大批遣送回国后,他照顾了全所的工作。我对他说:\\

“我交出的那批首饰,政府应该正式收下来了。存条我也早丢掉了。”\\

我以为副所长对这件事的过程未必清楚,想从头再说一遍,不料他立刻笑着说:\\

“这件事我知道。怎么,你已经有了自食其力的信心了?”\\

这天,我用了一整天的时间讲了四百六十八件珍宝(这些东西后来进了陈列室)的每件的来历,由一位文书人员做了记录。完成了这件工作,我走到院子里,浑身轻松地想:\\

“副所长的那句话,无疑的是一句宝贵的鉴定。看来,我进步的不错吧?快到了做个正经人的那天了吧?”