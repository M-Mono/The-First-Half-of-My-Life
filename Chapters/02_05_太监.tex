\fancyhead[LO]{{\scriptsize 1908-1917: 我的童年 · 太监}} %奇數頁眉的左邊
\fancyhead[RO]{} %奇數頁眉的右邊
\fancyhead[LE]{} %偶數頁眉的左邊
\fancyhead[RE]{{\scriptsize 1908-1917: 我的童年 · 太监}} %偶數頁眉的右邊
\chapter*{太监}
\addcontentsline{toc}{chapter}{\hspace{1cm}太监}
\thispagestyle{empty}
讲我的幼年生活,就不能少了太监。他们服侍我吃饭、穿衣和睡觉,陪我游戏,伺候我上学,给我讲故事,受我的赏也挨我的打。别人还有不在我面前的时间,他们却整天不离我的左右。他们是我幼年的主要伴侣,是我的奴隶,也是我最早的老师。\\

役使太监的历史起于何年,我说不准,但我知道结束的日子,是在二次大战取得胜利,我从帝王宝座上第三次摔下来的那天,那时可能是太监最少的时候,只有十名左右。据说人数最多的是明朝,达十万名。清朝使用太监,在职务和数量上虽有过限制,但西太后时代也还有三千多名。\xpinyin*{辛亥}以后,太监大量逃亡,虽然优待条件上规定不许再招阉人,内务府仍旧偷着收用。据我最近看到的一份“\xpinyin*{宣统}十四年(即一九二二年)正月行二月分小建津贴口分单”上的统计,还有一千一百三十七名。两年后,经我一次大遣散,剩下了二百名左右,大部分服侍太妃和我的妻子(她们还有近百名宫女,大体未动)。从那以后,宫中使用的差役只是数量小得多的护军和被称为“随侍”的男性仆役。\\

在从前,禁城以内,每天到一定时刻,除了值班的乾清宫侍卫之外,上自王公大臣下至最低贱的扶役“苏拉”,全走得干干净净,除了皇帝自家人之外。再没有一个真正的男性。太监的职务非常广泛,除了伺候起居饮食、随侍左右、执伞提炉等事之外,用《宫中则例》上的话来说,还有:传宣谕旨、引带召对巨工、承接题奏事件;承行内务府各衙门文移、收复外库钱粮、巡查火烛;收掌文房书籍、古玩字画、冠袍履带、鸟枪弓箭;收贮古玩器皿、赏用物件、功臣黄册、于鲜果品;带领御医各宫请脉、外匠营造一切物件;供奉列祖实录圣训、御容前和神前香烛;稽查各门大小巨工出入;登记翰林入值和侍卫值宿名单;遵藏御宝;登载起居注;鞭\xpinyin*{笞}犯规宫女太监;饲养各种动物;打扫殿宇、收拾园林;验自呜钟时刻;请发;煎药;唱戏;充当道士在城隍庙里念经焚香;为皇帝做替身在雍和宫里充当喇嘛,等等。\\

宫中太监按系统说,大致可分为两大类,一类是在太后、帝、后、妃身边的太监,一类是其他各处的太监。无论哪一类太监,都有严格的等级,大致可分为总管、首领、一般太监。太后和帝后身边都有总管、首领,妃宫只有首领。品级最高的是三品,但从\xpinyin*{李莲英}起,开了赏戴二品顶戴的例,所以我所用的大总管\xpinyin*{张谦恭}也得到了这个“荣誉”。三品花翎都领侍,是各处太监的最高首领,统管宫内四十八处的太监,在他下面是九个区域的所谓九堂总管,由三品到五品,再下面是各处的首领太监,由四品到九品,也有无品级的,再下面是一般的太监。一般太监里等级最低的是打扫处的太监,犯了过失的太监就送到这里充当苦役。太监的月银按规定最高额是银八两、米八斤、制钱一贯三百,最低的月银二两、米一斤半、制钱六百。对于大多数太监,特别是上层太监说来,这不过是个名义上的规定,实际上他们都有各种各样的,集团的或个人的,合法的或非法的“外快”,比名义上的月银要多到不知多少倍。像\xpinyin*{隆裕}太后的总管太监\xpinyin*{张兰德}\footnote{\xpinyin*{张兰德}(1876-1957),中国清朝末代太监总管,名\xpinyin*{祥斋},字\xpinyin*{云亭},在内宫太监里排辈兰字,序号\xpinyin*{张兰德},\xpinyin*{慈禧}太后赐名“\xpinyin*{恒泰}”,宫号\xpinyin*{小德张}。天津静海县南吕官屯人。1898年被提升为后宫太监回事。\xpinyin*{庚子}事变中,随\xpinyin*{慈禧}太后西狩,回京后升任御膳房掌案,三品顶戴。1909年,按照\xpinyin*{慈禧}的懿旨,\xpinyin*{小德张}升为长春宫四司八处大总管。各王公贵族,朝廷大臣晋见\xpinyin*{隆裕}太后,必须得到\xpinyin*{小德张}的首肯,权倾一时。他的前任是\xpinyin*{李莲英}。},即绰号叫\xpinyin*{小德张}的,所谓“贵敌王侯,富埒天子”,是尽人皆知的。我用的一个二总管\xpinyin*{阮进寿},每入冬季,一天换一件皮袍,什么貂翎眼、貂爪仁、貂脖子,没有穿过重样儿的。仅就新年那天他穿的一件反毛的全海龙皮褂,就够一个小京官吃上一辈子的。宫中其他总管太监和一些首领太监,也莫不各有自己的小厨房,各有一些小太监伺候,甚至有的还有外宅“家眷”,老妈、丫头一应俱全。而低层太监则特别苦,他们一年到头吃苦受累挨打受罪,到老无依无靠,只能仗着极有限的“恩赏”过日子,如果犯了过失撵了出去,那就惟有乞讨和饿死的一条路了。\\

和我接触最多的是养心殿的太监,其中最亲近的是伺候我穿衣吃饭的御前小太监,他们分住在殿后东西两个夹道,各有首领一名管理。专管打扫的所谓殿上太监,也有首领一名。这两种太监统归大总管\xpinyin*{张谦恭}和二总管\xpinyin*{阮进寿}所管。\\

\xpinyin*{隆裕}太后在世时,曾派都领侍总管太监\xpinyin*{张德安}做我的“诸达”,这个职务是照顾我的生活,教给我一切宫中礼节等等。但我对他的感情和信任却远不如\xpinyin*{张谦恭}。\xpinyin*{张谦恭}当时是个五十多岁、有些驼背的老太监,是我的实际的启蒙老师。我进\xpinyin*{毓}庆宫读书之前,他奉太后之命先教我认字块,一直教我念完了《三字经》和《百家姓》。我进\xpinyin*{毓}庆宫以后,他每天早晨要立在我的卧室外面,给我把昨天的功课念一遍,帮助我记忆。像任何一个皇帝的总管太监一样,他总要利用任何机会,来表示自己对主子的忠心和深挚的感情。因此,在他喋喋不休的聒噪中,我在进\xpinyin*{毓}庆宫之前就懂得了\xpinyin*{袁世凯}的可恨、\xpinyin*{孙文}的可怕,以及民国是大清“让”出来的,民国的大官几乎都是大清皇帝的旧臣,等等。外面时局的变化,也往往从他的忧喜的感情变化上传达给我。我甚至还可以从他每天早晨给我背书的声音上,知道他是在为我担忧,还是在为我高兴。\\

\xpinyin*{张谦恭}也是我最早的游伴之一。和他一起做竞争性的游戏,胜利的永远是我。记得有一次过年的时候,\xpinyin*{敬懿}太妃叫我去玩押宝,\xpinyin*{张谦恭}坐庄,我押哪一门,哪一门准赢,结果总是庄家的钱都叫我赢光。他也不在乎,反正钱都是太妃的。\\

我和别的孩子一样,小时候很爱听故事。\xpinyin*{张谦恭}以及许多其他太监讲的故事,总离不开两类:一是宫中的鬼话;一是“圣天子百灵相助”的神话。总之,都是鬼怪故事,如果我能都写下来,必定比一部聊斋还要厚。照他们说来,宫里任何一件物件,如铜鹤、金缸、水兽、树木、水井、石头等等无一未成过精,显过灵,至于宫中供的关帝菩萨、\xpinyin*{真武}大帝等等泥塑木雕的神像,就更不用说了。我从那些百听不厌的故事中,很小就得到这样一个信念:一切鬼神对于皇帝都是巴结的,甚至有的连巴结都巴结不上,因此皇帝是最尊贵的。据太监们说,储秀宫里那只左腿上有个凹痕的铜鹤,在\xpinyin*{乾隆}爷下江南的时候,它成了精,跑到江南去保驾,不料被\xpinyin*{乾隆}射了一箭,讨了一场没趣,只好溜回原处站着。那左腿上生了红锈的凹痕便是\xpinyin*{乾隆}射的箭伤。又说御花园西鱼池附近靠墙处有一棵古松,在\xpinyin*{乾隆}某次下江南时,给\xpinyin*{乾隆}遮了一路太阳,\xpinyin*{乾隆}回京之后,赐了这松树一首诗在墙上。\xpinyin*{乾隆}亲笔诗里说的是什么,这个不识字的太监就不管了。\\

御花园钦安殿西北角台阶上,从前放着一块砖,砖下面有一个脚印似的凹痕。太监们说,\xpinyin*{乾隆}年间有一次乾清宫失火,\xpinyin*{真武}大帝走出殿门,站在台阶上向失火的方向用手一指,火焰顿息,这个脚印便是\xpinyin*{真武}大帝救火时踏下的。这当然是胡说八道。\\

我幼时住在长春宫的西厢房台阶上有一块石枕,据一位太监解释,因为附近的中正殿顶上那四条金龙,有一条常在夜间到长春宫喝大金缸里的水,不知是哪一代皇帝造了这个石枕,供那条金龙休息之用。对这种无稽之谈,我也听得津津有味。\\

皇帝的帽子上的一颗大珠子也有神话。说是有一天\xpinyin*{乾隆}在圆明园一条小河边散步,发现河里放光,他用鸟枪打了一枪,光不见了,叫人到河里去摸,结果摸出一只大蛤蜊,从中发现了这颗大珍珠。又说这颗珠子做了帽珠之后,常常私自外出,飞去飞回,后来根据“高人”的指点,在珠子上钻了孔,安上金顶,从此才把它稳住。关于这颗珠子,《阅微草堂笔记》另有传说,自然全是胡扯。用这颗珠子做的珠顶冠,我曾经戴用过,伪满垮台时把它丢失在通化大栗子沟了。\\

这类故事和太监的种种解说,我在童年时代是完全相信的。相信的程度可以用下面这个故事表明。我八九岁时,有一次有点不舒服,\xpinyin*{张谦恭}拿来一颗紫红色的药丸让我吃。我问他这是什么药,他说:“奴才刚才睡觉,梦见一个白胡子老头儿,手里托着一丸药,说这是长生不老丹,特意来孝敬万岁爷的。”我听了他这话,不觉大喜,连自己不舒服也忘了,加之这时由神话故事又联想到二十四孝的故事,我便拿了这个长生不老丹到四位太妃那里,请她们也分尝一些。这四位母亲大概从\xpinyin*{张谦恭}那里先受到了暗示,全都乐哈哈的,称赞了我的孝心。过了一个时期,我偶然到御药房去找药,无意间发现了这里的紫金锭,和那颗长生不老丹一模一样,虽然我感到了一点失望,但是,信不信由你,这个白胡子神仙给我送药的故事,我仍不肯认做是编造的。\\

太监们的鬼神故事一方面造成了我的自大狂,另方面也从小养成了我怕鬼的心理。照太监们说,紫禁城里无处没有鬼神在活动。永和宫后面的一个夹道,是鬼掐脖子的地方;景和门外的一口井,住着一群女鬼,幸亏景和门上有块铁板镇住了,否则天天得出来;三海中间的金鳌玉陈桥,每三年必有一个行人被桥下的鬼拉下去……这类故事越听越怕,越怕越要听。十二岁以后,我对于“怪力乱神”的书(都是太监给我买来的)又入了迷,加上宫内终年不断地祭神拜佛、萨满跳神等等活动,弄得我终日疑神疑鬼,怕天黑,怕打雷,怕打闪,怕屋里没人。\\

每当夕阳西下,禁城进入了暮色苍茫之中,进宫办事的人全都走净了的时候,静悄悄的禁城中央——乾清宫那里便传来一种凄厉的呼声:“搭闩,下钱粮\footnote{“下钱粮”可能是“下千两”,意思是“下锁”,宫中忌讳“锁”字,故以“下千两”代替;“下锁”,后又讹传为“下钱粮”。总之,已经没有人说得清。},灯火小——心——”随着后尾的余音,禁城各个角落里此起彼伏地响起了值班太监死阴活气的回声。这是\xpinyin*{康熙}皇帝给太监们规定的例行公事,以保持警惕性。这种例行公事,把紫禁城里弄得充满了鬼气。这时我再不敢走出屋子,觉得故事里的鬼怪都聚到我的窗户外面来了。\\

太监们用这些鬼话来喂养我,并非全是有意地奉承我和吓唬我,他们自己实在是非常迷信的。\xpinyin*{张谦恭}就是这样的人,他每有什么疑难,总要翻翻《玉匣记》才能拿主意。一般的太监也都很虔诚地供奉着“殿神”,即长虫、狐狸、黄鼠狼和刺猬这四样动物。本来宫里供的神很多,除了佛、道、儒,还有“王爹爹、王妈妈”,以及坤宁宫外的“神杆”、上\xpinyin*{驷}院的马、什么宫的蚕,日月星辰,牛郎织女,五花八门,无一不供,但惟有殿神是属于太监的保护神,不在皇室供奉之列。照太监们的说法,殿神是皇帝封的二品仙家。有个太监告诉我说,有一天晚上,他在乾清宫丹陛上走,突然从身后来了一个二品顶戴、蟒袍补褂的人,把他抓起来一把扔到丹陛下面,这就是殿神。太监们不吃牛肉,据一个太监说,吃牛肉是犯了大五荤,殿神会罚他们在树皮上蹭嘴,直蹭到皮破血流为止。太监若是进入无人去的殿堂,必先大喊一声“开殿!”才动手去开门,免得无意中碰见殿神,要受惩罚。太监每到初一、十五,逢年过节都要给殿神上供,平常是用鸡蛋、豆腐干。烧酒和一种叫“二五眼”的点心,年节还要用整猪整羊和大量果品,对于收入微薄的底层太监说来,均摊供品的费用,虽是个负担,但他们都心甘情愿,因为这些最常挨打受气的底层太监,都希望殿神能保佑他们,在福祸难测的未来,能少受点罪。\\

太监们为了取得额外收入,有许多办法。戏曲和小说里描写过,\xpinyin*{光绪}要花银子给西太后宫的总管太监,否则\xpinyin*{李莲英}就会刁难他,请安时不给他通报,其实这是不会有的。至于太监敲大臣竹杠,我倒听了不少。据说\xpinyin*{同治}结婚时,内务府打点各处太监,漏掉了一处,到了喜日这天,这处的太监便找了内务府的堂郎中来,说殿上一块玻璃裂了一条纹。按规矩,内务府司员不经传召,不得上丹陛,这位堂郎中只是站在下面远远地瞧了一下,果然瞧见玻璃上有条纹。这位司员吓得魂不附体,大喜日子出这种破像,叫西太后知道必定不得了。这时太监说了,不用找工匠,他可以悄悄想办法去换一块。内务府的人明白这是敲竹杠,可是没办法,只好送上一笔银子。银子一到,玻璃也换好了。其实玻璃并没有裂,那条纹不过是贴上的一根头发。\xpinyin*{世续}的父亲\xpinyin*{崇纶}当内务府大臣的时候,有一次也是由于办什么事,钱没有送周全,没吃饱的太监这天便等在\xpinyin*{崇纶}上朝见太后的路上,等\xpinyin*{崇纶}走过,故意从屋里设出一盆洗脸水,把\xpinyin*{崇纶}的貂褂泼得水淋淋的。那太监故作惊慌,连忙请罪。\xpinyin*{崇纶}知道这不是发脾气的时候,因为太后正等着他去\xpinyin*{觐}见,因此很着急地叫太监想办法。太监于是拿出了一件预备好的貂褂说:“咱们这苦地方,还要托大人的福,多恩典。”原来太监们向例预备有各种朝服冠带,专供官员临时使用时租赁的,这回\xpinyin*{崇纶}也只好让他们敲一笔竹杠,花了一笔可观的租衣费。\\

据内务府一位旧人后来告诉我,在我结婚时,内务府曾叫我的大总管(刚代替\xpinyin*{张谦恭}升上来的)\xpinyin*{阮进寿}敲了一笔。因为我事先规定了婚费数目,不得超过三十六万元,内务府按照这个数目在分配了实用额之后,可以分赠太监的,数目不多,因此在大总管这里没通过,事情僵住了。堂郎中锺凯为此亲自到\xpinyin*{阮进寿}住的地方,左一个阮老爷,右一个阮老爷,央求了半天,\xpinyin*{阮进寿}也没答应,最后还是按\xpinyin*{阮进寿}的开价办事,才算过了关。那位朋友当时是在场人,他过于年轻,又刚去“学习”不久,许多行话听不懂,所以\xpinyin*{阮进寿}得到了多少外快,他没有弄清楚。\\

不过我相信,像\xpinyin*{张谦恭}和\xpinyin*{阮进寿}这些“老爷”,比起\xpinyin*{小德张}来,在各方面都差得很远。我在天津时,\xpinyin*{小德张}也住在天津。他在英租界有一座豪华的大楼,有几个姨太太和一大群奴仆伺候他,威风不下于一个军阀。据说一个姨太太因为受不住他的虐待,逃到英国巡捕房请求保护。\xpinyin*{小德张}钱能通神,巡捕房不但没有保护那个女人,反而给送回了阎王殿,结果竟被\xpinyin*{小德张}活活打死。那女人死后,也没有人敢动他一下。
