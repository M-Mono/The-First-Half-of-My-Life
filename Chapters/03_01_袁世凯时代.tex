\fancyhead[LO]{{\scriptsize 1917-1924: 北京的“小朝廷” · 袁世凯时代}} %奇數頁眉的左邊
\fancyhead[RO]{} %奇數頁眉的右邊
\fancyhead[LE]{} %偶數頁眉的左邊
\fancyhead[RE]{{\scriptsize 1917-1924: 北京的“小朝廷” · 袁世凯时代}} %偶數頁眉的右邊
\chapter*{袁世凯时代}
\addcontentsline{toc}{chapter}{\hspace{1cm}袁世凯时代}
\thispagestyle{empty}
紫禁城中的早晨,有时可以遇到一种奇异的现象,处于深宫但能听到远远的市声。有很清晰的小贩叫卖声,有木轮大车的隆隆声,有时还听到大兵的唱歌声。太监们把这现象叫做“响城”。离开紫禁城以后,我常常回忆起这个引起我不少奇怪想象的响城。响城给我印象最深的,是有几次听到中南海的军乐演奏。\\

“\xpinyin*{袁世凯}吃饭了。”总管太监\xpinyin*{张谦恭}有一次告诉我,“\xpinyin*{袁世凯}吃饭的时候还奏乐,简直是‘钟鸣鼎食’,比皇上还神气!”\\

\xpinyin*{张谦恭}的光嘴巴抿得扁扁的,脸上带着愤愤然的神色。我这时不过九岁上下,可是已经能够从他的声色中感到类似悲凉的滋味。军乐声把我引进到耻辱难忍的幻象中:\xpinyin*{袁世凯}面前摆着比太后还要多的菜肴,有成群的人伺候他,给他奏乐,扇着扇子……\\

但也有另外一种形式的响城,逐渐使我发生了浓厚的兴趣。这种“响城”是我在\xpinyin*{毓}庆宫里从老师们的嘴里听到的。这就是种种关于复辟的传说。\\

复辟——用紫禁城里的话说,也叫做“恢复祖业”,用遗老和旧臣们的话说,这是“光复故物”,“还政于清”——这种活动并不始于尽人皆知的“\xpinyin*{丁巳}事件”,也并不终于民国十三年被揭发过的“\xpinyin*{甲子}阴谋”。可以说从颁布退位诏起到“满洲帝国”成立止\footnote{严格地说,复辟活动到此时尚未停止。伪满改帝制后,虽然我的活动告一段落,但关内有些人仍不死心,后来日本发动了全面侵略,占领了平津,这些人在建立“后清”的幻想下,曾有一度活动。因为日本主子不同意,才没闹起来。},没有一天停顿过。起初是我被大人指导着去扮演我的角色,后来便是凭着自己的阶级本能去活动。在我少年时期,给我直接指导的是师傅们,在他们的背后,自然还有内务府大臣们,以及内务府大臣\xpinyin*{世续}商得民国总统同意,请来照料皇室的“王爷”(他们这样称呼我的父亲)。这些人的内心热情,并不弱于任何紫禁城外的人,但是后来我逐渐地明白,实现复辟理想的实际力量并不在他们身上。连他们自己也明白这一点。说起来滑稽,但的确是事实:紫禁城的希望是放在取代大清而统治天下的新贵们身上的。第一个被寄托这样幻想的人,却是引起紫禁城忿忿之声的\xpinyin*{袁世凯}大总统。\\

我到现在还记得很清楚,紫禁城里是怎样从绝望中感到了希望,由恐惧而变为喜悦的。在那短暂的时间里,宫中气氛变化如此剧烈,以致连我这八岁的孩子也很诧异。\\

我记得太后在世时,宫里很难看到一个笑脸,太监们个个是唉声叹气的,好像祸事随时会降临的样子。那时我还没搬到养心殿,住在太后的长春宫,我给太后请安时,常看见她在擦眼泪。有一次我在西二长街散步,看见成群的太监在搬动体元殿的自鸣钟和大瓶之类的陈设。\xpinyin*{张谦恭}愁眉苦脸地念叨着:\\

“这是太后叫往颐和园搬的。到了颐和园,还不知怎么样呢!”\\

这时太监逃亡的事经常发生。太监们纷纷传说,到了颐和园之后,大伙全都活不成。\xpinyin*{张谦恭}成天地念叨这些事,每念叨一遍,必然又安慰我说:“万岁爷到哪儿,奴才跟哪儿保驾,决不像那些胆小鬼!”我还记得,那些天早晨,他在我的“龙床”旁替我念书的声音,总是有气无力的。\\

民国二年的新年,气氛开始有了变化。阳历除夕这天,陈师傅在\xpinyin*{毓}庆宫里落了座,一反常态,不去拿朱笔圈书,却微笑着瞅了我一会,然后说:\\

“明天阳历元旦,民国要来人给皇上拜年。是他们那个大总统派来的。”\\

这是不是他第一次向我进行政务指导,我不记得了,他那少有的得意之色,大概是我第一次的发现。他告诉我,这次接见民国礼官,采用的是召见外臣之礼,我用不着说话,到时候有内务府大臣\xpinyin*{绍英}照料一切,我只要坐在龙书案后头看着就行了。\\

到了元旦这天,我被打扮了一下,穿上金龙袍褂,戴上珠顶冠,挂上朝珠,稳坐在乾清宫的宝座上。在我两侧立着御前大臣、御前行走和带刀的御前侍卫们。总统派来的礼官\xpinyin*{朱启铃}走进殿门,遥遥地向我鞠了一个躬,向前几步立定,再鞠一躬,走到我的宝座台前,又深深地鞠了第三躬,然后向我致贺词。贺毕,\xpinyin*{绍英}走上台,在我面前跪下。我从面前龙书案上的黄绢封面的木匣子里,取出事先写好的答辞交给他。他站起身来向\xpinyin*{朱启铃}念了一遍,念完了又交还给我。\xpinyin*{朱启铃}这时再鞠躬,后退,出殿,于是礼成。\\

第二天早晨,气氛便发生了进一步的变化,首先是我的床帐子外边\xpinyin*{张谦恭}的书声朗朗,其次是在\xpinyin*{毓}庆宫里,陈师傅微笑着捻那乱成一团的白胡须,摇头晃脑地说:\\

“优待条件,载在盟府,为各国所公认,连他总统也不能等闲视之!”\\

过了新年不久,临到我的生日,阴历正月十四这天,大总统\xpinyin*{袁世凯}又派来礼官,向我祝贺如仪。经过\xpinyin*{袁世凯}这样连续的捧场,民国元年间一度销声匿迹的王公大臣们,又穿戴起蟒袍补褂、红顶花翎,甚至于连顶马开路、从骑簇拥的仗列也有恢复起来的。神武门前和紫禁城中一时熙熙攘攘。在民国元年,这些人到紫禁城来大多数是穿着便衣,进城再换上朝服袍褂,从民国二年起,又敢于翎翎顶顶、袍袍褂褂地走在大街上了。\\

完全恢复了旧日城中繁荣气象的,是\xpinyin*{隆裕}的寿日和丧日那些天。\xpinyin*{隆裕}寿日是在三月十五,过了七天她就去世了。在寿日那天,\xpinyin*{袁世凯}派了秘书长\xpinyin*{梁士诒}前来致贺,国书上赫然写着:“大中华民国大总统致书大清\xpinyin*{隆裕}皇太后陛下”。\xpinyin*{梁士诒}走后,国务总理\xpinyin*{赵秉钧}率领了全体国务员,前来行礼。\xpinyin*{隆裕}去世后,\xpinyin*{袁世凯}的举动更加动人:他亲自在衣袖上缠了黑纱,并通令全国下半旗一天,文武官员服丧二十七天,还派全体国务员前来致祭。接着,在太和殿举行了所谓国民哀悼大会,由参议长\xpinyin*{吴景濂}主祭;军界也举行了所谓全国陆军哀悼大会,领衔的是袁的另一心腹,上将军\xpinyin*{段祺瑞}。在紫禁城内,在太监乾嚎的举哀声中,清朝的玄色袍褂和民国的西式大礼服并肩进出。被赏穿孝服百日的亲贵们,这时脸上洋溢着得意的神色。最让他们感到兴奋的是\xpinyin*{徐世昌}也从青岛赶到,接受了清室赏戴的双眼花翎。这位清室太傅在颁布退位后,拖着辫子跑到德国人盘踞的青岛当了寓公,起了一个有双关含意的别号“东海”。他在北京出现的意义,我在后面还要谈到。\\

\xpinyin*{隆裕}的丧事未办完,南方发起了讨袁运动,即所谓“二次革命”。不多天,这次战争以\xpinyin*{袁世凯}的胜利而告终。接着,\xpinyin*{袁世凯}用军警包围国会,强迫国会选他为正式大总统。这时他给我写了一个报告:\\

\begin{quote}
	大清皇帝陛下:中华民国大总统谨致书大清皇帝陛下:前于\ruby{宣统}{\textcolor{PinYinColor}{\Man ᡤᡝᡥᡠᠩᡤᡝ ᠶᠣᠰᠣ}}三年十二月二十五日奉大清\xpinyin*{隆裕}皇太后\xpinyin*{懿旨},将统治权公诸全国,定为共和立宪国体,命\xpinyin*{袁世凯}以全权组织临时共和政府,合满汉蒙回藏五族,完全领土为一大中华民国。旋经国民公举,为中华民国临时大总统。受任以来,两稳于兹,深虞险越。今幸内乱已平,大局安定,于中华民国二年十月六日经国民公举为正式大总统。国权实行统一,友邦皆已承认,于是年十月十日受任。凡我五族人民皆有进于文明、跻于太平之希望。此皆仰荷大清\xpinyin*{隆裕}皇太后暨大清皇帝天下为公,唐虞揖让之盛轨,乃克臻此。我五族人民感戴兹德,如日月之照临,山河之涵育,久而弥昭,远而弥挚。维有董督国民,幸新治化,烙守优待条件,使民国巩固,五族协和,庶有以慰大清\xpinyin*{隆裕}皇太后在天之灵。用特报告,并祝万福。\\

\begin{flushright}
	中华民国二年十月十九日\\

	\xpinyin*{袁世凯}
\end{flushright}


\end{quote}

由于这一连串的新闻,遗老中间便起了多种议论。\\

“\xpinyin*{袁世凯}究竟是不是\xpinyin*{曹操}?”\\

“项城当年和徐、冯、段说过,对民军只可智取不可力敌,徐。冯、段才答应办共和。也许这就是智取?”\\

“我早说过,那个优待条件里的辞位的辞字有意思。为什么不用退位、逊位,袁宫保单要写成个辞位呢?辞者,暂别之谓也。”\\

“大总统常说‘办共和’办的怎样。既然是办,就是试行的意思。”\\

这年冬天,\xpinyin*{光绪}和\xpinyin*{隆裕}“奉安”,在梁格庄的灵棚里演出了一幕活剧。主演者是那位最善表情的\xpinyin*{梁鼎芬},那时他还未到宫中当我的师傅,配角是另一位自命孤臣的\xpinyin*{劳乃宣},是\ruby{宣统}{\textcolor{PinYinColor}{\Man ᡤᡝᡥᡠᠩᡤᡝ ᠶᠣᠰᠣ}}三年的学部副大臣兼京师大学堂总监督,\xpinyin*{辛亥}后曾躲到青岛,在德国人专为收藏这流人物而设的“尊孔文社”主持社事。在这出戏里被当做小丑来捉弄的是前清朝山东巡抚、袁政府里的国务员\xpinyin*{孙宝琦},这时他刚当上外交总长(\xpinyin*{孙宝琦}的父亲\xpinyin*{孙诒经}被遗老们视为\xpinyin*{同治}时代的名臣之一)。那一天,这一批国务员由\xpinyin*{赵秉钧}率领前来。在致祭前\xpinyin*{赵秉钧}先脱下大礼服,换上清朝素袍褂,行了三跪九叩礼。孤臣孽子\xpinyin*{梁鼎芬}一时大为兴奋,也不知怎么回事,在那些没穿清朝袍褂来的国务员之中,叫他一眼看中了\xpinyin*{孙宝琦}。他直奔这位国务员面前,指着鼻子问:\\

“你是谁?你是哪国人?”\\

\xpinyin*{孙宝琦}给这位老朋友问得怔住了,旁边的人也都给弄得莫名其妙。\xpinyin*{梁鼎芬}的手指头哆嗦着,指点着孙宝倚,嗓门越说越响:\\

“你忘了你是\xpinyin*{孙诒经}的儿子!你做过大清的官,你今天穿着这身衣服,行这样的礼,来见先帝先后,你有廉耻吗?你——是个什么东西!”\\

“问得好!你是个什么东西?!”\xpinyin*{劳乃宣}跟了过来。这一唱一帮,引过来一大群人,把这三个人围在中心。\xpinyin*{孙宝琦}面无人色,低下头连忙说:\\

“不错,不错,我不是东西!我不是东西!”\\

后来梁师傅一谈起这幕活剧时,就描述得有声有色。这个故事和后来的“结庐守松”、“凛然退刺客”,可算是他一生中最得意的事迹。他和我讲了不知多少次,而且越讲情节越完整,越富于传奇性。\\

到民国三年,就有人称这年为复辟年了。孤臣孽子感到兴奋的事情越来越多:\xpinyin*{袁世凯}祀孔,采用三卿士大夫的官秩,设立清史馆,\xpinyin*{擢}用前清旧臣。尤其令人眼花缭乱的,是前东三省总督\xpinyin*{赵尔巽}被任为清史馆馆长。陈师傅等人视他为贰臣,他却自己宣称:“我是清朝官,我编清朝史,我吃清朝饭,我做清朝事。”那位给\xpinyin*{梁鼎芬}在梁格庄配戏的\xpinyin*{劳乃宣},在青岛写出了正续《共和解》,公然宣传应该“还政于清”,并写信给\xpinyin*{徐世昌},请他劝说\xpinyin*{袁世凯}。这时\xpinyin*{徐世昌}既是清室太傅同时又是民国政府的国务卿,他把劳的文章给袁看了。袁叫人带信给\xpinyin*{劳乃宣},请他到北京做参议。前京师大学堂的\xpinyin*{刘廷琛},也写了一篇《复礼制馆书》,还有一位在国史馆当协修的\xpinyin*{宋育仁},发表了还政于清的演讲,都一时传遍各地。据说在这个复辟年里,连四川一个绰号叫十三哥的土匪,也穿上清朝袍褂,坐上绿呢大轿,俨然以遗老自居,准备分享复辟果实了。\\

在紫禁城里,这时再没有人提起搬家的事。谨慎稳健的内务府大臣\xpinyin*{世续}为了把事情弄牢靠些,还特地找了他的把兄弟\xpinyin*{袁世凯}一次。他带回的消息更加令人兴奋,因为\xpinyin*{袁世凯}是这样对他说的:“大哥你还不明白,那些条条不是应付南边的吗?太庙在城里,皇上怎么好搬?再说皇宫除了皇上,还能叫谁住?”这都是很久以后,在内务府做过事的一位遗少告诉我的。当时\xpinyin*{世续}和王爷根本不和我谈这类事情,要谈的也要经过陈师傅。师傅当时的说法是:“看样子,他们总统,倒像是优待大清的。优待条件本是载在盟府……”\\

师傅的话,好像总没有说完全。现在回想起来,这正是颇有见地的“慎重”态度。和紫禁城外那些遗老比起来,紫禁城里在这段时期所表现的乐观,确实是谨慎而有保留的。\xpinyin*{袁世凯}的种种举动——从公开的不忘\xpinyin*{隆裕}“在天之灵”,到私下认定“皇上”不能离开皇宫和太庙,这固然给了紫禁城里的人不少幻想,但是紫禁城从“袁宫保”这里所能看到的也只限于此。因此,紫禁城里的人就不能表现出太多的兴奋。到了复辟年的年底,北京开始变风头的时候,证明了这种“审慎”确实颇有见地。\\

风头之变换,始于一个肃政史提出要追查复辟传闻。\xpinyin*{袁世凯}把这一案批交内务部“查明办理”,接着,演讲过还政于清的\xpinyin*{宋育仁}被步军统领衙门递解回籍。这个消息一经传出,不少人便恐慌了,劝进文章和还政于清的言论都不见了,在青岛正准备进京赴任的\xpinyin*{劳乃宣}也不敢来了。不过人们还有些惶惑不解,因为\xpinyin*{袁世凯}在查办复辟的民政部呈文上,批上了“严禁复辟谣言,既往不咎”这样奇怪的话,而\xpinyin*{宋育仁}被递解回籍时,\xpinyin*{袁世凯}送了他三千块大洋,一路上又大受各衙门的酒宴迎送,叫人弄不清他到底是受罚还是受奖。直到民国四年,总统府的美国顾问\ruby{古德诺}{\textcolor{PinYinColor}{Goodnow}}发表了一篇文章,说共和制不适中国国情\footnote{\ruby{古德诺}{\textcolor{PinYinColor}{Goodnow}}原为美国政治大学教授。他发表的这篇文章的题目叫做《共和与君主论》,胡说什么“中国如用君主制较共和制为宜”,作为\xpinyin*{袁世凯}实行帝制的理论根据。},继而又有“筹安会”\footnote{筹安会是\xpinyin*{袁世凯}实行帝制的御用机关,由\xpinyin*{杨度}建议,吸收\xpinyin*{孙毓筠}、\xpinyin*{严复}、\xpinyin*{刘师培}、\xpinyin*{李燮和}、\xpinyin*{胡瑛}等组成,为袁称帝进行鼓吹和筹备工作。}出现,主张推\xpinyin*{袁世凯}为中华帝国的皇帝,这才扫清了满天疑云,使人们明白了\xpinyin*{袁世凯}要复的是什么辟。风头所向弄明白了,紫禁城里的气氛也变了。\\

我从响城中听见中南海的军乐声,就是在这时候。那时,三大殿正进行油缮工程,在养心殿的台阶上,可以清清楚楚地望见脚手架上油工们的活动。\xpinyin*{张谦恭}告诉我,那是为\xpinyin*{袁世凯}登极做准备。后来,“伦贝子”(\ruby{溥伦}{\textcolor{PinYinColor}{Pu Lun}})代表皇室和八旗向\xpinyin*{袁世凯}上劝进表,\xpinyin*{袁世凯}许给他亲王双俸,接着他又到宫里来向太妃索要仪仗和玉玺。这些消息使我感到心酸、悲忿,也引起了我的恐惧。虽然陈师傅不肯明讲,我也懂得“天无二日,国无二君”这句老话。\xpinyin*{袁世凯}自己做了皇帝,还能让我这多余的皇帝存在吗?历史上的例子可太多了,太史公就统计过“春秋之中,弑君三十六”哩!\\

在那些日子里,乾清门外的三大殿的动静,牵连着宫中每个人的每根神经。不论谁在院子里行走,都要关心地向那边张望一下,看看关系着自己命运的油缮工程,是否已经完工。太妃们每天都要烧香拜佛,求大清的护国神“协天大帝关圣帝君”给以保佑。仪仗是忙不迭地让\ruby{溥伦}{\textcolor{PinYinColor}{Pu Lun}}搬走了,玉玺因为是满汉合壁的,并不合乎\xpinyin*{袁世凯}的要求,所以没有拿去。\\

这时\xpinyin*{毓}庆宫里最显著的变化,是师傅们对\ruby{毓崇}{\textcolor{PinYinColor}{Yū Cong}}特别和气,没有人再拿他当伯禽来看待。他在太妃那里竟成了红人,常常被叫进去赏赐些鼻烟壶、搬指之类的玩艺儿。每逢我说话提到\xpinyin*{袁世凯}的时候,师傅就向我递眼色,暗示我住嘴,以免让\ruby{毓崇}{\textcolor{PinYinColor}{Yū Cong}}听见,传到他父亲\ruby{溥伦}{\textcolor{PinYinColor}{Pu Lun}}耳朵里去。\\

有一天,\ruby{毓崇}{\textcolor{PinYinColor}{Yū Cong}}应召到太妃那里去了,\xpinyin*{陈宝琛}看见窗外已经没有了他的影子,从怀里拿出一张纸条,神秘地对我说:\\

“臣昨天卜得的易卦,皇上看看。”\\

我拿过来,看见这一行字:\\

“我仇有疾,不我能即,吉!”\\

他解释说,这是说我的仇人\xpinyin*{袁世凯}前途凶恶,不能危害于我,是个吉卦。他还烧了龟背,弄过\xpinyin*{蓍}草,一切都是吉利的,告诉我可以大大放心。这位老夫子为了我的命运,把原始社会的一切算命办法都使用过了。因此,他乐观地做出结论:\\

“天作孽,犹可违,自作孽,不可活。元凶大憝的\xpinyin*{袁世凯}作孽如此,必不得善终!‘我仇有疾,终无尤也!’何况优待条件藏在盟府,为各国所公认,\xpinyin*{袁世凯}焉能为害于我乎?”\\

为了“不我能即”和保住优待条件,师傅、王爷和内务府大臣们在算卦之外的活动,他们虽没有告诉我,我也多少知道一些。他们和\xpinyin*{袁世凯}进行了一种交易,简单地说,就是由清室表示拥护袁皇帝,袁皇帝承认优待条件。内务府给了袁一个正式公文,说:“现由全国国民代表决定君主立宪国体,并推戴大总统为中华帝国大皇帝,为除旧更新之计,作长治久安之谋,凡我皇室极表赞成。”这个公文换得了\xpinyin*{袁世凯}亲笔写在优待条件上的一段跋语:\\

先朝政权,未能保全,仅留尊号,至今耿耿。所有优待条件各节,无论何时断乎不许变更,容当列入宪法。\xpinyin*{袁世凯}志,\xpinyin*{乙卯}孟冬。\\

这两个文件的内容后来都见于民国四年十二月十六日的“大总统令”中。这个“令”发表之前不多天,我父亲日记里就有了这样一段记载:\\

\begin{quote}
	十月初十日(即阳历十一月十六日)上门。偕世太傅公见四皇贵妃,禀商皇室与袁大总统结亲事宜,均承认可,命即妥行筹办一切云。在内观秘件,甚妥,一切如恒云云。\\
\end{quote}

所谓秘件,就是袁的手书跋语。所谓亲事,就是\xpinyin*{袁世凯}叫步兵统领\xpinyin*{江朝宗}向我父亲同\xpinyin*{世续}提出的让他女儿当皇后。太妃们心里虽不愿意,也不得不从。其结果是,优待条件既没列入宪法,我也没跟袁家女儿结婚,因为\xpinyin*{袁世凯}只做了八十三天的皇帝,就在一片反袁声中气死了。
