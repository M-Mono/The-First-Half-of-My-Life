\fancyhead[LO]{{\scriptsize 1955-1959: 接受改造 · 怎样做人}} %奇數頁眉的左邊
\fancyhead[RO]{} %奇數頁眉的右邊
\fancyhead[LE]{} %偶數頁眉的左邊
\fancyhead[RE]{{\scriptsize 1955-1959: 接受改造 · 怎样做人}} %偶數頁眉的右邊
\chapter*{怎样做人}
\addcontentsline{toc}{chapter}{\hspace{1cm}怎样做人}
\thispagestyle{empty}
“新的一年开始了,你有什么想法?”\\

一九五五年的元旦,所长这样问我。\\

我说惟有束身待罪,等候处理。所长听了,不住摇头,大不以为然地说:\\

“何必如此消极?应当积极改造,争取重新做人!”\\

一九五四年年底,我在检察人员拿来的最后的文件上签字时,也听到这样的话:“努力改造吧,争取做个新人。”\\

这些话使我感到了安心,却没有从根本上改变我的悲观消极态度。我陷入了深深自卑的境地里,相形之下,对于宣判的担心倒在其次了。\\

有一天,在院子里休息的时候,来了一位新闻记者,拿着照相机在球场上照相。“检举认罪”结束之后,管理所里恢复了从前的办法,不再是分组轮流而是全体同时休息,而且比从前多了半小时。院子里很热闹,打排球的、打乒乓球的、谈天说地的、唱歌的,干什么的都有,都被记者收进了镜头。他捧着相机东照西照,后来镜头对着我来了。跟我站在一起看球的一个前伪满人员发现了记者的企图,忽然转身走开,并且说了一句:“我可不跟他照在一块儿!”接着,别人也走开了。\\

三月间,一些解放军高级将领到抚顺来视察沈阳军区管辖下的战犯管理所。所长把我和\xpinyin*{溥杰}叫了去。我一看见满屋是金晃晃的肩章,先以为是要开军事法庭了,后来才知道是将军们要听听我的学习情况。将军们的态度都非常和蔼,听得似乎很有兴趣,并且问了我的童年时代和伪满时期的生活。最后有一位带胡子的首长说:“好好学习、改造吧,你将来能亲自看到社会主义建设实况的!”在回去的路上,我想起说话的好像是位元帅,而\xpinyin*{溥杰}告诉我说,其中怕还不止一位元帅。我心中无限感慨,曾经被我看做最不容我的共产党人,事实上从看守员到元帅无一不是拿我当做人看的,可是同犯们连跟我站在一起都觉得不能容忍,好像我连人都不是了。\\

回到屋里,我把元帅的谈话告诉了同伴们。当过伪满驻日大使的老元,是脑子最快的人,他说:“恭喜你啦,老溥!元帅说你看得见社会主义,可见你是保险了!”\\

别人一听这话全活跃起来,因为像我这样的头号汉奸能保险,他们自然更保险了。\\

检举认罪结束后,很多人心里都结着个疙瘩,对前途感到不安。老宪从开始检举认罪以来就没笑过,现在也咧开嘴,亲热地拍着我的肩膀说:“恭喜恭喜,老溥!”\\

检举认罪结束后,不但在院中休息时不禁止交谈,而且白天监房不上锁,偶尔也有人串房门,因此这个喜讯很快地传到了别的组,一所里全知道了。到了休息时间,院子里还有人在议论。我这时想起了我的侄子们和大李,从检举认罪以来总不爱答理我,这个消息必定也会让他们高兴,可以用这个题目找他们叙叙。我听到了小固唱歌的声音——这个最活跃的小伙子,跟看守员和卫兵们已学了不少的歌曲,现在正唱着《二小放牛郎》这支歌。我顺着声音,在操场角上的一棵大树旁找到了他和小秀。可是不等我走到跟前,他们已离开了那地方。\\

四月间,所方让我们一所按照七所日本战犯那样选举出了学委会。学委会是在所方指导下,由犯人们自己管理自己的学习、生活的组织。学习与生活中发生的问题,学习讨论会和生活检讨会的情况,由它负责集中起来向所方反映,并且要提出它的看法和意见。学委会有委员五名,由选举产生,经所方认定。除一名主委外,四名委员分工管学习、生活、体育和文娱。各组的学习组长和生活组长跟它的学习委员和生活委员每天联系一次,汇报情况。这个组织的成立,让犯人们感到很兴奋,觉得这是所方对我们的改造具有信心的证明,有些人从这上面更意识到了思想改造是自己的事。后来事实证明,这个组织对我们的改造具有重要意义。不过在它刚成立的那段时间里,我的心情却跟别人不一样。这五名委员中,有两名是我的家族,他们是在检举时对我最不留情面、最使我感到无地自容的人:一个是老万,担任主委;一个是小瑞,担任生活委员。\\

学委会成立不久,便通过了一项决议,要修一座运动场。我们原先用的运动场是日本战犯修的,现在要自己平整出一块地方,做我们一所的运动场。生活委员小瑞负责组织了这次劳动。第一次上工,我就挨了他一顿当众申斥。在站队点名时,我忘了是为了什么琐碎事,照例拖拖拉拉,落在别人后头。我边系着衣扣,边向队伍这里跑着,忽然听见了一声喊:“\xpinyin*{溥仪}!”\\

“来了来了!”我答应着,跑到排尾站下。\\

“每次集合,你都是迟到,这么多的人只等你一个,一点都不自觉!”他板着脸,大声地向我申斥,“看你这一身上下,邋里邋遢!扣子是怎么扣的?”\\

我低头看了一下,原来扣子都扣错了眼儿。这时全队的人都扭过头来看着我,我的手指哆嗦得连扣子都摸不准了。\\

我甚至担心过,生活检讨会的记录到了他们手里,会给我增添一些更不利的注解。这时我们组里的生活检讨会,已经很少有从前那种不是吵嚷一气,就是彼此恭维一番的情形了,比较能做到言之有物,至少是比以前采取了较为认真的态度。其原因,一则是有些人去掉了思想负担,或者是对改造有了些认识,因而出现了积极性,另则是像过去那种隔靴搔痒的发言,到了学委会那里首先过不了关。我这时对生活检讨会感到的变化,是别人对我发言完全没有了顾忌,特别是由于新编进这组来的伙伴中,有一个是最熟悉我的大李,而且当了生活组长。人们批评起我的缺点来,经他一介绍、分析,就更能打中要害,说出病根。有了大李的分析、介绍,加上同组人提出的事实材料,再经学委会里老万和小瑞的注解,我还像个人吗?\\

我从前在遇到外界的刺激,感到十分沮丧的时候,有时自怨自艾,把这看做是自作自受,有时则怨天尤人,怨命运,怨别人成心跟我过不去,最早的时候,则怨共产党,怨人民政府,怨所方。现在我虽然也怨天尤人,但更多的是怨自作自受,对共产党和政府,对所方,却越来越怨不上了。在检举认罪期间,我看完别人给我写的检举材料,知道我一切不愿人知道的全露出来了,政府方面原先不知道的全知道了,想不到我竞是这样的人,照理说即使不报复我,也要放弃改造我的念头。可是,检察人员、所长以至元帅却仍对我说,要学习、改造,重新做人,而且这种意思贯串在每个工作人员的思想中,表现在每件具体事实上。\\

操场完工后,学委会决定再美化一下我们的院子,要栽花修树,清除杂草,垫平洼坑,迎接五一节。大家都很高兴地干起来了。我起先参加垫大坑的工作,江看守员说我眼睛不好,恐怕掉到坑里去,便把我的工作改为拔草。我被分配到一块花畦边上,干了一会儿,蒙古人老正走到我身边,忽然一把抢走我手里刚拔下的东西,大叫大嚷起来:\\

“你拔的是什么?呵?”\\

“不是叫我拔草吗?”\\

“这是草吗?你真会挑,拔的全是花秧子!”\\

我又成了周围人们视线的焦点。我蹲在那里,抬不起头来。我真愿意那些花草全部从世界上消失掉。\\

“你简直是个废物!”老正拿着我拔的花秧子指着我,继续叫嚷。\\

这时江看守员走过来了。他从老正手里接过花秧子,看了看,扔到地上。\\

“你骂他有什么用?”他对老正说,“你应该帮助他,教给他怎么拔,这样他下次才不会弄错。”\\

“想不到还有人认不出花和草来。”老正讪讪的。\\

“我原先也想不到,那用不着说。现在看到了,就要想办法帮助。”\\

从前,我脑子里这“想不到”三个字总是跟可怕的结论连着的:“想不到\xpinyin*{溥仪}这样蠢笨——不堪救药!”“想不到\xpinyin*{溥仪}这样虚伪,这样坏——不能改造!”“想不到\xpinyin*{溥仪}有这样多的人仇恨他——不可存留!”现在,我在“想不到”这三个字后面听到的却是:“现在看到了,就要想办法帮助!”\\

而且是不止一次听到,不只从一个人口中听到,而且说的还不仅是要对我帮助。\\

有一天,我的眼镜又坏了。我经过一番犹豫,最后还是不得不去求大李。\\

“请你帮帮忙吧,”我低声下气地对他说,“我自己弄了几次,总也弄不好,别人也不行,求你给修修。”\\

“你还叫我伺候你!”他瞪眼说,“我还把你伺候的不够吗?你还没叫人伺候够吗?”\\

说罢,他忿然躲开了我,从桌子的这面转到另一面去了。\\

我呆呆地立着,恨不得一下子撞在墙上。\\

过了没有两分钟,只见大李从桌子那面又走回来,气哼哼地拿起了我的眼镜说:\\

“好,给你修。不过可要说明,这不是为了别的,不过为了帮助你改造。要不是为了这个,我才没功夫呢!”\\

后来,我在休息时间到新成立的小图书室去想独自散散心,在那里碰见了\xpinyin*{溥杰}。我跟他谈起了心事,说到我曾因为家里人们的态度,难过得整夜睡不着觉。他说:“你为什么不跟所方谈谈呢?”我说:“谈什么呢?人们从前受够了我的罪,自然应该恨我。”\xpinyin*{溥杰}说:“我听说所方也劝过他们,应该不念旧恶,好好帮助你。”我这才明白了大李为什么带着气又从桌子那边转回来。\\

我那时把帮助分做两类:一类是行动上的,比如像大李给我修眼镜,比如每次拆洗被褥后,别人帮助我缝起来,——否则我会弄一天,影响了集体活动;另一类是口头上的,我把别人对我的批评,放在这类里。所方常常说,要通过批评与自我批评,交换意见,进行互相帮助。我很少这样“帮助”人,而且这时也很不愿意接受别人的“帮助”。总之,尽管大李说他修眼镜的目的是帮助我改造,尽管所方说批评是改造思想的互助形式之一,我还是看不出任何一类的帮助与我改造思想、重新做人的关系。不但如此,我认为修眼镜、缝被子只能证明自己的无能,换得别人的鄙夷,在批评中也只能更显出我的伤疤和隐痛。不帮助还好,越帮助越做不得人了。\\

政府人员每次谈到“做人”,总是跟“改造思想”、“洗心革面”连着的,但我总想到“脸面”问题,总想到我的家族和社会上如何看待我,能否容忍我。我甚至想到,共产党和人民政府即使要把我留在世上,到了社会上也许还是通不过;即使没有人打我,也会有人骂我、啐我。\\

所方人员每次谈到思想改造,总是指出:人的行为都受一定思想的支配,必须找到犯罪行为的思想根源,从思想上根本解决它,才不至于再去犯罪。但我总是想,我过去做的那些事是决不会再做了,如果新中国的人容我,我可以保证永不再犯,何须总是挖思想。\\

我把“做人”的关键问题摆在这上面:对方对我如何,而不是我自己要如何如何。\\

但是所长却是这样说的:如果改造好,人民会给以宽大。改造不好,不肯改造,人民就不答应。事实上,问题在于自己。\\

这个事实引起我的注意,或者说,我开始知道一点怎样做人的问题,却是在我苦恼了多少日子之后,从一件小事上开始的。