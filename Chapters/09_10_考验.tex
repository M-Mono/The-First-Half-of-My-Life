\fancyhead[LO]{{\scriptsize 1955-1959: 接受改造 · 考验}} %奇數頁眉的左邊
\fancyhead[RO]{} %奇數頁眉的右邊
\fancyhead[LE]{} %偶數頁眉的左邊
\fancyhead[RE]{{\scriptsize 1955-1959: 接受改造 · 考验}} %偶數頁眉的右邊
\chapter*{考验}
\addcontentsline{toc}{chapter}{\hspace{1cm}考验}
\thispagestyle{empty}
我的自我估计,又过高了。我遇到了考验。\\

全国各个生产战线上都出现了大跃进的形势,所方在这时向我们提出,为了让思想跟上形势,加紧进行学习改造,有必要进行一次思想检查,清除思想前进途中的障碍。办法是在学习会上每人谈谈几年来思想认识上的变化,谈谈还有些什么问题弄不通。别人可以帮助分析,也可以提出问题要本人讲清楚。在轮到我的时候,发生了问题。\\

我谈了过去的思想,谈了对许多问题的看法的变化,在征求意见时,有人问我:\\

“像我们这样出身的人,跟日本帝国主义的关系是深远的,在思想感情上还可能有些藕断丝连。你跟日本人的关系不比我们浅,别人都谈到了这个问题,你怎么一点没谈?难道你就没有吗?”\\

“我对日本人只有痛恨,没什么感情可言,我跟你们不一样。”\\

我的话引起了很多人的反感。有人说:“你为什么这样不虚心?你是不是还以为比人高一等?”有人说:“你现在是什么感情?难道你比谁都进步?”有的人举出许多过去的例子,如我去日本作的诗,我扶日本皇太后上台阶等等,说明我当时比谁都感激日本人,现在却全不承认,令人难以置信。我回答说,我过去与日本人是互相利用,根本不是有感情;我并非看不起在座的人,只是直话直说。这番解释,并没有人同意。后来,当我谈到逃亡大栗子沟心中惧怕的情形,有人问我:\\

“日本人要送你去东京,先给你汇走了三亿日圆准备着,你不感激日本帝国主义吗?”\\

“三亿日圆?”我诧异起来,“我不知道什么三亿日圆!”\\

其实,这不是一件多大的问题。日本关东军从伪满国库里提走了最后的准备金,对外宣称是给“满洲国皇帝”运到日本去的。这笔钱我连一分钱都没看见过,别人都知道这件事,井不当做我的罪行,不过是想了解一下我当时的思想感情而已。我如果能够冷静地回忆一下,或者虚心地向别人打听一下,我会想起来的,但是我并没有这样做,而是非常自信、非常坚决地宣称:“我根本不知道这回事!”\\

“不知道?”许多知道这回事的人都叫起来了,“这是\xpinyin*{张景惠}和\ruby{武部}{たけべ}\ruby{六藏}{ろくぞう}经手的事,\xpinyin*{张景惠}这才死,你就不认账啦?”又有人问我:“你在认罪时难道没写这事吗?”我说没有,他们就更惊异了:“这件事谁不知道呀!”“这可不是三百三千,这是三亿呀!”\\

到了晚上,我这才认真地回忆了一下。这一想,我忽然想起来了。在大栗子沟时,\xpinyin*{熙洽}和我说过,关东军把伪满银行的黄金全弄走了,说是给我去日本准备日后生活用的。这一定就是那三亿日元了。那时我正担心生命的危险,竟没把这回事放在心里。第二天,我又向别人问过,确实是这回事,因此在小组会上向大家说了。\\

“你从前为什么隐瞒呢?”几个人一齐问。\\

“谁隐瞒?我本来就是忘了!”\\

“现在还说忘了?”\\

“现在想起来了。”\\

“怎么从前想不起来?”\\

“忘了就是忘了!不是也有忘事的时候吗?”\\

这一句话,引起了难以应付的议论:\\

“时间越久越记得起,越近倒越忘,这真奇怪。”“原来明明是有顾虑,却不敢承认。”“毫无认错的勇气,怎么改造?”“没有人相信你的话。政府保险再不上你的当。”“你太喜欢狡辩了,太爱撒谎了!”“这样不老实的人,能改造吗?”……\\

我越辩论,大家越不信,我想这可麻烦了,人人都认为我在坚持错误,坚持说谎了,如果反映到所方去,众口一词,所方还能相信我说的吗?脑子里这样一想,活像有了鬼似的,马上昏了头。我本来没有他们所说的顾虑,现在却真的有了顾虑。“以曾子之贤,曾母之信,而三人疑之,则慈母不能信也!”想起这个故事,我失掉了所有的勇气,于是我的旧病发作了——只要能安全地逃过这个难关,什么原则都不要了。不是检讨一下就可以混过去吗?好,我承认:我从前是由于顾虑到政府惩办,没有敢交代,现在经大家一说,这才没有顾虑了。\\

三亿元的事固然是真的忘了,然而在这个问题上,却正好把我灵魂深处的东西暴露了出来。\\

以后小组里再没有人对我的问题发生兴趣了,可是我自己却无法从脑子里把这件事抛开。我越想越不安,觉得事情越糟。明明是忘了,却给说成是隐瞒;我害怕政府说我不老实,偏偏又不老实,说了假话。这件事成了我的心病,我又自作自受地遇到了折磨。\\

在从前,我心中充满了疑惧,把所方人员每件举动都看成包含敌意的时候,我总被死刑的恐惧所折磨。现在,我明白了政府不但不想叫我死,而且扶植我做人,我心中充满了希望,不想又遇到了另一种折磨。越是受到所方人员的鼓励,这种折磨越是厉害。\\

有一天,看守员告诉我,所长找我去谈话。我当时以为一定是问我那三亿日元的事。我估计所长可能很恼火,恼我受到如此待遇,却仍旧隐瞒罪行不说。如果是这样,我真不知怎样办才好。但同时也另有一种可能,就是所长会高兴,认为我承认了错误,做了检讨,说不定因此称赞我几句。如果是这样,那就比骂我一顿还难受。我心里这样捣了一阵鬼,等进了所长的接待室,才知道所长谈的完全是另外一件事。\\

然而由于这次见所长的结果,却使我陷进了更深的苦闷中。\\

老所长已经许多日子不见了。这次他是陪着另一位首长来的。他们问过我的学习和劳动情况后,又问起我关于除四害的活动情形。\\

所长说,他听说我在捕蝇方面有了进步,完成了任务,不知在这次开展的捕鼠运动中有什么成绩。我说还没有订计划,不过我想我们组里每人至少可以消灭一只。\\

“你呢?”坐在所长旁边的那位首长问。我这才认出来,原来这是在哈尔滨时,问我为什么对日本鬼子的屠杀不提抗议的那位首长,不禁有些心慌。没等我回答,他又问:“你现在还没开‘杀戒’吗?”说罢,他大笑起来。笑声冲散了我的慌乱情绪,我回答说,我早没那些想法了,这次打算在捕鼠运动中一定消灭一只老鼠。\\

“你的计划太保守了!”他摇头说,“现在连小学生订的计划都不只每人一只。”\\

“我可以争取消灭两只。”我认真地说。\\

这时所长接口说,不给我订指标,我可以尽量去做。谈到这里,就叫我回来了。\\

从所长那里回来,我心头有了一种沉重感。这倒不是因为对平生未试过的捕鼠任务感到为难,而是我由这次谈话联想起许多事情。我想起不久前的一次消灭蚊蝇运动中,所方特意检查过我的计划,我想起了由于学会了洗衣服而受到了所长的鼓励,……所方在每件事情上对我一点一滴地下功夫,无非是为了我“做人”。可是,我却又骗了一次人,我想,即使捉到一百只老鼠,也不能抵消我的错误。\\

刚下班的江看守员见我在俱乐部里独自发呆,问我是不是有了捕鼠办法,并且说他可以帮助我做个捕鼠器。老实说,我不但没办法捉老鼠,就连老鼠藏在哪儿全不懂。我巴不得地接受了他的帮助。在跟他学做捕鼠器的时候,我刚放下的心事又被勾起来了。\\

我们一边做捕鼠器,一边聊起天来。江看守员从捉老鼠说起了他的幼年生活。我无意间知道了他的少年时代的悲惨境遇。我完全想不到这个平素非常安静、待人非常和气的青年,原来在伪满时期受了那么大的罪。他是“集家并屯”政策的典型牺牲者。由于连续三次集家并屯,寒天住在窝铺里,他全家感染上伤寒,弟兄八个,死得只剩下了他一个。死掉的那七个弟弟,全是光着身子埋掉的。\\

我们把捕鼠器具做好,他的故事也断了。他领着我去找鼠洞,我默默地跟着他,想着这个被伪满政权夺去七个兄弟生命的青年,何以今天能这样心平气和地帮我捉老鼠?这里所有的看守员都是这样和气,他们过去的境遇又是怎样的?后来,我忍不住地问他:\\

“王看守员和刘看守员,都在伪满受过罪吗?”\\

“那时候谁不受罪?”他说,“王看守员给抓了三次劳工,刘看守员被逼得无路可走,投了抗日联军。”\\

我现在明白了,不用问,东北籍的所方人员在伪满时期全是受过罪的。\\

我按着他的指导,果然完成了任务,而且是超额两倍。王看守员和刘看守员听说我捉住了老鼠,都像发现了奇迹似地来看我的“俘虏”,都称赞我有了进步。听着他们的称赞,我心里很不受用。这些在伪满时期受够了罪的人,把我的“进步”看得这样重要,而我却仍在骗着他们!\\

我每天照常到医务室工作,照常打扫屋子,给病人量血压,施行电疗,学习中医,那个矮个子日本战犯照常每天向我鞠躬。可是我听不清他的话了,《中医概论》变得难解起来了,给人量血压时常常要反复几次。妹妹和妹夫们来信继续告诉了他们的新成就,屡次向我表示祝愿,盼望我早日改造好,与他们共享幸福生活。这些话现在听来好像都成了责备。\\

秋天来了,我们像去年一样突击制作煤砖,副所长和干部们又一齐动手给温室准备过冬燃料。我尽量多抬煤,却尽量不想让所长看见,怕听到他的夸奖。这时如果听到了夸奖是比挨骂还要难受的。\\

有一天,到了施行电疗的时间,我忙一些别的事,晚到了一步,已经有两个人等在那里了。其中一个是那个每次鞠躬的日本人。我知道他是每次先来的,就让他先做。出乎我的意料,他却向另外那个做了个手势,同时说了一句中国话:\\

“您请,我不忙。”\\

“按次序,你先来的。”被他推让的那个\xpinyin*{蒋介石}集团的战犯说。\\

“不客气,我不忙。我可以多坐一会儿。”他又像解释似地加了一句:“我就要释放了。”\\

我这还是头一次知道他会说这样好的中国话。我给那个\xpinyin*{蒋介石}集团的战犯弄着器械,一边瞟了那日本人几眼。只见他面容严肃地望着对面的墙壁。过了一会儿,他的视线又移向天花板。\\

“这间屋子,伪满时候是刑讯室的一间,”他用低低的声音说,听不出他是自言自语,还是跟人说话,“不知有多少爱国的中国人,在这里受过刑呵!”\\

过了一会儿,他又指指屋顶说:\\

“那时候,这上面吊着铁链。墙上都是血。”他环视着墙壁,目光最后停在玻璃柜上。静默了一会儿又说,“中国的先生们修理这间屋子的时候,我们还以为是恢复刑讯室,报复我们,后来看见穿白衣服的大夫先生,又以为是要拿我们做解剖试验。谁知道,是给我们治病的医务室……”\\

他的声音哽咽起来。\\

\xpinyin*{蒋介石}集团的战犯病号疗完走了,我让这日本人电疗。他恭恭敬敬地站立着说:\\

“我不用了。我是来看看这间屋子。我没有见到温大夫,请您转告他,我没有资格向他致谢,我是替我的母亲谢谢他。谢谢您,大夫先生。”\\

“我不是大夫,我是\ruby{溥仪}{Pu I}。”\\

也不知他听见了没有,只见他鞠完躬,弯身退出了房门。\\

我觉着再也支持不下去了。无论所方如何难于理解,我也要把我的假话更正过来。\\

正在这时,老所长到管理所来了,要找我谈话。\\

我推开了接待室的门。书桌后是那个熟悉的头发花白的人。他正看着一堆材料,叫我先坐下。过了一会儿,他合上材料,抬起头来。\\

“你们小组的记录我看了。怎样?你最近思想上有什么问题没有?”\\

事到临头,我又犹豫起来了。我望望那些小组记录材料,想起了众口一词的小组会,我不禁想:他听了我一个人的话,总是不相信的,我说了真话,有什么好处?不过,我又怎么好再骗人呢?\\

“你说说吧,这次小组会开的怎样?”\\

“很好。”我说,“这是系统的总结思想,结论都是正确的。”\\

“嗯?”所长扬起了眉毛,“详细说说好不好?”\\

我觉得自己喘气都不自然了。\\

“我说的是真话,”我说,“说我有过顾虑,这结论很对,只是个别例子……”\\

“为什么不说下去?你知道,我是很想多了解一下你的思想情况的。”\\

我觉得再不能不说了。我一口气把事情的经过说完,心里怦怦地跳个不停。老所长十分注意地听着。听完,他问道:\\

“这有什么难说的?你是怎样想的?”\\

“我怕众口一词……”\\

“只要你说的是实话,怕什么呢?”所长神色十分严肃,“难道政府就不能进行调查研究,不能做出自己的分析判断吗?你还不够明白,做人就是要有勇气的。要有勇气说老实话。”\\

我流下了眼泪。我没料到在他的眼里,一切都是这样清楚。我还有什么说的呢?