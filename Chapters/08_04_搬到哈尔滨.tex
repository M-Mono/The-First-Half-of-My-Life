\fancyhead[LO]{{\scriptsize 1950-1954: 由抗拒到认罪 · 搬到哈尔滨}} %奇數頁眉的左邊
\fancyhead[RO]{} %奇數頁眉的右邊
\fancyhead[LE]{} %偶數頁眉的左邊
\fancyhead[RE]{{\scriptsize 1950-1954: 由抗拒到认罪 · 搬到哈尔滨}} %偶數頁眉的右邊
\chapter*{搬到哈尔滨}
\addcontentsline{toc}{chapter}{\hspace{1cm}搬到哈尔滨}
\thispagestyle{empty}
在开往哈尔滨的列车上,只有几个年轻些的人还有点兴趣谈天说笑,愿意跟看守员打打“百分”,其他的人则很少说话,即使说起来声调也不高。车厢里大部分时间都是沉寂的。有不少人夜里睡不着,白天吃不下。我虽然不像回国时那样恐怖,却仍是比任何人都紧张。这时,正是朝鲜战场上的美国军队逼近了鸭绿江,中国人民志愿军出国抗美援朝不久。有一天夜里,我见\xpinyin*{溥杰}跟我一样睡不着觉,便悄悄地问他对战局的看法。他死阴活气地回答说:“出国参战,简直是烧香引鬼。眼看就完啦!”我领会他所谓“完啦”的意思:一方面指中国必然吃败仗,至少东北要被美国军队占领;一方面担心共产党看到“大势已去,江山难保”,先动手收拾我们这批人,免得落到美国人手里去。后来才知道,这是当时犯人们的共同想法。\\

到了哈尔滨,看到管理所的房子,我越发绝望了。管理所的房子原是伪满遗留下来的监狱,看见了它,大有“以其人之道,还治其人之身”的滋味。这所监狱是经日本人设计,专门关押“反满抗日犯”的地方,共两层,中心是岗台,围着岗台的是两层扇面形的监房,监房前后都是直径一寸的铁栏杆。由洋灰墙隔成一间间小屋,每屋可容七八人。我这屋里住了五个人,不算拥挤,不过由于是日本式的,只能睡地铺。我在这里住了大约两年,后来听说拆掉了。刚住进去的时候,我还不知道伪满时关在这里的“犯人”很少有活着出去的,不过单是听到那铁栏杆的开关声,就已经够我受的了,这种金属响声总让我联想到酷刑和枪杀。\\

我们受到的待遇仍和抚顺一样,看守员仍旧那样和善,伙食标准丝毫没有变化,报纸、广播、文娱活动一切如常。看到这些,我的心情虽然有了缓和,却仍不能稳定下来。记得有一天夜里,市区内试放警报器,那凄厉的响声,在我脑里久久不能消失。一直到我相信了中朝人民军队确实连获胜利之前,我总认为自己不死于中国人之手,就得死在美国飞机的轰炸中。总之,我那时只想到中国必败、我必死,除此以外,别无其他结果。\\

我还清清楚楚地记得,我们从报上看到了中国人民志愿军在朝鲜前线取得第一次战役胜利的消息,当时谁也不相信;到了年末,第二次战役大捷的消息来了,中朝人民军队把美国军队赶到三八线附近,我们还抱有很大的怀疑。过了年,有一天一位所方干部站在岗台上,向大家宣读了中朝军队光复汉城的新闻号外,各监房爆发出激烈的掌声。那时我心中仍旧半信半疑。二月间,报上公布了“惩治反革命条例”,所方恐怕引起我们惊慌不安,停止我们阅报,我们不了解内情,便断定是在朝鲜前线打了败仗,怀疑以前的捷报全是假的。我由此认为自己的厄运快来了。\\

一天半夜,我突然被铁门声惊醒,见栏杆外来了好些人,从隔壁监房里拥着一个人走出去。我认为这必是美国军队逼近了哈尔滨,共产党终于对我们下手了,不由地浑身战栗起来。好容易度过了这一夜,天亮后听同屋子的人议论,才明白这是个天大的误会。原来前“四平省长”老曲半夜小肠疵气病发作,看守员发现后,报告了所长,所长带着军医和护士们来检查了一下,最后送他进了医院。我当时由于恐惧和联想,弄得神魂颠倒,所以只看见军装的裤腿,竟没看见医生和护士们的白衣衫。\\

这个误会的解除并没给我带来多大的安慰。我怕听的除了夜里的铁门声之外,还有白天的汽车声。每逢听见外面有汽车响,我就疑心是来装我们去公审的。\\

我白天把精力放在倾听、观察铁栏杆外边的一切动静上,夜里时常为噩梦惊醒。和我同屋的四个伪满“将官”,情形不比我好多少。他们跟我一样,饭量越来越小,声气越来越低。我记得那些日子,每逢楼梯那边有响声,大家都一齐转头向栏杆外窥探,如果楼梯上出现一个陌生面孔,各个监房里一定自动停止一切声息,好像每个人都面临着末日宣判一样。正在大家最感绝望的时候,公安机关的一位首长来到监狱,代表政府向我们讲了一次话。听了这次讲话我们才重新看到了生机。\\

这位首长站在岗台前对着各个监房讲了一个多小时。他代表政府明确地告诉我们,人民政府并不想叫我们死,而是要我们经过学习反省,得到改造。他说共产党和人民政府相信在人民的政权下,多数的罪犯是可能改造成为新人的。他说共产主义的理想,是要改造世界,就是改造社会和改造人类。他说完,所长又讲了一会儿。记得他说过这样一段话:\\

“你们只想到死,看什么都像为了让你们死才安排的。你们可以想想,如果人民政府打算处决你们,又何必让你们学习?\\

“你们对于朝鲜战争有很多奇怪的想法。有人可能认为,志愿军一定打不过美国军队,美国军队一定会打进东北,因此担心共产党先下手杀了你们;有人还可能迷信美国的武力,认为美国侵略者是不可战胜的。我可以明确地告诉你们:中朝人民一定会打败美帝国主义,中国共产党的改造罪犯的政策也一定得到胜利。共产党人从来不说空话,事实就是事实!\\

“你们也许会说,既然不想杀我们,就把我们放出去不好吗?不好!如果不经改造就放你们出去,不仅你们还会犯罪,而且人民也不答应,人民见了你们不会加以饶恕。所以,你们必须好好地学习、改造。”\\

我对那位首长和所长的话虽然不完全懂,甚至不完全相信,但关于政府不想处决我们的这段话,却是越想越有道理。是呵,如果是存心杀掉我们,在抚顺时何必为我们扩建监狱的澡堂?到哈尔滨又何必抢救垂危的病人?又何必一直对我和年纪大的给以伙食方面的照顾?\\

对于像治病、洗澡之类的这些生活待遇,后来才知道,在新中国的监狱里不是什么稀奇事,但在当时,我们确实感到很新奇,把它看做是对我们的特殊照顾。因此听到了政府人员正面说出不想消灭我们的话来,我们顿时觉得轻松了不少。\\

关于首长和所长说的学习、改造,在当时我们没有一个人加以理睬。在我看来,叫我们看书看报不过是为了让我们消磨时间,免得胡思乱想。说看几本书就可以改变一个人的思想,我觉得实在不可思议。对于美国军队可以打败的话,我更不相信。同屋的四个自命懂得军事的“将官”,则一致认为,美国或许没有胆量冒天下之大不韪,不敢拿出原子弹,然而美国仅仅用常规武器就足以称霸世界、无敌于天下;说可以打败美国军队,只不过是句空话。可是后来,我们渐渐觉得,共产党人不大像是说空话的人。过了不久我们重新看到了报纸,觉得那些有关朝鲜战场的消息不像是假的。那些“将官”们也说,历来编造战报,双方死伤人数可以造假,而地域的得失却不能做长时间的谎报,特别是美军总司令表示愿意谈判的消息,更是不能编造的。美国军队也要谈判停战问题,还能说是无敌的吗?“将官”怀疑起来了,不用说,我更解释不通了。\\

“兵不厌诈,”一个当过“旅长”的战犯说,“也许这里面还有问题呢!我不相信美国是‘纸老虎’。”\\

可是不管怎么不信,朝鲜战争越来越不像我们原先那样想的,美国越弄越不像个真老虎。这种出乎意料的情况越明显,我反而越感到了安心,因为我认为如果共产党没有溃败,就不至于急于消灭我这个累赘。\\

这时的学习也与以前不同了。以前的学习是自流的,所方并不过问,现在是所方管学习的干部亲自领导我们学习。他给我们做了“什么是封建社会”的专题讲话,然后由我们讨论。每人还要写学习笔记。\\

有一天,讲课的干部对我们说:\\

“我已经讲过,改造思想首先要了解自己原来是什么思想。每个人的思想是跟他的出身、历史分不开的,因此,要从自己的出身、历史上去研究。为了进行思想改造,每个人要客观地无保留地反省一下自己的历史,写一份自传。……”\\

我心里对自己说:“这就是改造吗?这是不是借口改造来骗我的供词呢?共产党看战局稳定下来,大概就要慢慢收拾我了吧?”\\

这就是我当时的思想。我正是在这种对立的思想支配下,写下了我的第一份自传的。