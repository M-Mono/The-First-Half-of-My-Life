\fancyhead[LO]{{\scriptsize 1931-1932: 到东北去 · 会见板垣}} %奇數頁眉的左邊
\fancyhead[RO]{} %奇數頁眉的右邊
\fancyhead[LE]{} %偶數頁眉的左邊
\fancyhead[RE]{{\scriptsize 1931-1932: 到东北去 · 会见板垣}} %偶數頁眉的右邊
\chapter*{会见板垣}
\addcontentsline{toc}{chapter}{\hspace{1cm}会见板垣}
\thispagestyle{empty}
\ruby{板垣征四郎}{いたがき せいしろう}是一九二九年调到关东军当参谋的,据远东国际军事法庭揭露,他在一九三零年五月就对人说,他对解决“满洲问题”已有了一个“明确的想法”,他认为必须以武力解决中日间的问题。至少在“九·一八”事变前一年,他就主张驱逐\xpinyin*{张学良},在东北建立一个“新国家”。判决书上说:他“自一九三一年起,以大住地位在关东军参谋部参加了当时以武力占领满洲为直接目的的阴谋,他进行了支持这种目标的煽动,他协助制造引起所谓‘满洲事变’的口实,他压制了若干防止这项军事行动的企图,他同意了和指导了这种军事行动。嗣后,他在鼓动‘满洲独立’的欺骗运动中以及树立傀儡伪‘满洲国’的阴谋中,都担任了主要的任务。”\\

他于一九三四年任关东军副参谋长,一九三七年“七·七”事变后是师团长,一九三八年做了陆军大臣,一九三九年任中国派遣军的参谋长,以后做过朝鲜司令官、驻新加坡的第七方面军司令官。在华北内蒙树立伪政权、进攻中国内地、树立\xpinyin*{汪精卫}伪政权、发动哈桑湖对苏联进攻等等重大事件中,他都是重要角色。\\

二月二十三日下午,我会见了\ruby{板垣}{いたがき},由关东军通译官\xpinyin*{中岛比多吉}任翻译。\ruby{板垣}{いたがき}是个小矮个,有一个剃光的头,一张刮得很干净的青白色的脸,眉毛和小胡子的黑色特别显眼。在我见过的日本军官中,他的服装算是最整洁的了,袖口露出白得刺眼的衬衫,裤腿管上的圭角十分触目,加上他的轻轻搓手的习惯动作,给了我一个颇为斯文和潇洒的印象。\ruby{板垣}{いたがき}先对我送他礼物表示了谢意,然后表明,他奉关东军\ruby{本庄}{ほんじょう}司令官之命,向我报告关于“建立满洲新国家”的问题。\\

他慢条斯理地从什么“张氏虐政不得人心,日本在满权益丝毫没有保障”谈起,大谈了一阵日军行动的“正义性”,“帮助满洲人民建立王道乐土的诚意”。我听着他的话,不断地点头,心里却希望他快些把我关心的答案说出来。好不容易,他总算谈到了正题:\\

“这个新国家名号是‘满洲国’,国都设在长春,因此长春改名为新京,这个国家由五个主要民族组成,即满族、汉族、蒙古族、日本族和朝鲜族。日本人在满洲花了几十年的心血,法律地位和政治地位自然和别的民族相同,比如同样地可以充当新国家的官吏。……”\\

不等\ruby{中岛}{なかじま}翻译完,他从皮包里又拿出了《满蒙人民宣言书》以及五色的“满洲国国旗”,放到我面前的茶几上。我气得肺都要炸了。我的手颤抖着把那堆东西推了一下,问道:\\

“这是个什么国家?难道这是大清帝国吗?”\\

我的声音变了调。\ruby{板垣}{いたがき}照样地不紧不慢地回答:“自然,这不是大清帝国的复辟,这是一个新国家,东北行政委员会通过决议,一致推戴阁下为新国家的元首,就是‘执政’。”\\

听到从\ruby{板垣}{いたがき}的嘴里响出个“阁下”来,我觉得全身的血都涌到脸上来了。这还是第一次听日本人这么称呼我呢!“\xpinyin*{宣统}帝”或者“皇帝陛下”的称谓原来就此被他们取消了,这如何能够容忍呢?在我的心里,东北二百万平方里的土地和三千万的人民,全抵不上那一声“陛下”呀!我激动得几乎都坐不住了,大声道:\\

“名不正则言不顺,言不顺则事不成!满洲人心所向,不是我个人,而是大清的皇帝,若是取消了这个称谓,满洲人心必失。这个问题必须请关东军重新考虑。”\\

\ruby{板垣}{いたがき}轻轻地搓着手,笑容满面地说:\\

“满洲人民推戴阁下为新国家的元首,这就是人心所归,也是关东军所同意的。”\\

“可是日本也是天皇制的帝国,为什么关东军同意建立共和制呢?”\\

“如果阁下认为共和制不妥,就不用这个字眼。这不是共和制,是执政制。”\\

“我很感谢贵国的热诚帮助,但是别的都可说,惟有这个执政制却不能接受。皇帝的称谓是我的祖宗所留下的,我若是把它取消了,即是不忠不孝。”\\

“所谓执政,不过是过渡而已,”\ruby{板垣}{いたがき}表示十分同情,“\xpinyin*{宣统}帝是大清帝国的第十二代皇帝陛下,这是很明白的事,将来在议会成立之后,我相信必定会通过恢复帝制的宪法,因此目前的执政,不过是过渡时期的方法而已。”\\

我听到“议会”这两字,像挨了一下火烫似的,连忙摇头说:“议会没有好的,再说大清皇帝当初也不是什么议会封的!”\\

我们争来争去,总谈不到一起。\ruby{板垣}{いたがき}态度平和,一点不着急,青白脸上浮着笑容,两只手搓来搓去;我不厌其烦地重复着那十二条不得不正统系的道理,翻来覆去地表示,不能放弃这个皇帝的身分。我们谈了三个多钟头,最后,\ruby{板垣}{いたがき}收拾起了他的皮包,表示不想再谈下去了。他的声调没变,可是脸色更青更白了,笑容没有了,一度回到他口头上的\xpinyin*{宣统}帝的称呼又变成了阁下:“阁下再考虑考虑,明天再谈。”他冷冷地说完,便告辞走了。\\

这天晚上,根据郑氏父子和\xpinyin*{上角}的意见,我在大和旅馆里专为\ruby{板垣}{いたがき}举行了一个宴会。照他们的话说,这是为了联络感情。\\

我在宴会上的心情颇为复杂。我所以敢于拒绝执政的名义,多少是受了\xpinyin*{胡嗣瑗}、\xpinyin*{陈曾寿}这些人的影响,即认为日本人把东北弄成目前这种局面,非我出来就不能收拾,因此,只要我坚持一下,日本人就会让步。但是,在我拒绝了\ruby{板垣}{いたがき}之后,\xpinyin*{郑孝胥}就提醒我,无论如何不能和日本军方伤感情,伤了感情一定没有好处,\xpinyin*{张作霖}的下场就是殷鉴。我一听这话,又害怕起来。我原来认为,土匪出身的\xpinyin*{张作霖}和我这“自与常人殊”的“龙种”按理不能并列,现在我看出了,在日本人心里并不把我当做“龙种”看待,因此我不得不时时注意着\ruby{板垣}{いたがき}的那张青白脸。那张脸竟是个没有春夏秋冬的脸。他大口喝酒,对任何人的敬酒都表现十分豪爽,绝口不提白天的争论,就好像根本不曾发生过什么似的。这天晚上犹如约定好了一样,宴会上的人除了风花雪月,烟酒饮食,没有人说别的。一直到晚上十点钟结束宴会,我还没看出\ruby{板垣}{いたがき}脸上的气候。\\

可是用不着我再费多少时间去试探,第二天早晨,\ruby{板垣}{いたがき}把\xpinyin*{郑孝胥}、\xpinyin*{罗振玉}、\xpinyin*{万绳栻}和\xpinyin*{郑垂}都叫到大和旅馆,让他们向我传达了他的“气候”:\\

“军部的要求再不能有所更改。如果不接受,只能被看做是敌对态度,只有用对待敌人的手段做答复。这是军部最后的话!”\\

听到了这个回答,我怔住了。我的腿一软,跌坐在沙发上,半晌说不出话来。\\

\xpinyin*{罗振玉}垂头丧气,不发一言,\xpinyin*{万绳栻}惊慌不安地立在一旁,别人也都不言语。静了一回,只听见\xpinyin*{郑孝胥}说:“臣早说过,不可伤日本的感情……不过现在还来得及,臣已经在\ruby{板垣}{いたがき}面前极力担承,说皇上必能乾纲独断。”\\

我没有作声。\\

“不人虎穴焉得虎子?”\xpinyin*{郑垂}走了过来,满面春风地说,“识时务者为俊杰。咱君臣现在是在日本人掌心里,不能吃眼前亏,与其跟他们决裂,不如索性将计就计,以通权达变之方,谋来日之宏举。”\\

昨晚在宴会上\xpinyin*{郑垂}是最活跃的一个,池和\ruby{板垣}{いたがき}一再干杯,宴会后又拉着\ruby{板垣}{いたがき}喝酒。今天他的通权达变、将计就计论说得如此娓娓动听,我没把它和昨晚的特殊举动联系起来,只奇怪他和他老子去沈阳之前,还说过非大清复辟不干,怎么变的这么快呢?\\

\xpinyin*{郑孝胥}着我不作声,又换上了激昂的声调说:“日本人说得出做得出,眼前这个亏不能吃,何况日本人原是好意,让皇上当元首,这和做皇帝是一样。臣伺候皇上这些年,还不是为了今天?若是一定不肯,臣只有收拾铺盖回家。”听了他这话,我发了慌。他儿子接着说:“现在答应了日本军部,将来把实力培植起来,不愁没有办法按着咱的意思去办。”这时\xpinyin*{罗振玉}垂头丧气地说:“事已如此,悔之不及,只有暂定以一年为期,如逾期仍不实行帝制,到时即行退位,看以此为条件,\ruby{板垣}{いたがき}还怎么说。”我再没有办法,叹一口气,便叫\xpinyin*{郑孝胥}去和\ruby{板垣}{いたがき}说说看。\\

过了不多时,\xpinyin*{郑孝胥}头顶闪着光回来了,说\ruby{板垣}{いたがき}已经同意,并且今晚要“为未来的执政举行一个小规模的宴会!”\\

我就是这样,一方面是浑身没有一根骨头是硬的,一方面还幻想着未来的“复位登极”,公开走上了这条卑鄙无耻的道路,确定了头号汉奸的身分,给血腥的统治者充当了遮羞布。在这块布底下,从一九三二年二月二十三日这天起,祖国的东北完全变成了殖民地,三千万同胞开始了染满血泪的苦难生活。同时,我也给\ruby{本庄}{ほんじょう}、\ruby{板垣}{いたがき}之流增添了信心,奠定了他们“发家”的基石。\xpinyin*{郑孝胥}日记里这样记下了\ruby{本庄}{ほんじょう}、\ruby{板垣}{いたがき}等人的命运关头:\\

\begin{quote}
	上乃决,复命\xpinyin*{万绳栻}往召\ruby{板垣}{いたがき}。遂改“暂为维持”四字。\ruby{板垣}{いたがき}退而大悦。昨日\ruby{本庄}{ほんじょう}两次电话来询情形,\ruby{板垣}{いたがき}今日十一时当去。暂许之议,十时乃定。危险之机,间不容发。盖此议不成,则\ruby{本庄}{ほんじょう}、\ruby{板垣}{いたがき}皆当引咎辞职,而日本陆军援立之策败矣。\\
\end{quote}
