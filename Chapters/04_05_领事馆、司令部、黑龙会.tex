\fancyhead[LO]{{\scriptsize 1924-1930: 天津的“行在” · 领事馆、司令部、黑龙会}} %奇數頁眉的左邊
\fancyhead[RO]{} %奇數頁眉的右邊
\fancyhead[LE]{} %偶數頁眉的左邊
\fancyhead[RE]{{\scriptsize 1924-1930: 天津的“行在” · 领事馆、司令部、黑龙会}} %偶數頁眉的右邊
\chapter*{领事馆、司令部、黑龙会}
\addcontentsline{toc}{chapter}{\hspace{1cm}领事馆、司令部、黑龙会}
\thispagestyle{empty}
敬陈管见,条列于后:\\

\begin{quote}
	……对日本宜暗中联合而外称拒绝也。关东之人恨日本刺骨,日本禁关东与党军和协,而力足以取之。然日本即取关东不能自治,非得皇上正位则举措难施。今其势日渐紧张,关东因无以图存,日人亦无策善后,此\ruby{田中}{たなか}之所以屡示善意也。\\

我皇上并无一成一旅,不用日本何以恢复?机难得而易失,天子不取,后悔莫追。故对日本只有联合之诚,万无拒绝之理。所难者我借日本之力而必先得关东之心。若令关东之人,疑我合日谋彼,则以后欲由东三省拥戴,势有所难。此意不妨与日本当机要人明言之,将来皇上复位,日本于三省取得之权,尚须让步方易办理。……\\
\end{quote}

这是一九二八年我收到的一份奏折中的一段。这段话代表了张园里多数人的想法,也是我经过多年的活动后,日益信服的结论。\\

前面已经说过,我自从进了北府,得到了日本人的“关怀”以来,就对日本人有了某些信赖。我在日本公使馆里住了些日子,到了天津之后,我一天比一天更相信,日本人是我将来复辟的第一个外援力量。\\

我到天津的第一年,日本总领事\ruby{古田}{ふるた}\ruby{茂}{しげる}曾请我参观了一次日本侨民小学。在我往返的路上,日本小学生手持纸旗,夹道向我欢呼万岁。这个场面使我热泪满眶,感叹不已。当军阀内战的战火烧到了天津的边缘,租界上的各国驻军组织了联军,声言要对付敢于走近租界的国民军的时候,天津日本驻屯军司令官\ruby{小泉}{こいずみ}\ruby{六一}{ろくいち}中将特意来到张园,向我报告说:“请\xpinyin*{宣统}帝放心,我们决不让中国兵进租界一步。”我听了,大为得意。\\

每逢新年或我的寿辰,日本的领事官和军队的将住们必定到我这里来祝贺。到了日本“天长节”,还要约我去参观阅兵典礼。记得有一次“天长节”阅兵,日本军司令官\ruby{植田}{うえだ}\ruby{谦古}{へりくだふる}邀请了日租界不少高级寓公,如\xpinyin*{曹汝霖}、\xpinyin*{陆宗舆}、\xpinyin*{靳云鹏}等人都去了。我到场时,\ruby{植田}{うえだ}司令官特意骑马过来行致敬礼。当阅兵完毕,我们这些中国客人凑在一起,竟然随着日本人同声高呼“天皇万岁”。\\

日军司令部经常有一位住级参谋来给我讲说时事,多年来十分认真,有时还带来专门绘制的图表等物。第一个来讲的大概是名叫\ruby{河边}{かわなべ}的参谋,他调走之后继续来讲的是\ruby{金子}{きんす}\ruby{定一}{さだいち},接\ruby{金子}{きんす}的是后来在伪满当我的“御用挂”的\ruby{吉冈}{よしおか}\ruby{安直}{やすなお}。这个人在伪满与我相处十年,后面我要用专门的一节来谈他。\\

日军参谋讲说的时事,主要是内战形势,在讲解中经常出现这样的分析:“中国的混乱,根本在于群龙无首,没有了皇帝。”并由此谈到日本的天皇制的优越性,谈到中国的“民心”惟有“\xpinyin*{宣统}帝”才能收拾。中国军队的腐败无力是不可或缺的话题,自然也要用日本皇军做对比。记得济南惨案发生后,\ruby{吉冈}{よしおか}\ruby{安直}{やすなお}至少用了一个小时来向我描述\xpinyin*{蒋介石}军队的无能。日本布告的抄件,就是那次他给我拿来的。这些讲话加上历次检阅日军时获得的印象,使我深信日本军队的强大,深信日本军人对我的支持。\\

有一次我到白河边上去游逛,眺望停在河中心的日本兵舰。不知兵舰舰长怎么知道的,突然亲自来到岸上,虔敬地邀请我到他的舰上参观。到了舰上,日本海军将校列队向我致敬。这次由于仓猝间双方都没有准备翻译,我们用笔谈了一阵。这条兵舰舰名“藤”,船长姓\ruby{蒲田}{かまた}。我回来之后,\ruby{蒲田}{かまた}和一些军官向我回访,我应他的请求送了他一张签名照片,他表示这是他的极大的荣幸。从这件事情上,我觉得日本人是从心眼里对我尊敬的。我拉拢军阀、收买政客、任用客卿全不见效之后,日本人在我的心里的位置,就更加重要了。\\

起初,“日本人”三个字在我心里是一个整体。这当然不包括日本的老百姓,而是日本公使馆、天津日本总领事馆和天津日本“驻屯军”司令部里的日本人,以及和\xpinyin*{罗振玉}、\xpinyin*{升允}来往的那些非文非武的日本浪人。我把他们看成整体,是因为他们同样地“保护”我,把我当做一个“皇帝”来看待,同样地鄙夷民国,称颂大清,在我最初提出要出洋赴日的时候,他们都同样地表示愿意赞助。一九二七年,我由于害怕北伐军的逼近,一度接受\xpinyin*{罗振玉}劝告,决定赴日。经过日本总领事的接洽,日本总领事馆向国内请示,\ruby{田中}{たなか}内阁表示了欢迎,并决定按对待君主之礼来接待我。据\xpinyin*{罗振玉}说,日本军部方面已准备用军队保护我启程。只是由于形势的缓和,也由于\xpinyin*{陈宝琛}、\xpinyin*{郑孝胥}的联合劝阻,未能成行。后来,南京的国民党政府成立了,官方的“打倒帝国主义”、“废除不平等条约”之类的口号消失了,我逐渐发现,尽管日本人的“尊敬”、“保护”仍然未变,但是在我出洋之类的问题上,他们的态度却有了分歧。这种分歧甚至达到了令我十分愤慨的程度。\\

一九二七年下半年,有一天\xpinyin*{罗振玉}向我说:“虽然日租界比较安全,但究竟是鱼龙混杂。据日本司令部说,革命党(这是一直保留在张园里的对于国民党和共产党的笼统称呼)的便衣(这是对于秘密工作者的称呼,而且按他们解释,都是带有武器的)混进来了不少,圣驾的安全,颇为可虑。依臣所见,仍以暂行东幸为宜,不妨先到旅顺。恭亲王在那边有了妥善筹备,日本军方也愿协助,担当护驾之责。”这时我正被“革命党便衣”的谣言弄得惶惶不安,听了\xpinyin*{罗振玉}的话,特别是\ruby{溥伟}{Pu Wei}又写来了信,我于是再一次下了出行的决心。我不顾\xpinyin*{陈宝琛}和\xpinyin*{郑孝胥}的反对,立刻命令\xpinyin*{郑孝胥}去给我找日本总领事,我要亲自和他见面谈谈。\\

\xpinyin*{郑孝胥}听了我的吩咐,怔了一下,问道:“皇上请\ruby{加藤}{かとう},由谁做翻译呢?是\xpinyin*{谢介石}吗?”\\

我明白了他的意思。\xpinyin*{谢介石}是个台湾人,由于\xpinyin*{升允}的引见,在北京时就出入宫中,\xpinyin*{张勋}复辟时做了十二天的外务部官员,后来由日本人的推荐,在\xpinyin*{李景林}部下当秘书官,这时跟\xpinyin*{罗振玉}混在一起,不断地给我送来什么“便衣队行将举事”,革命党将对我进行暗杀等等情报。劝说我去旅顺避难的,也有他一份。\xpinyin*{郑孝胥}显然不喜欢\xpinyin*{罗振玉}身边的人给我当翻译,而同时,我知道在这个重要问题上,\xpinyin*{罗振玉}也不会喜欢\xpinyin*{郑孝胥}的儿子\xpinyin*{郑垂}或者\xpinyin*{陈宝琛}的外甥刘嚷业当翻译。我想了一下,便决定道:“我用英文翻译。\ruby{加藤}{かとう}会英文。”\\

总领事\ruby{加藤}{かとう}和副领事\ruby{冈本}{おかもと}\ruby{一策}{いっさく}、\ruby{白井}{しらい}\ruby{康}{こう}都来了。听完我的话,\ruby{加藤}{かとう}的回答是:\\

“陛下提出的问题,我还不能立即答复,这个问题还须请示东京。”\\

我心里想:这本是日本司令部对\xpinyin*{罗振玉}说没有问题的事,再说我又不是到日本去,何必去请示东京?天津的高级寓公也有到旅顺去避暑的,他们连日本总领事馆也不用通知就去了,对我为什么要多这一层麻烦?我心里的话没完全说出来,\ruby{加藤}{かとう}却又提出了一个多余的问题:\\

“请问,这是陛下自己的意思吗?”\\

“是我自己的。”我不痛快地回答。我又说,现在有许多对我不利的消息,我在这里不能安心。据日本司令部说,现在革命党派来不少便衣,总领事馆一定有这个情报吧?\\

“那是谣言,陛下不必相信它。”\ruby{加藤}{かとう}说的时候,满脸的不高兴。他把司令部的情报说成谣言,使我感到很奇怪。我曾根据那情报请他增派警卫,警卫派来了,他究竟相信不相信那情报?我实在忍不住地说:\\

“司令部方面的情报,怎么会是谣言?”\\

\ruby{加藤}{かとう}听了这话,半天没吭气。那两位副领事,不知道他们懂不懂英文,在沙发上像坐不稳似地蠕动了一阵。\\

“陛下可以确信,安全是不会有问题的。”\ruby{加藤}{かとう}最后说,“当然,到旅顺的问题,我将遵命去请示敝国政府。”\\

这次谈话,使我第一次觉出了日本总领事馆和司令部方面之间的不协调,我感觉到奇怪,也感觉到很气人。我把\xpinyin*{罗振玉}。\xpinyin*{谢介石}叫了来,又问了一遍。他们肯定说,司令部方面和接近司令部方面的日本人,都是这样说的。并且说:\\

“司令部的情报是极其可靠的。关于革命党的一举一动,向来都是清清楚楚的。不管怎么说,即使暗杀是一句谣言,也要防备。”\\

过了不多几天,我岳父\ruby{荣源}{Žung Yuwan}向我报告说,外边的朋友告诉他,从英法租界里来了\xpinyin*{冯玉祥}的便衣刺客,情况非常可虑。我的“随侍”\xpinyin*{祁继忠}又报告说,他发现大门附近,有些形迹可疑的人,伸头向园子里张望。我听了这些消息,忙把管庶务的\xpinyin*{佟济煦}和管护军的\xpinyin*{索玉山}叫来,叫他们告知日警,加紧门禁,嘱咐护军留神门外闲人,并禁止晚间出入。第二天,我听一个随侍说,昨晚上还有人外出,没有遵守我的禁令,我立刻下令给\xpinyin*{佟济煦}记大过一次,并罚扣违令外出者的\xpinyin*{饷}银\footnote{这时张园管柬“底下人”的办法,根据师傅们的谏劝和\xpinyin*{佟济煦}的恳求,已经取消了鞭\xpinyin*{笞},改为轻者罚跪,重者罚扣\xpinyin*{饷}银。为了管束,我还亲自订了一套“规则”,内容见第六章。},以示警戒。总之,我的神经紧张起来了。\\

有一天夜里,我在睡梦中忽然被一声枪响惊醒,接着,又是一枪,声音是从后窗外面传来的。我一下从床上跳起,叫人去召集护军,我认为一定是\xpinyin*{冯玉祥}的便衣来了。张园里的人全起来了,护军们被布置到各处,大门上站岗的日本巡捕(华人)加强了戒备,驻园的日本警察到园外进行了搜索。结果,抓到了放枪的人。出乎我的意料,这个放枪的却是个日本人。\\

第二天,\xpinyin*{佟济煦}告诉我,这个日本人名叫\ruby{岩田}{いわた},是黑龙会分子,日本警察把他带到警察署,日本司令部马上把他要去了。我听了这话,事情明白了七八分。\\

我对黑龙会的人物,曾有过接触。一九二五年冬季,我接见过黑龙会的重要人物\ruby{佃}{つくだ}\ruby{信夫}{のぶお}。事情的缘起,也是由于\xpinyin*{罗振玉}的鼓吹。\xpinyin*{罗振玉}对我说,日本朝野对于我这次被迫出宫和避难,都非常同情,日本许多权势人物,连军部在内,都在筹划赞助我复辟,现在派来了他们的代表\ruby{佃}{つくだ}\ruby{信夫}{のぶお},要亲自和我谈一谈。他说这个机会决不可失,应当立刻召见这位人物。\ruby{佃}{つくだ}\ruby{信夫}{のぶお}是个什么人,我原先并非毫无所闻,内务府里有人认识他,说他在\xpinyin*{辛亥}之后,常常在各王府跑出跑进,和宗室王公颇有些交情。\xpinyin*{罗振玉}的消息打动了我,不过我觉得日本总领事是日本正式的代表,又是我的保护人,理应找他来一同谈谈,于是叫人通知了\ruby{有田}{ありた}\ruby{八郎}{はちろう}总领事,请他届时出席。谁知那位\ruby{佃}{つくだ}\ruby{信夫}{のぶお}来时一看到有田在座,立刻返身便走,弄得在座的\xpinyin*{陈宝琛}、\xpinyin*{郑孝胥}等人都十分惊愕。后来\xpinyin*{郑孝胥}去责问他何以敢如此在“圣前非礼”,他的回答是:“把有田请来,这不是成心跟我过不去吗?既然如此,改日再谈。”现在看来,\xpinyin*{罗振玉}这次的活动以及\ruby{岩田}{いわた}的鸣枪制造恐怖气氛,就是那次伯信夫的活动的继续。这种活动,显然有日军司令部做后台。\\

后来我把\xpinyin*{陈宝琛}、\xpinyin*{郑孝胥}找来,要听听他们对这件事的看法。\xpinyin*{郑孝胥}说:“看起来,日本军、政两界,都想请皇上住在自己的势力范围之内加以保护。他们虽然不合作,却也于我无损。不过\xpinyin*{罗振玉}做事未免荒唐,他这样做法,有败无成,万不可过于重用。”\xpinyin*{陈宝琛}说:“不管日军司令部也罢,黑龙会也罢,做事全不负责任。除了日本公使和总领事,谁的话也别听!”我考虑了一下,觉得他们的话很有道理,便不想再向总领事要求离津了。从此,我对\xpinyin*{罗振玉}也不再感兴趣了。第二年,他便卖掉了天津的房子,跑到了大连。\\

说也奇怪,\xpinyin*{罗振玉}一走,谣言也少了,连\ruby{荣源}{Žung Yuwan}和\xpinyin*{祁继忠}也没有惊人的情报了。事隔很久以后,我才明白一点其中的奥妙。\\

这是我的英文翻译告诉我的。他和\ruby{荣源}{Žung Yuwan}是连襟,由于这种关系,也由于他和日军司令部翻译有事务上的交往,探听到一点内幕情况,后来透露了给我。原来,日军司令部专门设了一个特务机关,长期做张园的工作,和这个机关有关系的,至少有\xpinyin*{罗振玉}、\xpinyin*{谢介石}、\ruby{荣源}{Žung Yuwan}这几个人。我的英文翻译曾由这三个人带到这个特务机关的一处秘密地方,这地方对外的名称,叫做“三野公馆”。\\

他是在那天我接见了\ruby{加藤}{かとう}之后被他们带去的。他的翻译工作做完之后,被罗、谢、荣三人截住,打听会谈情况。\xpinyin*{罗振玉}等人听说\ruby{加藤}{かとう}对我出行毫不热心,立刻鼓噪起来。从他们的议论中,英文翻译听出了司令部方面有人对\xpinyin*{罗振玉}他们表示的态度完全不同,是说好了要把我送到旅顺去住的。为了向司令部方面的人汇报\ruby{加藤}{かとう}的谈话,\xpinyin*{罗振玉}等三人把英文翻译带到“三野公馆”去找那人,结果没找见,而英文翻译却发现了这个秘密地方。以后他从\ruby{荣源}{Žung Yuwan}和别的方面探听出,这是个有鸦片烟、女人、金钱的地方。\ruby{荣源}{Žung Yuwan}是这里的常客,有一次他甚至侮辱过一个被叫做大熊的日本人的妻子,大熊把他告到司令部,也没有能动他。至于\ruby{荣源}{Žung Yuwan}等人和三野公馆有些什么具体活动,\ruby{荣源}{Žung Yuwan}却不肯透露。\\

三野的全名是三野友吉,我认识这个人,他是司令部的一名少住,常随日军司令官来张园做客。当时我绝没想到,正是这个人,通过他的“公馆”,与张园的某些人建立了极亲密的来往,把张园里的情形摸得透熟,把张园里的\ruby{荣源}{Žung Yuwan}之流哄得非常听话,以至后来能通过他们,把谣言送到我耳朵里,弄得我几次想往旅顺跑。我听到我的翻译透露出来三野公馆的一些情况后,只想到日军司令部如此下功夫拉拢\ruby{荣源}{Žung Yuwan}等人,不过是为了和领事馆争夺我,他们两家的争夺,正如\xpinyin*{郑孝胥}所说,是于我有益无损的事。\\

事实上,我能看到的现象也是如此:司令部与领事馆的勾心斗角,其激烈与错综复杂,是不下于我身边的遗老们中间所发生的。比如司令部派了参谋每周给我讲说时事,领事馆就介绍了\ruby{远山}{とーやま}\ruby{猛雄}{たけお}做皇室教师;领事馆每次邀请我必同时请\xpinyin*{郑孝胥},司令部的邀请中就少不了\xpinyin*{罗振玉};领事馆在张园派驻了日本警官,而司令部就有专设的三野公馆,为\ruby{荣源}{Žung Yuwan}、\xpinyin*{罗振玉}、\xpinyin*{谢介石}等人预备了女人、鸦片,等等。\\

至于黑龙会,我了解得最晚,还是\xpinyin*{郑孝胥}告诉我的。这个日本最大的浪人团体,前身名为“玄洋社”,成立于中法战争之后,由日本浪人\ruby{平冈}{ひらおか}\ruby{浩太郎}{こうたろう}所创立,是在中国进行间谍活动的最早的特务组织,最初在福州、芝\xpinyin*{罘}(烟台)、上海都有机关,以领事馆、学校、照相馆等为掩护,如上海的“东洋学校”和后来的“同文书院”都是。“黑龙会”这个名字的意思是“超越黑龙江”,出现于一九零一年。在日俄战争中,这个团体起了很大作用,传说在那时黑龙会会员已达几十万名,拥有巨大的活动资金。\ruby{头山}{とうやま}\ruby{满}{みつる}是黑龙会最出名的领袖,在他的指挥下,他的党羽深入到中国的各阶层,从清末的王公大臣如\xpinyin*{升允}之流的身边,到贩夫走卒如张园的随侍中间,无一处没有他们在进行着深谋远虑的工作。日本许多著名的人物,如\ruby{土肥原}{どいはら}、\xpinyin*{广田}、\xpinyin*{平沼}、\ruby{有田}{ありた}、\xpinyin*{香月}等人都是\ruby{头山}{とうやま}\ruby{满}{みつる}的门生。据\xpinyin*{郑孝胥}说,\ruby{头山}{とうやま}\ruby{满}{みつる}是个佛教徒,有一把银色长须,面容“慈祥”,平生最爱玫瑰花,终年不愿离开他的花园。就是这样的一个佛教徒,在玫瑰花香气的\xpinyin*{氲氤}中,持着银须,面容“慈祥”地设计出骇人的阴谋和惨绝人寰的凶案。\\

\xpinyin*{郑孝胥}后来能认识到黑龙会和日本军部系统的力量,是应该把它归功于\xpinyin*{罗振玉}的。郑、罗、陈三人代表了三种不同的思想。\xpinyin*{罗振玉}认为军部人物以及黑龙会人物的话全是可靠的(他对\ruby{谢米诺夫}{Семёнов}和\xpinyin*{多布端}的信任,也一半是出于谢、多二人和黑龙会的关系),\xpinyin*{陈宝琛}则认为除了代表日本政府的总领事馆以外,别的日本人的话全不可信。\xpinyin*{郑孝胥}公开附和着\xpinyin*{陈宝琛},以反对\xpinyin*{罗振玉}。他心里起初也对司令部和黑龙会存着怀疑,但他逐渐地透过\xpinyin*{罗振玉}的吹嘘和黑龙会的胡作非为,看出了东京方面某种势力的动向,看出了日本当局的实在意图,最后终于看出了这是他可以仗恃的力量。因此,他后来决定暂时放下追求各国共管的计划,而束装东行,专门到日本去找黑龙会和日本参谋总部。
