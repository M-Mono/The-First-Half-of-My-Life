\fancyhead[LO]{{\scriptsize 1917-1924: 北京的“小朝廷” · 在北府里}} %奇數頁眉的左邊
\fancyhead[RO]{} %奇數頁眉的右邊
\fancyhead[LE]{} %偶數頁眉的左邊
\fancyhead[RE]{{\scriptsize 1917-1924: 北京的“小朝廷” · 在北府里}} %偶數頁眉的右邊
\chapter*{在北府里}
\addcontentsline{toc}{chapter}{\hspace{1cm}在北府里}
\thispagestyle{empty}
我说了那几句漂亮话,匆匆走进了国民军把守着的北府大门。我在父亲的书房里坐定,心想我这不是在王府里,而是进了虎口。我现在第一件要办的事,就是弄清楚究竟我的处境有多大危险。我临出宫以前,曾叫人送信给宫外的那些“股肱之臣”,让他们从速设法,营救我逃出国民军的掌握。这时,不但他们的奔走情形毫无消息,就连外边的任何消息也都无法知道。我很想找人商量商量,哪怕听几句安慰话也好。在这种情势下,我的父亲让我感到了极大的失望。\\

他比我还要惊慌。从我进了北府那一刻起,他就没有好好地站过一回,更不用说安安静静地坐一坐了。他不是喃喃自语地走来走去,就是慌慌张张地跑出跑进,弄得空气格外紧张,后来,我实在忍不下去了,请求他说:\\

“王爷,坐下商量商量吧!得想想办法,先打听一下外边的消息呀!”\\

“想想办法?好!好!”他坐了下来,不到两分钟,忽然又站起来,“\ruby{载洵}{zǎi xún}也不露面了!”说了这句牛头不对马嘴的话,又来来去去地转了起来。\\

“得打听打听消息呵!”\\

“打,打听消息?好,好!”他走出去了,转眼又走进来,“外边不,不让出去了!大门上有兵!”\\

“打电话呀!”\\

“打,打电话,好,好!”走了几步,又回来问:“给谁打电话?”\\

我看实在没办法,就叫太监传内务府大臣们进来。这时内务府大臣\xpinyin*{荣源}住进了外国医院,治神经病去了(两个月后才出来),\xpinyin*{耆龄}忙着搬移我的衣物,处理宫监、宫女的问题,宝熙在照顾未出宫的两位太妃,只剩下\xpinyin*{绍英}在我身边。他的情形比王爷好不了多少,一个电话也没打出去。幸亏后来其他的王公大臣和师傅们陆续地来了,否则北府里的慌乱还不知要发展到什么地步。\xpinyin*{庄士敦}在傍晚时分带来的消息是最好的:经过他的奔走,公使团首席公使荷兰的欧登科、英国公使麻克类、日本公使\xpinyin*{芳泽}已经向摄政内阁外交总长王正廷提出了“抗议”,王正廷向他们保证了我的生命财产的安全。这个消息对北府里的人们起了镇定作用,但是对于我父亲,好像“剂量”还不足。\xpinyin*{庄士敦}在他的著作里曾描写过那天晚上的情形:\\

皇帝在一间大客厅里接见了我,那间屋子挤满了满洲贵族和内务府的官员。……我的第一个任务,是说明三位公使拜访外交部的结果。他们已经从\xpinyin*{载涛}那里,知道了那天早晨我们在荷兰使馆进行了磋商,所以他们自然急于要知道,和王博士(正廷)会见时的情形。他们全神贯注地听我说话,只有\xpinyin*{醇亲王}一人,在我说话的时候不安地在屋里转来转去,显然是漫无目的。有好几次忽然加快脚步,跑到我跟前,说了几句前言不搭后语的话。他的口吃似乎比平时更加厉害了。他每次说的话都是那几句,意思是\\

“请皇上不要害怕”——这句话从他嘴里说出,完全是多余的,因为他显然要比皇帝惊慌。当他把这种话说到四五次的时候,我有点不耐烦了,我说,‘皇帝陛下在这里,站在我旁边,你为什么不直接和他说呢?’可是,他太心慌意乱了,以致没有注意到我说话的粗鲁。接着,他又漫无目的地转起圈子来。……\\

那天晚上,我父亲的另一举动,尤其令我不能满意。\\

\xpinyin*{庄士敦}到了不久,\xpinyin*{郑孝胥}带着两个日本人来了。从“东京震灾”捐款时起,东交民巷的日本公使馆就和我的“股\xpinyin*{肽}”们有了交际,\xpinyin*{罗振玉}和\xpinyin*{郑孝胥}来到紫禁城之后,又和日本兵营有了往来。\xpinyin*{郑孝胥}这时和东交民巷的竹本多吉大性商定了一条计策,由竹本的副官中平常松大尉,穿上便衣,带着一名医生,假装送我进医院,把我运出北府,接进日本兵营。\xpinyin*{郑孝胥}带着中平大尉和日本医生村田到了北府,说出了他们的计策,但是遭到了王公大臣和师傅们的一致反对。他们认为这个办法很难混过大门口的士兵,即使混过了他们,街上还有国民军的步哨,万一被发现,那就更糟糕。我父亲的态度最为激烈,他的反对理由是这样:“就算跑进了东交民巷,可是\xpinyin*{冯玉祥}来找我要人,我怎么办?”结果是\xpinyin*{郑孝胥}和日本人被送出大门去了。\\

到了次日,北府的门禁突然加严,只准进,不准出。后来稍放松一点,只许陈、朱两师傅和内务府大臣出进,外国人根本不许进来。这一下子,北府里的人又全慌了神,因为既然国民军不把洋人放在眼里,那就没有可保险的了。后来两个师傅分析了一下,认为历来还没有不怕洋人的当局,王正廷既向三国公使做出保证,料想他不会推翻。大家听了,觉得有理,我却仍不放心。话是不错,不过谁知道大门口的大兵是怎么想的呢?那年头有句话:“秀才遇见兵,有理讲不清!”\xpinyin*{黄郛}和王正廷尽管如何保证,离我最近的手持凶器的还是门口的大兵。万一他们发作起来,就怕一切保证都不顶事。我越想越怕,后悔没有跟\xpinyin*{郑孝胥}带来的日本人出去,同时心里也埋怨父亲只考虑自己,却不顾我的安危。\\

正在这时候,\xpinyin*{罗振玉}从天津回来了。他是在冯军接管内城守卫的时候乘坐京津国际列车\footnote{内战中,火车常被军阀扣留,京津间交通很不正常,因这趟车是根据东交民巷的意思组成的,所以交战双方都不敢动它}到天津求援去的。他到了天津日本驻屯军司令部,司令部的金子参谋告诉他,鹿钟麟已进了宫,日本司令官叫他去找\xpinyin*{段祺瑞}。这时\xpinyin*{段祺瑞}也接到了北京竹本大住转来\xpinyin*{郑孝胥}的求援电报。\xpinyin*{段祺瑞}发出了一封反对\xpinyin*{冯玉祥}“逼宫”的通电。\xpinyin*{罗振玉}看了那个电稿,明白了\xpinyin*{段祺瑞}马上就要出山,觉得形势并不那么严重,不过他仍然要求日军司令部出面“保护”。日军司令部告诉他,北京的竹本大往会有办法。根据日本驻屯军司令部的指示,他返回北京找到竹本大住,竹本大住叫他告诉我,日本骑兵将在北府附近巡逻,如国民军对北府有什么异样举动,日本兵营会立即采取“断然措施”。\xpinyin*{陈宝琛}也告诉我,日本兵营想把日本军用信鸽送进北府,以备报警之用(后来因为怕国民军知道,没敢收),于是我对日本人的“感情”又发展了一步。这样一来,\xpinyin*{罗振玉}在我心里得到了与\xpinyin*{郑孝胥}相等的地位,而王爷就被挤得更远了。\\

我看到了\xpinyin*{段祺瑞}指摘\xpinyin*{冯玉祥}“逼宫”的通电,又听到了奉军将要和冯军火并的消息,这两件事给我带来了新的希望。与此同时,\xpinyin*{陈宝琛}给我拿来了日本兵营转来的\xpinyin*{段祺瑞}的密电,上面说:“皇室事余全力维持,并保全财产。”接着门禁有了进一步的松动,允许更多的王公大臣以至宗室人等进来,甚至连没有“顶戴”“功名”的\xpinyin*{胡适}也没受到阻拦,只有\xpinyin*{庄士敦}还是不让进来。\\

不久,北府所最关心的张、冯关系,有了新的发展,传来了\xpinyin*{冯玉祥}在天津被奉军扣押的消息。后来虽然证明是谣传,但是接踵而至的消息更鼓舞了北府里的人:国民军所支持的黄部摄政内阁,在北京宴请东交民巷的公使,遭到了拒绝。北府里乐观地估计,这个和我过不去的摄政内阁的寿命快完了,代替他的自然是东交民巷(至少是日本人)所属意的\xpinyin*{段祺瑞}。果然,第二天的消息证实了\xpinyin*{罗振玉}的情报,\xpinyin*{冯玉祥}不得不同意\xpinyin*{张作霖}的决定,让\xpinyin*{段祺瑞}出山。过了不多天,张、段都到北京来了。那几天的情形,\xpinyin*{郑孝胥}的日记里是这样记载的:\\

乙\xpinyin*{巳}廿六日(十一月二十二日)。小雪。作字。日本兵营中平电话云:\\

\begin{quote}
	\xpinyin*{段祺瑞}九点自天津开车,十二点半可到京。偕大七(郑的长子\xpinyin*{郑垂})往迎\xpinyin*{段祺瑞}于车站。……三点车始到,投刺而已。……\\

丙午分七日(二十二日)。……曹襄衡(段的幕僚)电话云:段欲公为阁员,今日请过其居商之。答之曰:不能就,请代辞,若晤面恐致龃龉。至北府入对。泽公,伒贝子、耆寿民(龄)询余:就段否?余曰:\\

拟就其顾问,犹虑损名,苟不能复辟,何以自解于天下?伒贝子曰:若有利于皇室,虽为总统何害?……\\

丁未廿八日(二十三日)。……北府电话召。入对。上(\xpinyin*{溥仪})赐膳,裁两器、两盘、数小碟而已。段派\xpinyin*{荫昌}来,守卫兵得其长官令:不禁止洋员(指\xpinyin*{庄士敦})入见。涛贝勒云:顷已见段,求撤卫兵,但留警察。使垂访池部(日公使馆书记官)。上云:今日已派柯助志、\xpinyin*{罗振玉}商购裱褙胡同盛昱之屋,将为行在。……\\

戊申二十九日(二十五日)。……至吉兆胡同段宅晤段芝泉(棋瑞),\\

谈久之。至北府,入对。……\\

己百三十日(二十六日)……召见,草赐\xpinyin*{张作霖}诏,\xpinyin*{罗振玉}书之。诏云:“奉军入京,人心大定,威望所及,群邪敛迹。昨闻\xpinyin*{庄士敦}述及厚意,备悉一切。予数年以来,固守官中,囤子闻见,乘此时会,拟为出洋之行,惟筹备尚须时日,日内欲择暂驻之所,即行移出醇邸。\xpinyin*{俟}料理粗定,先往盛京,恭谒陵寝。事竣之日,再谋游学海外,以补不足。所有详情,已属\xpinyin*{庄士敦}面述。”……北府冯军撤回。\xpinyin*{冯玉祥}求免职,段批假一月。闻冯已赴西山。……\\
\end{quote}

段、张合作的消息一传出,北府的气氛就变了。王公们首先给\xpinyin*{张作霖}秘密地写了一封信,请求他庇护。张、段入京后,王公们派了代表和\xpinyin*{郑孝胥}一齐表示欢迎,然后又分头进行活动。由\xpinyin*{郑孝胥}去找\xpinyin*{段祺瑞},北府的管家\xpinyin*{张文治}去找他的盟见\xpinyin*{张作霖}。让北府最高兴的,是\xpinyin*{张作霖}托\xpinyin*{张文治}特别邀请\xpinyin*{庄士敦}去一趟。结果\xpinyin*{庄士敦}去了两趟。\xpinyin*{张作霖}找\xpinyin*{庄士敦}的目的,是想通过\xpinyin*{庄士敦}探一探东交民巷对他的态度,而北府里则希望通过\xpinyin*{庄士敦}探一探\xpinyin*{张作霖}对我的态度。我让\xpinyin*{庄士敦}带去了我的一张签名照片,一个大钻石戒指。\xpinyin*{张作霖}留下照片,退了戒指,表示了同情。与此同时,\xpinyin*{段祺瑞}向\xpinyin*{郑孝胥}表示了,可以考虑恢复优待条件。既有了东交民巷的“同情”,又有了这两位当权人物的支持,虽然\xpinyin*{冯玉祥}的国民军还在北京城里,而北府的人们已经敢于“反攻”了。\\

十一月二十八日,即大门上的国民军撤走、\xpinyin*{冯玉祥}通电辞职的第二天,北府里用内务府的名义发出了致国民内务部的一封公函:\\

\begin{quote}
	……查法理原则关于刑律之规定,凡以强暴胁迫人者,应负加害之责任,其民法原理凡出于强暴胁迫,欺\xpinyin*{罔}恐吓之行为,法律上不能发生效力。\\

兹特专函声明:所有摄阁任意修正之五条件,清室依照法理不能认为有效。\\

……\\
\end{quote}

与此同时还发出了向外国公使们呼吁支援的公函。对摄阁成立时组成的“清室善后委员会”,虽清室代表已参加开了几次会,现在也否认了。\\

这天,日本人办的《顺天时报》记者来访问我,我向他发表了谈话,与出宫那天所说的完全相反:\\

\begin{quote}
	此次国民军之行动,以假冒国民之巡警团体,武力强迫余之签字,余决不如外间所传之欣然快诺。……\footnote{这是记者报道的文字,登在民国十四年十一月二十九日的《顺天时报》上,基本和我当时的思想一致。}\\
\end{quote}

《顺天时报》是日本公使馆支配下的日商报纸。说到当时日本人对我的“热心”,决不能忽略了这份报纸。它不像竹本大住那样的一切在暗中进行,而是依仗特权公然地大嚷大叫,极尽耸动听闻之能事。从我进了北府的第二天起,《顺天时报》连续发出了对“皇室”无限“同情”,对摄政内阁和国民军无限“激愤”的消息和评论。里面大量地使用了“逼宫”、“蒙难”之类的字眼,以及“泰山压卵”、“欺凌寡妇孤儿”、“绑票”等等的比喻,大力渲染和编造了“旗人纷纷自杀”,“蒙藏发生怀疑”等等的故事,甚至还编造了“某太妃流血殉清朝”,“淑妃断指血书,愿以身守宫门”和“淑妃散发攀轮,阻止登车”的惊人奇闻。其他外文报纸虽也登过类似的文字,但比起《顺天时报》来,则大为逊色。
