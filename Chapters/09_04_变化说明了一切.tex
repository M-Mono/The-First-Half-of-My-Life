\fancyhead[LO]{{\scriptsize 1955-1959: 接受改造 · 变化说明了一切}} %奇數頁眉的左邊
\fancyhead[RO]{} %奇數頁眉的右邊
\fancyhead[LE]{} %偶數頁眉的左邊
\fancyhead[RE]{{\scriptsize 1955-1959: 接受改造 · 变化说明了一切}} %偶數頁眉的右邊
\chapter*{变化说明了一切}
\addcontentsline{toc}{chapter}{\hspace{1cm} 变化说明了一切}
\thispagestyle{empty}
三天参观结束归来时的情绪,和第一天出发时正是一个强烈的对比。兴奋的谈论代替了抑郁的沉默。一进监房就开始谈论,吃饭时谈,开小组会时谈,开完会还是谈,第二天也是谈,谈的全是参观。从各号的议论里可以不断听到的是这句话:\\

“变了!社会全变了,中国人全变了!”\\

这真是一句最有概括力的话。“变了!”这本是几年来我们从报上,从所方的讲话,以及从通信中常常接触到的事实,但是有些饱经世故者越是间接知道得多,越是想直接地核对一下,我们组里的老元就是这样的人。这回,他也服了。\\

这天晚上,我们谈到工人保健食堂的蛋糕,那是我们亲自尝过的,谈到工人的伙食,那是我们亲眼看到了的,说到工人宿舍的瓦斯灶,有人说可惜只看见烧水,没看见做的是什么饭,这时候老元接口道:“我倒看了一下。”\\

大家先很惊异,他是和别人一起走的,怎么他会看见?经他一说才明白,原来别人注意工人宿舍里的陈设,他却走到屋子后面,看了人家的垃圾箱。他发现了那里面有鱼骨头、鸡蛋皮以及其它东西。\\

做过东北军小粮秣官的“兴农部大臣”老南,平常话很少,今天他也显得比平常活跃了,他说:“不但在伪满,工人家里找不出鱼肉来,就是‘九·一八’以前也不多见。我可是小职员出身的……”\\

从小被日本人培养起来的老正,坦率地说出了心里话:“我以前看报纸、学文件,有时信,有时就怀疑,我总想,什么东北工业基地,还不是日本人给留下的?这回看见了工业学校附属的工厂,把日本老皮带式的车床挤到一边,到处都是国产的崭新设备,我才相信真是中国人翻了身。这真是变了!”\\

变了!——这句话引起我的共鸣,我另有自己的感受。\\

我受到了人民的宽恕,由于过分出乎意料,这三天参观当中老是想着:这是真的吗?他们受了汉奸那么多的罪,竟肯拉倒了吗?他们相信毛主席的改造罪犯的政策,竟是到了这种程度吗?这是什么原因?\\

方素荣和台山堡的过去和今天,也是东北人民的过去和今天。标志着这种由悲苦到欢乐的变化的,在抚顺到处都可以遇到。平顶山上的烈士碑和新生的丛林,露天矿四周残留火区的尘烟和新建的电气火车轨道,地下矿一百五十多公里巷道中的每根旧坑木和每段新砌的混凝土顶壁,露天矿旧址上“臭油房”的残迹和人民政府新建的工人宿舍大楼,以及市区里用日本高级旅馆改造的工人养老院,用日本高级员司宿舍改造的托儿所,还有各矿场新建的保健食堂、太阳灯室,等等,总之,每条街道。每座建筑、每台机器、每串数目字以至每块石头,都向我诉说着过去的血泪和今天的幸福,都告诉我这里经历了怎样的天翻地覆的变化。一切都让我思索着,刘大娘为什么要说“过去的让它过去”?那个残废青年为什么会说他相信我们能改造?……\\

变化说明了一切。\\

变了!——这句话里包含着抚顺矿工过去多少血泪!\\

抚顺,这个过去闻名于关内的千金寨(现在露天矿矿址),在大半个世纪之前,关内就有一首歌谣形容它的富饶:“都说关外好,千里没荒草,头上另有天,金银挖不了。”但是从一九零一年开采以来,挖出来的“金银”就不是矿工的,对矿工来说,是另一首歌谣里的生活:“一到千金寨,就把铺盖卖,新的换旧的,旧的换麻袋。”一九零五年帝俄在辽东失败,这地方就成了日本人的囊中物。在整整四十年的岁月中,抚顺矿工被折磨死的据估计有二十五万至三十万人。\\

从山东、河北被骗来的和东北当地破产的农民,每年成批地来到抚顺矿区,大多数是住在一二百人一间的“大房子”里,无论春夏秋冬只有一身破烂,每天十二小时以上的劳动,得到的有限的工资还得由大柜、把头剥几层。矿工说:“鬼子吃咱肉,把头啃骨头,腿子横着走,工人难抬头。”\\

有家室的工人住在“臭油房”里,过着少吃无穿的生活。有的孩子生下来,光着身子长到几岁;饿死了,还是光着身子埋掉。\\

更多的人是结不起婚,龙凤矿在解放前百分之七十的人是单身汉。\\

矿井里谈不上安全设备。爆炸、冒顶、片帮是常事。工人说:“要想吃煤饭,就得拿命换。”一九一七年,有一次大山坑发生瓦斯爆炸,日本人为了减少煤炭损失,把坑口封闭,九百十七个矿工被活活烧死在里面。一九二三年,老万坑内发火,又因同样的措施有六十九个工人死在里面。一九二八年大山坑透水,淹死工人四百八十二名。\\

伪满政权做过统计:一九一六至一九四四年,伤亡人数共计二十五万一千九百九十九名。\\

每次事故发生,矿工家属从四面八方涌向井边,哭声震野……\\

矿工被炸死的、烧死的、冻死的、饿死的、病死的,除了在井里埋在煤堆和泥沙里的,全被扔到一个叫南花园的地方的北面山沟里。这个山沟早被死人填满了,因此有了一个“万人坑”的名称。\\

日本人给工人们除了皮鞭、臭油房之外,还弄了一个叫“欢乐园”的地方,那里有上千名妓女,有赌场,有鸦片馆和吗啡馆,还有老君庙。\\

抚顺不仅有日本人的华丽的住宅、高耸人云的卷扬塔,还有老君庙旁成堆的乞丐、杨柏河旁和臭水沟里的死猫和死婴。冬天,天天有新尸体出现在杨柏桥下,——这里是被剥夺得无路可走的失业工人过宿的地方,它的外号叫“大官旅馆”。今夜在这里睡下的人,明早也许就是一具新的“路倒”。\\

伪满时期,抚顺增添了一个机构:矫正辅导院。这是“反满抗日”的矿工的集中营,进去的人在毒打之后,就在刺刀、机枪。警犬包围下从事奴隶劳动。他们像牲畜一样住在一起,冬天常有人冻死在炕上。\\

“变了!”这句话又包含着多少翻天覆地的事件!多少令人激动的欢乐!\\

在露天矿,有日本人在三十一年间给工人建筑的三千五百平方米的臭油房的遗迹,也有解放后七年间新建的十七万平方米的宿舍大楼。\\

第三天参观龙凤矿,我看见了工人宿舍里面的工人家庭的住室。这家也许就是从前那百分之七十里的一个。墙上的双影照片上,那个中年男人拘谨地微笑着,大概他就是解放后已婚的百分之八十中的一个吧?\\

在这个家庭的厨房里,我看见了瓦斯灶的蓝色的火苗……\\

这个给人以安定、温暖感觉的火苗,它原先是多么令人恐怖,它曾毁灭了多少家庭,叫多少妻子哭断肝肠呵!它今天给了人们温暖和幸福,但人们谈起那次征服瓦斯的斗争,人们心中的温暖和幸福,更是无比巨大的!\\

我们走在空气新鲜的、略觉微风迎面的龙凤矿的巷道里,在一望无际的日光灯照明之下,矿办公室王主任一边走着一边给我们讲了下面这个动人心弦的故事。\\

瓦斯,这一直是各国采煤史中的最凶恶的敌人,已不知有多少矿工的生命被它夺去。龙凤、胜利、老虎台三矿都是超级瓦斯矿。解放初期,三个矿井仍处在瓦斯的严重的威胁之中,尤其是龙风矿,被日本鬼子和国民党先后破坏,井下巷道大部崩坍堵塞,窝满了浓烈的瓦斯,以致采煤都不敢用爆破和电动设备。矿区当局为迅速消除瓦斯威胁,保证生产安全,采取了各种措施,依靠有经验的老工人对瓦斯进行了不懈的斗争,取得了初步的胜利,曾使采煤每吨的瓦斯喷出量由六十四点八立米降到三十六立米。后来,在矿区当局工人们不断努力和斗争的情况下,又出现了新的奇迹。\\

一九四九年秋天,东北工业部门掀起了一个热火朝天的新纪录运动,原龙凤矿的一位工程师向党委提出一项在旧时代根本没有人理睬、而工人们多少年来梦想过的理想,这个具有科学根据的理想是:开辟井下瓦斯巷道,根据瓦斯比空气轻、能透过煤层上升的原理,使煤层中的大量瓦斯自动聚在巷道里,然后用铁管引到地面上来,这样既可以把瓦斯用于福利,也为解决瓦斯为害问题找出了一条道路。\\

这个建议立刻得到矿区党委的重视,党相信这个建议,并且给工程师以最大的鼓励和支持。这个理想也引起了工人们,特别是老工人们和工人家属的热烈支持,有经验的老工人纷纷表示要为实现这理想贡献自己的全部力量。于是在党委组织下,这位工程师和一批勇敢的工人们进行了伟大的试验。工人党员们走在战斗的最前面,在浓厚的瓦斯巷道里夜以继日地奋战着。起初,他们遇到了不少的困难,受到过多次浓烈瓦斯的包围,也受到过胆怯和保守的议论冷风的吹袭,但一个个困难都被克服了,终于在一九五零年七月一日前夕完成了试验工程。“七一”进行试验那天,在瓦斯出口管周围附近,自动集聚了越来越多的工人家属和歇班工人,也来了无数的机关干部和上学的孩子们,人们都要亲眼看着自己的梦想如何变成现实。当一根火柴在管口燃起了猛烈的蓝色火苗时,欢呼声响遍了矿区,震动了矿山。人们向工程师和勇敢的工人祝贺。后来,他们的眼睛从蓝色的火焰上移开,都不约而同地集中到卷扬塔上光芒四射的红星上了\footnote{龙凤矿每逢有重大新成就,卷杨塔上红星即放光,全矿可见。}。此时老工人和老大娘们个个泪流满面,年轻的工人高呼着:“我们又胜利了!”\\

这个故事立刻让我想起,我在抚顺工人养老院看见的那位残废的老人。这是一次瓦斯爆炸中的幸免者。他逃脱了死亡,但是逃不脱困残废被赶出矿山的厄运。他过着乞讨生活,一直到解放;他几次几乎变成杨柏桥下的“路倒”。老人辛苦一生,没有结过婚,世上没有一个亲人。在他的床头上方,这个照例是放置亲人照片的地方,老人也有一个用精致的镜框镶起的照片,这也是他的房间里推一的一张照片:毛主席。\\

这个故事立刻让我想起,上午在一个幼儿院里,系着雪白小围巾的孩子挥动着小胖手唱的歌曲:“没有共产党,就没有新中国……”\\

从这些联想中,使我从老人和孩子那里得到了一个统一的回答。我明白了为什么刘大娘要说过去的让它过去,我明白了为什么她的儿子会相信我们可以改造……\\

我们随着王主任在巷道里继续前进着。在一个拐角的地方出现了一个灯光耀眼的小卖部——里面有水果点心,毛巾手绢,木梳香皂——王主任在这里停下来,指着小卖部说:\\

“在伪满时,从这里起是一条长长的臭水沟。沟里沟外到处有老鼠跑,可是谁也不敢碰它,因为那时很多工人很迷信,说它是老君爷的马。工人们都是混过今天不知能不能混过明天的人,因此,有的为了求平安,就敬信了老君爷。那时我们是又受鬼子的气,又受二把头的气,还要受老鼠的气。现在当然谁家也没老君爷了,把老君爷扔了,家家挂上毛主席像了。”\\

他指着混凝土的干净平整的地面继续说:“那时到处是水,浅处也有一尺左右。工人一下井,就得光脚蹚水走。在‘掌子’里,工人浑身都不穿一点衣服,精光光的。坑下又问又热,再说只有一身破烂,烂掉了也没人给你添。”\\

我们继续向前走,走到电车道旁,载运着发光的煤块的列车开过去了,穿着深蓝色工作服的司机和王主任笑着打个招呼,驶过去了。王主任继续说:\\

“那时候有电车走的道,没人走的道。电车在这个地方就常撞死人。不过比起爆炸死人,那又不算什么了。矿工过去有句话:说自己是‘四块石头夹一块肉’。在井下干了十几个钟头回到井上来,就算这一天又混过来了。在井口外面,天天下工时候有一群女人孩子等着,要是等不到自己的人,那就是完了。连尸首都不一定找到,不是压在石头底下,就是叫水沙埋了。在这里,”他停下了,指着路边说:“我亲自看见在这里压死了四个人。我十四岁就下井,自己也说不清跟阎王老子打了多少次交道。”\\

我这才知道这位精通业务的年轻主任原是矿工出身。他是个爽朗、活泼的人,他最后那句话是笑着说的。我决没料到站在我们面前的这个爱笑的人,过去的经历是那样悲惨,简直难以想象他是怎么熬过来的。为了生活,当年,这个十四岁的少年每天要干十几个钟头的活,有了病,不敢躺下,因为怕被看做有传染病隔离起来。工人们住的大房子,冬天没有火,大多数人没铺没盖,有条麻袋算好的,吃的也不够,每天每人只有八个蜂窝似的窝窝头,因此,传染病是极容易发生的。一九四二年,这里发生的一场流行病,工人们到今天提起来还是余悸未定。可怕的不是疫病,而是日本人的毒手,日本人曾把发生疫情的工人住宅区用层层刺网封锁起来,不准外出求医,然后又逐家检查,如果谁家有病不报告,日本鬼子就把大门钉起来,把人封锁在里头。如果有病报告了,又不管什么病一律填个霍乱,送进隔离所。人一进了隔离所就不用想出来,外面有电网围着,洋狗看守着,每人每顿一碗粥,有的半死不活,就送到炼人炉里烧死,或者和死人一起扔到万人坑里。\\

“刚才你们看见的煤车上的那个工人,”王主任脸上的笑容消失了,“他叫邢福山,他的父亲就是被活埋的一个。”\\

我们慢慢走着,巷道里有轻风迎面拂来,这是清新的温暖的气流,但我的心被过去的事冻结住了。经过一阵短暂的沉默,王主任继续说:“从前这里的空气是混浊的,不干活也可以把人间出病来。有一回我刚从井里上来,问得要死,有了病了,二把头非叫我再下去不行,我不去,他举起皮鞭打我。我在大房子里最小,大伙全疼我,有人过来要和二把头拆命,那小子一看就吓跑了。日本鬼子和二把头最怕的是特殊工人——这是鬼子送给被俘的八路军战俘的名称,鬼子把他们押到矿上做工,这些战士对鬼子不买账,谁凶他们在井底下就揍谁,揍死了就埋在里面。他们暴动了好多次,鬼子只好让步,给他们吃好一点,客气一点。鬼子和二把头怕普通工人受到特殊工人的影响,总设法隔离开,可是我们也知道了他们的斗争,也就摸透了鬼子和二把头的底,所以二把头只好扔下鞭子跑了,倒真像臭沟里的老鼠一样。从那天起,我就看透这些人日子长不了……”\\

这个当初生活在爆炸、冒顶和二把头皮鞭下的少年,他怎么熬过来的,我明白了,而且我的问题又一次得到了回答。在他身上有多么强烈的自信!当初他在那样艰难的朝不保夕的生活中,就已经看透了鬼子和二把头的底细,而我在那时是什么样子呢?是已吃腻了荤腥,丢尽了尊严,天天打针吃药,内心充满了末日的情绪。这和当初的这个少年的心情是多么强烈的对照!在那样的日子里,他就把我们这类人看成了老鼠,微不足道,在今天又是怎样呢?\\

我想起了试验瓦斯胜利的那个故事,想起故事里的老年工人和家属们的眼泪,想起故事里的青年工人高呼的那句话:“我们又胜利了!”这句话里充满了多大的自豪和自信!在他们的眼里,社会、人类、自然,一切奥秘都是可以揭穿的,一切都是可以改造的!一个皇帝又算得了什么?未来是他们的!这是为什么方素荣、刘大娘和他的儿子所以能宽恕我的又一个原因。\\

一切都变了!变化是反映在任何事物上的。从平顶山上的新生的丛林到矿山上的每块石头,都有了变化。变化也反映在我们所看到的各种人身上:养老院里正展开比健康、比长寿的老人是变化,工人宿舍的瓦斯灶和结婚照片是变化,年轻的王主任也是一个变化……一切变化中最根本的,是人的变化。\\

说明这一切变化发生的原因的,是老人床头的照片,是幼儿院孩子们唱的歌,是龙凤矿卷扬塔上的那颗星……\\

在那颗红星下发生了这一切——伟大的胸怀,对领袖的无限信仰,看透了一切的自信。有了这一切,才有了那个声出如雷鸣,耀眼如闪电的宽恕。