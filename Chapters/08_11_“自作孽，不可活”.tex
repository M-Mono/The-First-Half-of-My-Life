\fancyhead[LO]{{\scriptsize 1950-1954: 由抗拒到认罪 · “自作孽,不可活”}} %奇數頁眉的左邊
\fancyhead[RO]{} %奇數頁眉的右邊
\fancyhead[LE]{} %偶數頁眉的左邊
\fancyhead[RE]{{\scriptsize 1950-1954: 由抗拒到认罪 · “自作孽,不可活”}} %偶數頁眉的右邊
\chapter*{“自作孽,不可活”}
\addcontentsline{toc}{chapter}{\hspace{1cm} “自作孽,不可活”}
\thispagestyle{empty}
问题之严重,还不仅限于此。\\

日本战犯的坦白、揭发和东北人民群众的控诉、检举,使我们“一所”激动起来了。尤其是那些年纪轻的人,反应分外强烈。在这种情形下,我遭到了侄子、妹夫和大李的揭发。我陷入了来自四面八方的仇恨中,其中包括了家族的仇恨。我犹如置身镜子的包围中,从各种角度上都可以看到自己不可入目的形象。\\

这是从我们一所的一次全体大会开始的。那天我们参加过日本战犯的学习大会,工作团的人员把我们召集起来,要大家谈谈感想和认识。许多人从日本战犯大会上感染到的激情犹未消失,这时纷纷起立发言,自动坦白出自己的罪行,并且检举了别人。人们检举比较集中的是前伪满司法大臣张焕相。他在“九·一八”事变前,做过东北讲武堂教育长、哈尔滨特区行政长官和东北军航空司令。“九·一八”事变后,他从关内跑到抚顺老家,千方百计地巴结日本人,给统治者献计献策,上了四十二件条陈,因此,得到了关东军的赏识,并由军政部嘱托爬上司法大臣的位子。他有许多出名的举动,其中一件是他在被起用之前,在家里首先供奉日本神武天皇的神\xpinyin*{龛},每逢有日本人来找他,他必先跪在神\xpinyin*{龛}前做好姿势等着。另一件是,他曾在抚顺亲督民工修造神武天皇庙,修成后和他老婆每天亲自打扫。在人们的检举声中,他吓得面无人色。后来人们提到他人所以来的种种对抗举动,例如故意糟踏饭菜、破坏所内秩序、经常对看守员大喊大叫,等等,引起了全场人的忿怒。有人向他提出警告,如果今后再不老实,还要随时揭发他,政府也不会饶他。我很怕也被别人这样当场检举,很怕别人也认为我不老实。由于这次检举与认罪,不准彼此透露材料,我怕别人不知道我已做了坦白,觉得有必要在大会上谈谈,表明我的态度。因此,我也发了言。在我讲完了坦白材料之后,刚要说几句结束话,再表明一下认罪决心的时候,不想小固忽然从人丛中站起来,向我提出了质问:\\

“你说了这么多,怎么不提那个纸条呢?”\\

我一下怔住了。\\

“纸条!小瑞的纸条!”小秀也起来了,“那些首饰珍宝你刚才说是自动交出的,怎么不说是小瑞动员的呢?”\\

“对,对,”我连忙说,“我正要说这件事。这是由于小瑞的启发……”\\

我匆匆忙忙补充了这件事,而小固、小秀还是怒目相视,好像犹未甘心的样子。幸亏这个大会到此就结束了。\\

我回到监房里,赶紧提笔写了一个检讨书给所方。我想到所长知道了一定很生气的,心里不由得埋怨小瑞,于什么把这件事告诉小固和小秀呢?小固和小秀未免太无情了,咱们到底是一家人,你们不跟老万和老润学,竟连大李也比不上!过了不久,我看到了他们写的书面检举材料,才知道家里人的变化比我估计到的还要可怕。\\

按照规定,每份检举材料都要本人看过。赵讯问员拿了那堆检举材料,照例地说:\\

“你看完,同意的签字,不同意的可以提出申辩。”\\

我先看过了一些伪大臣写的。这都是伪满政权的公开材料,我都签了字。接着便看我的家族写的。我看了不多页,手心就冒汗了。\\

老万的检举材料里,有一条是这样写着的:\\

\begin{quote}
	一九四五年八月九日,晚上我入宫见溥仪。溥正在写一纸条,此时张景惠及武部六藏正在外间屋候见。溥向我出示纸条,内容大意是:令全满军民与日本皇军共同作战,击溃来侵之敌人(苏军)。溥谓将依此出示张景惠等,问我有何见解。我答云:只有此一途,别无他策。\\
\end{quote}

我心想这可毁了!我原把这件事算在吉冈的账上了。\\

大李的检举,更令我吃惊。他不但把我离开天津的详情写了,而且把我写自传前跟他订“攻守同盟”的事情也写上了。\\

事情不仅仅是如此。他们对我过去的日常行为——我怎么对待日本人,又怎样对待家里的人——揭露得非常具体。如果把这类事情个别地说出一件两件,或者还不算什么,现在经他们这样一集中起来,情形就不同了。例如老万写的有这么一段:\\

\begin{quote}
	在伪宫看电影时,有天皇出现即起立立正,遇有日兵攻占镜头即大鼓掌。原因是放电影的是日本人。\\

一九四四年实行节约煤炭时,溥仪曾令\xpinyin*{缉熙}楼停止升火,为的做给吉冈看,但在自己卧室内,背着吉冈用电火取暖。\\
\end{quote}

溥仪逃亡大栗子沟,把倭神与裕仁母亲像放在车上客厅内,他从那里经过必行九十度和,并命我们也如此。小瑞的检举里有这样一段:\\

\begin{quote}
	他用的孤儿,有的才十一二岁,有的父母被日寇杀害后收容到博济总会,前后要来使用的有二十名。工作十七八小时,吃的高粱米咸菜,尝尽非刑,打手板是经常的、最轻的。站木笼、跪铁链、罚劳役……平时得互相监视。孤儿长到十八九岁仍和十一二岁一般高矮。溥仪手下人曾将一名孤儿打死,而他却吃斋念佛,甚至不打苍蝇蚊子。\\
\end{quote}

在语气上流露出仇恨的,是大李写的:\\

\begin{quote}
	溥仪这个人既残暴又怕死,特别好疑心,而且很好用权术,十分伪善。他对佣人不当人待,非打即骂,打骂也不是因为犯了什么错,完全是以他个人情绪如何而定。如有点不舒服啦,累一点啦,用的人就倒楣了。拳打脚踢是轻的。可是他见了外人的时候,那种伪善样,就像再好也没有的。打人别具,在天津时有木板子、马鞭子,到伪满又加上许多新花样。……\\

他把大家都教成他的帮凶,如要是打某人,别人没有动手打,或动作稍慢一些,他都认为是结党袒护,那未动手打的人,要被打得厉害多少倍。\\

侄子与随侍没有没打过人的。一个十二三岁的周博仁(孤儿)有一次被打得两腿烂了一尺长的口子,叫黄子正大夫治了两三个月才好。这孩子治疗时,溥仪叫我送牛奶等物,还让我对孩子说:皇上对你多好呵!你在孤儿院能吃到这么好的东西吗?\\
\end{quote}

我把最后这批检举材料看完,过去那一套为自己做辩护的道理,从根本上发生了动摇。\\

在从前,我把自己的行为都看做是有理由的。我屈服于日本人的压力,顺从它的意志,是不得已而为之的;我对家里人的作福作威、予取予夺、动辄打骂以至用刑,也当做我的权力。总之,对强者的屈服,对弱者的发威,这都被我看做是自然的、合理的,我相信人人处于我的境地都会那样做。现在,我明白了除了我这样的人,别人并非如此;我的道理是拿不出去的。\\

说到弱者,没有比被剥夺权利的囚犯更“弱”的了,然而掌握着政权的共产党人对手下的这些囚犯,并没有打,没有骂,没有不当人看。说到强者,具有第一流装备的美国军队可算是“强”的了,然而装备远逊于它的共产党军队硬是不怕它,竟敢于跟它打了三年之久,一直打得它在停战协定上签了字。\\

就在刚才,我还看到了新的例子。在人民群众的控诉检举材料里,我知道了原来有许多普普通通的人,在强暴压力面前并不曾按着我的信条办事。\\

巴颜县有个叫李殿贵的农民,受尽了鬼子和汉奸的欺压,他把希望放在抗日联军身上。一九四一年的春节,他给抗联队伍送去了一斗小米、四十七根麻花、一百二十个鸡蛋和两包烟卷。后来被伪警察知道了,把他抓去,成天上“大挂”、吊打、过电,并且把打得血淋淋的死难者放在他身边恐吓他,叫他供出抗联的线索。这个顽强不屈的农民没有吐露出任何关于抗联的口供,在监狱里受尽折磨,一直坚持到光复得救。\\

姜树发,是天增屯的抗日救国会的副会长,给抗联送过饭,带过路,他被特务们抓去了,一连过了七堂,上“大挂”、打钉板。过电、灌凉水全经过了,没有供出一点线索,特务拿他没法,最后判了他两年徒刑。\\

萧振芳也是一个普通农民,帮助他叔叔萧坤一同给抗联送饭、带路,做秘密的抗日工作。一九四三年四月二十一日的半夜里,六个伪警察突然闯进他的家,没寻找到他叔叔萧坤,把他绑送到警察署追问。他说:“我不知道!”警察们把他打死过去,然后浇凉水,醒过来又打,这样死而复活,活了又打死,折腾到第四次,凉水也浇不活了,就用“卫生车”拉到烂尸岗子,扔在那里。这个顽强的人在烂尸岗又活了,被一个拉卫生车的工人救了去。他的叔父萧坤到后来也被抓了去,至死不屈。他住的那个监狱,就是我在哈尔滨住过的那个地方。\\

一九四三年,金山屯的李英华还是个孩子,他曾给过路的抗联军队送过鸡蛋,被特务告发,捉到警察署里。特务们先给他点烟、倒茶,请他吃饺子,说:“你是个孩子,不懂事,说了就放你。”李英华吸了烟,喝了茶,吃了饺子,然后说:“我是庄稼人,啥也不知道!”特务们便把他头朝下挂起来打,又过电、火烧,脱光了身子撞钉板,可是从这个孩子身上什么也没得到。\\

总之,我知道世界上的人并非骨头都是软的。我过去的所作所为,除了说明是欺软怕硬和贪生怕死之外,没有任何其他别的解释。\\

我从前还有一条最根本的理由,为欺软怕硬、贪生怕死做解释,就是我的命最贵重,我比任何人都更有存在的价值。几年来,经过洗衣、糊纸盒,我已懂得了自己的价值,今天我更从东北老百姓和家族的检举中看出了自己的价值。\\

我在镜子的围屏中看出我是有罪的人,是没有光彩的人,是个没有理由可以为自己做任何辩解的人。\\

我在最后一份材料上签完字,走在甬道上,心中充满了懊悔与悲伤:\\

“天作孽,犹可违,自作孽,不可活!”