\fancyhead[LO]{{\scriptsize 1917-1924: 北京的“小朝廷” · 由“使馆区”到“租界地”}} %奇數頁眉的左邊
\fancyhead[RO]{\thepage} %奇數頁眉的右邊
\fancyhead[LE]{\thepage} %偶數頁眉的左邊
\fancyhead[RE]{{\scriptsize 1917-1924: 北京的“小朝廷” · 由“使馆区”到“租界地”}} %偶數頁眉的右邊
\chapter*{由“使馆区”到“租界地”}
\addcontentsline{toc}{chapter}{\hspace{11mm}由“使馆区”到“租界地”}
%\thispagestyle{empty}
在那个时代,“使馆区”和“租界地”正是“好客”的地方。我进了日本公使馆才知道,我并不是惟一的客人,当时还住着一个名叫\xpinyin*{王毓芝}的人物,他是贿选大总统\xpinyin*{曹锟}的心腹谋士。\xpinyin*{曹锟}没有来得及逃往使馆区,被国民军软禁了起来。\xpinyin*{王毓芝}的腿快,做了这里的客人。我还记得,七年前我第二次做皇帝的时候,被\xpinyin*{张勋}赶走的\xpinyin*{黎元洪}也在这里住过,我第二次退位以后,被\xpinyin*{段祺瑞}赶走的\xpinyin*{张勋}做过荷兰使馆的客人。每逢使馆里到了必须接待来客的时候,使馆区里的饭店和医院总免不了跟着热闹一番,因为每次总有一批神经脆弱而又身价够不上进使馆的人们往这里跑,把这里塞得满满的,甚至于连楼梯底下都有人愿意付租金。\xpinyin*{辛亥}、\xpinyin*{丁巳}和我这次被赶出紫禁城,有不少的满族贵族都争先恐后地到这里做客。有一次饭店老板贴出了一张很不礼貌的告示:“查本店寄居者过多,楼梯下亦已住满,卫生状况殊为不佳,且有随地吐痰、极不文明者,……兹规定,如再有人吐痰于地,当罚款十元,决不宽贷!”尽管如此,还是有人趋之若鹜,流连忘返。\\

我在这里遇到的热情是空前的,也许还是绝后的。有一件小事我在前面没有说到,是我从北府出来的时候,在我汽车上还有北府的两名警察,他们按照当时“要人”们乘车的习惯,站在车外踏脚板上,一边一个,一直陪我到了德国医院。后来知道我不回去了,他们不能回去交差,便要求留在日本使馆。他们得到了准许,作为我的随侍被收留了。后来我派人再去北府接\xpinyin*{婉容}和\xpinyin*{文绣}的时候,那边的警察再不肯放走她们。使馆里派了一名书记官特意去交涉,也没有成功,最后还是\ruby{芳泽}{\textcolor{PinYinColor}{よしざわ}}公使亲自去找了段执政,\xpinyin*{婉容}和\xpinyin*{文绣}才带着她们的太监、宫女来到了我的身边。\\

使馆主人看我周围有那么一大群人,三间屋子显然住不开,特意腾出了一所楼房,专供我使用。于是我那一班人马——南书房行走和内务府大臣以及几十名随侍、太监、宫女、妇差、厨役等等又各得其所。在日本公使馆里,“大清皇帝”的奏事处和值班房又全套恢复了。\\

更重要的是,\ruby{芳泽}{\textcolor{PinYinColor}{よしざわ}}公使给我取得了执政府的谅解。执政府除了向\ruby{芳泽}{\textcolor{PinYinColor}{よしざわ}}公使做了表示之外,并且派了陆军中将曲同丰,亲自到日本兵营的\ruby{竹本}{\textcolor{PinYinColor}{たけもと}}大佐那里,再次表明:“执政府极愿尊重逊帝的自由意志,并于可能范围内,保护其生命财产及其关系者之安全。”\\

以我父亲为首的王公们曾来劝我口去,说北府现在已经安全,有\xpinyin*{段祺瑞}和\xpinyin*{张作霖}在,国民军决不敢任意行事,还说段和张都向他们做了保证。但我相信\xpinyin*{罗振玉}他们的话,段和张的保证都是因为我进了使馆才说的,我如果还在北府,而国民军还在北京,什么保证都靠不住。我拒绝了他们。事实上,王公们也正在向使馆区里找住处,后来有的进了德国兵营,有的进了六国饭店。我父亲一面劝我,又一面在西什库教堂租库房,存放他的珍贵财物,后来北府里的弟妹们也都跑到西什库教堂住去了。\\

看见日本使馆对我的殷勤照料,连许多不知名的遗老也活跃起来了。他们从各地给段执政打电报,要求恢复优待;他们给我寄钱(这叫做“进奉”),供我使用;有的人从外地跑到北京,给我请安,密陈大计。蒙古工公好像吃了兴奋剂似的,发出通电并上呈文给执政府,质问对他们的优待怎么办,执政府连忙答复说照旧不变。王公大臣们的腰板也硬起来了,拒绝出席“清室善后委员会”的会议。这个刚成立不久的委员会,由代表民国方面的\xpinyin*{李石曾}(委员长)、\xpinyin*{易培基}(代表\xpinyin*{汪精卫})、\xpinyin*{俞同奎}、\xpinyin*{沈兼士}、\xpinyin*{范源濂}、\xpinyin*{鹿钟麟}、\xpinyin*{张璧}和代表清室方面的\xpinyin*{绍英}、\ruby{载}{\textcolor{PinYinColor}{zǎi}}\ruby{润}{\textcolor{PinYinColor}{rùn}}、\ruby{耆}{\textcolor{PinYinColor}{Ci}}\ruby{龄}{\textcolor{PinYinColor}{Ling}}、\ruby{宝熙}{\textcolor{PinYinColor}{\Man ᠪᠣᠣ ᡥᡞ}}等组成,并请了\xpinyin*{罗振玉}列席。委员会要清点财物,划分公产私产以决定处理,\xpinyin*{绍英}等四人不但不去参加,并再次向当局声明不承认这个组织。\ruby{宝熙}{\textcolor{PinYinColor}{\Man ᠪᠣᠣ ᡥᡞ}}后来通过他的门生从宫里弄出十几箱东西运到了日本使馆,\xpinyin*{罗振玉}立刻反对说:“这岂不是从强盗手里讨施舍?如果要就全要,否则就全不要!”原来他另有打算,想把宫里的东西弄到他可以支配的地方去。那时我不知道这个底细,只觉得他说的有理,有骨气。至于后来又弄了没弄,弄出了什么来,我就全不知道了。\\

这些表示骨气的,请安的,送进奉的,密陈各种“中兴大计”的,敢于气势汹汹质问执政府的遗老遗少们,出进日本使馆的一天比一天多。到了旧历的元旦,我的小客厅里陡然间满眼都是辫子。我坐在坐北朝南、以西式椅子代替的宝座上,接受了朝贺。\\

许多遗老对使馆主人怀着感激之情。他们从使馆的招待上看出了希望,至少得到了某种心理上的满足。\xpinyin*{王国维}在奏折里说:“日使……非徒以皇上往日之余尊,亦且视为中国将来之共主,凡在臣僚,谁不庆幸?”\\

旧历元旦那天,小客厅里是一片庆幸的脸色。那天有段插曲值得一提。正当第三班臣僚三跪九叩行礼如仪之际,突然在行列里发出一声干嚎,把人们都吓了一跳,接着,有一个用袖掩面的人推开左右,边嚎边走,夺门而出。当时我还以为是谁碰瞎了眼睛,众人也愕然不知所措。有人认出这是前内务府大臣\xpinyin*{金梁},他干嚎个什么,没有一个人知道。到第二天,《顺天时报》上刊出了他写的诗来,人们这才恍然大悟,原来昨天这一幕怪剧,是为了写这首诗而做的苦心准备。诗曰:\\

\begin{quote}
	元旦朝故主,不觉哭失声;虑众或骇怪,急归掩面行。闭门恣痛哭,血泪自纵横。自晨至日午,伏地不能兴;家人惊欲死,环泣如送生。急梦至天上,双忠(文忠、忠武\footnote{文忠、忠武是\xpinyin*{梁鼎芬}和\xpinyin*{张勋}的谥法。})下相迎;携手且东指,仿佛见蓬瀛;波涛何汹涌,风日倏已平。悠悠如梦觉,夕阳昏复明,徐生惟一息,叩枕徒哀鸣。\\
\end{quote}

过了旧历元旦,眼看我的生日又要到了,而且是二十(虚岁)整寿。我本来不打算在别人家做寿,不料主人偏要凑趣,硬要把使馆里的礼堂让出来,作为接受朝贺之用。礼堂布置起来了,地板上铺上了豪华的地毯,作为宝座的太师椅上铺了黄缎子坐垫,椅后一个玻璃屏风贴上了黄纸,仆役们一律是清朝的红缨大帽。到了生日这天,从天津、上海、广东、福建等地来的遗老竟达一百以上,东交民巷各使馆的人员也有人参加,加上王公大臣、当地遗老,共有五六百人之多。因为人多,只得仍照例写出秩序单,分班朝贺。下面就是当时的礼单:\\

\begin{quote}
	一班:近支王公世爵,\ruby{载}{\textcolor{PinYinColor}{Dzai}}\ruby{涛}{\textcolor{PinYinColor}{Tao}}领衔;\\

二班:蒙古王公、活佛喇嘛,\ruby{那彦图}{\textcolor{PinYinColor}{\Meng ᠨᠠᠶᠠᠨᠲᠥ}}领衔;\\

三班:内廷司员、师傅及南书房翰林,\xpinyin*{陈宝琛}领衔;\\

四班:前清官吏在民国有职务者,\xpinyin*{志琦}领衔;\\

五班:前清遗臣,\xpinyin*{郭曾炘}领衔;\\

六班:外宾,\ruby{庄士敦}{\textcolor{PinYinColor}{Johnston}}领衔。\\
\end{quote}

那天我穿的是蓝花丝葛长袍,黑缎马褂,王公大臣和各地遗老们也是这种装束。除了这点以外,仪节上就和在宫里的区别不大了。明黄色、辫子、三跪九叩交织成的气氛,使我不禁伤感万分,愁肠百结。仪式完毕之后,在某种冲动之下,我在院子里对这五六百人发表了一个即席演说。这个演说在当时的上海报纸上刊载过,并不全对,但这一段是大致不差的:\\

\begin{quote}
	余今年二十岁,年纪甚轻,不足言寿,况现在被难之时,寄人篱下,更有何心做寿,但你们远道而来,余深愿乘此机会,与尔等一见,更愿乘此机会,与尔等一谈。照世界大势,皇帝之不能存在,余亦深知,决不愿冒此危险。平日深居大内,无异囚犯,诸多不能自由,尤非余所乐为。余早有出洋求学之心,所以平日专心研究英文,原为出洋之预备,只以其中牵掣太多,是以急切不能实行。至优待条件存在与否,在余视之,无关轻重,不过此事在余自动取消则可,在他人强迫则不可。优待条件系双方所缔结,无异国际之条约,断不能一方面下令可以更改。此次\xpinyin*{冯玉祥}派兵入宫,过于强迫,未免不近人情,此事如好好商量,并不难办到。余之不愿拥此虚名,出于至诚,蓄之久矣,若胁之兵威,余心中实感不快。即为民国计,此等野蛮举动,亦大失国家之体面,失国家之信用,况逐余出官,另有作用,余虽不必明言,大约尔等亦必知之。余此时系一极无势力之人,\xpinyin*{冯玉祥}以如此手段施之于余,胜之不武,况出官时所受威胁情形,无异凌辱,一言难尽。逐余出宫,犹可说也,何以历代祖宗所遗之衣物器具文字,一概扣留,甚至日用所需饭碗茶盅及厨房器具,亦不许拿出,此亦为保存古物平?此亦可值金钱乎?此等举动,恐施之盗贼罪国,未必如此苛刻。\\

在彼一方面,言\xpinyin*{丁巳}复辟为破坏优待条件,须知\xpinyin*{丁巳}年余方十二岁,有无自动复辟之能力,姑不具论,但自优待条件成立以来,所谓岁费,曾使时付过一次否?王公世爵俸银,曾照条件支给否?八旗生计,曾照条件办理否?破坏之责,首先民国,今舍此不言,专借口于\xpinyin*{丁巳}之复辟,未免太不公允!余今日并非发牢骚,不过心中抑郁,不能不借此机会宣泄,好在将有国民会议发现,如人心尚有一线光明,想必有公平之处置,余惟有静以\xpinyin*{俟}之。余尚有一言郑重声明,有人建议劝余运动外交,出为干涉,余至死不从,余决不能假借外人势力干涉中国内政。\\
\end{quote}

在我做生日的前后,许多报纸上出现了抨击我这伙人的\xpinyin*{舆}论,反映了社会上多数人的义愤。这种义愤无疑是被我的投靠日人,被小朝廷在当局的姑息和外人的包庇下的嚣张举动刺激出来的。这时“清室善后委员会”在清查宫内财物时发现了一些材料,如\xpinyin*{袁世凯}做皇帝时写在优待条件上的亲笔跋语,内务府抵押、变卖、外运古物的文据等等,公布了出来,于是\xpinyin*{舆}论大哗。当然最引人愤慨的,还是小朝廷和日本人的关系以及遗老们发起的要求恢复优待条件的运动(在我过生日的时候,报上刊登的已有十五个省三百余人十三起联名呈请)。为了对付小朝廷,北京出现了一个叫“反对优待清室大同盟”的团体,展开了针锋相对的活动。这些社会义愤在报纸上表现出的有“别馆珍闻”的讽刺小品,也有严肃激昂的正面指责;有对我的善意忠告,也有对日本使馆和民国当局的警告式的文字。今天看来,哪怕我从这些文章中接受一条意见,也不会把我的前半生弄成那样。记得有几篇是揭发日本人的阴谋的,现在我把它找出来了。这是一份登在《京报》上的“新闻编译社”的消息,其中有一段说到日本人对我的打算,它和后来发生的事情竟是那么吻合,简直令我十分惊讶:\\

其极大黑幕,为专养之以\xpinyin*{俟}某省之有何变故,某国即以强力护送之到彼处,恢复其祖宗往昔之地位名号,与民国脱离,受某国之保护,第二步再实施与某被合并国家同样之办法。\\

这个文章后面又说:“此次\ruby{溥仪}{\textcolor{PinYinColor}{\Man ᡦᡠ ᡳ}}之恐慌与出亡,皆有人故意恫吓,人其圈套,即早定有甚远之计划”,“其目前之优待,供应一切,情愿破钞,侍从人员,某国个个皆买其欢心,不知皆已受其牢宠,为将来之机械也”。这些实在话,在当时我的眼里,都一律成了诬蔑、陷害,是为了把我骗回去加以迫害的阴谋。当时有些文章,显然其作者既不是共产党人也不是国民党人,例如下面《京报》的一篇短评,或者还是一位讲究封建忠义之士的手笔,对我的利益表现了关心,说的又是实在事:\\

\begin{quote}
	遗老与\ruby{爱新觉罗}{\textcolor{PinYinColor}{\Man ᠠᡳᠰᡳᠨ ᡤᡳᠣᡵᠣ}}氏有何仇恨胡为必使倾家败产而后快?\\
\end{quote}

点查清官之结果,而知大宗古物多数业已抵卖,即历代之金宝金册皆在抵押中,虽以细人非至极穷,尚或不至卖其饲庙坟墓之碑额,奈何以煌煌历代皇后金册,亦落于大腹长袖者手?……吾敬为一班忠臣设计,应各激发忠义,为故主之遗嗣图安宁,勿徒硜硜自诩,以供市井觅利者流大得其便宜货,使来路不明之陈设品遍置堂室也。看了这样的文章,我已经不是像在宫里时那样,感到内务府人的不可信任,我对于这份《京报》和短评作者,只看成是我的敌人。至于那些指责文章,更不用说,在我心里引起的反应惟有仇恨。\\

我在日本使馆住着,有几次由于好奇,在深夜里带上一两名随侍,骑自行车外游(后来使馆锁了大门,不让出去了)。有一次我骑到紫禁城外的筒子河边上,望着角楼和城堞的轮廓,想起了我刚离开不久的养心殿和乾清宫,想起了我的宝座和明黄色的一切,复仇和复辟的欲望一齐涌到我的心头,不由得心如火烧。我的眼睛噙着泪水,心里发下誓愿,将来必以一个胜利的君王的姿态,就像第一代祖先那样,重新回到这里来。“再见!”我低低地说了这两个双关含意的字,然后跳上车子疾驶而去……\\

在使馆的三个月里,我日日接触的,是日本主人的殷勤照拂,遗老们的忠诚信誓和来自社会的抗议。我的野心和仇恨,在这三种不同的影响下,日夜滋长着。我想到这样呆下去是不行的,我应该为我的未来进行准备了,原先的打算又回到我的心中——我必须出洋到日本去。\\

使馆对我的想法表示了支持。公使正面不做什么表示,而\ruby{池部}{\textcolor{PinYinColor}{いけべ}}书记官公开表现了极大的热情。\xpinyin*{罗振玉}在他的自传《集寥编》中提过这个\ruby{池部}{\textcolor{PinYinColor}{いけべ}},他说:“予自随待人使馆后,见\ruby{池部}{\textcolor{PinYinColor}{いけべ}}君为人有风力,能断言,乃推诚结纳,\ruby{池部}{\textcolor{PinYinColor}{いけべ}}君亦推诚相接,因密与商上行止,\ruby{池部}{\textcolor{PinYinColor}{いけべ}}君谓:异日中国之乱,非上不能定,宜早他去,以就宏图,于是两人契益深。……”\\

关于\xpinyin*{郑孝胥}和\xpinyin*{罗振玉}这两位“宠臣”的事,这里要补述一下。这时以我为目标的争夺战,在日使馆中又进入了新的阶段,这次是以\xpinyin*{郑孝胥}的失败和\xpinyin*{罗振玉}的胜利而收场的。\\

\xpinyin*{郑孝胥}曾经拍过胸脯,说以他和段的关系,一定可以把优待条件恢复过来,段的亲信幕僚\xpinyin*{曾毓雋}、\xpinyin*{王揖唐}都是他的同乡,\xpinyin*{王揖唐}等人跟他半师半友,这些人从旁出力,更不在话下。后来\xpinyin*{段祺瑞}许下的空口愿不能兑现,使\xpinyin*{郑孝胥}大为狼狈。对\xpinyin*{郑孝胥}的微词就在我耳边出现了。从天津来的旧臣\xpinyin*{升允}首先表示了对郑的不满,他向我说了不少\xpinyin*{郑孝胥}“清谈误国”、“妄谈诳上”、“心怀叵测”、“一手遮天”之类的话。当时我并不知道,在前一个回合中失败的\xpinyin*{罗振玉},和这些反郑的议论,有什么关系。经过\xpinyin*{升允}这位先朝老臣的宣传,我对\xpinyin*{郑孝胥}是冷淡下来了,而对\xpinyin*{罗振玉}增加了好感。\\

\xpinyin*{罗振玉}在我面前并没有十分激烈地攻击\xpinyin*{郑孝胥},他多数时间是讲他自己,而这样做法比攻击别人的效果还大。我从他的自我表白中得到的印象,不仅他是这场风险中救驾的大功臣,而且相形之下,\xpinyin*{郑孝胥}成了个冒功取巧的小人。据\xpinyin*{罗振玉}自己说,\xpinyin*{段祺瑞}从天津发出反对\xpinyin*{冯玉祥}赶我出宫的电报,乃是他的活动结果之一。他回到北京,找到了他的好朋友\ruby{竹本}{\textcolor{PinYinColor}{たけもと}}大佐,因此才有了迎我人日本兵营的准备。后来北府门前国民军的撤走,据他说也是他找执政府交涉的结果。甚至我到东交民巷前决定的“先随便出入,示人以无他”的计策,也是他事先授给\xpinyin*{陈宝琛}的。\\

\xpinyin*{罗振玉}后来在《集家编》中,关于我进日本使馆的这一段,对\xpinyin*{郑孝胥}一字未提,只是在叙述我进日本使馆后的情形时,说了一句:“自谓能令\xpinyin*{段祺瑞}恢复优待者,以不能实其言,亦不告而南归矣!”事实上,那时我一心想出洋,\xpinyin*{郑孝胥}并没有支持我,在\ruby{庄士敦}{\textcolor{PinYinColor}{Johnston}}已经不宣传去伦敦做客的情形下,主张“东幸”的\xpinyin*{罗振玉}自然更受到我的重视,我对\xpinyin*{郑孝胥}因此不再感兴趣。于是\xpinyin*{郑孝胥}终于有一天郁郁地向我请假,说要回上海料理私事去,我当时还不明白他的意思,所以没挽留他,他一气就跑了。\\

生日过后不多天,\xpinyin*{罗振玉}来告诉我说:他和\ruby{池部}{\textcolor{PinYinColor}{いけべ}}已商量妥当,出洋的事应该到天津去做准备,在这里住着是很不方便的;到天津,最好还是在日本租界里找一所房子,早先买好的那房子在英租界,地点很不合适。我听他说得有理,也很想看看天津这个大都市,他的主意正中下怀,便立即同意了。我派“南书房行走”\xpinyin*{朱汝珍}去天津日租界找房子,结果看中了张园。不多天,\xpinyin*{罗振玉}又说,张园那里已经准备好,现在国民军在换防,铁路线上只有少数的一些奉军,正是个好机会,可以立即动身。我向\ruby{芳泽}{\textcolor{PinYinColor}{よしざわ}}公使谈了,他表示同意我去天津。为了我这次转移,他派人通知了\xpinyin*{段祺瑞}。段表示同意,还要派军队护送。\ruby{芳泽}{\textcolor{PinYinColor}{よしざわ}}没有接受他的好意,他决定由天津日本总领事馆的警察署长和便衣警察来京,由他们先护送我去,然后\xpinyin*{婉容}她们再去。事情就这样谈妥了。\\

民国十四年二月二十三日下午七时,我向\ruby{芳泽}{\textcolor{PinYinColor}{よしざわ}}公使夫妇辞行。我们照了相,我向他们表示了谢意,他们祝我一路平安,然后由\ruby{池部}{\textcolor{PinYinColor}{いけべ}}和便衣日警们陪着,出了日本公使馆的后门,步行到了北京前门车站。我在火车上找到了\xpinyin*{罗振玉}父子。火车在行进的一路上,每逢到站停车,就上来几个穿黑便衣的日本警察和特务,车到了天津,车厢里大半都被这样的人占满了。日本驻天津总领事\ruby{吉田}{\textcolor{PinYinColor}{よしだ}}\ruby{茂}{\textcolor{PinYinColor}{しげる}}和驻屯军的军官士兵们,大约有几十名,把我接下了车。\\

第三天,《顺天时报》上便出现了日本公使馆的声明:\\

\begin{quote}
	本公使馆滞在中之前清\ruby{宣统}{\textcolor{PinYinColor}{\Man ᡤᡝᡥᡠᠩᡤᡝ ᠶᠣᠰᠣ}}皇帝,于二十三日夜,突然向天津出发,本馆即于二十四日午后,将此旨通知段执政及外交总长,备作参考。原\ruby{宣统}{\textcolor{PinYinColor}{\Man ᡤᡝᡥᡠᠩᡤᡝ ᠶᠣᠰᠣ}}皇帝怀有离京之意,早为执政之政府所熟知,而无何等干涉之意,又为本馆所了解,但豫想迄实行之日,当尚有多少时日,不意今竟急遽离开北京,想因昨今一二新闻,频载不稳之记事,致促其行云云。
\end{quote}
