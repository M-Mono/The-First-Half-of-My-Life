\fancyhead[LO]{{\scriptsize 1950-1954: 由抗拒到认罪 · 坦白从宽}} %奇數頁眉的左邊
\fancyhead[RO]{} %奇數頁眉的右邊
\fancyhead[LE]{} %偶數頁眉的左邊
\fancyhead[RE]{{\scriptsize 1950-1954: 由抗拒到认罪 · 坦白从宽}} %偶數頁眉的右邊
\chapter*{坦白从宽}
\addcontentsline{toc}{chapter}{\hspace{1cm}坦白从宽}
\thispagestyle{empty}
“我\xpinyin*{溥仪}没有良心。政府给我如此人道待遇,我还隐瞒了这些东西,犯了监规,不,这是犯了国法,这东西本来不是我的,是人民的。我到今天才懂得,才想起了坦白交代。”\\

在所长的接待室里,我站在所长面前,低着头。在靠窗的一张桌子上,那四百六十八件首饰,发射着令人惋惜的光彩。假如我的“主动坦白”可以挽救我,假如宽大政策对我有效验的话,那么光彩就让它光彩去吧。\\

所长注视了我一阵,点点头说:“坐下来吧!”从这一声里,我听出了希望。\\

“你为了这件事,经过了很多思想斗争吧?”所长问。\\

我避开了那个纸条,说我一直为这件事心中不安。在我说的那些话里,只有最后一句是真的:“我不敢坦白,我怕坦白了也得不到宽大处理。”\\

“那为什么呢?”所长的嘴角上漾着笑意,“是不是因为你是个皇帝?”\\

我怔了一下,承认了:“是的,所长。”\\

“也难怪你会这样想,”所长笑起来了,“你有你的独特历史,自然有许多独特想法。我可以再告诉你一次:共产党和人民政府的政策是说到做到的,不管从前是什么身分,坦白的都可以从宽,改造好的还可以减刑,立功的还可以受奖。事在人为。你这些东西当初没交出来,犯了监规,并且藏在箱底里一年多,如今你既然自己来坦白,承认了错误,这说明你有了悔悟,我决定不给你处分。”\\

说罢,他命令门外的看守员去找保管员来。保管员到了,他命令道:\\

“你把那堆东西点收下来,给\xpinyin*{溥仪}开一个存条。”\\

我感到太出乎意料了。我连忙站起来:\\

“不,我不要存条。政府不肯没收,我也要献出来。”\\

“还是给你存起来吧。你在这里点交。”所长站起来要走,“我早已告诉过你,对我们说来,更有价值的是经过改造的人。”\\

我带着四百六十八件首饰的存条,回到了监房。同伴们正开讨论会,讨论着正在学习的《中国怎样降为殖民地半殖民地》这本书里的问题。他们看见我回来了,停下讨论,给了我前所未有的待遇,庆贺我有了进步。\\

“老溥,佩服你!”他们现在已经不叫我溥先生,而是一视同仁地以“老”字相呼了。刚一听到这称呼,我比听叫“先生”更觉着不是滋味,不过今天被他们叫得很舒服。“老溥,从你这件举动上,给了我启发!”“老溥,没看出你真有勇气。”“老溥,我有你这例子,更相信宽大政策了。我向你表示感谢。”等等。\\

我这里该补充说明一件事。自从我的衣物自洗自缝以来,我的外形比以前更加狼狈不堪了,而同伴们对我的尊敬也随着“先生”的称呼去了一大半,有人甚至于背后叫起我“八杂市”(哈尔滨从前一个专卖破烂的地方)来。在学习上表现出的无知,也时常引起他们的毫无顾忌的笑声。总之,我明白了自己在他们心目中的身分。现在他们再三对我表扬,我顿时有了扬眉吐气之感。\\

这天休息时,我在院子里听见前伪满驻日大使老元对别人谈论这件事。老元这人心眼极多,可以说眼珠一转就够别人想一天的。这个多心眼的人说出一段话,大大触动了我的心事:\\

“老溥是个聪明人,一点不笨。他争取了主动,坦白那些首饰,做的极对。其实,这种事瞒也瞒不住,政府很容易知道的。政府掌握着我们的材料,比我们想象的还要多。你们想想报上的那些三反、五反的案子就知道。千百万人都给政府提供材料,连你忘了的都变成了材料,飞到政府手里去了。”\\

照他这话说来,我在自传里扯的谎,看来也瞒不住了。\\

如果我说了出来,会不会像交出珠宝一样的平安无事呢?一个是政治问题,一个是经济问题,能一样对待吗?所长可没说。可是似乎用不着说,犯了法就是犯了法,经济上犯罪也是犯罪,三反、五反案件重的重办、轻的轻办,坦白的从宽,应该全是一样的。\\

话是这样说,事情不到临头,我还是下不了决心。跟上回不同的是,报上一出现“宽大”二字,我比以前更加想看个究竟了。\\

三反、五反运动接近了尾声,结案的消息多了起来,而且尽是“宽大处理”的。老王是干过“法官”的,我曾跟他研究过报上的那些案件。每次研究,我总在心里跟我自己的事情联系起来,反复考虑,能否援用这项政策。后来所方叫我们写日寇在东北的罪行材料时,我想的就更多了。\\

政府为了准备对日本战犯的处理,开始进行有关调查,号召伪满战犯提供日寇在东北的罪行材料。那天所方干部宣布这件事的时候,有人提出一个问题:“除了日寇的,别的可不可以写?”干部回答:“当然可以写,不过主要的是日寇罪行。”我听了,不由得犯了嘀咕:他要写什么别的?别的当然是中国人的,中国人最大的罪犯当然是我!我家里的人会不会也要写点“别的”?\\

伪满战犯对于写日寇在东北的罪行,都很积极。我们这个组,头一天就写出了十多份。组长老王收齐了写好的材料,满意地说:“我们的成绩不错!明天一定还可以写出这么多。”有人接口说:“如果让东北老百姓写,那不知可以写出多少来。”老王说:“那还用说,政府一定会向东北人民调查的!你看呢,老溥?”我说:“我看是一定的,可不知道这次除了日寇,还调查别人不?”“不调查别人,可是准有人要写到我们。老百姓恨我们这些人不下于恨日本人呢!”\\

吃晚饭的时候,是大李来送饭。我觉着他好像特别有气似的,他不等我把饭菜接过来,放在地上就走了。他走开以后,我立刻想起了我离开静园的时候,是他帮助我钻进车厢里去的。\\

第二天,我们又写了一天材料。我知道的不多,写的也少了。老王收材料时,仍很满意,因为别人写的还是不少。他说:“你们瞧吧,以此推想,东北人民写的会有多少!政府掌握了多少材料!干过司法工作的就知道,有了证据就不怕你不说。从前,旧社会司法机关认为顶难的就是证据,可是在人民政府这里,老百姓都来提供材料,情形就不同了。”我听了这话,心里又是一跳。\\

“政府掌握了材料!”这话我不是第一次听说了。今天早晨,我们议论报上一条关于捕获暗藏的反革命分子的消息时,我不由得又想起了这句话。报上这条消息中说,一九三五年杀害了红军将领\xpinyin*{方志敏}的刽子手,已经在湖南石门的深山中捕获了。这个刽子手在湖南解放后,先藏在常德县,后来躲到石门的深山里,继续干反革命活动,但是终于给公安机关侦察出来。怎么查出来的,报上没说。我心想,这大概又是掌握了材料,大概共产党从一九三五年就把这个刽子手的材料记下来了。我跟老王学得了一句司法术语,这叫“备案存查”。\\

第三天,当我写下了最后的一条材料,忽然听到楼梯口上有人声。我扭过头来,看见有个陌生的中年人出现在岗台的附近,后面随着所长。根据经验,我判断出这是上级机关来人视察。这位视察人员挨次察看了每间监房,听着看守长报告每个监房犯人的名字,面上毫无表情。他没穿军衣,我却觉得他像一位军人,这与其说是由于他的精确适度的每个动作和他的端正的体型,无宁说是由于他的严肃的面容。他大约不到五十岁。\\

“你在于什么?”他在我们的监房外停下了,这样问着,眼睛看着我。我没料到他的声调很温和,而且他脸上浮着一丝笑容。\\

我站了起来,报告说我正写日寇的罪行。他对我的回答感到兴趣:“你知道些什么日寇罪行?”\\

我把刚写好的,从前听\xpinyin*{佟济煦}说的那段屠杀建筑秘密工程工人的故事说了。\\

也许是我的神经过敏,也许事实就是如此,我觉得他脸上的那一丝笑容突然消失了,他的目光变得非常严峻。我万没料到这个故事引起了他这么强烈的反应。\\

“我当时听了很刺激,我原没想到日本人这样残忍。”我不安地说。\\

“你为什么不向日本人抗议呢?”他逼视着我的眼睛。\\

我觉出他在生气,赶紧低下了头,轻声说:\\

“我……不敢”\\

“你不敢,害怕,是吗?”他不要我回答,自顾说下去,“唉,害怕,害怕就能把一个人变成这样!”末后这句,又恢复了平静的声调。\\

我低声说:“这都是由于我的罪过造成的,我只有向人民认罪,我万死不足以蔽其辜!”\\

“也不要这样,把一切揽到自己头上。你只能负你自己那部分责任。应当实事求是。是你的,你推不掉,不是你的,也不算在你的账上。”\\

我仍继续说,我的罪是深重的,我感激政府对我的待遇,我已认识自己的罪恶,决心改造好。我不知道他是否在听我的话,只见他察看我们的监房各处,并且叫一个犯人拿过漱口杯看了一看。等我说完,他摇摇头,说道:\\

“应当实事求是。只要真正认罪,有了悔改表现,一定可以得到宽大。共产党说话算数,同时重视事实。人民政府对人民负责。你应当用事实和行动而不是用嘴巴来说明自己的进步。努力吧。”\\

他对我写的那堆东西看了一眼,然后向隔壁的监房走去了。\\

我的心沉重得厉害。我拿起写好的那堆材料重看了一遍,似乎今天我才感到这类事情的严重性。\\

从这以后,那双严峻的目光似乎总也离不开我,那几句话也总冲击着我的心:“是你的,你推不掉!”“应当实事求是!”“用事实和行动而不是用嘴巴来说明自己的进步!”我觉得自己正处在一个无法抗拒的冲力面前。是的,这是一种不追究到底誓不罢休的冲力。就是由于这股冲力,一九三五年杀害\xpinyin*{方志敏}的刽子手藏在深山中也没能逃脱掉。我觉得在这股冲力面前,日寇在东北的罪行必将全部结算清楚,伪满大小汉奸的!日账都无法逃掉。\\

这天是星期日,我在院子里晾晒洗好的衣服,忽然看见大率和小瑞。还有一位所方干部从远处走过来。他们三个人在花台附近立了一会儿,分手走开了。小瑞向我晾衣服的地方走来,我想跟他招呼一下,他却看也不看我一眼,一直走了过去。我不禁狐疑起来:“这是怎么回事?难道——他们真往绝处走吗?”\\

我回到屋里,找出了一些旧报纸,专挑上面关于宽大处理三反、五反案件的消息和文章来阅读。看了一阵,老王过来说:\\

“你干什么?研究五反?”\\

“不研究了。”我放下报纸,下了决心,“我想起过去的一些事,以前认识不到它的性质,现在看起来正是罪恶,把这些写到感想里你看好不好?”\\

“怎么不好?当然好啦!”他又放低声音说:“再说政府掌握咱们很多材料,还是先说了好。”\\

我拿起笔来了。在这份学习感想中,有一段的大意是:帝国主义侵略中国,离不开利用封建和买办的势力,我的经历就是个典型例子。以我为招牌的封建势力在复辟的主观幻想下,勾结日本帝国主义,而日本帝国主义则用这招牌,把东北变成了它的殖民地。我把在天津张园、静园的活动,我把我那一伙人与日本人的关系,以及我和\ruby{土肥原}{どいはら}见面的详情,原原本本地写了出来。\\

两天之后,组长老王告诉我,所方看到了我写的东西,认为我有了重大的进步,值得在本组里表扬。\\

“拿出一件真正的物证,比说一万句空话还有用。”干过“法官”的老王说。