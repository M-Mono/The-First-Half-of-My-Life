\fancyhead[LO]{{\scriptsize 1955-1959: 接受改造 · 日本战犯}} %奇數頁眉的左邊
\fancyhead[RO]{} %奇數頁眉的右邊
\fancyhead[LE]{} %偶數頁眉的左邊
\fancyhead[RE]{{\scriptsize 1955-1959: 接受改造 · 日本战犯}} %偶數頁眉的右邊
\chapter*{日本战犯}
\addcontentsline{toc}{chapter}{\hspace{1cm}日本战犯}
\thispagestyle{empty}
六七月间,我和几个同伴去沈阳,出席军事法庭,为审判日本战犯向法庭作证。\\

从报上知道,在中国共关押了一千多名日本战犯,一部分在抚顺,一部分在太原,都是日本帝国主义侵华战争时期中的犯罪分子。一九五六年的六月和七月,有四十五名分别在太原和沈阳判了徒刑,其余都受到了免诉处理,由中国红十字会协助他们回了国。在沈阳审判的是押在抚顺的战犯,两批审判共三十六名。有的是我在伪满时即已知名,有的是在抚顺管理所的大会讲坛上看见过。前伪满洲国总务厅次长\ruby{古海忠之}{ふるみ ただゆき}就是其中之一。他和伪总务厅长官\xpinyin*{武部六藏}是我和四名伪满大臣作证的对方。\xpinyin*{古海}是到庭的第一名被告人。他后来被法庭判处徒刑十八年\footnote{\ruby{古海忠之}{ふるみ ただゆき}已于1963年2月提前释放。}。\\

我走进这个审判侵略者的法庭的时候,忽然想起了朝鲜战争的胜利,想起了日内瓦谈判的胜利,想起了建国以来的外交关系。如今,在中国的土地上审判日本战犯,这更是历史上从来没有过的事情。\\

在志愿军和朝鲜人民军一起打胜仗的日子,我那时只想到,我除了向中国人民认罪求恕外,别无其它出路。到这次审判日本战犯时,出现在我心头的已不是出路问题,而是远远超过了个人问题的民族自豪感!\\

不,我得到的还不只限于民族自豪感。我从这件巨大的事件中,想到了更多更多的问题。\\

\xpinyin*{古海}在宣判前的最后陈述中说了这样的话:\\

“在东北全境,没有一寸土地没留下惨无人道的日本帝国主义者的暴行痕迹。帝国主义的罪行就是我的罪行。我深深认识到我是一个公然违反国际法和人道原则,对中国人民犯下了重大罪行的战争犯罪分子,我真心地向中国人民谢罪。对于我这样一个令人难以容忍的犯罪分子,六年来,中国人民始终给我以人道主义待遇,同时给了我冷静地认识自己的罪行的机会。由于这些,我才恢复了良心和理性。我知道了真正的人应该走的道路。我认为这是中国人民给我的,我不知道怎样来感激中国人民。”\\

我到如今还记得,我在法庭上作证发言后,庭上叫他陈述意见时,他深深鞠了一个躬,流着泪说道:\\

“证人所说的完全是事实。”\\

这情景不由我不想起东京国际法庭。在那里,日本战犯通过他们的律师叫嚣着,攻击着证人,为着减轻罪罚,百般设法,掩盖自己的罪行。而在这里,不仅是古海,不仅是我的作证对方而是所有受到审判的战犯全部认罪服刑。\\

关于日本战犯,我的弟弟和妹夫们,特别是记性好的老万,讲它几天也讲不完。他们从检举认罪开始,便参加翻译日本战犯大量的认罪材料,大批日本战犯遣送回国后,他们又协助管理所翻译大量的日本来信。妹夫们释放之后,这工作由\xpinyin*{溥杰}和老邦几个人担任。从一九五六年起,我就不断地零碎地从他们嘴里听到不少日本战犯的故事。\\

有个日本战犯,是前陆军将官,在一九五四年检察机关开始调查时,也许是由于他怕,也许是由于敌视,是从他嘴里查不出多少东西的。甚至在大会上,受到他的部下官兵的指控时,他还没放下自己的将官架子。但是这次在法庭上,他承认了指挥他的部队在冀东地区和河南浚县等地,进行过六次集体屠杀和平居民的罪行。例如,一九四二年十月,他属下的一个联队,在潘家戴庄屠杀了一千二百八十多名居民、烧掉民房一千多间的罪行。他在法庭面前承认了所有这些事实。他被判处二十年徒刑之后,向记者说:“在进行判决时,我按照我过去的罪行来判断,认为中国对我这样悖逆人道、违反国际公法的人,当然要从严处断,处以死刑。”他又说,在调查犯罪事实的时候,是非常正确而公正的,完全是用了他们在旧社会未曾见闻过的方法进行调查的。他说,尽管自己的罪恶没有什么辩护余地,可是法庭还是派了辩护人来,起诉书也是几天前送交他的,他觉得这是对他的人格的尊重。说到犯罪,他说:“当我想到我曾经杀害过很多的中国人民,使他们的遗属的生活遭到困难,而目前照顾我的正是被害者的亲人,这时候我的心有如刀割一般。”\\

有个日本前大住,受到了不起诉处分而被释放。我的三妹夫曾翻过一封从日本的来信,是和这位大住同船回国的一个战犯写的,信里提到日本记者知道了这个大住在监狱里被他的部下(也是战犯)追问过去的罪行时,很是恼怒,所以在船上访问了他,希望他说点和别人不同的话,因为战犯们对新中国的称赞和感激,已经使某些记者早不耐烦了。他们从大佐的嘴里并没有得到希望得到的任何东西,记者问他:“你为什么还是说那些话?你现在还怕中国吗?”他答:“我现在是坐在日本船上,对中国有什么怕的?我说的不过是事实罢了。”\\

三妹夫曾经担任过病号室的组长,他遇见过一个住病号室的日本兵战犯,他整天捣乱,不守监规,经常找护士和看守员的麻烦。到宣布了释放,开送别会的时候,他忽然哭了起来,当众讲出了自己的错误。还有个病号,虽然不像这个小兵那样捣乱,也是根本不想认罪的。他得的是直肠癌,因病情恶化把他送到医院里去急救,动了两次手术,做了人工肛门,而且医生为他输了自己的血,把他救活了。出院之后,他在一次大会上,当众叙述了他过去如何残杀和拷打中国人的罪行,又对照了中国人民在他病危中如何抢救了他。他在台上一面哭一面讲,台下的人也一面哭一面听……\\

有一天,我们平整场地、修建花坛,从院子里的土坑里挖出了一具白骨,头骨上有一个弹孔。学过西医的老元和老宪都判断死者生前是一个少女。后来,老万翻译了一个日本战犯的文章,这人是从前抚顺监狱的典狱长,他描述了那时关押爱国志士时的地狱景象:那时这里只有拷打声、镣铐声、惨叫声;那时这里又臭又脏,冬天墙上一层冰,夏天到处是蚊蝇;那时每个囚犯每天只给一小碗高粱米,还要终日做苦役,许多人被打死、累死。他说:“现在这里只有唱歌声、音乐声、欢笑声,如果有人走到围墙外,决不会想到这里是监狱;现在冬天有暖气,夏天有纱窗,过去苦役工厂成了锅炉房和面包房,从前爱国志士受折磨的暗室现在成了医务室的药房,从前的仓库现在修成了浴室,现在他们的人格受到尊重,他们每天可以学习,可以演奏乐器,可以绘画,可以打球,谁会相信这里是监狱?”他说:“现在中国正在建设给全人类带来幸福的事业,让我们走正当道路,不再犯罪,重新做人。”\\

在不少战犯写的文章中都说过,当他们被苏联送到中国来的时候,是恐惧的,是不服气的,甚至是仇恨的。有的人和我的心理一样,刚来的时候只会用自己的思想方法来推测,完全不理解为什么中国人民这样对待他们。他们看到修建锅炉房时,以为是盖杀人房,看到修建医务所、安装医疗设备时,以为也像他们干的那样,要用俘虏做试验。还有人把宽大和人道待遇看做是软弱。有个宪兵,在刚从苏联押到中国时是被日本战犯看做“日本好男子”的,终日大声叫骂。所方找他谈话,他侧身站在所方干部面前说:“我是苏联军队俘虏的,你们有什么资格来问我?”所方的人员对他说:“我们中国人民并没有请你到中国来杀人,但是有权利来向你追究你的血债!现在没资格说话的是你。你自己想想去吧。人到世界上来应该给人类做些有益的事,你做的什么呢?”他还以为要给他动刑,再给他一次逞硬的机会,可是就叫他这样去了,再没理他。不久,朝鲜战场上中国人民志愿军胜利的消息接二连三地传来了,他再也不闹了,因为他知道了讲道理的人并不是软弱,而野蛮却正是虚弱的表现。他变成了不声不响,终于自己主动地讲出了他的罪行。\\

日本战犯这些故事流传出来之前,日本战犯的变化是几乎人人皆知的。但我那时只顾考虑自己的问题,就像从前看报和看家信一样,无心认真去思索。其实从一九五四年前后起,日本战犯们的变化就不断地显露出来。我不如从\xpinyin*{溥杰}的残缺的一九五五年日记里抄些有关段落,借以说明(方括弧中的话是我的注解):\\

\begin{quote}
	一月二十六日晚间看日本战犯演舞踊及音乐剧[这是我们第一次看他们表演,以前是他们自演自看,他们这时已拥有一个相当规模的管弦乐队。乐器是所方为他们筹办的],都是取材我国人民解放军如何爱护人民、反帝及国际主义精神,和反对原子战争的日本人民的奋斗实例而成的。[剧终后]日本战犯们不少声泪俱下的表示反对美帝的原子能垄断[不少战犯说到自己亲人是死在原子弹之下的],并感谢我国人民政府之宽大政策。\\

五月二日白天仍是游戏了一天(因为过“五一”节,连着两天举行娱乐庆祝活动),晚间看日本战犯们的歌舞晚会,第六所的及第五所的前佐官级的战犯,也都参加了表演,这是向来所无的事,使我深刻地感到“新社会把鬼变成人”——“白毛女”影片上的话。\\

五月五日晚间看了(日本)战犯们的演剧“原爆之子”,才演了一场,因为晚间院内太冷(这天忽然起了风),所方怕出演者及观众(演出者只有日本战犯,观众是全体战犯)受了凉,遂临时中止,\xpinyin*{俟}天气好时再演(这个露天会场,是日本战犯用了不过三四天,就建筑起来的)。\\

五月六日今晚看了“原爆之子”,……情节颇感动人……(这写的是长崎受到战争惨祸的故事)。\\

五月十五日……参加亚洲会议的日本代表二十余人到这里参观,其代表团长声泪俱下地感谢了我国政府之对于战犯们的人道待遇。战犯代表也致答词,声言其改邪归正今后誓为保卫和平而斗争的决心,战犯们有很多人都感动得落下泪。所方并允许该代表团员与所认识的战犯们会见。\\

六月十一日终日看(日本)战犯所举行的运动会(这个运动场也是日本战犯自己修的),其组织性并其创意工夫,是可以供我们作参考的(在运动会上,他们的啦啦队很出色)。\\

七月四日晚间看(日本)战犯们的歌唱、音乐、舞蹈会。大约是片山哲来了罢,至深夜仍听到他们在欢呼拍掌。\\
\end{quote}

回想了一下,就觉出了他们的变化是很明显的。为什么这些身为囚犯的人变得那样高兴,那样生气勃勃?为什么在释放之后,坐在兴安丸上,还带着管理所送他们的那套管弦乐器,流着泪向逝去的中国的海岸吹奏?为什么他们最爱唱“东京——北京”?为什么连每个被判刑的人都在反复地说着:“我感激中国人民!”“我悔恨……”?\\

古海这样说,骂过人的这样说,耍过无赖的也这样说。从日本来的信里,常有这样的话:“我从中国知道了应当怎样活着”,“我认识了人生”,“在我踏出人生的第一步时,对于祝福我的身心健康与我握手的所长先生,你那手上的温暖是永不会失去的”\\

有几个战犯,从日本报纸、杂志上知道美国军队占领了他们的土地之后,出现了一种叫“胖胖女郎”的妇女职业,这是和我国解放前“吉普女郎”类似的现象,他们恼怒起来,骂那些女人不要脸。有人写信给他的妻子,问她是不是也干了这个。这封信经过检查,被所方管教人员留下来,拿着找到他,十分耐心地说:“你再考虑考虑,这样给妻子写,合适不合适?不用说你问得毫无根据,即使有根据,你也要想一想,这是谁的罪过?难道要叫一个女人负责吗?”这个战犯听了一声不响,突然他把那封信团起来扔在地上,然后抱头大哭起来。\\

是的,那些感激中国人民的人,不只是感激中国人民的宽大,他们更感激中国人民给他们认识了真理,明白了许多事情的真相。就像我认识了皇帝是怎么回事似的,他们也明白了军国主义的真相和日本的现实。他们回国之后来信谈到了少年犯罪数字的惊人,谈到了胖胖女郎的命运。在管理所放映过的日本电影《基地的儿童》、《战火中的妇女》都是现实。塞班岛的妇女在刺刀逼迫下走进海水,绝望的母亲用双手把自己刚出生的婴儿举到水面上,这些现实刚过去,美军的基地出现了,美国坦克轧着他们的土地,美军的飞机染污他们的天空,美国大兵奸污他们的妇女,……\\

一个回到农村的人,来信沉痛地说:“村中一部分青年变了,有当强盗的,有为了妇女问题而杀人的,有的参加了自卫队,沉溺在酒和妇女的堕落生活中。到了夜晚,如不把门窗关好就不敢安然地入睡。文化方面是腐败的,电影也是诲淫诲盗的多,还有从前时代的戏以及剑道柔道和射击的游戏。儿童做着杀人的游戏,对父母的吩咐也是不大听从。物资应有尽有,可是穷人是没钱买的……”\\

他们在中国认识到了真理,他们回去又看到了自己的祖国蒙受灾难的真相,他们一明白了这些道理,就组织起来、行动起来了。他们到处讲演,讲新中国,讲日本军国主义的罪恶,反对复活军国主义,要求独立民主与和平。他们何以如此呢?他们受到许多的限制、监视,但是他们并不畏缩,他们有很多办法对付那些限制。反动派不准他们演出中国的舞蹈,他们就把蒙古舞、扇舞、秧歌舞、红绸舞教给职业歌舞伎座,于是中国的红绸舞和秧歌舞传遍了日本全国各地。他们何以有这些办法呢?力量是哪里来的呢?\\

从妹夫们零星的但是兴奋的谈话中,我知道了在日本发生的许多关于归国战犯们的故事,这些故事归结出一个事实:他们到处受到日本人民的欢迎,他们把真理告诉了人民,人民支持了他们。\\

有许多人来信叙述他如何被他的家人、亲友、同乡,以及团体、学校邀请去讲他的监狱生活,讲中国的事情。他们讲了中国人民对日本人民的友情,讲了强大起来的中国对战争是什么态度,中国人民的希望和理想是什么。对他的话,有人怀疑,有人采取保留态度,有人相信。但越来越多的是相信,是肯定,是对于回去的人的信任。对于回去的人,亲美的反动统治者越不喜欢,人民却是越相信他……\\

他们一回国便出版了一本书:《三光政策》。那些亲身参与了日本军队在中国暴行的人写下了他们如何在中国土地上制造无人区,如何拿中国人民做细菌武器的试验,如何把活人解剖,……这本书第一版五万册,在一个星期里便卖光了!\\

有几位前军人、退伍的将军们,听了他们一位回国的旧同事谈了几年来的生活和感受后,默然良久,最后说:“凭了我们的良知和对你的了解,我们相信你所说的每一句话。不过,这些话只能是在屋里说。”\\

有一个村庄,在听了刚从中国回去的这位同乡说完以后,凡是有什么问题,人们总爱说:“找××去吧。他是我们村里懂得最多的人。”\\

有一个村庄,他们的刚刚回来的同乡不大爱说话,只是改变过去在家的习惯,乡亲们很诧异这个人为什么如今这样和善,这样爱帮助别人。当知道了这是在中国发生的变化以后,他成了村中更加有威信的人。\\

还有一个村庄,他们拿着“武运长久”的旗子,像欢迎凯旋的将军似地欢迎回国的人。但是这个受欢迎的人,一下了火车,就向他的乡亲们发表了一篇沉痛的演讲,结果人们明白了广岛的灾难原因,都流下了眼泪,“武运长久”的旗子也跌落在地上了……\\

有一个母亲,听她被释放回去的儿子讲述了十多年来的生活之后,便问道:“北京在哪里?”儿子告诉了她。她于是发现了褥垫放的不对头,不应当让双脚朝着这个方向,便急忙把褥垫调动过来,叫头朝着北京——那里是真理与希望。这是一个母亲的希望。\\

许许多多的战犯家属——他们许多都是朴实的劳动人民,或者具有良知的人。他们从前有不少人给中国政府写过信,要求释放他们的丈夫或儿子,说他们都是无罪的人。后来他们有人要求到中国来看他们的亲人,他们来了,听了亲人们的讲述,有的听了中国人民在法庭上控诉的录音,他们和监狱里的亲人一齐哭了,他们承认了监狱里的人是有罪的,明白了他们是上了军国主义的当。\\

日本战犯的变化,犹如我的家族的变化一样,给了我极大的震动。我从这些变化中看出了一个事实:共产党人是以理服人的。
