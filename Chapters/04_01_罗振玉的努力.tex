\fancyhead[LO]{{\scriptsize 1924-1930: 天津的“行在” · 罗振玉的努力}} %奇數頁眉的左邊
\fancyhead[RO]{} %奇數頁眉的右邊
\fancyhead[LE]{} %偶數頁眉的左邊
\fancyhead[RE]{{\scriptsize 1924-1930: 天津的“行在” · 罗振玉的努力}} %偶數頁眉的右邊
\chapter*{罗振玉的努力}
\addcontentsline{toc}{chapter}{\hspace{1cm} 罗振玉的努力}
\thispagestyle{empty}
到了天津,才知道并不像罗振玉所说的那样,“住处准备妥当”,因此我先在大和旅馆住了一天。次日婉容、文绣和日本使馆里的那一套人马都来了,才一同搬进匆忙布置起来的张园。\\

张园是一座占地约有二十亩的园子,中间有一座天津人称之为八楼八底的楼房。这是前清驻武昌第八镇统制张彪做游艺场用的地方。武昌起义时,张彪吓得连官印也不要了,带着他的金银财宝和家眷溜到天津,在日本租界里当了寓公。我刚住进了张园,这位前清的“名将”,坚决不收房钱,每天清晨都要带着一把扫帚,亲自来给我扫院子,大概是表示自己一贯矢忠之意。后来不知是经谁的劝阻,他才丢下那把扫帚。我在这里住了五年。后来张彪死了,他的儿子拿出房东的面孔要房租,我也嫌他的房子不好,于是又搬到了陆宗\xpinyin*{舆}的“静园”。\\

我到天津来的目的原是为了出洋,结果却一连住了七年。这是我在各派遗老、各种主意之间摇摆的七年。这时,王公们对我的左右力量,早已大为减弱;我父亲起初不大来天津,后来虽然常来(住在我原先买的英租界戈登路的房子里),对我也不发生什么作用。在这期间,庄士敦老师离开了我,又到威海卫当专员去了。威海卫被中国政府收回后,一九二六年他与北洋政府办理庚款问题时,到天津和我见过一次面。他曾为我奔走于吴佩孚等人之间,毫无结果。后来他回英国接受爵士爵位,做了伦敦大学的汉学教授兼英国外交部顾问。这七年间,在我身边进行勾心斗角的人物,大致可分为这几派:起初把希望放在恢复优待条件方面,后来又退缩为维持原状的,是以\xpinyin*{陈宝琛}为首的一批“旧臣”,可以称之为“还宫派”;把希望放在出洋以取得外国(主要是日本)援助上的,是以罗振玉为首,其中有遗老遗少,也有个别王公如溥伟之流,按当时的说法,可以称之为“联日”或“出洋”派;把希望放在联络、收买军阀方面,即所谓“用武人”一派,这派人物颇复杂,有前清遗老,也有民国的政客,中心人物却是我自己。后来又回到我身边的\xpinyin*{郑孝胥},起先并不属于哪一派,好像哪一派的主张他都赞成过,也反对过,他更提出过任何一派不曾提过的如所谓“用客卿”(外国人)、“门户开放”(同任何肯帮助复辟的国家勾结)等主张,因而也受过各派人的反对。当他后来一拿定了投靠日本这个主意,就战胜了一切对手。他不但胜过了他们,而且连他的老对手、“联日派”的老首领罗振玉,在这个阶段的争夺中又被他将多年经营来的成果,轻轻攫取到手。不过这也是后话,现在还是先把罗振玉说一说。\\

罗振玉到宫里来的时候,五十出头不多,中高个儿,戴一副金丝近视镜(当我面就摘下不戴),下巴上有一\xpinyin*{绺}黄白山羊胡子,脑后垂着一条白色的辫子。我在宫里时,他总是袍褂齐全,我出宫后,他总穿一件大襟式马褂,短肥袖口露出一截窄袍袖。一口绍兴官话,说话行路慢条斯理,节奏缓慢。他在清末做到学部参事,是原学部侍郎宝熙的旧部,本来是和我接近不上的,在我婚后,由于升允的推荐,也由于他的考古学的名气,我接受了\xpinyin*{陈宝琛}的建议,留作南书房行走,请他参加了对官中古彝器的鉴定。和他前后不多时间来的当时的名学者,有他的\xpinyin*{姻}亲王国维和以修元史闻名的柯劭囗。\xpinyin*{陈宝琛}认为南书房有了这些人,颇为清室增色。当然,罗振玉在复辟活动方面的名气比他在学术上的名气,更受到我的注意。他在辛亥革命那年东渡,在日本做了十年寓公,考古写书,自名“仇亭老民”。升允和善耆到日本活动,寻求复辟支援时,和他搅在一起,结了缘。后来,升允灰了心,在青岛住了一阵后,跑到天津日本租界里当寓公;善耆定居在旅顺大连,受日本人的\xpinyin*{豢}养。罗振玉比他们都活跃,他一九一九年回国,先住在天津,结交日本人,后来在大连码头开设了一个叫墨缘堂的古玩铺,一边走私贩卖古玩、字画,一边继续和日本人拉拉扯扯,广泛寻求复辟的同情者。\\

罗振玉在古玩、字画、金石、甲骨方面的骗钱行径,是由来已久的。他出身于浙江上虞县一个旧式书商之家,成年后在江西一个丘姓巨绅家教书。这位巨绅是个藏书家。罗振玉任西席的第三年,东翁突然去世,他利用女东家的无知,一方面装作十分哀痛的样子,拒绝接受这一年的束俯,要用以充做奠仪,另方面表示,愿留下东家的几件旧书和字画,作为纪念。女东家认为这位先生心眼太好,就请他自己到藏书楼任意挑选。于是这位书贾世子就精选出几筐“纪念品”,内有百余卷唐人写经,五百多件唐宋元明的字画,满载而归。在这个基础上,他由刻三字经、百家姓的书铺变成了古玩字画商,生意越做越好,古玩字画的鉴赏家的名声越来越大,后来更通过售卖古籍文物的路子,和日本人拉上了关系。他在日本的那些年,靠日本书商关系结交了一批朝野名流,有许多日本人把他看成了中国古文物学术的权威,常拿字画请他鉴定。他便刻了一些“罗振玉鉴定”、“罗振玉审定”的图章,日本古玩商拿字画请他盖一次,付他三元日金,然后再拿去骗人。后来他竟发展到仿刻古人名章印在无名字画上,另加上“罗振玉鉴定”章,然后高价出卖。他时常借口忙,把人家拿来请他鉴定的珍贵铜器,拖压下来,尽量多拓下一些拓片出卖。他的墨缘堂出售的宋版书,据说有一些就是用故宫的殿版《图书集成》里的扉页纸伪造的。殿版纸是成化纸或罗纹纸,极像宋版书用纸。据说内务府把那批殿版书交罗振玉代卖时,他把那一万多卷书的空白扉页全弄了下来,用仿宋体的刻版印了“宋版”书。我当时对这事是根本不知道的。有人说,罗振玉人品固然不佳,才学还好。据我看,他的才学究竟有多少,也很值得怀疑。在伪满时有一次他拿来一批汉玉请我观赏。我对汉玉说不上有什么研究,只是因为十分爱好,收藏了不少,所谓不怕不识货,就怕货比货。当然,所谓汉玉,并不是非汉朝的不可,这只不过是对古玉的惯称。我看过罗振玉拿来的汉玉,不禁对他的“才学”暗吃一惊,因为全部都是假货。\\

罗振玉并不经常到宫里来,他的\xpinyin*{姻}亲王国维能替他“当值”,经常告诉他当他不在的时候,宫里发生的许多事情。王国维对他如此服服帖帖,最大的原因是这位老实人总觉得欠罗振玉的情,而罗振玉也自恃这一点,对王国维颇能指挥如意。我后来才知道,罗振玉的学者名气,多少也和他们这种特殊瓜葛有关。王国维求学时代十分清苦,受过罗振玉的帮助,王国维后来在日本的几年研究生活,是靠着和罗振玉在一起过的。王国维为了报答他这份恩情,最初的几部著作,就以罗振玉的名字付\xpinyin*{梓}问世。罗振玉后来在日本出版、轰动一时的《殷墟书契》,其实也是窃据了王国维甲骨文的研究成果。罗、王二家后来做了亲家,按说王国维的债务更可以不提了,其实不然,罗振玉并不因此忘掉了他付出过的代价,而且王国维因他的推荐得以接近“天颜”,也要算做欠他的情分,所以王国维处处都要听他的吩咐。我到了天津,王国维就任清华大学国文教授之后,不知是由于一件什么事情引的头\footnote{我在特赦后,听到一个传说,因已无印象,故附记于此,聊备参考。据说绍英曾托王国维替我卖一点字画,罗振玉知道了,从王手里要了去,说是他可以办。罗振玉卖完字画,把所得的款项(一千多元)作为王国维归还他的债款,全部扣下。王国维向他索要,他反而算起旧账,王国维还要补给他不足之数。王国维气愤已极,对绍英的催促无法答复,因此跳水自尽。据说王遗书上“义无再辱”四字即指此而言。},罗振玉竟向他追起债来,后来不知又用了什么手段再三地去逼迫王国维,逼得这位又穷又要面子的王国维,在走投无路的情况下,于一九二七年六月二日跳进昆明湖自尽了。\\

王国维死后,社会上曾有一种关于国学大师殉清的传说,这其实是罗振玉做出的文章,而我在不知不觉中,成了这篇文章的合作者。过程是这样:罗振玉给张园送来了一份密封的所谓王国维的“遗折”,我看了这篇充满了孤臣孽子情调的临终忠谏的文字,大受感动,和师傅们商议了一下,发了一道“上谕”说,王国维“孤忠耿耿,\xpinyin*{深堪恻悯},……加恩\xpinyin*{谥予忠悫},派贝子\xpinyin*{溥伒}即日前往\xpinyin*{莫缀},赏给陀罗经被并洋二千元……”。罗振玉于是一面广邀中日名流、学者,在日租界日本花园里为“忠悫公”设灵公祭,宣传王国维的“完节”和“恩遇之隆,为振古所未有”,一面更在一篇祭文里宣称他相信自己将和死者“九泉相见,谅亦匪遥”。其实那个表现着“孤忠耿耿”的遗折,却是假的,它的翻造者正是要和死者“九泉相见”的罗振玉。\\

那时我身边的几个最善于勾心斗角的人,总在设法探听对手的行动,手法之一是收买对手的仆役,因而主人的隐私,就成了某些仆人的获利资本。在这上面最肯下功夫的,是\xpinyin*{郑孝胥}和罗振玉这一对冤家。罗振玉假造遗折的秘密,被\xpinyin*{郑孝胥}通过这一办法探知后,很快就在某些遗老中间传播开了。这件事情的真相当时并没有传到我耳朵里来,因为,一则\xpinyin*{谥法}业已踢了,谁也不愿担这个“欺君之罪”,另则这件事情传出去实在难听,这也算是出于遗老们的“爱国心”吧,就这样把这件事情给压下去了。一直到罗振玉死后,我才知道这个底细。近来我又看到那个遗折的原件,字写得很工整,而且不是王国维的手笔。一个要自杀的人居然能找到别人代缮绝命书,这样的怪事,我当初却没有察觉出来。\\

罗振玉给王国维写的祭文,很能迷惑人,至少是迷惑了我。他在祭文里表白了自己没有看见王国维的“封奏”内容之后,以臆测其心事的手法渲染了自己的忠贞,说他自甲子以来曾三次“犯死而未死”。在我出宫和进日本使馆的时候,他都想自杀过,第三次是最近,他本想清理完未了之事就死的,不料“公竟先我而死矣,公死,思遇之隆,为振古所未有,予若继公而死,悠悠之口或且谓予希冀恩泽”,所以他就不便去死了,好在“医者谓右肺大衰,知九泉相见,谅亦匪遥”。这篇祭文的另一内容要点,是说他当初如何发现和培养了那个穷书记\footnote{王国维在光绪\xpinyin*{戊戌}年为汪康年的司书,后入罗所办的“东文学社”求学。},这个当时“黯然无力于世”的青年如何在他的资助指点之下,终于“得肆力于学,蔚然成硕儒”。总之,王国维无论道德、文章,如果没有他罗振玉就成不了气候。那篇祭文当时给我的印象,就是这样。\\

但是,尽管我长久以来弄不清罗振玉的底细,而罗振玉在我身上所打的政治算盘,却一直不能如愿。在他最后败给\xpinyin*{郑孝胥}之前,仅\xpinyin*{陈宝琛}、\xpinyin*{胡嗣瑗}一伙就弄得他难于招架。在那一连串的、几起几落的争吵中,我自己则是朝三暮四,犹豫不决。\\

这两伙人起初的争论焦点,是出洋不出洋的问题。我从北京日本使馆跑到天津日本租界后,社会上的抨击达到一个新高潮。天津出现了一个“反清大同盟”专门和我作对。罗振玉这一伙人乘此机会便向我说,无论为了安全还是为了复辟,除了出洋别无他路可走。这一伙人的声势阵容,一时颇为浩大,连广东一位遗老陈伯陶也送上奏折说,“非外游不足以保安全,更不足以谋恢复”,并主张游历欧美之后可定居日本,以待时机变化。\xpinyin*{陈宝琛}这一伙则认为这完全是轻举妄动。他们认为一则冯玉祥未必能站得住脚,危险并不那么大;另则出洋到日本,日本未必欢迎。倘若在日本住不成,而国内又不能容,更不用想段祺瑞和张作霖之流会让我回到紫禁城,恢复以前的状况。我对\xpinyin*{陈宝琛}等人的意见不感兴趣,但他们提出的警告却引起了我的注意,对罗振玉的主张犯了犹豫。\\

一九二六年,政局曾经一度像\xpinyin*{陈宝琛}这一伙所希望的那样发生了变动,张作霖转而和吴佩孚联合,张、冯终于发生冲突,冯军遭到了奉军的攻击。冯玉祥撤走了天津的军队,北京的冯军处于包围之中。段祺瑞与张作霖勾结,被冯军发现,段祺瑞逃走了,随后冯军也在北京站不住脚,退往南口,奉军张宗昌进了北京。七月间,张、吴两“大帅”在北京的会面,引起“还宫派”无限乐观,还宫派活跃起来了。我身边的\xpinyin*{陈宝琛}亲自到北京,找他的旧交,新任的内阁总理杜锡珪去活动,在外面的康有为也致电吴佩孚、张作霖、张宗昌等人,呼吁恢复优待条件。康有为给吴佩孚写了一封长信,信中历数清朝的“功德”,并以“中华之为民国,以清朝让之,非民国自得之也”为理由,请吴佩孚乘机复辟。他对吴说,张作霖等人都没问题,外交方面也有同心,甚至“国民党人私下亦无不以复辟为然”,“全国士大夫无不疑民国而主复辟”,因此,“今但待决于明公矣”!\\

其实,这时已到了北洋军阀的回光返照时期。虽然北方各系军人忽然又合作了,张作霖又被公推为安国军总司令了,但一九二四年开始了国共第一次合作,一九二五年开始了国民革命军的北伐,到一九二六年,北伐军前锋势如破竹,孙传芳、吴佩孚、张作霖的前线军队,不住地溃败下来,他们正自顾不暇,哪有心思管什么优待条件?\xpinyin*{陈宝琛}没有活动出什么结果,吴佩孚给康有为的回信也很简单,敷衍说:“金石不渝,曲高无和必矣。”过了一年,康有为便抱着未遂之志死在青岛了。\\

还宫希望破灭了,\xpinyin*{陈宝琛}这一伙泄了气,罗振玉这边又活跃起来。一九二六年三月,当我正因北伐军的迫近而陷入忧虑之际,溥伟派人从旅顺给我送来奏折和致罗振玉的一封信,说他已和日方官绅接洽好,希望我迁到旅顺去住,“先离危险,再图远大”,“东巡西幸亦必先有定居”。我因为对罗振玉的闲话听得多了,已经对他有些不放心,不过我对溥伟的印象颇好。我到天津不久,溥伟从旅顺跑来给我请安,这位初次见面的“恭亲王”,向我说了一句很令我感动的话:“有我溥伟在,大清就不会亡!”我看了他劝我到旅顺的信,自然有些动心。因为他通过了罗振玉来劝我,所以我对罗的怀疑也消除了不少。后来,北伐军占领了武昌,北方军队全线动摇,罗振玉更向我宣传革命军全是“洪水猛兽”,“杀人放火”,倘若落在他们手里,决无活路。我听了这些话,已经决定随他去大连了,但由于\xpinyin*{陈宝琛}的劝告,又决定暂缓。\xpinyin*{陈宝琛}从日本公使馆得到的消息,事情似乎并不那么令人悲观。我观望了不久,果然,国民党的清党消息来了,蒋介石在成批地屠杀被指做“洪水猛兽”的共产党人,在这前后时间里,还接二连三地传来了英国军舰炮轰南京,日本出兵山东,阻挡南方军队北上的消息。这些消息让我相信了\xpinyin*{陈宝琛}那伙人的稳健,觉得事情确不像罗振玉这伙人说得那么严重。蒋介石既然和袁世凯。段祺瑞、张作霖一样的怕洋人,我住在外国租界,不是和以前一样的保险吗?\\

“还宫”和“出洋”这两派人的最终理想,其实并不矛盾,他们是一致希望复辟的。\xpinyin*{陈宝琛}这一伙人在还宫希望破灭之后,重弹起“遵时养晦”的老调,主张采取“静待观变”的政策,但是他们在“联日”方面,也并非反对罗振玉那伙人的主张。例如一位南书房行走叫温肃的遗老(张勋复辟时做过十二天的都察院副都御史),曾上奏说,“\xpinyin*{陈宝琛}有旷世之才,与芳泽甚密”,“行在”设在天津,可由陈与芳泽就近联系“密商协助\xpinyin*{饷}械,规定利权”,以“厚结外援,暗树势力”,“津京地近,往返可无痕迹”。有一个比温肃更讨厌罗振玉的张琨(前清顺天府文安县知县,候补知州),他对于出洋之所以不太支持,原因不过如此:“出洋如为避祸,以\xpinyin*{俟}复辟转圜则可,若再以彼道义之门、治平之范,弃其学而学焉,则大不可也”。可见他并不完全反对罗振玉的出洋理由。甚至\xpinyin*{陈宝琛}也曾一度让步说,倘若非要出洋不可,只望我选可靠的扈从人员。原来问题的真正焦点,还是在于反对罗振玉这个人。现在我能记得起的最坚决反对出洋的遗老,是极个别的,甚至也有人说过“日本推利是图,不会仗义协助复辟”的话,他们认为复辟只能放在“遗臣遗民”身上,在他们的遗臣遗民里,是要把罗振玉剔除出去的。\\

两伙人既然不是什么主张、办法上的争执,而是人与人的争执,因此在正面的公开条陈议论之外,暗地里勾心斗角就更为激烈。在这方面,罗振玉尽管花样再多,结果仍是个失败者。\\

有一天,罗振玉得到我的召见允许,到我的小召见室里来了。他拿着一个细长的布包儿,对我说:\\

“臣罪该万死,不当以此扰乱天心,然而臣若为了私交,只知隐恶扬善,则又不忠不义。”\\

“你说的什么呀?”\\

我莫名其妙地望着他,只见他慢慢腾腾,就像个老太监洗脸梳头似地,动手解那个包儿。包儿打开了,里面是一副对联,他不慌不忙地把它展开,还没展完,我就认出来,这是我写给\xpinyin*{陈宝琛}的。\\

“臣在小市上发现的\xpinyin*{宸翰御墨},总算万幸,被臣请回来了……”\\

那时我还不知道,罗振玉这些人一贯收买敌对者的仆役,干些卑鄙的勾当,我只想到\xpinyin*{陈宝琛}居然对皇上的“恩赐”如此不敬,居然使我的御笔摆到小市的地摊上!我心中十分不快,一时烦恼之至,不知说什么是好,只好挥挥手,叫罗振玉赶快走开。\\

这时\xpinyin*{陈宝琛}到北京去了。\xpinyin*{胡嗣瑗}知道了这件事,他坚持说,这决不是\xpinyin*{陈宝琛}的过失,他不相信陈家的仆人敢把它拿到小市上去,但又说陈家的仆人偷出去卖倒是可能。至于不卖给小市又卖给谁?为什么会到了罗振玉手里?他却不说出来。在我追问之下,他只说了一个叫我摸不着头脑的故事。\\

“嘉庆朝大学士松筠,皇上必能知道,是位忠臣。松筠的故事,皇上愿意听,臣就讲一讲。嘉庆二十四年,仁宗睿皇帝要御驾巡幸出关,大学士松筠知道了,心中不安,一则仁宗圣躬违和,如何能经这番奔波?另则和珅虽然伏诛,君侧依然未净,只怕仁宗此去不吉。松筠心中有话不能向上头明说,只好在奏折上委婉其词,托词夜观天象,不宜出巡。仁宗阅奏大怒,下谕一道,说自古以孝治天下,朕出关祭祀祖宗,岂有不吉之理?因此松筠夺官,降为骁骑校。仁宗后来在热河行宫龙驭上宾,宣宗(道光)即位还朝,一进西直门,看见了\xpinyin*{松筠},带着兵了设街,想起了\xpinyin*{松筠}进\xpinyin*{谏}大行皇帝的那些话,明白了话中的含意,才知道这才是忠心耿耿的重臣,立即官复原职……”\\

说到这里,\xpinyin*{胡嗣瑗}停住了。我着急地问:\\

“你说的什么呀?这跟\xpinyin*{陈宝琛}有什么关系?”\\

“臣说的是\xpinyin*{陈宝琛},跟\xpinyin*{松筠}一样,有话不好明说。”\\

“那么我是仁宗还是宣宗?”\\

“不,不……”\xpinyin*{胡嗣瑗}吓得不知说什么是好了。我不耐烦地说:\\

“你是个干脆人,别也学那种转弯抹角的,干脆说吧!”\\

“嗻,臣说的\xpinyin*{陈宝琛},正是忠心耿耿,只不过他对上头进谏,一向是迂回的,皇上天直聪明,自然是能体察到的。”\\

“行啦,我知道陈师傅是什么人。”\\

我虽然还不明白松筠的故事的含义,也乐意听\xpinyin*{胡嗣瑗}说陈师傅的好话,至少这可以除去那副对联所引起我心里的不舒服,但愿它真是贼偷去的就好了。\\

罗振玉经过一连串的失败,特别是在后面将要讲到的另外一件事上,更大大失掉了我的信任,他终于在一九二八年末搬到旅顺另觅途径去了。\\

这里暂且不叙遗老们之间的争斗,先谈一谈使我留津而不想出洋的另外的原因,这就是我对军阀的希望。
