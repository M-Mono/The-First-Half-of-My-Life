\fancyhead[LO]{{\scriptsize 1955-1959: 接受改造 · 特赦}} %奇數頁眉的左邊
\fancyhead[RO]{} %奇數頁眉的右邊
\fancyhead[LE]{} %偶數頁眉的左邊
\fancyhead[RE]{{\scriptsize 1955-1959: 接受改造 · 特赦}} %偶數頁眉的右邊
\chapter*{特赦}
\addcontentsline{toc}{chapter}{\hspace{1cm}特赦}
\thispagestyle{empty}
\begin{quote}
	中国共产党中央委员会的建议\\

全国人民代表大会常务委员会:\\

中国共产党中央委员会向全国人民代表大会常务委员会建议:在庆祝伟大的中华人民共和国成立十周年的时候,特赦一批确实已经改恶从善的战争罪犯、反革命罪犯和普通刑事罪犯。\\

我国的社会主义革命和社会主义建设已经取得了伟大胜利。我们的祖国欣欣向荣,生产建设蓬勃发展,人民生活日益改善。人民民主专政的政权空前巩固和强大。全国人民的政治觉悟和组织程度空前提高。国家的政治经济情况极为良好。党和人民政府对反革命分子和其他罪犯实行的惩办和宽大相结合、劳动改造和思想教育相结合的政策,已经获得伟大的成绩。\\

在押各种罪犯中的多数已经得到不同程度的改造,有不少人确实已经改恶从善。根据这种情况,中国共产党中央委员会认为,在庆祝伟大的中华人民共和国成立十周年的时候,对于一批确实已经改恶从善的战争罪犯、反革命罪犯和普通刑事罪犯,宣布实行特赦是适宜的。采取这个措施,将更有利于化消极因素为积极因素,对于这些罪犯和其他在押罪犯的继续改造,都有重大的教育作用。这将使他们感到在我们伟大的社会主义制度下,只要改恶从善,都有自己的前途。\\

中国共产党中央委员会提请全国人民代表大会常务委员会考虑上述建议,并且作出相应的决议。\\

\begin{flushright}
	中国共产党中央委员会主席\\

\xpinyin*{毛泽东}\\

一九五九年九月十四日\\
\end{flushright}
\end{quote}

毛主席的建议和刘主席的特赦令所引起的欢腾景象,我至今是难忘的。\\

广播员的最后一句话说完,广播器前先是一阵短暂的沉寂,然后是一阵欢呼、口号和鼓掌所造成的爆炸声,好像是一万挂鞭同时点燃,响成一片,持久不停。\\

从九月十八日清晨这一刻起,全所的人就安静不下来了。\\

战犯们议论纷纷。有的说党和政府永远是说一是一,说二是二。有的说,这下子可有奔头了。有的说,奔不了多久,出去的日子就到了。有的说,总要分批特赦,有先有后。有的说,也许是全体一齐出去。有的说,第一批里一定有某某、某某……。然而更多的人都明白,特赦与否是看改恶从善的表现的,因此不少人对最近以来自己的松懈倾向,有些后悔。同时也有的人口头上“谦虚”地说自己不够标准,暗地里却悄悄整理衣物,烧掉废笔记本,扔掉了破袜子。\\

休息的时候,院子里人声嘈杂。我听见老元对老宪说:\\

“头一批会有谁呢?”\\

“这次学习成绩评比得奖的没问题吧?你很可能。”\\

“我不行。我看你行。”\\

“我吗?如果我出去,一定到北京给你们寄点北京土产来。我可真想吃北京蜜枣。”\\

在院子里的另一头传来了大下巴的声音:\\

“要放都放,要不放就都别放!”\\

“你是自己没信心,”有人对他说,“怕把你剩下!”\\

“剩我?”大下巴又红了眼睛,“除非剩下\ruby{溥仪}{\textcolor{PinYinColor}{\Man ᡦᡠ ᡳ}},要不剩他就不会剩我。”\\

他说的不错,连我自己也是这样看的。大概是第二天,副所长问我对特赦的想法,我说:\\

“我想我只能是最后一个,如果我还能改好的话。但是我一定努力。”\\

特赦释放,对一般囚犯说来,意味着和父母子女的团聚,但这却与我无太大的关系。我母亲早已去世,父亲殁于一九五一年,最后一个妻子也于一九五六年跟我办了离婚手续。即使这些人仍在,他们又有谁能像这里的人那样了解我呢?把我从前所有认识的人都算上,有谁能像这里似的,能把做人的道理告诉我呢?如果说,释放就是获得自由和“阳光”,那么我要说,我正是在这里获得了真理的阳光,得到了认识世界的自由。\\

特赦对我说来,就是得到了做人的资格,开始了真正有意义的新生活。\\

在不久以前,我刚接到老万一封信,那信中说他的学地质的儿子,一个大学的登山队队长,和同学们在征服了祁连山的雪峰之后到了西藏,正巧碰上了农奴主的叛乱,他和同学们立即同农奴们站在一起,进行了战斗。叛乱平息后,他又和同学们向新的雪峰前进了。在老万的充满自豪和幸福的来信中,屡次谈到他的孩子是生长在今天,幸而不是那个值得诅咒的!日时代。今天的时代,给他的孩子铺开了无限光明的前程。如果不是这样的时代,他不会有这样的儿子,他自己也不会有今天,他如今被安排到一个编译工作部门做翻译工作,成了一名工作人员,一名社会主义事业的建设者,和每个真正的中国人一样了。他祝愿我早日能和他一同享受这种从前所不知道的幸福。他相信,这正是我日夜所向往的。……\\

特赦令颁布的一个月后,我们一所和七所的人一同又外出参观。我们又一次到了大伙房水库。上次一九五七年我们来看大伙房水库时,只看到一望无际的人群,活动在山谷间,那时,我们从桌子上的模型上知道它将蓄水二十一点一亿公方,可以防护千年一遇的洪水(一万零七百秒公方),同时还可灌溉八万顷土地。我们这次参观时,已是完工了一年的伟大杰作——一座展开在我们面前的浩瀚的人造海,一条高出地面四十八米、顶宽八米、底宽三百三十米、长达一千三百六十七米的大坝。日本战犯、伪满总务厅次长\ruby{古海}{\textcolor{PinYinColor}{ふるみ}}\ruby{忠之}{\textcolor{PinYinColor}{ただゆき}}这次参观回来,在俱乐部大厅里向全体战犯发表他的感想时,他说了这样一段话:\\

\begin{quote}
	“站在大伙房水库的堤坝上四面眺望,我感觉到的是雄伟。美丽、和平,我还深深地感到这是与自然界作斗争的胜利,这是正在继续战胜自然的中国人民的自豪和喜悦。……看到这样的水库,使我脑海里回忆起来,在伪满时代当总务厅主计处长、经济部次长、总务厅次长等职务时,站在水丰水库堤坝上眺望的往事;那时也认为是对大自然作斗争,认为能做这样世界上大工程的在亚洲只有日本人,而感到骄傲;蔑视中国人是绝对不可办到的(那时,为了准备战争非做不可的工作很多,在劳力方面虽强迫征用仍感不足,材料也没有,这个大伙房水库计划就打消了)。中国工人衣服破烂不堪,我认为自己和这些人比,完全是另一种人;我以‘伟大的、聪明的、高尚的’人的姿态,傲慢地看着他们。”\\

“在大伙房水库劳动着的人们,由于他们充满了希望,有着冲天的干劲,忘我的劳动,蓬勃的朝气,眉宇间显示出无比的自豪和喜悦。站在小高堤的一角眺望着的我,就是对中国人民犯下严重罪行的战争罪犯。哪一方面是对的呢?……”\\

一边站的是“眉宇间显示出无比的自豪和喜悦”的中国人民,一边站的是犯下严重罪行的战犯。我心里向往的就是脱离了后一边,丢掉这一边的身份,站到前一边,即“对的”那一边来。这是我经过十年来的思索,找出的惟一道路。\\
\end{quote}

十年来的经历和学习,使我弄清了根本的是非。这十年间,抗美援朝的胜利,日本战犯的认罪,中国在外交上的胜利和国际声誉的空前提高,国家、社会、民族的变化,包括我的家族以及往最小处说,例如我自己体质上的变化,这一切奇迹都是在共产党——十年以前我对它只有成见、敌意和恐惧——的领导下发生的。这十年来的事实以及一百多年的历史,对我说明:决定历史命运的,正是我原先最看不起的人民;我在前半生走向毁灭是必然的,我从前恃靠的帝国主义和北洋反动势力的崩溃也是必然的。我明白了从前\xpinyin*{陈宝琛}、\xpinyin*{郑孝胥}、\ruby{吉冈}{\textcolor{PinYinColor}{よしおか}}\ruby{安直}{\textcolor{PinYinColor}{やすなお}}以及神仙菩萨所不能告诉我的所谓命运,究竟是什么,这就是老老实实做一个自食其力、有益于人类的人。和人民的命运联结在一起的命运,才是最好的命运。\\

“哪一方面是对的,便站到哪一方面去。”\\

这是需要勇气的。特赦令给我鼓起了勇气。而且对每个人都一样。\\

我们学习、劳动更起劲了。许多人等待着下次的学习评比。食品加工组做出的豆腐又白又嫩,畜牧组的猪喂得更上膘了,我们医务组消除了任何差误,甚至连大下巴也老实起来,没跟人吵过嘴。\\

又一个多月过去了。一天晚上,副所长找我谈话,谈起特赦问题,问我:“这两个月你怎么想的?”\\

我把我前面想的说了,并且认为有几个人改造得不坏,我举出了畜牧组的、食品加工组的,以及上次学习评比得奖的几个人。\\

“你现在比较容易想到别人的长处了。”副所长笑着说,“如果特赦有你,你如何想呢?”\\

“不可能的。”我笑笑说。\\

不可能的。我回到屋里还是这样想。“如果……有呢?”一想到这里,我忽然紧张起来。后来想,将来会有的,还要一个相当长的时间。总之,希望是更大了。我不禁幻想起来,幻想着我和老万、小瑞他们一样,列身在一般人之间,做着一般人的事,我幻想着可能由劳动部门分配到一个医疗单位,当一名医务助理员,就像报上所描写的那样,……但是,这是需要一个相当长的时间的,需要等到人民批准了我,承认我是他们中间的一分子。想着未来的幸福,我几乎连党都睡不着了。\\

第二天,得到了集合的通知,我们走进了俱乐部大厅,迎面看见了台上的巨幅大红横披,我的呼吸急促了。横披上写着的是:“抚顺战犯管理所特赦大会”。\\

台上坐着最高人民法院的代表、两位所长和其他一些人。台下是静悄悄的,似乎可以听见心跳的声音。\\

首长简短地谈了几句话之后,最高人民法院的代表走到讲台当中,拿出一张纸来,念道:\\

\begin{quote}
	“\ruby{爱新觉罗}{\textcolor{PinYinColor}{\Man ᠠᡳᠰᡳᠨ ᡤᡳᠣᡵᠣ}}·\ruby{溥仪}{\textcolor{PinYinColor}{\Man ᡦᡠ ᡳ}}!”\\
\end{quote}

我心里激烈地跳动起来。我走到台前,只听上面念道:\\

\begin{quote}
	中华人民共和国最高人民法院特赦通知书\\

遵照一九五九年九月十七日中华人民共和国主席特赦令,本院对在押的伪满洲国战争罪犯\ruby{爱新觉罗}{\textcolor{PinYinColor}{\Man ᠠᡳᠰᡳᠨ ᡤᡳᠣᡵᠣ}}·\ruby{溥仪}{\textcolor{PinYinColor}{\Man ᡦᡠ ᡳ}}进行了审查。

罪犯\ruby{爱新觉罗}{\textcolor{PinYinColor}{\Man ᠠᡳᠰᡳᠨ ᡤᡳᠣᡵᠣ}}·\ruby{溥仪}{\textcolor{PinYinColor}{\Man ᡦᡠ ᡳ}},男性,五十四岁,满族,北京市人。该犯关押已经满十年,在关押期问,经过劳动改造和思想教育,已经有确实改恶从善的表现,符合特赦令第一条的规定,予以释放。\\

\begin{flushright}
	中华人民共和国最高人民法院\\

一九五九年十二月四日\\
\end{flushright}
\end{quote}

不等听完,我已痛哭失声。祖国,我的祖国阿,你把我造就成了人!……