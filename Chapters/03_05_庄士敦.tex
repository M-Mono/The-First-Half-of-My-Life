\fancyhead[LO]{{\scriptsize 1917-1924: 北京的“小朝廷” · 庄士敦}} %奇數頁眉的左邊
\fancyhead[RO]{} %奇數頁眉的右邊
\fancyhead[LE]{} %偶數頁眉的左邊
\fancyhead[RE]{{\scriptsize 1917-1924: 北京的“小朝廷” · 庄士敦}} %偶數頁眉的右邊
\chapter*{庄士敦}
\addcontentsline{toc}{chapter}{\hspace{1cm}庄士敦}
\thispagestyle{empty}
我第一次看见外国人,是在\xpinyin*{隆裕}太后最后一次招待外国公使夫人们的时候。我看见那些外国妇女们的奇装异服,特别是五颜六色的眼睛和毛发,觉得他们又寒怆,又可怕。那时我还没看见过外国的男人。对于外国男人,我是从石印的画报上,得到最初的了解的:他们嘴上都有个八字胡,裤腿上都有一条直线,手里都有一根棍子。据太监们说,外国人的胡子很硬,胡梢上可以挂一只灯笼,外国人的腿根直,所以\xpinyin*{庚子}年有位大臣给西太后出主意说,和外国兵打仗,只要用竹竿子把他们捅倒,他们就爬不起来了。至于外国人手里的棍子,据太监说叫“文明棍”,是打人用的。我的\xpinyin*{陈宝琛}师傅曾到过南洋,见过外国人,他给我讲的国外知识,逐渐代替了我幼时的印象和太监们的传说,但当我听说要来个外国人做我的师傅的时候,我这个十四岁的少年仍满怀着新奇而不安之感。\\

我的父亲和中国师傅们“引见”\ruby{雷湛奈尔德}{\textcolor{PinYinColor}{Reginald}}·\ruby{约翰}{\textcolor{PinYinColor}{John}}·\ruby{弗莱明}{\textcolor{PinYinColor}{Fleming}}·\ruby{庄士敦}{\textcolor{PinYinColor}{Johnston}}先生的日子,是一九一九年三月四日,地点在\xpinyin*{毓}庆宫。首先,按着接见外臣的仪式,我坐在宝座上,他向我行鞠躬礼,我起立和他行握手礼,他又行一鞠躬礼,退出门外。然后,他再进来,我向他鞠个躬,这算是拜师之礼。这些礼都完了,在\xpinyin*{朱益藩}师傅陪坐下,开始给我讲课。\\

我发现\ruby{庄士敦}{\textcolor{PinYinColor}{Johnston}}师傅倒并不十分可怕。他的中国话非常流利,比陈师傅的福建话和朱师傅的江西话还好懂。庄师傅那年大约四十岁出头,显得比我父亲苍老,而动作却敏捷灵巧。他的腰板根直,我甚至还怀疑过他衣服里有什么铁架子撑着。虽然他没有什么八字胡和文明棍,他的腿也能打弯,但总给我一种硬梆梆的感觉。特别是他那双蓝眼睛和淡黄带白的头发,看着很不舒服。\\

他来了大概一个多月之后,一天他讲了一会书,忽然回过头去,恶狠狠地看了立在墙壁跟前的太监一眼,涨红了脸,忿忿地对我说:\\

“内务府这样对待我,是很不礼貌的。为什么别的师傅上课没有太监,惟有我的课要一个太监站在那里呢?我不喜欢这样。”他把“喜”的音念成See,“我不喜欢,我要向徐总统提出来,因为我是徐总统请来的!”\\

他未必真的去找过总统。清室请他当我的师傅,至少有一半是为着靠他“保镖”,因此不敢得罪他。他一红脸,王爷和大臣们马上让了步,撤走了太监。我感到这个外国人很厉害,最初我倒是规规矩矩地跟他学英文,不敢像对中国师傅那样,念得腻烦了就瞎聊,甚至叫师傅放假。\\

这样的日子只有两三个月,我就发现,这位英国师傅和中国师傅们相同的地方越来越多。他不但和中国师傅一样恭顺地称我为皇上,而且一样地在我念得厌烦的时候,推开书本陪我闲聊,讲些山南海北古今中外的掌故。根据他的建议,英文课添了一个伴读的学生。他也和中国师傅的做法一模一样。\\

这位苏格兰老夫子是英国牛津大学的文学硕士。他到宫里教书是由老洋务派\xpinyin*{李经迈}(\xpinyin*{李鸿章}之子)的推荐,经\xpinyin*{徐世昌}总统代向英国公使馆交涉,正式被清室聘来的。他曾在香港英总督府里当秘书,入宫之前,是英国租借地威海卫的行政长官。据他自己说,他来亚洲已有二十多年,在中国走遍了内地各省,游遍了名山大川,古迹名胜。他通晓中国历史,熟悉中国各地风土人情,对儒、墨、释、老都有研究,对中国古诗特别欣赏。他读过多少经史子集我不知道,我只看见他像中国师傅一样,摇头晃脑抑扬顿挫地读唐诗。\\

他和中国师傅们同样地以我的赏赐为荣。他得到了头品顶戴后,专门做了一套清朝袍褂冠带,穿起来站在他的西山樱桃沟别墅门前,在我写的“乐静山斋”四字匾额下面,拍成照片,广赠亲友。内务府在地安门油漆作一号租了一所四合院的住宅,给这位单身汉的师傅住。他把这个小院布置得俨然像一所遗老的住宅。一进门,在门洞里可以看见四个红底黑字的“门封”,一边是“\xpinyin*{毓}庆宫行走”、“赏坐二人肩\xpinyin*{舆}”,另一边是“赐头品顶戴”、“赏穿带股貂褂”。每逢受到重大赏赐,他必有谢恩折。下面这个奏折就是第一次得到二品顶戴的赏赐以后写的:\\

\begin{quote}
	臣\ruby{庄士敦}{\textcolor{PinYinColor}{Johnston}}跪奏为叩谢天恩事。\ruby{宣统}{\textcolor{PinYinColor}{\Man ᡤᡝᡥᡠᠩᡤᡝ ᠶᠣᠰᠣ}}十三年十二月十三日钦奉谕旨:\ruby{庄士敦}{\textcolor{PinYinColor}{Johnston}}教授英文,三年匪懈,著加恩赏给二品顶戴,仍照旧教授,并赏给带膆貂褂一件,钦此。闻命之下,实不胜感激之至,谨恭折叩谢皇上天恩。谨奏。\\
\end{quote}

\ruby{庄士敦}{\textcolor{PinYinColor}{Johnston}}采用《论语》“士志于道”这一句,给自己起了个“志道”的雅号。他很欣赏中国茶和中国的牡丹花,常和遗老们谈古论今。他回国养老后,在家里专辟了一室,陈列我的赐物和他的清朝朝服、顶戴等物,并在自己购置的小岛上悬起“满洲国”的国旗,以表示对皇帝的忠诚。然而最先造成我们师生的融洽关系的,还是他的耐心。今天回想起来,这位爱红脸的苏格兰人能那样地对待我这样的学生,实在是件不容易的事。有一次他给我拿来了一些外国画报,上面都是关于第一次世界大战的图片,大都是显示协约国军威的飞机坦克大炮之类的东西。我让这些新鲜玩艺吸引住了。他看出了我的兴趣,就指着画报上的东西给我讲解,坦克有什么作用,飞机是哪国的好,协约国军队怎样的勇敢。起初我听得还有味道,不过只有一会儿功夫我照例又烦了。我拿出了鼻烟壶,把鼻烟倒在桌子上,在上面画起花来。庄师傅一声不响地收起了画报,等着我玩鼻烟,一直等到下课的时候。还有一次,他给我带来一些外国糖果,那个漂亮的轻铁盒子,银色的包装纸,各种水果的香味,让我大为高兴。他就又讲起那水果味道是如何用化学方法造成的,那些整齐的形状是机器制成的。我一点也听不懂,也不想懂。我吃了两块糖,想起了桧柏树上的蚂蚁,想让他们尝尝化学和机器的味道,于是跑到跨院里去了。这位苏格兰老夫子于是又守着糖果盒子,在那里一直等到下课。\\

庄师傅教育我的苦心,我逐渐地明白了,而且感到高兴,愿意听从。他教的不只是英文,或者说,英文倒不重要,他更注意的是教育我像个他所说的英国绅士那样的人。我十五岁那年,决心完全照他的样来打扮自己,叫太监到街上给我买了一大堆西装来。“我穿上一套完全不合身、大得出奇的西服,而且把领带像绳子似地系在领子的外面。当我这样的走进了\xpinyin*{毓}庆宫,叫他看见了的时候,他简直气得发了抖,叫我赶快回去换下来。第二天,他带来了裁缝给我量尺寸,定做了英国绅士的衣服。后来他说:\\

“如果不穿合身的西装,还是穿原来的袍褂好。穿那种估衣铺的衣服的不是绅士,是……”是什么,他没说下去。\\

“假如皇上将来出现在英国伦敦,”他曾对我说,“总要经常被邀请参加茶会的。那是比较随便而又重要的聚会,举行时间大都是星期三。在那里可以见到贵族、学者、名流,以及皇上有必要会见的各种人。衣裳不必太讲究,但是礼貌十分重要。如果喝咖啡像灌开水,拿点心当饭吃,或者叉子勺儿叮叮当当的响。那就坏了。在英国,吃点心、喝咖啡是Refreshment(恢复精神),不是吃饭……”\\

尽管我对\ruby{庄士敦}{\textcolor{PinYinColor}{Johnston}}师傅的循循善诱不能完全记住,我经常吃到第二块点心就把吃第一块时的警惕忘得一干二净,可是画报上的飞机大炮、化学糖果和茶会上的礼节所代表的西洋文明,还是深深印进了我的心底。从看欧战画报起,我有了看外国画报的爱好。我首先从画报上的广告得到了冲动,立刻命令内务府给我向外国定购画报上那样的洋犬和钻石,我按照画报上的样式,叫内务府给我买洋式家具,在养心殿装设地板,把紫檀木装铜活的炕几换成了抹着洋漆、装着白瓷把手的炕几,把屋子里弄得不伦不类。我按照\ruby{庄士敦}{\textcolor{PinYinColor}{Johnston}}的样子,大量购置身上的各种零碎:怀表、表链、戒指、别针、袖扣、领带,等等。我请他给我起了外国名字,也给我的弟弟妹妹们和我的“后”“妃”起了外国名字,我叫\ruby{亨利}{\textcolor{PinYinColor}{Henry}},\xpinyin*{婉容}叫\ruby{伊莉莎白}{\textcolor{PinYinColor}{Elizabeth}}。我模仿他那种中英文夹杂着的说话方法,成天和我的伴读者交谈:\\

“\ruby{威廉姆}{\textcolor{PinYinColor}{Williams}}(\ruby{溥杰}{\textcolor{PinYinColor}{\Man ᡦᡠ ᡤᡳᠶᡝ}}的名字),快给我把Pencil(铅笔)削好,……好,放在Desk(桌子)上!”\\

“\ruby{阿瑟}{\textcolor{PinYinColor}{Arthur}}(\xpinyin*{溥佳}的名字),Today(今天)下晌叫莉莉(我三妹的名字)他们来,Hear(听)外国军乐!”\\

说的时候,洋洋得意。听得\xpinyin*{陈宝琛}师傅皱眉闭目,像酸倒了牙齿似的。\\

总之,后来在我眼里,\ruby{庄士敦}{\textcolor{PinYinColor}{Johnston}}的一切都是最好的,甚至连他衣服上的樟脑味也是香的。\ruby{庄士敦}{\textcolor{PinYinColor}{Johnston}}使我相信西洋人是最聪明最文明的人,而他正是西洋人里最有学问的人。恐怕连他自己也没料到,他竟能在我身上发生这样大的魅力:他身上穿的毛呢衣料竟使我对中国的丝织绸缎的价值发生了动摇,他口袋上的自来水笔竟使我因中国人用毛笔宣纸而感到自卑。自从他把英国兵营的军乐队带进宫里演奏之后,我就更觉中国的丝弦不堪入耳,甚至连丹陛大乐的威严也大为削弱。只因\ruby{庄士敦}{\textcolor{PinYinColor}{Johnston}}讥笑说中国人的辫子是猪尾巴,我才把它剪掉了。\\

从民国二年起,民国的内务部就几次给内务府来函,请紫禁城协助劝说旗人剪掉辫子,并且希望紫禁城里也剪掉它,语气非常和婉,根本没提到我的头上以及大臣们的头上。内务府用了不少理由去搪塞内务部,甚至辫子可做识别进出宫门的标志,也成了一条理由。这件事拖了好几年,紫禁城内依旧是辫子世界。现在,经\ruby{庄士敦}{\textcolor{PinYinColor}{Johnston}}一宣传,我首先剪了辫子。我这一剪,几天功夫千把条辫子全不见了,只有三位中国师傅和几个内务府大臣还保留着。\\

因为我剪了辫子,太妃们痛哭了几场,师傅们有好多天面色阴沉。后来\ruby{溥杰}{\textcolor{PinYinColor}{\Man ᡦᡠ ᡤᡳᠶᡝ}}和\ruby{毓}{\textcolor{PinYinColor}{Yū}}\ruby{崇}{\textcolor{PinYinColor}{Cong}}也借口“奉旨”,在家里剪了辫子。那天陈师傅面对他的几个光头弟子,怔了好大一阵,最后对\ruby{毓}{\textcolor{PinYinColor}{Yū}}\ruby{崇}{\textcolor{PinYinColor}{Cong}}冷笑一声,说道:“把你的辫子卖给外国女人,你还可以得不少银子呢!”\\

顶不喜欢\ruby{庄士敦}{\textcolor{PinYinColor}{Johnston}}的,是内务府的人们。那时宫内开支仍然十分庞大,而优待条件规定的经费,年年拖欠。内务府为了筹办经费,每年都要拿出古玩字画金银瓷器去变卖和抵押。我逐渐地从\ruby{庄士敦}{\textcolor{PinYinColor}{Johnston}}口中,知道了里面有鬼。有一次内务府要卖掉一座有一人高的金塔,我想起了\ruby{庄士敦}{\textcolor{PinYinColor}{Johnston}}的话,内务府拿出去的金银制品,如果当做艺术品来卖都是有很高价值的,可是每次都是按重量卖,吃了很大的亏。据\ruby{庄士敦}{\textcolor{PinYinColor}{Johnston}}说,除非是傻子才这样干。我把内务府的人叫来,问这个金塔是怎么卖法。果然他们说是按重量卖的,我立刻大发脾气:\\

“这除非是傻子才干的事!你们就没有一个聪明人吗?”\\

内务府的人认为这是\ruby{庄士敦}{\textcolor{PinYinColor}{Johnston}}拆他们的台,他们便想出一个办法,把金塔抬到\ruby{庄士敦}{\textcolor{PinYinColor}{Johnston}}的家里,说是皇上请他代售。\ruby{庄士敦}{\textcolor{PinYinColor}{Johnston}}立刻看穿了这个把戏,大怒道:“假如你们不拿走,我马上奏明皇上!”结果是内务府的人乖乖地把金塔抬走了。他们拿\ruby{庄士敦}{\textcolor{PinYinColor}{Johnston}}没有办法,因为他既是清室的保镖,又得到了我的充分信任。\\

在\xpinyin*{毓}庆宫的最后一年,\ruby{庄士敦}{\textcolor{PinYinColor}{Johnston}}已是我的灵魂的重要部分。我们谈论课外问题,越来越多地占用着上课时间,谈论的范围也越来越广泛。他给我讲过英国王室的生活,各国的政体国情,大战后的列强实力,世界各地风光,“日不落的大英帝国”土地上的风物,中国的内战局势,中国的“白话文运动”(他这样称呼五四新文化运动)和西方文明的关系,他还谈到了复辟的可能性和不可靠的军阀态度。……\\

有一次他说:“从每种报纸上都可以看得出来,中国人民思念大清,每个人都厌倦了共和。我想暂且不必关心那些军人们的态度,皇帝陛下也不必费那么多时间从报纸上去寻找他们的态度,也暂且不必说,他们拥护复辟和拯救共和的最后目的有什么区别,总而言之,陈太傅的话是对的,皇帝陛下圣德日新是最要紧的。但是圣德日新,不能总是在紫禁城里。在欧洲,特别是在英王陛下的土地上,在英王太子读书的牛津大学里,皇帝陛下可以得到许多必要的知识,展开宽阔的眼界……”\\

在我动了留学英国的念头之前,他已给我打开了不小的“眼界”。经过他的介绍,紫禁城里出现过英国海军司令、香港英国总督,每个人都对我彬彬有礼地表示了对我的尊敬,称我为皇帝陛下。\\

我对欧化生活的醉心,我对\ruby{庄士敦}{\textcolor{PinYinColor}{Johnston}}亦步亦趋的模仿,并非完全使这位外国师傅满意。比如穿衣服,他就另有见解,或者说,他另有对我的兴趣。在我结婚那天,我在招待外国宾客的酒会上露过了面。祝了酒,回到养心殿后,脱下我的龙袍,换上了便装长袍,内穿西服裤,头戴鸭舌帽。这时,\ruby{庄士敦}{\textcolor{PinYinColor}{Johnston}}带着他的朋友们来了。一位外国老太太眼尖,她首先看见了我站在廊子底下,就问\ruby{庄士敦}{\textcolor{PinYinColor}{Johnston}}:\\

“那个少年是谁?”\\

\ruby{庄士敦}{\textcolor{PinYinColor}{Johnston}}看见了我,打量了一下我这身装束,立刻脸上涨得通红,那个模样简直把我吓一跳,而那些外国人脸上做出的那种失望的表情,又使我感到莫名其妙。外国人走了之后,\ruby{庄士敦}{\textcolor{PinYinColor}{Johnston}}的气还没有消,简直是气急败坏地对我说:\\

“这叫什么样子呵?皇帝陛下!中国皇帝戴了一顶猎帽!我的上帝!
