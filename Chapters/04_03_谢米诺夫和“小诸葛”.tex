\fancyhead[LO]{{\scriptsize 1924-1930: 天津的“行在” · 谢米诺夫和“小诸葛”}} %奇數頁眉的左邊
\fancyhead[RO]{} %奇數頁眉的右邊
\fancyhead[LE]{} %偶數頁眉的左邊
\fancyhead[RE]{{\scriptsize 1924-1930: 天津的“行在” · 谢米诺夫和“小诸葛”}} %偶數頁眉的右邊
\chapter*{谢米诺夫和“小诸葛”}
\addcontentsline{toc}{chapter}{\hspace{1cm}谢米诺夫和“小诸葛”}
\thispagestyle{empty}
我在拉拢、收买军人方面,花了多少钱,送了多少珠宝玉器,都记不起来了,只记得其中比较大的数目,是白俄\ruby{谢米诺夫}{Семёнов}拿去的。\\

\ruby{谢米诺夫}{Семёнов}是沙俄的一个将军,被苏联红军在远东击溃以后,率残部逃到中国满蒙边境一带,打家劫舍,奸淫烧杀,无恶不作。这批土匪队伍一度曾想侵入蒙古人民共和国,被击溃后,想在中蒙边境建立根据地,又遭到中国当地军队的扫荡。到一九二七年,实际上成了人数不多的股匪。这期间,\ruby{谢米诺夫}{Семёнов}本人往来于京、津、沪、旅顺以及香港、日本等地,向中国军阀和外国政客活动,寻找主顾,终于因为货色不行,变成了纯粹的招摇撞骗。第二次世界大战之后,\ruby{谢米诺夫}{Семёнов}被苏联军队捉了去,我在苏联被拘留时期曾听到过关于他被处绞刑的消息。我在天津的七年间,和这个双手沾满了中苏蒙三国人民鲜血的刽子手一直没有断过往来。我在他身上花了大量的钱,对他寄托了无限的希望。\\

\ruby{谢米诺夫}{Семёнов}起先由\xpinyin*{升允}和\xpinyin*{罗振玉}向我推荐过,我由于\xpinyin*{陈宝琛}的反对,没有见他。后来,\xpinyin*{郑孝胥}经\xpinyin*{罗振玉}的介绍,和谢会了面,认为谢是大可使用的“客卿”人才,给他“用客卿”的计划找到了第一个目标。他向我吹嘘了一通,主张不妨先把谢给\xpinyin*{张宗昌}撮合一下。那时正是我对\xpinyin*{张宗昌}抱着希望的时候,因此同意了\xpinyin*{郑孝胥}的办法。就这样,在\xpinyin*{郑孝胥}的直接活动下,\xpinyin*{张宗昌}接受了\ruby{谢米诺夫}{Семёнов}提供的外国炮灰,扩大了白俄军队。后来张、谢之间还订了一项《中俄讨赤军事协定》。\\

经过\xpinyin*{郑孝胥}的怂恿,一九二五年的十月,我在张园和\ruby{谢米诺夫}{Семёнов}会了面,由他带来的蒙古人\xpinyin*{多布端}(汉名\xpinyin*{包文渊})当翻译。我当时很满意这次谈话,相信了他的“犯难举事、反赤复国”的事业必能实现,立时给了五万元,以助其行。后来\xpinyin*{郑孝胥}、\ruby{谢米诺夫}{Семёнов}、\xpinyin*{毕瀚章}、\xpinyin*{刘凤池}等人在一起照了相,结成盟兄弟,表示一致矢忠清室。\\

那时正是继十四国进军苏联失败,世界上又一次出现大规模反苏反共高潮之时。我记得\ruby{谢米诺夫}{Семёнов}和\xpinyin*{郑孝胥}对我谈过,英美日各国决定以\ruby{谢米诺夫}{Семёнов}作为反苏的急先锋,要用军火、财力支持\ruby{谢米诺夫}{Семёнов},“俄国皇室”对\ruby{谢米诺夫}{Семёнов}正抱着很大希望。皇室代表曾与\xpinyin*{郑孝胥}有过来往,但详情我已不记得。我记得的是,\ruby{谢米诺夫}{Семёнов}和\xpinyin*{多布端}有个计划与我有莫大关系,是要使用他们在满蒙的党羽和军队,夺取满蒙地区建立起“反赤”根据地,由我在那里就位统治。为了供应\ruby{谢米诺夫}{Семёнов}活动费,我专为他立了一个银行存折,由\xpinyin*{郑孝胥}经手,随时给他支用。存款数字大约第一次是一万元。\ruby{谢米诺夫}{Семёнов}曾经表示,他本来并不需要我供给他活动费,因为他将要得到白俄侨民捐助的一亿八千万(后来又说是三亿)卢布,以后还会有美英日各国的财政支援;但是,这些钱一时还拿不到手,故此先用一点我的钱。后来他屡次因为“钱没到手”,总是找\xpinyin*{郑孝胥}支钱,而每次用钱都有一套动人的用途。记得有一次他说,日本驻津司令官高田丰树给他联络好了\xpinyin*{张作霖},他急待去奉天商讨大计,一时没有川资;又一次说,苏联的驻沪领事奉上级命令找了他,为了取得妥协,表示愿把远东某个地区给他成立自治区,他因此需要一笔路费,以便动身到东京研究这件事。\ruby{谢米诺夫}{Семёнов}究竟拿去了多少钱,我已经无法计算,只记得直到“九·一八”事变前两三个月,还要去了八百元。\\

在\ruby{谢米诺夫}{Семёнов}和我的来往期间,出现了不少的中间联络人物。其中有个叫\xpinyin*{王式}的,据这个人自称,不但\ruby{谢米诺夫}{Семёнов}对他十分信赖,而且日本要人和中国军阀都与他有密切关系。我从他嘴里最常听到的是这几句话:“这是最紧要的关头”,“这是最后的机会”,“此真千载一时之机,万不可失”,“机不可失,时不再来”等等,总是把我说得心眼里发痒。下面是他写的两个奏折:\\

\begin{quote}
	臣\xpinyin*{王式}跪\\

奏为外交军事,具有端倪,旋乾转坤在此一举,恭折仰祈圣鉴事。窃臣于五月十二日面奉谕旨,致书俄臣\ruby{谢米诺夫}{Семёнов},询其近状。臣行抵上海即驰书东京,并告以遣使赴德及联络军队二事,旋得其复函,言即将来华,不必东渡。既又接其电报,约会于大连。臣得电驰往与之晤见。据称:自昔年面奉温诏并赏厚\xpinyin*{帑},即感激天恩,誓图报称。后在沪上与臣相见,彼此以至诚相感,而订互助之口约,始终不渝。东旋以后,谋与彼邦士大夫游,渐复与被执政贵族日益亲近,屡以言情之,迄不得其要领。至今年春末,始获得苏俄扰乱满蒙及朝鲜日本之确据,出以示彼,日本方有所觉悟,毅然决然为其招募朝鲜子弟八千人,一切\xpinyin*{饷}糈器械,悉已完备,更欲为其招募俄国白党万余人,现散处于满蒙一带者,其\xpinyin*{饷}糈器械等等亦已筹备。\\

英人闻此更首先与苏俄绝交,愿以香港汇丰银行所存八千万元,\xpinyin*{俟}调查实在即予提取,故特电英国政府派遣参谋部某官至奉天,候其同往察看。法意二国亦有同情均愿加入;美国则愿先助美金五百万元,后再接济,共同在满蒙组织万国反赤义勇团,推其为盟主,共灭赤俄。今闻臣\xpinyin*{张宗昌}已归顺朝廷,曾造臣\xpinyin*{金卓}至大连,订期面商,加入团中,两月之间成军可必,成军之后即取东三省,迎銮登极,或\xpinyin*{俟}赤俄削平,再登大宝。所拟如此,不敢擅专,嘱臣请旨遵行。臣又同日臣\xpinyin*{田野丰}云,彼国政府虑赤祸蔓延将遍中国,中国共和以来乱益滋甚,知中国必不能无君,\xpinyin*{张学良}勾结南京伪政府,必不能保三省治安,必不能为中国之主,故朝野一致力助\ruby{谢米诺夫}{Семёнов},使\ruby{谢米诺夫}{Семёнов}力助皇上,光复旧物,\xpinyin*{戡}定大乱,共享承平。臣闻其言,十七年积愤为之顿释……臣道出大连,有\xpinyin*{沈向荣}者现充\xpinyin*{张宗昌}部下三十军军长,来见臣于逆旅之中,谓已纠集南北军长十人,有众十万,枪炮俱全,布列七省,愿为皇上效力,待臣返大连共同讨论,听臣指挥。此真千载一时之机,万不可失。伏愿皇上效法太祖皇帝,罗举七大恨,告庙誓众,宣布中外,万众一心,扫荡赤化。皇上纯孝格天,未始非天心厌乱,特造此机,使皇上还践帝官,复亿万年有道之基也。不然此机一失,人心懈矣。……\\

倘蒙皇上召见臣,更有\ruby{谢米诺夫}{Семёнов}、\xpinyin*{周善培}诸臣密陈之言,并臣与\xpinyin*{郑孝胥}、\xpinyin*{罗振玉}、\xpinyin*{荣源}诸臣所商筹款之法,谨当缕陈,请旨定夺,谨奏。\\

奏为兴复之计,在此一举,坐失时机,恐难再得,恭折仰祈圣鉴事。\\

窃臣于本月初一日谨将俄臣\ruby{谢米诺夫}{Семёнов}、日臣\xpinyin*{田野丰}在大连所拟办法及臣\xpinyin*{沈向荣}在彼\xpinyin*{俟}臣进行诸事,已恭折具呈御览。惟\ruby{谢米诺夫}{Семёнов}因英人在奉天久待,无可托辞,故需款至急,皇上行在\xpinyin*{帑}藏难支,臣断不敢读请,连日商诸臣\xpinyin*{罗振玉}愿将其在津房产抵押,约可得洋四万元以充经费,不足之数臣拟\xpinyin*{俟}皇上召见,面陈一切未尽之言,并有至密之事请旨定夺后,即赴大连上海再行设法……不然\xpinyin*{田野丰}已有微词,倘日人稍变初衷,\ruby{谢米诺夫}{Семёнов}即萌退志,各国不能越俎,\xpinyin*{张宗昌}即不能支持,纵使\ruby{谢米诺夫}{Семёнов}他日再起,我亦不能再责其践盟,九仞之山将全功尽弃。……更有日人要求之事,\ruby{谢米诺夫}{Семёнов}预定之谋,内部小有参商之处,均当面请乾断,惟祈训示抵遵,谨奏。\\

\begin{flushright}
	\xpinyin*{宣统}二十年八月初九日\\
\end{flushright}
\end{quote}

\xpinyin*{王式}写这几个奏折的日子,正是\xpinyin*{郑孝胥}出门,不在张园的时候。由于\xpinyin*{陈宝琛}、\xpinyin*{胡嗣瑗}这一派人的阻拦,他进不了张园的门,并且遇到了最激烈的攻击。\\

攻击\xpinyin*{王式}最激烈的是\xpinyin*{胡嗣瑗}。\xpinyin*{胡嗣瑗}在清末是个翰林,\xpinyin*{张勋}复辟时与\xpinyin*{万绳栻}同任内阁阁丞,在我到天津之后到了张园,被人起了个外号叫“胡大军机”,因为凡是有人要见我或递什么折子给我,必先经他过滤一下,这是由于我相信他为人“老实”而给他的职务,名义是管理“驻津办事处”。他最反对我和郑、罗等人接触。他看见了\xpinyin*{王式}的折子,就给我上奏折,逐条分析\xpinyin*{王式}和\xpinyin*{谢介石}等的言行前后矛盾之处,指出这纯粹是一场骗局。\xpinyin*{陈宝琛}向我摇头叹气,不满意\xpinyin*{郑孝胥}和这些人的来往,说:“\xpinyin*{苏龛}(\xpinyin*{郑宇}),\xpinyin*{苏龛},他真是疏忽不堪!”我被他们说动了心,决定不理这个\xpinyin*{王式}和\ruby{谢米诺夫}{Семёнов}的任何代表了,可是\xpinyin*{郑孝胥}一回到天津,经他三说两说,我又信了他的话,又拿出了钱供客卿们花用。记得后来\xpinyin*{郑孝胥}还推荐过一个叫\ruby{阿克第}{Acton}的奥国人和一个叫\ruby{罗斯}{Rose}的英国人。\ruby{阿克第}{Acton}是奥国从前的贵族,在天津奥国租界工部局任过职,据他自称在欧洲很有地位,可以为我在欧洲展开活动,取得复辟的声援。因此我派他做我的顾问,叫他到欧洲去活动,并且一次支给了这位客卿半年俸金一千八百元。\ruby{罗斯}{Rose}是个记者,说要复辟必得有报,要我拿两万元给他办报。我给了他三千元,后来报是出来了,叫做《诚报》,可是没几天就关了门。\\

事实就是如此,尽管有个“胡大军机”拦关,还是有不少人只要是拿着“联络军人、拥护复辟”这张“门票”,便可走进张园。特别是从一九二六年起,一批批的光杆司令和失意政客涌进了租界,我的门客更是有增无减。\\

这些人物里最值得一说的是“小诸葛”\xpinyin*{刘凤池}。我和刘的相识,是由于\xpinyin*{张勋}手下的奉系老军阀\xpinyin*{许兰洲}的介绍。\xpinyin*{刘是许}的旧部下,在许的嘴里,刘是个“现代的\xpinyin*{诸葛亮},得此一人,胜于\xpinyin*{卧龙凤雏},复辟大业,已有九成把握”。\xpinyin*{刘凤池}那年大约四十岁左右,他见了我,在吹嘘了自己的通天手眼之后,立时建议我拿出些古玩字画和金表给他,去联络台上人物。“那些福寿字、春条,对这类人是不行的”,这句话我还是从他嘴里第一个听到,虽然有点不舒服,但又赏识这个人直率。我认为他敢于讲别人不敢讲的,可见他的话一定可靠。于是我慷慨解囊,叫他一批一批地拿去那些最值钱的东西。后来,他竟指名要这要那,例如有一次他说要去活动\xpinyin*{张作霖}的部下\xpinyin*{邹作华},给我来信说:\\

\begin{quote}
	姓邹者才甚大,\xpinyin*{张作霖}胜,彼功甚大,张待之甚厚,小物品不能动其心也,应进其珍珠、好宝石或钻石,按万元左右贵重物予之,当有几十倍之大利在也。\\
\end{quote}

为了拉拢奉系的\xpinyin*{荣臻}、\xpinyin*{马占山}、\xpinyin*{张作相},他指明要各送十颗朝珠;为了拉拢一个姓穆的,他指明要珠顶冠上的那颗珠子。这种信,三五天必有一封,内中不少这类词句:“要真才就得多花钱,求俭遭人轻,做大事不拘小节”,“应送端砚、细瓷,外界不易得之物”。如果他报告的活动情况都如实的话,差不多奉系的旅长以上(甚至包括团长,如\xpinyin*{富双英}当团长时),以及拥有四十万众的红枪会首领、占山为王的草莽英雄等等,都拿到了我的珍珠。古瓷、钻石,都在我“不拘小节”之下大受感动,只待我一声令下,就可以举事了。但是他拿了无数的东西,人马却总不见动静。后来,我在\xpinyin*{陈宝琛}劝阻之下,发生了动摇,钱给的就不太积极,于是小诸葛无论面谈和来信中多了一种词句:“已耗费若干,旅费及招待,尚不在数”,“已倾家荡产,实难再代垫补”,“现在情况万分紧急,成败在此一举,无论如何先接济二万元”,“需款万分紧急,望无论如何将此款赐下,以免误此良机”。我后来觉出了事情不对,不肯再给钱,不久便接到了他这样的信:“皇上若每日不知研究,亦不十分注意时局,敢望其必成乎?若不猛进,亦不期望必成,又何必设此想乎?……试将中国史记打开,凡创业中兴之主,有如此之冷淡者乎?……”\\

我已忘记这个“小诸葛”是如何离开我的了,只记得他后来向我哭穷,只要十块钱救济。后来听说他在东北各地招摇,给奉系万福麟枪毙了。\\

像\xpinyin*{刘凤池}这类人物,我还可以举出一串名字,比如\xpinyin*{华瀚章}之类的人们,都用过差不多的手法,吊起了我的重登大宝的胃口,骗走了不少现款、古玩、珍珠、宝石等等。这些人最后和我的分手,是各式各样的,有的不告而别,有的被“胡大军机”或其他人硬给拦住,也有的是我自己不叫进门。其中有个绰号“费胖子”的安福系小政客\xpinyin*{费毓楷},他曾向我报告,他和炸死\xpinyin*{张作霖}的日本\xpinyin*{河本}大佐取上了联系,已组织好\xpinyin*{张学良}的侍卫,即将举行暴动,在东北实行武装复辟,迎我“正位”。这个动人的然而难于置信的大话叫\xpinyin*{陈宝琛}知道了,自然又加劝阻,连我岳父\xpinyin*{荣源}也反对我再和他来往。费胖子最后和张园分手时,比别人多了一场戏。他遭到拒绝进园,立刻大怒,气势汹汹地对拦门的\xpinyin*{荣源}嚷:“我出这么大的力,竟不理我了,好,我要到国民政府,去控告你们皇上颠覆民国的罪状!”\xpinyin*{荣源}和三教九流颇有来往,听了毫不在乎,反而笑道:“我劝你算了吧,你写的那些东西都还存在皇上的手里呢!”费胖子听了这话,只好悻悻而退。\\

这些人物在我身边真正的绝迹,已经是接近“九·一八”事变的时候,也就是在北方军阀全换上了青天白日旗之后,再过了一段时间。这时我对他们已经真正放弃幻想,同时由于其他后面谈到的原因,我已把希望放在别处去了。
