\fancyhead[LO]{{\scriptsize 1924-1930: 天津的“行在” · 谢米诺夫和“小诸葛”}} %奇數頁眉的左邊
\fancyhead[RO]{} %奇數頁眉的右邊
\fancyhead[LE]{} %偶數頁眉的左邊
\fancyhead[RE]{{\scriptsize 1924-1930: 天津的“行在” · 谢米诺夫和“小诸葛”}} %偶數頁眉的右邊
\chapter*{谢米诺夫和“小诸葛”}
\addcontentsline{toc}{chapter}{\hspace{1cm} 谢米诺夫和“小诸葛”}
\thispagestyle{empty}
我在拉拢、收买军人方面,花了多少钱,送了多少珠宝玉器,都记不起来了,只记得其中比较大的数目,是白俄谢米诺夫拿去的。\\

谢米诺夫是沙俄的一个将军,被苏联红军在远东击溃以后,率残部逃到中国满蒙边境一带,打家劫舍,奸淫烧杀,无恶不作。这批土匪队伍一度曾想侵入蒙古人民共和国,被击溃后,想在中蒙边境建立根据地,又遭到中国当地军队的扫荡。到一九二七年,实际上成了人数不多的股匪。这期间,谢米诺夫本人往来于京、津、沪、旅顺以及香港、日本等地,向中国军阀和外国政客活动,寻找主顾,终于因为货色不行,变成了纯粹的招摇撞骗。第二次世界大战之后,谢米诺夫被苏联军队捉了去,我在苏联被拘留时期曾听到过关于他被处绞刑的消息。我在天津的七年间,和这个双手沾满了中苏蒙三国人民鲜血的刽子手一直没有断过往来。我在他身上花了大量的钱,对他寄托了无限的希望。\\

谢米诺夫起先由升允和罗振玉向我推荐过,我由于陈宝琛的反对,没有见他。后来,郑孝胥经罗振玉的介绍,和谢会了面,认为谢是大可使用的“客卿”人才,给他“用客卿”的计划找到了第一个目标。他向我吹嘘了一通,主张不妨先把谢给张宗昌撮合一下。那时正是我对张宗昌抱着希望的时候,因此同意了郑孝胥的办法。就这样,在郑孝胥的直接活动下,张宗昌接受了谢米诺夫提供的外国炮灰,扩大了白俄军队。后来张、谢之间还订了一项《中俄讨赤军事协定》。\\

经过郑孝胥的怂恿,一九二五年的十月,我在张园和谢米诺夫会了面,由他带来的蒙古人多布端(汉名包文渊)当翻译。我当时很满意这次谈话,相信了他的“犯难举事、反赤复国”的事业必能实现,立时给了五万元,以助其行。后来郑孝胥、谢米诺夫、毕瀚章、刘凤池等人在一起照了相,结成盟兄弟,表示一致矢忠清室。\\

那时正是继十四国进军苏联失败,世界上又一次出现大规模反苏反共高潮之时。我记得谢米诺夫和郑孝胥对我谈过,英美日各国决定以谢米诺夫作为反苏的急先锋,要用军火、财力支持谢米诺夫,“俄国皇室”对谢米诺夫正抱着很大希望。皇室代表曾与郑孝胥有过来往,但详情我已不记得。我记得的是,谢米诺夫和多布端有个计划与我有莫大关系,是要使用他们在满蒙的党羽和军队,夺取满蒙地区建立起“反赤”根据地,由我在那里就位统治。为了供应谢米诺夫活动费,我专为他立了一个银行存折,由郑孝胥经手,随时给他支用。存款数字大约第一次是一万元。谢米诺夫曾经表示,他本来并不需要我供给他活动费,因为他将要得到白俄侨民捐助的一亿八千万(后来又说是三亿)卢布,以后还会有美英日各国的财政支援;但是,这些钱一时还拿不到手,故此先用一点我的钱。后来他屡次因为“钱没到手”,总是找郑孝胥支钱,而每次用钱都有一套动人的用途。记得有一次他说,日本驻津司令官高田丰树给他联络好了张作霖,他急待去奉天商讨大计,一时没有川资;又一次说,苏联的驻沪领事奉上级命令找了他,为了取得妥协,表示愿把远东某个地区给他成立自治区,他因此需要一笔路费,以便动身到东京研究这件事。谢米诺夫究竟拿去了多少钱,我已经无法计算,只记得直到“九·一八”事变前两三个月,还要去了八百元。\\

在谢米诺夫和我的来往期间,出现了不少的中间联络人物。其中有个叫王式的,据这个人自称,不但谢米诺夫对他十分信赖,而且日本要人和中国军阀都与他有密切关系。我从他嘴里最常听到的是这几句话:“这是最紧要的关头”,“这是最后的机会”,“此真千载一时之机,万不可失”,“机不可失,时不再来”等等,总是把我说得心眼里发痒。下面是他写的两个奏折:\\

\begin{quote}
	臣王式跪\\

奏为外交军事,具有端倪,旋乾转坤在此一举,恭折仰祈圣鉴事。窃臣于五月十二日面奉谕旨,致书俄臣谢米诺夫,询其近状。臣行抵上海即驰书东京,并告以遣使赴德及联络军队二事,旋得其复函,言即将来华,不必东渡。既又接其电报,约会于大连。臣得电驰往与之晤见。据称:自昔年面奉温诏并赏厚帑,即感激天恩,誓图报称。后在沪上与臣相见,彼此以至诚相感,而订互助之口约,始终不渝。东旋以后,谋与彼邦士大夫游,渐复与被执政贵族日益亲近,屡以言情之,迄不得其要领。至今年春末,始获得苏俄扰乱满蒙及朝鲜日本之确据,出以示彼,日本方有所觉悟,毅然决然为其招募朝鲜子弟八千人,一切饷糈器械,悉已完备,更欲为其招募俄国白党万余人,现散处于满蒙一带者,其饷糈器械等等亦已筹备。\\

英人闻此更首先与苏俄绝交,愿以香港汇丰银行所存八千万元,俟调查实在即予提取,故特电英国政府派遣参谋部某官至奉天,候其同往察看。法意二国亦有同情均愿加入;美国则愿先助美金五百万元,后再接济,共同在满蒙组织万国反赤义勇团,推其为盟主,共灭赤俄。今闻臣张宗昌已归顺朝廷,曾造臣金卓至大连,订期面商,加入团中,两月之间成军可必,成军之后即取东三省,迎銮登极,或俟赤俄削平,再登大宝。所拟如此,不敢擅专,嘱臣请旨遵行。臣又同日臣田野丰云,彼国政府虑赤祸蔓延将遍中国,中国共和以来乱益滋甚,知中国必不能无君,张学良勾结南京伪政府,必不能保三省治安,必不能为中国之主,故朝野一致力助谢米诺夫,使谢米诺夫力助皇上,光复旧物,戡定大乱,共享承平。臣闻其言,十七年积愤为之顿释……臣道出大连,有沈向荣者现充张宗昌部下三十军军长,来见臣于逆旅之中,谓已纠集南北军长十人,有众十万,枪炮俱全,布列七省,愿为皇上效力,待臣返大连共同讨论,听臣指挥。此真千载一时之机,万不可失。伏愿皇上效法太祖皇帝,罗举七大恨,告庙誓众,宣布中外,万众一心,扫荡赤化。皇上纯孝格天,未始非天心厌乱,特造此机,使皇上囗践帝官,复亿万年有道之基也。不然此机一失,人心懈矣。……\\

倘蒙皇上召见臣,更有谢米诺夫、周善培诸臣密陈之言,并臣与郑孝胥、罗振玉、荣源诸臣所商筹款之法,谨当缕陈,请旨定夺,谨奏。\\

奏为兴复之计,在此一举,坐失时机,恐难再得,恭折仰祈圣鉴事。\\

窃臣于本月初一日谨将俄臣谢米诺夫、日臣田野丰在大连所拟办法及臣沈向荣在彼俟臣进行诸事,已恭折具呈御览。惟谢米诺夫因英人在奉天久待,无可托辞,故需款至急,皇上行在帑藏难支,臣断不敢读请,连日商诸臣罗振玉愿将其在津房产抵押,约可得洋四万元以充经费,不足之数臣拟俟皇上召见,面陈一切未尽之言,并有至密之事请旨定夺后,即赴大连上海再行设法……不然田野丰已有微词,倘日人稍变初衷,谢米诺夫即萌退志,各国不能越俎,张宗昌即不能支持,纵使谢米诺夫他日再起,我亦不能再责其践盟,九仞之山将全功尽弃。……更有日人要求之事,谢米诺夫预定之谋,内部小有参商之处,均当面请乾断,惟祈训示抵遵,谨奏。\\

\begin{flushright}
	宣统二十年八月初九日\\
\end{flushright}
\end{quote}

王式写这几个奏折的日子,正是郑孝胥出门,不在张园的时候。由于陈宝琛、胡嗣瑗这一派人的阻拦,他进不了张园的门,并且遇到了最激烈的攻击。\\

攻击王式最激烈的是胡嗣瑗。胡嗣瑗在清末是个翰林,张勋复辟时与万绳栻同任内阁阁丞,在我到天津之后到了张园,被人起了个外号叫“胡大军机”,因为凡是有人要见我或递什么折子给我,必先经他过滤一下,这是由于我相信他为人“老实”而给他的职务,名义是管理“驻津办事处”。他最反对我和郑、罗等人接触。他看见了王式的折子,就给我上奏折,逐条分析王式和谢介石等的言行前后矛盾之处,指出这纯粹是一场骗局。陈宝琛向我摇头叹气,不满意郑孝胥和这些人的来往,说:“苏龛(郑字),苏龛,他真是疏忽不堪!”我被他们说动了心,决定不理这个王式和谢米诺夫的任何代表了,可是郑孝胥一回到天津,经他三说两说,我又信了他的话,又拿出了钱供客卿们花用。记得后来郑孝胥还推荐过一个叫阿克第的奥国人和一个叫罗斯的英国人。阿克第是奥国从前的贵族,在天津奥国租界工部局任过职,据他自称在欧洲很有地位,可以为我在欧洲展开活动,取得复辟的声援。因此我派他做我的顾问,叫他到欧洲去活动,并且一次支给了这位客卿半年俸金一千八百元。罗斯是个记者,说要复辟必得有报,要我拿两万元给他办报。我给了他三千元,后来报是出来了,叫做《诚报》,可是没几天就关了门。\\

事实就是如此,尽管有个“胡大军机”拦关,还是有不少人只要是拿着“联络军人、拥护复辟”这张“门票”,便可走进张园。特别是从一九二六年起,一批批的光杆司令和失意政客涌进了租界,我的门客更是有增无减。\\

这些人物里最值得一说的是“小诸葛”刘凤池。我和刘的相识,是由于张勋手下的奉系老军阀许兰洲的介绍。刘是许的旧部下,在许的嘴里,刘是个“现代的诸葛亮,得此一人,胜于卧龙凤雏,复辟大业,已有九成把握”。刘凤池那年大约四十岁左右,他见了我,在吹嘘了自己的通天手眼之后,立时建议我拿出些古玩字画和金表给他,去联络台上人物。“那些福寿字、春条,对这类人是不行的”,这句话我还是从他嘴里第一个听到,虽然有点不舒服,但又赏识这个人直率。我认为他敢于讲别人不敢讲的,可见他的话一定可靠。于是我慷慨解囊,叫他一批一批地拿去那些最值钱的东西。后来,他竟指名要这要那,例如有一次他说要去活动张作霖的部下邹作华,给我来信说:\\

\begin{quote}
	姓邹者才甚大,张作霖胜,彼功甚大,张待之甚厚,小物品不能动其心也,应进其珍珠、好宝石或钻石,按万元左右贵重物予之,当有几十倍之大利在也。\\
\end{quote}

为了拉拢奉系的荣臻、马占山、张作相,他指明要各送十颗朝珠;为了拉拢一个姓穆的,他指明要珠顶冠上的那颗珠子。这种信,三五天必有一封,内中不少这类词句:“要真才就得多花钱,求俭遭人轻,做大事不拘小节”,“应送端砚、细瓷,外界不易得之物”。如果他报告的活动情况都如实的话,差不多奉系的旅长以上(甚至包括团长,如富双英当团长时),以及拥有四十万众的红枪会首领、占山为王的草莽英雄等等,都拿到了我的珍珠。古瓷、钻石,都在我“不拘小节”之下大受感动,只待我一声令下,就可以举事了。但是他拿了无数的东西,人马却总不见动静。后来,我在陈宝琛劝阻之下,发生了动摇,钱给的就不太积极,于是小诸葛无论面谈和来信中多了一种词句:“已耗费若干,旅费及招待,尚不在数”,“已倾家荡产,实难再代垫补”,“现在情况万分紧急,成败在此一举,无论如何先接济二万元”,“需款万分紧急,望无论如何将此款赐下,以免误此良机”。我后来觉出了事情不对,不肯再给钱,不久便接到了他这样的信:“皇上若每日不知研究,亦不十分注意时局,敢望其必成乎?若不猛进,亦不期望必成,又何必设此想乎?……试将中国史记打开,凡创业中兴之主,有如此之冷淡者乎?……”\\

我已忘记这个“小诸葛”是如何离开我的了,只记得他后来向我哭穷,只要十块钱救济。后来听说他在东北各地招摇,给奉系万福麟枪毙了。\\

像刘凤池这类人物,我还可以举出一串名字,比如华瀚章之类的人们,都用过差不多的手法,吊起了我的重登大宝的胃口,骗走了不少现款、古玩、珍珠、宝石等等。这些人最后和我的分手,是各式各样的,有的不告而别,有的被“胡大军机”或其他人硬给拦住,也有的是我自己不叫进门。其中有个绰号“费胖子”的安福系小政客费毓楷,他曾向我报告,他和炸死张作霖的日本河本大佐取上了联系,已组织好张学良的侍卫,即将举行暴动,在东北实行武装复辟,迎我“正位”。这个动人的然而难于置信的大话叫陈宝琛知道了,自然又加劝阻,连我岳父荣源也反对我再和他来往。费胖子最后和张园分手时,比别人多了一场戏。他遭到拒绝进园,立刻大怒,气势汹汹地对拦门的荣源嚷:“我出这么大的力,竟不理我了,好,我要到国民政府,去控告你们皇上颠覆民国的罪状!”荣源和三教九流颇有来往,听了毫不在乎,反而笑道:“我劝你算了吧,你写的那些东西都还存在皇上的手里呢!”费胖子听了这话,只好悻悻而退。\\

这些人物在我身边真正的绝迹,已经是接近“九·一八”事变的时候,也就是在北方军阀全换上了青天白日旗之后,再过了一段时间。这时我对他们已经真正放弃幻想,同时由于其他后面谈到的原因,我已把希望放在别处去了。\\