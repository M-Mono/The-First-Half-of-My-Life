\fancyhead[LO]{{\scriptsize 1917-1924: 北京的“小朝廷” · 遣散太监}} %奇數頁眉的左邊
\fancyhead[RO]{} %奇數頁眉的右邊
\fancyhead[LE]{} %偶數頁眉的左邊
\fancyhead[RE]{{\scriptsize 1917-1924: 北京的“小朝廷” · 遣散太监}} %偶數頁眉的右邊
\chapter*{遣散太监}
\addcontentsline{toc}{chapter}{\hspace{1cm}遣散太监}
\thispagestyle{empty}
紫禁城在表面上是一片平静,内里的秩序却是糟乱一团。从我懂事的时候起,就时常听说宫里发生盗案、火警,以及行凶事件。至于烟赌,更不用说。到我结婚的时候,偷盗已发展到这种程度:刚行过婚礼,由珍珠玉翠装嵌的皇后凤冠上的全部珍宝,竟整个被换成了赝品。\\

我从师傅们那里知道,清宫中的财宝早已在世界上闻名。只说古玩字画,那数量和价值就是极其可观的。明清两代几百年帝王搜刮来的宝物,除了两次被洋兵弄走的以外,大部分还存放在宫里。这些东西大部分没有数目,就是有数目的也没有人去检查,所以丢没丢,丢了多少,都没有人知道。这就给偷盗者大开了方便之门。\\

今天想起来,那简直是一场浩劫。参加打劫行径的,可以说是从上而下,人人在内。换言之,凡是一切有机会偷的人,是无一不偷,而且尽可放胆地偷。偷盗的方式也各不同,有拨门撬锁秘密地偷,有根据合法手续,明目张胆地偷。太监大都采用前一种方式,大臣和官员们则采用办理抵押、标卖或借出鉴赏,以及请求赏赐等等,即后一种方式。至于我和溥杰采用的一赏一受,则是最高级的方式。当然,那时我决不会有这样想法,我想的只是,别人都在偷盗我的财物。\\

我十六岁那年,有一天由于好奇心的驱使,叫太监打开建福官那边一座库房。库门封条很厚,至少有几十年没有开过了。我看见满屋都是堆到天花板的大箱子,箱皮上有嘉庆年的封条,里面是什么东西,谁也说不上来。我叫太监打开了一个,原来全是手卷字画和非常精巧的古玩玉器。后来弄清楚了,这是当年乾隆自己最喜爱的珍玩。乾隆去世之后,嘉庆下令把那些珍宝玩物全部封存,装满了建福官一带许多殿堂库房,我所发现的不过是其中的一库。有的库尽是彝器,有的库尽是瓷器,有的库尽是名画,意大利人郎世宁给乾隆画的许多画也在内。在养心殿后面的库房里,我还发现了许多很有趣的“百宝匣”,据说这也是乾隆的玩物。这种百宝匣用紫檀木制成,外形好像一般的书箱,打开了像一道楼梯,每层梯上分成几十个小格子,每个格子里是一样玩物,例如一个宋磁小瓶,一部名人手抄的寸半本四书,一个精刻的牙球,一个雕着古代故事的核桃,几个刻有题诗绘画的瓜子,以及一枚埃及古币等等。一个百宝匣中,举凡字画、金石。玉器、铜器、瓷器、牙雕等等,无一不备,名为百宝,实则一个小型的匣子即有几百种,大型的更不只千种。还有一种特制的紫檀木炕几,上面无一处没有消息,每个消息里盛着一件珍品,这个东西我没看见,我当时只把亲自发现的百宝匣,大约有四五十匣,都拿到养心殿去了。这时我想到了这样的问题:我究竟有多少财宝?我能看到的,我拿来了,我看不到的又有多少?那些整库整院的珍宝怎么办?被人偷去的有多少?怎样才能制止偷盗?\\

庄士敦师傅曾告诉我,他住的地安门街上,新开了许多家古玩铺。听说有的是太监开的,有的是内务府官员或者官员的亲戚开的。后来,别的师傅也觉得必须采取措施,杜绝盗患。最后,我接受了师傅们的建议,决定清点一下。这样一来,麻烦更大了。\\

首先是盗案更多了。毓庆宫的库房门锁给人砸掉了,乾清宫的后窗户给人打开了。事情越来越不像话,我刚买的大钻石也不见了。为了追查盗案,太妃曾叫敬事房都领侍组织九堂总管,会审当事的太监,甚至动了刑,但是无论是刑讯还是悬重赏,都未获得一点效果。不但如此,建福官的清点刚开始,六月二十七日的夜里便突然发生了火警,清点的和未清点的,全部烧个精光。\\

据说火警是东交民巷的意大利公使馆消防队首先发现的。救火车开到紫禁城叫门时,守门的还不知是怎么回事。这场大火经各处来的消防队扑救了一夜,结果还是把建福宫附近一带,包括静怕轩、慧曜楼、吉云楼、碧琳馆、妙莲花室、延春阁、积翠亭、广生楼、凝辉楼、香云亭等一大片地方烧成焦土。这是清宫里贮藏珍宝最多的地方,究竟在这一把火里毁掉了多少东西,至今还是一个谜。内务府后来发表的一部分胡涂账里,说烧毁了金佛二千六百六十五尊,字画一千一百五十七件,古玩四百三十五件,古书几万册。这是根据什么账写的,只有天晓得。\\

在救火的时候,中国人,外国人,紫禁城里的人,城外的人,人来人往,沸腾一片,忙成一团。除了救火还忙什么,这是可以想象的。但紫禁城对这一切都表示了感谢。有一位来救火的外国太太,不知为什么跟中国消防队员发生了争执,居然动手把对方打得鼻子出了血,手里的扇子也溅上了血。后来她托人把这扇子拿给我看,以示其义勇,我还在上面题了诗,以示感谢。这场火灾过去之后,内务府除用茶点招待了救火者,还送给警察和消防队六万元“酬劳”费。\\

要想估计一下这次的损失,不妨说一下那堆烧剩和“摸”剩下的垃圾的处理。那时我正想找一块空地修建球场,由庄士敦教我打网球,据他说这是英国贵族都会的玩艺。这片火场正好做这个用场,于是叫内务府赶快清理出来。那堆灰烬里固然是找不出什么字画、古瓷之类的东西了,但烧熔的金银铜锡还不少。内务府把北京各金店找来投标,一个金店以五十万元的价格买到了这片灰烬的处理权,把熔化的金块金片拣出了一万七千多两。金店把这些东西拣走之后,内务府把余下的灰烬装了许多麻袋,分给了内务府的人们。后来有个内务府官员告诉我,他叔父那时施舍给北京雍和宫和柏林寺每庙各两座黄金“坛城”,它的直径和高度均有一尺上下,就是用麻袋里的灰烬提制出来的。\\

起火的原因和损失真相同样的无从调查。我疑心这是偷盗犯故意放火灭迹的。过不多天,养心殿东套院无逸斋的窗户上又发生火警,幸好发现得早,一团浸过煤油的棉花刚烧着,就被发现扑灭。我的疑心立刻更加发展起来。我认为不但是有人想放火灭迹,而且还想要谋害我了。\\

事实上,偷窃和纵火灭迹都是事实,师傅们也没有避讳这一点,而对我的谋害则可能是我自己神经过敏。我的多疑的性格,这时已显露出来了。按清宫祖制,皇帝每天无论如何忙,也要看一页《圣训》(这些东西一年到头摆在皇帝寝宫里)。我这时对雍正的《殊批谕旨》特别钦佩。雍正曾说过这样的话:“可信者人,而不可信者亦人,万不可信人之必不负于己也。不如此,不可以言用人之能”。他曾在亲信大臣鄂尔泰的奏折上批过:“其不敢轻信人一句,乃用人第一妙诀。朕从来不知疑人,亦不知信人”。又说,对人“即经历几事,亦只可信其已往,犹当留意观其将来,万不可信其必不改移也”。这些话都深深印人我的脑子里。我也记得康熙的话:“为人上者,用人虽宜信,然亦不可遽信”。康,熙特别说过太监不可信,他说:“朕观古来,太监良善者少,要在人主防微杜渐,慎之于始”。祖宗们的这些训谕,被这几场火警引进了我的思索中。\\

我决定遵照雍正皇帝“察察为明”的训示行事。我想出了两条办法,一条是向身边的小太监们套问,另一条是自己去偷听太监们的谈话。后来我用第二条办法,在东西夹道太监住房窗外,发现了他们背后议论我,说我脾气越来越坏,这更引起了我的猜疑。在无逸斋发生火警这天晚上,我再到太监窗下去偷听,不料听到他们的议论更发展了一步,竟说这把火是我自己放的。我觉得他们真是居心叵测,我如果不先采取措施,后害实在无穷。\\

这时刚刚发生了一起行凶案。有个太监因为被人告发了什么过失,挨了总管的责打,于是怀恨在心,一天早晨趁告发人还没起身,拿了一把石灰和一把刀,进了屋子,先撒石灰在那人脸上,迷了他的眼,然后用刀戳那人的脸。这个行凶的人后来未被捉住,受伤的人送进了医院。我这时想起许多太监都受过我的责打,他们会不会对我行凶呢?想到这里,我简直连觉都不敢睡了。从我的卧室外间一直到抱厦,都有值更太监打地铺睡着,这里面如果有谁对我不怀好心,要和我过不去,那不是太容易下手了吗?我想挑一个可靠的人给我守夜,挑来挑去,只挑出一个皇后来。我从这天起让婉容整夜为我守卫,如果听见了什么动静,就叫醒我。同时我还预备了一根棍子,放在床边,以便应变。一连几天,婉容整夜不能睡觉,我看这究竟不是个办法。为了一劳永逸,最后我决定,把太监全都赶走不要!\\

我知道这件事必定要引起一场风波。不把父亲对付好,是行不通的。我想好了一个主意,亲自去找我的父亲。他没有办法和内务府大臣以及师傅们商量,突然遇到了这个问题,他的口才就更加不行,变得更加结巴了。他非常吃力地讲出了一些零七八碎的理由,什么祖制如此咧,这些人当差多年不致图谋不轨咧,等等,来进行劝服。并且说:“这这也得慢慢商议,皇帝先回到宫,过两天……”\\

我不管他怎么说,只用这一句话来回答:\\

“王爷不答应,我从今天起就再不回宫啦!”\\

他见我这样对付他,急得坐也不是,站也不是,又抓头,又挠腮,直在地上打转儿,桌上的一瓶汽水给他的袖子碰倒掉在地上,砰地一声炸了。瞅他这副模样,我禁不住反倒格格乐起来,并且从容不迫地打开书桌上的一本书,装作决心不想离开的样子。\\

父亲终于屈服了。最后决定,除了太妃身边离不开的一些以外,其他太监全部遣散。\\