\fancyhead[LO]{{\scriptsize 1931-1932: 到东北去 · 不静的“静园”}} %奇數頁眉的左邊
\fancyhead[RO]{} %奇數頁眉的右邊
\fancyhead[LE]{} %偶數頁眉的左邊
\fancyhead[RE]{{\scriptsize 1931-1932: 到东北去 · 不静的“静园”}} %偶數頁眉的右邊
\chapter*{不静的“静园”}
\addcontentsline{toc}{chapter}{\hspace{1cm}不静的“静园”}
\thispagestyle{empty}
一九二九年七月,我从日租界官岛街的张园,迁到协昌里的“静园”。这是租的安福系政客\xpinyin*{陆宗舆}的房子,原名“乾园”,我给它改了名字,是含有一层用意的。\\

北伐后,国民党的势力伸到了北方,和我有交情的军阀纷纷垮台,被我寄托过希望的东三省,宣布“易帜”。张园上下因此一度感到一片悲观失望。一部分遗老门客作鸟兽散了,和我厮守着的近臣们,除了\xpinyin*{郑孝胥}和\xpinyin*{罗振玉}等人之外,几乎再没有别人谈论什么复辟。像\xpinyin*{陈宝琛}这样的人,从前嘴边上挂着的“天与人归”、“卧薪尝胆”的话,也不说了。人们惟一考虑的问题,是得到了江山的新王朝,将会怎样对待我这个末代皇帝。我自己陷入了深沉的忧虑之中。但是,这种情形并没有继续多久。我们很快就看到,五色旗才摘下来,打着青天白日旗的人又彼此厮杀起来,今天甲乙联合反丙,明天乙丙又合作倒甲,情形和从前并没有什么两样。\xpinyin*{蒋介石}所达到的“统一”,越看越不像那么回事,\xpinyin*{蒋介石}脚底下的江山,越看越不像料想中的那么稳固。张园有了绝路逢生之感,不免渐渐重温旧梦,觉得“定于一”的大业,似乎仍然非我莫属。不但遗老和门客们后来恢复了这个论调,就连每周“进讲”时局的日本参谋们,也不避讳这种观点。我把新居取名“静园”的意思,并非是求清静,而是要在这里“静观变化,静待时机。”\\

静园里日日望着,月月盼着。一九三一年的夏天,真盼来了消息。\\

“九·一八”事变前的两个月,在日本东京“学习院”读书的\ruby{溥杰}{\textcolor{PinYinColor}{Pu Giye}}正待回国度假,忽然接到鹿儿岛来的一封信。鹿儿岛驻军某联队的\ruby{吉冈}{\textcolor{PinYinColor}{よしおか}}\ruby{安直}{\textcolor{PinYinColor}{やすなお}}大队长,曾经是天津日军司令部的参谋,常到张园来讲说时局,与\ruby{溥杰}{\textcolor{PinYinColor}{Pu Giye}}也认识,这时他向\ruby{溥杰}{\textcolor{PinYinColor}{Pu Giye}}发出邀请,请\ruby{溥杰}{\textcolor{PinYinColor}{Pu Giye}}到鹿儿岛做几天客,然后再回国。\ruby{溥杰}{\textcolor{PinYinColor}{Pu Giye}}应邀到了鹿儿岛,受到了\ruby{吉冈}{\textcolor{PinYinColor}{よしおか}}少佐夫妇的殷勤招待。到了告别的时候,\ruby{吉冈}{\textcolor{PinYinColor}{よしおか}}单独对\ruby{溥杰}{\textcolor{PinYinColor}{Pu Giye}}神秘而郑重地说:“你到了天津,可以告诉令兄:现在\xpinyin*{张学良}闹的很不像话,满洲在最近也许就要发生点什么事情。……请\xpinyin*{宣统}皇帝多多保重,他不是没有希望的!”七月十日\ruby{溥杰}{\textcolor{PinYinColor}{Pu Giye}}到了天津,把这个消息告诉了我。七月二十九日,日本华族\ruby{水野}{\textcolor{PinYinColor}{みずの}}\ruby{胜邦}{\textcolor{PinYinColor}{かつくに}}子爵前来访问,在\xpinyin*{郑孝胥}和\ruby{溥杰}{\textcolor{PinYinColor}{Pu Giye}}的陪侍下,我接见了他。在这次平常的礼貌的会见中,客人送了我一件不平常的礼物:一把日本扇子,上面题着一联诗句:“天莫空勾践,时非无范蠡”。\\

原来\ruby{溥杰}{\textcolor{PinYinColor}{Pu Giye}}回国之前,\ruby{水野}{\textcolor{PinYinColor}{みずのし}}子爵亲自找过他,接洽送扇子的事,因此,\ruby{溥杰}{\textcolor{PinYinColor}{Pu Giye}}明白了这两句诗的来历,并且立即写信报告了我。这是发生在日本南北朝内乱中的故事。受控制于镰仓幕府的后醍醐天皇,发动倒幕失败,被幕府捕获,流放隐歧。流放中,有个武士把这两句诗刻在樱树干上,暗示给他。后来,这位日本“勾践”果然在一群“范蠡”的辅佐下,推翻了幕府,回到了京都。以后即开始了“建武中兴”。\ruby{水野}{\textcolor{PinYinColor}{みずのし}}说的故事到此为止,至于后醍醐天皇回京都不过三年,又被新的武士首领足利尊氏赶了出来,他就没再说。当然,那时我也不会有心思研究日本历史。重要的是,这是来自日本人的暗示。那时正当“山雨欲来风满楼”之际,东北局势日益紧张,我的“重登大宝”的美梦已连做了几天晚上。这时来了这样的暗示——无论它是出于单纯的私人关怀,还是出于某方的授意——对我说来,事实上都是起着行动信号的作用。\\

\begin{quote}
	“九·一八”前后那几天的静园动态,\xpinyin*{郑孝胥}日记里留下了一些记载:\\

\xpinyin*{乙亥}初六日(九月十七日)。\xpinyin*{诣}行在。召见,商派\xpinyin*{刘骧业}、\xpinyin*{郑垂}往大连。……\\

\xpinyin*{丙子}初七日(九月十八日)。\xpinyin*{诣}行在。召见,咨询出行事宜。\\

\xpinyin*{丁丑}初八日(九月十九日)。日本《日日新闻》送来号外传单云:夜三时二十三分奉天电云:中日交战。召见\xpinyin*{刘骧业}、\xpinyin*{郑垂},命\xpinyin*{刘骧业}先赴大连。作字。遇\xpinyin*{弢庵}(\xpinyin*{陈宝琛}),谈预料战事恐复成日俄之战。午原(\xpinyin*{刘骧业})来,求作书二纸,遗满铁总裁\ruby{内田}{\textcolor{PinYinColor}{うちだ}}及日军司令\ruby{本庄}{\textcolor{PinYinColor}{ほんじょう}}。\xpinyin*{大七}(\xpinyin*{郑垂})往行日领馆。云:昨日军已占奉天,华军自退,长春亦有战事。……\\

\xpinyin*{戊寅}初九日(九月二十日)。\xpinyin*{诣}行在。进讲。报言日军据沈阳,同时据长春、营口、安东、辽阳。东三省民报送致十八号,报中毫无知觉。……\\

己卯初十日(九月二十一日)。\xpinyin*{诣}行在。进讲。\xpinyin*{蒋介石}返南京,对日本抗议,\xpinyin*{张学良}令奉军勿抵抗。……\xpinyin*{佟揖先}(\xpinyin*{济煦})来,自言欲赴奉天,谋复辟事。余曰:若得军人商人百余人倡议,脱离张氏,以三省、内蒙为独立国,而向日本上请愿书,此及时应为之事也。……\\
\end{quote}

我从一听见事变的消息时起,每分钟都在想到东北去,但我知道不经日本人的同意是办不到的。\xpinyin*{郑孝胥}对我说,沈阳情况还不明朗,不必太着忙,日本人迟早会来请皇上,最好先和各方面联络一下。因此我决定派\xpinyin*{刘骧业},去找日本人在东北的最高统治者\ruby{内田}{\textcolor{PinYinColor}{うちだ}}和\ruby{本庄}{\textcolor{PinYinColor}{ほんじょう}}。另叫我的管家头目\xpinyin*{佟济煦},去东北看看遗老们那边的情形。这时\xpinyin*{商衍瀛}也想去找那些有过来往的东北将领。这些办理“及时应为之事”的人走后,过了不久,\xpinyin*{郑孝胥}的话应验了,关东军派人找我来了。\\

九月三十日下午,日本天津驻屯军司令部通译官\ruby{吉田}{\textcolor{PinYinColor}{よしだ}}\ruby{忠太郎}{\textcolor{PinYinColor}{ちゅうたろう}}来到静园,说司令官\ruby{香椎}{\textcolor{PinYinColor}{こうへい}}\ruby{浩平}{\textcolor{PinYinColor}{こうへい}}中将请我到司令部谈一件重要的事情。他告诉我不要带随从,单独前往。我怀着喜事临门的预感,到了海光寺日本兵营,\ruby{香椎}{\textcolor{PinYinColor}{こうへい}}正立在他的住宅门外等着我。我进了他的客厅,在这里我看见了两个人恭恭敬敬地站着,一个是长袍马褂的\xpinyin*{罗振玉},另一个是穿西服的陌生人,从他鞠躬姿势上就可以看出是个日本人。\ruby{香椎}{\textcolor{PinYinColor}{こうへい}}介绍说,他是关东军参谋\ruby{板垣}{\textcolor{PinYinColor}{いたがき}}大住派来朝见我的,名叫\ruby{上角}{\textcolor{PinYinColor}{うえすみ}}\ruby{利一}{\textcolor{PinYinColor}{としかず}}。介绍了之后,\ruby{香椎}{\textcolor{PinYinColor}{こうへい}}就出去了。\\

屋子里只剩下我们三个人。\xpinyin*{罗振玉}恭恭敬敬地给我请过安,拿出一个大信封给我。这是我的远支宗室,东北保安副总司令\xpinyin*{张作相}的参谋长\ruby{熙洽}{\textcolor{PinYinColor}{hsi chia}}写来的。\xpinyin*{张作相}是兼职的吉林省主席,因为到锦州奔父丧,不在吉林,\ruby{熙洽}{\textcolor{PinYinColor}{hsi chia}}便利用职权,乘机下令开城迎接日军,因此,他的日本士官学校时代的老师多门师团长的军队,不费一枪一弹,就占领了吉林。他在信里说,他期待了二十年的机会,今天终于来到了,请我匆失时机,立即到“祖宗发祥地”主持大计,还说可以在日本人的支持下,先据有满洲,再图关内,只要我一回到沈阳,吉林即首先宣布复辟。\\

\xpinyin*{罗振玉}等我看完了信,除了重复了一遍信中的意思,又大讲了一番他自己的奔走和关东军的“仗义协助”。照他说,东北全境“光复”指日可待,三千万“子民”盼我回去,关东军愿意我去复位,特意派了\ruby{上角}{\textcolor{PinYinColor}{うえすみ}}来接我。总之是一切妥善,只等我拔起腿来,由日本军舰把我送到大连了。他说得兴高采烈,满脸红光,全身颤动,眼珠子几乎都要从眼眶子里跳出来了。他的兴奋是有来由的。他不仅有\ruby{熙洽}{\textcolor{PinYinColor}{hsi chia}}的欲望,而且有吕不韦的热衷。他现在既相信不久可以大过其蟒袍补褂三跪九叩之瘾,而且看到利润千万倍于“墨缘堂”的“奇货”。他这几年来所花费的“苦功”,后来写在他的自传《集蓼编》里了:\\

\begin{quote}
	予自\xpinyin*{辛亥}避地海东,意中日唇齿,彼邦人士必有明辅车之相依,燎原之将及者,乃历八年之久,竟无所遇,于是浩然有归志。遂以己未(1919年)返国,寓天津者又十年,目击军人私斗,连年不已,邪说横行,人纪扫地,不忍见闻。事后避地辽东又三年。衰年望治之心日迫,私意关内麻乱,无从下手,惟有东三省尚未糜烂,莫如吁恳皇上先拯救满蒙三千万民众,然后再以三省之力,\xpinyin*{戡}定关内。惟此事非得东三省有势力明大义者,不能相期有成。乃以\xpinyin*{辛未}(1931年)春赴吉林,与照君格民(洽)密商之。\\

熙君夙具匡复之志,一见相契合,勉以珍重待时。又以东三省与日本关系甚深,非得友邦谅解,不克有成。故居辽以后,颇与日本关东军司令官相往还,力陈欲谋东亚之和平,非中日协力从东三省下手不可;欲维持东三省,非请我皇上临御,不能洽民望。友邦当道闻之,颇动听。\\
\end{quote}

关于\xpinyin*{罗振玉}在一九二八年末搬到旅顺大连以后的活动,他曾来信大略向我说过,那时在\xpinyin*{郑孝胥}和\xpinyin*{陈宝琛}等人的宣传下,我对这个“言过其实,举止乖\xpinyin*{戾}”的人,并没抱太大的希望。正巧在几个月之前,他刚刚又给我留下了一个坏印象。几个月以前,他忽然兴冲冲地从大连跑来,拿着日本浪人\ruby{田野}{\textcolor{PinYinColor}{たの}}\ruby{丰}{\textcolor{PinYinColor}{ゆたか}}写的“劝进表”对我说,\ruby{田野}{\textcolor{PinYinColor}{たの}}\ruby{丰}{\textcolor{PinYinColor}{ゆたか}}在日本军部方面手眼通天,最近与一个叫\ruby{高山公通}{\textcolor{PinYinColor}{たかやま}}\ruby{公通}{\textcolor{PinYinColor}{きみみち}}的军界宿耆共同活动,得到军部的委托,拟定了一个计划,要根据所谓“赤党举事”的情报,派\ruby{谢米诺夫}{\textcolor{PinYinColor}{Семёнов}}率白俄军在日军支援下乘机夺取“奉天”,同时将联络东北当地官吏“迎驾归满,宣诏收复满蒙,复辟大清”。为了实现这个计划,希望我拿出一些经费给他。我听了这个计划,很觉蹊跷,未敢置信。过了两天,日本驻北京的武官森赳忽然来找\xpinyin*{郑孝胥},要我千万不要相信\ruby{田野}{\textcolor{PinYinColor}{たの}}\ruby{丰}{\textcolor{PinYinColor}{ゆたか}}的计划,\xpinyin*{郑孝胥}连忙告诉了我,并且把\xpinyin*{罗振玉}又攻击了一顿。这件事情才过去不久,现在\xpinyin*{罗振玉}又来和我谈迎驾的问题,我自然不能不有所警惕。\\

我瞧瞧\xpinyin*{罗振玉},又瞧瞧生疏的\ruby{上角}{\textcolor{PinYinColor}{うえすみ}}\ruby{利一}{\textcolor{PinYinColor}{としかず}},心中犹豫不定。显然,\xpinyin*{罗振玉}这次的出现,与以往任何一次不同,一则谈话的地点是在日军司令部,同来的还有关东军\ruby{板垣}{\textcolor{PinYinColor}{いたがき}}大住的代表;二则他手里拿着\xpinyin*{照治}的亲笔信;再则,前一天我从大连报纸上看到了“沈阳各界准备迎立前清皇帝”的新闻,天津报上不断登载的中国军队节节退让,英国在国际联盟袒护日本的消息。看来日军对东北的统治是可能实现的,这一切都是我所希望的。但是,我觉得这件事还是和\xpinyin*{陈宝琛}、\xpinyin*{郑孝胥}他们商量一下的好。\\

我向\xpinyin*{罗振玉}和\ruby{上角}{\textcolor{PinYinColor}{うえすみ}}说,等我回去考虑一下再答复他们。这时,不知躲在哪里的香推出场了,他向我表示,天津的治安情形不好,希望我能考虑关东军\ruby{板垣}{\textcolor{PinYinColor}{いたがき}}大住的意见,动身到东北去。他这几句话,使我在坐进汽车之后,越想越觉得事情不像是假的。我的疑惑已经完全为高兴所代替了。不料回到了静园,马上就碰见了泼冷水的。\\

头一个表示反对的是\xpinyin*{陈宝琛},追随他的是\xpinyin*{胡嗣瑗}、\xpinyin*{陈曾寿}(\xpinyin*{婉容}的师傅)。他们听了我的叙述,立即认为\xpinyin*{罗振玉}又犯了鲁莽乖\xpinyin*{戾}的老病,认为对于关东军的一个大住的代表,并不能贸然置信。他们说,东北的局势变化、国际列强的真正态度,以及“民心”的趋向等等,目前还未见分晓,至少要等\xpinyin*{刘骧业}探得真相之后,才能决定行止。听了这些泄气话,我颇不耐烦地直摇头:\\

“\ruby{熙洽}{\textcolor{PinYinColor}{hsi chia}}的信,决不会说谎。”\\

八十四岁的\xpinyin*{陈宝琛}听了我的话,样子很难过,任了一阵之后,很沉痛地说:\\

“天与人归,势属必然,光复故物,岂非小臣终身之愿?惟局势混沌不分,贸然从事,只怕去时容易回时难!”\\

我看和这几个老头子说不通,叫人马上催\xpinyin*{郑孝胥}来。\xpinyin*{郑孝胥}虽然七十一岁了,却是劲头十足的,他的“开门户”、“借外援”。“三共论”以及“三都计划”等等,已使我到了完全倾倒的程度。不久前,我按他的意思,给他最崇拜的意大利首相\ruby{墨索里尼}{\textcolor{PinYinColor}{Mussolini}}写了一块“国士无双”的横幅。他曾说:“意大利必将成为西方一霸,大清帝国必将再兴于东方,两国分霸东西,其天意乎?”为了嘉勉我未来的黑衣宰相,这年春天我特授意我的父亲,让我的二妹和\xpinyin*{郑孝胥}的长孙订了亲,给以“皇亲”的特殊荣誉。我估计他现在听到\ruby{熙洽}{\textcolor{PinYinColor}{hsi chia}}和关东军请我出关“主持大计”的消息,必定是与\xpinyin*{陈宝琛}的反应不同,该是大大高兴的。没料到,他并没表现出我所料想的那种兴奋。\\

“展转相垂,至有今日。满洲势必首先光复,日本不迎圣驾,也不能收场。”他沉吟一下说,“不过,何时启驾,等\xpinyin*{佟济煦}回来之后再定,更为妥帖。”\\

这意思,竟跟\xpinyin*{陈宝琛}一样,也以为时机未臻成熟。\\

其实,\xpinyin*{郑孝胥}脑袋里所想的,并不是什么时机问题。这可以由他不多天前的一篇日记来证明:\\

\begin{quote}
	报载美国罗斯安吉(洛杉矶)十月四日合众社电:罗斯安吉之出版人\ruby{毕德}{\textcolor{PinYinColor}{Peter}},为本社撰一文称:世界恢复之希望(按资本主义世界从一九二九年起发生了经济大恐慌,报上经常有谈论如何把资本主义世界从危机中拯救出来之类的文章)端赖中国。氏引英国著名小说家\ruby{韦尔斯}{\textcolor{PinYinColor}{Wells}}之最近建议,“需要一世界之独裁者将自世界经济萧条中救出”,氏谓此项计划,无异幻梦,不能实现。华德建议美政府,应考虑极端之独裁办法,以拯救现状。第一步,应组一国际经济财政银行团,以美国为领袖,供给资金,惟一目的,为振兴中国。氏主张美政府应速草一发展中国计划。中国工业交通之需要如能应付,将成为世界之最大市场,偿还美国之投资,当不在远。此时集中注意于中国,美国社会经济制度皆有改正,繁荣可以恢复,人类将受其福利云。\\

今年为民国二十年。……彼以双十为国庆,这二十年整矣。此试巧合,天告之也:民国亡,国民党灭,开放之期已至!谁能为之主人者?计亚洲中有资格者,一为日本天皇,一为\xpinyin*{宣统}皇帝,然使日本天皇提出开放之议,各国闻之者,其感念如何?安乎?不安乎?日本皇帝自建此议,安乎?不安乎?若\xpinyin*{宣统}皇帝,则已闲居二十年,其权力已失,正以权力已失,而益增其提议之资格。以其无种族国际之意见,且无逞强凌弱之野心故也。\\
\end{quote}

可见,他不但看到满洲,而且看到全中国,全国的“开放之期已至”,更何论东北!那时他考虑的主要问题,不在于去东北的时机,而在于如何应付\xpinyin*{罗振玉}的新挑战。\\

挑战是从我去日军司令部的前几天就开始了的。那天,我接到了从东北来的两封信,一封是\xpinyin*{罗振玉}的,一封是给\ruby{溥伟}{\textcolor{PinYinColor}{Pu Wei}}当秘书的\xpinyin*{周善培}(在清末给\xpinyin*{岑春煊}做过幕僚)的,都要求我“给以便宜行事”的“手谕”,以便为我活动。照他们的话说,时机已至,各方面一联络即成,目前只差他们的代表身分证明了。我把这事告诉了\xpinyin*{郑孝胥},他慌忙拦阻道:“此事万不可行!此类躁进之人见用,必有损令名!”\\

\xpinyin*{郑孝胥}胥我被\xpinyin*{罗振玉}垄断了去,对这一点,我当时自然理会不到,我只觉得既然都主张等一下去东北的人,而去东北的人也快回来了,不妨就等一等。这时的\xpinyin*{陈曾寿}惟恐我变了主意,忙给我上了一个奏折。这个奏折可说是代表了\xpinyin*{陈宝琛}这派人当时思想的一个典型材料:\\

\begin{quote}
	奏为密规近日情势,宜慎赴机宜,免误本谋,恭摺仰祈圣鉴事。窃同凡事不密则害成。所当暗中着着进行,不动声色,使人无从窥其际。待机会成熟,然后一举而起。故不动则已,动则必期于成。若事未实未稳,已显露于外,使风声四播,成为众矢之的,未有不败者也。今皇上安居天津,毫无举动,已远近传言,多所揣测。若果有大连之行,必将中外喧腾,指斥无所不至,则日本纵有此心,亦将阻而变计。彼时进既不能,退又不可,其为危险岂堪设想。且事之进行,在人而不在地。苟机有可乘,在津同一接洽;若机无可图,赴连亦属\xpinyin*{罔}济。且在津则暗中进行,而易混群疑,赴连则举世惊哗,而横生阻碍。在津则事虽不成,犹有余地以自处;赴连则事苟无着,即将悬寄而难归。事理昭然,有必至者。抑在今日局势未定,固当沉机以观变,即将来东省果有拥戴之诚,日本果有敦请皇上复位之举,亦当先察其来言者为何如人。若仅出于一部分军人之意,而非由其政府完全谅解,则歧异可虑,变象难测。万一其政府未能同意,中道改计,将若之何?是则断不可冒万险以供其军人政策之尝试。若来者实由其政府举动,然后探其真意所在。如其确出仗义扶助之诚,自不可失此良机;如其怀有利用欺诱之意,则朝鲜覆辙具在,岂可明知其为陷阱而甘蹈之。应付之计,宜与明定约言,确有保障而后可往。大抵路、矿、商务之利,可以酌量许让。用人行政之权,必须完全自主。对外可与结攻守之同盟,内政必不容丝毫之干预。此当预定一坚决不移之宗旨,以为临事应付之根本者也。昔晋文公借秦力以复国,必有栾、卻、狐,先为之内主;楚昭王借秦兵以却吴,亦有子西等旧臣收合余烬,以为先驱。自古未有专恃外力,而可以立国者。此时局势,亦必东省士绅将帅先有拥戴归向之表示,而后日本有所凭借,以为其扶助之资。此其时机,似尚未至。今日东省人士犹怀观望之心,若见日本与民国政府交涉决裂,当有幡然改图者矣。今列强外相群集于日内瓦,欲借国联局面施其调停。日本不肯开罪于列强,闻已提出条款大纲,若民国政府应允,即许退兵。在民国政府虽高唱不屈之论,实则色厉内荏,恐终出于屈服之途。日本苟尝所欲,必将借以收场。若交涉不能妥协,则或别有举动。此时形势犹徘徊歧路之间,万不可冒昧轻动,陷于进退维谷之地也。观今日民国情形,南京与广东虽趋合并,而彼此仇恨已深,同处一堂,互相猜忌,其合必不能久。彼等此时若与日本决裂,立将崩溃。如允日本要求,则与其平日夸示国人者完全背驰,必将引起内乱,无以自立。日本即一时撤兵,仍将伺隙而动。故此时我之所谋,即暂从缓动,以后机会甚多。若不察真相,轻于一试,一遭挫折,反永绝将来之望,而无以立足矣。皇上天纵英明,饱经忧患,必能坚持定见,动合机宜,不致轻为所摇。臣愚见所及,是否有当,理会恭折密陈,伏祈圣鉴。谨奏。\\
\end{quote}

在这各种不同的想法里,静园里越加不能安静了。与此同时,又发生了一件出乎意外的事情。
