\fancyhead[LO]{{\scriptsize 1950-1954: 由抗拒到认罪 · 小家族起变化}} %奇數頁眉的左邊
\fancyhead[RO]{} %奇數頁眉的右邊
\fancyhead[LE]{} %偶數頁眉的左邊
\fancyhead[RE]{{\scriptsize 1950-1954: 由抗拒到认罪 · 小家族起变化}} %偶數頁眉的右邊
\chapter*{小家族起变化}
\addcontentsline{toc}{chapter}{\hspace{1cm}小家族起变化}
\thispagestyle{empty}
所长这段话的含意,我是过了许多年以后才明白的。当时我只是想,他既然说“需要改造”,那么我眼前就没有什么危险。\\

可是万没想到,在我觉得已经没了危险的时候,危险就来了。\\

有一天,我的眼镜腿掉了,我请看守员代我送到大李那里去修理。大李是个很巧的人,他常给人修理些小玩意,像眼镜、钟表、自来水笔等等,到他手里都能整旧如新。我的眼镜每逢有了毛病,他总是很认真地给我修好。没想到,这一次他的态度变了。\\

我们这个管理所的建筑有个特点,楼上楼下的声响可以互相听到。看守员拿了我的眼镜下楼不久,我就听见了大李嘟嘟囔囔的声音。语音虽不清楚,但可以听出是不高兴。过了一会儿,看守员把眼镜带回来了,无可奈何地对我说:“你是不是自己想想办法?他说没办法修。”\\

我听到大李的嘟嚷声时,就满肚子是气,心想他竟然敢对我端架子,太可恶了。我倒要看看他是不是敢端下去。我对看守员说:“我自己会就不找他了。上次就是他修好的,还是请江先生跟他再说说吧。”这位江看守员年纪很轻,个子瘦小,平常很少说话。我们同屋的人都说他为人老实。他果然很老实,听了我的话又下楼去了。\\

这回大李没推,给我修理了。可是拿回来一看,修得非常马虎,只是用一根线系了一下,连原来的螺丝都不见了。\\

我仔细地琢磨了一下,终于明白了大李是变了,而且不是从今天开始的。我记起了不久前的一天,我因为多日不见大李,散步时想问他在忙什么,就叫小瑞去找,不料小瑞回来说:“大李说他忙,没功夫。”刚才从他拒绝修眼镜的嘟囔声音里,我模糊地听到这样一句话:“我不能老伺候他,我没功夫!”\\

修眼镜的事过去不久,便到了一九五二年的新年。所方让我们组织一个新年晚会,自己演唱一些小节目,作为娱乐。舞台就是岗台前的空地。我在“三人快板”这个节目上,又发现了不祥之兆。\\

这是小秀小固和大李三个人自编自演的。他们那间屋子里,除了小瑞,全都上了台。他们三个人用问答的形式,数说着发生在犯人中的引人发笑的故事,讽刺了某些犯人不得人心的行为。比如被人们称做大下巴的前伪满司法大臣张焕相,他最爱对人发脾气,吵起来弄得四邻不安,他在吃饭时常洒一地饭粒,别人如果给他指出来,他就洒得更多。又比如有些人当看守员经过的时候拚命提高嗓门读书,其实不是为自己读,而是做给所方看。他们一面念着快板,一面模仿着被讽刺者的姿态,引起了一阵阵的笑声。我一听就知道这主要是小固编的。起初我也觉得很好笑,可是听到后来就笑不起来了。他们讽刺起一些迷信鬼神的人。他们说,这种人不明白从前算卦、求神并没有挽救了自己,进了管理所还偷偷地念咒求神。这段快板的讽刺对象,显然也把我包括了进去,因为我这时还没有完全停止念咒求神的活动。这段快板,说的虽然并非毫无道理,可是,我怎么可以被讽刺呢?不错,从前我确实是上过卦、\xpinyin*{乩}、经、咒的当,我们现在关在监狱里,渐渐明白了求神不如求人的道理,可是又何必当众影射我?这简直是“没上没下”了!\\

问题还不仅限于此。接着,他们又讽刺了一种人,这种人进了监狱,明白了许多道理,政府拿他当人看待,“但是他仍要给别人当奴才”,“百依百顺地伺候别人”,结果不能帮助“别人”改造,只能“帮助别人维持主人架子,对抗改造”。我一听立刻就明白了这个被讽刺的人是谁,这个“别人”又是谁。同时也明白了小瑞不参加这个节目演出的原因。我心里疼惜起小瑞来,我更担心小瑞会撑不下去。\\

事实上,小瑞跟别人一样,也有了一些变化。最近大李、小秀和小固在院子里不露面了,小瑞也减少了露面的次数,我的脏衣服逐渐积压起来,多日送不出去。\\

开过这次晚会,小瑞索性不来拿我的衣服去洗了。紧接着,又出了一件大事。\\

这天该我值日,我蹲在栏杆边上等着接饭菜。送饭菜的是小瑞。他把一样样饭菜递完,最后拿出一张叠成小块的纸条,放在我手里。我怔了一下,忙悄悄地藏起来,然后回身送饭,尽力不动声色。饭后,我装作上厕所,在屋角矮墙后的马桶上,偷偷地打开纸条。只见那上面写着:\\

我们都是有罪的,一切应该向政府坦白。我从前给您藏在箱底的东西,您坦白了没有?自己主动交代,政府一定宽大处理。\\

一股怒火,陡然在我胸中升起。但是过了不大时间,这股怒火就被一股冷气压熄了。我看到了众叛亲离的预兆。\\

纸条扔到马桶里被水冲走了,纸条所带来的心思却去不掉。我默默地回想着这几个青年人的过去和现在,觉得他们的变化简直不可思议。小秀不必说了,其余的几个是怎么变的呢?\\

大李,他的父亲原在颐和园当差,侍奉过西太后,由于这个关系,在宫里裁汰太监时,他得以进宫当差,那年他才十四岁。后来随我到天津,和另外几个童仆一起,在我请来的汉文教师教导下念书。他正式做了我的随侍,是我认为最可靠的仆人之一。我离大栗子沟时,挑了他做跟随。在苏联,他曾因一个日本人不肯让路而动过拳头,对我却始终恭顺,俯首贴耳地听我训斥。他为我销毁珠宝,做得涓滴不留,一丝不苟。对这样的一个人,我实在想象不出他发生变化的理由。现在事实就是如此,在他的眼里,已经没有了“上边”和“下边”了。\\

小固,是恭亲王\ruby{溥伟}{Pu Wei}的儿子,\ruby{溥伟}{Pu Wei}去世后,我以大清皇帝的身分赐他袭爵,把他当做未来“中兴”的骨干培养,他也以此为终身志愿,到了苏联还写过述志诗以示不忘。他在我的教育下,笃信佛教,曾人迷到整天对着骷髅像参“白骨禅”,而且刚到哈尔滨那天,还不忘表示过忠诚。没想到这样的人,竟会编出那样的快板来讽刺我,显然,他的忠诚是不存在了。\\

最不可思议的是小瑞的变化。如果说大李是“非我族类,其心必异”,小秀是由于“\xpinyin*{睚眦}之仇”,小固是看穿了“白骨样”之类的欺骗,那么小瑞是为了什么呢?\\

小瑞是清朝停亲王的后人,他家这一支自从他祖父\ruby{载}{zǎi}\ruby{濂}{lián}、叔祖父\ruby{载漪}{Dzai I}和\ruby{载澜}{Dzai Lan}被列为“\xpinyin*{庚子}肇祸诸臣”之后,败落了下来。他十九岁那年被我召到长春,与其他的贫穷“宗室子弟”一起念书。在那批被称为“内廷学生”的青年中,他被我看做是最听话、最老实的一个。我觉得他天资低些,心眼少些,而服侍我却比心眼多的更好。在苏联,他表现出的忠诚,五年如一日。记得我曾经试验过他一次,我对他说:“你如果真的忠于皇上,心里有什么,都该说出来。你有没有不敬的想头?”他听了,立刻满脸通红,连声说“有罪有罪”,经我一追问,这老实人说出了一件使他不安已久的事。原来有一次我为了一件事不称心,叫几个侄子一齐跪了一个钟头,他那时心里喊了一声冤枉,埋怨我不好伺候。他说出了这个秘密,满脸流汗,恐惶万状。如果我这时下令叫他痛打自己一顿,他必是乐于执行的。我只点点头说:“你只要知罪就行了,姑且宽赦你这一回!”他忙磕头谢恩,好像从地狱回到天堂一样的快乐。从苏联临回国时,我断定性命难保,曾和妹夫、弟弟们商量“立\xpinyin*{嗣}”问题,决定叫小瑞做我的承继人。他听到这个决定后的表现就更不用说了。如果说,在苏联时我有时还叫别人干点什么,那么回国之后,别人就不用想插手,因为我身边的事全被他包办下来了。这样的一个人,今天却教训起我来,说我“有罪”了!\\

这些不可思议的变化,其实只要细想一下,是可以看出一些端倪来的。新年晚会那天,小固有一段快板诗,里面反映了他们的思想变化。大概意思是说他从少年时期到了伪满,终日在“内廷”里听着反宣传,受着奴化教育,久而久之认为日本人是天底下最强大的,中国老百姓是天生无能、该受摆布的,以及人是生来要分等级的等等。他们回国之后,才明白过去是受了骗。回国的第一天,在绥芬河车站上发现火车司机是中国人,这就大大出乎他们的意料之外。以后,几乎天天发现有出乎意料的事情。他们最感到意外的,是所方人员的态度和抗美援朝的胜利。……\\

小固的这段唱词,我当时只当做是一般的开场白,未加注意。然而这不正是他们对我“背叛”的原因吗?他们不是发现被我欺骗了吗?但这都不是我当时能理解的。我最不明白的是,他们离开了我以后,与所方人员——所长、干部、看守员、炊事员、医生、护士们接触时,都强烈感觉出与前不同的地位:在这里,虽然是个犯人,却是个有人格的人,而从前虽然被看做是个贵族,被看做是“一人之下、万人之上”的人,实际上却是个不折不扣的奴才。他们越想越觉得自己的青春时代过得不光彩。我们回国,列车在沈阳站停下时,正赶上与我们同车的一位女工\footnote{即大连化工厂女工\xpinyin*{赵桂兰}。赵因用身体掩盖了一瓶将要爆炸的化学物品,被炸去了一只手,保住了工厂。}下车,这位女工因保护祖国财产而负伤,在站上她受到了各界人士们的热烈欢迎。他们听车上的公安战士们讲述了那位青年女工的故事,第一次知道了原来还有这样不同的青年生活在人间。以后,他们又听到了志愿军的英雄事迹,祖国建设事业中的英雄事迹,这给他们打开了视野。他们经过不断的对比,不由得不开始思索起许多问题:为什么从前不知道世界上还有这样的人?为什么同样是青年人,人家会那样生活,而自己却只知参禅、磕头?为什么人家那样尊严地、光荣地生活着,而自己却受到无理打骂还要谢恩认罪?为什么人家这样有本事,而自己却什么也不懂?……\\

这样想着想着,他们就变了。他们开始认真地学习,开始向所方讲出了过去的一切。\\

我消灭了纸条,靠墙坐着,忧间地想:共产党真厉害,不知是使了什么法儿,让他们变成这个样儿。我惟一感到一点安慰的,是妹夫和弟弟们还没有什么异状,不过这点安慰,却抵不上我的忧虑:小瑞会不会向所方检举我?\\

一想到检举,我心里除了气恼、忧虑,更感到了左右为难。我藏在皮箱底层的东西,都是经过精选的白金、黄金、钻石、珍珠之类的首饰,共计四百六十八件。我把它看做后半生生活的依靠,如果没有了它,即使放了我,我也无法活下去。“自食其力”这四个字,在我脑子里根本就不存在。把珠宝交出去吗?我隐瞒了这么长时间,忽然拿了出来,这就证明了我过去全是骗人。继续隐瞒下去吗?除了小瑞,其他人也都知道这个秘密。即使小瑞不说,其他人说不说,我更没有把握。如果被别人揭发出来,那就更糟!\\

“主动交代,可以宽大处理。”这句话在我心里浮现出来,随后又渐渐消失了。\\

那时在我看来,“共产党”三个字和“宽大”总像调和不起来似的。尽管进入管理所以来受到的待遇大大出乎意料,尽管从报上屡次看到从宽处理“五反”案件的消息,但是我还是不能相信。在“三反”、“五反”运动开始不久,有个别罪大恶极的贪污犯被判处了死刑,接着,报上揭露了许多资本家盗窃国家资财、窃取经济情报、走私、行贿,以及偷漏国税等等罪行,这时我不由得把这些案件拿来跟我的加以比较。我对“首恶必办,胁从不问,立功受奖”这几句话也另有自己想法。我认为即使那些宽大事例全是真的,也不会适用于我,因为我是“首恶”,属于必办之类的。\\

“坦白从宽”吗?——我苦笑了一下。在我的设想中,管理所长听我说出了这件事,知道受了骗,立刻会勃然大怒,狠狠地责罚我,而且追究我还有什么别的欺骗行为。我当初对待处于自己权威下的人,就是如此。\\

我不能去坦白,——我对自己说,小瑞他们还不至于真的能“绝情绝义”到检举我的地步。我把这件事拖下来了。\\

过了一个星期,又轮到小瑞给我们送饭。我偷偷地注意到,他的神色十分严肃,连看也不看我一眼。不但如此,他还对我的皮箱狠狠地盯了一阵。\\

不好,——我心里嘀咕着,他别是要有什么举动吧?\\

过了不到两个小时,我们刚刚开始学习,小瑞忽然匆匆地又来了。他在我们房外停了一下,然后匆匆地走开。我看得清清楚楚,他的两眼刚才正是搜索那只皮箱的。\\

我断定他刚才一定到所长那里去过。我沉不住气了。“与其被揭发出来,倒不如主动交代的好。”我心里说。\\

我抓住了组长老王的手,忙不迭地说:\\

“我有件事情要向政府坦白。我现在就告诉你……”
