\fancyhead[LO]{{\scriptsize 1859-1908: 我的家世 · 亲王之家}} %奇數頁眉的左邊
\fancyhead[RO]{} %奇數頁眉的右邊
\fancyhead[LE]{} %偶數頁眉的左邊
\fancyhead[RE]{{\scriptsize 1859-1908: 我的家世 · 亲王之家}} %偶數頁眉的右邊
\chapter*{亲王之家}
\addcontentsline{toc}{chapter}{\hspace{1cm}亲王之家}
\thispagestyle{empty}
我一共有四位祖母,所谓醇贤亲王的嫡福晋\ruby{叶赫那拉}{\textcolor{PinYinColor}{\Man ᠶᡝᡥᡝ ᠨᠠᡵᠠ}}氏,并不是我的亲祖母。她在我出生前十年就去世了。听说这位老太太秉性和她姊姊完全不同,可以说是墨守成规,一丝不苟。\ruby{同治}{\textcolor{PinYinColor}{\Man ᠶᠣᠣᠨᡳᠩᡤᠠ ᡩᠠᠰᠠᠨ}}死后,\xpinyin*{慈禧}照常听戏作乐,她却不然,有一次这位祖母奉召进宫看戏,坐在戏台前却闭上双眼,\xpinyin*{慈禧}问她这是干什么,她连眼也不睁地说:“现在是国丧,我不能看戏!”\xpinyin*{慈禧}给她顶的也无可奈何。她的忌讳很多,家里人在她面前说话都要特别留神,什么“完了”“死”这类字眼要用“得了”“喜”等等代替。她一生拜佛,成年放生烧香,夏天不进花园,说是怕踩死蚂蚁。她对蚂蚁仁慈如此,但是打起奴仆来,却毫不留情。据说醇王府一位老太监的终身不治的颜面抽搐病,就是由她的一顿藤鞭打成的。\\

她一共生了五个孩子。第一个女儿活到六岁,第一个儿子还不到两周岁,这两个孩子在\ruby{同治}{\textcolor{PinYinColor}{\Man ᠶᠣᠣᠨᡳᠩᡤᠠ ᡩᠠᠰᠠᠨ}}五年冬天相隔不过二十天都死了。第二个儿子就是\ruby{光绪}{\textcolor{PinYinColor}{\Man ᠪᠠᡩᠠᡵᠠᠩᡤᠠ ᡩᠣᡵᠣ}},四岁离开了她。\ruby{光绪}{\textcolor{PinYinColor}{\Man ᠪᠠᡩᠠᡵᠠᠩᡤᠠ ᡩᠣᡵᠣ}}进宫后,她生下第三个儿子,只活了一天半。第四个男孩\ruby{载}{\textcolor{PinYinColor}{zǎi}}\ruby{洸}{\textcolor{PinYinColor}{guāng}}出世后,她不知怎样疼爱是好,穿少了怕冻着,吃多了怕撑着。朱门酒肉多得发臭,朱门子弟常生的毛病则是消化不良。《红楼梦》里的贾府“净钱一天”是很有代表性的养生之道。我祖母就很相信这个养生之道,总不肯给孩子吃饱,据说一只虾也要分成三段吃,结果第四个男孩又因营养不够,不到五岁就死了。王府里的老太监\xpinyin*{牛祥}曾说过:“要不然怎么五爷(\ruby{载沣}{\textcolor{PinYinColor}{\Man ᡯᠠᡳ ᡶᡝᠩ}})接了王爷呢,就是那位老福晋,疼孩子,反倒把前面几位小爷给耽误了。”\\

我父亲\ruby{载沣}{\textcolor{PinYinColor}{\Man ᡯᠠᡳ ᡶᡝᠩ}}虽非她的亲生子,但依宗法,要受她的管教。她对我父亲和叔父们的饮食上的限制没有了,精神上的限制仍然没有放松。据那位牛太监说:“五爷六爷在她老人家跟前连笑也要小心,如果笑出声来,就会听见老人家吆喝:笑什么?没个规矩!”\\

醇贤亲王的第一侧福晋\ruby{颜扎}{\textcolor{PinYinColor}{Yanja}}氏去世很早。二侧福晋\xpinyin*{刘佳}氏,即是我的亲祖母,她在\ruby{那拉}{\textcolor{PinYinColor}{\Man ᠨᠠᡵᠠ}}氏祖母去世后当了家。她虽不像\ruby{那拉}{\textcolor{PinYinColor}{\Man ᠨᠠᡵᠠ}}氏祖母那样古板,却是时常处于精神不正常的状态。造成这种病症的原因同样是与儿孙命运相关。这位祖母也夭折过一个两岁的女儿。而使她精神最初遭受刺激以致失常的,却是由于幼子的出嗣。她一共生了三个儿子,即\ruby{载沣}{\textcolor{PinYinColor}{\Man ᡯᠠᡳ ᡶᡝᠩ}}、\ruby{载}{\textcolor{PinYinColor}{Dzai}}\ruby{洵}{\textcolor{PinYinColor}{Xun}}、\ruby{载}{\textcolor{PinYinColor}{Dzai}}\ruby{涛}{\textcolor{PinYinColor}{Tao}}。七叔\ruby{载}{\textcolor{PinYinColor}{Dzai}}\ruby{涛}{\textcolor{PinYinColor}{Tao}}从小在她自己怀里长大,到十一岁这年,突然接到\xpinyin*{慈禧}太后旨意,让他过继给我祖父的堂兄弟\ruby{奕}{\textcolor{PinYinColor}{I}}\ruby{谟}{\textcolor{PinYinColor}{Mu}}贝子为子。我祖母接到这个“\xpinyin*{懿}旨”,直哭得死去活来。经过这次刺激,她的精神就开始有些不正常了。\\

\ruby{奕}{\textcolor{PinYinColor}{I}}\ruby{谟}{\textcolor{PinYinColor}{Mu}}膝下无儿无女,得着一个过继儿子,自然非常高兴,当做生了一个儿子,第三天大做弥月,广宴亲朋。这位贝子平时不大会奉承\xpinyin*{慈禧},\xpinyin*{慈禧}早已不满,这次看到他如此高兴,更加生气,决定不给他好气受。\xpinyin*{慈禧}曾有一句“名言”:“谁叫我一时不痛快,我就叫他一辈子不痛快。”不知道\ruby{奕}{\textcolor{PinYinColor}{I}}\ruby{谟}{\textcolor{PinYinColor}{Mu}}以前曾受过她什么折磨,他在发牢骚时画了一张画,画面只有一只脚,影射\xpinyin*{慈禧}专门胡搅,搅得家事国事一团糟,并且题了一首发泄牢骚的打油诗:“老生避脚实堪哀,竭力经营避脚台,避脚台高三百尺,高三百尺脚仍来。”不知怎的,被\xpinyin*{慈禧}知道了,\xpinyin*{慈禧}为了泄忿,突然又下一道\xpinyin*{懿}旨,让已经过继过去五年多的七叔,重新过继给我祖父的八弟钟郡王\ruby{奕詥}{\textcolor{PinYinColor}{I Ho}}。\ruby{奕}{\textcolor{PinYinColor}{I}}\ruby{谟}{\textcolor{PinYinColor}{Mu}}夫妇受此打击,一同病倒。不久,\ruby{奕}{\textcolor{PinYinColor}{I}}\ruby{谟}{\textcolor{PinYinColor}{Mu}}寿终正寝,\xpinyin*{慈禧}又故意命那个抢走的儿子\ruby{载}{\textcolor{PinYinColor}{Dzai}}\ruby{涛}{\textcolor{PinYinColor}{Tao}}代表太后去致祭,\ruby{载}{\textcolor{PinYinColor}{Dzai}}\ruby{涛}{\textcolor{PinYinColor}{Tao}}有了这个身份,在灵前自然不能下跪。接着不到半年,\ruby{奕}{\textcolor{PinYinColor}{I}}\ruby{谟}{\textcolor{PinYinColor}{Mu}}的老妻也气得一命呜呼。\\

在第二次指定七叔过继的同时,\xpinyin*{慈禧}还指定把六叔\ruby{载}{\textcolor{PinYinColor}{Dzai}}\ruby{洵}{\textcolor{PinYinColor}{Xun}}过继出去,给我另一位堂祖叔敏郡王\xpinyin*{奕志}为嗣。正像漠贝子诗中所说的那样:“避脚台高三百尺,高三百尺脚仍来。”\xpinyin*{刘佳}氏祖母闭门家中坐,忽然又少掉了一个儿子,自然又是一个意外打击。事隔不久,又来了第三件打击。我祖母刚给我父亲说好一门亲事,就接到\xpinyin*{慈禧}给我父亲指婚的\xpinyin*{懿}旨。原来我父亲早先订了亲,\xpinyin*{庚子}年八国联军进北京时,许多旗人因怕洋兵而全家自杀,这门亲家也是所谓殉难的一户。我父亲随\xpinyin*{慈禧}\ruby{光绪}{\textcolor{PinYinColor}{\Man ᠪᠠᡩᠠᡵᠠᠩᡤᠠ ᡩᠣᡵᠣ}}在西安的时候,祖母重新给他订了一门亲,而且放了“大定”,即把一个如意交给了未婚的儿媳。按习俗,送荷包叫放小定,这还有伸缩余地,到了放大定,姑娘就算是“婆家的人”了。放大定之后,如若男方死亡或出了什么问题,在封建礼教下就常有什么望门寡或者殉节之类的悲剧出现。\xpinyin*{慈禧}当然不管你双方本人以及家长是否同意,她做的事,别人岂敢说话。\xpinyin*{刘佳}氏祖母当时是两头害怕,怕\xpinyin*{慈禧}怪罪,又怕退“大定”引起女方发生意外,这就等于对太后抗旨,男女两方都是脱不了责任的。尽管当时有人安慰她,说奉太后旨意去退婚不会有什么问题,她还是想不开,精神失常的病患又发作了。\\

过了六年,她的病又大发作了一次,这就是在军机大臣送来\xpinyin*{懿}旨叫送我进宫的那天。我一生下来,就归祖母抚养。祖母是非常疼爱我的。听乳母说过,祖母每夜都要起来一两次,过来看看我。她来的时候连鞋都不穿,怕木底鞋的响声惊动了我。这样看我长到三岁,突然听说\xpinyin*{慈禧}把我要到宫里去,她立即昏厥了过去。从那以后,她的病就更加容易发作,这样时好时犯地一直到去世。她去世时五十九岁,即我离京到天津那年。\\

醇亲王\ruby{载沣}{\textcolor{PinYinColor}{\Man ᡯᠠᡳ ᡶᡝᠩ}}自八岁丧父,就在醇贤亲王的遗训和这样两位老人的管教下,过着传统的贵族生活。他当了摄政王,享受着俸禄和采邑的供应,上有母亲管着家务,下有以世袭散骑郎二品长史\footnote{二品长史是皇室内务府派给各王府的名义上的最高管家,是世袭的二品官。其实他并不管事憋了王府中有婚丧大事时去一下之外,平日并不去王府。}为首的一套办事机构为他理财、酬应,有一大批护卫、太监、仆妇供他役使,还有一群清客给他出谋划策以及聊天游玩。他用不着操心家庭生活,也用不上什么生产知识。他和外界接触不多,除了依例行事的冠盖交往,谈不到什么社会阅历。他的环境和生活就是如此。\\

我父亲有两位福晋,生了四子七女。我的第二位母亲是\xpinyin*{辛亥}以后来的,我的三胞妹和异母生的两个弟弟和四个妹妹出生在民国时代。这一家人到现在,除了大妹和三弟早故外,父亲殁于一九五一年年初,母亲早于一九二一年逝世。\\

母亲和父亲是完全不同的类型。有人说旗人的姑奶奶往往比姑爷能干,或许是真的。我记得我的妻子\xpinyin*{婉容}和我的母亲\ruby{瓜尔佳}{\textcolor{PinYinColor}{\Man ᡤᡡᠸᠠᠯᡤᡳᠶᠠ}}氏就比我和父亲懂得的事多,特别是会享受,会买东西。据说旗人姑娘在家里能主事,能受到兄嫂辈的尊敬,是由于每个姑娘都有机会选到宫里当上嫔妃(据我想,恐怕也是由于兄弟辈不是游手好闲就是忙于宦务,管家理财的责任自然落在姊妹们身上,因此姑娘就比较能干些)。我母亲在娘家时很受宠,\xpinyin*{慈禧}也曾说过“这姑娘连我也不怕”的话。母亲花起钱来,使祖母和父亲非常头痛,简直没办法。父亲的收入,不算田庄;亲王双俸和什么养廉银\footnote{清代制度官吏于常俸之外,朝廷为示要求官吏清廉之意,另给银钱,叫做养廉银。}每年是五万两,到民国时代的小朝廷还是每年照付。每次俸银到手不久,就被母亲花个精光。后来父亲想了很多办法,曾经和她在财物上分家,给她规定用钱数目,全不生效。我父亲还用过摔家伙的办法,比如拿起条几上的瓶瓶罐罐摔在地上,以示忿怒和决心。因为总摔东西未免舍不得,后来专门准备了一些摔不碎的铜壶铅罐之类的东西(我弟弟见过这些“道具”),不久,这些威风也被母亲识破了,结果还是父亲再拿出钱来供她花。花得我祖母对着账房送来的账条叹气流泪,我父亲只好再叫管事的变卖古玩、田产。\\

母亲也时常拿出自己贵重的陪嫁首饰去悄悄变卖。我后来才知道,她除了生活享受之外,曾避着父亲,把钱用在政治活动上,通过\ruby{荣禄}{\textcolor{PinYinColor}{\Man ᠯᡠᠩᡤᡠ}}的旧部如民国时代步兵统领衙门的总兵\xpinyin*{袁得亮}之流,去运动奉天的将领。这种活动,是与太妃们合谋进行的。她们为了复辟的梦想,拿出过不少首饰,费了不少银子。\ruby{溥杰}{\textcolor{PinYinColor}{\Man ᡦᡠ ᡤᡳᠶᡝ}}小时候曾亲眼看见过她和太妃的太监鬼鬼祟祟地商议事情,问她是什么事,她说:“现在你还小呢,将来长大了,就明白我在做着什么了。”她却不知道,她和太妃们的那些财宝,都给太监和\xpinyin*{袁得亮}中饱了。她对她父亲的旧部有着特殊的信赖,对\xpinyin*{袁世凯}也能谅解。\xpinyin*{辛亥}后,醇王府上下大小无不痛骂\xpinyin*{袁世凯},\xpinyin*{袁世凯}称帝时,孩子们把报纸上的\xpinyin*{袁世凯}肖像的眼睛都抠掉了,惟独母亲另有见解:“说来说去不怪\xpinyin*{袁世凯},就怪\xpinyin*{孙文}!”\\

我的弟弟妹妹们从小并不怕祖母和父亲,而独伯母亲。佣仆自然更不用说。有一天,我父亲从外面回来,看见窗户没有关好,问一个太监:“怎么不关好?”这太监回答说:“奶奶还没回来呢,不忙关。”父亲生了气,罚他蹲在地上。一个女仆说:“要是老爷子,还不把你打成稀烂!”老爷子是指母亲而言,她和\xpinyin*{慈禧}一样,喜欢别人把她当做男人称呼。\\

我三岁进宫,到了十一岁才认得自己的祖母和母亲,那次她们是奉太妃之召进宫的。我见了她们,觉得很生疏,一点不觉得亲切。不过我还记得祖母的眼睛总不离开我,而且好像总是闪着泪光。母亲给我的印象就完全不同,我见了她的时候生疏之外更加上几分惧怕。她每次见了我总爱板着脸说些官话:“皇帝要多看些祖宗的圣训”,“皇帝别贪吃,皇帝的身子是圣体,皇帝要早睡早起……”现在回想起来,那硬梆梆的感觉似乎还存在着,低贱出身的祖母和大学士府小姐出身的母亲,流露出的人情,竟是如此的不同。
