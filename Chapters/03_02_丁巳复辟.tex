\fancyhead[LO]{{\scriptsize 1917-1924: 北京的“小朝廷” · 丁巳复辟}} %奇數頁眉的左邊
\fancyhead[RO]{} %奇數頁眉的右邊
\fancyhead[LE]{} %偶數頁眉的左邊
\fancyhead[RE]{{\scriptsize 1917-1924: 北京的“小朝廷” · 丁巳复辟}} %偶數頁眉的右邊
\chapter*{丁巳复辟}
\addcontentsline{toc}{chapter}{\hspace{1cm}丁巳复辟}
\thispagestyle{empty}
\xpinyin*{袁世凯}去世那天,消息一传进紫禁城,人人都像碰上了大喜事。太监们奔走相告,太妃们去护国协天大帝关圣帝君像前烧香,\xpinyin*{毓庆宫}无形中停了一天课……\\

接着,紫禁城中就听见了一种新的响城声:\\

“\xpinyin*{袁世凯}失败,在于动了鸠占鹊巢之念。”\\

“帝制非不可为,百姓要的却是旧主。”\\

“\xpinyin*{袁世凯}与\ruby{拿破仑}{Napoléon}三世不同,他并不如拿氏有祖荫可恃。”\\

“与其叫姓袁的当皇帝,还不如物归旧主哩。”\\

……\\

这些声音,和师傅们说的“本朝深仁厚泽,全国人心思旧”的话起了共鸣。\\

这时我的思想感情和头几年有了很大的不同。这年年初,我刚在\xpinyin*{奕劻}\xpinyin*{谥法}问题上表现出了“成绩”,这时候,我又对报纸发生了兴趣。\\

袁死了不多天之后,报上登了“宗社党起事未成”、“满蒙匪势猖獗”的消息。我知道这是当初公开反抗共和的王公大臣——\xpinyin*{善耆}、\xpinyin*{溥伟}、\xpinyin*{升允}、\xpinyin*{铁良},正在为我活动。他们四人当初是被称做\xpinyin*{申包胥}的,哭\xpinyin*{秦庭}都没成功。后来\xpinyin*{铁良}躲到天津的外国租界,其余的住在日本租借地旅顺和大连,通过手下的日本浪人,勾结日本的军阀、财阀,从事复辟武装活动。四人中最活跃的是\xpinyin*{善耆},他任民政部尚书时聘用的警政顾问日本人\xpinyin*{川岛浪速},一直跟他在一起,给他跑合拉纤。日本财主大仓喜八郎男爵给了他一百万日圆活动费。日本军人\xpinyin*{青森}、\xpinyin*{土井}等人给他召募满蒙土匪,编练军队,居然有了好几千人。\xpinyin*{袁世凯}一死,就闹起来了。其中有一支由蒙古贵族\xpinyin*{巴布扎布}率领的队伍,一度逼近了张家口,气势十分猖獗。直到后来\xpinyin*{巴布扎布}在兵变中被部下刺杀,才告终结。在闹得最凶的那些天,出现了一种很奇特的现象:一方面“勤王军”和民国军队在满蒙几个地方乒乒乓乓地打得很热闹,另方面在北京城里的民国政府和清室小朝廷照旧祝贺往来,应酬不绝。紫禁城从\xpinyin*{袁世凯}去世那天开始的兴隆气象,蒸蒸日上,既不受\xpinyin*{善耆}和\xpinyin*{巴布扎布}的兴兵作乱的影响,更不受他们失败的连累。\\

袁死后,\xpinyin*{黎元洪}继任总统,\xpinyin*{段祺瑞}出任国务总理。紫禁城派了曾向\xpinyin*{袁世凯}劝进的\xpinyin*{溥伦}前去祝贺,\xpinyin*{黎元洪}也派了代表来答谢,并且把\xpinyin*{袁世凯}要去的皇帝仪仗仍送回紫禁城。有些王公大臣们还得到了民国的勋章。有些在\xpinyin*{袁世凯}时代东躲西藏的王公大臣,现在也挂上了嘉禾章,又出现于交际场所。元旦和我的生日那天,大总统派礼官前来祝贺,我父亲也向黎总统段总理赠送\xpinyin*{肴馔}。这时内务府比以前忙多了,要拟旨赐\xpinyin*{谥法},赏朝马、二人肩\xpinyin*{舆}、花翎、顶戴,要授什么“南书房行走”\footnote{行走即是已有一定官职,又派到别的机构去办事的意思。南书房在乾清宫之右,原为\xpinyin*{康熙}读书处,\xpinyin*{康熙}十六年始选翰林等官入内当值,凡被选入值者,叫做“入值南书房”或“南书房行走”,这是大臣难得的待遇。}、乾清门各等侍卫,要带领秀女供太妃挑选,也偷偷地收留下优待条件上所禁止的新太监。当然还有我所无从了解的各种交际应酬,由个别的私宴到对国会议员们的公宴。……\\

总之,紫禁城又像从前那样活跃起来。到了\xpinyin*{丁巳}年(民国六年)\xpinyin*{张勋}进宫请安,开始出现了复辟高潮。\\

在这以前,我亲自召见请安的人还不多,而且只限于满族。我每天的活动,除了到\xpinyin*{毓庆宫}念书,在养心殿看报,其余大部分时间还是游戏。我看见神武门那边翎顶袍褂多起来了,觉着高兴,听说勤王军发动了,尤其兴奋,而勤王军溃灭了,也感到泄气。但总的说来,我也很容易把这些事情忘掉。肃亲王逃亡旅顺,消息不明,未免替他担心,可是一看见骆驼打喷嚏很好玩,肃亲王的安危就扔到脑后去了。既然有王爷和师傅大臣们在,我又何必操那么多的心呢?到了事情由师傅告诉我的时候,那准是一切都商议妥帖了。阴历四月二十七日这天的情形就是如此。\\

这天新授的“太保”\xpinyin*{陈宝琛}和刚到紫禁城不久的“\xpinyin*{毓庆宫}行走”\xpinyin*{梁鼎芬},两位师傅一齐走进了\xpinyin*{毓庆宫}。不等落座,陈师傅先开了口:\\

“今天皇上不用念书了。有个大臣来给皇上请安,一会奏事处太监会上来请示的。”\\

“谁呀?”\\

“前两江总督兼摄江苏巡抚\xpinyin*{张勋}。”\\

“\xpinyin*{张勋}?是那个不剪辫子的定武军\xpinyin*{张勋}吗?”\\

“正是,正是。”\xpinyin*{梁鼎芬}点头赞许,“皇上记性真好,正是那个\xpinyin*{张勋}。”梁师傅向来不错过颂扬的机会,为了这个目的,他正在写我的起居注。\\

其实我并没有什么好记性,只不过前不久才听师傅们说起这个\xpinyin*{张勋}的故事。民国开元以来,他和他的军队一直保留着辫子。\xpinyin*{袁世凯}在民国二年扑灭“二次革命”,就是以他的辫子兵攻陷南京而告成功的。辫子兵在南京大抢大烧,误伤了日本领事馆的人员,惹起日本人提出抗议,辫帅赶忙到日本领事面前赔礼道歉,答应赔偿一切损失,才算了事。\xpinyin*{隆裕}死后,他通电吊唁称为“国丧”,还说了“凡我民国官吏莫非大清臣民”的话。\xpinyin*{袁世凯}死后不久,报上登出了\xpinyin*{张勋}的一封通电。这封通电表示了徐州的督军会议对袁死后政局的态度,头一条却是“尊重优待清室各条”。总之,我相信他是位忠臣,愿意看看他是个什么样儿。\\

按照清朝的规矩,皇帝召见大臣时,无关的人一律不得在旁。因此每次召见不常见的人之前,师傅总要先教导一番,告诉我要说些什么话。这次陈师傅用特别认真的神气告诉我,要夸赞\xpinyin*{张勋}的忠心,叫我记住他现在是长江巡间使,有六十营的军队在徐州、兖州一带,可以问问他徐、兖和军队的事,好叫他知道皇上对他很关心。末了,陈师傅再三嘱咐道:\\

“\xpinyin*{张勋}免不了要夸赞皇上,皇上切记,一定要以谦逊答之,这就是示以圣德。”\\

“满招损,谦受益。”梁师傅连忙补充说,“越谦逊,越是圣明。上次\xpinyin*{陆荣廷}\xpinyin*{觐}见天颜,到现在写信来还不忘称颂圣德……”\\

\xpinyin*{陆荣廷}是两广巡阅使,他是历史上第一个被赏赐紫禁城骑马的民国将领。两个月前,他来北京会晤\xpinyin*{段祺瑞},不知为什么,跑到宫里来给我请了安,又报效了崇陵植树一万元。我在回养心殿的轿子里忽然想起来,那次\xpinyin*{陆荣廷}\xpinyin*{觐}见时,师傅们的神色和对我的谆谆教诲,也是像这次似的。那次\xpinyin*{陆荣廷}的出现,好像是紫禁城里的一件了不起的大事。内务府和师傅们安排了不同平常的赏赐,有我写的所谓御笔福寿字和对联,有无量寿金佛一\xpinyin*{龛},三镶玉如意一柄,玉陈设二件和尺头四件。\xpinyin*{陆荣廷}走后来了一封信,请\xpinyin*{世续}“代奏叩谢天恩”。从那时起,“南陆北张”就成了上自师傅下至太监常提的话头。\xpinyin*{张谦恭}对我说过:“有了南陆北张两位忠臣,大清有望了。”\\

我根据太监给我买的那些石印画报,去设想\xpinyin*{张勋}的模样,到下轿的时候,他在我脑子里也没成型。我进养心殿不久,他就来了。我坐在宝座上,他跪在我面前磕了头。\\

“臣\xpinyin*{张勋}跪请圣安……”\\

我指指旁边一张椅子叫他坐下(这时宫里已不采取让大臣跪着说话的规矩了),他又磕了一个头谢恩,然后坐下来。我按着师傅的教导,问他徐、兖地方的军队情形,他说了些什么,我也没用心去听。我对这位“忠臣”的相貌多少有点失望。他穿着一身纱袍褂,黑红脸,眉毛很重,胖呼呼的。看他的短脖子就觉得不理想,如果他没胡子,倒像御膳房的一个太监。我注意到了他的辫子,的确有一根,是花白色的。\\

后来他的话转到我身上,不出陈师傅所料,果然恭维起来了。\\

他说:“皇上真是天在聪明!”\\

我说:“我差的很远,我年轻,知道的事挺少。”\\

他说:“本朝圣祖仁皇帝也是冲龄践柞,六岁登极呀!”\\

我连忙说:“我怎么比得上祖宗,那是祖宗……”\\

这次召见并不比一般的时间长,他坐了五六分钟就走了。我觉得他说话粗鲁,大概不会比得上\xpinyin*{曾国藩},也就觉不到特别高兴。可是第二天\xpinyin*{陈宝琛}、\xpinyin*{梁鼎芬}见了我,笑眯眯地说\xpinyin*{张勋}夸我聪明谦逊,我又得意了。至于\xpinyin*{张勋}为什么要来请安,师傅们为什么显得比\xpinyin*{陆荣廷}来的那次更高兴,内务府准备的赏赐为什么比对陆更丰富,太妃们为什么还赏赐了酒宴等等这些问题,我连想也没去想。\\

过了半个月,阴历五月十三这天,还是在\xpinyin*{毓庆宫},\xpinyin*{陈宝琛}、\xpinyin*{梁鼎芬}和\xpinyin*{朱益藩}三位师傅一齐出现,面色都十分庄严,还是陈师傅先开的口:\\

“\xpinyin*{张勋}一早就来了……”\\

“他又请安来啦?”\\

“不是请安,是万事俱备,一切妥帖,来拥戴皇上复位听政,大清复辟啦!”\\

他看见我在发怔,赶紧说:“请皇上务要答应\xpinyin*{张勋}。这是为民请命,天与人归……”\\

我被这个突如其来的喜事弄得昏昏然。我呆呆地看着陈师傅,希望他多说几句,让我明白该怎么当这个“真皇帝”。\\

“用不着和\xpinyin*{张勋}说多少话,答应他就是了。”陈师傅胸有成竹地说,“不过不要立刻答应,先推辞,最后再说:既然如此,就勉为其难吧。”\\

我回到养心殿,又召见了\xpinyin*{张勋}。这次\xpinyin*{张勋}说的和他的奏请复辟折上写的差不多,只不过不像奏折说的那么斯文就是了。\\

“\xpinyin*{隆裕}皇太后不忍为了一姓的尊荣,让百姓遭殃,才下诏办了共和。谁知办的民不聊生……共和不合咱的国情,只有皇上复位,万民才能得救。……”\\

听他念叨完了,我说:“我年龄太小,无才无德,当不了如此大任。”他夸了我一顿,又把\xpinyin*{康熙}皇帝六岁做皇帝的故事念叨一遍。听他叨叨着,我忽然想起了一个问题:\\

“那个大总统怎么办呢?给他优待还是怎么着?”\\

“\xpinyin*{黎元洪}奏请让他自家退位,皇上准他的奏请就行了。”\\

“唔……”我虽然还不明白,心想反正师傅们必是商议好了,现在我该结束这次召见了,就说:“既然如此,我就勉为其难吧!”于是我就又算是“大清帝国”的皇帝了。\\

\xpinyin*{张勋}下去以后,陆续地有成批的人来给我磕头,有的请安,有的谢恩,有的连请安带谢恩。后来奏事处太监拿来了一堆已写好的“上谕”。头一天一气下了九道“上谕”:\\

即位诏;\\

\xpinyin*{黎元洪}奏请奉还国政,封黎为一等公,以彰殊典;\\

特设内阁议政大臣,其余官制暂照\xpinyin*{宣统}初年,现任文武大小官员均著照常供职;\\

授七个议政大臣(\xpinyin*{张勋}、\xpinyin*{王士珍}、\xpinyin*{陈宝琛}、\xpinyin*{梁敦彦}、\xpinyin*{刘迁琛}、\xpinyin*{袁大化}、\xpinyin*{张镇芳})和两名内阁阎丞(\xpinyin*{张勋}的参谋长\xpinyin*{万绳栻}和\xpinyin*{冯国璋}的幕僚\xpinyin*{胡嗣瑗});\\

授各部尚书(外务部\xpinyin*{梁敦彦}、度支部\xpinyin*{张镇芳}、参谋部\xpinyin*{王士珍}。陆军部\xpinyin*{雷震春}、民政部\xpinyin*{朱家宝});\\

授\xpinyin*{徐世昌}、\xpinyin*{康有为}为粥德院正、副院长;\\

授原来各省的督军为总督、巡抚和都统(\xpinyin*{张勋}兼任直隶总督北洋大臣)。\\

据老北京人回忆当时北京街上的情形说:那天早晨,警察忽然叫各户悬挂龙旗,居民们没办法,只得用纸糊的旗子来应付;接着,几年没看见的清朝袍褂在街上出现了,一个一个好像从棺材里面跑出来的人物;报馆出了复辟消息的号外,售价比日报还贵。在这种奇观异景中,到处可以听到报贩叫卖“\xpinyin*{宣统}上谕”的声音:“六个子儿买古董咧!这玩艺过不了几天就变古董,六个大铜子儿买件古董可不贵咧!”\\

这时前门外有些铺子的生意也大为兴隆。一种是成衣铺,赶制龙旗发卖;一种是估衣铺,清朝袍褂成了刚封了官的遗老们争购的畅销货;另一种是做戏装道具的,纷纷有人去央求用马尾给做假发辫。我还记得,在那些日子里,紫禁城里袍袍褂褂翎翎顶顶,人们脑后都拖着一条辫子。后来讨逆军打进北京城,又到处可以拣到丢弃的真辫子,据说这是\xpinyin*{张勋}的辫子兵为了逃命,剪下来扔掉的。\\

假如那些进出紫禁城的人,略有一点儿像报贩那样的眼光,能预料到关于辫子和上谕的命运,他们在开头那几天就不会那样地快活了。\\

那些日子,内务府的人员穿戴特别整齐,人数也特别多(总管内务府大臣特别指示过),因人数仍嫌不够,临时又从候差人员中调去了几位。有一位现在还健在,他回忆说:“那两天咱们这些写字儿的散班很晚,总是写不过来。每天各太妃都赏饭。到赏饭的时候总少不了传话:不叫谢恩了,说各位大人的辛苦,四个宫的主子都知道。”他却不知道,几个太妃正乐得不知如何是好,几乎天天都去神佛面前烧香,根本没有闲工夫来接见他们。\\

在那些日子里,没有达到政治欲望的王公们,大不高兴。\xpinyin*{张勋}在发动复辟的第二天做出了一个禁止亲贵干政的“上谕”,使他们十分激忿。醇亲王又成了一群贝勒贝子们的中心,要和\xpinyin*{张勋}理论,还要亲自找我做主。\xpinyin*{陈宝琛}听到了消息,忙来嘱咐我说:\\

“本朝\xpinyin*{辛亥}让国,就是这般王公亲贵干政闹出来的,现在还要闹,真是胡涂已极!皇上万不可答应他们!”\\

我当然信从了师傅。然而自知孤立的王公们并不死心,整天聚在一起寻找对策。这个对策还没想好,讨逆军已经进了城。这倒成全了他们,让他们摆脱了这次复辟的责任。\\

陈师傅本来是个最稳重、最有见识的人。在这年年初发生的一件事情上,我对他还是这个看法。那时\xpinyin*{劳乃宣}悄悄地从青岛带来了一封信。发信者的名字已记不得了,只知道是一个德国人,代表德国皇室表示愿意支持清室复辟。\xpinyin*{劳乃宣}认为,这是个极好的机缘,如果再加上德清两皇室结亲,就更有把握。陈师傅对于这件事,极力表示反对,说\xpinyin*{劳乃宣}太荒唐,是个成事不足败事有余的人;即使外国人有这个好意,也不能找到\xpinyin*{劳乃宣}这样的人。谁知从复辟这天起,这个稳重老练的老夫子,竟完全变了。\\

“触孤臣孽子,其操心也危,其虑患也深,故达!”\\

复辟的第一天,我受过成群的孤臣孽子叩贺,回到\xpinyin*{毓庆宫},就听见陈师傅这么念叨。他拈着白胡子团儿,老光镜片后的眼睛眯成一道缝,显示出异乎寻常的兴奋。\\

然而使我最感到惊奇的,倒不是他的兴奋,也不是他在“亲贵于政”问题上表现出的与王公们的对立(虽然直接冒犯的是我的父亲),而是在处理\xpinyin*{黎元洪}这个问题上表现出的激烈态度。先是\xpinyin*{梁鼎芬}曾自告奋勇去见\xpinyin*{黎元洪},劝\xpinyin*{黎元洪}立即让出总统府,不料遭到拒绝,回来忿然告诉了\xpinyin*{陈宝琛}和\xpinyin*{朱益藩}。\xpinyin*{陈宝琛}听了这个消息,和\xpinyin*{梁鼎芬}、\xpinyin*{朱益藩}一齐来到\xpinyin*{毓庆宫},脸上的笑容完全没有了,露出铁青的颜色,失去了控制地对我说:\\

“\xpinyin*{黎元洪}竟敢拒绝,拒不受命,请皇上马上踢他自尽吧!”\\

我吃了一惊,觉得太过分了。\\

“我刚一复位,就赐\xpinyin*{黎元洪}死,这不像话。国民不是也优待过我吗?”\\

\xpinyin*{陈宝琛}这是第一次遇到我对他公开的驳斥,但是同仇敌忾竟使他忘掉了一切,他气呼呼地说:“\xpinyin*{黎元洪}岂但不退,还赖在总统府不走。乱臣贼子,元凶大憝,焉能与天子同日而语?”\\

后来他见我表示坚决,不敢再坚持,同意让\xpinyin*{梁鼎芬}再去一次总统府,设法劝他那位亲家离开。\xpinyin*{梁鼎芬}还没有去,\xpinyin*{黎元洪}已经抱着总统的印玺,跑到日本公使馆去了。\\

讨逆军逼近北京城,复辟已成绝望挣扎的时候,\xpinyin*{陈宝琛}和\xpinyin*{王士珍}、\xpinyin*{张勋}商议出了一个最后办法,决定拟一道上谕给\xpinyin*{张作霖},授他为东三省总督,命他火速进京勤王。\xpinyin*{张作霖}当时是奉天督军,对\xpinyin*{张勋}给他一个奉天巡抚是很不满足的。陈师傅对\xpinyin*{张作霖}这时寄托了很大的希望。这个上谕写好了,在用“御宝”时发生了问题,原来印盒的钥匙在我父亲手里。若派人去取就太费时间了,于是,陈师傅当机立断,叫人把印盒上的锁头索性砸开,取出了刻着“法天立道”的“宝”。(这道上谕并未送到\xpinyin*{张作霖}手里,因为带信的\xpinyin*{张海鹏}刚出城就被讨逆军截住了。)我对陈师傅突然变得如此果断大胆,有了深刻的印象。\\

复辟的开头几天,我每天有一半时间在\xpinyin*{毓庆宫}里。念书是停了,不过师傅们是一定要见的,因为每样事都要听师傅们的指导。其余半天的时间,是看看待发的上谕和“内阁官报”,接受人们的叩拜,或者照旧去欣赏蚂蚁倒窝,叫上\xpinyin*{驷}院\footnote{上\xpinyin*{驷}院是内务府管辖的三院之一,管理养牧马驼等事务。顺治初叫御马监,后改为阿登衙门,\xpinyin*{康熙}时才改名上\xpinyin*{驷}院。}太监把养的骆驼放出来玩玩。这种生活过了不过四五天,宫中掉下了讨逆军飞机的炸弹,局面就完全变了。磕头的不来了,上谕没有了,大多数的议政大臣们没有了影子,纷纷东逃西散,最后只剩下了\xpinyin*{王士珍}和\xpinyin*{陈宝琛}。飞机空袭那天,我正在书房里和老师们说话,听见了飞机声和从来没听见过的爆炸声,吓得我浑身发抖,师傅们也是面无人色。在一片混乱中,太监们簇拥着我赶忙回到养心殿,好像只有睡觉的地方才最安全。太妃们的情形更加狼狈,有的躲进卧室的角落里,有的钻到桌子底下。当时各宫人声噪杂,乱成几团。这是中国历史上第一次出现空袭,内战史上第一次使用中国空军。如果第一次的防空情形也值得说一下的话,那就是:各人躲到各人的卧室里,把廊子里的竹帘子(即雨搭)全放下来——根据太监和护军的知识,这就是最聪明的措施了。幸亏那次讨逆军的飞机并不是真干,不过是恐吓了一下,只扔下三个尺把长的小炸弹。这三个炸弹一个落在隆宗门外,炸伤了抬“二人肩\xpinyin*{舆}”的轿夫一名,一个落在御花园里的水池里,炸坏了水池子的一角,第三个落在西长街隆福门的瓦檐上,没有炸,把聚在那里赌钱的太监们吓了个半死。\\

给\xpinyin*{张作霖}发出上谕的第二天,紫禁城里听到了迫近的枪炮声,\xpinyin*{王士珍}和\xpinyin*{陈宝琛}都不来了,宫内宫外失掉了一切联系。后来,枪炮声稀疏下来,奏事处太监传来了“护军统领”\xpinyin*{毓逖}禀报的消息:“奏上老爷子,\xpinyin*{张勋}的军队打了胜仗,\xpinyin*{段祺瑞}的军队全败下去了!”这个消息也传到了太妃那里。说话之间,外边的枪炮声完全没有了,这一来,个个眉开眼笑,太监们的鬼话都来了,说关老爷骑的赤兔马身上出了汗,可见关帝显圣保过驾,\xpinyin*{张勋}才打败了\xpinyin*{段祺瑞}。我听了,忙到了关老爷那里,摸了摸他那个木雕的坐骑,却是干巴巴的。还有个太监说,今早上,他听见养心殿西暖阁后面有叮叮当当的盔甲声音,这必是关帝去拿那把青龙偃月刀。听了这些话,太妃和我都到钦安殿叩了头。这天晚上大家睡了一个安稳觉。第二天一清早,内务府报来了真的消息:“\xpinyin*{张勋}已经逃到荷兰使馆去了!……”\\

我的父亲和陈师傅在这时出现了。他们的脸色发灰,垂头丧气。我看了他们拟好的退位诏书,又害怕又悲伤,不由得放声大哭。下面就是这个退位诏书:\\

\begin{quote}
	\xpinyin*{宣统}九年五月二十日,内阁奉上谕:前据\xpinyin*{张勋}等奏称,国本动摇,人心思旧,恳请听政等语。朕以幼冲,深居宫禁,民生国计,久未与闻。我孝定景皇后逊政恤民,深仁至德,仰念遗训,本无丝毫私天下之心,惟据以救国救民为词,故不得已而九如所请,临朝听政。乃昨又据\xpinyin*{张勋}奏陈,各省纷纷称兵,是又将以政权之争致开兵衅。年来我民疾苦,己如火热水深,何堪再\xpinyin*{罹}干戈重兹困累。言念及此,辗转难安。朕断不肯私此政权,而使生灵有涂炭之虞,致负孝定景皇后之盛德。著\xpinyin*{王士珍}会同\xpinyin*{徐世昌},迅速通牒\xpinyin*{段祺瑞},商办一切交接善后事宜,以靖人心,而弭兵祸。\\

\begin{flushright}
	钦此!\\
\end{flushright}
\end{quote}
