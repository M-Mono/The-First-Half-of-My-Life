\fancyhead[LO]{{\scriptsize 1932-1945: 伪满十四年 · 尊严与职权}} %奇數頁眉的左邊
\fancyhead[RO]{} %奇數頁眉的右邊
\fancyhead[LE]{} %偶數頁眉的左邊
\fancyhead[RE]{{\scriptsize 1932-1945: 伪满十四年 · 尊严与职权}} %偶數頁眉的右邊
\chapter*{尊严与职权}
\addcontentsline{toc}{chapter}{\hspace{1cm}尊严与职权}
\thispagestyle{empty}
在《满洲国组织法》里,第一章“执政”共十三条,条条规定着我的权威。第一条是“执政统治满洲国”,第二至第四条规定由我“行使立法权”、“执行行政权”、“执行司法权”,以下各条规定由我“颁布与法律同一效力之紧急训令”,“制定官制、任命官吏”,“统帅陆海空军”,以及掌握“大赦、特赦、减刑及复权之权”,等等。实际上,我连决定自己出门行走的权力都没有。\\

有一天,我忽然想到外面去逛逛,便带着\xpinyin*{婉容}和两个妹妹来到以我的年号命名的“大同公园”。不料进了公园不久,日本宪兵队和“执政府警备处”的汽车便追来了,请我回去。原来他们发现了我不在执政府里,就告诉了日本宪兵司令部,宪兵司令部便出动了大批军警到处搜寻,弄得满城风雨。事后执政府顾问官\xpinyin*{上角利一}向我说,为了我的安全和尊严,今后再不要私自外出。从那以后,除了关东军安排的以外,我再没出过一次大门。\\

我当时被劝驾回来,听日本人解释说,这都是为了我的安全和尊严,觉得很有道理。可是等我在勤民楼办了一些日子的“公事”之后,我便对自己的安全和尊严发生了怀疑。\\

我自从发过誓愿之后,每天早起,准时到勤民楼办公。从表面上看来,我是真够忙的,从早到晚,总有人要求谒见。谒见者之中,除少数前来请安的在野旧臣或宗室觉罗之外,多数是当朝的新贵,如各部总长、特任级的参议之流。这些人见了我,都表白了忠心,献纳了贡物,可就是不跟我谈公事。我每次问起“公事”时,他们不是回答“次长在办着了”,就是“这事还要问问次长”。次长就是日本人,他们是不找我的。\\

\xpinyin*{胡嗣瑗}首先表示了气忿。他向\xpinyin*{郑孝胥}提出,各部主权应在总长手里,重要公事还应由执政先做出决定,然后各部再办,不能次长说什么是什么。\xpinyin*{郑孝胥}回答说:“我们实行的是责任内阁制,政务须由‘国务会议’决定。责任内阁对执政负责,每周由总理向执政报告一次会议通过的案件,请执政裁可。在日本就是如此。”至于总长应有主权问题,他也有同感。他说此事正准备向日本关东军司令官提出,加以解决。原来他这个总理与国务院的总务厅长官之间,也存在着这个问题。\\

\xpinyin*{郑孝胥}后来跟关东军怎么谈的,我不知道。但是\xpinyin*{胡嗣瑗}后来对我说的一次国务会议的情形,使我明白了所谓“责任内阁制”是怎么一回事,总长与次长是什么关系。\\

那是一次讨论关于官吏俸金标准问题的国务会议。一如往昔,议案是总务厅事先准备好了,印发给各部总长的。总长们对于历次的议案,例如接管前东北政府的财产、给日本军队筹办粮\xpinyin*{秣}、没收东北四大银号以成立中央银行等等,都是毫不费劲立表赞同的,但是这次的议案关系到自己的直接利害,因此就不是那么马虎了。总长们认真地研究了议案,立刻议论纷纷,表示不满。原来在《给与令草案》中规定,“日系官吏”的俸金与“满系”的不同,前者比后者的大约高出百分之四十左右。财政总长\xpinyin*{熙洽}最沉不住气,首先发表意见说:“这个议案,简直不像话。咱们既然是个复合民族国家,各民族一律平等,为什么日本人要受特殊待遇?如果说是个亲善国家的国民,就该表示亲善,为什么拿特别高的俸金?”实业总长\xpinyin*{张燕卿}也说:“\xpinyin*{本庄繁}司令官说过,日满亲善,同心同德,有福同享,有难同当。假若待遇不同,恐非\xpinyin*{本庄}司令官的本意。”其他总长,如交通总长\xpinyin*{丁鉴修}等人,也纷纷表示希望一视同仁,不分薄厚。总务厅长官\xpinyin*{驹井德三}一看情形不好,便止住了总长们的发言,叫议案起草人人事课长\xpinyin*{古海忠}之为草案做解答。古海不慌不忙,谈出了一番道理,大意是,要想讲平等,就要先看能力平等不平等,日本人的能力大,当然薪俸要高,而且日本人生活程度高,生来吃大米,不像“满”人吃高粱就能过日子。他又说:“要讲亲善,请日本人多拿一些俸金,这正是讲亲善!”总长们听了,纷纷表示不满。\xpinyin*{驹井}不得不宣布休会,改为明天再议。\\

第二天复会时,\xpinyin*{驹井}对大家说,他跟次长们研究过,关东军也同意,给总长们把俸额一律提高到与次长们同一标准。“但是,”他又补充说,“日系官吏远离本乡,前来为满洲人建设王道乐土,这是应该感激的,因此另外要付给日籍人员特别津贴。这是最后决定,不要再争执了。”许多总长听了这番话,知道再闹就讨没趣了,好在已经给加了钱,因此都不再做声,可是照洽自认为与\xpinyin*{本庄繁}有点关系,没把\xpinyin*{驹井}放在眼里,当时又顶了两句:“我不是争两个钱,不过我倒要问问,日本人在哪儿建设王道乐土?不是在满洲吗?没有满洲人,能建设吗?”\xpinyin*{驹井}听了,勃然大怒,拍着桌子吼道:“你知道满洲的历史吗?满洲是日本人流血换来的,是从俄国人手里夺回来的,你懂吗?”\xpinyin*{熙洽}面色煞白,问道:不让说话吗?\xpinyin*{本庄}司令官也没对我喊叫过。”\xpinyin*{驹井}依然喊叫道:“我就是要叫你明白,这是军部决定的!”这话很有效,\xpinyin*{熙洽}果然不再说话,全场一时鸦雀无声。\\

这件事情发生后,所谓“内阁制”和“国务会议”的真相,就瞒不住任何人了。\\

“国务院”的真正“总理”不是\xpinyin*{郑孝胥},而是总务厅长官\xpinyin*{驹井德三}。其实,日本人并不隐讳这个事实。当时日本《改造》杂志就公然称他为“满洲国总务总理”和“新国家内阁总理大臣”。\xpinyin*{驹井}原任职于“满铁”,据说他到东北不久即以一篇题为《满洲大豆论》的文章,得到了东京军部和财阀的赏识,被视为“中国通”。他被军部和财阀选中为殖民地大总管,做了实际上的总理,他眼中的顶头上司当然是关东军司令官,并不是我这个名义上的执政。\\

我和\xpinyin*{郑孝胥}是名义上的执政与总理,总长们是名义上的总长,所谓国务会议也不过是走走形式。国务会议上讨论的议案,都是“次长会议”上已做出决定的东西。次长会议又称“火耀会议”,是总务厅每星期二召集的各部次长的会议,这才是真正的“内阁会议”,当然这是只对“太上皇”关东军司令官负责的会议。每次会议有关东军第四课参加,许多议案就是根据第四课的需要拟订的。\\

这些事情,后来对谁都不是秘密了,按说我是应该能够清醒过来的,但我却不是这样的人。我身边有个爱说话的\xpinyin*{胡嗣瑗},由于他的时常提醒,我总也忘不了唯我独尊的身分,更忘不了早在张园就确立的一种思想,即“日本非我皇上正位,则举措难施”。日本人表面上对我的态度,也经常给我一种错觉,使我时常信以为真,认为我毕竟不同于\xpinyin*{熙洽},日本人不尊重我不行。例如在“协和会”的建立问题上,我就是这样想的。\\

我就职一个多月以后的一天,\xpinyin*{郑孝胥}向我做例行报告,提到关东军决定要成立一个政党,定名为“协和党”。这个党的任务是“组织民众协力建国”,培育民众具有“尊重礼教、乐听天命”的精神。我每逢听到有人提到“党”,总有谈虎色变的感觉,因此听了\xpinyin*{郑孝胥}的报告,比听到\xpinyin*{驹井}拍桌子的消息更紧张,连忙打断他的话,播手反对道:“要什么党?要党有什么好处?\xpinyin*{辛亥}亡国不就是‘党’闹的吗?\xpinyin*{孔子}说,君子矜而不争,群而不党,难道这些你全忘了吗?”\xpinyin*{郑孝胥}搭拉着脸说:“皇上的话很对,可是这是军部决定的。”他以为这句话可以堵上我的嘴了,没想到这次我把这件事看做生命攸关的问题,说什么也不肯同意。我对于他口口声声地说“军部决定的”,早已厌烦之至,不愿意再听,就生气地说:“你不去对日本人说,就给我把他们叫来!”\\

\xpinyin*{郑孝胥}走后,我把这件事告诉了\xpinyin*{胡嗣瑗}。这位秘书处长对我的做法大加恭维,并且说:\\

“依臣管见,不见得如\xpinyin*{郑孝胥}所说,事事皆军部做主。\xpinyin*{罗振玉}说过,\xpinyin*{郑孝胥}是依恃军部,跋扈犯上。皇上若是向军部据理而争,军部未必敢于专横。何况党之不利于我,犹不利于日本,日本军方焉能不明此理?”\\

我听他说得有理,就更有了主意。两天后,关东军第四课的参谋\xpinyin*{片仓衷}、参谋长\xpinyin*{桥本虎之助}、高参\xpinyin*{板垣征四郎}先后来向我做解释,都没有说服我。事情就拖下来了。\\

过了三个月,即这年的七月间,我相信我是胜利了。关东军决定不成立“协和党”,只成立一个“协和会”,作为“翼赞”政府的组织。这个会包括所有居民在内,具体地说,凡年满二十岁的男子均为会员,妇女均为其附属的“妇女会”会员,十五至二十岁的青年均为附属的“青年团”团员,十至十五岁的少年为附属的“少年团”团员。\\

事实上,关东军把“党”改为“会”,并非是对我有什么让步,而是认为这比弄个不伦不类的政党更便于统治东北人民,通过这样一个网罗一切人口的组织,更便于进行奴化宣传、特务监视和奴役人民。我眼中看不到这样的事实,只觉得日本人毕竟是要听我的。\\

有了这样的错觉,就无怪要再碰钉子了。这是订立《日满密约》以后的事。
