\fancyhead[LO]{{\scriptsize 1955-1959: 接受改造 · 再次参观}} %奇數頁眉的左邊
\fancyhead[RO]{} %奇數頁眉的右邊
\fancyhead[LE]{} %偶數頁眉的左邊
\fancyhead[RE]{{\scriptsize 1955-1959: 接受改造 · 再次参观}} %偶數頁眉的右邊
\chapter*{再次参观}
\addcontentsline{toc}{chapter}{\hspace{1cm}再次参观}
\thispagestyle{empty}
一九五七年下半年,我们再次出去参观,这次参观,我们到过沈阳、鞍山、长春和哈尔滨四个城市,看了一个水库工地(沈阳大伙房),十八个工厂,六个学术单位和学校,三个医院,两个展览馆,一个体育宫。在哈尔滨访问了受过日本七三一细菌部队灾害的平房区,晋谒了东北烈士馆。这次参观我们获得了比上次更加深刻的印象。我这里只想说说其中的几点观感。\\

我们看到的企业,除了少数是日本人遗留下来的以外,大多数是新建的。日本人遗留下来的企业在接收时几乎全是一堆破烂,像鞍钢和沈阳机床厂,就都经日本人和国民党破坏过,到了人民政府手里重新恢复、扩建,才成为今天这样巨大的规模。许多见过那些旧日企业规模的伪大臣,都感到非常惊奇。使我最感惊奇的,是从许多新设备上看到了用中国文字写的牌号、规格。我虽然没有别人那样多的阅历,但是从前一提到机器,在心里永远是跟洋文联系着:MADE IN USA,MADE IN GERMANY,……现在,我看到了中国自己制造的成套装备,而且这些企业的产品,就有一部分是要出口的。在那些产品上,赫然写着:“中华人民共和国制造”。\\

在鞍山钢铁公司里,我站在庞大的钢铁建筑面前,简直无法想象它是怎样从一堆破烂中恢复和扩建起来的。然而这是事实。日本人在离开的时候说:“把鞍山给中国人种高粱去吧!要想恢复,平心静气地说,要二十年!”中国人在这里没有种高粱,三年时间,把它恢复起来了,而且达到了一百三十五万吨的年产量,远远超过了伪满时期的最高纪录,又过了一个五年,年产量达到了五百三十五万吨,等于从一九一七年日本在鞍山创办昭和制钢所起,一直到一九四七年国民党最后撤走止,这三十一年的累计产量。\\

在参观中,我看到了无数这类的例子。每个例子都向我说明:中国人站起来了。中国人不但在战场上可以打胜仗,而且在经济建设上一样能打胜仗。如果不是我亲眼看到这个事实,如果十年前向我做出这样预言,不仅劝中国人种高粱的日本人不信,连我也不信。\\

在过去的四十年间,我根本忘掉了自己的国籍,忘掉了自己是中国人。我曾随着日本人一起称颂大和民族是最优秀的民族,我曾跟\xpinyin*{郑孝胥}一起幻想由“客卿”、“外力”来开发中国的资源,我曾与\ruby{溥杰}{Pu Giye}多次慨叹中国人之愚蠢与白种人之聪明。我进了管理所,还不相信新中国能在世界上站得住。在朝鲜战场上中朝人民军队打了胜仗,我不是觉得扬眉吐气而是提心吊胆,担心美国人会扔原子弹。我不明白,在联合国讲坛上,中国共产党人何以敢于控诉美帝国主义,而不怕把事情闹大。我不明白在板门店的谈判桌上,朝中方面的代表何以敢于对美国人说:“从战场上得不到的东西,休想从会议桌上得到。”总之,我患了严重的软骨病。\\

美国在朝鲜停战协定上签了字,日内瓦会议上显示出新中国在国际事务上的作用,这时我不由地想起了从鸦片战争以来的外交史,想起了西太后“量中华之物力,结与国之欢心”的政策,想起了\xpinyin*{蒋介石}勒令人民对帝国主义凶犯忍辱吞声以表示“泱泱大国民风”的“训示”。中国近代一百零九年的对外史,就是从我曾祖父\xpinyin*{道光}帝到国民党\xpinyin*{蒋介石}的软骨症的病历。从一八七一年清朝为了天津教案事件正式派遣外交使节\xpinyin*{崇厚}到法国去赔礼道歉起,到\xpinyin*{李鸿章}去日本马关,我父亲去德国,以至北洋政府外交官参加巴黎和会,\xpinyin*{孔祥熙}参加英王加冕典礼,哪一个不是去伺候洋人颜色的呢?

在那一百零九年间,那些带着从大炮、鸦片一直到十字架和口香糖的自以为文明、高尚的人,他们到中国来,任意地烧、杀。抢、骗,把军队驻扎在京城、口岸、通都大邑、要道、要塞上,无一不把中国人看做奴隶、野人和靶子。他们在中国的日历上,留下了数不清的“国耻纪念日”。他们和\xpinyin*{道光}帝、西太后、\ruby{奕劻}{I Kuwang}、\xpinyin*{李鸿章}、\xpinyin*{袁世凯}、\xpinyin*{段祺瑞}、\xpinyin*{蒋介石}订了成堆的变中国人为奴隶的条约。以致在近百年的外交关系史上,出现了各种耻辱的字眼:利益均沾、机会均等、门户开放、最惠国待遇、租借地、关税抵押、领事裁判权、驻军权、筑路权、采矿权、内河航行权、空运权……除此而外,他们得到的还有伤驴一条赔美金百元,杀人一命偿美金八十元,强奸中国妇女而不受中国法庭审判等等特权。\\

现在,那种屈辱的历史是一去不复返了。中国人扬眉吐气地站起来了,正满怀信心地建设自己的祖国,让一个个发出过耻笑声的“洋人”闭上了嘴。\\

在长春第一汽车制造厂,我们听到了一个小故事。汽车厂刚开始生产时,有个小学校的孩子们要来参观。汽车厂打算派车去接,孩子们打电话来问是不是新造的车,厂方回答说,新造的是运货卡车,坐着不舒服,准备派去的是进口的大轿车。孩子们表示了不同的看法,说:“进口轿车不如运货卡车舒服,我们要坐祖国造的卡车!”\\

祖国,她在孩子们的心里是如何崇高呵!而在我过去的心中,却四十多年一直没个影子。\\

作为一个中国人,今天无论是站在世界上,还是生活在自己的社会里,都是最尊严的。\\

关于别人日常怎么样地生活,我在过去(除了伪满后期一段时间以外)对这问题总怀有好奇心。我有生第一次出去满足这种好奇心,是到我父亲的北府,第二次是借探病为名去看\xpinyin*{陈宝琛}。我对他们的自由自在的生活很羡慕。后来我在天津,从西餐馆和外国娱乐场所观察过那些“高等华人”,觉得他们可能比我“自由”,但是不如我“尊贵”,我不太羡慕他们,但好奇心仍在。在伪满,只顾担忧,不大好奇了。回国之后,起初根本没想过这类问题,别人如何生活,与我无关,后来我感到前途明亮起来,这个问题又对我有了现实性,所以在这次参观中,我特别留心了这个问题。结果是,勾起了我无数回忆,心中起了无限感慨。\\

获得印象最深的是在哈尔滨。哈尔滨儿童公园里的儿童铁道,使我想起了跟蚂蚁打交道的童年。我从儿童医院的婴儿出生统计和保健情况上,看出了这在当年清朝皇族家庭中,也是不可企望的。我坐在哈尔滨太阳岛的条椅上,遥望江中的游艇,听着草地上男女青年们的手风琴声和唱歌声,想起了我前半生的岁月。我不但没高兴地唱过,就连坐在草上晒晒太阳的兴致都没有,更不用说是随意地走走了。那时我担心厨子赚我的菜钱,担心日本人要我的命……而这里,一切都是无忧无虑的。在我前面几丈远的水滨上,有个青年画家在专心致志地写生。我们坐在他身后,一直就没看见他回过一次头。他的提包和备用的画布都堆在条椅脚下,根本没有人替他看管,他似乎很有把握地知道,决没有人会拿走他的东西。这样的事,在旧社会里简直不可想象,而在这里却是个事实。\\

这也是一个事实:公园里的电话亭里,有一个小木箱,上面贴着一张写着“每次四分,自投入箱”的纸条。\\

据一个同伴说,太阳岛上从前有个俱乐部,上一次厕所都要给小费的。但是现在,家里人来信说,你无论在哪个饭馆、旅店。澡堂等等地方,如果给服务人员小费,那就会被服务员看做是对他们的侮辱。这也是事实。\\

在哈尔滨最后几天的参观,我从两个地方看出了世界上两类人的不同。一个地方是日本七三一细菌部队造过孽的平房区,另一个地方是东北烈士馆。\\

二次大战后,日本出版了一本《七三一细菌部队》,作者署名\ruby{秋山}{あきやま}\ruby{浩}{ひろし},是七三一部队的成员,写的是他在部队时,从一个角落上所看到的事情。据书上说,这是一座周围四公里的建筑群,主楼比日本丸之内大厦大四倍,里面有三千名工作人员,养着数以万计的老鼠,拥有所谓\ruby{石井}{いしい}式孵育器四千五百具,用鼠血繁殖着天文数字的跳蚤,每月生产鼠疫病菌三百公斤。“工场”里设有可容四五百人的供试验用的活人监狱,囚禁的人都是战俘和抗日爱国的志士们,有中国人,苏联人,也有蒙古人民共和国的公民。这些人不被称为人,只是被他们叫做“木头”。每年至少有六百人被折磨死在里面,受到的试验令人惨不忍闻:有的被剥得净光,在输进冷气的柜子里受冻伤试验,举着冻掉了肌肉只剩下骨头的手臂哆嗦着;有的像青蛙似地放在手术台上,被那些穿着洁白的工作服的人解剖着;有的被绑在柱子上,只穿一件小裤衩,忍受着细菌弹在面前爆炸;有的被喂得很肥壮,然后接受某种病菌的感染,如果不死,就再试验,这样一直到死掉为止……\\

那个作者在七三一部队时听说,培养这些病菌,威力可超过任何武器,可以杀掉一亿人口,这是日本军人引以自豪的。\\

在苏联红军进逼哈尔滨的时候,这个部队为了消灭罪证,将遗下的几百名囚犯一次全都毒死,打算烧成灰埋进一个大坑里。由于这些刽子手过于心慌,大部分人没有烧透,坑里埋不下,于是又把半熟的尸体从坑里扒出来,分出骨肉,把肉烧化,把人骨用粉碎机碾碎,然后又用炸药把主建筑炸毁。\\

不久以后,附近的村庄里有人走过废墟,看到一个破陶磁罐子里尽是跳蚤。这人受到了跳蚤叮咬,万没想到,刽子手遗下的鼠疫菌已进到他的体内。于是这个村庄便发生了鼠疫。人民政府马上派出了医疗大军进行防治抢救,可是这个一百来户的村子还是被夺去了一百四十二条性命。\\

这是我访问的一个社员,劳动模范\xpinyin*{姜淑清}亲眼看到的血淋淋的事实。她给我们讲了这个村子在伪满时期受过的罪之后,说:“日本小鬼子投了降,缴了枪,人民政府带着咱过上了好日子,有了地,给自个儿收下了庄稼,大伙高高兴兴地都说从这可好了,人民政府领导咱们就要过好日子了,谁知道小鬼子的坏心眼子还没有使完,走了还留下这一手!狠毒哪!”\\

“人活在世上,总应该做些对人类有益的事,才活得有意义,有把握。”\\

这是有一次所长说的话。这句话现在从我心底发出了响声。制造鼠疫菌的“瘟神”们和供奉“瘟神”的奴仆们,原是同一类的人,同是为了私欲,使出了一切毒辣和卑鄙的手段,不惜让成亿人走进毁灭。然而,这是枉然的,没有“把握”的。“瘟神”的最科学的武器并不万能,最费心机的欺诈并不能蒙住别人的眼睛。被毁灭的不是人民,而是“瘟神”自己。“瘟神”的武器和它的供奉者没留下来,留下来的是今天正在建设幸福生活的人民,包括曾住在离“瘟神”不过几百米地方的金星农业社这个村庄。这真是活得最有“把握”的人。由于他们是同样地有“把握”,所以姜大娘说的是台山堡刘大娘同样的话:\\

“听毛主席的话,好好学习,好好改造吧!”\\

无论是在姜大娘的干净明亮的小屋里,还是农业社的宽阔的办公室里,我都有这样一个感觉:金星社的社员们谈到过去,是简短的、缓慢的,但是一提到现在和未来,那气氛就完全不同了。谈到今天的收成,特别是他们的蔬菜生产,那真是又仔细,又生动。为了证明他们的话,社员们还领我们去看了他们的暖窖设备,看了新买来的生产资料——排灌机、载重汽车、各种各样的化肥,看了新建的学校、卫生所和新架设的电线。当他们谈到明年的计划指标时,更是神采飞扬。社长说得很谨慎,他向我指着一排一排新建的瓦房说:“明年大秋之后,我想可能多盖几间。”他说到几间时,我们谁也不相信那仅仅是三五间或十来间。\\

在我们离开这个村庄的时候,社员们搬来了整筐的黄瓜、小红萝卜送给我们。“留下吧,这是咱社里刚收的,东西不值钱,可是很新鲜。”社长不顾我们的辞谢,硬把筐子送进我们的车里。\\

我在车窗口凝视着逐渐远去的金星社新建的瓦房顶,回想着金星社长说到的那几句:“我想着……。”不知为什么,这句非常平凡的话,听在耳朵里,曾给我一种不同凡响的感觉。现在我明白了。这些曾被我轻视过的认为最没文化的人,他们用自己的双手勤勤恳恳地劳动着,他们做的事情是平凡而又伟大的,因为他们让大地给人类生长出粮食和蔬菜瓜果;他们的理想也是平凡而又伟大的,因为他们要让茅屋变成瓦房,以便让人们生活得更加美好。而那些曾被我敬畏过、看做优秀民族代表的日本军国主义者,他们掌握着近代的科学技术,干的却是制造瘟疫。制造死亡的勾当,他们也有理想,这理想便是奴役和消灭掉被压迫的民族。这两种人,究竟是谁文明谁野蛮呢?\\

平房区“细菌工场”遗留下的瓦砾,告诉了人们什么叫做丑恶,东北烈士馆里每一件烈士的遗物又告诉了人们什么叫做善良。这里的每件陈列品都在告诉人们:它的主人当初为了人类最美好的理想,如何流尽了最后一滴鲜血,让生命发出了最灿烂的光辉。无论是细菌工场的残砖烂铁还是东北烈士馆里的血衣、遗墨,都是一面镜子,从这面镜子里照出了我们这群参观者过去的丑陋形象。\\

东北烈士馆是一座庄严的罗马式建筑,当初被伪满哈尔滨警察署占用过十四年。在那血腥的年代里,这里不知有多少骨头最硬的中国人被审问、拷打、送上刑场。陈列在这里的烈士照片和遗物,仅仅是极小的一部分。烈士馆中每件实物和每件事迹,所指出的具体时间和地点,都可以引起一件使我羞愧的回忆。事变发生的第三天——一九三一年九月二十一日,中国共产党满洲省委召开紧急会议,号召东北的党员和一切爱国士兵立即武装起来,和敌人作斗争。那个决议书和哈尔滨小戎街三号省委故居的照片,把我引回到二十多年前静园的日子。为了挽救民族于危亡,东北人民在党的领导下,不顾\xpinyin*{蒋介石}的阻拦,自己起来战斗了,而我在静园里却加紧了卖国的罪恶活动。我想起了\ruby{土肥原}{どいはら}和\ruby{板垣}{いたがき},\xpinyin*{郑孝胥}父子和\xpinyin*{罗振玉},汤岗子和旅顺……\\

在讲解人员介绍\xpinyin*{杨靖宇}将军的事迹的时候,我又回忆起那几次“巡幸”到东边道——\xpinyin*{杨靖宇}、\xpinyin*{李红光}等将军的抗联第一军活动地区——的情形。我在那里看见过长白山的顶峰,看见过朝雾和初升的太阳。祖国的山野美景没动我的心,引起我注意的倒是铁路两侧的日本宪兵、伪满国兵和警察。日本人办的报纸上总在报道东边道的“土匪”已剿净,但是那次“巡幸”到这一带,还是如临大敌,惶惶不安。一直到最后逃亡到通化、大栗子沟,我还听说这里“不太平”。抗日联军在这一带一直战斗到日本投降。最后被消灭的不是抗联,而是自称胜利者的日本皇军。抗联当时面对着强大的关东军和装备优越的伪满国兵,处境的艰苦是难以想象的,但是从陈列的当时使用过的饭锅、水壶、自制斧头、磨得漆皮都没有了的缝纫机等等生活用具上,我似乎看到了这些用具的主人的声容笑貌——这是我从龙凤矿那位青年主任的脸上看见过的,是只有充满着坚强信心的人才可能有的声容笑貌。在一双用\xpinyin*{桦}树皮做的鞋子面前,我似乎听到了那种自信、高亢的声调,唱出了那首流传过的歌谣:\\

\begin{quote}
	\xpinyin*{桦}皮鞋,是国货,自己原料自己做。野麻搓成上鞋绳,皮子就在树上剥。\xpinyin*{桦}皮鞋,不简单,战士穿上能爬山;时髦小姐买不到,有钱太太没福穿。\xpinyin*{桦}皮鞋,真正好,战士穿上满山跑,追得鬼子丧了胆,追得汽车嘟嘟叫!\\
\end{quote}

日本人当初叫我“裁可”一批批的法令,然后据此施行了集家并屯、统制粮谷等等政策,封锁了山区,用尽一切办法去断绝抗联军队与外界的经济联系。它也确实做到了这一点,甚至\xpinyin*{杨靖宇}将军和一部分部队被包围起来了,绝粮的情况是千真万确的事实了,但是战斗还是在继续着,继续到日本人怀疑了自己所有的情报和所有的常识。为什么这些人没有粮还在打?他们吃什么?\xpinyin*{杨靖宇}将军不幸牺牲了,日本人为了解开这个谜,破开了将军的肚子,他们从这个坚强不屈的人的胃里,找到的是草根。树叶……\\

我记起了\ruby{吉冈}{よしおか}\ruby{安直}{やすなお}发出过的叹息:“共产军,真是可怕!”在拥有飞机、坦克的日本皇军眼里,草根竟然是可怕的东西。\\

在\xpinyin*{杨靖宇}将军和他的战友们歌唱着烨皮鞋,嚼着草根,对着那张旧地图上展望着祖国大地未来的时候,我正在害着怕,怕日本人的抛弃,怕夜间的噩梦,我正吃烦了荤腥,终日打卦念经\xpinyin*{杨靖宇}将军遗下的地图、图章、血衣和他小时候写的作文本,在我的眼前模糊起来。在我身后——我的同伴和日本战犯们中间传过来哭泣声,声音越来越响。参观到\xpinyin*{赵一曼}烈士遗像面前的时候,有人从行列中挤了出来,跪在烈士像前一面痛哭一面碰头在地。\\

“我就是那个伪警署长……”\\

这是伪勤劳部大臣\xpinyin*{于镜涛},他原先是这个哈尔滨的警察署长,\xpinyin*{赵一曼}烈士当初就押在这个警察署,就是在这间陈列室里受的审讯,而审讯者之中正有这个\xpinyin*{于镜涛}。\\

当年的审讯者,今天成了囚犯,受到了历史的审判。不用说,应该哭的决不仅是\xpinyin*{于镜涛}一个人。
