\fancyhead[LO]{{\scriptsize 1859-1908: 我的家世 · 慈禧太后的决定}} %奇數頁眉的左邊
\fancyhead[RO]{} %奇數頁眉的右邊
\fancyhead[LE]{} %偶數頁眉的左邊
\fancyhead[RE]{{\scriptsize 1859-1908: 我的家世 ·慈禧太后的决定}} %偶數頁眉的右邊
\chapter*{慈禧太后的决定}
\addcontentsline{toc}{chapter}{\hspace{1cm}慈禧太后的决定}
\thispagestyle{empty}
庚子后,\ruby{载漪}{zǎi yī}被列为祸首之一,发配新疆充军,他的儿子也失去了大阿哥的名号。此后七年间没有公开提起过废立的事。光绪三十四年十月,西太后在颐和园渡过了她的七十四岁生日,患了痢疾,卧病的第十天,突然做出了立\xpinyin*{嗣}的决定。跟着,光绪和慈禧就在两天中相继去世。我父亲这几天的日记有这样的记载:\\

\begin{quote}
	十九日。上朝。致庆邸急函一件……\\

二十日。上疾大渐。上朝。奉旨派\ruby{载沣}{zǎi fēng}恭代批折,钦此。\\

庆王到京,午刻同\xpinyin*{诣}仪鸾殿面承召见,钦奉\xpinyin*{懿旨}:\xpinyin*{醇亲王}\ruby{载沣}{zǎi fēng}著授为摄政王,钦此。又面承\xpinyin*{懿旨}:\xpinyin*{醇亲王}\ruby{载沣}{zǎi fēng}之子溥仪著在宫内教养,并在上书房读书,钦此。叩辞至再,未邀俞允,即命携之入宫。万分无法,不敢再辞,钦遵于申刻由府携溥仪入宫。又蒙召见,告知已将溥仪交在隆裕皇后宫中教养,钦此。即谨退出,往谒庆邸。\\

二十一日。\xpinyin*{癸酉}百刻,小臣\ruby{载沣}{zǎi fēng}跪闻皇上崩于\xpinyin*{瀛}台。亥刻,小臣同庆王、世相、\xpinyin*{鹿协揆}、张相、袁尚书、增大臣崇\xpinyin*{诣}福昌殿。仰蒙皇太后召见。面承\xpinyin*{懿旨}:摄政王\ruby{载沣}{zǎi fēng}之子溥仪著入承大统为\xpinyin*{嗣}皇帝,钦此。\\

又面承\xpinyin*{懿旨}:前因穆宗毅皇帝未有储贰,曾于同治十三年十二月初五日降旨,大行皇帝生有皇子即承继穆宗教皇帝为嗣。现在大行皇帝龙驭上宾,亦未有储贰,不得已以摄政王\ruby{载沣}{zǎi fēng}之子溥仪承继穆宗毅皇帝为嗣并兼承大行皇帝之挑。钦此。\\

又面承\xpinyin*{懿旨}:现在时势多艰,\xpinyin*{嗣}皇帝尚在冲龄,正宜专心典学,著摄政王\ruby{载沣}{zǎi fēng}为监国,所有军国政事,悉秉予之训示裁度施行,\xpinyin*{\xpinyin*{俟}嗣}皇帝年岁渐长,学业有成,再由\xpinyin*{嗣}皇帝亲裁政事,钦此。是日住于西苑军机处。\\
\end{quote}

这段日记,我从西太后宣布自己的决定的头一天,即十九日抄起,是因为十九日那句“致庆邸急函”和二十日的“庆王到京”四个字,与立\xpinyin*{嗣}大有关系。这是西太后为了宣布这个决定所做的必要安排之一。为了说清楚这件事,不得不从远处说起。\\

庆王就是以办理卖国外交和卖官\xpinyin*{鬻爵}而出名的\xpinyin*{奕劻}\footnote{\xpinyin*{奕劻}(1838-1917),是乾隆第十七子庆值亲王永璘之孙。初袭辅国将军,咸丰二年封贝子,十年封贝勒,同治十一年加郡王衔,光绪十年总理各国事务衙门,并封庆郡王,二十年封亲王。}。在西太后时代,能得到太后欢心就等于得到了远大前程。要想讨西太后的欢心,首先必须能随时摸得着太后的心意,才能做到投其所好。荣禄贿赂太监李莲英,让太太陪伴太后游乐,得到不少最好最快的情报,因此他的奉承和孝敬,总比别人更让太后称心满意。如果说\xpinyin*{奕劻}的办法和他有什么不同的话,那就是\xpinyin*{奕劻}在李莲英那里花了更多的银子,而\xpinyin*{奕劻}的女儿即著名的四格格\footnote{格格是清代皇族女儿的统一称呼,皇帝的女儿封公主称固伦格格,亲王女儿郡主称和硕格格,郡王女儿封县主称多罗格格,贝勒女儿封郡君亦称多罗格格,贝子女儿封县君称固山格格,镇国公、辅国公女儿封乡君称格格。格格又有汉族“小姐”之意,故旗人家女儿也叫格格。}也比荣禄太太更机灵。如果西太后无意中露出了她喜欢什么样的坎肩,或者嵌镶着什么饰品的鞋子,那么不出三天,那个正合心意的坎肩、鞋子之类的玩艺就会出现在西太后的面前。\xpinyin*{奕劻}的官运就是从这里开始的。在西太后的赏识下,\xpinyin*{奕劻}一再加官晋爵,以一个远支宗室的最低的爵位辅国将军,逐步进到亲王,官职做到总理各国事务衙门。他得到了这个左右逢源的差使,身价就更加不同,无论在太后眼里和洋人的眼里,都有了特殊的地位。辛丑议和是他一生中最重要的事件。在这一事件中,他既为西太后尽了力,使她躲开了祸首的名义,也让八国联军在条约上满了意。当时人们议论起王公们的政治本钱时,说某王公有德国后台,某王公有日本后台……都只不过各有一国后台而已,一说到庆王,都认为他的后台谁也不能比,计有八国之多。因此西太后从那以后非常看重他。光绪二十九年,他进入了军机处,权力超过了其他军机大臣,年老的礼亲王的领衔不过是挂个虚名。后来礼王告退,\xpinyin*{奕劻}正式成了领衔军机大臣,他儿子载振也当了商部尚书,父子显赫不可一世。尽管有反对他的王公们暗中搬他,御史们出面参他贪赃枉法,卖官\xpinyin*{鬻爵},都无济于事,奈何他不得。有位御史弹劾他“自任军机,门庭若市,细大不捐,其父子起居饮食车马衣服异常挥霍,……将私产一百二十万两送往东交民巷英商汇丰银行存储”,有位御史奏称有人送他寿礼十万两,花一万二千两买了一名歌妓送给他儿子。结果是一个御史被斥回原衙门,一个御史被夺了官。\\

西太后对\xpinyin*{奕劻}是否就很满意?根据不少遗老们侧面透露的材料,只能这样说:西太后后来对于\xpinyin*{奕劻}是又担心又依赖,所以既动不得他,并且还要笼络他。\\

使西太后担心的,不是贪污纳贿,而是袁世凯和\xpinyin*{奕劻}的特殊关系。单从袁在\xpinyin*{奕劻}身上花钱的情形来看,那关系就很不平常。袁世凯的心腹朋友徐世昌后来说过:庆王府里无论是生了孩子,死了人,或是过个生日等等,全由直隶总督衙门代为开销。\xpinyin*{奕劻}正式领军机处之前不久,有一天庆王府收到袁家送来十万两(一说二十万两)白银,来人传述袁的话说:“王爷就要有不少开销,请王爷别不赏脸。”过了不久,\xpinyin*{奕劻}升官的消息发表了,人们大为惊讶袁世凯的未卜先知。\\

戊戌政变后,西太后对袁世凯一方面是十分重视的,几年功夫把他由直隶按察使提到直隶总督、外务部尚书,恩遇之隆,汉族大臣中过去只有曾、胡、左、李才数得上。另一方面,西太后对这个统率着北洋新军并且善于投机的汉族大臣,并不放心。当她听说袁世凯向贪财如命的庆王那里大量地送银子时,就警惕起来了。\\

西太后曾经打过主意,要先把\xpinyin*{奕劻}开缺。她和军机大臣\xpinyin*{瞿鸿禨}\footnote{瞿鸿禨(1850-1918),字子玖,号止庵,湖南省长沙府善化县人。}露出了这个意思,谁知这位进士出身后起的军机,太没阅历,竟把这件事告诉了太太。这位太太有位亲戚在一家外文报馆做事,于是这个消息便辗转传到了外国记者的耳朵里,北京还没有别人知道,伦敦报纸上就登出来了。英国驻北京的公使据此去找外务部,讯问有无此事。西太后不但不敢承认,而且派铁良和鹿传霖追查,结果,\xpinyin*{瞿鸿禨}被革了职。\\

西太后倒\xpinyin*{奕劻}不成,同时因\xpinyin*{奕劻}有联络外国人的用途,所以也就不再动他,但对于袁世凯,她没有再犹豫。光绪三十三年,内调袁为外务部尚书,参加军机。明是重用,实际是解除了他的兵权。袁世凯心里有数,不等招呼,即主动交出了北洋新军的最高统帅权。\\

西太后明白,袁对北洋军的实际控制能力,并非立时就可以解除,袁和\xpinyin*{奕劻}的关系也不能马上斩断。正在筹划着下一个步骤的时候,她自己病倒了,这时又忽然听到这个惊人消息:袁世凯准备废掉光绪,推戴\xpinyin*{奕劻}的儿子载振为皇帝。不管\xpinyin*{奕劻}如何会办外交和会奉承,不管袁世凯过去对她立过多大的功,也不管他们这次动手的目标正是被她痛恨的光绪,这个以袁世凯为主角的阴谋,使她马上意识到了一种可怕的厄运——既是爱新觉罗皇朝的厄运,也是她个人的厄运。因此她断然地做出了一项决定。为了实现这个决定,她先把\xpinyin*{奕劻}调开,让他去东陵查看工程,然后把北洋军段祺瑞\footnote{段祺瑞(1864-1936),字芝泉,安徽合肥人,是袁世凯创办的北洋军的得力将领。在民国后成为北洋军阀皖系首领。袁世凯死后,在日本帝国主义支持下数度把持北京政权,是日本帝国主义的忠实走狗。一九三一年“九·一八”后又企图在日本帝国主义支持下组织华北汉奸政权,旋被抛弃,不久被蒋介石软禁在上海,一直到死。}的第六镇全部调出北京,开往深水,把陆军部尚书铁良统辖的第一镇调进来接防。等到\xpinyin*{奕劻}回来,这里一切大事已定:慈禧宣布了立我为嗣,封我父亲为摄政王。但是为了继续笼络住这位有八国朋友的庆王,给了他亲王世袭罔替的思荣。\\

关于袁、庆的阴谋究竟确不确,阴谋的具体内容又是什么,我说不清。但是我有一位亲戚亲自听铁良事后说起过西太后的这次安排。铁良说,为了稳定段祺瑞的第六镇北洋军,开拔之先发给了每名士兵二两银子,一套新装和两双新鞋。另外,我还听见一个叫李长安的老太监说起光绪之死的疑案。照他说,光绪在死的前一天还是好好的,只是因为用了一剂药就坏了,后来才知道这剂药是袁世凯使人送来的。按照常例,皇帝得病,每天太医开的药方都要分抄给内务府大臣们每人一份,如果是重病还要抄给每位军机大臣一份。据内务府某大臣的一位后人告诉我,光绪死前不过是一般的感冒,他看过那些药方,脉案极为平常,加之有人前一天还看到他像好人一样,站在屋里说话,所以当人们听到光绪病重的消息时都很惊异。更奇怪的是,病重消息传出不过两个时辰,就听说已经“晏驾”了。总之光绪是死得很可疑的。如果太监李长安的说法确实的话,那么更印证了袁庆确曾有过一个阴谋,而且是相当周密的阴谋。\\

还有一种传说,是西太后自知病将不起,她不甘心死在光绪前面,所以下了毒手。这也是可能的。但是我更相信的是她在宣布我为\xpinyin*{嗣}皇帝的那天,还不认为自己会一病不起。光绪死后两个小时,她还授命监国摄政王:“所有军国政事,悉秉承予之训示裁度施行。”到次日,才说:“现予病势危笃,恐将不起,嗣后军国政事均由摄政王裁定,遇有重大事件有必须请皇太后(指光绪的皇后,她的侄女那拉氏)\xpinyin*{懿旨}者,由摄政王随时面请施行。”她之所以在发现了来自袁世凯那里的危险之后,或者她在确定了光绪的最后命运之后,从宗室中单单挑选了这样的一个摄政王和这样一个\xpinyin*{嗣}皇帝,也正是由于当时她还不认为自己会死得这么快。在她来说当了太皇太后固然不便再替皇帝听政,但是在她与小皇帝之间有个听话的摄政王,一样可以为所欲为。\\

当然,她也不会认为自己老活下去。在她看来,她这个决定总算为保全爱新觉罗的宝座而尽了力。她甚至会认为,这个决定之正确,就在于她选定的摄政王是光绪的亲兄弟。因为按常情说,只有这样的人,才不至于上袁世凯的当。
