\fancyhead[LO]{{\scriptsize 1859-1908: 我的家世 · 慈禧太后的决定}} %奇數頁眉的左邊
\fancyhead[RO]{} %奇數頁眉的右邊
\fancyhead[LE]{} %偶數頁眉的左邊
\fancyhead[RE]{{\scriptsize 1859-1908: 我的家世 · 慈禧太后的决定}} %偶數頁眉的右邊
\chapter*{慈禧太后的决定}
\addcontentsline{toc}{chapter}{\hspace{1cm}慈禧太后的决定}
\thispagestyle{empty}
\xpinyin*{庚子}后,\ruby{载漪}{\textcolor{PinYinColor}{\Man ᡯᠠᡳ ᡳ}}被列为祸首之一,发配新疆充军,他的儿子也失去了大阿哥的名号。此后七年间没有公开提起过废立的事。\ruby{光绪}{\textcolor{PinYinColor}{\Man ᠪᠠᡩᠠᡵᠠᠩᡤᠠ ᡩᠣᡵᠣ}}三十四年十月,西太后在颐和园渡过了她的七十四岁生日,患了痢疾,卧病的第十天,突然做出了立\xpinyin*{嗣}的决定。跟着,\ruby{光绪}{\textcolor{PinYinColor}{\Man ᠪᠠᡩᠠᡵᠠᠩᡤᠠ ᡩᠣᡵᠣ}}和\xpinyin*{慈禧}就在两天中相继去世。我父亲这几天的日记有这样的记载:\\

\begin{quote}
	十九日。上朝。致庆邸急函一件……\\

二十日。上疾大渐。上朝。奉旨派\ruby{载沣}{\textcolor{PinYinColor}{\Man ᡯᠠᡳ ᡶᡝᠩ}}恭代批折,钦此。\\

庆王到京,午刻同\xpinyin*{诣}仪鸾殿面承召见,钦奉\xpinyin*{懿}旨:醇亲王\ruby{载沣}{\textcolor{PinYinColor}{\Man ᡯᠠᡳ ᡶᡝᠩ}}著授为摄政王,钦此。\\

又面承\xpinyin*{懿}旨:醇亲王\ruby{载沣}{\textcolor{PinYinColor}{\Man ᡯᠠᡳ ᡶᡝᠩ}}之子\ruby{溥仪}{\textcolor{PinYinColor}{\Man ᡦᡠ ᡳ}}著在宫内教养,并在上书房读书,钦此。叩辞至再,未邀俞允,即命携之入宫。万分无法,不敢再辞,钦遵于申刻由府携\ruby{溥仪}{\textcolor{PinYinColor}{\Man ᡦᡠ ᡳ}}入宫。又蒙召见,告知已将\ruby{溥仪}{\textcolor{PinYinColor}{\Man ᡦᡠ ᡳ}}交在\xpinyin*{隆裕}皇后宫中教养,钦此。即谨退出,往谒庆邸。\\

二十一日。\xpinyin*{癸酉}百刻,小臣\ruby{载沣}{\textcolor{PinYinColor}{\Man ᡯᠠᡳ ᡶᡝᠩ}}跪闻皇上崩于瀛台。亥刻,小臣同庆王、世相、\xpinyin*{鹿协揆}、张相、袁尚书、增大臣崇\xpinyin*{诣}福昌殿。仰蒙皇太后召见。面承\xpinyin*{懿}旨:摄政王\ruby{载沣}{\textcolor{PinYinColor}{\Man ᡯᠠᡳ ᡶᡝᠩ}}之子\ruby{溥仪}{\textcolor{PinYinColor}{\Man ᡦᡠ ᡳ}}著入承大统为\xpinyin*{嗣}皇帝,钦此。\\

又面承\xpinyin*{懿}旨:前因穆宗毅皇帝未有储贰,曾于\ruby{同治}{\textcolor{PinYinColor}{\Man ᠶᠣᠣᠨᡳᠩᡤᠠ ᡩᠠᠰᠠᠨ}}十三年十二月初五日降旨,大行皇帝生有皇子即承继穆宗教皇帝为嗣。现在大行皇帝龙驭上宾,亦未有储贰,不得已以摄政王\ruby{载沣}{\textcolor{PinYinColor}{\Man ᡯᠠᡳ ᡶᡝᠩ}}之子\ruby{溥仪}{\textcolor{PinYinColor}{\Man ᡦᡠ ᡳ}}承继穆宗毅皇帝为嗣并兼承大行皇帝之挑。钦此。\\

又面承\xpinyin*{懿}旨:现在时势多艰,\xpinyin*{嗣}皇帝尚在冲龄,正宜专心典学,著摄政王\ruby{载沣}{\textcolor{PinYinColor}{\Man ᡯᠠᡳ ᡶᡝᠩ}}为监国,所有军国政事,悉秉予之训示裁度施行,\xpinyin*{俟嗣}皇帝年岁渐长,学业有成,再由\xpinyin*{嗣}皇帝亲裁政事,钦此。是日住于西苑军机处。\\
\end{quote}

这段日记,我从西太后宣布自己的决定的头一天,即十九日抄起,是因为十九日那句“致庆邸急函”和二十日的“庆王到京”四个字,与立\xpinyin*{嗣}大有关系。这是西太后为了宣布这个决定所做的必要安排之一。为了说清楚这件事,不得不从远处说起。\\

庆王就是以办理卖国外交和卖官\xpinyin*{鬻}爵而出名的\ruby{奕劻}{\textcolor{PinYinColor}{\Man ᡳ ᡴᡠᠸᠠᠩ}}\footnote{\ruby{奕劻}{\textcolor{PinYinColor}{\Man ᡳ ᡴᡠᠸᠠᠩ}}(1838-1917),是\ruby{乾隆}{\textcolor{PinYinColor}{\Man ᠠᠪᡴᠠᡳ ᠸᡝᡥᡳᠶᡝᡥᡝ}}第十七子庆值亲\xpinyin*{王永璘}之孙。初袭辅国将军,\ruby{咸丰}{\textcolor{PinYinColor}{\Man ᡤᡠᠪᠴᡳ ᡝᠯᡤᡳᠶᡝᠩᡤᡝᡝ}}二年封贝子,十年封贝勒,\ruby{同治}{\textcolor{PinYinColor}{\Man ᠶᠣᠣᠨᡳᠩᡤᠠ ᡩᠠᠰᠠᠨ}}十一年加郡王衔,\ruby{光绪}{\textcolor{PinYinColor}{\Man ᠪᠠᡩᠠᡵᠠᠩᡤᠠ ᡩᠣᡵᠣ}}十年总理各国事务衙门,并封庆郡王,二十年封亲王。}。在西太后时代,能得到太后欢心就等于得到了远大前程。要想讨西太后的欢心,首先必须能随时摸得着太后的心意,才能做到投其所好。\ruby{荣禄}{\textcolor{PinYinColor}{\Man ᠯᡠᠩᡤᡠ}}贿赂太监\xpinyin*{李莲英},让太太陪伴太后游乐,得到不少最好最快的情报,因此他的奉承和孝敬,总比别人更让太后称心满意。如果说\ruby{奕劻}{\textcolor{PinYinColor}{\Man ᡳ ᡴᡠᠸᠠᠩ}}的办法和他有什么不同的话,那就是\ruby{奕劻}{\textcolor{PinYinColor}{\Man ᡳ ᡴᡠᠸᠠᠩ}}在\xpinyin*{李莲英}那里花了更多的银子,而\ruby{奕劻}{\textcolor{PinYinColor}{\Man ᡳ ᡴᡠᠸᠠᠩ}}的女儿即著名的四格格\footnote{格格是清代皇族女儿的统一称呼,皇帝的女儿封公主称固伦格格,亲王女儿郡主称和硕格格,郡王女儿封县主称多罗格格,贝勒女儿封郡君亦称多罗格格,贝子女儿封县君称固山格格,镇国公、辅国公女儿封乡君称格格。格格又有汉族“小姐”之意,故旗人家女儿也叫格格。}也比\ruby{荣禄}{\textcolor{PinYinColor}{\Man ᠯᡠᠩᡤᡠ}}太太更机灵。如果西太后无意中露出了她喜欢什么样的坎肩,或者嵌镶着什么饰品的鞋子,那么不出三天,那个正合心意的坎肩、鞋子之类的玩艺就会出现在西太后的面前。\ruby{奕劻}{\textcolor{PinYinColor}{\Man ᡳ ᡴᡠᠸᠠᠩ}}的官运就是从这里开始的。在西太后的赏识下,\ruby{奕劻}{\textcolor{PinYinColor}{\Man ᡳ ᡴᡠᠸᠠᠩ}}一再加官晋爵,以一个远支宗室的最低的爵位辅国将军,逐步进到亲王,官职做到总理各国事务衙门。他得到了这个左右逢源的差使,身价就更加不同,无论在太后眼里和洋人的眼里,都有了特殊的地位。\xpinyin*{辛丑}议和是他一生中最重要的事件。在这一事件中,他既为西太后尽了力,使她躲开了祸首的名义,也让八国联军在条约上满了意。当时人们议论起王公们的政治本钱时,说某王公有德国后台,某王公有日本后台……都只不过各有一国后台而已,一说到庆王,都认为他的后台谁也不能比,计有八国之多。因此西太后从那以后非常看重他。\ruby{光绪}{\textcolor{PinYinColor}{\Man ᠪᠠᡩᠠᡵᠠᠩᡤᠠ ᡩᠣᡵᠣ}}二十九年,他进入了军机处,权力超过了其他军机大臣,年老的礼亲王的领衔不过是挂个虚名。后来礼王告退,\ruby{奕劻}{\textcolor{PinYinColor}{\Man ᡳ ᡴᡠᠸᠠᠩ}}正式成了领衔军机大臣,他儿子\ruby{载}{\textcolor{PinYinColor}{zǎi}}\ruby{振}{\textcolor{PinYinColor}{zhèn}}也当了商部尚书,父子显赫不可一世。尽管有反对他的王公们暗中搬他,御史们出面参他贪赃枉法,卖官\xpinyin*{鬻}爵,都无济于事,奈何他不得。有位御史弹劾他“自任军机,门庭若市,细大不捐,其父子起居饮食车马衣服异常挥霍,……将私产一百二十万两送往东交民巷英商汇丰银行存储”,有位御史奏称有人送他寿礼十万两,花一万二千两买了一名歌妓送给他儿子。结果是一个御史被斥回原衙门,一个御史被夺了官。\\

西太后对\ruby{奕劻}{\textcolor{PinYinColor}{\Man ᡳ ᡴᡠᠸᠠᠩ}}是否就很满意?根据不少遗老们侧面透露的材料,只能这样说:西太后后来对于\ruby{奕劻}{\textcolor{PinYinColor}{\Man ᡳ ᡴᡠᠸᠠᠩ}}是又担心又依赖,所以既动不得他,并且还要笼络他。\\

使西太后担心的,不是贪污纳贿,而是\xpinyin*{袁世凯}和\ruby{奕劻}{\textcolor{PinYinColor}{\Man ᡳ ᡴᡠᠸᠠᠩ}}的特殊关系。单从袁在\ruby{奕劻}{\textcolor{PinYinColor}{\Man ᡳ ᡴᡠᠸᠠᠩ}}身上花钱的情形来看,那关系就很不平常。\xpinyin*{袁世凯}的心腹朋友\xpinyin*{徐世昌}后来说过:庆王府里无论是生了孩子,死了人,或是过个生日等等,全由直隶总督衙门代为开销。\ruby{奕劻}{\textcolor{PinYinColor}{\Man ᡳ ᡴᡠᠸᠠᠩ}}正式领军机处之前不久,有一天庆王府收到袁家送来十万两(一说二十万两)白银,来人传述袁的话说:“王爷就要有不少开销,请王爷别不赏脸。”过了不久,\ruby{奕劻}{\textcolor{PinYinColor}{\Man ᡳ ᡴᡠᠸᠠᠩ}}升官的消息发表了,人们大为惊讶\xpinyin*{袁世凯}的未卜先知。\\

\xpinyin*{戊戌}政变后,西太后对\xpinyin*{袁世凯}一方面是十分重视的,几年功夫把他由直隶按察使提到直隶总督、外务部尚书,恩遇之隆,汉族大臣中过去只有曾、胡、左、李才数得上。另一方面,西太后对这个统率着北洋新军并且善于投机的汉族大臣,并不放心。当她听说\xpinyin*{袁世凯}向贪财如命的庆王那里大量地送银子时,就警惕起来了。\\

西太后曾经打过主意,要先把\ruby{奕劻}{\textcolor{PinYinColor}{\Man ᡳ ᡴᡠᠸᠠᠩ}}开缺。她和军机大臣\xpinyin*{瞿鸿禨}\footnote{\xpinyin*{瞿鸿禨}(1850-1918),字\xpinyin*{子玖},号\xpinyin*{止庵},湖南省长沙府善化县人。}露出了这个意思,谁知这位进士出身后起的军机,太没阅历,竟把这件事告诉了太太。这位太太有位亲戚在一家外文报馆做事,于是这个消息便辗转传到了外国记者的耳朵里,北京还没有别人知道,伦敦报纸上就登出来了。英国驻北京的公使据此去找外务部,讯问有无此事。西太后不但不敢承认,而且派\ruby{铁良}{\textcolor{PinYinColor}{\Man ᠲᡞᠶᡝᡞ ᠯᡞᠶᠠᠩ}}和\xpinyin*{鹿传霖}追查,结果,\xpinyin*{瞿鸿禨}被革了职。\\

西太后倒\ruby{奕劻}{\textcolor{PinYinColor}{\Man ᡳ ᡴᡠᠸᠠᠩ}}不成,同时因\ruby{奕劻}{\textcolor{PinYinColor}{\Man ᡳ ᡴᡠᠸᠠᠩ}}有联络外国人的用途,所以也就不再动他,但对于\xpinyin*{袁世凯},她没有再犹豫。\ruby{光绪}{\textcolor{PinYinColor}{\Man ᠪᠠᡩᠠᡵᠠᠩᡤᠠ ᡩᠣᡵᠣ}}三十三年,内调袁为外务部尚书,参加军机。明是重用,实际是解除了他的兵权。\xpinyin*{袁世凯}心里有数,不等招呼,即主动交出了北洋新军的最高统帅权。\\

西太后明白,袁对北洋军的实际控制能力,并非立时就可以解除,袁和\ruby{奕劻}{\textcolor{PinYinColor}{\Man ᡳ ᡴᡠᠸᠠᠩ}}的关系也不能马上斩断。正在筹划着下一个步骤的时候,她自己病倒了,这时又忽然听到这个惊人消息:\xpinyin*{袁世凯}准备废掉\ruby{光绪}{\textcolor{PinYinColor}{\Man ᠪᠠᡩᠠᡵᠠᠩᡤᠠ ᡩᠣᡵᠣ}},推戴\ruby{奕劻}{\textcolor{PinYinColor}{\Man ᡳ ᡴᡠᠸᠠᠩ}}的儿子\ruby{载}{\textcolor{PinYinColor}{zǎi}}\ruby{振}{\textcolor{PinYinColor}{zhèn}}为皇帝。不管\ruby{奕劻}{\textcolor{PinYinColor}{\Man ᡳ ᡴᡠᠸᠠᠩ}}如何会办外交和会奉承,不管\xpinyin*{袁世凯}过去对她立过多大的功,也不管他们这次动手的目标正是被她痛恨的\ruby{光绪}{\textcolor{PinYinColor}{\Man ᠪᠠᡩᠠᡵᠠᠩᡤᠠ ᡩᠣᡵᠣ}},这个以\xpinyin*{袁世凯}为主角的阴谋,使她马上意识到了一种可怕的厄运——既是\ruby{爱新觉罗}{\textcolor{PinYinColor}{\Man ᠠᡳᠰᡳᠨ ᡤᡳᠣᡵᠣ}}皇朝的厄运,也是她个人的厄运。因此她断然地做出了一项决定。为了实现这个决定,她先把\ruby{奕劻}{\textcolor{PinYinColor}{\Man ᡳ ᡴᡠᠸᠠᠩ}}调开,让他去东陵查看工程,然后把北洋军\xpinyin*{段祺瑞}\footnote{\xpinyin*{段祺瑞}(1864-1936),字\xpinyin*{芝泉},安徽合肥人,是\xpinyin*{袁世凯}创办的北洋军的得力将领。在民国后成为北洋军阀皖系首领。\xpinyin*{袁世凯}死后,在日本帝国主义支持下数度把持北京政权,是日本帝国主义的忠实走狗。一九三一年“九·一八”后又企图在日本帝国主义支持下组织华北汉奸政权,旋被抛弃,不久被\xpinyin*{蒋介石}软禁在上海,一直到死。}的第六镇全部调出北京,开往深水,把陆军部尚书\ruby{铁良}{\textcolor{PinYinColor}{\Man ᠲᡞᠶᡝᡞ ᠯᡞᠶᠠᠩ}}统辖的第一镇调进来接防。等到\ruby{奕劻}{\textcolor{PinYinColor}{\Man ᡳ ᡴᡠᠸᠠᠩ}}回来,这里一切大事已定:\xpinyin*{慈禧}宣布了立我为嗣,封我父亲为摄政王。但是为了继续笼络住这位有八国朋友的庆王,给了他亲王世袭\xpinyin*{罔}替的恩荣。\\

关于袁、庆的阴谋究竟确不确,阴谋的具体内容又是什么,我说不清。但是我有一位亲戚亲自听\ruby{铁良}{\textcolor{PinYinColor}{\Man ᠲᡞᠶᡝᡞ ᠯᡞᠶᠠᠩ}}事后说起过西太后的这次安排。\ruby{铁良}{\textcolor{PinYinColor}{\Man ᠲᡞᠶᡝᡞ ᠯᡞᠶᠠᠩ}}说,为了稳定\xpinyin*{段祺瑞}的第六镇北洋军,开拔之先发给了每名士兵二两银子,一套新装和两双新鞋。另外,我还听见一个叫\xpinyin*{李长安}的老太监说起\ruby{光绪}{\textcolor{PinYinColor}{\Man ᠪᠠᡩᠠᡵᠠᠩᡤᠠ ᡩᠣᡵᠣ}}之死的疑案。照他说,\ruby{光绪}{\textcolor{PinYinColor}{\Man ᠪᠠᡩᠠᡵᠠᠩᡤᠠ ᡩᠣᡵᠣ}}在死的前一天还是好好的,只是因为用了一剂药就坏了,后来才知道这剂药是\xpinyin*{袁世凯}使人送来的。按照常例,皇帝得病,每天太医开的药方都要分抄给内务府大臣们每人一份,如果是重病还要抄给每位军机大臣一份。据内务府某大臣的一位后人告诉我,\ruby{光绪}{\textcolor{PinYinColor}{\Man ᠪᠠᡩᠠᡵᠠᠩᡤᠠ ᡩᠣᡵᠣ}}死前不过是一般的感冒,他看过那些药方,脉案极为平常,加之有人前一天还看到他像好人一样,站在屋里说话,所以当人们听到\ruby{光绪}{\textcolor{PinYinColor}{\Man ᠪᠠᡩᠠᡵᠠᠩᡤᠠ ᡩᠣᡵᠣ}}病重的消息时都很惊异。更奇怪的是,病重消息传出不过两个时辰,就听说已经“晏驾”了。总之\ruby{光绪}{\textcolor{PinYinColor}{\Man ᠪᠠᡩᠠᡵᠠᠩᡤᠠ ᡩᠣᡵᠣ}}是死得很可疑的。如果太监\xpinyin*{李长安}的说法确实的话,那么更印证了袁、庆确曾有过一个阴谋,而且是相当周密的阴谋。\\

还有一种传说,是西太后自知病将不起,她不甘心死在\ruby{光绪}{\textcolor{PinYinColor}{\Man ᠪᠠᡩᠠᡵᠠᠩᡤᠠ ᡩᠣᡵᠣ}}前面,所以下了毒手。这也是可能的。但是我更相信的是她在宣布我为\xpinyin*{嗣}皇帝的那天,还不认为自己会一病不起。\ruby{光绪}{\textcolor{PinYinColor}{\Man ᠪᠠᡩᠠᡵᠠᠩᡤᠠ ᡩᠣᡵᠣ}}死后两个小时,她还授命监国摄政王:“所有军国政事,悉秉承予之训示裁度施行。”到次日,才说:“现予病势危笃,恐将不起,嗣后军国政事均由摄政王裁定,遇有重大事件有必须请皇太后(指\ruby{光绪}{\textcolor{PinYinColor}{\Man ᠪᠠᡩᠠᡵᠠᠩᡤᠠ ᡩᠣᡵᠣ}}的皇后,她的侄女\ruby{那拉}{\textcolor{PinYinColor}{\Man ᠨᠠᡵᠠ}}氏)\xpinyin*{懿}旨者,由摄政王随时面请施行。”她之所以在发现了来自\xpinyin*{袁世凯}那里的危险之后,或者她在确定了\ruby{光绪}{\textcolor{PinYinColor}{\Man ᠪᠠᡩᠠᡵᠠᠩᡤᠠ ᡩᠣᡵᠣ}}的最后命运之后,从宗室中单单挑选了这样的一个摄政王和这样一个\xpinyin*{嗣}皇帝,也正是由于当时她还不认为自己会死得这么快。在她来说当了太皇太后固然不便再替皇帝听政,但是在她与小皇帝之间有个听话的摄政王,一样可以为所欲为。\\

当然,她也不会认为自己老活下去。在她看来,她这个决定总算为保全\ruby{爱新觉罗}{\textcolor{PinYinColor}{\Man ᠠᡳᠰᡳᠨ ᡤᡳᠣᡵᠣ}}的宝座而尽了力。她甚至会认为,这个决定之正确,就在于她选定的摄政王是\ruby{光绪}{\textcolor{PinYinColor}{\Man ᠪᠠᡩᠠᡵᠠᠩᡤᠠ ᡩᠣᡵᠣ}}的亲兄弟。因为按常情说,只有这样的人,才不至于上\xpinyin*{袁世凯}的当。
