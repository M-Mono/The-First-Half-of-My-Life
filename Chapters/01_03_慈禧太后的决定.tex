\fancyhead[LO]{{\scriptsize 1859-1908: 我的家世 · 慈禧太后的决定}} %奇數頁眉的左邊
\fancyhead[RO]{} %奇數頁眉的右邊
\fancyhead[LE]{} %偶數頁眉的左邊
\fancyhead[RE]{{\scriptsize 1859-1908: 我的家世 ·慈禧太后的决定}} %偶數頁眉的右邊
\chapter*{慈禧太后的决定}
\addcontentsline{toc}{chapter}{\hspace{1cm}慈禧太后的决定}
\thispagestyle{empty}
庚子后,载漪被列为祸首之一,发配新疆充军,他的儿子也失去了大阿哥的名号。此后七年间没有公开提起过废立的事。光绪三十四年十月,西太后在颐和园渡过了她的七十四岁生日,患了痢疾,卧病的第十天,突然做出了立嗣的决定。跟着,光绪和慈禧就在两天中相继去世。我父亲这几天的日记有这样的记载:\\

\begin{quote}
	十九日。上朝。致庆邸急函一件……\\

二十日。上疾大渐。上朝。奉旨派载沣恭代批折,钦此。\\

庆王到京,午刻同诣仪鸾殿面承召见,钦奉懿旨:醇亲王载沣著授为摄政王,钦此。又面承懿旨:醇亲王载沣之子溥仪著在宫内教养,并在上书房读书,钦此。叩辞至再,未邀俞允,即命携之入宫。万分无法,不敢再辞,钦遵于申刻由府携溥仪入宫。又蒙召见,告知已将溥仪交在隆裕皇后宫中教养,钦此。即谨退出,往谒庆邸。\\

二十一日。癸酉百刻,小臣载沣跪闻皇上崩于瀛台。亥刻,小臣同庆王、世相、鹿协揆、张相、袁尚书、增大臣崇诣福昌殿。仰蒙皇太后召见。面承懿旨:摄政王载沣之子溥仪著入承大统为嗣皇帝,钦此。\\

又面承懿旨:前因穆宗毅皇帝未有储贰,曾于同治十三年十二月初五日降旨,大行皇帝生有皇子即承继穆宗教皇帝为嗣。现在大行皇帝龙驭上宾,亦未有储贰,不得已以摄政王载沣之子溥仪承继穆宗毅皇帝为嗣并兼承大行皇帝之挑。钦此。\\

又面承懿旨:现在时势多艰,嗣皇帝尚在冲龄,正宜专心典学,著摄政王载沣为监国,所有军国政事,悉秉予之训示裁度施行,俟嗣皇帝年岁渐长,学业有成,再由嗣皇帝亲裁政事,钦此。是日住于西苑军机处。\\
\end{quote}

这段日记,我从西太后宣布自己的决定的头一天,即十九日抄起,是因为十九日那句“致庆邸急函”和二十日的“庆王到京”四个字,与立嗣大有关系。这是西太后为了宣布这个决定所做的必要安排之一。为了说清楚这件事,不得不从远处说起。\\

庆王就是以办理卖国外交和卖官鬻爵而出名的奕劻\footnote{奕劻是乾隆第十七子庆值亲王永璘之孙。初袭辅国将军,咸丰二年封贝子,十年封贝勒,同治十一年加郡王衔,光绪十年总理各国事务衙门,并封庆郡王,二十年封亲王。}。在西太后时代,能得到太后欢心就等于得到了远大前程。要想讨西太后的欢心,首先必须能随时摸得着太后的心意,才能做到投其所好。荣禄贿赂太监李莲英,让太太陪伴太后游乐,得到不少最好最快的情报,因此他的奉承和孝敬,总比别人更让太后称心满意。如果说奕劻的办法和他有什么不同的话,那就是奕劻在李莲英那里花了更多的银子,而奕劻的女儿即著名的四格格\footnote{格格是清代皇族女儿的统一称呼,皇帝的女儿封公主称固伦格格,亲王女儿郡主称和硕格格,郡王女儿封县主称多罗格格,贝勒女儿封郡君亦称多罗格格,贝子女儿封县君称固山格格,镇国公、辅国公女儿封乡君称格格。格格又有汉族“小姐”之意,故旗人家女儿也叫格格。}也比荣禄太太更机灵。如果西太后无意中露出了她喜欢什么样的坎肩,或者嵌镶着什么饰品的鞋子,那么不出三天,那个正合心意的坎肩、鞋子之类的玩艺就会出现在西太后的面前。奕劻的官运就是从这里开始的。在西太后的赏识下,奕劻一再加官晋爵,以一个远支宗室的最低的爵位辅国将军,逐步进到亲王,官职做到总理各国事务衙门。他得到了这个左右逢源的差使,身价就更加不同,无论在太后眼里和洋人的眼里,都有了特殊的地位。辛丑议和是他一生中最重要的事件。在这一事件中,他既为西太后尽了力,使她躲开了祸首的名义,也让八国联军在条约上满了意。当时人们议论起王公们的政治本钱时,说某王公有德国后台,某王公有日本后台……都只不过各有一国后台而已,一说到庆王,都认为他的后台谁也不能比,计有八国之多。因此西太后从那以后非常看重他。光绪二十九年,他进入了军机处,权力超过了其他军机大臣,年老的礼亲王的领衔不过是挂个虚名。后来礼王告退,奕劻正式成了领衔军机大臣,他儿子载振也当了商部尚书,父子显赫不可一世。尽管有反对他的王公们暗中搬他,御史们出面参他贪赃枉法,卖官鬻爵,都无济于事,奈何他不得。有位御史弹劾他“自任军机,门庭若市,细大不捐,其父子起居饮食车马衣服异常挥霍,……将私产一百二十万两送往东交民巷英商汇丰银行存储”,有位御史奏称有人送他寿礼十万两,花一万二千两买了一名歌妓送给他儿子。结果是一个御史被斥回原衙门,一个御史被夺了官。\\

西太后对奕劻是否就很满意?根据不少遗老们侧面透露的材料,只能这样说:西太后后来对于奕劻是又担心又依赖,所以既动不得他,并且还要笼络他。\\

使西太后担心的,不是贪污纳贿,而是袁世凯和奕劻的特殊关系。单从袁在奕劻身上花钱的情形来看,那关系就很不平常。袁世凯的心腹朋友徐世昌后来说过:庆王府里无论是生了孩子,死了人,或是过个生日等等,全由直隶总督衙门代为开销。奕劻正式领军机处之前不久,有一天庆王府收到袁家送来十万两(一说二十万两)白银,来人传述袁的话说:“王爷就要有不少开销,请王爷别不赏脸。”过了不久,奕劻升官的消息发表了,人们大为惊讶袁世凯的未卜先知。\\

戊戌政变后,西太后对袁世凯一方面是十分重视的,几年功夫把他由直隶按察使提到直隶总督、外务部尚书,恩遇之隆,汉族大臣中过去只有曾、胡、左、李才数得上。另一方面,西太后对这个统率着北洋新军并且善于投机的汉族大臣,并不放心。当她听说袁世凯向贪财如命的庆王那里大量地送银子时,就警惕起来了。\\

西太后曾经打过主意,要先把奕劻开缺。她和军机大臣瞿鸿囗露出了这个意思,谁知这位进士出身后起的军机,太没阅历,竟把这件事告诉了太太。这位太太有位亲戚在一家外文报馆做事,于是这个消息便辗转传到了外国记者的耳朵里,北京还没有别人知道,伦敦报纸上就登出来了。英国驻北京的公使据此去找外务部,讯问有无此事。西太后不但不敢承认,而且派铁良和鹿传霖追查,结果,瞿鸿囗被革了职。\\

西太后倒奕劻不成,同时因奕劻有联络外国人的用途,所以也就不再动他,但对于袁世凯,她没有再犹豫。光绪三十三年,内调袁为外务部尚书,参加军机。明是重用,实际是解除了他的兵权。袁世凯心里有数,不等招呼,即主动交出了北洋新军的最高统帅权。\\

西太后明白,袁对北洋军的实际控制能力,并非立时就可以解除,袁和奕劻的关系也不能马上斩断。正在筹划着下一个步骤的时候,她自己病倒了,这时又忽然听到这个惊人消息:袁世凯准备废掉光绪,推戴奕劻的儿子载振为皇帝。不管奕劻如何会办外交和会奉承,不管袁世凯过去对她立过多大的功,也不管他们这次动手的目标正是被她痛恨的光绪,这个以袁世凯为主角的阴谋,使她马上意识到了一种可怕的厄运——既是爱新觉罗皇朝的厄运,也是她个人的厄运。因此她断然地做出了一项决定。为了实现这个决定,她先把奕劻调开,让他去东陵查看工程,然后把北洋军段祺瑞\footnote{段祺瑞(1864—1936),字芝泉,安徽合肥人,是袁世凯创办的北洋军的得力将领。在民国后成为北洋军阀皖系首领。袁世凯死后,在日本帝国主义支持下数度把持北京政权,是日本帝国主义的忠实走狗。一九三一年“九·一八”后又企图在日本帝国主义支持下组织华北汉奸政权,旋被抛弃,不久被蒋介石软禁在上海,一直到死。}的第六镇全部调出北京,开往深水,把陆军部尚书铁良统辖的第一镇调进来接防。等到奕劻回来,这里一切大事已定:慈禧宣布了立我为嗣,封我父亲为摄政王。但是为了继续笼络住这位有八国朋友的庆王,给了他亲王世袭罔替的思荣。\\

关于袁、庆的阴谋究竟确不确,阴谋的具体内容又是什么,我说不清。但是我有一位亲戚亲自听铁良事后说起过西太后的这次安排。铁良说,为了稳定段祺瑞的第六镇北洋军,开拔之先发给了每名士兵二两银子,一套新装和两双新鞋。另外,我还听见一个叫李长安的老太监说起光绪之死的疑案。照他说,光绪在死的前一天还是好好的,只是因为用了一剂药就坏了,后来才知道这剂药是袁世凯使人送来的。按照常例,皇帝得病,每天太医开的药方都要分抄给内务府大臣们每人一份,如果是重病还要抄给每位军机大臣一份。据内务府某大臣的一位后人告诉我,光绪死前不过是一般的感冒,他看过那些药方,脉案极为平常,加之有人前一天还看到他像好人一样,站在屋里说话,所以当人们听到光绪病重的消息时都很惊异。更奇怪的是,病重消息传出不过两个时辰,就听说已经“晏驾”了。总之光绪是死得很可疑的。如果太监李长安的说法确实的话,那么更印证了袁庆确曾有过一个阴谋,而且是相当周密的阴谋。\\

还有一种传说,是西太后自知病将不起,她不甘心死在光绪前面,所以下了毒手。这也是可能的。但是我更相信的是她在宣布我为嗣皇帝的那天,还不认为自己会一病不起。光绪死后两个小时,她还授命监国摄政王:“所有军国政事,悉秉承予之训示裁度施行。”到次日,才说:“现予病势危笃,恐将不起,嗣后军国政事均由摄政王裁定,遇有重大事件有必须请皇太后(指光绪的皇后,她的侄女那拉氏)懿旨者,由摄政王随时面请施行。”她之所以在发现了来自袁世凯那里的危险之后,或者她在确定了光绪的最后命运之后,从宗室中单单挑选了这样的一个摄政王和这样一个嗣皇帝,也正是由于当时她还不认为自己会死得这么快。在她来说当了太皇太后固然不便再替皇帝听政,但是在她与小皇帝之间有个听话的摄政王,一样可以为所欲为。\\

当然,她也不会认为自己老活下去。在她看来,她这个决定总算为保全爱新觉罗的宝座而尽了力。她甚至会认为,这个决定之正确,就在于她选定的摄政王是光绪的亲兄弟。因为按常情说,只有这样的人,才不至于上袁世凯的当。\\