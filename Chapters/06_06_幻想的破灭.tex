\fancyhead[LO]{{\scriptsize 1932-1945: 伪满十四年 · 幻想的破灭}} %奇數頁眉的左邊
\fancyhead[RO]{} %奇數頁眉的右邊
\fancyhead[LE]{} %偶數頁眉的左邊
\fancyhead[RE]{{\scriptsize 1932-1945: 伪满十四年 · 幻想的破灭}} %偶數頁眉的右邊
\chapter*{幻想的破灭}
\addcontentsline{toc}{chapter}{\hspace{1cm}幻想的破灭}
\thispagestyle{empty}
日本自一九三三年初退出国际联盟之后,更加肆无忌惮地进行扩军备战,特别是加紧了全面侵华的部署和后方的准备。在“七·七”事变之前,日本在华北连续使用武力和制造事变,国民党南京政府步步屈服,签订了出让华北控制权的“何(应钦)梅(津)协定”、“秦(德纯)土(肥原)协定”等密约,听任“冀东防共自治政府”、“内蒙自治军政府”等等伪组织的存在和活动,再三地向日本表白“不但无排日之行动与思想,亦本无排日必要的理由”,并且对国人颁布了“效睦邻邦命令”,重申抗日者必严惩之禁令。这样,日本在关内的势力有了极大的加强,人人可以看出,只要时间一到,五省即可彻底变色。我在前面说过,这正是关内关外复辟迷们跃跃欲试的时候,正是我第三次“登极”前后得意忘形的时候。然而,日本在张牙舞爪于关内的同时,它在“满洲国”内也正采取着步步加紧的措施,这些措施终于临到我这“皇帝”的头上。\\

在东北彻底殖民地化的过程中,公平地说,汉奸们是得到不少便宜的。例如改帝制,这个措施不仅使复辟迷们得到了一定心理满足,而也成了一次发财的机缘,自\xpinyin*{郑孝胥}以下的大汉奸都得到一笔自五万至六十万不等的“建国功劳金”,总数共为八百六十万元(以后每逢一次大规模的掠夺,如“粮谷出荷”、“献金报国”等等,必有一次“奖金”分给上自“总理大臣”下至保甲长)。我现在不想对日本的各种措施做全面的叙述,只把我恢复祖业思想的幻灭以及深感恐惧的事情说一说。\\

按情理说,日本关东军在决定帝制时正式告诉我不是恢复清朝,在“登极”时不准我穿龙袍,在决定“总理大臣”人选时根本不理睬我的意见,我就该明白了我的“尊严”的虚假性,但是我却由于过分“陶醉”,竟没有因此而清醒过来。使我开始感到幻灭滋味的,还是“\xpinyin*{凌升}事件”。\\

\xpinyin*{凌升}是清末蒙古都统\xpinyin*{贵福}之子,原为\xpinyin*{张作霖}东三省保安总司令部和蒙古宣抚使署顾问。他是在旅顺的“请愿代表”之一,因此被列入“建国元勋”之内。事件发生时他是伪满兴安省省长。一九三六年春天,他突然遭到了关东军的拘捕。拘捕的原因,据关东军派来的\ruby{吉冈}{よしおか}\ruby{安直}{やすなお}说,他有反满抗日活动,但是据\xpinyin*{佟济煦}听来的消息,却是他在最近一次省长联席会上发过牢骚,以致惹恼了日本人。据说他在这次会上,抱怨日本关东军言行不一,说他在旅顺时曾亲耳听\ruby{板垣}{いたがき}说过,日本将承认“满洲国”是个独立国,可是后来事实上处处受关东军干预,他在兴安省无权无职,一切都是日本人做主。开过这个会,他回到本省就被抓去了。我听到这些消息,感到非常不安,因为半年前我刚刚与他结为亲家,我的四妹与他的儿子订了婚。我正在犹豫着,是不是要找关东军说说情的时候,新任的司令官兼第四任驻“满”大使\ruby{植田}{うえだ}\ruby{谦吉}{けんきち}先找我来了。\\

“前几天破获了一起案件,罪犯是皇帝陛下认得的,兴安省省长\xpinyin*{凌升}。他勾结外国图谋叛变,反对日本。军事法庭已经查实他的反满抗日罪行,宣判了死刑。”\\

“死刑?”我吃了一惊。\\

“死刑。”他向他的翻译点头重复一遍,意思是向我说清楚。然后又对我说:“这是杀一儆百,陛下,杀一儆百是必需的!”\\

他走后,关东军\ruby{吉冈}{よしおか}\ruby{安直}{やすなお}参谋又通知我,应该立刻跟\xpinyin*{凌升}的儿子解除四妹的婚约。我连忙照办了。\\

\xpinyin*{凌升}被处决时,使用的是斩首之刑。一同受刑的还有他的几个亲属。这是我所知道的第一个被日本人杀害的显要官员,而且还是刚跟我做了亲家的。我从\xpinyin*{凌升}跟我攀亲的举动上,深信他是最崇拜我的,也是最忠心于我的人,而关东军衡量每个人的惟一标准却是对日本的态度。不用说,也是用这统一标准来看待我的。想到这里,我越发感到\ruby{植田}{うえだ}“杀一儆百”这句话的阴森可怕。\\

我由此联想到不久前的一件事。一九三五年末,有一些人为图谋复辟清朝而奔波于关内关外,如\xpinyin*{康有为}的徒弟任祖安,我从前的奏事官吴天培等,引起了关东军的注意。关东军曾就此向我调查。“\xpinyin*{凌升}事件”提醒了我,日本人是不喜欢这类事的,还是要多加小心为是。\\

日本人喜欢什么?我自然地联想到一个与\xpinyin*{凌升}命运完全不同的人,这就是\xpinyin*{张景惠}。这实在是日本人有意给我们这伙人看的两个“榜样”。一福一祸,对比鲜明。\xpinyin*{张景惠}之所以能得日本人的欢心,代替了\xpinyin*{郑孝胥},是有他一套功夫的。这位“胡子”出身的“总理大臣”的为人,和他得到日本人的赏识,可以从日本人传诵他的“警句”上知道。有一次总务厅长官在国务会议上讲“日满一心一德”的鬼道理,作为日本掠夺工矿原料行为的“道义”根据,临末了,请“总理大臣”说几句。\xpinyin*{张景惠}说:“咱是不识字的大老粗,就说句粗话吧:日满两国是两只蚂冷(蜻蜓)拴在一根绳上。”这“两只蚂冷一根绳”便被日本人传诵一时,成为教训“满”籍官员的“警句”。日本在东北实行“拓殖移民”政策的时候,在“国务会议”上要通过法案,规定按地价四分之一或五分之一的代价强购东北农田,有些“大臣”如韩云阶等一则害怕造成“民变”,另则自己拥有大量土地,不愿吃亏,因此表示了反对。这时\xpinyin*{张景惠}却出来说话了:“满洲国土地多的不得了,满洲人是老粗,没知识,让日本人来开荒教给新技术,两头都便宜。”提案就此通过了。“两头便宜”这句话于是又被日本人经常引用着。后来,“粮谷出荷”加紧推行,东北农民每季粮食被征购\xpinyin*{殆}尽,有些“大臣”们因为征购价过低,直接损害到他们的利益,在“国务会议”上借口农民闹饥荒,吵着要求提高收购价格。日本人自然又是不干,\xpinyin*{张景惠}于是对大家说:“日本皇军卖命,我们满洲出粮,不算什么。闹饥荒的勒一下裤腰带,就过去了。”“勒腰带”又成了日本人最爱说的一句话,当然,不是对他们自己说的。关东军司令官不断地对我称赞\xpinyin*{张景惠}为“好宰相”,是“日满亲善身体力行者”。我当时很少想到这对我有什么意义,现在有了\xpinyin*{凌升}的榜样,在两者对比之下,我便懂得了。\\

“\xpinyin*{凌升}事体”过去了,我和德王的一次会见造成了我更大的不安。\\

德王即由日本操纵成立了“内蒙自治军政府”伪组织的德穆楚克栋鲁普。他原是一个蒙古王公。我在天津时,他曾送钱给我,送良种蒙古马给\xpinyin*{溥杰},多方向我表示过忠诚。他这次是有事找关东军,乘机取得关东军司令官的允许,前来看望我的。他对我谈起这几年的经历和成立“自治军政府”的情形,不知不觉地发开了牢骚,埋怨他那里的日本人过分跋扈,说关东军事先向他许了很多愿,到头来一样也不实现。尤其使他感到苦恼的是自己样样不能做主。他的话勾起了我的牢骚,不免同病相怜,安慰了他一番。不想第二天,关东军派到我这里专任联络的参谋,即以后我要谈到的“帝室御用挂”\ruby{吉冈}{よしおか}\ruby{安直}{やすなお},走来板着脸问我:\\

“陛下昨天和德王谈了些什么?”\\

我觉得有些不妙,就推说不过是闲聊而已。\\

他不放松我,追问道:“昨天的谈话,对日本人表示不满了没有?”\\

我心里砰砰跳了起来。我知道惟一的办法就是坚不承认,而更好的办法则是以进为退,便说:“那一定是德王故意编排出什么假话来了吧?”\\

\ruby{吉冈}{よしおか}虽然再没穷追下去,我却一连几天心惊肉跳,疑虑丛生。我考虑这件事只有两个可能,不是日本人在我屋里安上了什么偷听的机器,就是德王在日本人面前说出了真话。我为了解开这个疑团,费了好大功夫,在屋里寻找那个可能有的机器。我没有找到什么机器,又怀疑是德王成心出卖我,可是也没有什么根据。这两种可能都不能断定,也不能否定,于是都成了我的新魔障。\\

这件事发生之后,我懂得的事就比“\xpinyin*{凌升}事件”告诉我的更多了。我再不跟任何外来人说真心话,我对每位客人都有了戒心。事实上,自从我访日回来发表讲演之后,主动来见的人即逐渐减少,到德王会见之后,更近于绝迹。到了一九三七年,关东军更想出了一个新规矩,即每逢我接见外人,须由“帝室御用挂”在旁侍立。\\

进入了一九三七年,我一天比一天感到紧张。\\

在“七·七”事变前这半年间,日本加紧了准备工作。为了巩固它的后方基地的统治,对东北人民的抗日爱国活动,进行了全面的镇压。一月四日,以“满洲国皇帝\xpinyin*{敕}令”颁行了“满洲帝国刑法”,接着便开始了“大检举”、“大讨伐”,实行了“保甲连坐法”,“强化协和会”,修“警备道”,建“碉堡”,归屯并村。日本这次调来大量队伍,用大约二十个日本师团的兵力来对付拥有四万五千余人的抗日联军。与此同时,各地大肆搜捕抗日救国会会员,搜捕一切被认做“不稳”的人。这一场“大检举”与“大讨伐”,效果并不理想,关东军司令官向我夸耀了“皇军”威力和“赫赫战果”之后不到一年,又以更大的规模调兵遣将(后来知道是七十万日军和三十万伪军),举行了新“讨伐”,同时据我的亲信、警卫处长\xpinyin*{佟济煦}告诉我,各地经常有人失踪,好像反满抗日的分子老也抓不完。\\

我从关东军司令官的谈话中,从“总理大臣”的例行报告中,向来是听不到什么真消息的,只有\xpinyin*{佟济煦}还可以告诉我一些。他曾经告诉过我,关东军司令官对我谈的“讨伐”胜利消息,不一定可靠,消灭的“土匪”也很难说是什么人。他说,他有个被抓去当劳工的亲戚,参加修筑过一件秘密工程,据这个亲戚说,这项工程完工后,劳工几乎全部遭到杀害,只有他和少数几个人幸免于难,逃了出来。照他看来,报纸上有一次吹嘘某地消灭了多少“土匪”,说的就是那批劳工。\\

\xpinyin*{佟济煦}的故事说过不久,给我当过英文翻译的吴沆业失踪了。有一天\xpinyin*{溥杰}来告诉我,吴是因为在驻东京大使馆时期与美国人有来往被捕的,现在已死在宪兵队。还说,吴死前曾托看守带信给他,求他转请我说情,但他当时没有敢告诉我。我听了,赶紧叫他不要再说下去。\\

在这段时间里,我经手“裁可”的政策法令,其中有许多关于日本加紧备战和加强控制这块殖民地的措施,但无论是“第一五年开发产业计划”,还是“产业统制法”,也无论是为适应进一步控制需要而进行的“政府机构大改组”,还是规定日本语为“国语”,都没有比\xpinyin*{溥杰}的结婚更使我感到刺激的。\\

\xpinyin*{溥杰}在日本学习院毕业后,就转到士官学校学陆军。一九三五年冬他从日本回到长春,当了禁卫军中尉,从这时起,关东军里的熟人就经常向他谈论婚\xpinyin*{姻}问题,什么男人必须有女人服侍啦,什么日本女人是世界上最理想的妻子啦,不断地向他耳朵里灌。起初,我听他提到这些事时不过付之一笑,并没拿它当回事。不料后来关东军派到我身边来的\ruby{吉冈}{よしおか}\ruby{安直}{やすなお}果真向我透露了关东军的意思,说为了促进日满亲善,希望\xpinyin*{溥杰}能与日本女人结婚。我当时未置可否,心里却十分不安,赶忙找我的二妹一起商量对策。我们一致认为,这一定是一项阴谋,日本人想要笼络住\xpinyin*{溥杰},想要一个日本血统的孩子,必要时取我而代之。为了打消关东军的念头,我们决定赶快动手,抢先给\xpinyin*{溥杰}办亲事。我把\xpinyin*{溥杰}找来,先进行了一番训导,警告他如果家里有了个日本老婆,自己就会完全处于日本人监视之下,那是后患无穷的,然后告诉他我一定要给他找一个好妻子,他应该听我的话,不要想什么日本女人。\xpinyin*{溥杰}恭恭敬敬地答应了,我便派人到北京去给他说亲。后来经我岳父家的人在北京找到一位对象,\xpinyin*{溥杰}也表示满意,可是\ruby{吉冈}{よしおか}突然找到\xpinyin*{溥杰},横加干涉地说,关东军希望他跟日本女子结婚,以增进“日满亲善”,他既身为“御弟”,自应做出“亲善”表率,这是军方的意思,\ruby{本庄}{ほんじょう}大将在东京将要亲自为他做媒,因此他不可再去接受北京的亲事,应该等着东京方面的消息。结果,\xpinyin*{溥杰}只得服从了关东军。\\

一九三七年四月三日,\xpinyin*{溥杰}与嵯峨胜侯爵的女儿嵯峨浩在东京结了婚。过了不到一个月,在关东军的授意下,“国务院”便通过了一个“帝位继承法”,明文规定:皇帝死后由子继之,如无子则由孙继之,如无子无孙则由弟继之,如无弟则由弟之子继之。\\

\xpinyin*{溥杰}和他的妻子回东北后,我拿定了一个主意:不在\xpinyin*{溥杰}面前说出任何心里话,\xpinyin*{溥杰}的妻子给我送来的食物我一口也不吃。假若\xpinyin*{溥杰}和我一起吃饭,食桌上摆着他妻子做的菜,我必定等他先下春之后才略动一点。\\

后来,\xpinyin*{溥杰}快要做父亲的时候,我曾提心吊胆地为自己的前途算过卦,我甚至也为我的弟弟担忧。我相信那个帝位继承法,前面的几条都是靠不住的,靠得住的只是“弟之子继之”这句话。关东军要的是一个日本血统的皇帝,因此我们兄弟两个都可能做牺牲品。后来听说他得的是个女儿,我这才松了一口气。\\

当时我曾想过,假若我自己有了儿子,是不是会安全?想的结果是,即使真的有了儿子,也不见得对我有什么好处,因为关东军早叫我写下了字据:若有皇子出生,五岁时就必须送到日本,由关东军派人教养。\\

可怕的事情并没有就此终结。六月二十八日,即“七·七”事变九天前,又发生了一起有关“护军”的事件。\\

所谓护军,是我自己出钱养的队伍,它不同于归“军政部”建制的“禁卫军”。我当初建立它,不单是为了保护自己,而是跟我当初送\xpinyin*{溥杰}他们去日本学陆军的动机一样,想借此培养我自己的军事骨干,为建立自己所掌握的军队做准备。我这支三百人的队伍全部都是按照军官标准来训练的。负责管理护军的\xpinyin*{佟济煦}早就告诉过我,关东军对这支队伍是不喜欢的。我对\xpinyin*{佟济煦}的预感,过去一直未能理解,直到出了事情这才明白。六月二十八日那天,一部分护军到公园去游玩,因租借游艇,与几个穿便衣的日本人发生了口角。这时一群日本人一拥而上,不容分说,举手就打。他们被逼急了,便使出武术来抵抗。日本人见不能奈何他们,就放出狼狗来咬。他们踢死狼狗,冲出重围,逃回队里。他们没想到,这一来便闯下了祸。过了不大时间,宫内府外边便来了一些日本宪兵,叫\xpinyin*{佟济煦}把今天去公园的护军全部交出来。\xpinyin*{佟济煦}吓得要命,忙把那些护军交日本宪兵带走。日本宪兵逼他们承认有“反满抗日”活动,那些护军不肯承认,于是便遭到了各种酷刑虐待。到这时那些护军才明白过来,这一事件是关东军有意制造的:那些穿便衣的日本人原是关东军派去的,在双方斗殴中受伤者有两名关东军参谋,被踢死的狼狗即关东军的军犬。我听到护军们被捕,原以为是他们无意肇祸,忙请\ruby{吉冈}{よしおか}\ruby{安直}{やすなお}代为向关东军说情。\ruby{吉冈}{よしおか}去了一趟,带口来关东军参谋长\ruby{东条}{とうじょう}\ruby{英机}{ひでき}的三个条件,即:一、由管理护军的\xpinyin*{佟济煦}向受伤的关东军参谋赔礼道歉;二、将肇事的护军驱逐出境;三、保证以后永不发生同类事件。我按照东条的条件—一照办之后,关东军接着又逼我把警卫处长\xpinyin*{佟济煦}革职,由日本人\ruby{长尾}{ながお}吉五郎接任,把警卫处所辖的护军编制缩小,长武器一律换上了短枪。\\

从前,我为了建立自己的实力,曾送过几批青年到日本去学陆军,不想这些人回来之后,连\xpinyin*{溥杰}在内,都由军政部派了差,根本不受我的支配。现在,作为骨干培养的护军已完全掌握在日本人手里,我便不再做这类可笑的美梦了。\\

“七·七”事变爆发,日军占领了北京之后,北京的某些王公。遗老曾一度跃跃欲试,等着恢复旧日冠盖,但是我这时已经明白,这是决不可能的了。我这时的惟一的思想,就是如何在日本人面前保住安全,如何应付好关东军的化身——帝室御用挂\ruby{吉冈}{よしおか}\ruby{安直}{やすなお}。
