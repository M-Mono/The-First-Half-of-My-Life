\fancyhead[LO]{{\scriptsize 1950-1954: 由抗拒到认罪 · 写自传与献“宝”}} %奇數頁眉的左邊
\fancyhead[RO]{} %奇數頁眉的右邊
\fancyhead[LE]{} %偶數頁眉的左邊
\fancyhead[RE]{{\scriptsize 1950-1954: 由抗拒到认罪 · 写自传与献“宝”}} %偶數頁眉的右邊
\chapter*{写自传与献“宝”}
\addcontentsline{toc}{chapter}{\hspace{1cm}写自传与献“宝”}
\thispagestyle{empty}
我认为写自传是审判的前奏。既然要审判,那就是说生死尚未定局,在这上面我要力争一条活路。\\

对于应付审判,我早有了既定的打算。刚到哈尔滨那天,我们走下汽车,还没进入监房,这时侄子小固凑近我,在我耳边悄悄地说:“问起来,还是在苏联那套说法!”我略略点了一下头。\\

所谓在苏联的那套说法,就是隐瞒我投敌的行径,把自己说成是一个完全善良无辜的、爱国爱民的人。我明白现在的处境与在苏联时不同,我必须编造得更加严密,决不能有一点点漏洞。\\

小固那天的话,是代表同他住在一起的侄子们和随侍大李的。那几句话说明了他们早已有了准备,同时也说明了他们对我的忠心,一如往昔。不过要想不出漏洞,光是忠心还不够,我觉得还必须再嘱咐一下。特别是要嘱咐一下大李,因为他是我的自传中最关键的部分——我从天津怎样到的东北——的实际见证人。我从静圆溜走前,事先他给我准备的行李衣物,我钻进汽车的后箱后,是他给我盖的箱盖。这些事一旦被泄漏出去,那个\ruby{土肥原}{\textcolor{PinYinColor}{どいはら}}强力绑架的故事就不会有人相信了。\\

这件事只能在休息时间,利用我和我的家族合法的见面机会去办。这时情况与以前已经有些不同了,一些年纪较轻的犯人开始干起杂活,如挑水、送饭、帮厨之类。我的家族除了\ruby{荣源}{\textcolor{PinYinColor}{Žung Yuwan}}这时已死,黄医生因风湿性关节炎经常休息外,其余都参加了这种服务性的劳动。我在休息时间,不大容易全看到他们,不是这个在帮厨,就是那个在送开水。不过,也有个好处,这就是他们行动比较自由,可以为我传话找人。我就是利用这种便利让小瑞把大李给我悄悄找来的。\\

大李来了,恭顺地走近了我,带着听候吩咐的样儿。我压低嗓音问他:\\

“你还记得从天津搬家的事吗?”\\

“是说到关外吧?是我收拾的东西,是吧?”\\

“如果所方问起我是怎么从天津走的,你就说全不知道。你收拾东西,是在我走后,知道吗?”\\

“走后?”\\

“对啦,走后,你是听了\xpinyin*{胡嗣瑗}的吩咐,把我用的衣物行车送到旅顺的。”\\

大李点点头,表示心领神会,悄悄走了。\\

第二天,小瑞在院子里告诉我,大李请他转报,昨天晚上他和所方贾科员谈天,他告诉贾科员我在东北时待底下人很厚道,从不打人骂人。又说我在旅顺时,成天锁门,不见日本人。我听了这话,觉着这个大李做得太过分了,为什么提旅顺的事呢!我叫小瑞告诉他:别多嘴,如果问起旅顺的情形,就说什么也不知道。\\

我对大李的忠诚很满意。我对重要的问题有了把握,又向侄子们分别嘱咐过了,这才动手写起我的自传。在这份自传里,我写下了我的家世,写下了西太后如何让我做了皇帝,我在紫禁城如何度过了童年,我如何“完全不得已”地躲进了日本公使馆,我如何在天津过着“与世无争”的生活,然后是按外界传说写成的“绑架”和“不幸的”长春岁月。记得我在最后是这样结束的:\\

\begin{quote}
	我看到人民这样受苦受难,自己没一点办法,心中十分悲忿。我希望中国军队能打过来,也希望国际上发生变化,使东北得到解救。这个希望,终于在一九四五年实现了。\\
\end{quote}

这份自传经过再三地推敲和修改,最后用恭楷缮清,送了上去。从这篇文字上我相信任何人都可以看出,我是个十分悔罪的人。\\

送出自传之后,我又想,仅仅这篇文字还不够,还必须想个办法让政府方面相信我的“诚实”和“进步”才行。怎么办呢?依靠大李他们替我吹嘘吗?这显然不够,最重要的是我自己还必须有实际上的成绩。\\

一想到成绩,我不禁有些泄气。自从回国以来,即使火车上的那段不算,抚顺的那段也不算,单说自从到了哈尔滨,我参加了监房内的值日以来,那成绩就连我自己也不满意,更不用说所方了。\\

原来犯人们自从听了公安机关的首长和所长的讲话之后,每个人都在设法证明自己有了“觉悟”,都把所谓的“觉悟”看做活命的手段。现在回想起来,感到非常可笑,人们当时竟把事情看得那么简单:好像只要作假做得好,就可以骗得过政府。在我存有这种妄想的时候,最使我引为悲哀的,就是我处处不如别人。\\

当时大家都从学习、值日和生活这三方面,努力表现自己,希图取信所方。我们这个组,在学习方面“成绩”最好的要算我们的组长老王。他原是伪满军法少将,在北平学过几年法政,文化程度比较高,对新理论名词懂得比较快。其他三名“将官”起初跟我一样,连“主观”“客观”都闹不清,可是“进步”也比我快。在开讨论会时,他们都能说一套。最要命的是学完“什么叫封建社会”的专题后,每人要写一篇学习心得(或称学习总结),把自己对这个问题的领会、感想,用自己的话说出来。在讨论时,我还可以简单地说一说,知道多少说多少,写心得可就不这么容易了。老实说,这时我对于学习还没感到有什么需要,学习对于我,非但没解决什么认识上的问题,反而让我对于书上关于封建社会的解释感到害怕。例如,封建帝王是地主头子,是最大的地主,这些话都像是对我下判决似的。如果我是最大的地主,那么不但从叛国投敌上说该法办,而且从土地改革的角度上说也赦不了,那不是更没活路了吗?我在这种不安的情绪中,简直连一个字也写不下去。在我勉强安下心东抄西凑地写完这篇心得后,又看了看别人写的,觉得我的学习成绩是决不会使所方满意的。\\

到哈尔滨后,我自动地参加了值日,这是惟一可以证明“进步”的地方。在这里,所方再没有人宣布我“有病”,而我也发现这里每间屋的屋角上都有抽水马桶,没有提马桶这个难题了。值日工作只是接递外面送来的三顿饭、开水和擦地铺,我不再感到怵头,当轮到我的时候,就动手干起来了。我有生以来第一次为别人服务,就出了一个岔子,在端饭菜的时候,几乎把一碗菜汤全洒在人家头上。因此,以后每逢轮到我,总有人自动帮忙。他们一半是好意,一半也是不甘再冒菜汤浇顶的危险。\\

生活上的情形,就更不能跟别人比了。我的服装依旧不整洁,我的衣服依旧靠小瑞给我洗缝。自从所长当众指出我的邋里邋遏以后,我心里总有一种混杂着羞耻和怨恨的感情。我曾试着练习照顾自己,给自己洗衣服,可是当我弄得满身是水,仍然制服不了肥皂和搓板的时候,心中便充满了怨气;而当我站在院里等待小瑞,别人的目光投向我手中待洗的衣袜时,我又感到羞耻。\\

交上自传不久,我忽然下定决心,再试一次。我觉得这件事再困难也要干,否则所方看我一点出息都没有,还怎么相信我呢?我以满头大汗的代价,洗好了一件白衬衣。等晾干了一看,白衬衣变成了花衬衣,好像八大山人的水墨画。我对着它发了一阵呆,小瑞过来,把“水墨画”从晾衣绳上拉下来,夹在怀里悄悄地说:“这不是上头干的事,还是给瑞干吧。”\\

他的话很顺耳,——我边散步边思索着,不错,这不是我干的,而且也干不好。可是,我不干这个,干什么才能向所方表现一下自己呢?我必须找一件可以干、而且干得出色的事情才行。\\

我正苦苦地思索着,忽然旁边几个人的议论引起了我的注意。\\

这是我五妹夫老万那屋里的几个。他们正谈论着关于各界人民捐献飞机大炮支援志愿军的事。那时按规定,不同监房的人不得交谈,但听别人的谈话并不禁止。那堆人里有个姓张的前伪满大臣,在抚顺时曾跟我同过屋,他有个儿子从小不肯随他住在伪满,反对他这个汉奸父亲,连他的钱也不要。他现在估计这个儿子一定参加了抗美援朝。他每提起儿子,总是流露出不安的心情,现在又是如此。\\

“如果政府还没有没收我的财产,我要全部捐献给抗美援朝。我儿子既然不要,我只好这样。”\\

有人笑道:“这岂不是笑话!我们的财产本来就该没收的。”\\

“那怎么办呢?”老张愁眉苦脸地说,“也许我那孩子就在朝鲜拚命呢!”\\

“你想的太多,毫无根据。”另一个说,“你以为汉奸的儿女可以参军吗?”\\

这句话别人听了显然不是味儿,一时都不再做声,可是老张还想他的主意:\\

“咱们随身带的财物,政府并没充公,是代为保存的。我把它捐出去好不好!”\\

“那有多一点?”又有人笑他,“除了皇上和总理大臣,谁的东西都值不了多少钱!……”\\

这句话把我提醒了。不错,我还有许多珠宝首饰呢,这可是任何人都无法跟我较量的。不说藏在箱子底的那些,就说露在外面的一点也是很值钱的。其中那套\xpinyin*{乾隆}皇帝当太上皇时用的“宝”,就是无价之宝。这是用田黄石刻的三颗印,由三条田黄石链条连结在一起,雕工极为精美。我不想动用藏在箱底的财宝,决定把这三颗印拿出来以证明我的“觉悟”。\\

决定了就赶快做。我记得从前有一次,所方人员在岗台上宣布志愿军取得第五次战役胜利的消息时,不知是哪个犯人听完之后立刻向干部要求到朝鲜去参战,接着有好些人都提出这个要求,还有人立时扯本子写申请书。当然,所方没有接受。我后来不免有些嫉妒地想:这些人既表现了“觉悟”,又实际担不上什么风险,心眼真是不少。我想起那回事,决定这回不能落后于人,不要让他们抢先办了,显得我是跟着学的。正好,这天政府负责人员来巡视,我透过栏杆,看出来人正是在沈阳叫我不要紧张的那位。根据所长陪伴的形势,我断定他必是所长的上级,虽然他并没穿军装。我觉得向这样人拿出我的贡品,是效果更好的。等他巡视到我们监房跟前的时候,我向他深鞠一躬,说道:\\

“请示首长先生,我有件东西,想献给人民政府……”\\

我拿出了\xpinyin*{乾隆}的日黄石印给他,他却不接过去,只点点头:\\

“你是\ruby{溥仪}{\textcolor{PinYinColor}{\Man ᡦᡠ ᡳ}}吧?好,这件事你跟所方谈吧。”\\

他又问了几句别的话,就走开了。我想,他如果看到我的东西,知道它的价值,就不会如此冷淡了。没有办法,我只好找所方办这件事。我写了一封信,连同那套石印,交给看守员请他转送给所长。\\

这套田黄石印送出之后,犹如石沉大海,一连多日没有消息。我不禁起了疑心,是不是看守员偷着匿起来了呢?\\

我犯了老毛病,疑心什么就相信是什么。这天晚上,别人下棋的下棋,打扑克的打扑克,我却独自寻思田黄石印的去向,已经完全肯定是被贪污了。我考虑着是否直接问一下所长。这时矮墩墩的刘看守员从外面经过,站住了。\\

“你怎么不玩?”他问。\\

“我不会。”我答。这是实话。\\

“你学嘛,打百分一学就会。”\\

“我学也学不会。”这也是实话。\\

“哪里的话!我不信还有学不会打扑克的。等一等,”他热情地说,“我交了班来教你。”\\

过了一会儿,他果真带着一副扑克牌来了。他一屁股坐在栏杆外面,兴致勃勃地洗起牌来。我那套田黄石印就是交给他的。我心里对他原有的好印象全没有了。我当时的心情——现在想起来还是难受的——竟是充满了厌恶。\\

“我就不相信这个学不会,”刘看守员发着牌说,“再说,不会玩怎么行?你将来重新做人,重新生活,不会玩那可怎么生活!”\\

我心想:“你可真会说,装的真像呵!”\\

“\ruby{溥仪}{\textcolor{PinYinColor}{\Man ᡦᡠ ᡳ}}并不笨,”高个子老王也凑过来,嘴里叼着个小烟袋,笑着说。这就是在抚顺给\ruby{荣源}{\textcolor{PinYinColor}{Žung Yuwan}}找回沈阳烟的那个看守员,他的烟瘾很大,终日不离烟袋,那烟袋只有一柞长。他到痰盂那里敲掉了烟灰,又开始装新的一袋,一边装一边说:“\ruby{溥仪}{\textcolor{PinYinColor}{\Man ᡦᡠ ᡳ}}不笨,只要学,什么都学的会。”\\

他点上了烟。隔壁有个人对他说:“王先生,你的烟挺香呵!”\\

“怎么,大概你的烟卷又没啦?”他挪过一步对隔壁看看。不知是谁笑着又说:“我抽烟太没计划。”王看守员笑笑,解下了小烟荷包,扔了过去:“好吧,拿纸卷一支过过瘾。”\\

王看守员每逢犯人抽光了规定的纸烟,总要解下烟荷包让人卷烟过瘾。这种举动原来使我很不理解,而现在则有了解释:“你们全是骗人!我就不信你们这一套!”\\

事实上,一心想骗人的不是别人,正是我自己,而弄得别人不能相信的,也是我自己。过了不久,所长在院子里对我说:\\

“你的信和田黄石的图章,我全看到了。你从前在苏联送出去的那些东西,现在也在我们这里。不过,对于人民说来,更有价值的是人,是经过改造的人。”