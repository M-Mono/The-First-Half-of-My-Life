\fancyhead[LO]{{\scriptsize 1931-1932: 到东北去 · 白河偷渡}} %奇數頁眉的左邊
\fancyhead[RO]{} %奇數頁眉的右邊
\fancyhead[LE]{} %偶數頁眉的左邊
\fancyhead[RE]{{\scriptsize 1931-1932: 到东北去 · 白河偷渡}} %偶數頁眉的右邊
\chapter*{白河偷渡}
\addcontentsline{toc}{chapter}{\hspace{1cm}白河偷渡}
\thispagestyle{empty}
动身日期是十一月十日。按照计划,我必须在这天傍晚,瞒过所有的耳目,悄悄混出静园的大门。我为这件事临时很费了一番脑筋。我先是打算不走大门,索性把汽车从车房门开出去。我命令最亲近的随侍大李去看看能不能打开车房门,他说车房门久未使用,门外已经被广告招贴糊住了。后来还是\xpinyin*{祁继忠}想出了个办法,这就是把我藏进一辆跑车(即只有双座的一种敞篷车)的后箱里,然后从随侍里面挑了一个勉强会开车的,充当临时司机。他自己坐在司机旁边,押着这辆“空车”,把我载出了静园。\\

在离静园大门不远的地方,\xpinyin*{吉田忠太郎}坐在一辆汽车上等着,一看见我的汽车出了大门,他的车便悄悄跟在后面。\\

那时正是天津骚乱事件的第三天。日本租界和邻近的中国管区一带整日戒严。这次骚乱和戒严,究竟是有意的布置还是偶合,我不能断定,总之给我的出奔造成了极为顺利的环境。在任何中国人的车辆不得通行的情况下,我这辆汽车走到每个路口的铁丝网前,遇到日本兵阻拦时,经后面的\xpinyin*{吉田}一打招呼,便立刻通过。所以虽然\xpinyin*{祁继忠}找来的这个二把刀司机技术实在糟糕(一出静园大门车就撞在电线杆子上,我的脑袋给箱盖狠狠碰了一下,一路上还把我颠撞得十分难受),但是总算顺利地开到了预定的地点——敷岛料理店。\\

汽车停下之后,\xpinyin*{祁继忠}把开车的人支到一边,\xpinyin*{吉田}过来打开了车箱,扶我出来,一同进了敷岛料理店。早等候在这里的日本军官,叫\xpinyin*{真方勋大尉},他拿出了一件日本军大衣和军帽,把我迅速打扮了一下,然后和\xpinyin*{吉田}一同陪我坐上一部日军司令部的军车。这部车在白河岸上畅行无阻,一直开到一个码头。车子停下来之后,\xpinyin*{吉田}和\xpinyin*{真方勋}扶我下了车。我很快就看出来,这不是日租界,不觉有点发慌。\xpinyin*{吉田}低声安慰我说:“不要紧,这是英租界。”我在他和\xpinyin*{真方勋}二人的夹扶下,快步在水泥地面上走了一段,一只小小的没有灯光的汽船出现在眼前。我走进船舱,看见了\xpinyin*{郑孝胥}父子俩如约候在里面,心里才稳定下来。坐在这里的还有三个日本人,一个是\xpinyin*{上角利一},一个是从前在\xpinyin*{升允}手下做过事的日本浪人\xpinyin*{工藤铁三郎},还有一个叫\xpinyin*{大谷}的,现在忘了他的来历。我见到了船长西长次郎,知道了船上还有十名日本士兵,由一个名叫诹访绩的军曹带领着,担任护送之责。这条船名叫“比治山丸”,是日军司令部运输部的。为了这次特殊的“运输”任务,船上堆了沙袋和钢板。过了二十年之后,我从日本的《文艺春秋》杂志上看到了\xpinyin*{工藤}写的一篇回忆录。据他说当时船上暗藏了一大桶汽油,准备万一被中国军队发现,无法脱逃的时候,日本军人就放火烧,让我们这几个人证与船同归于尽。那时我的座位距离汽油桶大概不会超过三米远,我还认为离着“幸福”是越来越近了呢!\\

\xpinyin*{吉田}和\xpinyin*{真方勋大尉}离开了汽船,汽船离了码头。电灯亮了,我隔窗眺望着河中的夜景,心中不胜感慨。白天的白河我曾到过几次,在东北海军\xpinyin*{毕庶澄}的炮舰上和日本的驱逐舰上,我曾产生过幻想,把白河看做我未来奔向海洋彼岸,寻找复辟外援的通路。如今我真的航行在这条河上了,不禁得意忘形,高兴得想找些话来说说。\\

可是我高兴得未免太早,\xpinyin*{郑垂}告诉我:“外国租界过去了,前边就是中国人的势力。军粮城那边,可有中国军队守着哩!”\\

听了这话,我的心一下子提到了嗓子眼。看看郑氏父子和那几个日本人,全都板着脸,一语不发。大家在沉默中过了两个小时,突然间从岸上传来一声吆喝:“停——船!”\\

像神经一下子被切断了似的,我几乎瘫在地上。舱里的几个日本兵忽噜忽噜地上了甲板,甲板上传来低声的口令和零乱的脚步声。我探头到窗外,看见每个沙包后都有人伏着,端枪做出准备射击的姿势。这时我觉出船的行速在下降,航向好像是靠近河岸。我正不解其故,忽然电灯全熄了,岸上响起了枪声,几乎是同时,机器声突然大作,船身猛然加速,只觉一歪,像跳起来似地掠岸而过,岸上的喊声,枪声,渐渐远了。原来日本人早准备好了这一手,先装作听命的样子,然后乘岸上不备,一溜烟逃过去了。\\

过了一会,灯光亮起来,舱里又有了活气。半夜时到了大沽口外。在等待着商轮“淡路九”出口外接我们的时候,日本兵拿出了酱汤、咸白菜和日本酒。\xpinyin*{郑孝胥}活跃起来了,高谈其同文同种的谬论,把这一场惊险经历描绘成“英雄事业”的一部分。他和日本兵干杯,诗兴大发,即兴吟了一首诗道:\\

\begin{quote}
	同洲二帝欲同尊,七客同舟试共论;\\

人定胜天非浪语,相看应在不多言。\\
\end{quote}

因为这天晚上吃了大米和大麦合制的日本饭,\xpinyin*{郑孝胥}后来刻了两个图章给我,一个是“不忘在莒”,一个是“\xpinyin*{滹}沱麦饭”。前者是借鲁昭公奔莒的故事,暗示我安不忘危,别忘了我和他在一起的这一晚;后者是借刘秀败走\xpinyin*{滹}沱河,大树将军冯异为他烤衣服、做麦饭充饥的故事。\xpinyin*{郑孝胥}把我比做刘秀,他自己自然是比做大树将军了。\\

\xpinyin*{郑孝胥}这天晚上的高兴,除了由于他成了一个胜利者外,大概还有另一层不便说出的原因,这就是他从日本军政的表面摩擦和分歧中,比任何人更早地看出了他们的一致。在我会见\xpinyin*{土肥原}后的第二天(十一月三日),他的日记上写道:\\

\begin{quote}
	\xpinyin*{大七}(即\xpinyin*{郑垂})至日本领事馆,\xpinyin*{后藤}言:\xpinyin*{土肥原}谓此来即为迎上赴奉天,领事馆可佯为不知。\\

二次大战后被发现的日本外务省的档案,其中有十一月六日外相币原给天津\xpinyin*{桑岛}总领事的一封密电稿,说明了白河偷渡的戏剧性:\\

关于拥戴\xpinyin*{宣统}帝的运动。认为如果过度拘束皇帝的自由,对内、外的关系反会不好。曾把这种意见在外务方面协议过,外务方面虽然也同意,但关于满洲目前的局势,各方面都有拥戴皇帝的运动,因此,对于帝国国策的执行上,难保不受到连累。同时,皇帝身边的保护也属必要,所以做了相当的警备。再外务方面也表示,现在满洲方面的政局,也稍安稳,东三省的民众总的意志,也想拥戴皇帝。如果对于国策的执行没有妨碍,听其自然也无不可。\\
\end{quote}
