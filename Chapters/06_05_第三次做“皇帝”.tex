\fancyhead[LO]{{\scriptsize 1932-1945: 伪满十四年 · 第三次做“皇帝”}} %奇數頁眉的左邊
\fancyhead[RO]{} %奇數頁眉的右邊
\fancyhead[LE]{} %偶數頁眉的左邊
\fancyhead[RE]{{\scriptsize 1932-1945: 伪满十四年 · 第三次做“皇帝”}} %偶數頁眉的右邊
\chapter*{第三次做“皇帝”}
\addcontentsline{toc}{chapter}{\hspace{1cm} 第三次做“皇帝”}
\thispagestyle{empty}
\begin{quote}
	京津旧臣,闻皇上就任执政,疑尊号自此取消,同深悲愤。即曾任民国官吏如曹汝霖、汪荣宝等,亦以名义关系甚重为言。臣以皇上屡次坚拒,及最后不得已允许之苦心,详为解释,闻者始稍知此中真相,而终无以尽祛其疑。\\
\end{quote}

这是我就任执政一个月后,请假回天津的陈曾寿寄来的“封奏”中的一段。从京津寄来的这类封奏还有好几件,都曾给了我无限烦恼。\\

按照约定,我当执政一年期满,如果关东军不实行帝制,我是可以辞职的。但是我没有这样干。我没有这样的胆量,而且即便关东军让我辞职,我能到哪里去呢?\\

在就职一周年的头几天,出乎我的意料,在一次例行会见中,武藤先向我提起了这个问题。他说,日本现在正研究着满洲国国体问题,到时机成熟,这个问题自然会解决的。\\

过了不久,即三月二十七日,日本为了更便于自由行动,退出了国际联盟。同时,攻人长城各口的日军加紧军事行动,形成了对平津的包围形势。五月末,忙于打内战的南京政府进一步对日本妥协,签订了“塘沽协定”,将长城以南、冀东地区划为非武装区,撤走中国军队,使日本势力进一步控制了华北。在这种形势下,热心复辟的人们得到了巨大的鼓舞,都以为时机已成熟了,纷纷活动起来。熙洽在三月间曾指使他的心腹林鹤皋,邀集了一批满族“遗民”和前东三省的议员们,在长春聚会,打算弄出一个“劝进表”来,当时被日本宪兵制止了,这时又恢复了活动。华北一些前直系人物和一些日本特务浪人酝酿“拥戴”吴佩孚出山,平津某些与谋的遗老为此派了人来跟郑孝胥联络,研究在华北、东北实现复辟。七月间,总务厅长官驹井德三下台,拿了一百万元退职金,另又要去了一笔巨额机密费,去找黄郛活动华北独立。他临走时向郑孝胥表示还要到上海,为我将来在全国复辟之事进行活动。总之,在那些日子里,经常可以听见关于复辟或帝制的传说,这些传说鼓舞着我,鼓舞着跟我一样的野心家们。郑孝胥这年重阳节写了一首诗,其中有这样的句子:“燕市再游非浪语,异乡久客独关情;西南豪杰休相厄,会遣遗民见后清。”他这种将在“燕市”恢复“后清”的“志气”,使我对他减弱不少恶感。\\

我的“皇帝梦”又做起来了。我非常关心各方面的消息,我进一步把希望放在屠杀自己同胞的日本军队身上。日军全部占领了热河之后,我曾大摆庆功宴席,慰问武藤和参加作战的日军将领们,祝他们“武运长久”,“再接再厉”。后来有一路日本军队占领了距北京只有百里之遥的密云,即按兵不动,我对此不禁大感失望。这时郑孝胥告诉我,日军占领华北以至华南只是迟早间的事,当务之急还是应该先办满洲国体问题。他又说,此事之决定,不在关东军而在东京方面,他已听说东京元老派许多人都是主张我正位的。听了他的话,我觉得应该派个人到东京从侧面去活动一下,至少应该打探些消息来。\\

接受这个使命的是我的警卫官工藤忠。此人即陪我从天津到东北来的工藤铁三郎。他在清末时即跟随升允,在升允后来的复辟活动中,他是积极的赞助者。我在旅顺时,他不像上角和甘粕那样以军方代理人的面目出现,而是处处站在我一边说话,甚至背地里还表示过对关东军的不满。有一次,我看到杯子里的茶水似乎颜色不对,怕有人下了毒,要叫人拿去化验一下,这时工藤立即端起杯子把茶喝了一口。我当了执政之后,他是惟一呼我为“皇上”的日本人,并且时常表示不满意关东军的跋扈,时常表示相信我定能恢复“大清皇帝”的名位。他所表现出的忠心,简直不下于最标准的遗老,因此我赐他改名为“忠”,拿他当自己家里人看待。他也感激涕零地表示誓死效忠,永世不变。他接受了我的使命,去了不多时间就回来了。他在日本见到了南次郎和黑龙会的重要人物,探听出军部方面当权人物是同意实行帝制的。根据他的消息,我相信时机是快到了。\\

一九三三年的十月间,工藤的消息得到了证实。继任的关东军司令官菱刈隆正式通知说,日本政府准备承认我为“满洲帝国皇帝”。\\

我得到了这个通知,简直乐得心花怒放。我考虑到的第一件事情,就是必须准备一套龙袍。\\

龙袍从北京的太妃那里拿来了,但是关东军却对我说,日本承认的是“满洲国皇帝”,不是“大清皇帝”,因此我不能穿清朝龙袍,只能穿关东军指定的“满洲国陆海空军大元帅正装”。\\

“这怎么行?”我对郑孝胥说,“我是爱新觉罗的后人,怎能不守祖制?再说北京的宗室觉罗都要来,看着我穿洋式服装登极算什么?”\\

“皇上说的是。”郑孝胥不住地点头,望着摊在桌上的龙袍。这位一心想做“后清”丞相的人,大概正盘算着正一品珊瑚顶和三眼花翎,最近以来对我顺从得多了。他点头说:“皇上说的是,可是关东军方面怎么说?”\\

“给我交涉去。”\\

郑孝胥走后,我独自欣赏着荣惠太妃保存了二十二年的龙袍,心中充满了感情。这是光绪皇帝穿过的,真正的皇帝龙袍。这是我想了二十二年的龙袍。我必须穿它去登极,这是恢复清朝的起点。……\\

我的头脑还没冷过来,郑孝胥就回来了。他报告说,关东军坚持登极时要穿元帅正装。\\

“你是不是交涉过?”\\

“臣岂敢不去。这是板垣亲自对臣说的。”\\

“这怎么行?”我跳起来,“登极之前要行告天礼,难道叫我穿元帅服磕头祭天吗?”\\

“臣再去跟板垣说说。”\\

郑孝胥走后,胡嗣瑗过来提醒我,要争的不是服制,更重要的是跟军部说,要任免官吏的决定权。如果这问题解决了,赵武灵王的胡服骑射,也没什么不好。\\

其实胡嗣瑗同我一样,都不明白日本要这个帝制,不过为了使我更加傀儡化,为了更便利于统治这块殖民地。皇帝的名义哪里会给我带来什么权力,我这样的人又哪里会学什么骑射?除了依附在日本关东军的皮靴上,我简直什么也不会,什么也不想。所以后来关东军同意了我穿龙袍去祭天,我也就不再去争什么别的了。\\

一九三四年三月一日的清晨,在长春郊外杏花村,在用土垒起的“天坛”上,我穿着龙袍行了告天即位的古礼。然后,回来换了所谓大元帅正装,举行了“登极”典礼。这时执政府改称为“宫内府”,我住的地方因要避开日本天皇的“皇宫”称呼,称为“帝宫”。其中的房屋后来除增建了一所“同德殿”之外,其余的只是修缮了一下,楼名依旧未变。登极典礼是在勤民楼举行的。\\

那天勤民楼的大厅里铺着大红地毯,在北墙跟用丝帷幕装设成一个像神龛似的地方,中间放一特制的高背椅,上刻有作为徽号的兰花,所谓“御纹章”。我立在椅前,两旁站列着宫内府大臣宝熙、侍从武官长张海鹏、侍从武官石丸志都磨和金卓、侍卫处长工藤忠、侍卫官熙仑免(熙洽之子)和润良(婉容之兄)等人,以“总理大臣”郑孝胥为首的文武百官列队向我行三鞠躬礼,我以半躬答之。接着是日本大使菱刈隆向我呈递国书和祝贺。这些仪式完了,北京来的宗室觉罗(载、溥、毓字辈差不多全来了),以及前内务府的人又向我行三跪九叩之礼。当然,我是坐在椅子上受礼的。\\

关内各地遗老,如陈夔龙、叶尔恺、刘承干、朱汝珍、萧丙炎、章梫、黎湛枝、温肃、汪兆镛等等,都寄来祝贺的表章。上海的大流氓头子常玉清,也寄来奏折向我称臣。\\

六月六日,日本天皇的兄弟秩父宫雍仁代表天皇前来祝贺,赠我日本大勋位菊花大缓章,赠婉容宝冠章。\\

胡嗣瑗再三提醒我去要的权利一样也未到手,而我已经昏昏然了。七月间,我父亲带着弟、妹们来长春看我。我对他的接待,足可以说明我的自我陶醉程度。\\

他到达长春的时候,我派出了宫内府以宝熙为首的官员和由佟济煦率领的一队护军,到长春车站列队迎接。我和婉容则在“帝宫”中和门外立候。婉容是宫装打扮,我是身穿戎装,胸前挂满了勋章。我的勋章有三套:一套是日本赠的;一套是“满洲帝国”的;另一套则是我偷着派人到关内定制的“大清帝国”的。后一套当然不能当着关东军的面使用,只能利用这个机会佩戴。\\

我父亲的汽车来了,我立正等着他下了车,向他行了军礼,婉客行了跪安。然后我陪他进了客厅,此时屋内没有外人,我戎装未脱,给他补请了跪安。\\

这天晚上,大摆家宴。吃的是西餐,位次排列完全是洋规矩,由我与婉容分坐在男女主人位子上。另外,又按照我的布置,从我进入宴会厅时起,乐队即开始奏乐。这是宫内府的乐队,奏的什么曲子我已忘了,大概是没有做出什么规定,他们爱妻什么就奏什么,反正喇叭一吹起来,我就觉得够味。\\

在宴会进行到喝香摈的时候,溥杰按我的布置,起立举杯高呼:“皇帝陛下万岁,万岁,万万岁!”我的家族一起随声附和,连我父亲也不例外。我听了这个呼声,到了酒不醉人人自醉的地步了。\\

第二天,宫内府大臣宝熙告诉我,关东军司令部派了人来,以大使馆名义向我提出抗议,说昨天武装的护军去车站,是违反“满洲帝国”已承担义务的前东北当局与日本签订的协议的,这个协议规定,铁路两侧一定范围内是“满铁”的附属地,除日本外任何武装不准进入。关东军司令官——不,日本大使要求保证今后再不发生同类事件。\\

这件事本来是足以令我清醒过来的,可是日本人这时还很会给我面子,首先是没有公开抗议,其次是在我派人道歉和做了保证之后,就没再说什么。但更主要的是它给我规定的许多排场,很能满足我的虚荣心,以致我又陷入了昏迷之中。\\

最使我陶醉的是“御临幸”和“巡狩”。\\

按照关东军的安排,我每年要到外地去一两次,谓之“巡狩”。在“新京”(长春),我每年要去参加四次例行仪式,一次是去“忠灵塔”祭祀死于侵略战争的日军亡魂,一次是到“建国忠灵庙”祭祀伪满军亡魂,一次是到关东军司令部祝日皇寿辰“天长节”,一次是到“协和会”参加年会。这样的外出都称之为“御临幸”。就以去“协和会”为例,说说排场。\\

先说“卤簿”——即所谓“天子出,车驾次第”,是这样的:最先头的是军警的“净街车”,隔一段距离后是一辆红色的敞篷车,车上插一小旗,车内坐着“警察总监”,再后面,是我坐的“正车”,全红色,车两边各有两辆摩托伴随,再后面,则是随从人员和警卫人员的车辆。这是平时用的“略式卤簿”。\\

在出门的前一天,长春的军警、宪兵先借题逮捕“可疑分子”和“有碍观瞻”的“游民”。市民们根据这个迹象就可以判断是我要出门了。到了正日子,沿途预先布满了军警,面向外站着,禁止路人通行,禁止两旁店铺和住家有人出入,禁止在窗口上探头张望。在“协和会”的大门内外全铺了黄土。车驾动身前,广播电台即向全市广播:“皇帝陛下启驾出宫。”用中国话和日本话各说一遍。这时“协和会”里的人全体起立,自“总理”以下的特任官们则列队楼外“奉迎”。车驾到达,人们把身子弯成九十度,同时乐队奏“国歌”。我进入屋内,先在便殿休息一下,然后接见大臣们。两边侍立着宫内府大臣、侍从武官长、侍卫处长、掌礼处长和侍从武官、侍卫官等,后来另添上“帝室御用挂”吉冈安直。用的桌椅以及桌布都是从宫内府搬来的,上有特定的兰花“御纹章”。自总理以下有资格的官员们在我面前逐个行过礼,退出。走完这个过场,我即起身离便殿,此时乐声大作,一直到我进入会场,走上讲台为止。在这段时间内,会场上的人一直是在台下弯成九十度的姿势。关东军司令官此时在台上的一角,见我上台,向我弯身为礼,我点头答礼。我上台后,转过身来向台下答过礼,台下的人才直起身子来。此时宫内府大臣双手捧上“敕语”,我接过打开,向全场宣读。台下全场的人一律低头站着,不得仰视。读完,在我退出会场时,又是乐声大作,全体九十度鞠躬。我回到便殿稍息,这时特任官们又到楼外准备“奉送”。把我送走后,全市街道上的扩音器则又放出“皇帝陛下启驾还宫”的两国话音。我到了家,扩音器还要说一次:“皇帝陛下平安归宫。”\\

据说,这是仿效用于日本天皇的办法。在我照片上做的文章也是从日本搬来的。我的照片被称做“御容”,后来推广适应日本人习惯的那种不中不日的“协和语”,改称之为“御真影”。按规定,在机关、学校、军队和一切公共团体的特定处所,如机关的会议室,学校的校长室里,设立一个像神龛似的东西,外垂帷幕,里面悬着我的照片和“诏书”。任何人走进了这间屋子,都必须先向这个挂帷幕的地方行礼。在居民家里,虽无强制悬御真影的法令规定,但协和会曾强行派售过我与婉容的照片,并指定要悬在正堂上。\\

这种偶像崇拜教育的施行重点,是在军队和学校里。每天早晨,伪满各地的军队与学校都须举行朝会,要行两次遥拜礼,即先面向东方的“皇居”(东京日本天皇的地方),再向长春或帝宫方向,各行一个九十度鞠躬的最敬礼。此外逢到“诏书奉戴日”即颁布每个诏书的日子,还要读诏书。关于诏书我在后面还要谈到。\\

此外还有其他许多规定,还有外地“巡狩”时的种种排场,在这里我不—一赘述了。总之,日本军国主义者把这一套玩艺做得极为认真。据我的体验,这不仅是为了训练中国人,养成盲目服从的习惯和封建迷信思想,就是对下层的日本人也是一样。日本关东军曾经几次利用我去鼓励它的臣民。有一次我到阜新煤矿,日本人曾把日本工头召来,让我对他说几句勉励话。这工头受此“殊荣”,竟感动得流出眼泪。当然,我这时更觉得有身价了。\\

使我终于产生最大的错觉,自认有了极高的权威的,是在一九三五年四月访问日本之后。\\

其实这次访日,全是关东军安排的。他们说,为了答谢日本天皇派御弟秩父宫来对我“即位”的祝贺,也是为了对“日满亲善”的躬亲示范,需要这样办一办。\\

日本政府以枢密顾问官林权助男爵为首组织了十四人的接待委员会,派了战舰比睿丸来迎接,白云、丛云、薄云等舰护航。我从大连港起舰时,有球摩、第十二、第十五驱逐舰队接受我的检阅,到达横滨港时,有百架飞机编队的欢迎。记得我在这次晕头转向、受宠若惊的航程中,写下了一首谄媚的四言诗:\\

\begin{quote}
	海平如镜,万里远航。\\

两邦携手,永固东方。\\

\end{quote}
在航行的第四日,看了一次七十条舰艇的演习,又在晕船呕吐之中写了一首七言绝句:\\

\begin{quote}
	万里雄航破飞涛,碧苍一色天地交,\\

此行岂仅览山水,两国申盟日月昭。\\
\end{quote}

总之,还未上岸,我已受宠若惊。我不仅对日本所示之威力深感惊异,我还把这看做是对我的真心尊敬,真心帮助。过去的一些不愉快,只怪自己误会了。\\

到了日本东京,裕仁亲自到车站迎接我,并为我设宴。在我拜会他们后他又回拜了我。我接见了日本元老重臣,受了祝贺,又同格仁一起检阅了军队。我还参拜了“明治神宫”,慰问了日本陆军医院那些侵略中国挨了打的伤兵伤官。我到裕仁的母亲那里,献了殷勤。日本报纸曾报道过我和她散步的情形,说有一次上土坡,我用手搀扶了日本皇太后,这和我在长春宫内府中,搀我父亲上台阶有着同样的心情。其实,我还从来没有搀扶过自己的父亲,如果问到我搀扶裕仁的母亲的心情,坦白地说,那纯粹是为了巴结。\\

最后一天,雍仁代表他哥哥裕仁到车站向我送别,他致欢送词说:\\

“皇帝陛下这次到日本来,对于日满亲善,是有重大贡献的。我国天皇陛下对此感到非常满意。务请皇帝陛下抱定日满亲善一定能做到的确实信念而回国,这是我的希望。”\\

我又十分巴结地回答道:\\

“我对这次日本皇室的隆重接待和日本国民的热诚欢迎,实是感激已极。我现在下定决心,一定要尽我的全力,为日满的永久亲善而努力。我对这件事,是抱有确实信心的。”\\

临登船出发时,我请担任接待的林权助代向日本天皇和裕仁母亲致谢,这时我居然两眼含满了无耻的眼泪,这样一弄,把那个老头子也给逗哭了。回想起来,我连一点中国人味也没有了。\\

日本皇室这次对我的招待,使我头脑更加发热,感到自从当了皇帝之后,连空气都变了味。我脑子里出现了一个逻辑:天皇与我平等,天皇在日本的地位,就是我在满洲国的地位。日本人对我,当如对其天皇者同。\\

在这种昏昏然中,我一回到长春,立即发表了充满谀词的“回銮训民诏书”,同时请来新任的关东军司令长官南次郎大将,向他发表了我的感想。次日(即四月二十九日),兴高采烈地参加了裕仁的生日的庆祝会,再次日,便急不可待地下谕,把在长春的所有简任职以上的官吏,不论中国人日本人全召来,听我训话,发表访日感想。我在事先完全没有和日本人商议,也没预备讲话稿,到了时候却口若悬河。我讲了访日的经过,绘形绘声地描述了日本天皇对我的招待,讲了日本臣民对我的尊敬。然后大发议论。\\

“为了满日亲善,我确信:如果日本人有不利于满洲国者,就是不忠于日本天皇陛下,如果满洲人有不利于日本者,就是不忠于满洲国的皇帝;如果有不忠于满洲国皇帝的,就是不忠于日本天皇,有不忠于日本天皇的,就是不忠于满洲国皇帝……”\\

我想的实在太天真了。\\

我回到长春不到一个月,关东军司令官南次郎在一次例行会见中,告诉我“郑孝胥总理倦勤思退”,需要让他养老,换一位总理大臣。关于日本不满意郑孝胥的事,我已略有所闻,正想找机会赶走他,现在南次郎提出这事,我立时不假思索地说,让郑退休,我完全同意,总理之职可以由臧式毅继任。我以为听了我两次“日满亲善论”的南次郎一定会遵命的,谁知竟碰了钉子,他向我摇头说:“不,关东军已考虑妥了合适的人选,皇帝陛下不必操心,就让张景惠当总理大臣好了。”\\

郑孝胥不久前在他主办的“王道书院”里发了一次牢骚。他向听课的人说:“满洲国已经不是小孩子了,就该让它自己走走,不该总是处处不放手。”这话惹恼了日本主子,因此就把他一脚踢开。他后来连存在银行里的“建国功劳金”也取不出来,想迁离长春也不得准许,在宪兵队的监视下,只能在家里写写字,做做诗。这个连骨头都被“共管”虫子蛀透了的“诗人兼书法家”,三年之后,终于怀着未遂之愿暴死于长春。他的儿子郑垂也是暴卒的,早于他三年。据传说,他父子都是死于日本人的暗害。即使传闻不确,他的下场也足以打破我的恢复祖业的幻想了,而我到一年之后,即日本全面侵华的前夕,才渐渐明白过来。\\