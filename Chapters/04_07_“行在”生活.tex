\fancyhead[LO]{{\scriptsize 1924-1930: 天津的“行在” · “行在”生活}} %奇數頁眉的左邊
\fancyhead[RO]{} %奇數頁眉的右邊
\fancyhead[LE]{} %偶數頁眉的左邊
\fancyhead[RE]{{\scriptsize 1924-1930: 天津的“行在” · “行在”生活}} %偶數頁眉的右邊
\chapter*{“行在”生活}
\addcontentsline{toc}{chapter}{\hspace{1cm}“行在”生活}
\thispagestyle{empty}
我在张园里住了一段时间以后,就觉得这个环境远比北京的紫禁城舒服。我有了这样的想法:除非复辟的时机已经成熟,或者发生了不可抗拒的外力,我还是住在这里的好。这也是出洋念头渐渐冲淡的一个原因。\\

张园(和后来的静园)对我说来,没有紫禁城里我所不喜欢的东西,又保留了似乎必要的东西。在紫禁城里我最不喜欢的,首先是连坐车、上街都不自由的那套规矩,其次是令我生气的内务府那一批人。如今我有了任意行事的自由,别人只能进谏而无法干涉。在紫禁城里,我认为必要的东西,是我的威严,在这里也依然存在。虽然我已不穿笨拙的皇帝龙袍,经常穿的是普通的袍子马褂,更多的是穿西装,但是这并不影响别人来给我叩拜。我住的地方从前做过游艺场,没有琉璃瓦,也没有雕梁画栋,但还有人把它称做“行在”(我也觉得抽水马桶和暖气设备的洋楼远比养心殿舒服),北京的宗族人等还要轮流来这里给我“值班”,从前张园游艺场售票处的那间屋子,犹如从前的“\xpinyin*{乾}清门侍卫处”。虽然这里已没有了南书房、\xpinyin*{懋勤殿}、内务府这些名堂,但在人们的心目中,张园那块“清室驻津办事处”的牌子就是它们的化身。至于人们对我的称呼,园子里使用的\xpinyin*{宣统}年号,更是一丝不苟地保留着,这对我说来,都是自然而必要的。\\

在张园时代,内务府大臣们只剩下\xpinyin*{荣源}一个人,其余的或留京照料,或告老退休。我到天津后最初发出的谕旨有这两道:“\xpinyin*{郑孝胥}、\xpinyin*{胡嗣瑗}、\xpinyin*{杨锺羲}、\xpinyin*{温肃}、\xpinyin*{景方昶}、\xpinyin*{萧丙炎}、\xpinyin*{陈曾寿}、\xpinyin*{万绳栻}、\xpinyin*{刘骧业}皆驻津备顾问。”“设总务处,著\xpinyin*{郑孝胥}、\xpinyin*{胡嗣瑗}任事,庶务处著\xpinyin*{佟济煦}任事,收支处著\xpinyin*{景方昶}任事,交涉处著\xpinyin*{刘骧业}任事。”\xpinyin*{陈宝琛}、\xpinyin*{罗振玉}、\xpinyin*{郑孝胥}是每天必见的“近臣”,他们和那些顾问每天上午都要来一次,坐在楼外西边的一排平房里等着“召见”。在大门附近有一间屋子,是请求“\xpinyin*{觐}见”者坐候传唤的地方,曾经坐过的人,有武人、政客、遗老、各式“时新”人物、骚人墨客以及医卜星相。像青年党党魁曾传,网球名手\xpinyin*{林宝华},《新天津报》主笔\xpinyin*{刘冉公},国民党监察委员\xpinyin*{高友唐},……都曾加入\xpinyin*{张宗昌}、\xpinyin*{刘凤池}的行列,在这里恭候过“奏事官”的“引见”。驻园的日警,天津人称之为“白帽”的,驻在对面平房里,每日登记着这些往来的人物。每逢我外出,便有一个日警便衣跟随。\\

张园里的经济情况,和紫禁城比起来,自然差的多了,但是我还拥有一笔可观的财产。我从宫里弄出来的一大批财物,一部分换了钱,存在外国银行里生息,一部分变为房产,按月收租金。在关内外我还有大量的土地,即清朝入关后“跑马圈地”弄来的所谓“皇产”,数字我不知道,据我从一种历史刊物上看到的材料说,仅直隶省的皇产,不算八旗的,约有十二万\xpinyin*{饷}。即使把这数字打几个折扣,也还可观。为了处理这些土地的租赁与出售,民国政府直隶督办和清室专设了一个“私产管理处”,两家坐地分赃,卖一块分一笔钱,也是一项收入。此外,前面我已说过,我和\xpinyin*{溥杰}费了半年多功夫运出来的大批珍贵字画古籍,都在我手里。\\

我到天津之后,京、奉、津等地还有许多地方须继续开支月费,为此设立了“留京办事处”、“陵庙承办事务处”、“驻辽宁办事处”、“宗人府”、“私产管理处(与民国当局合组的)”、“东陵守护大臣”和“西陵守护大臣”等去分别管理。我找到了一份材料,这上面只算北京和东西陵这几处的固定月费、薪俸、饭食,就要开支一万五千八百三十七元八角四分\footnote{这个数字包括以下各项:\xpinyin*{敬懿}、\xpinyin*{荣惠}两太妃8000元,醇亲王2800元,寿皇殿总管太监等饭食72元,太庙首领太监等钱粮19.44元,东陵奉祀960元,西陵奉祀832元,东西陵守护大臣200元,醇贤亲王国寝祭品每季266.4元,园寝翼领官兵口分144元,太妃\xpinyin*{邸}内管领值班饭食80元,太妃\xpinyin*{邸}内护军住班饭食32元,留京办事处长官及留用司员薪水1932元,宗人府办公经费500元,以上共15837.84元。},至于天津一地的开支,每月大约需一万多元\footnote{员工薪资约为4000元,\xpinyin*{婉容}、\xpinyin*{文绣}月银1800元,房租约200元,其他开支,据“驻津办事处”的司房写的一份“谨将各项用项缮呈御览”的表格,其中核计出的每月平均开支如下:膳房536.511元,电灯234.947元,番菜膳房215.115元,邮费1.877元,茶房168.782元,自来水61.341元,办事人员饭食236.194元,车费110.642元,电话113.947元,旅费38.364元,奖赏142.902元,购物4128.754元,马\xpinyin*{乾}85元,杂费236.825元,合计6311.201元。\\},最大宗的开支即收买和运动军阀的钱,尚不在此数。每月平均开支中的购买一项,约占全月开支三分之二,也没有包括汽车、钻石之类项目。天津时期的购买用品的开支比在北京时大得多,而且月月增加,像钢琴、钟表、收音机、西装、皮鞋、眼镜,买了又买,不厌其多。\xpinyin*{婉容}本是一位天津大小姐,花钱买废物的门道比我多。她买了什么东西,\xpinyin*{文绣}也一定要。我给\xpinyin*{文绣}买了,\xpinyin*{婉容}一定又要买,而且花的钱更多,好像不如此不足以显示皇后的身份。\xpinyin*{文绣}看她买了,自然又叽咕着要。这种竞赛式的购买,弄得我后来不得不规定她们的月费定额,自然,给\xpinyin*{婉容}定的数目要比\xpinyin*{文绣}的大一些,记得起初是\xpinyin*{婉容}一千,\xpinyin*{文绣}八百,后来有了困难,减到三百与二百。至于我自己花钱,当然没有限制。\\

由于这种昏天黑地的挥霍,张园又出现了紫禁城时代的窘状,有时竟弄得过不了节,付不出房租,后来连近臣和“顾问”们的俸银都开支不出来了。\\

我花了无数的钱,买了无数用不着的东西,也同时买来了一个比\ruby{庄士敦}{Johnston}给我的更强烈的观念:外国人的东西,一切都是好的,而对照之下,我觉得在中国,除了帝制之外,什么都是不好的。\\

一块留兰香牌口香糖,或者一片拜耳的阿司匹灵,这几分钱的东西就足够使我发出喟叹,认为中国人最愚蠢,外国人最聪明。当然,我想到的中国人,并没有包括我自己,因为我自认自己是凌驾于一切臣民之上的。我认为就连那些聪明的外国人也是这样看我的。\\

那时我在外国租界里,受到的是一般中国人绝对得不到的待遇。除了日本人,美国、英国、法国、意大利等各国的总领事。驻军长官、洋行老板,对我也极为恭敬,称我“皇帝陛下”,在他们的国庆日请我去阅兵,参观兵营,参观新到的飞机、兵舰,在新年和我的生日都来向我祝贺……\\

\ruby{庄士敦}{Johnston}没走以前,给我介绍了英国总领事和英国驻军司令,以后他们辗转介绍,历任的司令官都和我酬酢往还不断。英王乔治五世的第三子过津时访问过我,带去了我送他父亲的照片,后来英王来信向我致谢,并把他的照片交英国总领事送给我。通过意大利总领事,我还和意大利国王互赠过照片。\\

我看过不少兵营,参加过多次外国军队的检阅。这些根据我的祖先——西太后承认的“\xpinyin*{庚子}条约”而驻在中国土地上的外国军队,耀武扬威地从我面前走过的时候,我却觉得颇为得意,认为外国人是如此的待我,可见他们还把我看做皇帝。\\

天津有一个英国人办的名叫“乡艺会”(Country Club)的俱乐部,是只准许外国大老板进出的豪华游乐场所,中国人是根本走不进那个大门的,只有对我是个例外\footnote{在后期也准许中国人去,但仅限买办资本家之流,由外国会员带去。这个地方在解放后被人民政府接收,改为人民俱乐部了。}。我可以自由出入,而且可以带着我的家人们,一起享受当“特殊华人”的滋味。\\

为了把我自己打扮得像个西洋人,我尽量利用惠罗公司、隆茂洋行等等外国商店里的衣饰、钻石,把自己装点成《老爷杂志》上的外国贵族模样。我每逢外出,穿着最讲究的英国料子西服,领带上插着钻石别针,袖上是钻石袖扣,手上是钻石戒指,手提“文明棍”,戴着德国蔡司厂出品的眼镜,浑身发着密丝佛陀、古龙香水和樟脑精的混合气味,身边还跟着两条或三条德国猎犬和奇装异服的一妻一妾……\\

我在天津的这种生活,曾引起过\xpinyin*{陈宝琛}、\xpinyin*{胡嗣瑗}这派遗老不少的议论。\\

他们从来没反对我花钱去买东西,也不反对我和外国人来往,但是当我到中原公司去理发,或者偶尔去看一次戏,或者穿着西服到外面电影院看电影,他们就认为大失帝王威仪,非来一番苦谏不可了。有一次,\xpinyin*{胡嗣瑗}竟因我屡谏不改,上了自劾的请求告退的奏折(原文抬头处,我都改成了空一格):\\

\begin{quote}
	奏为微臣积年溺职,致圣德不彰,恐惧自陈,仰恳恩准即予罢斥事。窃臣粗知廉耻,本乏才能,国变以还,宦情都尽,只以我朝三百年赫赫宗社,功德深入人心,又伏闻皇上天禀聪明,同符圣祖,虽贼臣幸窃成柄,必当有兴复之一时。辄谬与诸遗臣密围大计,\xpinyin*{丁巳}垂成旋败,良由策划多歧。\\

十年来事势日非,臣等不能不尸其咎。而此心耿耿,百折莫回者,所恃我皇上圣不虚生,龙潜成德也。泊乘\xpinyin*{舆}出狩,奔向北来,狠荷录其狂愚,置之密勿,时遭多难,义不敢辞。受事迄今,愆尤山积,或劾其才力竭蹶矣,或斥其妒贤嫉能矣,或病其性情褊急矣,或低其贪糜厚禄矣。经臣再三求退,用恤人言,乃承陛下屡予优容,不允所请。臣即万分不肖,具有天良,清夜扪心,能勿感惊?……前者臣以翠华俯临剧场,外议颇形轻侮,言之不觉垂涕。曾蒙褒责有加,奉谕嗣后事无大小,均望随时规益,等因,钦此!仰见皇上如天之度,葑菲不遗,宜如何披露腹心,力图匡护。记近来商场酒肆又传不时游幸,\xpinyin*{罗振玉}且扬言众中,谓有人亲见上至中原公司理发,并购求玩具,动费千数百金等语。道路流传,颇乖物听。论者因疑左右但知容悦,竟无一效忠骨鲠之臣。里既未能执奏于事前,更不获弁明于事后,则臣之溺职者又一也。……是臣溺职辜恩,已属百喙难解,诚如亮言,宜责之以彰其慢者也,若复靦颜不去,伴食浮沉,上何以弼圣功,下何以开贤路?长此因循坐误,更何以偷息于人间?茹鲠在喉,彷徨无已,惟有披沥愚悃,恳恩开去管理驻津办事处一差,即行简用勤能知大体人员,克日接管其事,则宗社幸甚!微臣幸甚!……\\
\end{quote}

\xpinyin*{胡嗣瑗}说的“俯临剧场”,是指我和\xpinyin*{婉容}到开明戏院看梅兰芳先生演《西施》的那一次。他老先生在戏园里看见了我,认为我失了尊严,回来之后就向我辞职。经我再三慰留,以至拿出了两件狐皮筒子赏他,再次表示我从谏的决心,他才转嗔为喜,称赞我是从谏如流的“英主”,结果双方满意,了事大吉。这次由中原公司理发引起的辞职,也是叫我用类似办法解决的。我初到天津那年,\xpinyin*{婉容}过二十整寿生日的时候,我岳父\xpinyin*{荣源}要请一洋乐队来演奏,遗老丁仁长闻讯赶忙进谏,说“洋乐之声,内有哀音”,万不可在“皇后千秋之日”去听。结果是罢用洋乐,丁仁长得到二百块大洋的赏赐。以物质奖赏谏臣,大概就是由这次开的头。\\

从此以后,直到我进了监狱,我一直没有在外面看过戏,理过发。我遵从了\xpinyin*{胡嗣瑗}的意见,并非是怕他再闹,而确实是接受了他的教育,把到戏园子看戏当做有失身份的事。有一个例子可证明我的“进步”。后来有一位瑞典王子到天津,要和我见面,我因为在报上看见他和梅兰芳的合照,便认为他失了身分,为了表示不屑,我拒绝了他的要求,没和他见面。\\

\xpinyin*{陈宝琛}一派的\xpinyin*{胡嗣瑗}、丁仁长这些遗老,到了后期,似乎对于复辟已经绝望,任何冒险的想法都不肯去试一试,这是他们和\xpinyin*{郑孝胥}、\xpinyin*{罗振玉}等不同之处,但他们对于帝王的威严,却比\xpinyin*{郑孝胥}等人似乎更重视,这也是使我依然信赖这些老头子的原因。尽管他们的意见常常被我视为迂腐,遇到他们有矢忠表现的时候,我总还采纳他们的意见。因此在那种十分新奇的洋场生活中,我始终没忘记自己的身分,牢固地记住了“皇帝”的“守则”。\\

一九二七年,\xpinyin*{康有为}去世,他的弟子\xpinyin*{徐良}求我赐以\xpinyin*{谥法}。按我起初的想法,是要给他的。康在去世前一年,常到张园来看我,第一次见到我的时候,曾泪流满脸地给我磕头,向我叙述当年“德宗皇帝隆遇之思”,后来他继续为我奔走各地,寻求复辟支持者,叫他的弟子向海外华侨广泛宣传:“欲救中国非\xpinyin*{宣统}君临天下,再造帝国不可”。他临死前不久,还向\xpinyin*{吴佩孚}以及其他当权派呼吁过复辟。我认为从这些举动上看来,给以\xpinyin*{谥法}是很应当的。但是\xpinyin*{陈宝琛}出来反对了。这时候在他看来,分辨忠奸不仅不能只看辫子,就连复辟的实际行动也不足为据。他说:“\xpinyin*{康有为}的宗旨不纯,曾有保中国不保大清之说。且当年\xpinyin*{忤}逆孝钦太皇太后(\xpinyin*{慈禧}),已不可赦!”\xpinyin*{胡嗣瑗}等人完全附和\xpinyin*{陈宝琛},\xpinyin*{郑孝胥}也说\xpinyin*{光绪}当年是受了\xpinyin*{康有为}之害。就这样,我又上了一次分辨“忠奸”的课,拒绝了赐谥给\xpinyin*{康有为}。据说后来\xpinyin*{徐良}为此还声言要和陈、郑等人“以老拳相见”哩。\\

一九三一年,\xpinyin*{文绣}突然提出了离婚要求,在得到解决之后,遗老们还没有忘记这一条:要发个上谕,贬淑妃为庶人。我自然也照办了。\\

说起\xpinyin*{文绣}和我离婚这一段,我想起了我的家庭夫妇间的不正常的生活。这与其说是感情上的问题,倒不如说是由于张园生活上的空虚。其实即使我只有一个妻子,这个妻子也不会觉得有什么意思。因为我的兴趣除了复辟,还是复辟。老实说,我不懂得什么叫爱情,在别人是平等的夫妇,在我,夫妇关系就是主奴关系,妻妾都是君王的奴才和工具。\\

这里是\xpinyin*{文绣}在宫里写的一篇短文,这篇短文中多少流露出了她当时的心情:\\

\begin{quote}
	哀苑鹿\\

春光明媚,红绿满园,余偶散步其中,游目骋怀,信可乐也。倚树稍憩,忽闻\xpinyin*{囿}鹿,悲鸣宛转,\xpinyin*{俛}而视之,\xpinyin*{奄奄}待毙,状殊可怜。余以此庞得入御园,受恩俸\xpinyin*{豢}养,永保其生,亦可谓之幸矣。然野畜不畜于家,如此鹿在园内,不得其自由,犹狱内之犯人,非遇赦不得而出也。庄子云:\\

宁其生而曳尾于涂中,不愿其死为骨为贵也。\\
\end{quote}

\xpinyin*{文绣}从小受的是三从四德的教育,不到十四岁,开始了“宫妃”生活,因此“君权”和“夫权”的观念很深。她在那种环境中敢于提出离婚,不能说这不是需要双重勇敢的行为。她破除万难,实现了离婚的要求,离婚之后,仍受到不少压力。有人说,她提出离婚是受了家里人的教唆,是为了贪图一笔可观的赡养费。事实上,她家里的人给她精神上的迫害不见得比外来的少。据说她拿到的五万元赡养费,经过律师、中间人以及家里人的克扣、占用、“求助”,剩不了好多,而她精神上受的损害更大。她的一个哥哥曾在天津《商报》上发表了一封公开信给她,其中竟有这样的话:\\

\begin{quote}
	我家受清帝厚恩二百余载,我祖我宗四代官至一品。且慢云逊帝对汝并无虐待之事,即果然虐待,在汝亦应耐死忍受。……汝随侍逊帝,身披绫罗,口\xpinyin*{餍}鱼肉,使用仆妇,工资由账房开支,购买物品物价由账房开支,且每月有二百元之月费,试问汝一闺阁妇女,果有何不足?纵中宫待汝稍严,不肯假以辞色,然抱\xpinyin*{衾}与调,自是小星本分,实命不犹,抑又何怨……?\\
\end{quote}

这封信曾在遗老们中间传诵一时。\xpinyin*{文绣}后来的情形不详,只听说她在天津当了小学教师,殁于一九五零年,终身未再结婚。\\

如果从表面现象上看,\xpinyin*{文绣}是被“中宫”挤跑了的。这虽非全部原因,也是原因之一。\xpinyin*{婉容}当时的心理状态,可以从她求的\xpinyin*{乩}辞上窥得一斑(文内\xpinyin*{金荣}氏指\xpinyin*{婉容},端氏指\xpinyin*{文绣}):\\

\begin{quote}
	\xpinyin*{婉容}求的\xpinyin*{乩}文\\

吾仙师叫\xpinyin*{金荣}氏听我劝,万岁与荣氏真心之好并无二意,荣氏不可多疑,吾仙师保护万岁,荣氏后有子孙,万岁后有大望,荣氏听我仙师话,吾保护尔的身体,万岁与端氏并无真心真意,荣氏你自管放心好了。\\
\end{quote}

顺便提一下,这种令人发笑的扶\xpinyin*{乩}、相面、算卦、批八字等等活动,在那时却是不足为怪的社会现象,在张园里更是日常生活不可少的玩意。在我后来住的静园里,就有房东\xpinyin*{陆宗舆}设的“\xpinyin*{乩}坛”。简直可以说,那时\xpinyin*{乩}坛和卜卦给我的精神力量,对我的指导作用,是仅次于师傅和其他近臣们对我的教育。我常常从这方面得到“某年人运”、“某岁大显”之类预言的鼓舞。北京商会会长孙学仕自称精通麻衣,曾预言我的“御容”何时将人运,何时又将握“大权”。日本领事馆里的一位日本相法家也说过我某某年必定成大事的话。这些都是我开倒车的动力。
