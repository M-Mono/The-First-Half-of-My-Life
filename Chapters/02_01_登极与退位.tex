\fancyhead[LO]{{\scriptsize 1908-1917: 我的童年 · 登极与退位}} %奇數頁眉的左邊
\fancyhead[RO]{} %奇數頁眉的右邊
\fancyhead[LE]{} %偶數頁眉的左邊
\fancyhead[RE]{{\scriptsize 1908-1917: 我的童年 · 登极与退位}} %偶數頁眉的右邊
\chapter*{登极与退位}
\addcontentsline{toc}{chapter}{\hspace{1cm}登极与退位}
\thispagestyle{empty}
\xpinyin*{光绪}三十四年旧历十月二十日的傍晚,\xpinyin*{醇王}府里发生了一场大混乱。这边老福晋不等听完新就位的摄政王带回来的\xpinyin*{懿旨},先昏过去了。王府太监和妇差丫头们灌姜汁的灌姜汁,传大夫的传大夫,忙成一团,那边又传过来孩子的哭叫和大人们哄劝声。摄政工手忙脚乱地跑出跑进,一会儿招呼着随他一起来的军机大臣和内监,叫人给孩子穿衣服,这时他忘掉了老福晋正昏迷不醒,一会被叫进去看老福晋,又忘掉了军机大臣还等着送未来的皇帝进宫。这样闹腾好大一阵,老福晋苏醒过来,被扶送到里面去歇了,这里未来的皇帝还在“抗旨”,连哭带打地不让内监过来抱他。内监苦笑着看军机大臣怎么吩咐,军机大臣则束手无策地等摄政工商量办法,可是摄政王只会点头,什么办法也没有……\\

家里的老人给我说的这段情形,我早已没有印象了。老人们说,那一场混乱后来还亏着乳母给结束的。乳母看我哭得可怜,拿出奶来喂我,这才止住了我的哭叫。这个卓越的举动启发了束手无策的老爷们。军机大臣和我父亲商量了一下,决定由乳母抱我一起去,到了中南海,再交内监抱我见\xpinyin*{慈禧}太后。\\

我和\xpinyin*{慈禧}这次见面,还能够模糊地记得一点。那是由一次强烈的刺激造成的印象。我记得那时自己忽然处在许多陌生人中间,在我面前有一个阴森森的帏帐,里面露出一张丑得要命的瘦脸——这就是\xpinyin*{慈禧}。据说我一看见\xpinyin*{慈禧},立刻嚎啕大哭,浑身哆嗦不住。\xpinyin*{慈禧}叫人拿冰糖葫芦给我,被我一把摔到地下,连声哭喊着:“要嫫嫫!要嫫嫫!”弄得\xpinyin*{慈禧}很不痛快,说:“这孩子真别扭,抱到哪儿玩去吧!”\\

我入宫后的第三天,\xpinyin*{慈禧}去世,过了半个多月,即旧历十一月初九,举行了“登极大典”。这个大典被我哭得大煞风景。\\

大典是在太和殿举行的。在大典之前,照章要先在中和殿接受领侍卫内大臣们的叩拜,然后再到太和殿受文武百官朝贺。我被他们折腾了半天,加上那天天气奇冷,因此当他们把我抬到太和殿,放到又高又大的宝座上的时候,早超过了我的耐性限度。我父亲单膝侧身跪在宝座下面,双手扶我,不叫我乱动,我却挣扎着哭喊:“我不挨这儿!我要回家!我不挨这儿!我要回家!”父亲急得满头是汗。文武百官的三跪九叩,没完没了,我的哭叫也越来越响。我父亲只好哄我说:“别哭别哭,快完了,快完了!”\\

典礼结束后,文武百官窃窃私议起来了:“怎么可以说‘快完了’呢?”“说要回家可是什么意思呵?”……一切的议论,都是垂头丧气的,好像都发现了不祥之兆。\\

后来有些笔记小品里加技添叶地说,我是在钟鼓齐鸣声中吓哭了的,又说我父亲在焦急之中,拿了一个玩具小老虎哄我,才止住了哭。其实那次大典因为处于“国丧”期,丹陛大乐只设而不奏,所谓玩具云者更无其事。不过说到大臣们都为了那两句话而惶惑不安,倒是真事。有的书上还说,不到三年,清朝真的完了,要回家的也真回了家,可见当时说的句句是谶语,大臣们早是从这两句话得到了感应的。\\

事实上,真正的感应不是来自偶然而无意的两句话。如果翻看一下当时历史的记载,就很容易明白文武百官们的忧心忡忡是从哪里来的。只要看看《清鉴纲目》里关于我登极前一年的大事提要就够了:\\

\begin{quote}
	\xpinyin*{光绪}三十三年,秋七月。广州钦州革命党起事,攻陷阳城,旋被击败。\\

冬十一月。孙文、黄兴合攻广西镇南关\footnote{现改名睦南关},克之,旋败退。\\

谕:禁学生干预政治及开会演说。\\

三十四年,春正月。广东缉获日本轮船,私运军火,寻命释之。\\

三月。孙文、黄兴遣其党攻云南河口,克之,旋败退。\\

冬十月,安庆炮营队官熊成基起事,旋败死。\\
\end{quote}

这本《清鉴纲目》是民国时代编出的,所根据的史料主要是清政府的档案。我从那个时期的档案里还看到不少“败死”“败退”的字样。这类字样越多,也就越说明风暴的加剧。这就是当时那些王公大臣们的忧患所在。到了\xpinyin*{宣统}朝,事情越加明显。后来起用了\xpinyin*{袁世凯},在一部分人心里更增加一重忧虑,认为外有革命党,内有\xpinyin*{袁世凯},历史上所出现过的不吉之兆,都集中到\xpinyin*{宣统}一朝来了。\\

我胡里胡涂地做了三年皇帝,又胡里胡涂地退了位。在最后的日子里所发生的事情,给我的印象最深的是:有一天在养心殿的东暖阁里,\xpinyin*{隆裕}太后坐在靠南窗的炕上,用手绢擦眼,面前地上的红毡子垫上跪着一个粗胖的老头子,满脸泪痕。我坐在太后的右边,非常纳闷,不明白两个大人为什么哭。这时殿里除了我们三个,别无他人,安静得很,胖老头很响地一边抽缩着鼻子一边说话,说的什么我全不懂。后来我才知道,这个胖老头就是\xpinyin*{袁世凯}。这是我看见\xpinyin*{袁世凯}惟一的一次,也是\xpinyin*{袁世凯}最后一次见太后。如果别人没有对我说错的话,那么正是在这次,\xpinyin*{袁世凯}向\xpinyin*{隆裕}太后直接提出了退位的问题。从这次召见之后,\xpinyin*{袁世凯}就借口东华门遇险\footnote{一九一二年一月十六日\xpinyin*{袁世凯}退朝回家,三个革命党人伺于东华门大街便宜坊酒楼上,掷弹炸袁未中,炸毙袁的侍卫长袁金标,炸伤护兵数人,事后袁以“久患心跳作烧及左\xpinyin*{骽}腰疼痛等症”为名请假,拒不入朝,让胡惟德等人代奏。}的事故,再不进宫了。\\

武昌起义后,各地纷纷响应,满族统帅根本指挥不动抵抗民军的北洋各镇新军,摄政王再也没办法,只有接受\xpinyin*{奕劻}这一伙人的推荐,起用了\xpinyin*{袁世凯}。待价而沽的\xpinyin*{袁世凯},有\xpinyin*{徐世昌}这位身居内阁协办大臣的心腹之交供给情报,摸透了北京的行情,对于北京的起用推辞再三,一直到被授以内阁总理大臣和统制全部兵权的钦差大臣,军政大权全已在握的时候,他才在彰德“遥领圣旨”,下令北洋军向民军进攻。夺回了汉阳后,即按兵不动,动身进京,受\xpinyin*{隆裕}太后和摄政王的召见。\\

这时候的\xpinyin*{袁世凯}和从前的\xpinyin*{袁世凯}不同了,不仅有了军政大权,还有了比这更为难得的东西,这就是洋人方面对他也发生了兴趣,而革命党方面也有了他的朋友。北洋军攻下汉阳之后,英国公使朱尔典得到本国政府的指示,告诉他:英国对袁“已经发生了极友好的感情”\footnote{一九一一年十一月十五日,英国外相格雷复驻华公使朱尔典电。其全文是“复你十二日电。我们对\xpinyin*{袁世凯}已发生了极友好的感情和崇敬。我们愿意看到一个足够有力的政府,可以不偏袒地处理对外关系,维持国内秩序以及革命后在华贸易的有利环境,这样的政府将要得到我们所能给予的一切外交援助。”(见蓝皮书中国第一号,一九一二年四十页)}。袁到北京不久,英国驻武昌的总领事就奉朱尔典之命出面调停民军和清军的战事。\xpinyin*{袁世凯}的革命党方面的朋友,主要的是谋刺摄政王不遂的\xpinyin*{汪精卫}。\xpinyin*{汪精卫}被捕之后,受到肃亲王\xpinyin*{善耆}的很好的招待。我父亲在自己的年谱中说这是为了“以安反侧之心”,其实并非如此。我有位亲戚后来告诉过我,当时有个叫西田耕一的日本人,通过\xpinyin*{善耆}那里的日本顾问关系告诉善,日本人是不同意杀掉\xpinyin*{汪精卫}的。摄政王在几方面压力之下,没有敢对\xpinyin*{汪精卫}下手。武昌事起,\xpinyin*{汪精卫}得到释放,他立刻抓住机会和\xpinyin*{善耆}之流的亲贵交朋友。\xpinyin*{袁世凯}到北京,两人一拍即合,\xpinyin*{汪精卫}也很快与袁的长公子克定变成了好朋友,从而变成了袁的谋士,同时也变成了\xpinyin*{袁世凯}和民军方面某些人物中间的桥梁。民军方面的消息经此源源地传到\xpinyin*{袁世凯}这边,立宪派人物也逐渐对他表示好感。\xpinyin*{袁世凯}有了许多新朋友,加上在国内外和朝廷内外的那一伙旧朋友,就成了对各方面情况最清楚而且是左右逢源的人物。\xpinyin*{袁世凯}口到北京后,不到一个月,就通过\xpinyin*{奕劻}在\xpinyin*{隆裕}面前玩了个把戏,把摄政王挤掉,返归藩邸。接着,以接济军用为名挤出了\xpinyin*{隆裕}的内帮,同时逼着亲贵们输财赡军。亲贵感到了切肤的疼痛,皇室的财力陷入了枯竭之境,至此,政、兵、财三权全到了袁的手里。接着,袁授意驻俄公使陆征祥联合各驻外公使致电清室,要求皇帝退位,同时以全体国务员名义密奏太后,说是除了实行共和,别无出路。我查到了这个密奏的日期,正是前面提到的与袁会面的那天,即十一月二十八日。由此我明白了太后为什么后来还哭个不停。密奏中让太后最感到恐怖的,莫过于这几句:“海军尽叛,天险已无,何能悉以六镇诸军,防卫京津?虽效周室之播迁,已无相容之地。”“东西友邦,有从事调停者,以我只政治改革而已,若等久事争持,则难免无不干涉。而民军亦必因此对于朝廷,感情益恶。读法兰西革命之史,如能早顺\xpinyin*{舆}情,何至路易之子孙,靡有孑造也。……”\\

\xpinyin*{隆裕}太后完全给吓昏了,连忙召集御前会议,把宗室亲贵们叫来拿主意。王公们听到了密奏的内容和\xpinyin*{袁世凯}的危言,首先感到震动的倒不是法兰西的故事,而是\xpinyin*{袁世凯}急转直下的变化。本来在民、清两军的议和谈判中,\xpinyin*{袁世凯}一直反对实行共和,坚决主张君主立宪。他曾在致\xpinyin*{梁鼎芬}的一封信中,表示了自己对清室的耿耿忠心,说“决不辜负孤儿寡妇(指我和太后)”。在他刚到北京不久,发布准许百姓自由剪发辫的上谕的那天,在散朝外出的路上,\xpinyin*{世续}指着自己脑后的辫子笑着问道:“大哥,您对这个打算怎么办?”他还肃然回答:“大哥您放心,我还很爱惜它,总要设法保全它!”因此一些对\xpinyin*{袁世凯}表示不信任的人很为高兴,说“袁宫保决不会当曹操!”民清双方的谈判,达成了把国体问题交临时国会表决的原则协议,国会的成员、时间和地点问题,则因清方的坚持而未决。正争执中,南京成立了临时政府,选了孙中山为临时大总统,第二天,\xpinyin*{袁世凯}忽然撤去唐绍仪代表的资格,改由他自己直接和民军代表用电报交涉。国体问题还远未解决,忽然出现了袁内阁要求清帝退位问题,自然使皇室大为震骇。\\

原来\xpinyin*{袁世凯}这时有了洋人的支持,在民军方面的朋友也多到可以左右民军行动的程度。那些由原先的立宪党人变成的革命党人,已经明白\xpinyin*{袁世凯}是他们的希望;这种希望后来又传染给某些天真的共和主义者。因此在民军方面做出了这个决议:只要袁赞成共和,共和很快就可成功;只要袁肯干,可以请袁做第一任大总统。这正符合了袁的理想,何况退位的摄政王周围,还有一个始终敌对的势力,无论他打胜了革命党还是败给革命党,这个势力都不饶他。他决定接受这个条件,但对清室的处置,还费考虑。这时他忽然听说孙中山在南京就任了临时大总统,不免着起急来。他的心腹助手赵秉钧后来透露:“项城本具雄心,又善利用时机。但虽重兵在握,却力避曹孟德欺人之名,故一面挟北方势力与南方接洽,一方面挟南方势力,以胁制北方。项城初以为南方易与,颇侧南方,及南方选举总统后,恍然南北终是两家,不愿南方势力增长,如国民大会成立,将终为其挟持,不能摆脱。乃决计专对清室着手,首先胁迫亲贵王公,进而胁迫清帝,又进而恫吓太后,并忖度其心理,诱饵之以优待条件,达到自行颁布退位,以全权组织临时政府。”这就是\xpinyin*{袁世凯}突然变化的真相。\\

变化尽管是变化,如果想从善于流泪的\xpinyin*{袁世凯}脸上,直接看到凶相,是办不到的。他最后和太后见了那次面,在东华门碰上了一个冒失的革命党人的炸弹,给了他一个借口,从此再不进宫,而由他的助手赵秉钩、胡惟德等人出面对付皇室。他自己不便于扮演的角色就由他们来扮演。\\

但是变化终归是变化。那些相信过\xpinyin*{袁世凯}的人,又改变了看法。\\

“谁说\xpinyin*{袁世凯}不是曹操?”\\

一直坚持这个说法的是恭王\xpinyin*{溥伟}、肃王\xpinyin*{善耆}、公爵\ruby{载泽}{zǎi zé}等人,还有\xpinyin*{醇王}周围的年轻的贝勒们。一位贵胄学堂的学生后来说,当时的民政大臣满人桂春曾宣称,为了回答外地对满人仇杀的行为,他已组织了满族警察和贵胄学堂的学生,对北京城的汉人实行报复。远在西安的总督蒙族人\xpinyin*{升允},这时带兵勤王,离了西安,\xpinyin*{袁世凯}去了一封表示赞许的电报,同时命令他停在潼关不得前进。以\xpinyin*{良弼}为首的一些贵族组织了宗社党\footnote{在\xpinyin*{辛亥}革命期间,满清皇族的最顽固最反动的集团,以\xpinyin*{良弼}、\xpinyin*{溥伟}、\xpinyin*{铁良}等为首组成了宗社党,其目的是挽救清朝的灭亡,反对清帝退位,反对\xpinyin*{袁世凯},反对议和。后\xpinyin*{良弼}被革命党人彭家珍炸死,\xpinyin*{袁世凯}又策动\xpinyin*{冯国璋}等发表通电,赞成共和,才被迫同意清帝沮位,\xpinyin*{隆裕}亦传谕,把它解散。宗社党解体之后,其中一些主要分子并不死心,分别投靠了帝国主义企图借外力来复辟。},宗社党将采取恐怖行动的传说也出现了。总之,一部分满蒙王公大臣做出了要拚命的姿态。太后召集的第一次御前会议,会上充满了忿恨之声。\xpinyin*{奕劻}和溥伦由于表示赞成退位,遭到了猛烈的抨击。第二天,\xpinyin*{奕劻}没有敢来,溥伦改变了口风,声明赞成君主。\\

这种情势没有保持多久。参加会议的\xpinyin*{毓朗}后来和他的后辈说过这个会议,\xpinyin*{溥伟}也有一篇日记做了一些记载,内容都差不多。其中的一次会议是这样开的:\\

太后问:“你们看是君主好还是共和好?”\\

大约有四五个人立刻应声道:“奴才都主张君主,没有主共和的道理。”接着别人也表示了这个态度,这次\xpinyin*{奕劻}和溥伦没参加,没有相反的意见。有人还说,求太后“圣断坚持,勿为\xpinyin*{奕劻}之流所惑”。太后叹气道:\\

“我何尝要共和,都是\xpinyin*{奕劻}跟\xpinyin*{袁世凯}说的,革命党太厉害,咱没枪炮役军\xpinyin*{饷},打不了这个仗。我说不能找外国人帮忙吗?他们说去问问。过了两天说问过了,外国人说摄政王退位他们才帮忙。\ruby{载沣}{zǎi fēng}你说是不是这样说的?”\\

\xpinyin*{溥伟}忿忿地说:“摄政王不是已退位了吗?怎么外国人还不帮忙?这显然是\xpinyin*{奕劻}欺君\xpinyin*{罔}上!”\\

那彦图接口道:“太后今后可别再听\xpinyin*{奕劻}的啦!”\\

\xpinyin*{溥伟}和\ruby{载泽}{zǎi zé}说:“乱党实不足惧,只要出军\xpinyin*{饷},就有忠臣去破贼杀敌。\xpinyin*{冯国璋}说过,发三个月的\xpinyin*{饷}他就能把革命党打败。”\\

“内\xpinyin*{帑}已经给\xpinyin*{袁世凯}全要了去,我真没有钱了!”太后摇头叹气。\\

\xpinyin*{溥伟}拿出日俄战争中日本帝后以首饰珠宝赏军的故事,劝清太后效法。\xpinyin*{善耆}支持\xpinyin*{溥伟}的意见,说这是个好主意。\xpinyin*{隆裕}说:“胜了固然好,要是败了,连优待条件不是也落不着了吗?”\\

这时优待条件已经由民清双方代表议出来了。\\

“优待条件不过是骗人之谈,”\xpinyin*{溥伟}说,“就和迎闯王不纳粮的话一样,那是欺民,这是欺君。即使这条件是真的,以朝廷之尊而受臣民优待,岂不贻笑千古,贻笑列邦?”说罢,他就地碰起头来。\\

“就是打仗,只有\xpinyin*{冯国璋}一个也不行呀!”太后仍然没信心。\xpinyin*{溥伟}就请求“太后和皇上赏兵去报国”。\xpinyin*{善耆}也说,有的是忠勇之士。太后转过头,问跪在一边一直不说话的\xpinyin*{载涛}:\\

“\xpinyin*{载涛}你管陆军,你知道咱们的兵怎么样。”\\

“奴才练过兵,没打过仗,不知道。”\xpinyin*{载涛}连忙碰头回答。\\

太后不做声了。停了一晌才说了一句:\\

“你们先下去吧。”\\

末了,\xpinyin*{善耆}又向太后嘱咐一遍:“一会,\xpinyin*{袁世凯}和国务大臣就觐见了,太后还要慎重降旨。”\\

“我真怕见他们。”太后摇头叹气。……\\

在这次会议上,本来\xpinyin*{溥伟}给太后想出了个应付国务大臣们的办法,就是把退位问题推到遥遥无期的国会身上。可是国务大臣赵秉钧带来了\xpinyin*{袁世凯}早准备好了的话:\\

“这个事儿叫大伙儿一讨论,有没有优待条件,可就说不准了!”\\

太后对于王公们主战的主意不肯考虑了。王公们曾千嘱咐万嘱咐不要把这件事和太监说起,可是太后一回宫,早被\xpinyin*{袁世凯}喂饱的总管太监\xpinyin*{小德张}却先开了口:\\

“照奴才看,共和也罢,君主也罢,老主子全是一样。讲君主,老主子管的事不过是用用宝。讲共和,太后也还是太后。不过这可得答应了那‘条件’。要是不应呵,革命党打到了北京,那就全完啦!”\\

在御前会议上,发言主战的越来越少,最后只剩下了四个人。据说我的二十几岁的六叔是主战者之一,他主张来个化整为零,将王公封藩,分踞各地进行抵抗。这个主张根本没人听。\xpinyin*{毓朗}贝勒也出过主意,但叫人摸不清他到底主张什么。他说:\\

“要战,即效命疆场,责无旁贷。要和,也要早定大计。”\\

御前会议每次都无结果而散。这时,袁的北洋军将领\xpinyin*{段祺瑞}等人突然从前线发来了要求“清帝”退位的电报,接着,\xpinyin*{良弼}被革命党人炸死了。这样一来,在御前会议上连\xpinyin*{毓朗}那样两可的意见也没有了。主战最力的\xpinyin*{善耆}、\xpinyin*{溥伟}看到大势已去,离了北京,他们想学\xpinyin*{申包胥}哭秦庭,一个跑到德国人占领的青岛,一个到了日本人占领的旅顺。他们被留在那里没让走。外国官员告诉他们,这时到他们国家去是不适宜的。问题很清楚,洋人已决定承认\xpinyin*{袁世凯}的政府了。\\

\xpinyin*{宣统}三年旧历十二月二十五日,\xpinyin*{隆裕}太后颁布了我的退位诏。一部分王公跑进了东交民巷,\xpinyin*{奕劻}父子带着财宝和姨太太搬进了天津的外国租界。\xpinyin*{醇王}在会议上一直一言不发,颁布退位诏后,就回到家里抱孩子去了。\xpinyin*{袁世凯}一边根据清皇太后的\xpinyin*{懿旨},组织了民国临时共和政府,一边根据与南方革命党达成的协议,由大清帝国内阁总理大臣一变而为中华民国的临时大总统。而我呢,则作为大总统的邻居,根据清室优待条件\footnote{与“关于清帝逊位后优待之条件”同时颁布的还有“关于满蒙回藏各族待遇之条件”和“关于清皇族待遇之条件”。}开始了小朝廷的生活。\\

这个清室优待条件如下:\\

\begin{quote}
	第一款:大清皇帝辞位之后,尊号仍存不废。中华民国以待各外国君主之礼相待。\\

第二款:大清皇帝辞位之后,岁用四百万两。\xpinyin*{俟}改铸新币后,改为四百万元,此款由中华民国拨用。\\

第三款:大清皇帝辞位之后,暂居官禁。日后移居颐和园。侍卫人等,照常留用。\\

第四款:大清皇帝辞位之后,其宗庙陵寝,永远奉祀。由中华民国酌设卫兵,妥慎保护。\\

第五款:德宗崇陵未完工程,如制妥修。其奉安典礼,仍如旧制。所有实用经费,并由中华民国支出。\\

第六款:以前官内所用各项执事人员,可照常留用,惟以后不得再招阉人。\\

第七款:大清皇帝辞位之后,其原有之私产由中华民国特别保护。\\

第八款:原有之禁卫军,归中华民国陆军部编制,额数俸\xpinyin*{饷},仍如其旧。\\
\end{quote}
