\fancyhead[LO]{{\scriptsize 1917-1924: 北京的“小朝廷” · 内部冲突}} %奇數頁眉的左邊
\fancyhead[RO]{} %奇數頁眉的右邊
\fancyhead[LE]{} %偶數頁眉的左邊
\fancyhead[RE]{{\scriptsize 1917-1924: 北京的“小朝廷” · 内部冲突}} %偶數頁眉的右邊
\chapter*{内部冲突}
\addcontentsline{toc}{chapter}{\hspace{1cm}内部冲突}
\thispagestyle{empty}
自从\xpinyin*{庄士敦}入宫以来,我在王公大臣们的眼里逐渐成了最不好应付的皇帝。到了我结婚前后这段时间,我的幻想和举动,越发叫他们觉得离奇,因而惊恐不安。我今天传内务府,叫把三万元一粒的钻石买进来,明天又申斥内务府不会过日子,只会贪污浪费。我上午召见大臣,命他们去清查古玩字画当天回奏,下午又叫预备车辆去游香山。我对例行的仪注表示了厌倦,甚至连金顶黄轿也不爱乘坐。为了骑自行车方便,我把祖先在几百年间没有感到不方便的宫门门槛,叫人统统锯掉。我可以为了一件小事,怪罪太监对我不忠,随意叫敬事房答打他们,撤换他们。王公大臣们的神经最受不了的,是我一会想励精图治,要整顿宫廷内部,要清查财务,一会我又扬言要离开紫禁城,出洋留学。王公大臣们被我闹得整天心惊肉跳,辫子都急成白的了。\\

我的出洋问题,有些工公大臣考虑得比我还早,这本来是他们给我请外国师傅的动机之一。我结婚后接到不少造老的奏折、条陈,都提到过这个主张。但到我亲自提出这个问题的时候,几乎所有的人都表示了反对。在各种反对者的理由中,最常听说的是这一条:\\

“只要皇上一出了紫禁城,就等于放弃了民国的优待。既然民国没有取消优待条件,为什么自己偏要先放弃它呢?”\\

无论是对出洋表示同情的,还是根本反对的,无论是对“恢复祖业”已经感到绝望的,还是仍不死心的,都舍不得这个优待条件。尽管优待条件中规定的“四百万岁费”变成了口惠而实不至的空话,但是还有“帝王尊号仍存不废”这一条。只要我留在紫禁城,保住这个小朝廷,对恢复祖业未绝望的人固然很重要,对于已绝望的人也还可以保留饭碗和既得的地位,这种地位的价值不说死后的恤典,单看看给人点主、写墓志铭的那些生荣也就够了。\\

我的想法和他们不同。我首先就不相信这个优待条件能永远保留下去。不但如此,我比任何人都更能感到自己处境的危险。自从新的内战又发生,\xpinyin*{张作霖}败退出关,\xpinyin*{徐世昌}下台,\xpinyin*{黎元洪}重新上台,我就觉得危险突然逼近前来。我想的只是新的当局会不会加害于我,而不是什么优待不优待的问题。何况这时又有了某些国会议员主张取消优待的传说。退一万步说,就算现状可以维持,又有谁知道,在瞬息万变的政局和此起彼伏的混战中,明天是什么样的军人上台,后天是什么样的政客组阁?我从许多方面——特别是\xpinyin*{庄士敦}师傅的嘴里已经有点明白,这一切政局的变化,没有一次不是列强在背后起作用。与其等待民国新当局的优待,何不直接去找外国人?如果一个和我势不两立的人物上了台,再去想办法,是不是来得及?对于历代最末一个皇帝的命运,从\xpinyin*{成汤}放\xpinyin*{夏桀}于南巢,\xpinyin*{商纣}自焚于鹿台,\xpinyin*{犬戎}弑幽王于骊山之下起,我可以一直数到\xpinyin*{朱由检}上煤山。没有人比我对这些历史更熟悉的了。\\

当然,我没有向王公大臣们说起这些晦气的故事,我这样和他们辩论:\\

“我不要什么优待,我要叫百姓黎民和世界各国都知道,我不希望民国优待我,这倒比人家先取消优待的好。”\\

“优待条件载在盟府,各国公认,民国倘若取消,外国一定帮助我们说话。”他们说。\\

“外国人帮我们,你们为什么不叫我到外国去?难道他们见了我本人不更帮忙吗?”\\

尽管我说的很有道理,他们还是不同意。我和父亲、师傅。王公们的几次辩论,只产生这个效果:他们赶紧忙着筹办“大婚”。\\

我所以着急要出洋,除上面对王公大臣们说的理由之外,另外还有一条根本没有和他们提,特别是不敢向我的父亲提,这就是我对我周围的一切,包括他本人在内,越来越看不顺眼。\\

这还是在我动了出洋的念头以前就发生的。自从\xpinyin*{庄士敦}入宫以后,由于他给我灌输的西洋文明的知识,也由于少年人好奇心理的发展,我一天比一天不满意我的环境,觉得自己受着拘束。我很同意\xpinyin*{庄士敦}做出的分析,这是由于王公大臣们的因循守旧。\\

在这些王公大臣们眼里,一切新的东西都是可怕的。我十五岁那年。\xpinyin*{庄士敦}发现我眼睛可能近视,建议请个外国眼科医生来检验一下,如果确实的话,就给我配眼镜。不料这个建议竟像把水倒进了热油锅,紫禁城里简直炸开了。这还了得?皇上的眼珠子还能叫外国人看?皇上正当春秋鼎盛,怎么就像老头一样戴上“光子”(眼镜)?从太妃起全都不答应。后来费了\xpinyin*{庄士敦}不少口舌,加之我再三坚持要办,这才解决。\\

我所想要的,即使是王公大臣早得到的东西,他们也要反对,这尤其叫我生气。比如安电话那一次就是这样。\\

我十五岁那年,有一次听\xpinyin*{庄士敦}讲起电话的作用,动了我的好奇心,后来听\xpinyin*{溥杰}说北府(当时称我父亲住的地方)里也有了这个玩艺儿,我就叫内务府给我在养心殿里也安上一个。内务府大臣\xpinyin*{绍英}听了我的吩咐,简直脸上变了色。不过他在我面前向例没说过抵触的话,“嗻”了一声,下去了。第二天,师傅们一齐向我劝导:\\

“这是祖制向来没有的事,安上电话,什么人都可以跟皇上说话了,祖宗也没这样干过……这些西洋奇技淫巧,祖宗是不用的……”\\

我也有我的道理:“宫里的自鸣钟、洋琴、电灯,都是西洋玩艺,祖制里没有过,不是祖宗也用了吗?”\\

“外界随意打电话,冒犯了天颜,那岂不有失尊严?”\\

“外界的冒犯,我从报上也看了不少,眼睛看和耳朵听不是一样的吗?”\\

当时或者连师傅们也没明白,内务府请他们来劝驾是什么用意。内务府最怕的并不是冒犯“天颜”,而是怕我经过电话和外界有了更多的接触。在我身边有了一个爱说话的\xpinyin*{庄士敦},特别是有了二十来种报纸,已经够他们受的了。打开当时的北京报纸,几乎每个月至少有一起清室内务府的辟谣声明,不是否认清室和某省当局或某要人的来往,就是否认清室最近又抵押或变卖了什么古物。这些被否认的谣言,十有九件确有其事,至少有一半是他们不想叫我知道的。有了那些报纸,加上一个\xpinyin*{庄士敦},早已弄得他们手忙脚乱,现在又要添上个电话,作为我和外界的第三道桥梁,岂不更使他们防不胜防?因此他们使尽力气来反对。看师傅说不服我,又搬来了王爷。\\

我父亲这时已经成了彻底的维持现状派,只要我老老实实住在紫禁城里,他每年照例拿到他的四万二千四百八十两岁银,便一切满足,因此他是最容易受内务府摆布的人。但是这位内务府的支持者,并没有内务府所希望的那种口才。他除重复了师傅们的话以外,没有任何新的理由来说服我,而且叫我一句话便问得答不上来了:\\

“王爷府上不是早安上电话了吗?”\\

“那是,那是,可是,可是跟皇帝并不一样。这件事还是过两天,再说吧……”\\

我想起他的辫子比我剪得早,电话先安上了,不让我买汽车而他却买了,我心里很不满意。\\

“皇帝怎么不一样?我就连这点自由也没有?不行,我就是要安!”我回头叫太监:“传内务府:今天就给我安电话!”\\

“好,好,”我父亲连忙点头,“好,好,那就安……”\\

电话安上了,又出了新的麻烦。\\

随着电话机,电话局送来了一个电话本。我高兴极了,翻着电话本,想利用电话玩一玩。我看到了京剧名演员\xpinyin*{杨小楼}的电话号码,对话筒叫了号。一听到对方回答的声音,我就学着京剧里的道白腔调念道:“来者可是\xpinyin*{杨}——\xpinyin*{小}——\xpinyin*{楼}——呵?”我听到对方哈哈大笑的声音,问:“您是谁呵?哈哈……”不等他说完,我就把电话挂上了。真是开心极了。接着,我又给一个叫\xpinyin*{徐狗子}的杂技演员开了同样的玩笑,又给东兴楼饭庄打电话,冒充一个什么住宅,叫他们送一桌上等酒席。这样玩了一阵,我忽然想起\xpinyin*{庄士敦}刚提到的\xpinyin*{胡适}博士,想听听这位“匹克尼克来江边”的作者用什么调儿说话,又叫了他的号码。巧得很,正是他本人接电话。我说:\\

“你是胡博士呵?好极了,你猜我是谁?”\\

“您是谁呵?怎么我听不出来呢?……”\\

“哈哈,甭猜啦,我说吧,我是\xpinyin*{宣统}阿!”\\

“\xpinyin*{宣统}?……是皇上?”\\

“对啦,我是皇上。你说话我听见了,我还不知道你是什么样儿。你有空到宫里来,叫我瞅瞅吧。”\\

我这无心的玩笑,真把他给引来了。据\xpinyin*{庄士敦}说,\xpinyin*{胡适}为了证实这个电话,特意找过了\xpinyin*{庄士敦},他没想到真是“皇上”打的电话。他连忙向\xpinyin*{庄士敦}打听了进宫的规矩,明白了我并不叫他磕头,我这皇上脾气还好,他就来了。不过因为我没有把这件事放在心上,也没叫太监关照一下守卫的护军,所以胡博士走到神武门,费了不少口舌也不放通过。后来护军半信半疑请奏事处来问了我,这才放他进来。\\

这次由于心血来潮决定的会见,只不过用了二十分钟左右时间。我问了他白话文有什么用,他在外国到过什么地方,等等。最后为了听听他对我的恭维,故意表示我是不在乎什么优待不优待的,我很愿意多念点书,像报纸文章上常说的那样,做个“有为的青年”。他果然不禁大为称赞,说:“皇上真是开明,皇上用功读书,前途有望,前途有望!”我也不知道他说的前途指的是什么。他走了之后,我再没费心去想这些。没想到王公大臣们,特别是师傅们,听说我和这个“新人物”私自见了面,又像炸了油锅似地背地吵闹起来了。\\

总之,随着我的年事日长,他们觉得我越发不安分,我也觉得他们越发不顺眼。这时我已经出紫禁城玩过一两次,这是从我借口母亲去世要亲往祭奠开始,排除了无穷的劝阻才勉强争得来的一点自由。这点自由刺激了我的胃口,我越发感到这些喜欢大惊小怪的人物迂腐不堪。到民国十一年的夏季,上面说的几件事所积下的气忿,便促成了我下决心出洋的又一股劲头。我和王公大臣们的冲突,以正式提出留学英国而达到高峰。\\

这件事和安电话就不同了,王公大臣们死也不肯让步。最后连最同情我的七叔\xpinyin*{载涛},也只允许给我在天津英租界准备一所房子,以供万一必要时去安身。我因为公开出紫禁城不可能,曾找\xpinyin*{庄士敦}帮忙。在上节我已说过,他认为时机不相宜,不同意我这时候行动。于是我就捺下性子等候时机,同时暗中进行着私逃的准备。我这时有了一个忠心愿意协助我的人,这就是我的弟弟\xpinyin*{溥杰}。\\

我和\xpinyin*{溥杰},当时真是一对难兄难弟,我们的心情和幻想,比我们的相貌还要相似。他也是一心一意想跳出自己的家庭圈子,远走高飞,寻找自己的出路,认为自己的一切欲望,到了外国都可以得到满足。他的环境和我的比起来,也像他的身体和我的身体比例一样,不过只小了一号。下面是他的自传的一段摘录:\\

\begin{quote}
	到二十岁离开为止,我的家庭一直是一个拥有房屋数百间、花园一大座、仆役七八十名的“王府”。家中一直使用\xpinyin*{宣统}年号,逢年过节还公然穿戴清朝袍褂,带着护卫、听差大摇大摆地走在街上。平日家庭往来无白丁,不是清朝遗老就是民国新贵……\\

四岁断乳,一直到十七岁,每天早晨一醒来,老妈子给穿衣服,自己一动不动,连洗脚剪指甲自己也不干,倘若自己拿起剪刀,老妈便大呼大叫,怕我剪了肉。平时老妈带着,不许跑,不许爬高,不许出大门,不给吃鱼怕卡嗓子,不给……\\

八岁开读。塾师是\xpinyin*{陈宝琛}介绍的一位贡生,姓赵,自称是宋太祖的嫡系后裔,工褚字。老师常声泪俱下地讲三纲五常,大义名分。十三四岁,老师开始骂民国,称革命党人“无父无君”。说中国除非“定于一”才有救,\\

军阀混战是由于群龙无首。激发我“恢复祖业”,以天下为己任的志气。\\

“英国灭了印度,印度王侯至今世袭不断,日本吞并朝鲜,李王一家现在也仍是殿下……”父亲常和我这样念叨。\\

母亲死前对我说,“你长大后好好帮助你哥哥,无论如何不可忘记你是\xpinyin*{爱新觉罗}的子孙,这样你才对得起我……”\\

时常听说满族到处受排斥,皇族改姓金,\xpinyin*{瓜尔佳}氏改姓关,不然就找不到职业。听到这些,心中充满了仇恨。\\

十四五岁时,祖母和父亲叫我把私蓄几千元存到银行吃息钱。自己研究结果,还是送外国银行好,虽然息钱太低,可是保险。\\

十四岁起,入宫伴读。……\\
\end{quote}

\xpinyin*{溥杰}比我小一岁,对外面的社会知识比我丰富,最重要的是,他能在外面活动,只要借口进宫,就可以骗过家里了。我们行动的第一步是筹备经费,方法是把宫里最值钱的字画和古籍,以我赏赐\xpinyin*{溥杰}为名,运出宫外,存到天津英租界的房子里去。\xpinyin*{溥杰}每天下学回家,必带走一个大包袱。这样的盗运活动,几乎一天不断地干了半年多的时间。运出的字画古籍,都是出类拔萃、精中取精的珍品。因为那时正值内务府大臣和师傅们清点字画,我就从他们选出的最上品中挑最好的拿。我记得的有王羲之、王献之父子的墨迹《曹娥碑》、《二谢帖》,有锺繇、僧怀素、欧阳询、宋高宗、米芾、赵孟頫、董其昌等人的真迹,有司马光的《资治通鉴》的原稿,有唐王维的人物,宋马远和夏珪以及马麟等人画的《长江万里图》,\xpinyin*{张择端}的《清明上河图》,还有\xpinyin*{阎立本}、宋徽宗等人的作品。古版书籍方面,乾清宫西昭仁殿的全部宋版明版书的珍本,都被我们盗运走了。运出的总数大约有一千多件手卷字画,二百多种挂轴和册页,二百种上下的宋版书。民国十三年我出宫后,“清室善后委员会”在点查\xpinyin*{毓庆宫}的时候,发现了“赏\xpinyin*{溥杰}单”,付印公布,其中说赏\xpinyin*{溥杰}的东西“皆属琳琅秘籍,缥细精品,天禄书目所载,宝籍三编所收,择其精华,大都移运宫外”,这是一点不错的。这批东西移到天津,后来卖了几十件。伪满成立后,日本关东军参谋\xpinyin*{吉冈安直}又把这些珍品全部运到了东北,日本投降后,就不知下文了。\\

我们的第二步计划,是秘密逃出紫禁城。只要我自己出了城,进到外国公使馆,就算术已成舟,不管是王公大臣还是民国当局,就全没有办法了,这是几年来的民国历史给了我们的一个最有用的知识。更重要的是,我的\xpinyin*{庄士敦}师傅给我想出了更具体的办法,他叫我先和公使团的首席公使荷兰的\xpinyin*{欧登科}联络好,好使他事先有所准备。庄师傅给我出这个主意已是民国十二年的二月了。九个月前他曾反对我出洋,认为时机不好,现在他何以认为时机已经到来,以及他另外和东交民巷的公使们如何商量的,我一点都不知道。我从他的指点上获得了很大的信心,这就很够我满足的了。我先请他代往公使那里通个消息,然后我亲自给\xpinyin*{欧登科}公使直接通了电话,为了把事情办得稳妥,我又派\xpinyin*{溥杰}亲自到荷兰公使馆去了一趟。结果一切都是满意的。\xpinyin*{欧登科}在电话里答应了我,并亲自和\xpinyin*{溥杰}约定好,虽然他不能把汽车一直开进宫里,但将在神武门外等我,只要我能溜出这个大门,那就一切不成问题;从我第一天的食宿到我的脚踏上英国的土地,进入英国学校的大门,他全可以负责。当下我们把出宫的具体日期钟点都规定好了。\\

到了二月二十五日这天,剩下的问题就是如何走出神武门了。紫禁城里的情形是这样,我身边有一群随身太监,各宫门有各宫门的太监,宫廷外围是护军的各岗哨,神武门外,还有由民国步兵统领指挥的“内城守卫队”巡逻守卫。我认为,最重要的是身边和宫门太监,只要这几关打通,问题就不大了。我想的实在是太简单了,我打通太监的办法,也不过是花点钱而已。拿到钱的太监欢天喜地地谢了恩,我就认为万事俱备,谁知在预定时间前一小时,不知是哪个收了钱的太监报知了内务府。我还没走出养心殿,就听说王爷传下令来,叫各宫门一律断绝出入,紫禁城全部进入戒严状态。我和\xpinyin*{溥杰}一听这消息,坐在养心殿里全傻了眼。\\

过了不大功夫,我父亲气急败坏地来了:\\

“听听听听说皇上,要要要走……”\\

看他这副狼狈的样子,做错事的倒好像是他,我忍不住笑起来了。\\

“没有那么回事。”我止住了笑说。\\

“这可不好,这可怎么好……”\\

“没那回事!”\\

我父亲疑心地瞅瞅\xpinyin*{溥杰},\xpinyin*{溥杰}吓得低下了头。\\

“没有那事儿!”我还这样说。父亲嘟嘟囔囔说了几句,然后领走了我的“同谋犯”。他们走了,我把御前太监叫来追问,是谁说出去的。我非要把泄底的打个半死不可。可是我没办法问出来,这件事,又不能叫敬事房去查,只好一个人生闷气。\\

从那以后,我最怕看见高墙。\\

“监狱!监狱!监狱!”我站在堆秀山上望着城墙,常常这么念叨。“民国和我过不去还犹可说,王公大臣、内务府也和我过不去,真是岂有此理。我为了城外的祖业江山才要跑出去的,你们为了什么呢?……最坏的是内务府,这准是他们把王爷弄来的!”\\

第二天见了\xpinyin*{庄士敦},我向他发了一顿牢骚。他安慰了我几句,说不如暂时不去想这些,还是现实一些,先把紫禁城整顿整顿。\\

“新来的\xpinyin*{郑孝胥},是个很有为的人。”他说,“郑很有抱负,不妨听听他对整顿的想法。”\\

我心中又燃起另一种希望。既然紫禁城外祖业不能恢复,就先整顿城里的财产吧。我对庄师傅的建议非常满意。我那时万想不到,他后来在他那本书里写到这次逃亡时,竟然把自己说成了毫无干系,而且还是个反对者呢。
