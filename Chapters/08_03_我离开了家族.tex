\fancyhead[LO]{{\scriptsize 1950-1954: 由抗拒到认罪 · 我离开了家族}} %奇數頁眉的左邊
\fancyhead[RO]{} %奇數頁眉的右邊
\fancyhead[LE]{} %偶數頁眉的左邊
\fancyhead[RE]{{\scriptsize 1950-1954: 由抗拒到认罪 · 我离开了家族}} %偶數頁眉的右邊
\chapter*{我离开了家族}
\addcontentsline{toc}{chapter}{\hspace{1cm}我离开了家族}
\thispagestyle{empty}
为什么把我和家族分开?我到很晚才明白过来,这在我的改造中,实在是个极其重要的步骤,可是在当时,我却把这看做是共产党跟我势不两立的举动。我认为这是要向我的家族调查我过去的行为,以便对我进行审判。\\

我被捕之后,在苏联一贯把自己的叛国行为说成是迫不得已的,是在暴力强压之下进行的。我把跟\ruby{土肥原}{\textcolor{PinYinColor}{どいはら}}的会谈改编成武力绑架,我把勾结日本帝国主义的行为和后来种种谄媚民族敌人的举动全部掩盖起来。知道底细的家族成员们一律帮我隐瞒真相,哄弄苏联人。现在回到了中国,我更需要他们为我保密,我必须把他们看管好,免得他们失言,说出不该说的话来。特别是小秀,更需加意防范。\\

到抚顺的第一天,我就发现小秀因为火车上的那点“\xpinyin*{眶毗}之仇”,态度有些异样。那天我进了监房不久,忽然觉着有什么东西在脖子上爬,忙叫小秀给我看看。要是在以往,他早就过来了,可是那天他却装作没听见,一动不动。不但如此,后来小瑞过来,从我脖子后头找到一个小毛虫,扔在地上,小秀在旁边还哼了一声:“现在还放生,放了生叫它害别人!”我听了,浑身都觉着不是劲。\\

过了几天,小瑞给我整理被褥,我叫他把被子抖一抖。这个举动很不得人心,把屋里抖得雾气腾腾。\ruby{溥杰}{\textcolor{PinYinColor}{Pu Giye}}鼓着嘴,躲到一边去了,小固捂着鼻子对小瑞说:“行行好吧。呛死人啦!”小秀则一把抓过被子,扔到铺上说:“这屋子里不只你们住着,别人也住着!为了你们就不顾别人,那可不行。”我沉下了脸,问道:“什么你们我们?你还懂规矩吗?”他不回答我,一扭头坐在桌子旁,闷着头不说话。过了一会儿,我看见他噘着嘴使劲在纸上画,想看看他画什么,不料刚走过去,他拿起纸来就扯了。恍惚之间,我看到了一行字:“咱们走着瞧!”\\

我想起了火车上的那回事,尝到了自作自受的后悔滋味。从这天起,我尽力向他表示好感,拿出和颜悦色对他。我找了个机会,单独向他解释了火车上那回事,并非出于什么恶意,我对他一向是疼爱的。此后,一有机会我就对三个侄子大谈伦常之不可废,大难当前,和衷共济之必要。当小秀不在跟前的时候,我更嘱咐别人:“对小秀多加小心!注意别让他有轨外行动!多哄哄他!”\\

经过一番努力,小秀没发生什么问题,后来报上那篇文章在我们脑子里引起了幻想,小秀的态度也完全正常了。可是我对他刚放下心,就调整监房了,看守员叫我一个人搬到另一间屋子里去。\\

小瑞和小固两人替我收拾起铺盖、皮箱,一人替我拿一样,把我送到新屋子。他们放下东西走了。我孤零零地站在一群陌生人面前,感到非常别扭,简直坐也不是,站也不是。这屋子里原来住着八个人,见我进来,都沉默不语,态度颇为拘谨。后来,大概是经过一致默契,有人把我的铺盖接过去,安放在靠近墙头的地方。以后我才明白,这个地方是冬暖夏凉的地方,冬天得暖气,夏天有窗户。我当时对这些好意连同他们的恭敬脸色全没注意,心里只想着这次分离对我的危险。我默默地坐了一会儿,觉得这里连板炕都似乎特别硬。我站起来,抱着胳臂踱开了。\\

我踱了一阵,想出一个主意,就走到房门前,敲了几下门板。\\

“什么事?”一位矮墩墩的看守员打开门问。\\

“请问先生,我能不能跟所长先生谈一件事?”\\

“哪类的事?”\\

“我想说说,我从来没跟家里人分开过,我离开他们,非常不习惯。”\\

他点点头,叫我等一等。他去了一会儿,回来说所长准许我搬回去。\\

我高兴极了,抱起铺盖,看守员帮我提上箱子,便往回里走。在甬道里,我碰见了所长。\\

“为照顾你和年岁大些的人,所里给你们定的伙食标准比较高些,”所长说,“考虑到你们住一起用不同的伙食,恐怕对他们有影响,所以才……”\\

我明白了所长原来是这样考虑的,不等他说完,就连忙说:“不要紧,我保险他们不受影响。”我差点说出来:“他们本来就该如此!”\\

所长微微一笑:“你想的很简单。你是不是也想过,你自己也要学一学照顾自己?”\\

“是的,是的,”我连忙说,“不过,我得慢慢练,一点一点地练……”\\

“好吧,”所长点头说,“你就练练吧。”\\

我回到家里人住的那间屋子,觉得分别了半天,就像分别了一年似的。见了面,大家都很高兴。我告诉了他们所长说要我“练一练”的话,大家从这句话里觉出政府似乎不急于处理我的意思,就更高兴了。\\

然而家里人并没有让我去练,我自己也不想去练。我只考虑所长那番话的意思,迟早还会叫我们分开,因此必须好好地想出个办法来应付这个问题。我竟没想到,所长给的时间是这样短,才过了十天,我的办法还没想好,看守员就又来叫我收拾铺盖了。\\

我决定趁小瑞给我收拾东西的时间,对家族嘱咐几句。因为怕门外的看守员听见,不好用嘴说,就写了一个纸条;又因屋子里这时多了两个汪伪政权的人,所以纸条写得特别含蓄。大意是;我们相处得很好,我走后仍要和衷共济,我对你们每人都很关怀。写罢,我交给\ruby{溥杰}{\textcolor{PinYinColor}{Pu Giye}},叫他给全体传阅。我相信他们看了,必能明白“和衷共济”的意思是不要互相乱说。我相信两个汪伪政权的人对我的举动并没有发生怀疑。\\

我的侄子又给我抱着铺盖提着箱子,把我送进上次那间屋子,人们又把我的铺盖接过去,安放在那个好地方。跟上次一样,我在炕上坐不住,又抱着胳臂踱了一阵,然后去敲门板。\\

还是那个矮墩墩的看守员打开了门。我现在已知道他姓刘,而且对他有了一些好感。这是由吃包子引起的。不久前,我们第一次吃包子,大家吃得特别有味,片刻间全吃光了。刘看守员觉着这件事很新鲜,笑着走过来,问我们够不够。有人不说话,有人吞吞吐吐地说“够了”。他说:“怎么忸忸怩怩的,要吃饱嘛!”说着,一阵风似地走了,过了一会儿,一桶热腾腾包子出现在我们的房门口。我觉得这个人挺热心,跟他说出我的新主意,谅不至于出岔子。\\

“刘先生,我有件事……”\\

“找所长?”他先说了。\\

“我想先跟刘先生商量一下,我,我……”\\

“还是不习惯?”他笑了。这时我觉出背后也似乎有人在发笑,不禁涨红了脸,连忙辩解说:\\

“不,我想说的不是再搬回去。我想,能不能让我跟家里人每天见一面。只要能见见,我就觉着好得多了。”\\

“每天在院里散步,不是可以见吗?这有什么问题?”\\

“我想跟他们在一起说说话儿,所长准许吗?”按照规定,不同监房是不得交谈的。\\

“我给你问问去。”\\

我得到了准许。从这天起,我每天在院子里散步时都能和家里人见一次面,说一会儿话儿。几个侄子每天都告诉我一点关于他们屋里的事情,所里的人跟他们说了什么,他们也照样告诉我。从接触中,小固还是那样满不在乎,小秀也没什么异样,小瑞仍然恭顺地为我洗衣服、补袜子。\\

我所担心的问题得到了解决,不想新的问题出现了。这就是,过去四十多年的“饭来张口、衣来伸手”的生活习惯,现在给我带来极大的苦恼。\\

四十多年来,我从来没叠过一次被,铺过一次床,倒过一次洗脸水。我甚至没有给自己洗过脚,没有给自己系过鞋带。像饭勺、刀把、剪子、针线这类东西,从来没有摸过。现在一切事都要我亲自动手,使我陷入了十分狼狈的境地。早晨起来,人家早已把脸洗完了,我才穿上衣服,等到我准备去洗脸了,有人提醒我应该先把被叠好;等我胡乱地卷起被子,再去洗脸,人家早洗完了;我漱口的时候,已经把牙刷放进嘴里,才发现没有蘸牙粉,等我把这些事情都忙完了,人家早饭都快吃完了。我每天总是跟在别人后面,忙得昏头胀脑。\\

仅仅是忙乱,倒还罢了,更恼人的是同屋人的暗笑。同屋的八个人,都是伪满的将官,有“军管区司令”、“旅长”,也有“禁卫军团长”,他们从前在我面前都是不能抬头的人物。我初到这间屋子的时候,他们虽然不像我的家族那样偷着叫我“上边”,但“你”字还不敢用,不是称我为“先生”,就是索性把称呼略掉,以表示对我的恭敬。这时他们的耻笑虽不是公然的,但是他们那种故做不看、暗地偷看的表情,常常让我感到格外不好受。\\

让我感到很不好受的还不仅限于此。我们从到抚顺的第一天起,各个监房都建立了值日制度,大家每天轮流打扫地板、擦洗桌子和倒尿桶。没跟家族分开时,这些事当然用不着我来干。我搬进了新屋之后,难题就来了,轮到我值日那天该怎么办呢?我也去给人倒尿桶?我跟日本关东军订立密约的时候,倒没觉得怎样,而现在把倒尿桶却当成了上辱祖宗、下羞子侄的要命事。幸好所方给我解了围,第二天,所方一位姓贾的干部走来对大家说:“\ruby{溥仪}{\textcolor{PinYinColor}{Pu I}}有病,不用叫他参加值日了!”我听到这句话,犹如绝路逢生,心中第一次生出了感激之情。\\

值日的事解决了,不想又发生了一件事。有一天,我们正在院子里三三两两地散步,所长出现了。我们每次散步他必定出现,而且总要找个犯人谈几句。这次我发现他注意到了我。他把我从上到下打量了一阵,打量得我心里直发毛。\\

“\ruby{溥仪}{\textcolor{PinYinColor}{Pu I}}!”他叫了一声。我从回国之后,开始听别人叫我的名字,很觉不习惯,这时仍感到刺耳,觉得还不如听叫号码好受。来这里的初期,看守员一般总是叫号码的(我的号码是“981”)。\\

“是,所长。”我走了过去。\\

“你的衣服是跟别人一块发的,怎么你这一身跟别人的不一样?”他的声调很和气。\\

我低头看看自己的衣服,再看看别人,原来别人身上整整齐齐,干干净净,而我的却是褶褶囊囊,邋里邋遢:口袋扯了半边,上衣少了一只扣子,膝盖上沾了一块蓝墨水,不知怎么搞的,两只裤腿也好像长短不一,鞋子还好,不过两只鞋只有一根半鞋带。\\

“我这就整理一下,”我低声说,“我回去就缝口袋、钉扣子。”\\

“你衣服上的褶子是怎么来的呢?”所长微笑着说,“你可以多留心一下,别人怎么生活。能学习别人的长处,才能进步。”\\

尽管所长说得很和婉,我却觉得很难堪,很气恼。我这是第一次被人公开指出我的无能,这是我第一次不是被当做尊严的形象而是作为“废物”陈列在众目注视之下。“我成了大伙研究的标本啦!”我难受地转过身,避开“大臣”和“将官”们的目光,希望天色快些暗下来。\\

我溜到墙根底下,望着灰色的大墙,心中感慨万千:我这一生一世总离不开大墙的包围。从前在墙里边,我还有某种尊严,有我的特殊地位,就是在长春的小圈子里,我也保持着生活上的特权,可是如今,在这个墙里,那一切全没有了,让我跟别人一样,给我造成了生存上的困难。总之一句话,我这时不是因感到自己无能而悲哀,而是由于被人看做无能而气恼。或者说,我不是怪自己无能,而是怨恨我一向认为天生应该由人来服侍的特权的丧失。我因免于值日而对所方发生的感激之情,这时一下子全消失了。\\

这天晚上,我发现了别人临睡时脱下衣服,都整整齐齐地叠好、放在枕头底下,而我却一向是脱下来顺手一团,扔到脚底下的。我想起所长说的话,确有几分道理,应该注意一下别人的长处,——我如果早知道这点的话,今天不是就不会碰到这种难堪了吗?我对伙伴们产生了不满,他们为什么对我这样“藏奸”,不肯告诉我呢?\\

其实,那些伪将官们连向我说话还感到拘谨,我既然不肯放下架子去请教,谁还敢先向我指指点点呢?\\

我就是这样的在抚顺度过了两个多月。十月末,管理所迁往哈尔滨,我们便离开了抚顺。