\fancyhead[LO]{{\scriptsize 1931-1932: 到东北去 · 所见与所思}} %奇數頁眉的左邊
\fancyhead[RO]{} %奇數頁眉的右邊
\fancyhead[LE]{} %偶數頁眉的左邊
\fancyhead[RE]{{\scriptsize 1931-1932: 到东北去 · 所见与所思}} %偶數頁眉的右邊
\chapter*{所见与所思}
\addcontentsline{toc}{chapter}{\hspace{1cm}所见与所思}
\thispagestyle{empty}
我到旅顺以后,感到最惶惑不安的,倒不是因为受到封锁。隔离,而是从\ruby{上角}{うえすみ}这几个日本人口中听到,关东军似乎连新国家的国体问题还没定下来。\\

这对我说来,比没有人在码头上迎接我更堵心。没有人迎接,还可以用“筹备不及”、“尚未公布”的话来解释。“国体未定”又是怎么回事呢?国体既然未定,\ruby{土肥原}{どいはら}干么要请我到满洲来呢?\\

\xpinyin*{郑孝胥}和\ruby{上角}{うえすみ}向我解释说,\ruby{土肥原}{どいはら}没有说谎,关东军支持我复位和主持大计的话全不错,不过这是满洲的事,当然还要和满洲人商量,没有商量好以前,自然叫做“未定”。\\

我已经不像在汤岗子那样容易相信这些人了,但我又找不到任何别人商议事情。这还是我第一次离开我的师傅。在没师傅指点的情形下,我只好采取\xpinyin*{商衍瀛}的办法,找神仙帮忙来解答问题。我拿出从天津带来的一本《未来预知术》,摇起了金钱神课。记得我摇出了一课“乾乾”卦,卦辞还算不坏。于是我就这样的在\xpinyin*{郑孝胥}、\xpinyin*{罗振玉}和\xpinyin*{诸葛亮}\footnote{《未来预知术》是香港出版的一本迷信书,伪称是\xpinyin*{诸葛亮}的著作,可是其中的封辞中有汉代以后的诗文典故。}的一致劝导下,捺着性子等待下去。\\

有一天,\ruby{上角}{うえすみ}来问我,是不是认识\xpinyin*{马占山}。我说在天津时,他到张园来过,算是认识吧。\ruby{上角}{うえすみ}说,\ruby{板垣}{いたがき}希望我能写一封信,劝\xpinyin*{马占山}归顺。我说在天津时已曾写过一封,如果需要,还可以再写。这第二封劝降书并没有用上,\xpinyin*{马占山}就投降了。虽然我的信未发生作用,可是关东军请我写信这件事给了我一种安慰,我心里这样解释:这显然是日本人承认我的威信,承认这块江山必须由我统治才行。我是谁呢,不就是大清的皇帝吗?这样一想,我比较安心了些。\\

这样等了三个月,到我过生日的第二天,即一九三二年二月十九日,忽然来了一个消息,刚刚复会的“东北行政委员会”通过了一项决议,要在满洲建立一个“共和国”。所谓东北行政委员会是二月十八日复会的,这个委员会由投降的原哈尔滨特区长官\xpinyin*{张景惠}、辽宁(这时被改称奉天)省主席\xpinyin*{臧式毅}、黑龙江省代理主席\xpinyin*{马占山}和被这委员会追认的吉林省主席\xpinyin*{熙洽}组成,\xpinyin*{张景惠}为委员长。二月十九日,这个委员会在\ruby{板垣}{いたがき}导演下通过了那项决议,接着又发表了一个“独立宣言”。这些消息传来之后,除了郑氏父子以外,我身边所有的人,包括\xpinyin*{罗振玉}在内无不大起恐慌,人人愤慨。\\

这时占据着我全心的,不是东北老百姓死了多少人,不是日本人要用什么办法统治这块殖民地。它要驻多少兵,要采什么矿,我一概不管,我关心的只是要复辟,要他们承认我是个皇帝。如果我不为了这点,何必千里迢迢跑来这里呢?我如果不当皇帝,我存在于世上还有什么意义呢?\xpinyin*{陈宝琛}老夫子以八十高龄的风烛残年之身来到旅顺时,曾再三对我说:“若非复位以正统系,何以对待列祖列宗在天之灵!”\\

我心中把\ruby{土肥原}{どいはら}、\ruby{板垣}{いたがき}恨得要死。那天我独自在前肃亲王的客厅里像发了疯似地转来转去,纸烟被我捏断了一根又一根,《未来预知术》被我扔到地毯上。我一下子想起了我的静园,想到假如我做不成皇帝,还不如去过舒适的寓公生活,因为那样我还可以卖掉一部分珍玩字画,到外国去享福。这样一想,我有了主意,我要向关东军表明态度,如果不接受我的要求,我就回天津去。我把这主意告诉了\xpinyin*{罗振玉}和\xpinyin*{郑孝胥},他们都不反对。\xpinyin*{罗振玉}建议我先送点礼物给\ruby{板垣}{いたがき},我同意了,便从随身带的小件珍玩中挑了几样叫他去办。恰好这时\ruby{板垣}{いたがき}来电话请郑罗二人去会谈,于是我便叫\xpinyin*{陈曾寿}为我写下必须“正统系”的理由,交给他们带给\ruby{板垣}{いたがき},叫他们务必坚持,向\ruby{板垣}{いたがき}说清楚我的态度。\\

我写的那些理由共十二条(后四条是\xpinyin*{陈曾寿}续上的):\\

\begin{quote}
	一、尊重东亚五千年道德,不得不正统系。\\

二、实行王道,首重伦常纲纪,不得不正统系。\\

三、统驭国家,必使人民信仰钦敬,不得不正统系。\\

四、中日两国为兄弟之邦,欲图共存共荣,必须尊崇固有之道德,使两国人民有同等之精神,此不得不正统系。\\

五、中国遭民主制度之害已二十余年,除少数自私自利者,其多数人民厌恶共和,思念本朝,故不得不正统系。\\

六、满蒙人民素来保存旧习惯,欲使之信服,不得不正统系。\\

七、共和制度日炽,加以失业人民日众,与日本帝国实有莫大之隐忧;若中国得以恢复帝制,于两国人民思想上。精神上保存至大,此不得不正统系。\\

八、大清在中华有二百余年之历史,(入关前)在满洲有一百余年之历史,从人民之习惯,安人民之心理,治地方之安靖,存东方之精神,行王政之复古,巩固贵国我国之皇统,不得不正统系。\\

九、贵国之兴隆,在\ruby{明治}{めいじ}大帝之王政。观其训谕群工,莫不推扬道德,教以忠义。科学兼采欧美,道德必本诸孔孟,保存东方固有之精神,挽回孺染欧风之弊习,故能万众人心亲上师长,保护国家,如手足之捍头目。此予之所敬佩者。为起步\ruby{明治}{めいじ}大帝,不能不正统系。\\

十、蒙古诸王公仍袭旧号,若行共和制度,欲取消其以前爵号,则因失望而人心涣散,更无由统制之,故不能不正统系。\\

十一、贵国扶助东三省,为三千万人民谋幸福,至可感佩。惟子之志愿,不仅在东三省之三千万人民,实欲以东三省为张本,而振兴全国之人心,以救民于水火,推至于东亚共存共荣,即贵国之九千万人民皆有息息相关之理,两国政体不得歧异。为振兴两国国势起见,不得不正统系。\\

十二、予自\xpinyin*{辛亥}逊政,退处民间,今已二十年矣,毫无为一己尊崇之心,专以救民为宗旨。只要有人出而任天下之重,以正道挽回劫运,子虽为一平民,亦所欣愿。若必欲予承之,本个人之意见,非正名定分,实有用人行政之权,成一独立国家,不能挽回二十年来之弊政。否则有名无实,诸多牵制,毫无补救于民,如水益深,如火益热,徒负初心,更滋罪\xpinyin*{戾},此万万不敢承认者也。倘专为一己尊荣起见,则二十年来杜门削迹,一旦加之以土地人民,无论为总统,为王位,其所得已多,尚有何不足之念。\\
\end{quote}

实以所主张者纯为人民,纯为国家,纯为中日两国,纯为东亚大局起见,无一毫私利存乎其间,故不能不正统系。\\

\xpinyin*{郑孝胥}知道,这次沈阳之行是决定自己命运的关键。因为关东军在叫东北行政委员会通过“国体”之前,要先排定一下“开国元勋”们的位置。因此,他在动身之前,对我尽量表示顺从,以免引起我对他发生戒心。但是等到他的目的已经达到,从沈阳返回来的时候,那情形就变了。他劝我不要和关东军争论,劝我接受共和制,出任“执政”。\\

“什么执政?叫我当共和国的执政?”我跳了起来。\\

“这事已成定局,臣再三向军方争论无效。军方表示,执政即元首……”\\

我不理他,转身问\xpinyin*{罗振玉},这是怎么回事。\xpinyin*{罗振玉}说:“臣就见了\ruby{板垣}{いたがき}一面,是\xpinyin*{郑孝胥}跟\ruby{板垣}{いたがき}谈的。”\\

后来据\xpinyin*{陈曾寿}说,\xpinyin*{郑孝胥}父子根本没把我的十二条“正统系”给\ruby{板垣}{いたがき}拿出来,而且还向\ruby{板垣}{いたがき}保证:“皇上的事,我全可以包下来,”“皇上如同一张白纸,你们军部怎么画都行,”等等。当时我还不知道这回事,只认为他们不会办事,都受了日本人的骗。\\

“你们都没用!”我大声喊道,“你们为什么不说,我的要求达不到,我就回天津!”\\

“皇上还是再三思考为好。”\xpinyin*{郑孝胥}说,“复辟必须依赖日本,眼前与日本反目,将来的希望也完了。将来复辟不是没有希望呵!”\\

他又讲了一些历史故事,劝我答应,可是那些故事我早就听够了,再说无论是\xpinyin*{刘秀}还是\xpinyin*{重耳},也都没有放弃君主称号的。最后他说:\\

“下午\ruby{板垣}{いたがき}就来\xpinyin*{觐}见,请皇上对\ruby{板垣}{いたがき}说吧!”\\

“让他来!”我气呼呼地回答。
