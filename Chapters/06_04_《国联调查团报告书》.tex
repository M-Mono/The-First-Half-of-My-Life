\fancyhead[LO]{{\scriptsize 1932-1945: 伪满十四年 · 《国联调查团报告书》}} %奇數頁眉的左邊
\fancyhead[RO]{\thepage} %奇數頁眉的右邊
\fancyhead[LE]{\thepage} %偶數頁眉的左邊
\fancyhead[RE]{{\scriptsize 1932-1945: 伪满十四年 · 《国联调查团报告书》}} %偶數頁眉的右邊
\chapter*{《国联调查团报告书》}
\addcontentsline{toc}{chapter}{\hspace{11mm}《国联调查团报告书》}
%\thispagestyle{empty}
一九三二年五月,国联调查团来到了东北。十月,发表了所谓“满洲问题”的调查报告。郑氏父子对于这个调查团曾抱有很大幻想,报告书公布的时候,他们简直以为实现国际共管的理想是指日可待的。他父子俩后来失宠于日人,终于被抛弃,与这种热衷于共管有很大关系。我当时并没有他们想的那么多,没有他们那样兴奋,但却从他们的议论中,知道了不少国际上的事情。我与他们的感受也不同。他们因调查团的态度而发生了共管的幻想,而我却由此发生了对日本强大的感觉。由于这种感觉,我越发认为自己的命运是无法跟它分开了。\\

关于西方列强在“满洲事件”上的态度,我早就听郑氏父子等人不断说过这类的话:“别看日内瓦、巴黎(国联)开会开得热闹,其实哪一国也不打算碰日本,欧战以后有实力的是美国,可是连美国也不想跟日本动硬的。”精通英文、日文的\xpinyin*{郑垂}不时地把外国报纸上的\xpinyin*{舆}论告诉我,说美国不少报纸言论是担日的。他曾有根有据地说了一些非公开消息,例如美日曾有密约,美对日本在东北的行动有谅解,等等。他还很具体地告诉我,早在事变前美国方面的重要人物就劝过\xpinyin*{蒋介石},把满洲卖给日本,让日本去碰苏联,以收其利\footnote{事实上,喜欢吹牛的郑氏父子并没有撒谎。在当时的《东方杂志》上,就可以找到《纽约论坛报》、《纽约日日新闻》等报纸上的袒日言论的译文。比如,前者有这样的话:“日人军事行动,乃对中国废除不平等条约政策所不能免之反响”,后者:“日本继承俄国在满洲开发,至于今日,其功绩之伟大,为世人公认。”国联通过派遣调查团的决议,确曾遭受到美国的反对,理由是:“此种行动足以刺激日本国民的情绪”,国联在一次会议上,打算做出要求日军退出满洲的决议时,美国国务卿\ruby{凯塞尔}{\textcolor{PinYinColor}{Castle}}就公开表示,对此并未附议。这些事实的记载可以从当时的许多报刊上看到。后来美国国务院发表了一些秘密文件,其中《一九三一年美国外交文件》一书,公布了那年十一月二十七日美驻日大使\ruby{福白斯}{\textcolor{PinYinColor}{Forbes}}交给日本外务大臣\ruby{币原}{\textcolor{PinYinColor}{しではら}}的一份党书,透露了美国政府当时“曾劝中国政府采取妥协步调”。至于日美对东北问题的秘密谈判,则在一九三五年十二月号的《国际事件》上据\ruby{西}{\textcolor{PinYinColor}{Q.}}·\ruby{莱特}{\textcolor{PinYinColor}{Wright}}的一篇文章《美国人对远东问题的观点》中揭露了出来。}。\\

“调查团要来了,”\xpinyin*{郑孝胥}是这样告诉我的,“国民党请他们来调查,想请他们帮忙对付日本,其实他们是不对付日本的。他们关心的一是门户开放、机会均等,二是对付赤俄。他们在东京跟\ruby{内田}{\textcolor{PinYinColor}{うちだ}}\ruby{康哉}{\textcolor{PinYinColor}{こうさい}}(这时已出任日本外相)谈的就是这个。用不着担心,到时候应付几句就行了。依臣看来,国民党也明知道调查团办不了什么事,说不定国民党看到了国际共管满洲的好处。”\\

后来事实证明,郑氏父子说的话大部正确。\\

沈阳事变发生后,\xpinyin*{蒋介石}一再电\xpinyin*{张学良}转命东北驻军:“为免事件扩大,绝对不抵抗。”四天后,即九月二十二日,蒋在南京全市国民党员大会上宣称:“以公理对强权,以和平对野蛮,忍辱含愤,暂取逆来顺受态度,以待国际公理之判断。”同时,对内却毫无和平与公理,用最野蛮的办法加紧进行内战。九月三十日,国民党向国联请求派中立委员会到满洲调查。经过几番讨论,到十二月十日才得到日本同意,做出组织调查团的决议。调查团由五国委员组成,即英国的\ruby{李顿}{\textcolor{PinYinColor}{Lytton}}爵士、美国的\ruby{佛兰克}{\textcolor{PinYinColor}{Frank}}\ruby{洛斯}{\textcolor{PinYinColor}{Ross}}·\ruby{麦考益}{\textcolor{PinYinColor}{McCoy}}少将、法国的\ruby{亨利}{\textcolor{PinYinColor}{Henri}}·\ruby{克劳德}{\textcolor{PinYinColor}{Claudel}}中将、意大利的\ruby{格迪}{\textcolor{PinYinColor}{Count}}伯爵和德国的\ruby{恩利克}{\textcolor{PinYinColor}{Heinrich}}\ruby{希尼}{\textcolor{PinYinColor}{Schnee}}博士。团长是\ruby{李顿}{\textcolor{PinYinColor}{Lytton}}。一九三二年二月三日调查团启程,先在日本、上海、南京、汉口、九江、宜昌、重庆等地转了一圈,又在北平住了十天,到东北的时候已是五月份了。在这期间,南京政府宣传着“等待公理的判断”,而日军则攻占了锦州,发动了淞沪战争,成立了“满洲国”。除了这些被“等待”来的结果之外,还有一个在各国干预下产生的《淞沪停战协定》。根据这个协定,南京政府的军队从此不得进驻淞沪地区。\\

五月三日这天,我和调查团的会见,用了大约一刻钟左右的时间。他们向我提出了两个问题:我是怎么到东北来的?“满洲国”是怎么建立起来的?在回答他们的问题之前,我脑子里闪过一个大概他们做梦也没想到的念头。我想起当年\ruby{庄士敦}{\textcolor{PinYinColor}{Johnston}}曾向我说过,伦敦的大门是为我打开着的,如果我现在对\ruby{李顿}{\textcolor{PinYinColor}{Lytton}}说,我是叫\ruby{土肥原}{\textcolor{PinYinColor}{どいはら}}骗来又被\ruby{板垣}{\textcolor{PinYinColor}{いたがき}}威吓着当上“满洲国元首”的,我要求他们把我带到伦敦,他们肯不肯呢?我这个念头刚一闪过,就想起来身边还坐着关东军的参谋长\ruby{桥本}{\textcolor{PinYinColor}{はしもと}}\ruby{虎之助}{\textcolor{PinYinColor}{とらのすけ}}和高参\ruby{板垣}{\textcolor{PinYinColor}{いたがき}}\ruby{征四郎}{\textcolor{PinYinColor}{せいしろう}}。我不由地向那青白脸瞄了一眼,然后老老实实按照他预先嘱咐过的说:“我是由于满洲民众的推戴才来到满洲的,我的国家完全是自愿自主的……”\\

调查团员们一齐微笑点头,再没问什么。然后我们一同照相,喝香摈,祝贺彼此健康。调查团走后,\ruby{板垣}{\textcolor{PinYinColor}{いたがき}}的青白脸泛满了笑意,赞不绝口地说:“执政阁下的风度好极了,讲话响亮极了!”\xpinyin*{郑孝胥}事后则晃着秃头说:“这些西洋人跟臣也见过面,所谈都是机会均等和外国权益之事,完全不出臣之所料。”\\

这年十月,日本《中央公论》上刊出了\ruby{驹井}{\textcolor{PinYinColor}{こまい}}的一篇文章,\xpinyin*{郑垂}把译文送来不久,《调查团报告书》也到了我手里,这两样东西,给了我一个统一的印象,正如郑氏父子所判断的,调查团所关心的是“机会”与“门户”问题。\\

\ruby{驹井}{\textcolor{PinYinColor}{こまい}}的文章题为《满洲国是向全世界宣称着》,内容是他与\ruby{李顿}{\textcolor{PinYinColor}{Lytton}}等人会见的情形。现在\xpinyin*{郑垂}的译文已不可得,只有借助于一篇不高明的译文,是\xpinyin*{陈彬龢}编印的《满洲伪国》里的。文章中说,\ruby{李顿}{\textcolor{PinYinColor}{Lytton}}第一个向他提出问题:“满洲国的建设不稍嫌早些么?”他回答了一大套非但不早,且嫌其晚的鬼道理,然后是:\\

其次\ruby{麦考益}{\textcolor{PinYinColor}{McCoy}}将军问:“满洲国宣扬着门户开放主义,果真实行了么?”\\

我立即回答说:“门户开放和机会均等是满洲立国的铁则。门户开放政策,在昔围绕着中国的诸国中,美国是率先所说的精神。但这主义政策是列国之所倡,中国本身是抱着门户闭锁主义,我们果在中国的何处可以看到门户开放的事实?现在我们以极强的钥匙使满洲国门户开放,我们只有受诸君感谢,而没有受抗议的道理。……不过我须附带声明的,就是关于国防事业断不能门户开放,即在世界各国亦断无此例。”\\

\ruby{李顿}{\textcolor{PinYinColor}{Lytton}}再询问:“满洲国实行着机会均等么?”\\

我略不踌躇地说:“机会均等,贵国在中国已有其先例,即前清末叶,中国内政极度糜烂,几全失统一之际,\ruby{罗浮脱}{\textcolor{PinYinColor}{Robert}}·\ruby{赫德}{\textcolor{PinYinColor}{Hart}}提议清廷说,倘然长此以往,中国将完全失其作用于国际间,不如依赖西洋人,海关行政,亦有确定之必要。于是清朝立即任命\ruby{罗浮脱}{\textcolor{PinYinColor}{Robert}}·\ruby{赫德}{\textcolor{PinYinColor}{Hart}}为总税司,海关行政方得确立。由于海关上使用着许多英、法、日等国人,在中国被认为是最确实的行政机关,因此列强借款给中国,中国道得在财政上有所弥补。英国人曾以海关为施行机会均等之所,但是我们日本人,要想做这海关的事务员,则非受等于拒绝的严格的英语试验不可。\\

“……我们满洲国,是满洲国人和日本人协力而建设的国家,因之新国家的公文,均以满洲国语和日本语发表。所以任何国人,金能完全使用满日两国语言,并能以满洲国所给与之待遇为满足,则我们当大大的欢迎。\\

这就是我所说的机会均等。”\\

我继续着问:“你们各位还有旁的询问么?”\\

旁的人都说:“此外已无何等询问的必要了,我们已能充分理解了满洲国的立场,愉快之至!”\\

国联调查委员在离开新京时,我送到车站上,那时候\ruby{李顿}{\textcolor{PinYinColor}{Lytton}}握了我的手小声地说:“恭祝新满洲国之健全的发达!”同时用力地握了下手就分别了。\\

这次谈话,使\xpinyin*{郑孝胥}父子感到了极大的兴奋,\xpinyin*{郑垂}甚至估计到,国联很可能做出一个国际共管满洲的决议来。后来调查团的报告书公布出来,使郑氏父子更有了信心。调查团的报告书中所代表的国联,正是以郑氏父子所希望的那种中国的管理者的态度出现的。报告书明白地说:“目前极端之国际冲突事件,业经中国再度要国联之干涉。……中国遵循与国际合作之道,当能得最确定及最迅速之进步,以达到其国家之理想。”这位管理者明确地表示:日本“为谋满洲之经济发展,要求建设一能维持秩序之巩固政权,此项要求,我等亦不以为无理”。但是,这位管理者认为最重要的是,“惟有在一种外有信仰内有和平,而与远东现有情形完全不同之空气中,为满洲经济迅速发展所必要之投资始可源源而来”。这就是说,要有列强各国共同认定的那种“信仰”才行,这就是郑氏父子所向往的由各国共同经营,利益均沾的局面。\\

郑氏父子关于反苏问题的估计,也得到了证实。调查团说,它理解日本称满洲为其生命线之意义,同情日本对“其自身安全之顾虑”,因此,“日本之欲谋阻止满洲被利用为攻击日本之根据地,以及为在某种情形之下满洲边境被外国军队冲过时,日本欲有采取适当军事行动之能力,吾人均可承认”。不过调查团又认为,这样做法日本的财政负担必大,而且日本在满军队受时怀反侧之民众包围,其后又有包含敌意之中国,日本军队能否不受重大困难,亦殊难言。因此可以考虑另外的办法,则“日本甚或又因世界之同情与善意,不须代价而获安全保障较现时以巨大代价换得者为更佳”。调查团于是提出意见说,问题的解决,恢复原状和维持现状都不是令人满意的办法,认为只要“由现时(满洲国)组织毋须经过极端之变更或可产生一种满意之组织”,这就是实行“获得高度自治权”的“满洲自治”,由各国洋人充当这个自治政府的顾问;由于日本人在东北的权益大些,日本人比例也大些,但其他外国也要有一定比例。为实现这个新政体,“讨论和提出一种特殊制度之设立,以治理东三省之详密议案”,要先成立一个由国联行政院掌握最高决定权的、由中日双方和“中立观察员”组成的顾问委员会。调查团并且认为“国际合作”的办法不但适于“满洲”,也适于对全中国使用。其根据理由也是郑氏父子屡次表示过的,是因为中国只有劳动力,而资本、技术。人才全要靠外国人,否则是建设不起来的。\\

在刚看到报告书的那几天,\xpinyin*{郑孝胥}曾兴致勃勃地告诉过我,“事情很有希望”,说\xpinyin*{胡适}也在关内发表论文,称誉报告书为“世界之公论”。可是后来日本方面的反响到了,他父子大为垂头丧气。尽管调查团再三谈到尊重日本在满洲的权益,甚至把“九·一八”事变也说成是日本的自卫行为,日本的外务省发言人却只表示同意一点,就是:“调查团关于满洲的建议,大可施于中国与列强间的关系而获得研益,如制定国际共管计划者,是也!”至于对“满洲”本身的共管方案,根本不加理睬。\xpinyin*{郑孝胥}后来的失宠和被弃,即种因在对于“门户开放、机会均等”的热衷上。\\

在国联调查团的报告书发表之前,我曾经设想过,假如真的像郑氏父子希望的那样,将东北归为国际共管,我的处境可能比日本独占情形下好得多。但是,我还有两点不同的考虑;一是怕“共管”之中,南京政府也有一份,如果这样,我还是很难容身;另一点是,即使南京管不上我,国际共管也未必叫我当皇帝,如果弄出个“自治政府”来,那还有什么帝制?更重要的是,日本的横蛮,在国际上居然不受一点约束,给我的印象极为深刻。因此,事后我一想起了调查团会见时我心里闪过的那个念头,不禁暗暗想道:“幸亏我没有傻干,否则我这条命早完了。……现在顶要紧的还是不要惹翻了日本人,要想重登大宝,还非靠日本人不可呀!”
