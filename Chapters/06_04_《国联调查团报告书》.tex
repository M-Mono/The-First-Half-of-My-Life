\fancyhead[LO]{{\scriptsize 1932-1945: 伪满十四年 · 《国联调查团报告书》}} %奇數頁眉的左邊
\fancyhead[RO]{} %奇數頁眉的右邊
\fancyhead[LE]{} %偶數頁眉的左邊
\fancyhead[RE]{{\scriptsize 1932-1945: 伪满十四年 · 《国联调查团报告书》}} %偶數頁眉的右邊
\chapter*{《国联调查团报告书》}
\addcontentsline{toc}{chapter}{\hspace{1cm} 《国联调查团报告书》}
\thispagestyle{empty}
一九三二年五月,国联调查团来到了东北。十月,发表了所谓“满洲问题”的调查报告。郑氏父子对于这个调查团曾抱有很大幻想,报告书公布的时候,他们简直以为实现国际共管的理想是指日可待的。他父子俩后来失宠于日人,终于被抛弃,与这种热衷于共管有很大关系。我当时并没有他们想的那么多,没有他们那样兴奋,但却从他们的议论中,知道了不少国际上的事情。我与他们的感受也不同。他们因调查团的态度而发生了共管的幻想,而我却由此发生了对日本强大的感觉。由于这种感觉,我越发认为自己的命运是无法跟它分开了。\\

关于西方列强在“满洲事件”上的态度,我早就听郑氏父子等人不断说过这类的话:“别看日内瓦、巴黎(国联)开会开得热闹,其实哪一国也不打算碰日本,欧战以后有实力的是美国,可是连美国也不想跟日本动硬的。”精通英文、日文的郑垂不时地把外国报纸上的舆论告诉我,说美国不少报纸言论是担日的。他曾有根有据地说了一些非公开消息,例如美日曾有密约,美对日本在东北的行动有谅解,等等。他还很具体地告诉我,早在事变前美国方面的重要人物就劝过蒋介石,把满洲卖给日本,让日本去碰苏联,以收其利\footnote{事实上,喜欢吹牛的郑氏父子并没有撒谎。在当时的《东方杂志》上,就可以找到《纽约论坛报》、《纽约日日新闻》等报纸上的袒日言论的译文。比如,前者有这样的话:“日人军事行动,乃对中国废除不平等条约政策所不能免之反响”,后者:“日本继承俄国在满洲开发,至于今日,其功绩之伟大,为世人公认。”国联通过派遣调查团的决议,确曾遭受到美国的反对,理由是:“此种行动足以刺激日本国民的情绪”,国联在一次会议上,打算做出要求日军退出满洲的决议时,美国国务卿凯塞尔就公开表示,对此并未附议。这些事实的记载可以从当时的许多报刊上看到。后来美国国务院发表了一些秘密文件,其中《一九三一年美国外交文件》一书,公布了那年十一月二十七日美驻日大使福白斯交给日本外务大臣币原的一份党书,透露了美国政府当时“曾劝中国政府采取妥协步调”。至于日美对东北问题的秘密谈判,则在一九三五年十二月号的《国际事件》(International Affairs, 1935 Dec.)上据西 · 莱特的一篇文章《美国人对远东问题的观点》(Q.Wright: American View of the Far Eastern Problem)中揭露了出来。}。\\

“调查团要来了,”郑孝胥是这样告诉我的,“国民党请他们来调查,想请他们帮忙对付日本,其实他们是不对付日本的。他们关心的一是门户开放、机会均等,二是对付赤俄。他们在东京跟内田康哉(这时已出任日本外相)谈的就是这个。用不着担心,到时候应付几句就行了。依臣看来,国民党也明知道调查团办不了什么事,说不定国民党看到了国际共管满洲的好处。”\\

后来事实证明,郑氏父子说的话大部正确。\\

沈阳事变发生后,蒋介石一再电张学良转命东北驻军:“为免事件扩大,绝对不抵抗。”四天后,即九月二十二日,蒋在南京全市国民党员大会上宣称:“以公理对强权,以和平对野蛮,忍辱含愤,暂取逆来顺受态度,以待国际公理之判断。”同时,对内却毫无和平与公理,用最野蛮的办法加紧进行内战。九月三十日,国民党向国联请求派中立委员会到满洲调查。经过几番讨论,到十二月十日才得到日本同意,做出组织调查团的决议。调查团由五国委员组成,即英国的李顿爵士、美国的佛兰克洛斯·麦考益少将、法国的亨利·克劳德中将、意大利的格迪伯爵和德国的恩利克希尼博士。团长是李顿。一九三二年二月三日调查团启程,先在日本、上海、南京、汉口、九江、宜昌、重庆等地转了一圈,又在北平住了十天,到东北的时候已是五月份了。在这期间,南京政府宣传着“等待公理的判断”,而日军则攻占了锦州,发动了淞沪战争,成立了“满洲国”。除了这些被“等待”来的结果之外,还有一个在各国干预下产生的《淞沪停战协定》。根据这个协定,南京政府的军队从此不得进驻淞沪地区。\\

五月三日这天,我和调查团的会见,用了大约一刻钟左右的时间。他们向我提出了两个问题:我是怎么到东北来的?“满洲国”是怎么建立起来的?在回答他们的问题之前,我脑子里闪过一个大概他们做梦也没想到的念头。我想起当年庄士敦曾向我说过,伦敦的大门是为我打开着的,如果我现在对李顿说,我是叫土肥原骗来又被板垣威吓着当上“满洲国元首”的,我要求他们把我带到伦敦,他们肯不肯呢?我这个念头刚一闪过,就想起来身边还坐着关东军的参谋长桥本虎之助和高参板垣征四郎。我不由地向那青白脸瞄了一眼,然后老老实实按照他预先嘱咐过的说:“我是由于满洲民众的推戴才来到满洲的,我的国家完全是自愿自主的……”\\

调查团员们一齐微笑点头,再没问什么。然后我们一同照相,喝香摈,祝贺彼此健康。调查团走后,板垣的青白脸泛满了笑意,赞不绝口地说:“执政阁下的风度好极了,讲话响亮极了!”郑孝胥事后则晃着秃头说:“这些西洋人跟臣也见过面,所谈都是机会均等和外国权益之事,完全不出臣之所料。”\\

这年十月,日本《中央公论》上刊出了驹井的一篇文章,郑垂把译文送来不久,《调查团报告书》也到了我手里,这两样东西,给了我一个统一的印象,正如郑氏父子所判断的,调查团所关心的是“机会”与“门户”问题。\\

驹井的文章题为《满洲国是向全世界宣称着》,内容是他与李顿等人会见的情形。现在郑垂的译文已不可得,只有借助于一篇不高明的译文,是陈彬龢编印的《满洲伪国》里的。文章中说,李顿第一个向他提出问题:“满洲国的建设不稍嫌早些么?”他回答了一大套非但不早,且嫌其晚的鬼道理,然后是:\\

其次麦考益将军问:“满洲国宣扬着门户开放主义,果真实行了么?”\\

我立即回答说:“门户开放和机会均等是满洲立国的铁则。门户开放政策,在昔围绕着中国的诸国中,美国是率先所说的精神。但这主义政策是列国之所倡,中国本身是抱着门户闭锁主义,我们果在中国的何处可以看到门户开放的事实?现在我们以极强的钥匙使满洲国门户开放,我们只有受诸君感谢,而没有受抗议的道理。……不过我须附带声明的,就是关于国防事业断不能门户开放,即在世界各国亦断无此例。”\\

李顿再询问:“满洲国实行着机会均等么?”\\

我略不踌躇地说:“机会均等,贵国在中国已有其先例,即前清末叶,中国内政极度糜烂,几全失统一之际,罗浮脱·赫德提议清廷说,倘然长此以往,中国将完全失其作用于国际间,不如依赖西洋人,海关行政,亦有确定之必要。于是清朝立即任命罗浮脱·赫德为总税司,海关行政方得确立。由于海关上使用着许多英、法、日等国人,在中国被认为是最确实的行政机关,因此列强借款给中国,中国道得在财政上有所弥补。英国人曾以海关为施行机会均等之所,但是我们日本人,要想做这海关的事务员,则非受等于拒绝的严格的英语试验不可。\\

“……我们满洲国,是满洲国人和日本人协力而建设的国家,因之新国家的公文,均以满洲国语和日本语发表。所以任何国人,金能完全使用满日两国语言,并能以满洲国所给与之待遇为满足,则我们当大大的欢迎。\\

这就是我所说的机会均等。”\\

我继续着问:“你们各位还有旁的询问么?”\\

旁的人都说:“此外已无何等询问的必要了,我们已能充分理解了满洲国的立场,愉快之至!”\\

国联调查委员在离开新京时,我送到车站上,那时候李顿握了我的手小声地说:“恭祝新满洲国之健全的发达!”同时用力地握了下手就分别了。\\

这次谈话,使郑孝胥父子感到了极大的兴奋,郑垂甚至估计到,国联很可能做出一个国际共管满洲的决议来。后来调查团的报告书公布出来,使郑氏父子更有了信心。调查团的报告书中所代表的国联,正是以郑氏父子所希望的那种中国的管理者的态度出现的。报告书明白地说:“目前极端之国际冲突事件,业经中国再度要国联之干涉。……中国遵循与国际合作之道,当能得最确定及最迅速之进步,以达到其国家之理想。”这位管理者明确地表示:日本“为谋满洲之经济发展,要求建设一能维持秩序之巩固政权,此项要求,我等亦不以为无理”。但是,这位管理者认为最重要的是,“惟有在一种外有信仰内有和平,而与远东现有情形完全不同之空气中,为满洲经济迅速发展所必要之投资始可源源而来”。这就是说,要有列强各国共同认定的那种“信仰”才行,这就是郑氏父子所向往的由各国共同经营,利益均沾的局面。\\

郑氏父子关于反苏问题的估计,也得到了证实。调查团说,它理解日本称满洲为其生命线之意义,同情日本对“其自身安全之顾虑”,因此,“日本之欲谋阻止满洲被利用为攻击日本之根据地,以及为在某种情形之下满洲边境被外国军队冲过时,日本欲有采取适当军事行动之能力,吾人均可承认”。不过调查团又认为,这样做法日本的财政负担必大,而且日本在满军队受时怀反侧之民众包围,其后又有包含敌意之中国,日本军队能否不受重大困难,亦殊难言。因此可以考虑另外的办法,则“日本甚或又因世界之同情与善意,不须代价而获安全保障较现时以巨大代价换得者为更佳”。调查团于是提出意见说,问题的解决,恢复原状和维持现状都不是令人满意的办法,认为只要“由现时(满洲国)组织毋须经过极端之变更或可产生一种满意之组织”,这就是实行“获得高度自治权”的“满洲自治”,由各国洋人充当这个自治政府的顾问;由于日本人在东北的权益大些,日本人比例也大些,但其他外国也要有一定比例。为实现这个新政体,“讨论和提出一种特殊制度之设立,以治理东三省之详密议案”,要先成立一个由国联行政院掌握最高决定权的、由中日双方和“中立观察员”组成的顾问委员会。调查团并且认为“国际合作”的办法不但适于“满洲”,也适于对全中国使用。其根据理由也是郑氏父子屡次表示过的,是因为中国只有劳动力,而资本、技术。人才全要靠外国人,否则是建设不起来的。\\

在刚看到报告书的那几天,郑孝胥曾兴致勃勃地告诉过我,“事情很有希望”,说胡适也在关内发表论文,称誉报告书为“世界之公论”。可是后来日本方面的反响到了,他父子大为垂头丧气。尽管调查团再三谈到尊重日本在满洲的权益,甚至把“九·一八”事变也说成是日本的自卫行为,日本的外务省发言人却只表示同意一点,就是:“调查团关于满洲的建议,大可施于中国与列强间的关系而获得研益,如制定国际共管计划者,是也!”至于对“满洲”本身的共管方案,根本不加理睬。郑孝胥后来的失宠和被弃,即种因在对于“门户开放、机会均等”的热衷上。\\

在国联调查团的报告书发表之前,我曾经设想过,假如真的像郑氏父子希望的那样,将东北归为国际共管,我的处境可能比日本独占情形下好得多。但是,我还有两点不同的考虑;一是怕“共管”之中,南京政府也有一份,如果这样,我还是很难容身;另一点是,即使南京管不上我,国际共管也未必叫我当皇帝,如果弄出个“自治政府”来,那还有什么帝制?更重要的是,日本的横蛮,在国际上居然不受一点约束,给我的印象极为深刻。因此,事后我一想起了调查团会见时我心里闪过的那个念头,不禁暗暗想道:“幸亏我没有傻干,否则我这条命早完了。……现在顶要紧的还是不要惹翻了日本人,要想重登大宝,还非靠日本人不可呀!”\\