\fancyhead[LO]{{\scriptsize 1908-1917: 我的童年 · 毓庆宫读书}} %奇數頁眉的左邊
\fancyhead[RO]{} %奇數頁眉的右邊
\fancyhead[LE]{} %偶數頁眉的左邊
\fancyhead[RE]{{\scriptsize 1908-1917: 我的童年 · 毓庆宫读书}} %偶數頁眉的右邊
\chapter*{毓庆宫读书}
\addcontentsline{toc}{chapter}{\hspace{1cm}毓庆宫读书}
\thispagestyle{empty}
我六岁那年,\xpinyin*{隆裕}太后为我选好了教书的师傅,钦天监为我选好了开学的吉日良辰。\xpinyin*{宣统}三年旧历七月十八日辰刻,我开始读书了。\\

读书的书房先是在中南海瀛合补桐书屋,后来移到紫禁城斋宫右侧的\xpinyin*{毓庆宫}——这是\xpinyin*{光绪}小时念书的地方,再早,则是\xpinyin*{乾隆}的皇子\xpinyin*{颙琰}(即后来的\xpinyin*{嘉庆}皇帝)的寝宫。\xpinyin*{毓庆宫}的院子很小,房子也不大,是一座工字形的宫殿,紧紧地夹在两排又矮又小的配房之间。里面隔成许多小房间,只有西边较大的两敞间用做书房,其余的都空着。\\

这两间书房,和宫里其他的屋子比起来,布置得较简单:南窗下是一张长条几,上面陈设着帽筒、花瓶之类的东西;靠西墙是一溜炕。起初念书就是在炕上,炕桌就是书桌,后来移到地上,八仙桌代替了炕桌。靠北板壁摆着两张桌子,是放书籍文具的地方;靠东板壁是一溜椅子、茶几。东西两壁上挂着醇贤亲王亲笔给\xpinyin*{光绪}写的诚勉诗条屏。比较醒目的是北板壁上有个大钟,盘面的直径约有二米,指针比我的胳臂还长,钟的机件在板壁后面,上发条的时候,要到壁后摇动一个像汽车摇把似的东西。这个奇怪的庞然大物是哪里来的,为什么要安装在这里,我都不记得了,甚至它走动起来是什么声音,报时的时候有多大响声,我也没有印象了。\\

尽管\xpinyin*{毓庆宫}的时钟大得惊人,\xpinyin*{毓庆宫}的人却是最没有时间观念的。看看我读的什么书,就可以知道。我读的主要课本是十三经,另外加上辅助教材《大学衍义》、《朱子家训》、《庭训格言》、《圣谕广训》、《御批通鉴辑览》、《圣武记》、《大清开国方略》等等。十四岁起又添了英文课,除了《英语读本》,我只念了两本书,一本是《爱丽思漫游奇境记》,另一本是译成英文的中国《四书》。满文也是基本课,但是连字母也没学会,就随老师\xpinyin*{伊克坦}的去世而结束。总之,我从\xpinyin*{宣统}三年学到民国十一年,没学过加减乘除,更不知声光化电。关于自己的祖国,从书上只看到“同光中兴”,关于外国,我只随着爱丽思游了一次奇境。什么华盛顿、拿破仑,瓦特发明蒸气机,牛顿看见苹果落地,全不知道。关于宇宙,也超不出“太极生两仪,两仪生四象,四象生八卦”。如果不是老师愿意在课本之外谈点闲话,自己有了阅读能力之后看了些闲书,我不会知道北京城在中国的位置,也不会知道大米原来是从地里长出来的。当谈到历史,他们谁也不肯揭穿长白山仙女的神话,谈到经济,也没有一个人提过一斤大米要几文钱。所以我在很长时间里,总相信我的祖先是由仙女佛库伦吃了一颗红果生育出来的,我一直以为每个老百姓吃饭时都会有一桌子菜肴。\\

我读的古书不少,时间不短,按理说对古文总该有一定的造\xpinyin*{诣},其实不然。首先,我念书极不用功。除了经常生些小病借题不去以外,实在没题目又不高兴去念书,就叫太监传谕老师,放假一天。在十来岁以前,我对\xpinyin*{毓庆宫}的书本,并不如对\xpinyin*{毓庆宫}外面那棵桧柏树的兴趣高。在\xpinyin*{毓庆宫}东跨院里,有棵桧柏树,夏天那上面总有蚂蚁,成天上上下下,忙个不停。我对它们产生了很大的好奇心,时常蹲在那里观察它们的生活,用点心渣子喂它们,帮助它们搬运食品,自己倒忘了吃饭。后来我又对蛐蛐、蚯蚓发生了兴趣,叫人搬来大批的古瓷盆缸喂养。在屋里念书,兴趣就没这么大了,念到最枯燥无味的时候,只想跑出来看看我这些朋友们。\\

十几岁以后,我逐渐懂得了读书和自己的关系:怎么做一个“好皇帝”,以及一个皇帝之所以为皇帝,都有什么天经地义,我有了兴趣。这兴趣只在“道”而不在“文”。这种“道”,大多是皇帝的权利,很少是皇帝的义务。虽然圣贤说过“民为重,社稷次之,君为轻”,“君视臣为草芥,臣视君为寇仇”之类的话,但更多的话却是为臣工百姓说的,如所谓“君君臣臣父父子子”等。在第一本教科书《孝经》里,就规定下了“始于事亲,终于事君”的道理。这些顺耳的道理,开讲之前,我是从师傅课外闲谈里听到的,开讲以后,也是师傅讲的比书上的多。所以真正的古文倒不如师傅的古话给我的印象更深。\\

许多旧学塾出身的人都背过书,据说这件苦事,确实给了他们好处。这种好处我却没享受到。师傅从来没叫我背过书,只是在书房里念几遍而已。\\

也许他们也考虑到念书是应该记住的,所以规定了两条办法:一条是我到太后面前请安的时候,要在太后面前把书从头念一遍给她听;另一条是我每天早晨起床后,由总管太监站在我的卧室外面,大声地把我昨天学的功课念几遍给我听。至于我能记住多少,我想记不想记,就没有人管了。\\

老师们对我的功课,从来不检查。出题作文的事,从来没有过。我记得作过几次对子,写过一两首律诗,做完了,老师也不加评语,更谈不上修改。其实,我在少年时代是挺喜欢写写东西的,不过既然老师不重视这玩艺,我只好私下里写,给自己欣赏。我在十三四岁以后,看的闲书不少,像明清以来的笔记、野史,清末民初出版的历史演义、剑仙快客、公案小说,以及商务印书馆出版的《说部丛书》等等,我很少没看过的。再大一点以后,我又读了一些英文故事。我曾仿照这些中外古今作品,按照自己的幻想,编造了不少“传奇”,并且自制插图,自编自看。我还化名向报刊投过稿,大都遭到了失败。我记得有一次用“\xpinyin*{邓炯麟}”的化名,把一个明朝诗人的作品抄寄给一个小报,编者上了我的当,给登出来了。上当的除了报纸编者还有我的英国师傅\xpinyin*{庄士敦},他后来把这首诗译成英文收进了他的著作《紫禁城的黄昏》,以此作为他的学生具有“诗人气质”的例证之一。\\

我的学业成绩最糟的,要数我的满文。学了许多年,只学了一个字,这就是每当满族大臣向我请安,跪在地上用满族语说了照例一句请安的话(意思是:奴才某某跪请主子的圣安)之后,我必须回答的那个:“伊立(起来)!”\\

我九岁的时候,他们想出一条促进我学业的办法,给我配上伴读的学生。伴读者每人每月可以拿到按八十两银子折合的酬赏,另外被“赏紫禁城骑马”\footnote{“赏紫禁城骑马”也叫赏朝马。军机处每年将一、二品大臣年六十以上者,开单请旨,一般皆可获准,推侍郎(正二品)以下的不一定全准,内廷官员往往“特蒙思礼”不复问年,亲王以下至贝子则皆可准许。准骑者由东华门入至话亭下马,由西华门人至内务府总管衙门前下马。这种赏赐也是封建朝廷给予臣下的一种巨大的荣誉。}。虽然那时已进入民国时代,但在皇族子弟中仍然被看做是巨大的荣誉。得到这项荣誉的有三个人,即:\xpinyin*{溥杰}、\xpinyin*{毓崇}(溥伦的儿子,伴读汉文)、\xpinyin*{溥佳}(七叔\xpinyin*{载涛}的儿子,伴读英文,从我十四岁时开始)。伴读者还有一种荣誉,是代书房里的皇帝受责。“成王有过,则挞伯禽”,既有此古例,因此在我念书不好的时候,老师便要教训伴读的人。实际上,皇弟\xpinyin*{溥杰}是受不到这个的,倒楣的是\xpinyin*{毓崇}。\xpinyin*{毓庆宫}里这三个汉文学生,\xpinyin*{溥杰}的功课最好,因为他在家里另有一位教师教他,他每天到\xpinyin*{毓庆宫}来,不过是白赔半天功夫。\xpinyin*{毓崇}的成绩最坏,这倒不是他没另请师傅,而是他由于念的好也挨说,念不好也挨说,这就使他念得没有兴趣。所以他的低劣成绩,可以说是职业原因造成的。\\

我在没有伴读同学的时候,确实非常淘气。我念书的时候,一高兴就把鞋袜全脱掉,把袜子扔到桌子上,老师只得给我收拾好,给我穿上。有一次,我看见徐坊老师的长眉毛好玩,要他过来给我摸摸。在他遵命俯头过来的时候,给我冷不防的拔下了一根。徐坊后来去世,太监们都说这是被“万岁爷”拔掉寿眉的缘故。还有一次,我的\xpinyin*{陆润庠}师傅竟被我闹得把“君臣”都忘了。记得我那次无论如何念不下书,只想到院子里看蚂蚁倒窝去,陆老师先用了不少婉转的话劝我,什么“文质彬彬,然后君子”,我听也听不懂,只是坐在那里东张西望,身子扭来扭去。陆师傅看我还是不安心,又说了什么“君子不重则不威;学则不固”,我反倒索兴站起来要下地了,这时他着急了,忽然大喝一声:“不许动!”我吓了一跳,居然变得老实一些。可是过了不久,我又想起了蚂蚁,在座位上魂不守舍地扭起来。\\

伴读的来了之后,果然好了一些,在书房里能坐得住了。我有了什么过失,师傅们也有了规劝和警戒的方法。记得有一次我蹦蹦跳跳地走进书房,就听见陈老师对坐得好好的\xpinyin*{毓崇}说:“看你何其轻佻!”\\

我每天念书时间是早八时至十一时,后来添了英文课,在下午一至三时。每天早晨八时前,我乘坐金顶黄轿到达\xpinyin*{毓庆宫}。我说了一声:“叫!”太监即应声出去,把配房里的老师和伴读者叫了来。他们进殿也有一定程序:前面是捧书的太监,后随着第一堂课的老师傅,再后面是伴读的学生。老师进门后,先站在那里向我注目一下,作为见面礼,我无须回礼,因为“虽师,臣也,虽徒,君也”,这是礼法有规定的。然后\xpinyin*{溥杰}和\xpinyin*{毓崇}向我请跪安。礼毕,大家就坐。桌子北边朝南的独座是我的,师傅坐在我左手边面西的位子上,顺他身边的是伴读者的座位。这时太监们把他们的帽子在帽筒上放好,鱼贯而退,我们的功课也就开始了。\\

我找到了十五岁时写的三页日记,可以看出那时念书的生活情况。\xpinyin*{辛亥}后,在我那一圈儿里一直保留着\xpinyin*{宣统}年号,这几页日记是“\xpinyin*{宣统}十二年十一月”的。\\

\begin{quote}
	二十七日,晴。早四时起,书大福字十八张。八时上课,同\xpinyin*{溥杰}、\xpinyin*{毓崇}共读论语、周礼、礼记、唐诗,听陈师讲通鉴辑览。九时半餐毕,复读左传、谷梁传,听朱师讲大学衍义及写仿对对联。至十一时功课毕,请安四官。是日\xpinyin*{庄士敦}未至,因微受感冒。遂还养。心殿,书福寿字三十张,复阅各报,至四时餐,六时寝。卧帐中又读古文观止,甚有兴味。\\

二十八日,睛。早四时即起,静坐少时,至八时上课。仍如昨日所记。至十二钟三刻余,\xpinyin*{庄士敦}至,即与\xpinyin*{溥佳}读英文。三时,功课毕,还养心殿。三时半,因微觉胸前发痛,召\xpinyin*{范一梅}来诊,开药方如左:薄荷八分,白芷一钱,青皮一钱五分炒,郁金一钱五分研,扁豆二钱炒,神曲一钱五分炒,焦查三钱,青果五枚研,水煎温服。晚餐后,少顷即服。五时半寝。\\

二十九日,晴。夜一时许,即被呼醒,觉甚不适。及下地,方知已受煤毒。二人扶余以行,至前室已晕去。卧于榻上,少顷即醒,又越数时乃愈。而在余寝室之二太监,亦晕倒,今日方知煤之当紧(警)戒也。八时,仍旧上课读书,并读英文。三时下学,餐毕,至六时余寝。\\
\end{quote}

\xpinyin*{陆润庠}师傅\footnote{\xpinyin*{陆润庠}(1841-1915),也是当时的一个工业资本家,\xpinyin*{光绪}末年,他在苏州创办了最早的纱厂丝厂。\xpinyin*{辛亥}革命后清室非法授以太保,并在死后追赠为太傅,谥文端。}是江苏元和人,做过大学士,教我不到一年就去世了。教满文的\xpinyin*{伊克坦}是满族正白旗人,满文翻译进士出身,教了我九年多满文。和陆、伊同来的\xpinyin*{陈宝琛}是福建闽县人,西太后时代做过内阁学士和礼部侍郎,是和我相处最久的师傅。陆死后添上教汉文的做过国子丞的徐坊,南书房翰林\xpinyin*{朱益藩}和以\xpinyin*{光绪}陵前植松而出名的\xpinyin*{梁鼎芬}\footnote{\xpinyin*{梁鼎芬}(1859-1919),字节庵又字星海,广东番禺人,\xpinyin*{宣统}三年委广东宣抚使,未上任清朝即倒台,赴易州哭谒\xpinyin*{光绪}陵,故小朝廷授他为“崇陵陵工大臣”。在他奔走之下,上海各地有不少想求得小朝廷的匾额或其他荣典的人大捐其钱,供奉崇陵工程。}。对我影响最大的师傅首先是\xpinyin*{陈宝琛},其次是后来教英文的英国师傅\xpinyin*{庄士敦}。陈在福建有才子之名,他是\xpinyin*{同治}朝的进士,二十岁点翰林,入阁后以敢于上谏太后出名,与张之洞等有清流党之称。他后来不像张之洞那样会随风转舵,\xpinyin*{光绪}十七年被借口南洋事务没有办好,降了五级,从此回家赋闲,一连二十年没出来。直到\xpinyin*{辛亥}前夕才被起用,原放山西巡抚,未到任,就被留下做我的师傅,从此没离开我,一直到我去东北为止。在我身边的遗老之中,他是最称稳健谨慎的一个。当时在我的眼中,他是最忠实于我、最忠实于“大清”的。在我感到他的谨慎已经妨碍了我之前,他是我惟一的智囊。事无巨细,咸待一言决焉。\\

“有王虽小而元子哉!”这是陈师傅常微笑着对我赞叹的话。他笑的时候,眼睛在老光镜片后面眯成一道线,一只手慢慢捋着雪白而稀疏的胡子。\\

更叫我感兴趣的是他的闲谈。我年岁大些以后,差不多每天早晨,总要听他讲一些有关民国的新闻,像南北不和,督军火并,府院交恶,都是他的话题。说完这些,少不得再用另一种声调,回述一下“同光中兴、康乾盛世”,当然,他特别喜欢说他当年敢于进谏西太后的故事。每当提到给民国做官的那些旧臣,他总是忿忿然的。像\xpinyin*{徐世昌}、赵尔巽这些人,他认为都应该列入\xpinyin*{贰}臣传里。在他嘴里,革命、民国、共和,都是一切灾难的根源,和这些字眼有关的人物,都是和盗贼并列的。“非圣人者无法,非孝者无亲,此大乱之道也”,这是他对一切不顺眼的总结论。记得他给我转述过一位遗老编的对联:“民犹是也,国犹是也,何分南北?总而言之,统而言之,不是东西”。他加上一个横批是:“旁观者清”。他在赞叹之余,给我讲了卧薪尝胆的故事,讲了“遵时养晦”的道理。他在讲过时局之后,常常如此议论:“民国不过几年,早已天怒人怨,国朝二百多年深仁厚泽,人心思清,终必天与人归。”\\

\xpinyin*{朱益藩}师傅教书的时候不大说闲话,记得他总有些精神不振的样子,后来才知道他爱打牌,一打一个通夜,所以睡眠有点不足。他会看病,我生病有时是请他看脉的。\xpinyin*{梁鼎芬}师傅很爱说话。他与陈师傅不同之处,是说到自己的地方比陈师傅要多些。有一个故事我就听他说过好几遍。他在\xpinyin*{光绪}死后,曾发誓要在\xpinyin*{光绪}陵前结庐守陵,以终晚年。故事就发生在他守陵的时候。有一天夜里,他在灯下读着史书,忽然院子里跳下一个彪形大汉,手持一把雪亮的匕首,闯进屋里。他面不改色地问道:“壮士何来?可是要取梁某的首级?”那位不速之客被他感动了,下不得手。他放下书,慨然引颈道:“我梁某能死于先帝陵前,于愿足矣!”那人终于放下匕首,双膝跪倒,自称是\xpinyin*{袁世凯}授命行刺的,劝他从速离去,免生不测。他泰然谢绝劝告,表示决不怕死。这故事我听了颇受感动。我还看见过他在崇陵照的一张相片,穿着清朝朝服,身边有一株松苗。后来\xpinyin*{陈宝琛}题过一首诗:“补天回日手何如?冠带临风自把锄,不见松青心不死,固应藏魄傍山庐。”他怎么把终老于陵旁的誓愿改为“不见松青心不死”,又怎么不等松青就跑进城来,我始终没弄明白。\\

当时弄不明白的事情很多,比如,子不语怪力乱神,但是陈师傅最信卜卦,并为我求过神签,向关帝问过未来祖业和我自己的前途;梁师傅笃信扶\xpinyin*{乩};朱师傅向我推荐过“天眼通”。\\

我过去曾一度认为师傅们书生气太多,特别是\xpinyin*{陈宝琛}的书生气后来多得使我不耐烦。其实,认真地说来,师傅们有许多举动,并不像是书生干的。书生往往不懂商贾之利,但是师傅们却不然,他们都很懂行,而且也很会沽名钓誉。现在有几张赏单叫我回忆起一些事情。这是“\xpinyin*{宣统}八年十一月十四日”的记录:\\

\begin{quote}
	赏\xpinyin*{陈宝琛}:\xpinyin*{王时敏}晴岚暖翠阁手卷一卷\\

\xpinyin*{伊克坦}:\xpinyin*{米元章}真迹一卷\\

\xpinyin*{朱益藩}:\xpinyin*{赵伯驹}王洞群仙图一卷\\

\xpinyin*{梁鼎芬}:\xpinyin*{阎立本}画孔子弟子像一卷\\
\end{quote}

还有一张“\xpinyin*{宣统}九年三月初十日”记的单子,上有赏\xpinyin*{伊克坦}、\xpinyin*{梁鼎芬}每人“唐宋名臣相册”一册,赏\xpinyin*{朱益藩}“\xpinyin*{范中正}夏峰图”一轴、“\xpinyin*{恽寿平}仿\xpinyin*{李成}山水”一轴。这类事情当时是很不少的,加起来的数量远远要超过这几张纸上的记载。我当时并不懂字画的好坏,赏赐的品目都是这些内行专家们自己提出来的。至于不经赏赐,借而不还的那就更难说了。\\

有一次在书房里,陈师傅忽然对我说,他无意中看到两句诗:“老鹤无衰貌,寒松有本心”。他想起了自己即将来临的七十整寿,请求我把这两句话写成对联,赐给他做寿联。看我答应了,他又对他的同事\xpinyin*{朱益藩}说:“皇上看到这两句诗,说正像陈师傅,既然是皇上这样说,就劳大笔一挥,写出字模供皇上照写,如何?”\\

这些师傅们去世之后,都得到了颇令其他遗老羡慕的\xpinyin*{谥法}。似乎可以说,他们要从我这里得到的都得到了,他们所要给我的,也都给我了。至于我受业的成绩,虽然\xpinyin*{毓庆宫}里没有考试,但是我十二岁那年,在一件分辨“忠奸”的实践上,让师傅们大为满意。\\

那年\xpinyin*{奕劻}去世,他家来人递上遗折,请求\xpinyin*{谥法}。内务府把拟好的字眼给我送来了。按例我是要和师傅们商量的,那两天我患感冒,没有上课,师傅不在跟前,我只好自己拿主意。我把内务府送来的\xpinyin*{谥法}看了一遍,很不满意,就扔到一边,另写了几个坏字眼,如荒谬的“谬”,丑恶的“丑”,以及幽王的“幽”,厉王的“厉”,作为恶谥,叫内务府拿去。过了一阵,我的父亲来了,结结巴巴地说:\\

“皇上还还是看在宗宗室的分上,另另赐个……”\\

“那怎么行?”我理直气壮地说,“\xpinyin*{奕劻}受\xpinyin*{袁世凯}的钱,劝太后让国,大清二百多年的天下,断送在\xpinyin*{奕劻}手里,怎么可以给个美谥?只能是这个:丑!谬!”\\

“好,好好。”父亲连忙点头,拿出了一张另写好字的条子来,递给我:“那就就用这这个,‘献’字,这这个字有个犬旁,这这字不好……”\\

“不行!不行!”我看出这是哄弄我,师傅们又不在跟前,这简直是欺负人了,我又急又气,哭了起来:“犬字也不行!不行不行!……不给了!什么字眼也不给了!”\\

我父亲慌了手脚,脑后的花翎跳个不停:“别哭别哭,我找找找上书房去!”\\

第二天我到\xpinyin*{毓庆宫}上课,告诉了\xpinyin*{陈宝琛},他乐得两只眼睛又眯成了一道缝,连声赞叹:\\

“皇上跟王爷争的对,争的对!……有王虽小而元子哉!”\\

南书房翰林们最后拟了一个“密”字,我以为这不是个好字眼,同意了,到后来从\xpinyin*{苏洵}的《\xpinyin*{谥法}考》上看到“追补前过曰密”时,想再改也来不及了。但是这次和父亲的争论,经师傅们的传播,竟在遗老中间称颂一时。\xpinyin*{梁鼎芬}在侍讲日记里有这样一段文字:\\

\xpinyin*{宣统}九年正月初七百,庆亲王\xpinyin*{奕劻}死。初八日遗折上,内务府大臣拟旨谥曰“哲”,上不可。……初十日,召见\xpinyin*{世续}、\xpinyin*{绍英}、\xpinyin*{耆龄},谕曰:\xpinyin*{奕劻}贪赃误国,得罪列祖列宗,我大清国二百余年之天下,一手坏之,不能予谥!已而谥之曰“密”。\xpinyin*{谥法}考追补前过曰密。\xpinyin*{奕劻}本有大罪,天下恨之。传闻上谕如此,凡为忠臣义士,靡不感泣曰:真英主也!
