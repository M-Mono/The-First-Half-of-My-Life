\fancyhead[LO]{{\scriptsize 1932-1945: 伪满十四年 · 吉冈安直}} %奇數頁眉的左邊
\fancyhead[RO]{} %奇數頁眉的右邊
\fancyhead[LE]{} %偶數頁眉的左邊
\fancyhead[RE]{{\scriptsize 1932-1945: 伪满十四年 · 吉冈安直}} %偶數頁眉的右邊
\chapter*{吉冈安直}
\addcontentsline{toc}{chapter}{\hspace{1cm}吉冈安直}
\thispagestyle{empty}
关东军好像一个强力高压电源,我好像一个精确灵敏的电动机,\xpinyin*{吉冈安直}就是传导性能良好的电线。\\

这个高颧骨、小胡子、矮身材的日本鹿儿岛人,从一九三五年起来到我身边,一直到一九四五年日本投降,和我一起被苏军俘虏时止,始终没有离开过我。十年间,他由一名陆军中住,步步高升到陆军中将。他有两个身分,一个是关东军高级参谋,另一个是“满洲国帝室御用挂”。后者是日本的名称,据说意思好像是“内廷行走”,又像是“皇室秘书”,究竟应当译成什么合适,我看这并没有什么关系,因为它的字面含意无论是什么,都不能说明\xpinyin*{吉冈}的实际职能。他的实际职能就是一根电线。关东军的每一个意思,都是通过这根电线传达给我的。我出巡、接见宾客、行礼、训示臣民、举杯祝酒,以至点头微笑,都要在\xpinyin*{吉冈}的指挥下行事。我能见什么人,不能见什么人,见了说什么话,以及我出席什么会,会上讲什么,等等,一概听他的吩咐。我要说的话,大都是他事先用日本式的中国话写在纸条上的。\\

日本发动了全面侵华战争,要伪满出粮、出人、出物资,我便命令\xpinyin*{张景惠}在一次“省长会议”上,按\xpinyin*{吉冈}的纸条“训勉”省长们“勤劳奉仕,支持圣战”。日本发动了太平洋战争,兵力不足,要伪满军队接替一部分中国战场上的任务,我便在军管区司令官宴会上,按纸条表示了“与日本共生共死,一心一德,断乎粉碎英美势力”的决心。\\

此外,日本在关内每攻占一个较大的城市,\xpinyin*{吉冈}必在报告了战果之后,让我随他一同起立,朝战场方向鞠躬,为战死的日军官兵致默哀。经他几次训练,到武汉陷落时我就再用不着他提醒,等他一报告完战果我就自动起立,鞠躬静默。\\

随着“成绩”不断进步,他也不断给我加添功课。例如这次武汉陷落,他又指示我给攻占武汉的大刽子手\xpinyin*{冈村宁次}写亲笔祝词,赞颂他的武功,并指示我给日本天皇去贺电。\\

后来修建了“建国神庙”,我每月去那里为日本军队祷告胜利,也是在这“电线”的授意下进行的。\\

在“七·七”事变前,我的私事家事,关东军还不多过问,可是事变后,情形不同了。\\

“七·七”事变前,我在关内的家族照例每年要来一些人,为我祝寿,平时也不兔来来往往。“七·七”事变后,关东军做出规定,只准列在名单上的几个人在一定时间到长春来。而且规定除了我的近支亲族之外,其余的人只能向我行礼,不准与我谈话。\\

同时,外面给我寄来的信件,也一律先送\xpinyin*{吉冈}的喽罗——宫内府的日系官吏看,最后由\xpinyin*{吉冈}决定是否给我。\\

当然,关东军也了解我不致于反满抗日,但是,他们仍旧担心我会跟关内勾结起来恢复清朝,而这是不符合他们的要求的。\\

在那时,要想瞒过\xpinyin*{吉冈}私自会见外人或收一封信,简直是办不到的。那时在宫内府设有“宪兵室”,住有一班穿着墨绿色制服的日本宪兵,不仅一切出入的人都逃不出他们的视线,就连院子里发生什么事也逃不过他们的耳朵。加之宫内府自次长以下所有的日本人都是\xpinyin*{吉冈}的爪牙,这就造成了对我的严格控制。\\

\xpinyin*{吉冈}之所以能作为关东军的化身,干了十年之久,是有他一套本领的。\\

有的书上说,\xpinyin*{吉冈}原是我在天津时的好友,后来当了关东军参谋,正好这时关东军要选一名帝室与关东军之间的“联络人”,以代替解职的侍从武官\xpinyin*{石丸志都磨},于是便选上了他。其实在天津时,他不过有一段时间常给我讲时事,谈不上是我的什么好友。他被派到我这里当“联络人”,也不是当了关东军参谋才恰逢其时的。如果说他是\xpinyin*{溥杰}的好友,倒有一半是真的。伪满成立之后,\xpinyin*{溥杰}进了日本陆军士官学校,\xpinyin*{吉冈}正在这个学校担任战史教官。他几乎每个星期日都请\xpinyin*{溥杰}去他家做客,殷勤招待。他们两人成了好友之后,他即向\xpinyin*{溥杰}透露,关东军有意请他到满洲,担任军方与我个人之间的联络人。\xpinyin*{溥杰}来信告诉了我,后来又把我回信表示欢迎的意思告诉了他。他这时表示,这是他的荣幸,不过假如他不能得到关东军高级参谋的身分,就不想干,因为从前干这差事的\xpinyin*{中岛比多吉}和\xpinyin*{石丸志都磨}没在满洲站住脚,就是由于没有在关东军里扎下根。\\

后来,不知他怎么活动的,他的愿望实现了,关东军决定任他为高级参谋,派他专任对我的联络职务。他在动身来满洲之前,请\xpinyin*{溥杰}写信把这消息告诉我,同时说:“如果令兄能预先给我准备好一间办公的屋子,我就更感到荣幸了。”我知道了这件事,满足了他的“荣幸”感。过了许久我才明白,原来他这是有意给关东军看的。他在关东军的眼里既有与我的不平凡关系,在我的眼中又有关东军高参这张老虎皮,自然就左右逢源,得其所哉了。\\

\xpinyin*{吉冈}很喜欢画水墨画。有一次他画了一幅墨竹,请\xpinyin*{郑孝胥}题诗,请我题字(什么字,早已忘了),然后带到日本,送给\xpinyin*{裕仁}的母亲日本皇太后。不久,日本报纸上刊登了这幅画,并称誉\xpinyin*{吉冈}为“采笔军人”。\xpinyin*{吉冈}的艺术声名是否由此出现的,我不知道,但我敢断定他指望这幅画带给他的,并不是什么艺术上的称号,却是比这称号更值钱的身价。我从日本访问回来,日本皇太后和我有了经常的往来,不断互相馈赠些小礼物,中间人就是这位\xpinyin*{吉冈}。从那次他送了墨竹之后,东京与长春的往来就更频繁了。\\

他大约每年都要往返东京几次,每次临走之前,总要叫我做点点心之类的食品,由他带去送给日本皇太后,回来时,再带回日本皇太后的礼物,其中必不可少的是日本点心。好在那位老太太和我都有现成的做点心师傅,彼此送来送去,都不费什么事。不过由于我的疑心病,\xpinyin*{吉冈}每次带回来的点心,我总是叫别人先吃了才敢动。\\

当然,\xpinyin*{吉冈}每年一次往返于日满皇室之间,这决不是他的擅自专断,但每次往返的内容,我相信主要是他的独创设计。比如有一次,他看见了我的四用联合收音机,忽然像发现了奇迹似地问我:\\

“这个机器能Record(录音)?”\\

他的中国话不大好,但我们交谈起来还不困难,因为他还会点英文。我们两人的英文程度差不多,平时说话中国话夹着英文,加上笔谈帮忙,倒也能把意思说清楚。\\

“Record是大大的好。”我说,并且拿出一片录音片试给他看。\\

“好,好!”他高兴地笑着,看我安好片子,便说:“我教陛下几句日本话吧!嗯!”接着就用日本话说出:“我祝天皇陛下身体健康!”\\

我照他说的日本话说一遍:“我祝天皇陛下身体健康……”,这句话录到唱片上了。他把那唱片放送了两遍,满意地拿了起来。\\

“好,这次我到东京,嗯!把它贡给天皇陛下!”\\

\xpinyin*{吉冈}说话,总带几个“嗯!哈!”眼眉同时挑起。这个毛病,越到后来越多,我也觉着越不受用。和这种变化同时发生的,还有他对于我们之间的关系的解释。\\

一九三四年我访问日本,日本皇太后给我写了几首和歌,那时\xpinyin*{吉冈}的话是我最顺耳的时候。\\

“皇太后陛下等于陛下的母亲,我如同陛下的准家属,也感到荣耀!”\\

他那时对\xpinyin*{溥杰}说:“我和你有如手足的关系。我和皇帝陛下,虽说不能以手足相论,也算是手指与足指关系。咱们是准家族呀!”\\

但是到了一九三六年前后,他的话却有了变化:\\

“日本犹如陛下的父亲,嗯,关东军是日本的代表,嗯,关东军司令官也等于是陛下的父亲,哈!”\\

日本军队前线景况越坏,我在关东军和\xpinyin*{吉冈}面前的辈份也越低,后来他竟是这样说的:\\

“关东军是你的父亲,我是关东军的代表,嗯!”\\

\xpinyin*{吉冈}后来每天进“宫”极为频繁,有时来了不过十分钟,就走了,走了不到五分钟,又来了。去而复返的理由都是很不成道理的,比如刚才忘了说一句什么话,或者忘了问我明天有什么事叫他办,等等。因此我不能不担心,他是否在用突然袭击的办法考查我。\\

为了使他不疑心,我只好一听说他到,立即接见,尽力减少他等候的时间。甚至正在吃饭,也立刻放下饭碗去见他。对于他,我真算做到了“一饭三吐哺、一沐三握发”的程度。
