\fancyhead[LO]{{\scriptsize 1908-1917: 我的童年 · 我的乳母}} %奇數頁眉的左邊
\fancyhead[RO]{} %奇數頁眉的右邊
\fancyhead[LE]{} %偶數頁眉的左邊
\fancyhead[RE]{{\scriptsize 1908-1917: 我的童年 · 我的乳母}} %偶數頁眉的右邊
\chapter*{我的乳母}
\addcontentsline{toc}{chapter}{\hspace{1cm}我的乳母}
\thispagestyle{empty}
\xpinyin*{梁鼎芬}给我写的“起居注”中,有一段“\xpinyin*{宣统}九年正月十六日”的纪事:\\

\begin{quote}
	上常\xpinyin*{笞}太监,近以小过前后答十七名,臣\xpinyin*{陈宝琛}等谏,不从。\\
\end{quote}

这就是说,在到我七周岁的时候,责打太监已成家常便饭,我的冷酷无情、惯发威风的性格已经形成,劝也劝不过来了。\\

我每逢发脾气,不高兴的时候,太监就要遭殃:如果我忽然高兴,想开心取乐的时候,太监也可能要倒楣。我在童年,有许多稀奇古怪的嗜好,除了玩骆驼、喂蚂蚁、养蚯蚓、看狗牛打架之外,更大的乐趣是恶作剧。早在我懂得利用敬事房打人之前,不少太监们已吃过我恶作剧的苦头。有一次,大约是八九岁的时候,我对那些百依百顺的太监们忽然异想天开,要试一试他们是否真的对“圣天子”听话。我挑出一个太监,指着地上一块脏东西对他说:“你给我吃下去!”他真的趴在地上吃下去了。\\

有一次我玩救火用的\xpinyin*{唧}筒,喷水取乐。正玩着,前面走过来了一个年老的太监,我又起了恶作剧的念头,把龙头冲着他喷去。这老太监蹲在那里不敢跑开,竟给冷水激死过去。后来经过一阵抢救,才把他救活过来。\\

在人们的多方逢迎和百般依顺的情形下,养成了我的以虐待别人来取乐的恶习。师傅们谏劝过我,给我讲过仁恕之道,但是承认我的权威,给我这种权威教育的也正是他们。不管他们用了多少历史上的英主圣君的故事来教育我,说来说去我还是个“与凡人殊”的皇帝。所以他们的劝导并没有多大效力。\\

在宫中惟一能阻止我恶作剧行为的,是我的乳母\xpinyin*{王焦氏}。她就是我在西太后面前哭喊着找的那位嫫嫫。她一个字不识,不会讲什么“仁恕之道”和历史上的英主圣君故事,但当她劝我的时候,我却觉得她的话是不好违拗的。\\

有一次,有个会玩木偶戏的太监,给我表演了一场木偶戏。我看得很开心,决心赏他一块鸡蛋糕吃。这时我的恶作剧的兴趣又来了,决定捉弄他一下。我把练功夫的铁砂袋撕开,掏出一些铁砂子,藏在蛋糕里。我的乳母看见了,就问我:“老爷子,那里头放砂子可叫人怎么吃呀?”“我要看看他咬蛋糕是什么模样。”“那不崩了牙吗?崩了牙就吃不了东西。人不吃东西可不行呵!”我想,这话也对,可是我不能取乐了,我说:“我要看他崩牙的模样,就看这一口吧!”乳母说:“那就换上绿豆,咬绿豆也挺逗乐的。”于是那位玩木偶的太监才免了一次灾难。\\

又有一次,我玩气枪,用铅弹向太监的窗户打,看着窗户纸打出一个个小洞,觉得很好玩。不知是谁,去搬了救兵——乳母来了。\\

“老爷子,屋里有人哪!往屋里打,这要伤了人哪!”\\

我这才想起了屋里有人,人是会被打伤的。\\

只有乳母告诉过我,别人和我同样是人。不但我有牙,别人也有牙,不但我的牙不能咬铁砂,别人也不能咬,不但我要吃饭,别人也同样不吃饭要饿肚子,别人也有感觉,别人的皮肉被铅弹打了会一样的痛。这些用不着讲的常识,我并非不懂,但在那样的环境里,我是不容易想到这些的,因为我根本就想不起别人,更不会把自己和别人相提并论,别人在我心里,只不过是“奴才”、“庶民”。我在宫里从小长到大,只有乳母在的时候,才由于她的朴素的言语,使我想到过别人同我一样是人的道理。\\

我是在乳母的怀里长大的,我吃她的奶一直到九岁,九年来,我像孩子离不开母亲那样离不开她。我九岁那年,太妃们背着我把她赶出去了。那时我宁愿不要宫里的那四个母亲也要我的“嫫嫫”,但任我怎么哭闹,太妃也没有给我把她找回来。现在看来,乳母走后,在我身边就再没有一个通“人性”的人。如果九岁以前我还能从乳母的教养中懂得点“人性”的话,这点“人性”在九岁以后也逐渐丧失尽了。\\

我结婚之后,派人找到了她,有时接她来住些日子。在伪满后期,我把她接到长春,供养到我离开东北。她从来没有利用自己的特殊地位索要过什么。她性情温和,跟任何人都没发生过争吵,端正的脸上总带些笑容。她说话不多,或者说,她常常是沉默的。如果没有别人主动跟她说话,她就一直沉默地微笑着。小时候,我常常感到这种微笑很奇怪。她的眼睛好像凝视着很远很远的地方。我常常怀疑,她是不是在窗外的天空或者墙上的字画里,看见了什么有趣的东西。关于她的身世、来历,她从来没有说过。直到我被特赦之后,访问了她的继子,才知道了这个用奶汁喂大了我这“大清皇帝”的人,经受过“大清朝”的什么样的苦难和屈辱。\\

\xpinyin*{光绪}十三年(1887),她出生在直隶河间府任丘县农村一个焦姓的贫农家里。那时她家里有父亲、母亲和一个比她大六岁的哥哥,连她一共四口。五十来岁的父亲种着佃来的几亩洼地,不雨受旱,雨大受涝,加上地租和赋税,好年成也不够吃。在她三岁那年(即\xpinyin*{光绪}十六年),直隶北部发生了一场大水灾。她们一家不得不外出逃难。在逃难的路上,她的父亲几次想把她扔掉,几次又被放回了破筐担里。这一担挑子的另一头是破烂衣被,是全家仅有的财产,连一粒粮食都没有。她后来对她的继子提起这次几乎被弃的厄运时,没有一句埋怨父亲的话,只是反复地说,她的父亲已经早饿得挑不动了,因为一路上要不到什么吃的,能碰见的人都和他们差不多。这一家四口,父亲、母亲、一个九岁的儿子和三岁的女儿,好不容易熬到了北京。他们到北京本想投奔在北京一位当太监的本家。不料这位本家不肯见他们,于是他们流浪街头,成了乞丐。北京城里成千上万的灾民,露宿街头,啼饥号寒。与此同时,朝廷里却在大兴土木,给西太后建颐和园。从《\xpinyin*{光绪}朝东华录》里可以找到这样的记载:这年祖父去世,西太后派大臣赐奠治丧,我父亲承袭王爵。醇王府花银子如淌水似地办丧事,我父亲蒙思袭爵,而把血汗给他们变银子的灾民们正在\xpinyin*{奄奄}待毙,卖儿\xpinyin*{鬻}女。焦姓这家要卖女儿,没有人买。这时害怕出乱子的顺天府尹办了一个粥厂,他们有了暂时的栖身之地,九岁的男孩被一个剃头匠收留下当徒弟,这样好不容易地熬过了冬天。春天来了,流浪的农民们想念着土地,粥厂要关门,都纷纷回去了。焦姓这一家回到家乡,渡过了几个半饥不暖的年头。\xpinyin*{庚子}年八国联军的灾难又降到河间保定两府,女儿这时已是十三岁的姑娘,再次逃难到北京,投奔当了剃头匠的哥哥。哥哥无力赡养她,在她十六岁这年,在半卖半嫁的情形下,把她给了一个姓王的差役做了媳妇。丈夫生着肺病,生活却又荒唐。她当了三年挨打受气的奴隶,刚生下一个女儿,丈夫死了。她母女俩和公婆,一家四口又陷入了绝境。这时我刚刚出生,醇王府给我找乳母,在二十名应选人中,她以体貌端正和奶汁稠厚而当选。她为了用工钱养活公婆和自己的女儿,接受了最屈辱的条件:不许回家,不许看望自己的孩子,每天吃一碗不许放盐的肘子,等等。二两月银,把一个人变成了一头奶牛。\\

她给我当乳母的第三年,女儿因营养不足死了。为了免于引起她的伤感以致影响奶汁质量,醇王府封锁了这消息。\\

第九年,有个妇差和太监吵架,太妃决定赶走他们,顺带着把我乳母也赶走了。这个温顺地忍受了一切的人,在微笑和凝视中渡过了沉默的九年之后,才发现她的亲生女儿早已不在人世了!
