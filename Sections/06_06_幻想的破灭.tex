\fancyhead[LO]{{\scriptsize 1932-1945: 伪满十四年 · 幻想的破灭}} %奇數頁眉的左邊
\fancyhead[RO]{} %奇數頁眉的右邊
\fancyhead[LE]{} %偶數頁眉的左邊
\fancyhead[RE]{{\scriptsize 1932-1945: 伪满十四年 · 幻想的破灭}} %偶數頁眉的右邊
\chapter*{幻想的破灭}
\addcontentsline{toc}{chapter}{\hspace{1cm} 幻想的破灭}
\thispagestyle{empty}
日本自一九三三年初退出国际联盟之后,更加肆无忌惮地进行扩军备战,特别是加紧了全面侵华的部署和后方的准备。在“七七”事变之前,日本在华北连续使用武力和制造事变,国民党南京政府步步屈服,签订了出让华北控制权的“何(应钦)梅(津)协定”、“秦(德纯)土(肥原)协定”等密约,听任“冀东防共自治政府”、“内蒙自治军政府”等等伪组织的存在和活动,再三地向日本表白“不但无排日之行动与思想,亦本无排日必要的理由”,并且对国人颁布了“效睦邻邦命令”,重申抗日者必严惩之禁令。这样,日本在关内的势力有了极大的加强,人人可以看出,只要时间一到,五省即可彻底变色。我在前面说过,这正是关内关外复辟迷们跃跃欲试的时候,正是我第三次“登极”前后得意忘形的时候。然而,日本在张牙舞爪于关内的同时,它在“满洲国”内也正采取着步步加紧的措施,这些措施终于临到我这“皇帝”的头上。\\

在东北彻底殖民地化的过程中,公平地说,汉奸们是得到不少便宜的。例如改帝制,这个措施不仅使复辟迷们得到了一定心理满足,而也成了一次发财的机缘,自郑孝胥以下的大汉奸都得到一笔自五万至六十万不等的“建国功劳金”,总数共为八百六十万元(以后每逢一次大规模的掠夺,如“粮谷出荷”、“献金报国”等等,必有一次“奖金”分给上自“总理大臣”下至保甲长)。我现在不想对日本的各种措施做全面的叙述,只把我恢复祖业思想的幻灭以及深感恐惧的事情说一说。\\

按情理说,日本关东军在决定帝制时正式告诉我不是恢复清朝,在“登极”时不准我穿龙袍,在决定“总理大臣”人选时根本不理睬我的意见,我就该明白了我的“尊严”的虚假性,但是我却由于过分“陶醉”,竟没有因此而清醒过来。使我开始感到幻灭滋味的,还是“凌升事件”。\\

凌升是清末蒙古都统贵福之子,原为张作霖东三省保安总司令部和蒙古宣抚使署顾问。他是在旅顺的“请愿代表”之一,因此被列入“建国元勋”之内。事件发生时他是伪满兴安省省长。一九三六年春天,他突然遭到了关东军的拘捕。拘捕的原因,据关东军派来的吉冈安直说,他有反满抗日活动,但是据佟济煦听来的消息,却是他在最近一次省长联席会上发过牢骚,以致惹恼了日本人。据说他在这次会上,抱怨日本关东军言行不一,说他在旅顺时曾亲耳听板垣说过,日本将承认“满洲国”是个独立国,可是后来事实上处处受关东军干预,他在兴安省无权无职,一切都是日本人做主。开过这个会,他回到本省就被抓去了。我听到这些消息,感到非常不安,因为半年前我刚刚与他结为亲家,我的四妹与他的儿子订了婚。我正在犹豫着,是不是要找关东军说说情的时候,新任的司令官兼第四任驻“满”大使植田谦吉先找我来了。\\

“前几天破获了一起案件,罪犯是皇帝陛下认得的,兴安省省长凌升。他勾结外国图谋叛变,反对日本。军事法庭已经查实他的反满抗日罪行,宣判了死刑。”\\

“死刑?”我吃了一惊。\\

“死刑。”他向他的翻译点头重复一遍,意思是向我说清楚。然后又对我说:“这是杀一儆百,陛下,杀一儆百是必需的!”\\

他走后,关东军吉冈安直参谋又通知我,应该立刻跟凌升的儿子解除四妹的婚约。我连忙照办了。\\

凌升被处决时,使用的是斩首之刑。一同受刑的还有他的几个亲属。这是我所知道的第一个被日本人杀害的显要官员,而且还是刚跟我做了亲家的。我从凌升跟我攀亲的举动上,深信他是最崇拜我的,也是最忠心于我的人,而关东军衡量每个人的惟一标准却是对日本的态度。不用说,也是用这统一标准来看待我的。想到这里,我越发感到植田“杀一儆百”这句话的阴森可怕。\\

我由此联想到不久前的一件事。一九三五年末,有一些人为图谋复辟清朝而奔波于关内关外,如康有为的徒弟任祖安,我从前的奏事官吴天培等,引起了关东军的注意。关东军曾就此向我调查。“凌升事件”提醒了我,日本人是不喜欢这类事的,还是要多加小心为是。\\

日本人喜欢什么?我自然地联想到一个与凌升命运完全不同的人,这就是张景惠。这实在是日本人有意给我们这伙人看的两个“榜样”。一福一祸,对比鲜明。张景惠之所以能得日本人的欢心,代替了郑孝胥,是有他一套功夫的。这位“胡子”出身的“总理大臣”的为人,和他得到日本人的赏识,可以从日本人传诵他的“警句”上知道。有一次总务厅长官在国务会议上讲“日满一心一德”的鬼道理,作为日本掠夺工矿原料行为的“道义”根据,临末了,请“总理大臣”说几句。张景惠说:“咱是不识字的大老粗,就说句粗话吧:日满两国是两只蚂冷(蜻蜓)拴在一根绳上。”这“两只蚂冷一根绳”便被日本人传诵一时,成为教训“满”籍官员的“警句”。日本在东北实行“拓殖移民”政策的时候,在“国务会议”上要通过法案,规定按地价四分之一或五分之一的代价强购东北农田,有些“大臣”如韩云阶等一则害怕造成“民变”,另则自己拥有大量土地,不愿吃亏,因此表示了反对。这时张景惠却出来说话了:“满洲国土地多的不得了,满洲人是老粗,没知识,让日本人来开荒教给新技术,两头都便宜。”提案就此通过了。“两头便宜”这句话于是又被日本人经常引用着。后来,“粮谷出荷”加紧推行,东北农民每季粮食被征购殆尽,有些“大臣”们因为征购价过低,直接损害到他们的利益,在“国务会议”上借口农民闹饥荒,吵着要求提高收购价格。日本人自然又是不干,张景惠于是对大家说:“日本皇军卖命,我们满洲出粮,不算什么。闹饥荒的勒一下裤腰带,就过去了。”“勒腰带”又成了日本人最爱说的一句话,当然,不是对他们自己说的。关东军司令官不断地对我称赞张景惠为“好宰相”,是“日满亲善身体力行者”。我当时很少想到这对我有什么意义,现在有了凌升的榜样,在两者对比之下,我便懂得了。\\

“凌升事体”过去了,我和德王的一次会见造成了我更大的不安。\\

德王即由日本操纵成立了“内蒙自治军政府”伪组织的德穆楚克栋鲁普。他原是一个蒙古王公。我在天津时,他曾送钱给我,送良种蒙古马给溥杰,多方向我表示过忠诚。他这次是有事找关东军,乘机取得关东军司令官的允许,前来看望我的。他对我谈起这几年的经历和成立“自治军政府”的情形,不知不觉地发开了牢骚,埋怨他那里的日本人过分跋扈,说关东军事先向他许了很多愿,到头来一样也不实现。尤其使他感到苦恼的是自己样样不能做主。他的话勾起了我的牢骚,不免同病相怜,安慰了他一番。不想第二天,关东军派到我这里专任联络的参谋,即以后我要谈到的“帝室御用挂”吉冈安直,走来板着脸问我:\\

“陛下昨天和德王谈了些什么?”\\

我觉得有些不妙,就推说不过是闲聊而已。\\

他不放松我,追问道:“昨天的谈话,对日本人表示不满了没有?”\\

我心里砰砰跳了起来。我知道惟一的办法就是坚不承认,而更好的办法则是以进为退,便说:“那一定是德王故意编排出什么假话来了吧?”\\

吉冈虽然再没穷追下去,我却一连几天心惊肉跳,疑虑丛生。我考虑这件事只有两个可能,不是日本人在我屋里安上了什么偷听的机器,就是德王在日本人面前说出了真话。我为了解开这个疑团,费了好大功夫,在屋里寻找那个可能有的机器。我没有找到什么机器,又怀疑是德王成心出卖我,可是也没有什么根据。这两种可能都不能断定,也不能否定,于是都成了我的新魔障。\\

这件事发生之后,我懂得的事就比“凌升事件”告诉我的更多了。我再不跟任何外来人说真心话,我对每位客人都有了戒心。事实上,自从我访日回来发表讲演之后,主动来见的人即逐渐减少,到德王会见之后,更近于绝迹。到了一九三七年,关东军更想出了一个新规矩,即每逢我接见外人,须由“帝室御用挂”在旁侍立。\\

进入了一九三七年,我一天比一天感到紧张。\\

在“七七”事变前这半年间,日本加紧了准备工作。为了巩固它的后方基地的统治,对东北人民的抗日爱国活动,进行了全面的镇压。一月四日,以“满洲国皇帝敕令”颁行了“满洲帝国刑法”,接着便开始了“大检举”、“大讨伐”,实行了“保甲连坐法”,“强化协和会”,修“警备道”,建“碉堡”,归屯并村。日本这次调来大量队伍,用大约二十个日本师团的兵力来对付拥有四万五千余人的抗日联军。与此同时,各地大肆搜捕抗日救国会会员,搜捕一切被认做“不稳”的人。这一场“大检举”与“大讨伐”,效果并不理想,关东军司令官向我夸耀了“皇军”威力和“赫赫战果”之后不到一年,又以更大的规模调兵遣将(后来知道是七十万日军和三十万伪军),举行了新“讨伐”,同时据我的亲信、警卫处长佟济煦告诉我,各地经常有人失踪,好像反满抗日的分子老也抓不完。\\

我从关东军司令官的谈话中,从“总理大臣”的例行报告中,向来是听不到什么真消息的,只有佟济煦还可以告诉我一些。他曾经告诉过我,关东军司令官对我谈的“讨伐”胜利消息,不一定可靠,消灭的“土匪”也很难说是什么人。他说,他有个被抓去当劳工的亲戚,参加修筑过一件秘密工程,据这个亲戚说,这项工程完工后,劳工几乎全部遭到杀害,只有他和少数几个人幸免于难,逃了出来。照他看来,报纸上有一次吹嘘某地消灭了多少“土匪”,说的就是那批劳工。\\

佟济煦的故事说过不久,给我当过英文翻译的吴沆业失踪了。有一天溥杰来告诉我,吴是因为在驻东京大使馆时期与美国人有来往被捕的,现在已死在宪兵队。还说,吴死前曾托看守带信给他,求他转请我说情,但他当时没有敢告诉我。我听了,赶紧叫他不要再说下去。\\

在这段时间里,我经手“裁可”的政策法令,其中有许多关于日本加紧备战和加强控制这块殖民地的措施,但无论是“第一五年开发产业计划”,还是“产业统制法”,也无论是为适应进一步控制需要而进行的“政府机构大改组”,还是规定日本语为“国语”,都没有比溥杰的结婚更使我感到刺激的。\\

溥杰在日本学习院毕业后,就转到士官学校学陆军。一九三五年冬他从日本回到长春,当了禁卫军中尉,从这时起,关东军里的熟人就经常向他谈论婚姻问题,什么男人必须有女人服侍啦,什么日本女人是世界上最理想的妻子啦,不断地向他耳朵里灌。起初,我听他提到这些事时不过付之一笑,并没拿它当回事。不料后来关东军派到我身边来的吉冈安直果真向我透露了关东军的意思,说为了促进日满亲善,希望溥杰能与日本女人结婚。我当时未置可否,心里却十分不安,赶忙找我的二妹一起商量对策。我们一致认为,这一定是一项阴谋,日本人想要笼络住溥杰,想要一个日本血统的孩子,必要时取我而代之。为了打消关东军的念头,我们决定赶快动手,抢先给溥杰办亲事。我把溥杰找来,先进行了一番训导,警告他如果家里有了个日本老婆,自己就会完全处于日本人监视之下,那是后患无穷的,然后告诉他我一定要给他找一个好妻子,他应该听我的话,不要想什么日本女人。溥杰恭恭敬敬地答应了,我便派人到北京去给他说亲。后来经我岳父家的人在北京找到一位对象,溥杰也表示满意,可是吉冈突然找到溥杰,横加干涉地说,关东军希望他跟日本女子结婚,以增进“日满亲善”,他既身为“御弟”,自应做出“亲善”表率,这是军方的意思,本庄繁大将在东京将要亲自为他做媒,因此他不可再去接受北京的亲事,应该等着东京方面的消息。结果,溥杰只得服从了关东军。\\

一九三七年四月三日,溥杰与嵯峨胜侯爵的女儿嵯峨浩在东京结了婚。过了不到一个月,在关东军的授意下,“国务院”便通过了一个“帝位继承法”,明文规定:皇帝死后由子继之,如无子则由孙继之,如无子无孙则由弟继之,如无弟则由弟之子继之。\\

溥杰和他的妻子回东北后,我拿定了一个主意:不在溥杰面前说出任何心里话,溥杰的妻子给我送来的食物我一口也不吃。假若溥杰和我一起吃饭,食桌上摆着他妻子做的菜,我必定等他先下春之后才略动一点。\\

后来,溥杰快要做父亲的时候,我曾提心吊胆地为自己的前途算过卦,我甚至也为我的弟弟担忧。我相信那个帝位继承法,前面的几条都是靠不住的,靠得住的只是“弟之子继之”这句话。关东军要的是一个日本血统的皇帝,因此我们兄弟两个都可能做牺牲品。后来听说他得的是个女儿,我这才松了一口气。\\

当时我曾想过,假若我自己有了儿子,是不是会安全?想的结果是,即使真的有了儿子,也不见得对我有什么好处,因为关东军早叫我写下了字据:若有皇子出生,五岁时就必须送到日本,由关东军派人教养。\\

可怕的事情并没有就此终结。六月二十八日,即“七七”事变九天前,又发生了一起有关“护军”的事件。\\

所谓护军,是我自己出钱养的队伍,它不同于归“军政部”建制的“禁卫军”。我当初建立它,不单是为了保护自己,而是跟我当初送溥杰他们去日本学陆军的动机一样,想借此培养我自己的军事骨干,为建立自己所掌握的军队做准备。我这支三百人的队伍全部都是按照军官标准来训练的。负责管理护军的佟济煦早就告诉过我,关东军对这支队伍是不喜欢的。我对佟济煦的预感,过去一直未能理解,直到出了事情这才明白。六月二十八日那天,一部分护军到公园去游玩,因租借游艇,与几个穿便衣的日本人发生了口角。这时一群日本人一拥而上,不容分说,举手就打。他们被逼急了,便使出武术来抵抗。日本人见不能奈何他们,就放出狼狗来咬。他们踢死狼狗,冲出重围,逃回队里。他们没想到,这一来便闯下了祸。过了不大时间,宫内府外边便来了一些日本宪兵,叫佟济煦把今天去公园的护军全部交出来。佟济煦吓得要命,忙把那些护军交日本宪兵带走。日本宪兵逼他们承认有“反满抗日”活动,那些护军不肯承认,于是便遭到了各种酷刑虐待。到这时那些护军才明白过来,这一事件是关东军有意制造的:那些穿便衣的日本人原是关东军派去的,在双方斗殴中受伤者有两名关东军参谋,被踢死的狼狗即关东军的军犬。我听到护军们被捕,原以为是他们无意肇祸,忙请吉冈安直代为向关东军说情。吉冈去了一趟,带口来关东军参谋长东条英机的三个条件,即:一、由管理护军的佟济煦向受伤的关东军参谋赔礼道歉;二、将肇事的护军驱逐出境;三、保证以后永不发生同类事件。我按照东条的条件—一照办之后,关东军接着又逼我把警卫处长佟济煦革职,由日本人长尾吉五郎接任,把警卫处所辖的护军编制缩小,长武器一律换上了短枪。\\

从前,我为了建立自己的实力,曾送过几批青年到日本去学陆军,不想这些人回来之后,连溥杰在内,都由军政部派了差,根本不受我的支配。现在,作为骨干培养的护军已完全掌握在日本人手里,我便不再做这类可笑的美梦了。\\

“七七”事变爆发,日军占领了北京之后,北京的某些王公。遗老曾一度跃跃欲试,等着恢复旧日冠盖,但是我这时已经明白,这是决不可能的了。我这时的惟一的思想,就是如何在日本人面前保住安全,如何应付好关东军的化身——帝室御用挂吉冈安直。\\