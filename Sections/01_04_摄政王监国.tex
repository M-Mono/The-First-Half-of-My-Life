\fancyhead[LO]{{\scriptsize 1859-1908: 我的家世 · 摄政王监国}} %奇數頁眉的左邊
\fancyhead[RO]{} %奇數頁眉的右邊
\fancyhead[LE]{} %偶數頁眉的左邊
\fancyhead[RE]{{\scriptsize 1859-1908: 我的家世 · 摄政王监国}} %偶數頁眉的右邊
\chapter*{摄政王监国}
\addcontentsline{toc}{chapter}{\hspace{1cm}摄政王监国}
\thispagestyle{empty}
我做皇帝、我父亲做摄政王的这三年间,我是在最后一年才认识自己的父亲的。那是我刚在毓庆宫读书不久,他第一次照章来查看功课的时候。有个太监进来禀报说:“王爷来了。”老师立刻紧张起来,赶忙把书桌整理一下,并且把见王爷时该做什么,指点了给我,然后告诉我站立等候。过了一会,一个头戴花翎、嘴上没胡须的陌生人出现在书房门口,挺直地立在我的面前,这就是我的父亲。我按家礼给他请了安,然后一同落坐。坐好,我拿起书按老师的指示念起来:\\

“孟子见梁惠王,王立于沼上,王立于沼上……”\\

不知怎的,我心慌得很,再也念不下去。梁惠王立于沼上是下不来了。幸好我的父亲原来比我还慌张,他连忙点头,声音含混地说:\\

“好,好,皇帝好,好好地念,念书吧!”说完,又点了一阵头,然后站起来走了。他在我这里一共呆了不过两分钟。\\

从这天起,我知道了自己的父亲是什么样:不像老师,他没胡子,脸上没皱纹,他脑后的花翎子总是跳动。以后他每隔一个月来一次,每次呆的时间也都不过两分钟。我又知道了他说话有点结巴,明白了他的花翎子之所以跳动,是由于他一说话就点头。他说话很少,除了几个“好,好,好”以外,别的话也很难听清楚。\\

我的弟弟曾听母亲说过,辛亥那年父亲辞了摄政王位,从宫里一回来便对母亲说:“从今天起我可以回家抱孩子了!”母亲被他那副轻松神气气得痛哭了一场,后来告诫弟弟:“长大了万不可学阿玛(满族语父亲)那样!”这段故事和父亲自书的对联“有书真富贵,无事小神仙”,虽都不足以证明什么真正的“退隐”之志,但也可以看出他对那三年监国是够伤脑筋的。那三年可以说是他一生最失败的三年。\\

对他说来,最根本的失败是没有能除掉袁世凯。有一个传说,光绪临终时向摄政王托付过心事,并且留下了“杀袁世凯”四字朱谕。据我所知,这场兄弟会见是没有的。摄政工要杀袁世凯为兄报仇,虽确有其事,但是被奕劻为首的一班军机大臣给拦阻住了。详情无从得知,只知道最让父亲泄气的是奕劻的一番话:“杀袁世凯不难,不过北洋军如果造起反来怎么办?”结果是隆裕太后听从了张之洞等人的主意,叫袁世凯回家去养“足疾”,把他放走了。\\

有位在内务府干过差使的“遗少”给我说过,当时摄政王为了杀袁世凯,曾想照学一下康熙皇帝杀大臣鳌拜的办法。康熙的办法是把鳌拜召来,赐给他一个座位,那座位是一个只有三条好腿的椅子,鳌拜坐在上面不提防给闪了一下,因此构成了“君前失礼”的死罪。和摄政王一起制定这个计划的是小恭亲王溥伟\footnote{溥伟(1880-1937),恭亲王奕訢之孙,光绪二十四年袭王爵,辛亥革命前为禁烟大臣,辛亥后在德帝国主义庇护下寓居青岛,青岛被日本占领后又投靠日本。在此期间与升允等组织宗社党,不断进行复辟活动,“九·一八”事变后出任沈阳四民维持会会长,企图在日本支持下组织“明光帝国”,但不久即被抛弃,拿了日本人赏的一笔钱老死于旅顺。}。溥伟有一柄咸丰皇帝赐给他祖父奕訢的白虹刀,他们把它看成太上宝剑一样的圣物,决定由溥伟带着这把刀,做杀袁之用。一切计议停当了,结果被张之洞等人拦住了。这件未可置信的故事至少有一点是真的,这就是那时有人极力保护袁世凯,也有人企图消灭袁世凯,给我父亲出谋划策的也大有人在。袁世凯在戊戌后虽然用大量银子到处送礼拉拢,但毕竟还有用银子消除不了的敌对势力。这些敌对势力,并不全是过去的维新派和帝党人物,其中有和奕劻争地位的,有不把所有兵权拿到手誓不甘休的,也有为了其他目的而把希望寄托在倒袁上面的。因此杀袁世凯和保袁世凯的问题,早已不是什么维新与守旧、帝党与后党之争,也不是什么满汉显贵之争了,而是这一伙亲贵显要和那一伙亲贵显要间的夺权之争。以当时的亲贵内阁来说,就分成庆亲王奕劻等人的一伙和公爵载泽等人的一伙。给我父亲出谋划策以及要权力地位的,主要是后面这一伙。\\

无论是哪一伙,都有一群宗室觉罗、八旗世家、汉族大臣、南北谋士;这些人之间又都互有分歧,各有打算。比如载字辈的泽公,一心一意想把堂叔庆王的总揆夺过来,而醇王府的兄弟们首先所瞩目的,则是袁世凯等汉人的军权。就是向英国学海军的兄弟和向德国学陆军的兄弟,所好也各有不同。摄政王处于各伙人句心斗角之间,一会儿听这边的话,一会儿又信另一边的主意,一会对两边全说“好,好”,过一会又全办不了。弄得各伙人都不满意他。\\

其中最难对付的是奕劻和载泽。奕劻在西太后死前是领衔军机,太后死后改革内阁官制,他又当了内阁总理大臣,这是叫度支部尚书载泽最为忿忿不平的。载泽一有机会就找摄政王,天天向摄政王揭奕劻的短。西太后既搬不倒奕劻,摄政王又怎能搬得倒他?如果摄政王支持了载泽,或者摄政王自己采取了和奕劻相对立的态度,奕劻只要称老辞职,躲在家里不出来,摄政王立刻就慌了手脚。所以在泽公和庆王间的争吵,失败的总是载泽。醇王府的人经常可以听见他和摄政王嚷:“老大哥这是为你打算,再不听我老大哥的,老庆就把大清断送啦!”摄政王总是半晌不出声,最后说了一句:“好,好,明儿跟老庆再说……”到第二天,还是老样子:奕劻照他自己的主意去办事,载泽又算白费一次力气。\\

载泽的失败,往往就是载沣的失败,奕劻的胜利,则意味着洹上垂钓\footnote{一九零九年袁世凯被清廷罢斥后,息影于彰德迈水(安阳河),表面上不谈政治,曾经著蓑衣竹笠,作渔翁状,驾扁舟一叶,垂竿洹水滨,以示志在山水之间,其实仍与旧部来往不断,尤其是有“军师”徐世昌经常秘密向他报告国事政局,朝廷动向,并得到他暗中部署,因此,武昌事起,就有了徐世昌等联名保举及袁讨价还价的故事。}的袁世凯的胜利。摄政王明白这个道理,也未尝不想加以抵制,可是他毫无办法。\\

后来武昌起义的风暴袭来了,前去讨伐的清军,在满族陆军大臣荫昌的统率下,作战不利,告急文书纷纷飞来。袁世凯的“军师”徐世昌看出了时机已至,就运动奕劻、那桐几个军机一齐向摄政王保举袁世凯。这回摄政王自己拿主意了,向“愿以身家性命”为袁做担保的那桐发了脾气,严肃地申斥了一顿。但他忘了那桐既然敢出头保袁世凯,必然有恃无恐。摄政王发完了威风,那桐便告老辞职,奕劻不上朝应班,前线紧急军情电报一封接一封送到摄政工面前,摄政王没了主意,只好赶紧赏那桐“乘坐二人肩舆”,挽请奕劻“体念时艰”,最后乖乖地签发了谕旨:授袁世凯钦差大臣节制各军并委袁的亲信冯国璋\footnote{冯国璋(1857-1919),字华南,河北河间人,在清末亦是协助袁世凯创办北洋军的得力将领。在辛亥革命后成为北洋军阀的直系首领之一,是英美帝国主义的走狗。}、段祺瑞为两军统领。他垂头丧气地回到府邸后,另一伙王公们包围了他,埋怨他先是放虎归山,这回又引狼入室。他后悔起来,就请这一伙王公们出主意。这伙人说,让袁世凯出来也还可以,但要限制他的兵权,不能委派他的旧部冯国璋、段祺瑞为前线军统。经过一番争论之后,有人认为冯国璋还有交情,可以保留,于是载洵贝勒也要求,用跟他有交情的姜桂题来顶替段祺瑞。王公们给摄政王重新拟了电报,摄政王派人连夜把电报送到庆王府,叫奕劻换发一下。庆王府回答说,庆王正歇觉,公事等明天上朝再说。第二天摄政王上朝,不等他拿出这一个上谕,奕劻就告诉他,头一个上谕当夜就发出去了。\\

我父亲并非是个完全没有主意的人。他的主意便是为了维持皇族的统治,首先把兵权抓过来。这是他那次出使德国从德国皇室学到的一条:军队一定要放在皇室手里,皇族子弟要当军官。他做得更彻底,不但抓到皇室手里,而且还必须抓在自己家里。在我即位后不多天,他就派自己的兄弟载涛做专司训练禁卫军大臣,建立皇家军队。袁世凯开缺后,他代替皇帝为大元帅,统率全国军队,派兄弟载洵为筹办海军大臣,另一个兄弟载涛管军谘处(等于参谋总部的机构),后来我这两位叔叔就成了正式的海军部大臣和军谘府大臣。\\

据说,当时我父亲曾跟王公们计议过,无论袁世凯镇压革命成功与失败,最后都要消灭掉他。如果他失败了,就借口失败诛杀之,如果把革命镇压下去了,也要找借口解除他的军权,然后设法除掉他。总之,军队决不留在汉人手里,尤其不能留在袁世凯手里。措施的背后还有一套实际掌握全国军队的打算。假定这些打算是我父亲自己想得出的,不说外界阻力,只说他实现它的才能,也和他的打算太不相称了。因此,不但跟着袁世凯跑的人不满意他,就连自己的兄弟也常为他摇头叹息。\\

李鸿章的儿子李经迈出使德国赴任之前,到摄政王这里请示机宜,我七叔载涛陪他进宫,托付他在摄政王面前替他说一件关于禁卫军的事,大概他怕自己说还没用,所以要借重一下李经迈的面子。李经迈答应了他,进殿去了。过了不大功夫,在外边等候着的载涛看见李经迈又出来了,大为奇怪,料想他托付的事必定没办,就问李经迈是怎么回事。李经迈苦笑着说:“王爷见了我一共就说了三句话:‘你哪天来的?’我说了,他接着就问:‘你哪天走?’我刚答完,不等说下去,王爷就说:‘好好,好好地干,下去吧!’——连我自己的事情都没说,怎么还能说得上你的事?”\\

我祖母患乳疮时,请中医总不见好,父亲听从了叔叔们的意见,请来了一位法国医生。医生打算开刀,遭到了醇王全家的反对,只好采取敷药的办法。敷药之前,医生点上了酒精灯准备给用具消毒,父亲吓坏了,忙问翻译道:\\

“这这这干么?烧老太太?”\\

我六叔看他这样外行,在他身后对翻译直摇头咧嘴,不让翻给洋医生听。\\

医生留下药走了。后来医生发现老太太病情毫无好转,觉得十分奇怪,就叫把用过的药膏盒子拿来看看。父亲亲自把药盒都拿来了,一看,原来一律原封未动。叔叔们又不禁摇头叹息一番。\\

醇王府的大管事张文治是最爱议论“王爷”的。有一回他说,在王府附近有一座小庙,供着一口井,传说那里住着一位“仙家”。“银锭桥案件”\footnote{银锭桥在北京地安门附近,是载沣每天上朝必经之地。一九一零年汪精卫、黄复生为刺杀载沣秘密埋藏自制炸弹于桥下,因被军警识破,计划未遂。汪、黄被捕后,清廷慑于当时民气,未敢处以极刑,南北议和时即予释放。当时把这案件叫做银锭桥案件。
}败露后,王爷有一次经过那个小庙,要拜一拜仙家,感谢对他的庇佑。他刚跪下去,忽然从供桌后面跳出个黄鼠狼来。这件事叫巡警知道了,报了上去,于是大臣们就传说王爷命大,连仙家都受不了他这一拜。张文治说完了故事就揭穿了底细,原来这是王爷叫庙里人准备好的。\\

醇王府的人在慈禧死后都喜欢自称是维新派,我父亲也不例外。提起父亲的生活琐事,颇有不少反对迷信和趋向时新风气的举动。我还听人说过,“老佛爷并不是反对维新的,戊戌以后办的那些事不都是光绪要办的吗?醇亲王也是位时新人物,老佛爷后来不是也让他当了军机吗?”慈禧的维新和洋务,办的是什么,不必说了。关于父亲的维新,我略知一些。他对那些曾被“老臣”们称为奇技淫巧的东西,倒是不采取排斥的态度。醇王府是清朝第一个备汽车、装电话的王府,他们的辫子剪得最早,在王公中首先穿上西服的也有他一个。但是他对于西洋事物真正的了解,就以穿西服为例,可见一斑。他穿了许多天西服后,有一次很纳闷地问我杰二弟:“为什么你们的衬衫那么合适,我的衬衫总是比外衣长一块呢?”经杰二弟一检查,原来他一直是把衬衫放在裤子外面的,已经忍着这股别扭劲好些日子了。\\

此外,他曾经把给祖母治病的巫婆赶出了大门,曾经把仆役们不敢碰的刺猬一脚踢到沟里去,不过踢完之后,脸上却一阵煞白。他反对敬神念佛,但是逢年过节烧香上供却非常认真。他的生日是正月初五,北京的风俗把这天叫做“破五”,他不许人说这两个字,并在日历的这一页上贴上红条,写上寿宇,把坚笔拉得很长。杰二弟问他这是什么意思,他说:“这叫长寿嘛!”\\

为了了解摄政王监国三年的情况,我曾看过父亲那个时候的日记。在日记里没找到多少材料,却发现过两类很有趣的记载。一类是属于例行事项的,如每逢立夏,必“依例剪平头”,每逢立秋,则“依例因分发”;此外还有依例换什么衣服,吃什么时鲜,等等。另一类,是关于天象观察的详细记载和报上登载的这类消息的摘要,有时还有很用心画下的示意图。可以看出,一方面是内容十分贫乏的生活,一方面又有一种对天文的热烈爱好。如果他生在今天,说不定他可以学成一名天文学家。但可惜的是他生在那样的社会和那样的家庭,而且从九岁起便成了皇族中的一位亲王。\\