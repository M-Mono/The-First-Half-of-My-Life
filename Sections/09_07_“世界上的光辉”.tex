\fancyhead[LO]{{\scriptsize 1955-1959: 接受改造 · “世界上的光辉”}} %奇數頁眉的左邊
\fancyhead[RO]{} %奇數頁眉的右邊
\fancyhead[LE]{} %偶數頁眉的左邊
\fancyhead[RE]{{\scriptsize 1955-1959: 接受改造 · “世界上的光辉”}} %偶數頁眉的右邊
\chapter*{“世界上的光辉”}
\addcontentsline{toc}{chapter}{\hspace{1cm} “世界上的光辉”}
\thispagestyle{empty}
从一九五六年下半年起,经常有些外国记者和客人来访问我,还有些外国人写信给我,向我要照片。一九五七年二月,我接到从法国斯梯林一温德尔寄来的一封信,请求我在照片上签字,信里除了附来几张我过去的照片外,还有一篇不知要做什么用的文章,文章全文如下:\\

\begin{quote}
	监牢里的中国皇帝\\

世界上的光辉是无意义的,这句话是对一个关在红色中国的抚顺监牢里,等待判决的政治犯人的一生写照。在孩童时期,他穿的是珍贵的衣料,然而现在却穿着破旧的棉布衣服,在监牢的园子里独自散步。这个人的名字是:亨利溥仪。五十年前,他的诞生伴随着奢华的节日的烟火,但是现在牢房却成了他的住处。亨利在两岁时做了中国的皇帝,但以后中国的六年内战把他从皇帝宝座上推了下来。一九三二年对于这位“天子”来说,又成为一个重要的时期:日本人把他扶起来做满洲国的皇帝。第二次世界大战以后,人们再也没有听到关于他的什么事,一直到现在这张引人注意的照片报道他的悲惨的命运为止……\\
\end{quote}

如果他早两年寄来,或者还能换得我一些眼泪,但是他寄来的太晚了。我在回信中回答他说:“对不起,我不能同意你的见解。我不能在照片上签字。”\\

不久前,在某些外国记者的访问中,我遇过不少奇怪的问题,例如:“作为清朝最末一位皇帝,你不觉得悲哀吗?”“长期不审判你,你不觉得不公平吗?这不令你感到惊奇吗?”等等,这里面似乎也包含着类似的同情声调。我回答他们说,如果说到悲哀,我过去充当清朝皇帝和伪满皇帝,那正是我的悲哀;如果说到惊奇,我受到这样的宽大待遇,倒是很值得惊奇的。记者先生们对我的答案,似乎颇不理解。我想那位从法国写信来的先生,看到了我的回答必然也有同感。\\

在我看来,世界上的光辉是什么呢?这是方素荣的那颗伟大的心,是台山堡那家农民的朴素语言,是在我们爱新觉罗下一代身上反映出来的巨大变化,是抚顺矿山的瓦斯灶上的火苗,是工业学校里的那些代替了日本设备的国产机床,是养老院里老工人的晚年,……难道这些对我都是没有意义的吗?\\

难道我被寄予做个正经人的希望和信任,这是对我没有意义的吗?难道这不是最宝贵的审判吗?\\

我相信,这不仅是我个人的心情,而是许多犯人共同的心情,甚至于是其中一些人早有的心情。事实上,这种争取重新做人的愿望与信念,正逐渐在日益增多的心中生长着(这时我们已经开始把改造当做是自己的事了),否则的话,一九五七年的新年就不会过得那样与前不同。\\

我们每次逢年过节,在文娱活动方面,除了日常的球、棋、牌和每周看两次的电影之外,照例要组织一次晚会,由几个具有这方面才干的人表演一些小节目,如伪满将官老龙的戏法,小固的快板,老佑的清唱,溥杰的《萧何月下追韩信》,蒙古人老正和老郭的蒙古歌曲,等等。溥杰偶尔也说一次自编的相声,大家有时也来个大合唱。观众就是我们一所的这几十个人,会场就在我们一所的甬道里或者小俱乐部里。甬道里从新年的前几天就开始张灯结彩,布置得花花绿绿。有了这些,再加上年节丰盛的伙食和糖果零食,使大家过得很满意。可是一九五七年这一次不行了,大家觉得别的全好,惟独这个甬道晚会有点令人不能满足;如果能像日本战犯似的在礼堂里组织一次大型晚会,那才过瘾。离着新年还很远,许多人就流露出了这种愿望。到了该着手筹备过年的时候,一些年轻的学委们憋不住了,向所方提出了这个意思。所方表示,倘若有信心,办个大型的也可以,并且说如果能办起来,可以让新调来的三、四两所的蒋介石集团的战犯做观众,把礼堂装得满满的。学委们得到了这样的答复,告诉了各组,于是大家兴高采烈地筹备起来了。\\

大家之所以高兴,是因为都想过个痛快的新年,而所方之所以支持,是因为这是犯人们进行自我教育的成功的方法之一。学委会是首先接受了这个思想的。他们早从日本战犯的演出得到了启发。日本战犯每次晚会除了一般的歌舞之外,必有一场戏剧演出,剧本大都是根据日本报刊上的材料自己编的。记得一出名叫《原爆之子》的戏,描写的是蒙受原子弹灾祸的日本人民的惨状,这出戏控诉了日本军国主义给世界人民而且也给日本人民造下的罪行,演到末尾,台上台下是一片控诉声和哭泣声。学委会看出了日本战犯们通过演戏的办法,编剧者、演剧者和观剧者达到了互相教育、互相帮助的效果,决心也要在这次晚会上演出一出这样的戏。学委会的计划得到了许多人的拥护,他们很快就把戏的大概内容和剧名都想出来了。一共是两出戏,一出名叫《侵略者的失败》,内容写的是英军侵略埃及、遭到埃及人民的反击而失败的故事,这是一出时事活报剧;另一出是写一个伪满汉奸,从当汉奸到改造的经历,这是一出故事剧,名叫《从黑暗走向光明》。剧作家也有了,这就是溥杰和一个前江伪政权的穆姓官员。事情一决定,他们便马上写起剧本来。\\

与剧本的创作同时进行的,是其他各项节目的准备工作。“幻术家”老龙的戏法向来最受欢迎,现在他对于以前玩的“帽子取蛋”、“吞乒乓球”之类的小戏法,觉得不过瘾了,声明要表演几个惊人的大型戏法。蒙古人老正兄弟和老郭等人在准备蒙古歌舞。我们组的学习组长老初,一个前汪伪政权的外交官,是位音乐爱好者,他带了一批人在练习合唱。还有一些人分头准备相声、快板、清唱等等传统节目。这几天最忙的是学委会主任老万,他忙于排节目,找演员,计划节日会场的布置。会场布置由小瑞负责,他是制作纸花和灯笼的巧手,在他的指导下,一些人用各色花纸做了灯笼、纸花以及张灯结彩所必需的一切饰物。全场的照明设备由大李负责,他现在成了一名出色的电工。我的固侄也够忙的,他除了做幻术家的助手之外,还要准备说相声,参加练唱。在各方带动之下,人人被卷入了筹备活动。\\

以前每次甬道里的晚会,任何一项准备工作都没有我的份。我不会说快板,也不会变戏法,别人也不找我去布置会场。就是帮人家拿拿图钉、递个纸条,人家还嫌我碍事哩。在这次筹备工作中,我原先认为不会有人找我去添麻烦,万没料到,我们的组长老初竟发现我唱歌发音还过得去,把我编进了歌咏队。我怀着感激之情,十分用心地唱熟了“东方红”、“歌唱祖国”、“全世界人民心一条”。歌曲刚练熟,又来了一件出乎意外的事,学委会主任找我来了。\\

“溥仪,第一出戏里有个角色,由你扮演吧!并不太难,台词不多,而且,这是文明戏,可以即兴编词,不太受约束。这件工作很有意义,这是自我教育,这……”\\

“不用说服啦,”我拦住了他,“只要你看我行,我就干!”\\

“行!”老万高兴得张开大嘴,“你行!你一定行,你的嗓音特别洪亮!你……”\\

“过奖过奖!你就说我演什么戏吧?”\\

“《侵略者的失败》——这是剧名。英国侵略埃及,天怒人怨,这是根据报上的一段新闻编的。主角老润,演外交大臣劳埃德。你演一个左派工党议员。”\\

我到溥杰那里了解了剧情,看了剧本,抄下了我的台词,然后便去挑选戏装。既然是扮演洋人,当然要穿洋装。这类东西在管理所的保管室里是不缺少的,因为许多人的洋装都存在这里。\\

我拿出了那套在东京法庭上穿过的藏青色西服,拿了衬衣。领带等物,回到了监房。监房里正空无一人,我独自打扮了起来。刚换上了一件箭牌的白府绸衬衣,老元进来了,他吓了一跳,怔了半晌才问我:\\

“你这是于什么?”\\

我一半是由于兴奋,一半是由于衬衣的领子太紧,一时说不出话来。\\

“我要演戏,”我喘吁吁地说,“来,帮我把马甲后面的带子松一松。”\\

他给我松了,可是前面的扣子还是系不上,我才知道自己比从前胖了。那双英国惠罗公司的皮鞋也夹脚,我懊恼地问老元:\\

“我演一个英国工党的议员,不换皮鞋行不行?”\\

“得啦吧!”老元说,“英国工党议员还擦香水哩,难道还能穿棉靴头吗?不要紧,你穿一会儿也许就不紧了,这马甲可以拾摄一下,你先去念台词吧。你也上台演戏,真是奇闻!哈哈!……”\\

我走到甬道里,还听见他的笑声。但我很高兴。我记着老万告诉我的话,这个演出是自我教育,也是一种互助。我这还是第一次被放在帮助别人的地位,过去我可总是被人帮助的。原来我也和别人一样,有我的才能,在互助中能处于平等的地位呢。\\

我走到俱乐部,开始背诵抄来的台词。从这一刻起,我无时不在背诵我的台词。老万说的不错,台词很短,大概这是台上说话的演员台词中最少的一个了。按照剧情,演到最后,劳埃德在议会讲台上为他的侵略失败做辩解时,一些反对党的议员们纷纷起立提出质问,后来群起而攻之,这时我在人群中起立,随便驳他几句,然后要说出这么几句话:“劳埃德先生,请你不用再诡辩了,事实这就是可耻,可耻,第三个还是可耻!”最后会议沸腾着怒骂声,纷纷要求劳埃德下台,我喊道:“滚下去!滚下去!”这个剧没有什么复杂情节,主要是会场辩论,从一个议员提出质问开始,到外交大臣被轰下台,用不了十五分钟。但是我为了那几句台词,费了大概几十倍的时间。我惟恐遗忘掉或说错,辜负了别人的期望。从前我曾为忧愁和恐惧搅得失眠、梦呓,现在我第一次因兴奋和紧张而睡不着觉了。\\

新年到了。当我走进了新年晚会会场时,我被那节日的气氛和漂亮的舞台完全吸引住了,忘掉了内心的紧张。五彩缤纷的装饰和巧夺天工的花朵,令人赞叹不止。灯光的装置是纯粹内行的章法,舞台的楣幅上红地白字“庆祝新年同乐晚会”,是艺术宋体,老万的手笔。布幕上的“今晚演出节目表”是最吸引人的:一、合唱,二、独唱,三、蒙古歌舞,四、相声,五、快板,六、戏法,七、活报剧《侵略者的失败》,八、话剧《从黑暗走向光明》。一切都不比日本战犯的晚会逊色。看到坐在中间的蒋军战犯的窃窃私议和赞赏的神态,我们这伙人也不禁高兴地互相递眼色。\\

扩音器里送来了老万的开场白,然后是合唱开始。一个个节目演下去,会场上掌声一阵接着一阵。轮到老龙的大型戏法,会场上的情绪进入了高潮。《大变活人》演到最后,活人小固从空箱子里钻了出来,笑声和掌声响成了一片。后来表演者从一个小纸盘里拉出无限多的彩带,最后拉出一幅彩旗,现出了“争取改造,重新做人”这几个大字时,欢呼声、掌声和口号声响得令人担心天花板会震下来。这时我走进了后台,开始化装。\\

会场休息片刻后,活报剧开场了。舞台上开始了关于苏伊士运河战争失败的辩论。老润扮的劳埃德像极了,他的鼻子本来就大,这个议会里所有的英国公民,只有他一个人最像英国人。他的表情也很出色,恼恨、忧惧、无可奈何而又外示矜持,活活是个失败的外交大臣。我身旁坐着老元,他也是一位议员,对外交大臣做出很不耐烦的样子。我们工党左派议员共有十几个人,在舞台上占据着正面,舞台侧面是保守党议席,那里的人较少,做出灰溜溜的样子。戏演了十多分钟,老元悄悄地对我说(这姿势是剧本里要求的):“你别那么楞着,来点动作!”我欠欠身,抬头张望了一下台下,这时发现那些观众们似乎对台上注意的不是劳埃德而是我这位左派议员,我心慌起来。在合唱时观众还没有人注意我,现在我成了视线的集中目标了。我的镇静尚未恢复过来,老元碰了我一下子:“你说呀,该你说几句驳他了!”我慌忙站了起来,面对信口开河的老润,一时想不起台词来了。正在紧张中,忽然情急智生,我用英文连声向他喊道:“NO!NO!NO!……”我这一喊,果然把他的话打断了,同时我也想起了下面的台词,连忙接下去说:“劳埃德先生,请你不用再诡辩了,”我一手叉腰,一手指着他:“事实这就是可耻,可耻,第三个还是可耻!”接着,我听见了台下一片掌声,台上一片“滚下去!滚下去!”的喊声,外交大臣劳埃德仓皇失措地跑下台去了。\\

“你演得不错!”老元下了台,第一个称赞了我。“虽然慌了一点,还真不错!”后来其他人也表示很满意,对我的即兴台词笑个不住。还有人提起当年我拒绝会见曾与梅兰芳先生合过影的瑞典王子的事,我也不禁哈哈大笑。\\

骚动着的会场逐渐平静下来,话剧《从黑暗走向光明》开场了。\\

这出戏的情节把人们引进了另外一个境界里。第一场写的是东北旧官僚吴奇节、卜世位二人在东北沦陷时,摇身一变为大汉奸,第二场写他们在日寇投降时正想勾搭国民党,被苏军逮捕了,第三场是被押回国后,在改造中还玩一套欺骗手法,但是终于无效,最后在政府的教育和宽大政策的感召下,认了罪,接受了改造。剧本编得并不算高明,但是战犯们对这个富有代表性的故事非常熟悉,每个人都可以从剧中人找到自己的影子,回忆起自己的过去,因此都被吸引住了,而且越看越觉得羞耻。戏里有一段是汉奸强迫民工修神武天皇庙,大下巴看出这是他的故事,不禁喃喃地说:“演这丢人事于什么?”演到汉奸们坐在一间会议室里,给日本人出主意掠夺东北人民的粮食,做出谄媚姿态的时候,我听到旁边有人唉声叹气,说:“太丑了!”我感到最丑的还数不上这个剧中人物,而是在那个伪机关会议室里的一个挂着布帘的木龛,那是伪满当时每个机关里不可少的东西,里面供奉着所谓“御真影”——汉奸皇帝的相片。当剧中人人场后对它鞠躬时,我觉得世界上没有比这更丑的东西了。\\

这出戏演到最后一幕,政府人员出来向吴奇节、卜世仁讲解了改造罪犯的政策时,会场上的情绪达到了整个晚会的最高峰,掌声和口号声超过了以往我听到的任何一次。这与其说是由于剧情,不如说是由于几年来生活的感受,特别是由于最近从家属来信、家属会见、外出参观、日本战犯在中国法庭上认罪等等一系列事情上直接获得的感受,今天一齐发生了作用。在震耳的口号声和鼓掌声中,还有被湮没的哭泣声。在我前面几排处一个矮胖的人,低垂着白头,两肩抽搐着,这是和溥杰同组的老刘,那个不亲眼看见女儿就不相信事实的人。在我后面哭得出了声的是那个恢复了父亲身份的老张,他的胸袋上的金笔正闪闪发光。\\

晚会上出现的激动情绪,充分地说明了这个世界对我们存在着“光辉”,而且是越来越明亮的光辉。新年过去不久,有一批人得到了免诉处理,被释放了。这一批共十三人,其中有我的三个侄子和大李。在热烈的送别之后,我们又度过了一个更欢腾的春节,我们又组织了更好的晚会(演出第二个自编的剧目《两个时代的春节》,这个剧描写的是一个东北村庄在伪满与解放后不同的景况)。春节过后,第二批四名犯人又得到了释放,其中有我的两个妹夫。在这时候,那位法国人却给我写来了那封说什么“世界上的光辉”的信!\\