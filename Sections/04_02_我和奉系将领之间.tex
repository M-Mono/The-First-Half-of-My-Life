\fancyhead[LO]{{\scriptsize 1924-1930: 天津的“行在” · 我和奉系将领之间}} %奇數頁眉的左邊
\fancyhead[RO]{} %奇數頁眉的右邊
\fancyhead[LE]{} %偶數頁眉的左邊
\fancyhead[RE]{{\scriptsize 1924-1930: 天津的“行在” · 我和奉系将领之间}} %偶數頁眉的右邊
\chapter*{我和奉系将领之间}
\addcontentsline{toc}{chapter}{\hspace{1cm} 我和奉系将领之间}
\thispagestyle{empty}
\begin{quote}
	八月初五日,早七时起,洗漱毕,萧丙炎诊脉。八时,郑孝胥讲《通鉴》。九时,园中散步,接见康有为。十时余,康辞去,这张宪及张庆昶至,留之早餐,赐每人福寿字一张,在园中合摄一影。张宪为李景林部之健将,张庆昶为孙传芳部之骁将。十二时辞去。接见济煦,少时即去。余用果品并用茶点,适英国任萨姆女士至,与之相谈。皇后所召之女画士亦至,余还寝室休息。在园中骑车运动,薄暮乘汽车出园,赴新购房地,少时即返。八时余晚餐,休息,并接见结保川医士。十一时寝\footnote{萧丙炎是清末都察院御史,任萨姆女士是婉容的英文教师。}。\\

八月初六日,早八时余起。十时召见袁励准。十一时早餐,并见结保川。\\

十二时接见康有为,至一时康辞去,陈师傅来见。三时休息。鲁军军长毕庶澄及其内兄旅长常之英来谒,少时辞去。少顷吴忠才至,托其南下时代向吴佩孚慰问。六时毕翰章来谒,六时余辞去。余在园内散步,适荣源至,稍谈,余即入室休息。\\
\end{quote}

从这仅存的一九二七年的一页日记中,可以看出当时我的日常生活和接见的人物。从一九二六到一九二八年,毕庶澄、张宗昌等人是张园的经常客人。除他们之外,我还接见过张学良、褚玉璞、徐源泉、李景林等等奉系将领。第一个和我见面的是李景林。我到天津时,正是刚战胜吴佩孚的奉军占领着天津,奉系的直隶督办李景林立即以地方官的身份来拜访我,表示了对我保护之意。尽管他和当时任何的中国将军一样,他们的军法政令是进不了“租界”的。\\

我在天津的七年间,拉拢过一切我想拉拢的军阀,他们都给过我或多或少的幻想。吴佩孚曾上书向我称臣,张作霖向我磕过头、段祺瑞主动地请我和他见过面。其中给过我幻想最大的,也是我拉拢最力、为时最长的则是奉系将领们。这是由张作霖向我磕头开始的。\\

我到天津的这年六月,荣源有一天很高兴地向我说,张作霖派了他的亲信阎泽溥,给我送来了十万元,并且说张作霖希望在他的行馆里和我见一见。这件事叫陈宝琛知道了,立刻表示反对,认为皇上到民国将领家去见人,而且去的地方是租界外面,那是万万不可以的。我也觉得不能降这种身份和冒这个险,所以拒绝了。不料第二天的夜里,荣源突然把阎泽溥领了来,说张作霖正在他住的地方等着我,并且说中国地界内决无危险,张作霖自己不便于走进租界,所以还是请我去一趟。经过荣源再三宣传张作霖的忠心,加之我想起了不久前他对我表示过的关怀,我又早在宫里就听说过,除了张勋(二张还是儿女亲家)之外,张作霖是对于清朝最有感情的。因此,我没有再告诉别人,就坐上汽车出发了。\\

这是初夏的一个夜晚,我第一次出了日本租界,到了张作霖的“行馆”曹家花园。花园门口有个奇怪的仪仗队——穿灰衣的大兵,手持古代的刀枪剑戟和现代的步枪,从大门外一直排列到大门里。汽车经过这个行列,开进了园中。\\

我下了汽车,被人领着向一个灯火辉煌的大厅走去。这时,迎面走来了一个身材矮小、便装打扮、留着小八字胡的人,我立刻认出这是张作霖。我迟疑着不知应用什么仪式对待他——这是我第一次外出会见民国的大人物,而荣源却没有事先指点给我——出乎意外的是,他毫不迟疑地走到我面前,趴在砖地上就向我磕了一个头,同时问:“皇上好?”\\

“上将军好?”我就着劲,扶起他,一同走向客厅门。我心里很高兴,而且多少——虽然这已不像一个皇上的心理——有点感激他刚才那个举动,这把我从“降贵纤尊”中感到的不自在消除了。当然,我更高兴的是,这个举足轻重的人物看来是并不忘旧的。\\

客厅里摆的是硬木桌椅、西式沙发、玻璃屏风,非常讲究而又不伦不类。我们在一个圆桌边对面坐下,张作霖一支接一支地抽着纸烟,打开了话匣子。他一张嘴先痛骂冯玉祥“逼宫”,说冯玉祥那是为了要拿宫中的宝物,而他是非常注意保护古代文化和财宝的,由于这个缘故,他不但把奉天的宫殿保护得很好,而且这次把北京的一套四库全书也要弄去,一体保护。他带着见怪的口气说,我不该在他带兵到了北京之后,还向日本使馆里跑,而他是有足够力量保护我的。他问我出来之后的生活,问我缺什么东西,尽管告诉他。\\

我说,张上将军对我的惦念,我完全知道,当时因为冯玉祥军队还在,实是不得已才进了日本使馆的。我又进一步说,奉天的宗庙陵寝和宫殿,我早已知道都保护得很好,张上将军的心意,我是明白的。\\

“皇上要是乐意,到咱奉天去,住在宫殿里,有我在,怎么都行。”\\

“张上将军真是太好了,……”\\

但是这位张上将军却没有接着再说这类话,就把话题转到我的生活上去了:“以后缺什么,就给我来信。”\\

我缺的什么?缺的是一个宝座,可是这天晚上我无法把它明说出来,这是显然的事。\\

我们谈话时,没有人在场,和我们在一起的只有一屋子的苍蝇。我立刻意识到,深夜里还有苍蝇飞,这在租界里是没有的现象。\\

后来,有个副官进来说:“杨参谋长(宇霆)求见。”张作霖挥挥手说:“不着忙,呆回儿再说!”我忙站起来说:“上将军很忙,我就告辞了。”他连忙说:“不着忙,不着忙。”这时似乎有个女人的脸在屏风后问了一下(后来听说是张作霖的五姨太太),我觉得他真是忙,再度告辞,这回他不拦阻了。\\

我每逢外出,驻张园的日本便衣警察必定跟随着,这次也没例外。我不知道张作霖看没看见站在汽车旁边的那个穿西服的日本人,他临送我上车时,大声地对我说:\\

“要是日本小鬼欺侮了你,你就告诉我,我会治他们!”\\

汽车又通过那个奇怪的仪仗队,出了曹家花园,开回到租界上。第二天日本总领事有田八郎向我提出了警告:\\

“陛下如再私自去中国地界,日本政府就再不能保证安全!”\\

虽然张作霖说他会治日本小鬼,虽然日本领事提出这样的抗议,但是当时任何人都知道日本人和张作霖的关系,如果不是日本人供给张作霖枪炮子弹,张作霖未必就能有这么多的军队。所以由这次会见在我心里所升起的希望,并没有受到这个抗议的影响,更不用说陈宝琛那一派的反对了。\\

我的复辟希望更被后来的事实所助长,这就是以提出“田中奏折”\footnote{“田中奏折”是田中上日本天皇的秘密奏折,奏折说:“吾人如被征服中国,要先征服满蒙,吾人如能征服中国,则其余所有亚洲国家及南洋诸国,均将畏惧于我,投降于我。……当吾人得以支配中国全部资源之后,吾人将更能进而征服印度、南洋群岛、小亚细亚以至欧洲。”又说:“第一步征服台湾,第二步征服朝鲜,现皆实现,惟第三步的灭亡满蒙以及征服中国全土,……则尚未完成。”
}出名的田中内阁,于一九二七年上台后所表现的态度。田中奏折返于一九二九年才揭发出来,其实它的内容在一九二七年就露出来了。这里我引述一段《远东国际军事法庭判决书》上对当时情势的叙述\footnote{判决书于一九四八年公布。}:\\

\begin{quote}
	田中首相所提倡的“积极政策”是借着与满洲当局、特别是与东北边防军总司令及满洲、热河的行政首长张作霖的合作,以扩大和发展日方认为已在满洲取得了的特殊权益。田中首相还曾声明说:尽管日本尊重中国对满洲的主权,并愿尽可能的实行对华“门户开放政策”,但日本具有充分的决心,绝对不允许发生扰乱该地的平静和损害日本重大权益的情势。\\

田中内阁强调必须将满洲看做和中国其他部分完全不同的地方,并声明如果争乱从中国其他地方波及满洲和蒙古时,日本将以武力来保护它在该地的权益。\\
\end{quote}

给我磕头的张作霖,在得到田中内阁的支持之后,成了北方各系军人的领袖,做了安国军总司令,后来又做了军政府的大元帅。当蒋介石的军队北上的时候,“保护”满蒙地区“权益”的日本军队,竟开到远离满蒙数千里的济南,造成了惊人的“济南惨案”。日本军队司令官冈村宁次还发了一份布告警告过蒋介石。天津日本驻屯军参谋官为表示对我的关切,曾特地抄了一份给我。蒋介石为了讨好帝国主义,刚杀过了共产党和工人、学生,看见了这份布告,又恭恭敬敬地退出了济南,并禁止民众有任何反日行动。\\

在此同时,我和奉系将领之间也进入了紧张的接触。\\

公开的酬酢往来,是从我见过张作霖之后开始的。我父亲的大管家张文治,在奉军将领中有不少的把兄弟,这时又和张宗昌换了帖,成了奉军将领的引见人之一。前内城守卫队军乐队长李士奎,这时也成了奉军人物,褚王璞和毕庶澄就是他引进的。胡若愚还给我带来了张学良。这些将领们到张园来,已和从前进紫禁城时不同,他们不用请安叩头,我不用赏朝马肩舆,他们只给我鞠个躬,或握一下手,然后平起平坐。我给他们写信,也不再过分端皇帝架子。我和奉军将领交往的亲疏,决定于他们对复辟的态度。最先使我发生好感的是毕庶澄,因为他比别人更热心于我的未来事业,什么“人心思旧”、“将来惟有帝制才能救中国,现在是群龙无首”,说的话跟遗老遗少差不了多少。他是张宗昌的一名军长,兼渤海舰队司令,曾请我到他的军舰参观过。我对他抱着较大的希望,后来听到他被褚玉璞枪毙的消息时,我曾大为伤感。他死后,我的希望便转移到了张宗昌身上。\\

张宗昌,字效坤,山东掖县人。我在天津见到他的时候,他有四十多岁,一眼看去,是个满脸横肉的彪形大汉,如果一细看,就会发现这个彪形大汉的紫膛面皮上,笼着一层鸦片中毒的那种青灰色。他十五、六岁时流浪到营口,在“宝棚”当过赌佣,成天与地痞流氓赌棍小偷鬼混,在关东当过胡匪的小头目,以后又流落到沙俄的海参崴,给华商总会当门警头目。由于他挥霍不吝和善于逢迎勾结,能和沙俄宪兵警察紧密合作,竟成了海参崴流氓社会的红人,成了包娼、包赌、包庇烟馆的一霸。武昌起义后,南方革命军派人到中俄边境,争取胡子头目刘弹子(王双)投效革命,双方谈判成功,将刘部编为一个骑兵团,授刘为骑兵团长。张是中间的介绍人,一同到了上海,不知道他怎么一弄,自己成了革命军的团长,刘弹子反而成了他下面的一名营长。“二次革命”爆发,他投了反革命的机,以屠杀革命军人之功,得到了冯国璋的赏识,当上了冯的卫队营营长,以后层层运动,又得到了十一师师长的位置。不久在江苏安徽战败,逃亡出关,投奔张作霖,当了旅长。从此以后,他即借奉军之势,从奉军进关那天起,步步登高,由师长、军长而山东军务督办、苏皖鲁剿匪总司令,一直做到了直鲁联军司令,成了割据一方的土皇帝。由于他流氓成性,南方报纸曾给了他一个“狗肉将军”的绰号,后来看他打仗一败即跑,又给了他一个“长腿将军”的别名。\\

一九二八年四月二日,在蒋介石和张学良夹击之下,张宗昌兵败滦河,逃往旅大,后来又逃到日本门司,受日本人的庇护。一九三二年他以回家扫墓的名义回到山东,暗地里运动刘珍年部下倒戈,打算以倒戈队伍为基础,重整旗鼓,夺取当时山东省主席韩复榘的地盘,恢复其对山东的统治。一九三二年九月三日,他在济南车站被一个叫郑继成的当场打死。这位凶手自首说是为叔父报仇(他的叔父是被张宗昌枪毙的冯玉祥部下军长郑金声),实际是山东省主席韩复榘主使下的暗杀。据说张被打死后,他的尸首横在露天地里,他的秘书长花钱雇不到人搬运他的尸体,棺材铺的老板也不愿意卖给他棺材,后来还是主持谋杀的省当局,叫人收了尸。这个国人皆曰可杀的恶魔,曾是张园的熟客,是一个被我寄托以重大希望的人物。\\

我在北府时,张宗昌就化装来看过我,向我表示过关心。我到天津后,只要他来天津,必定来看我。每次来都在深夜,因为他白天要睡觉,晚上抽了大烟,精神特别足。谈起来,山南海北,滔滔不绝。\\

一九二六年,张吴联合讨冯,与冯军激战于南口,冯军退后,首先占领南口的是张宗昌的队伍。我一听到这个好消息,立刻给张宗昌亲笔写了一封半信半谕的东西:\\

\begin{quote}
	字问效坤督办安好,\\

久未通信,深为想念,此次南口军事业已结束,讨赤之功十成八九,将军以十万之众转战直鲁,连摧强敌,当此炎夏,艰险备尝,坚持讨逆,竟于数日内,直捣贼穴,建此伟大功业,挽中国之既危,灭共产之已成。\\

今赤军虽已远飏,然根株不除,终恐为将来之患,仍望本除恶务尽之意,一鼓而荡平之,中国幸甚,人民幸甚。现派索玉山赠与将军银瓶一对,以为此次破南口之纪念,望哂纳。\\

汉卿、芳宸、蕴山\footnote{索玉山是前禁卫军的团长,汉卿是张学良,芳宸是李景林,蕴山是褚玉璞。}均望致意\\

\begin{flushright}
	丙寅七月十三日\\
\end{flushright}
\end{quote}

我得到张宗昌胜利的消息,并不慢于报纸上的报道,因为我有自己的情报工作。有一些人为我搜集消息,有人给我翻译外文报纸。我根据中外报纸和我自己得到的情报,知道了张宗昌的胜利和声势,简直是令我心花怒放。我希望张宗昌得到全面胜利,为我复辟打下基础。但是这位“狗肉将军”在飞黄腾达的时候,总不肯明确地谈这些事,好象只有变成了“长腿将军”的时候,才又想起它来。\\

一九二八年,蒋介石、冯玉祥、阎锡山等人宣告合作,向北方的地盘上扑了过来,津浦线的这一路,绕过了给张宗昌帮忙的日本人,把张宗昌的根据地山东吞没了。张宗昌兵败如山倒,一直向山海关跑。这时张作霖已被日本人炸死,“少帅”张学良拒绝张宗昌出关。张宗昌的军队被困在芦台、滦州一线,前后夹击,危在旦夕。这一天,他的参谋金卓来找我,带来了他的一封信,向我大肆吹嘘他还有许多军队、枪炮,规复京津实非难事,唯尚无法善其后,须先统筹兼顾,接着又说他正在训练军队,月需饷银二百五十万充,他“伏乞睿哲俯赐,巽令使疆场小卒,知所依附”。担当联络的金卓,一再陈说张宗昌胜利在望,只等我的支援。这时陈宝琛、胡嗣瑗听说我又要花钱了,都来劝阻我,结果只写了一个鼓励性的手谕。不久,张宗昌完全垮台,到日本去了。他离我越远越有人在我们中间自动地来递信传话,张宗昌的信也越来越表现了他矢忠清室之志,但都有一个特点,就是向我要钱。带信人除了前面说过的金卓(后来在伪满给我当侍从武官)之外,还有后来当了伪满外交大臣的谢介石、德州知县王继兴、津浦路局长朱耀、陈宝琛的外甥刘骧业、安福系政客费毓楷和自称是张的秘书长的徐观囗等人。他们给我带来关于张宗昌的各种消息。我已不记得给他们拿去了多少钱,我现在找到了一部分当时的来信和去信的底稿,挑两件抄在下面:\\

\begin{quote}
	朕自闻滦河囗师,苦不得卿消息,听夕忧悬。昨据朕派遣在大连之前外务部右丞谢介石专人奏陈,悉卿安抵旅顺,并闻与前俄谢米诺夫将军订彼此互助之约,始终讨赤,志不销挫,闻之差慰。胜负兵家之常,此次再起,务须筹备完密,不可轻率进取。谢米诺夫怀抱忠义与卿相同,彼此提挈呼应,必奏敷功。方今苍生倒悬,待援孔亟,朕每念及,寝食难安,望卿为国珍重以副朕怀。今命谢介石到旅顺慰劳,并赏卿巨鉴一部,其留心阅览,追踪古人,朕有厚望焉。\\

皇上圣鉴:敬陈者,宗昌月前观光东京,得晤刘骧业,恭读手谕,感激莫名,业经复呈,计达天聪。宗昌自来别府,荏苒经年,对于祖国民生之憔悴,国事之蜩螗,夙夜焦灼,寝馈难安。一遵我皇上忧国爱民之至意,积极规划,罔敢稍疏。惟凡举大事,非财政充裕,不能放手办理,即不能贯彻主张,一木难支,众掌易举,当在圣明洞鉴之中。去秋订购枪械一批,价洋日金贰百壹拾万元,当交十分之五,不料金票陡涨,以中国银币折合约须叁百万元。目前军事方面筹划妥协,确有彻底办法,不动则已,动出万全。惟枪械一项,需款甚巨,四处张罗,缓不济急。筹思再四,惟有恳乞俯鉴愚忱,颁发款项壹百万元。万一力有不及,或先筹济叁伍拾万,以资应用,而利进行。感戴鸿慈,靡有涯既。兹派前德州知事王继兴,驰赴行官,代陈一切。人极稳妥,且系宗昌至戚。如蒙俞允,即由该知事具领携回,一俟款到,即行发动。此款回国后两月内即可归还。时机已迫,望若云霓,披沥上陈,无任屏营待命之至,伏乞睿鉴。恭请\\

圣安\\

\begin{flushright}
	张宗昌谨呈\\
\end{flushright}
\end{quote}

上面说的那笔钱,我没有给那位德州县知事。经陈宝琛、胡嗣瑗的劝止,我也没有再去信。但同时,我仍不能忘情于奉系,虽然这时张作霖已经死了。\\

张作霖之死\footnote{关于张被杀经过及原因,参与这一阴谋的日本战犯河本大作有过一段供述。据河本称,是他亲自指挥关东军参谋部人员,事先在京奉和南满铁路交接点皇姑屯车站布下了“必死之阵”:在交接点埋了三十麻袋黄色炸药,以设在五百公尺外瞭望台上的电气机控制爆炸;并在交接点以北装置了脱轨机、在附近埋伏了一排冲锋队。1928年6月4日5时半,张所乘之蓝色铁甲列车开到,东宫大尉一按电钮,张与列车同时被毁。事后关东军为掩盖真相,立调工兵赶修铁路,同时杀了两个中国人扔在肇事地点,口袋里塞上伪造的北伐军信件,并逮捕了十余名无辜居民,诬陷北伐军所为。杀张之原因,河本说:“一切亲日的军阀,我们统统抓住。能利用的时候就援助;不能利用的时候就设法消灭!”一语道破了帝国主义的毒辣。}尽人皆知是日本人谋杀的。我后来听说,日本人杀张,是由于张越来越不肯听话,张的不听话,是由于少帅的影响,要甩掉日本,另与美国结成新欢。因此日本人说他“忘思负义,不够朋友”。他的遇害虽然当时也把我吓了一跳,有的遗老还提醒我注意这个殷鉴,但是后来我没有理会那些遗老的话,因为我自认是与张作霖不同的人。张是个带兵的头目,这样的人除了他还可以另外找得到。而我是个皇帝,这是日本人从中国人里再找不出第二个来的。那时在我身边的人就有这样一个论点:“关东之人恨日本刺骨,日本禁关东与党军(指张学良与国民党)协和,力足取之,然日本即取关东不能自治,非得皇上正位则举措难施”。我深信日本是承认这一点的。“我欲借日本之力,必先得关东之心”,这是随之而来的策略,因此,我就从奉系里寻找张作霖的旧头目们,为我复辟使用。有个叫商衍瀛的遗老,是广东驻防旗人,从前做过翰林,当时是东北红“囗”字会的名人,这时出来给我活动奉系的将领。因为张学良已明白表示了要与蒋介石合作,所以商衍瀛进行的活动特别诡密。简要地说,这个最后的活动并没有结果,只留下了下面一点残迹:\\

\begin{quote}
	上谕\\

数日来肝火上升,每于夜间耳鸣头闷,甚感疲怠,是以未能见卿。卿此去奉,表面虽为地款,实则主要不在此耳,此不待言而明也。余备玉数种,分与相(张作相)、惠(张景惠)等人,到行带去。\\

再如降乩时,可否一问,余身体常不适,及此次肝热,久不能豫。\\

俟后为款事,自当随时与办事处来函。惟关于大局事,若有来函,务须格外缤密。\\

商衍瀛的奏折及我的批语臣商衍瀛跪奏  皇上圣躬久安,务求静养,时局变幻不出三个月内。  今日  皇上之艰难,安知非他日之福?望  圣躬勿过优劳,以待时机之复。奉  谕各节,臣当敬谨遵  谕办理。古玉敬谨分  赐。臣拟明日出关。再往吉林,哈尔滨,如蒙俞允,即当就道,臣恭请\\

圣安\\

\begin{flushright}
	宣统二十一年二月初九日\\
\end{flushright}
\end{quote}

此去甚是。惟须借何题目,免启学良之疑。卿孤忠奋发,极慰朕志。当此时局扰乱,甚易受嫌,卿当珍重勤密,以释朕怀。\\