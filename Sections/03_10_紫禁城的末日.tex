\fancyhead[LO]{{\scriptsize 1917-1924: 北京的“小朝廷” · 紫禁城的末日}} %奇數頁眉的左邊
\fancyhead[RO]{} %奇數頁眉的右邊
\fancyhead[LE]{} %偶數頁眉的左邊
\fancyhead[RE]{{\scriptsize 1917-1924: 北京的“小朝廷” · 紫禁城的末日}} %偶數頁眉的右邊
\chapter*{紫禁城的末日}
\addcontentsline{toc}{chapter}{\hspace{1cm}紫禁城的末日}
\thispagestyle{empty}
这次整顿内务府宣告失败,并不能使我就此“停车”。车没有停,不过拐个弯儿。我自从上了车,就不断有人给我加油打气,或者指点路标方向。\\

遗老们向我密陈恢复“大计”,前面说的只不过是其中的一例。在我婚后,像那样想为我效力的人,到处都有。例如康有为和他的徒弟徐勤、徐良两父子,打着“中华帝国宪政党”的招牌,在国内国外活动。他们的活动情况,继续地通过庄士敦传到宫中。徐勤写来奏折吹牛说,这个党在海外拥有十万党员和五家报纸。在我出宫前两年,徐良曾到广西找军阀林俊廷去活动复辟,他给庄士敦来信说,广西的三派军人首领陆荣廷、林俊廷和沈鸿英“三人皆与我党同宗旨,他日有事必可相助对待反对党也”\footnote{民十三年我出宫后,接收清宫的清室善后委员会在养心殿搜出了康有力和徐良给庄士敦的信共二封,连同金梁的条陈和江亢虎请觐见的信都发表了出来,但当时却没发表这一封,也没发表康有为向吴佩孚进行活动的往来信件。}。民国十三年春节后,康有为给庄士敦的信中说:“经年奔走,至除夕乃归,幸所至游说,皆能见听,亦由各方厌乱,人有同心。”据他说陕西、湖北、湖南、江苏。安徽、江西、贵州、云南全都说好了,或者到时一说就行。他最寄予希望的是吴佩孚,说“洛(指吴,吴当时在洛阳)忠于孟德(指曹锟),然闻已重病,如一有它,则传电可以旋转”。又说湖北萧耀南说过“一电可来”的话,到他生日,“可一赏之”。现在看起来,康有为信中说了不少梦话,后来更成了没有实效的招摇行径。但当时我和庄士敦对他的话不仅没有怀疑,而且大为欢欣鼓舞,并按他的指点送寿礼、赏福寿字。我在他们指点之下,开始懂得为自己的“理想”去动用财富了。\\


同样的例子还有“慈善捐款”。这是由哪位师傅的指点,不记得了,但动机是很清楚的,因为我这时懂得了社会舆论的价值。那时在北京报纸的社会版上,差不多天天都有“宣统帝施助善款待领”的消息。我的“施助”活动大致有两种,一种是根据报纸登载的贫民消息,把款送请报社代发,另一种是派人直接送到贫户家里。无论哪一种做法,过一两天报上总有这样的新闻:“本报前登某某求助一事,荷清帝遣人送去X元……”既表彰了我,又宣传了“本报”的作用。为了后者,几乎无报不登吸引我注意的贫民消息,我也乐得让各种报纸都给我做宣传。以至有的报居然登出这样的文章来:\\

\begin{quote}
	时事小言 皇恩浩荡\\

皇恩浩荡,乃君主时恭维皇帝的一句普通话,不意改建民国后,又闻有皇恩浩荡之声浪也。今岁入冬以来,京师贫民日众,凡经本报披露者,皆得有清帝之助款,贫民取款时,无不口诉皇恩之浩荡也。即本报代为介绍,同人帮同忙碌,然尽报纸之天职,一方替贫民之呼吁,一方代清帝之布恩,同人等亦无不忻忻然而云皇恩浩荡也。或日清帝退位深官,坐拥巨款,既无若何消耗,只好救济贫民,此不足为奇也。惟民国之政客军阀无不坐拥巨款,且并不见有一救济慈善者,于此更可见宣统帝之皇恩浩荡也\footnote{见民国十二年十二月十五日《平报》,作者:秋隐。}。
\end{quote}

像这样的文章,对我的价值自然比十块八块的助款大得太多了。\\

我付出最大的一笔赈款,是对民国十二年九月发生的日本“震灾”。那次日本地震的损失惊动了世界,我想让全世界知道“宣统帝”的“善心”,决定拿出一笔巨款助赈。我的陈师傅看的比我更远,他在称赞了“皇恩浩荡,天心仁慈”之后,告诉我说:“此举之影响,必不仅限于此。”后来因为现款困难,便送去了据估价在美金三十万元上下的古玩字画珍宝。日本芳泽公使陪同日本国会代表团来向我致谢时,宫中出现的兴奋气氛,竟和外国使节来观大婚礼时相像。\\

在这个时期,我的生活更加荒唐,干了不少自相矛盾的事。比如我一面责怪内务府开支太大,一面又挥霍无度。我从外国画报上看到洋狗的照片,就叫内务府向国外买来,连同狗食也要由国外定购。狗生了病请兽医,比给人治病用的钱还多。北京警察学校有位姓钱的兽医,大概看准了我的性格,极力巴结,给我写了好几个关于养狗知识的奏折,于是得到了绿玉手串、金戒指、鼻烟壶等十件珍品的赏赐。我有时从报上看见什么新鲜玩艺,如四岁孩子能读《孟子》,某人发现一只异样的蜘蛛,就会叫进宫里看看,当然也要赏钱。我一下子喜欢上了石头子儿,便有人买了各式各样的石头子儿送来,我都给以巨额赏赐。\\

我一面叫内务府裁人,把各司处从七百人戴到三百人,“御膳房”的二百厨师减到三十七个人,另方面又叫他们添设做西餐的“番菜膳房”,这两处“膳房”每月要开支一千三百多元菜钱。\\

关于我的每年开支数目,据我婚前一年(即民国十年)内务府给我编造的那个被缩小了数字的材料,不算我的吃穿用度,不算内务府各处司的开销,只算内务府的“交进”和“奉旨”支出的“恩赏”等款,共计年支八十七万零五百九十七两。\\

这种昏天黑地的生活,一直到民国十三年十一月五日,冯玉祥的国民军把我驱逐出紫禁城,才起了变化。\\

这年九月由朝阳之战开始的第二次直奉战争,吴佩孚的直军起初尚处于优势,十月间,吴部正向山海关的张作霖的奉军发动总攻之际,吴部的冯玉祥突然倒戈回师北京,发出和平通电。在冯、张合作之下,吴佩孚的山海关前线军队一败涂地,吴佩孚自己逃回洛阳。后来吴在河南没站住脚,又带着残兵败将逃到岳州,直到两年后和孙传芳联合,才又回来,不过这已是后话。吴军在山海关败绩消息还未到,占领北京的冯玉祥国民军已经把贿选总统曹锟软禁了起来,接着解散了“猪仔国会”,颜惠庆的内阁宣告辞职,国民军支持黄郛\footnote{黄郛字膺白,浙江人,反动的投机政客,后来北伐战争时帮助蒋介石策划反革命政变,成为国民党亲日派,也是新政学系首领之一。}组成了摄政内阁。\\

政变消息刚传到宫里来,我立刻觉出了情形不对。紫禁城的内城守卫队被国民军缴械,调出了北京城,国民军接替了他们的营地,神武门换上了国民军的岗哨。我在御花园里用望远镜观察景山,看见了那边上上下下都是和守卫队服装不同的士兵们。内务府派去了人,送去茶水吃食,国民军收下了,没有什么异样态度,但是紫禁城里的人谁也放不下心。我们都记得,张勋复辟那次,冯玉祥参加了“讨逆军”,如果不是段祺瑞及时地把他调出北京城,他是要一直打进紫禁城里来的。段祺瑞上台之后,冯玉祥和一些别的将领曾通电要求把小朝廷赶出紫禁城。凭着这点经验,我们对这次政变和守卫队的改编有了不祥的预感。接着,听说监狱里的政治犯都放出来了,又听说什么“过激党”都出来活动了,庄士敦和陈师傅他们给我的种种关于“过激”“恐怖”的教育——最主要的一条是说他们要杀掉每一个贵族——这时发生了作用。我把庄士敦找来,请他到东交民巷给我打听消息,要他设法给我安排避难的地方。\\

王公们陷入惶惶不安,有些人已在东交民巷的“六国饭店”定了房间,但是一听说我要出城,却都认为目前尚无必要。他们的根据还是那一条:有各国公认的优待条件在,是不会发生什么事情的。\\

然而必须发生的事,终归是发生了。\\

那天上午,大约是九点多钟,我正在储秀宫和婉客吃着水果聊天,内务府大臣们突然踉踉跄跄地跑了进来。为首的绍英手里拿着一件公文,气喘吁吁地说:\\

“皇上,皇上,……冯玉祥派了军队来了!还有李鸿藻的后人李石曾,说民国要废止优待条件,拿来这个叫,叫签字,……”\\

我一下子跳了起来,刚咬了一口的苹果滚到地上去了。我夺过他手里的公文,看见上面写着:\\

\begin{quote}
	大总统指令\\

派鹿钟麟、张璧交涉清室优待条件修正事宜,此令。\\

中华民国十三年十一月五日\\

国务院代行国务总理黄郛……\\

修正清室优待条件\\

今因大清皇帝欲贯彻五族共和之精神,不愿违反民国之各种制度仍存于今日,特将清室优待条件修正如左:\\

第一条、大清宣统帝即日起永远废除皇帝尊号,与中华民国国民在法律上享有同等一切之权利;\\

第二条、自本条件修正后,民国政府每年补助清室家用五十万元,并特支出二百万元开办北京贫民工厂,尽先收容旗籍贫民;\\

第三条、清室应按照原优待条件第三条,即日移出官禁,以后得自由选择住居,但民国政府仍负保护责任;\\

第四条、清室之宗庙陵寝永远奉祀,由民国酌设卫兵妥为保护;\\

第五条、清室私产归清室完全享有,民国政府当为特别保护,其一切公产应归民国政府所有。\\

\begin{flushright}
	中华民国十三年十一月五日
\end{flushright}
\end{quote}

老实说,这个新修正条件并没有我原先想象的那么可怕。但是绍英说了一句话,立即让我跳了起来:“他们说限三小时内全部搬出去!”\\

“那怎么办?我的财产呢?太妃呢?”我急得直转,“打电话找庄师傅!”\\

“电话线断,断,断了!”荣源回答说。\\

“去人找王爷来!我早说要出事的!偏不叫我出去!找王爷!找王爷!”\\

“出不去了,”宝熙说,“外面把上了人。不放人出去了!”\\

“给我交涉去!”\\

“嗻!”\\

这时端康太妃刚刚去世不多天,官里只剩下敬懿和荣惠两个太妃,这两位老太太说什么也不肯走。绍英拿这个作理由,去和鹿钟麟商量,结果允许延到下午三点。过了中午,经过交涉,父亲进了宫,朱、陈两师傅被放了进来,只有庄士敦被挡在外面。\\

听说王爷进来了,我马上走出屋子去迎他,看见他走进了宫门口,我立即叫道:\\

“王爷,这怎么办哪?”\\

他听见我的叫声,像挨了定身法似的,粘在那里了,既不走近前来,也不回答我的问题,嘴唇哆嗦了好半天,才进出一句没用的话:\\

“听,听旨意,听旨意……”\\

我又急又气,一扭身自己进了屋子。后来据太监告诉我,他听说我在修正条件上签了字,立刻把自己头上的花翎一把揪下来,连帽子一起摔在地上,嘴里嘟囔着说:“完了!完了!这个也甭要了!”\\

我回到屋里,过了不大功夫,绍英回来了,脸色比刚才更加难看,哆哆嗦嗦地说:“鹿钟麟催啦,说,说再限二十分钟,不然的话,不然的话……景山上就要开炮啦……”\\

其实鹿钟麟只带了二十名手枪队,可是他这句吓唬人的话非常生效。首先是我岳父荣源吓得跑到御花园,东钻西藏,找了个躲炮弹的地方,再也不肯出来。我看见王公大臣都吓成这副模样,只好赶快答应鹿的要求,决定先到我父亲的家里去。\\

这时国民军已给我准备好汽车,一共五辆,鹿钟麟坐头辆,我坐了第二辆,婉容和文绣、张璧、绍英等人依次上了后面的车。\\

车到北府门口,我下车的时候,鹿钟麟走了过来,这时我才和他见了面。鹿和我握了手,问我:\\

“溥仪先生,你今后是还打算做皇帝,还是要当个平民?”\\

“我愿意从今天起就当个平民。”\\

“好!”鹿钟麟笑了,说:“那么我就保护你。”又说,现在既是中华民国,同时又有个皇帝称号是不合理的,今后应该以公民的身分好好为国效力。张璧还说:\\

“既是个公民,就有了选举权和被选举权,将来也可能被选做大总统呢!”\\

一听大总统三个字,我心里特别不自在。这时我早已懂得“韬光养晦”的意义了,便说:\\

“我本来早就想不要那个优待条件,这回把它废止了,正合我的意思,所以我完全赞成你们的话。当皇帝并不自由,现在我可得到自由了。”\\

这段话说完,周围的国民军士兵都鼓起掌来。\\

我最后的一句话也并非完全是假话。我确实厌恶王公大臣们对我的限制和阻碍。我要“自由”,我要自由地按我自己的想法去实现我的理想——重新坐在我失掉的“宝座”上。\\