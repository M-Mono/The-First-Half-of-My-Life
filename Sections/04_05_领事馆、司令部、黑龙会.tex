\fancyhead[LO]{{\scriptsize 1924-1930: 天津的“行在” · 领事馆、司令部、黑龙会}} %奇數頁眉的左邊
\fancyhead[RO]{} %奇數頁眉的右邊
\fancyhead[LE]{} %偶數頁眉的左邊
\fancyhead[RE]{{\scriptsize 1924-1930: 天津的“行在” · 领事馆、司令部、黑龙会}} %偶數頁眉的右邊
\chapter*{领事馆、司令部、黑龙会}
\addcontentsline{toc}{chapter}{\hspace{1cm} 领事馆、司令部、黑龙会}
\thispagestyle{empty}
敬陈管见,条列于后:\\

\begin{quote}
	……对日本宜暗中联合而外称拒绝也。关东之人恨日本刺骨,日本禁关东与党军和协,而力足以取之。然日本即取关东不能自治,非得皇上正位则举措难施。今其势日渐紧张,关东因无以图存,日人亦无策善后,此田中之所以屡示善意也。\\

我皇上并无一成一旅,不用日本何以恢复?机难得而易失,天子不取,后悔莫追。故对日本只有联合之诚,万无拒绝之理。所难者我借日本之力而必先得关东之心。若令关东之人,疑我合日谋彼,则以后欲由东三省拥戴,势有所难。此意不妨与日本当机要人明言之,将来皇上复位,日本于三省取得之权,尚须让步方易办理。……\\
\end{quote}

这是一九二八年我收到的一份奏折中的一段。这段话代表了张园里多数人的想法,也是我经过多年的活动后,日益信服的结论。\\

前面已经说过,我自从进了北府,得到了日本人的“关怀”以来,就对日本人有了某些信赖。我在日本公使馆里住了些日子,到了天津之后,我一天比一天更相信,日本人是我将来复辟的第一个外援力量。\\

我到天津的第一年,日本总领事古田茂曾请我参观了一次日本侨民小学。在我往返的路上,日本小学生手持纸旗,夹道向我欢呼万岁。这个场面使我热泪满眶,感叹不已。当军阀内战的战火烧到了天津的边缘,租界上的各国驻军组织了联军,声言要对付敢于走近租界的国民军的时候,天津日本驻屯军司令官小泉六一中将特意来到张园,向我报告说:“请宣统帝放心,我们决不让中国兵进租界一步。”我听了,大为得意。\\

每逢新年或我的寿辰,日本的领事官和军队的将住们必定到我这里来祝贺。到了日本“天长节”,还要约我去参观阅兵典礼。记得有一次“天长节”阅兵,日本军司令官植田谦古邀请了日租界不少高级寓公,如曹汝霖、陆宗舆、靳云鹏等人都去了。我到场时,植田司令官特意骑马过来行致敬礼。当阅兵完毕,我们这些中国客人凑在一起,竟然随着日本人同声高呼“天皇万岁”。\\

日军司令部经常有一位住级参谋来给我讲说时事,多年来十分认真,有时还带来专门绘制的图表等物。第一个来讲的大概是名叫河边的参谋,他调走之后继续来讲的是金子定一,接金子的是后来在伪满当我的“御用挂”的吉冈安直。这个人在伪满与我相处十年,后面我要用专门的一节来谈他。\\

日军参谋讲说的时事,主要是内战形势,在讲解中经常出现这样的分析:“中国的混乱,根本在于群龙无首,没有了皇帝。”并由此谈到日本的天皇制的优越性,谈到中国的“民心”惟有“宣统帝”才能收拾。中国军队的腐败无力是不可或缺的话题,自然也要用日本皇军做对比。记得济南惨案发生后,吉冈安直至少用了一个小时来向我描述蒋介石军队的无能。日本布告的抄件,就是那次他给我拿来的。这些讲话加上历次检阅日军时获得的印象,使我深信日本军队的强大,深信日本军人对我的支持。\\

有一次我到白河边上去游逛,眺望停在河中心的日本兵舰。不知兵舰舰长怎么知道的,突然亲自来到岸上,虔敬地邀请我到他的舰上参观。到了舰上,日本海军将校列队向我致敬。这次由于仓猝间双方都没有准备翻译,我们用笔谈了一阵。这条兵舰舰名“藤”,船长姓蒲田。我回来之后,蒲田和一些军官向我回访,我应他的请求送了他一张签名照片,他表示这是他的极大的荣幸。从这件事情上,我觉得日本人是从心眼里对我尊敬的。我拉拢军阀、收买政客、任用客卿全不见效之后,日本人在我的心里的位置,就更加重要了。\\

起初,“日本人”三个字在我心里是一个整体。这当然不包括日本的老百姓,而是日本公使馆、天津日本总领事馆和天津日本“驻屯军”司令部里的日本人,以及和罗振玉、升允来往的那些非文非武的日本浪人。我把他们看成整体,是因为他们同样地“保护”我,把我当做一个“皇帝”来看待,同样地鄙夷民国,称颂大清,在我最初提出要出洋赴日的时候,他们都同样地表示愿意赞助。一九二七年,我由于害怕北伐军的逼近,一度接受罗振玉劝告,决定赴日。经过日本总领事的接洽,日本总领事馆向国内请示,田中内阁表示了欢迎,并决定按对待君主之礼来接待我。据罗振玉说,日本军部方面已准备用军队保护我启程。只是由于形势的缓和,也由于陈宝琛、郑孝胥的联合劝阻,未能成行。后来,南京的国民党政府成立了,官方的“打倒帝国主义”、“废除不平等条约”之类的口号消失了,我逐渐发现,尽管日本人的“尊敬”、“保护”仍然未变,但是在我出洋之类的问题上,他们的态度却有了分歧。这种分歧甚至达到了令我十分愤慨的程度。\\

一九二七年下半年,有一天罗振玉向我说:“虽然日租界比较安全,但究竟是鱼龙混杂。据日本司令部说,革命党(这是一直保留在张园里的对于国民党和共产党的笼统称呼)的便衣(这是对于秘密工作者的称呼,而且按他们解释,都是带有武器的)混进来了不少,圣驾的安全,颇为可虑。依臣所见,仍以暂行东幸为宜,不妨先到旅顺。恭亲王在那边有了妥善筹备,日本军方也愿协助,担当护驾之责。”这时我正被“革命党便衣”的谣言弄得惶惶不安,听了罗振玉的话,特别是溥伟又写来了信,我于是再一次下了出行的决心。我不顾陈宝琛和郑孝胥的反对,立刻命令郑孝胥去给我找日本总领事,我要亲自和他见面谈谈。\\

郑孝胥听了我的吩咐,怔了一下,问道:“皇上请加藤,由谁做翻译呢?是谢介石吗?”\\

我明白了他的意思。谢介石是个台湾人,由于升允的引见,在北京时就出入宫中,张勋复辟时做了十二天的外务部官员,后来由日本人的推荐,在李景林部下当秘书官,这时跟罗振玉混在一起,不断地给我送来什么“便衣队行将举事”,革命党将对我进行暗杀等等情报。劝说我去旅顺避难的,也有他一份。郑孝胥显然不喜欢罗振玉身边的人给我当翻译,而同时,我知道在这个重要问题上,罗振玉也不会喜欢郑孝胥的儿子郑垂或者陈宝琛的外甥刘嚷业当翻译。我想了一下,便决定道:“我用英文翻译。加藤会英文。”\\

总领事加藤和副领事冈本一策、白井康都来了。听完我的话,加藤的回答是:\\

“陛下提出的问题,我还不能立即答复,这个问题还须请示东京。”\\

我心里想:这本是日本司令部对罗振玉说没有问题的事,再说我又不是到日本去,何必去请示东京?天津的高级寓公也有到旅顺去避暑的,他们连日本总领事馆也不用通知就去了,对我为什么要多这一层麻烦?我心里的话没完全说出来,加藤却又提出了一个多余的问题:\\

“请问,这是陛下自己的意思吗?”\\

“是我自己的。”我不痛快地回答。我又说,现在有许多对我不利的消息,我在这里不能安心。据日本司令部说,现在革命党派来不少便衣,总领事馆一定有这个情报吧?\\

“那是谣言,陛下不必相信它。”加藤说的时候,满脸的不高兴。他把司令部的情报说成谣言,使我感到很奇怪。我曾根据那情报请他增派警卫,警卫派来了,他究竟相信不相信那情报?我实在忍不住地说:\\

“司令部方面的情报,怎么会是谣言?”\\

加藤听了这话,半天没吭气。那两位副领事,不知道他们懂不懂英文,在沙发上像坐不稳似地蠕动了一阵。\\

“陛下可以确信,安全是不会有问题的。”加藤最后说,“当然,到旅顺的问题,我将遵命去请示敝国政府。”\\

这次谈话,使我第一次觉出了日本总领事馆和司令部方面之间的不协调,我感觉到奇怪,也感觉到很气人。我把罗振玉。谢介石叫了来,又问了一遍。他们肯定说,司令部方面和接近司令部方面的日本人,都是这样说的。并且说:\\

“司令部的情报是极其可靠的。关于革命党的一举一动,向来都是清清楚楚的。不管怎么说,即使暗杀是一句谣言,也要防备。”\\

过了不多几天,我岳父荣源向我报告说,外边的朋友告诉他,从英法租界里来了冯玉祥的便衣刺客,情况非常可虑。我的“随侍”祁继忠又报告说,他发现大门附近,有些形迹可疑的人,伸头向园子里张望。我听了这些消息,忙把管庶务的佟济煦和管护军的索玉山叫来,叫他们告知日警,加紧门禁,嘱咐护军留神门外闲人,并禁止晚间出入。第二天,我听一个随侍说,昨晚上还有人外出,没有遵守我的禁令,我立刻下令给佟济煦记大过一次,并罚扣违令外出者的饷银\footnote{这时张园管柬“底下人”的办法,根据师傅们的谏劝和佟济煦的恳求,已经取消了鞭笞,改为轻者罚跪,重者罚扣饷银。为了管束,我还亲自订了一套“规则”,内容见第六章。},以示警戒。总之,我的神经紧张起来了。\\

有一天夜里,我在睡梦中忽然被一声枪响惊醒,接着,又是一枪,声音是从后窗外面传来的。我一下从床上跳起,叫人去召集护军,我认为一定是冯玉祥的便衣来了。张园里的人全起来了,护军们被布置到各处,大门上站岗的日本巡捕(华人)加强了戒备,驻园的日本警察到园外进行了搜索。结果,抓到了放枪的人。出乎我的意料,这个放枪的却是个日本人。\\

第二天,佟济煦告诉我,这个日本人名叫岩田,是黑龙会分子,日本警察把他带到警察署,日本司令部马上把他要去了。我听了这话,事情明白了七八分。\\

我对黑龙会的人物,曾有过接触。一九二五年冬季,我接见过黑龙会的重要人物佃信夫。事情的缘起,也是由于罗振玉的鼓吹。罗振玉对我说,日本朝野对于我这次被迫出宫和避难,都非常同情,日本许多权势人物,连军部在内,都在筹划赞助我复辟,现在派来了他们的代表佃信夫,要亲自和我谈一谈。他说这个机会决不可失,应当立刻召见这位人物。佃信夫是个什么人,我原先并非毫无所闻,内务府里有人认识他,说他在辛亥之后,常常在各王府跑出跑进,和宗室王公颇有些交情。罗振玉的消息打动了我,不过我觉得日本总领事是日本正式的代表,又是我的保护人,理应找他来一同谈谈,于是叫人通知了有田八郎总领事,请他届时出席。谁知那位佃信夫来时一看到有田在座,立刻返身便走,弄得在座的陈宝琛、郑孝胥等人都十分惊愕。后来郑孝胥去责问他何以敢如此在“圣前非礼”,他的回答是:“把有田请来,这不是成心跟我过不去吗?既然如此,改日再谈。”现在看来,罗振玉这次的活动以及岩田的鸣枪制造恐怖气氛,就是那次伯信夫的活动的继续。这种活动,显然有日军司令部做后台。\\

后来我把陈宝琛、郑孝胥找来,要听听他们对这件事的看法。郑孝胥说:“看起来,日本军、政两界,都想请皇上住在自己的势力范围之内加以保护。他们虽然不合作,却也于我无损。不过罗振玉做事未免荒唐,他这样做法,有败无成,万不可过于重用。”陈宝琛说:“不管日军司令部也罢,黑龙会也罢,做事全不负责任。除了日本公使和总领事,谁的话也别听!”我考虑了一下,觉得他们的话很有道理,便不想再向总领事要求离津了。从此,我对罗振玉也不再感兴趣了。第二年,他便卖掉了天津的房子,跑到了大连。\\

说也奇怪,罗振玉一走,谣言也少了,连荣源和祁继忠也没有惊人的情报了。事隔很久以后,我才明白一点其中的奥妙。\\

这是我的英文翻译告诉我的。他和荣源是连襟,由于这种关系,也由于他和日军司令部翻译有事务上的交往,探听到一点内幕情况,后来透露了给我。原来,日军司令部专门设了一个特务机关,长期做张园的工作,和这个机关有关系的,至少有罗振玉、谢介石、荣源这几个人。我的英文翻译曾由这三个人带到这个特务机关的一处秘密地方,这地方对外的名称,叫做“三野公馆”。\\

他是在那天我接见了加藤之后被他们带去的。他的翻译工作做完之后,被罗、谢、荣三人截住,打听会谈情况。罗振玉等人听说加藤对我出行毫不热心,立刻鼓噪起来。从他们的议论中,英文翻译听出了司令部方面有人对罗振玉他们表示的态度完全不同,是说好了要把我送到旅顺去住的。为了向司令部方面的人汇报加藤的谈话,罗振玉等三人把英文翻译带到“三野公馆”去找那人,结果没找见,而英文翻译却发现了这个秘密地方。以后他从荣源和别的方面探听出,这是个有鸦片烟、女人、金钱的地方。荣源是这里的常客,有一次他甚至侮辱过一个被叫做大熊的日本人的妻子,大熊把他告到司令部,也没有能动他。至于荣源等人和三野公馆有些什么具体活动,荣源却不肯透露。\\

三野的全名是三野友吉,我认识这个人,他是司令部的一名少住,常随日军司令官来张园做客。当时我绝没想到,正是这个人,通过他的“公馆”,与张园的某些人建立了极亲密的来往,把张园里的情形摸得透熟,把张园里的荣源之流哄得非常听话,以至后来能通过他们,把谣言送到我耳朵里,弄得我几次想往旅顺跑。我听到我的翻译透露出来三野公馆的一些情况后,只想到日军司令部如此下功夫拉拢荣源等人,不过是为了和领事馆争夺我,他们两家的争夺,正如郑孝胥所说,是于我有益无损的事。\\

事实上,我能看到的现象也是如此:司令部与领事馆的勾心斗角,其激烈与错综复杂,是不下于我身边的遗老们中间所发生的。比如司令部派了参谋每周给我讲说时事,领事馆就介绍了远山猛雄做皇室教师;领事馆每次邀请我必同时请郑孝胥,司令部的邀请中就少不了罗振玉;领事馆在张园派驻了日本警官,而司令部就有专设的三野公馆,为荣源、罗振玉、谢介石等人预备了女人、鸦片,等等。\\

至于黑龙会,我了解得最晚,还是郑孝胥告诉我的。这个日本最大的浪人团体,前身名为“玄洋社”,成立于中法战争之后,由日本浪人平冈浩太郎所创立,是在中国进行间谍活动的最早的特务组织,最初在福州、芝罘(烟台)、上海都有机关,以领事馆、学校、照相馆等为掩护,如上海的“东洋学校”和后来的“同文书院”都是。“黑龙会”这个名字的意思是“超越黑龙江”,出现于一九零一年。在日俄战争中,这个团体起了很大作用,传说在那时黑龙会会员已达几十万名,拥有巨大的活动资金。头山满是黑龙会最出名的领袖,在他的指挥下,他的党羽深入到中国的各阶层,从清末的王公大臣如升允之流的身边,到贩夫走卒如张园的随侍中间,无一处没有他们在进行着深谋远虑的工作。日本许多著名的人物,如土肥原、广田、平沼、有田、香月等人都是头山满的门生。据郑孝胥说,头山满是个佛教徒,有一把银色长须,面容“慈祥”,平生最爱玫瑰花,终年不愿离开他的花园。就是这样的一个佛教徒,在玫瑰花香气的氲氤中,持着银须,面容“慈祥”地设计出骇人的阴谋和惨绝人寰的凶案。\\

郑孝胥后来能认识到黑龙会和日本军部系统的力量,是应该把它归功于罗振玉的。郑、罗、陈三人代表了三种不同的思想。罗振玉认为军部人物以及黑龙会人物的话全是可靠的(他对谢米诺夫和多布端的信任,也一半是出于谢、多二人和黑龙会的关系),陈宝琛则认为除了代表日本政府的总领事馆以外,别的日本人的话全不可信。郑孝胥公开附和着陈宝琛,以反对罗振玉。他心里起初也对司令部和黑龙会存着怀疑,但他逐渐地透过罗振玉的吹嘘和黑龙会的胡作非为,看出了东京方面某种势力的动向,看出了日本当局的实在意图,最后终于看出了这是他可以仗恃的力量。因此,他后来决定暂时放下追求各国共管的计划,而束装东行,专门到日本去找黑龙会和日本参谋总部。\\