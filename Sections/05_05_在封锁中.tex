\fancyhead[LO]{{\scriptsize 1931-1932: 到东北去 · 在封锁中}} %奇數頁眉的左邊
\fancyhead[RO]{} %奇數頁眉的右邊
\fancyhead[LE]{} %偶數頁眉的左邊
\fancyhead[RE]{{\scriptsize 1931-1932: 到东北去 · 在封锁中}} %偶數頁眉的右邊
\chapter*{在封锁中}
\addcontentsline{toc}{chapter}{\hspace{1cm} 在封锁中}
\thispagestyle{empty}
在淡路丸上,郑孝胥讲了一整天治国平天下的抱负。十三日早晨,我们到达了辽宁省营口市的“满铁”码头。\\

为什么去沈阳要从营口登陆,这个问题我根本不曾考虑过,我想到的只是东北民众将如何在营口码头上来接我。在我的想象中,那里必定有一场民众欢呼的场面,就像我在天津日租界日侨小学里看到的那样,人们摇着小旗,向我高呼万岁。但是船身越靠近码头,越不像那么回事。那里并没有人群,更没有什么旗帜。等到上了岸,这才明白,不但迎接的人很少,而且全是日本人。\\

经过上角利一的介绍,才知道这都是板垣派来的人,为首的叫甘粕正彦。此人在中国知道他的不多,在日本却大有名气。他原是个宪兵大尉。日本大地震时,日本军部趁着震灾造成的混乱,迫害进步人士,遭难的大杉荣夫妇和七岁的孩子就是死在他手里的。震灾后,这个惨案被人揭发出来,在社会舆论压力之下,军部不得不让他充当替罪羊,交付军事法庭会审,处以无期徒刑。过了不久,他获得了假释,被送往法国去念书。他在法国学的是美术和音乐,几年之后,这位艺术家回到日本,随即被派到关东军特务机关。据二次大战后日本出版的一本书上说,作为“九·一八”事变信号的柳条沟铁道的爆炸,就是他的一件杰作。在营口码头上,我怎么也不会想到,这个彬彬有礼的戴细腿近视眼镜的人,会有这么不平凡的经历。如果没有他的杰作,也许我还不会到东北来哩。\\

甘粕正彦把我和郑氏父子让进预备好的马车,把我们载到火车站。坐了大约一个多钟头的火车,又换上了马车。一路上没听到任何解释,稀里胡涂地到了汤岗子温泉疗养区。我怀着狐疑的心情走进了对翠阁温泉旅馆。\\

对翠阁旅馆是日本“满铁”的企业,日本风格的欧式洋楼,设备相当华丽,只有日本军官、满铁高级人员和中国的官僚有资格住。我被带进了楼上的非常讲究的客房,在这里见着了罗振玉、商衍瀛和佟济煦。罗振玉给我请安后即刻告诉我,他正在和关东军商洽复辟建国的事,又说在商谈结束前,不宜把我到达这里的消息泄露出去,而且除了他之外别人也不宜出头露面。他这话的真正用意我没有领会,我却自以为弄清了一个疑团:怪不得没有热烈欢迎的场面,原来人们还都不知我来。我相信和关东军的谈判是容易的,不久就可以宣布我这大清皇帝在沈阳故宫里复位的消息,那时就不会是这样冷冷清清的了。我想得很高兴,全然没有注意到郑氏父子的异样神色。我痛痛快快地吃了一餐别有风味的日本饭菜,在窗口眺望了一会这个风景区的夜色,就心旷神怡地睡觉去了。\\

过了一宿,我才明白这次又乐得太早了。\\

漱洗之后,我招呼随侍祁继忠,说我要出去蹓跶一下,看看左近的风景。\\

“不行呵,不让出去啦!”祁继忠愁眉苦脸地说。\\

“怎么不行?”我诧异地问。“谁说的?到楼下去问问!”\\

“连楼也不让下呵!”\\

我这时才知道,对翠阁旅馆已经被封锁起来,不但外面的人不准进到旅馆范围里来,就是住在楼下的人也休想上楼(楼上只有我们这几个人住)。尤其令人不解的是,为什么连楼上的人也不许下去呢?找罗振玉,罗振玉已不知何往。郑孝胥父子都很生气,请我找日本人问问这是怎么回事。陪我们住在这里的日本人,带头的是上角利一和甘粕正彦。祁继忠把上角找来了,他笑嘻嘻地用日本腔的中国话说:\\

“这是为了安全的,为了宣统帝安全的。”\\

“我们在这里住到什么时候?”郑孝胥问。\\

“这要听板垣大佐的。”\\

“熙洽他们呢?不是罗振玉说熙洽要接我到奉天吗?”\\

“这,也要听板垣大佐的。”\\

“罗振玉呢?”郑垂问。\\

“到沈阳找板垣大住去了。现在还在讨论着新国家的问题,讨论出一致的意见,就来请宣统帝去的。”\\

“糟!”郑垂一甩手,忿忿地走到一边去了。这个“君前失礼”的举动很使我看不惯,不过这时更引起我注意的,却是上角说的“新国家”问题还在讨论。这可太奇怪了,不是土肥原和熙洽都说一切没问题,就等我来主持大计了吗?上角现在说“还在讨论”,这是什么意思呢?我提出了这个问题,上角利一含糊其词地回答说:\\

“这样的大事,哪能说办就办的?宣统帝不要着急,到时候自然要请宣统帝去的。”\\

“到哪里去呢?”郑垂匆匆地走过来插嘴,“到奉天吗?”\\

“这要听板垣大佐的。”\\

我很生气地躲开了他们,到另一间屋子叫来了佟济煦,问他从沈阳拍来电报说“万事俱妥”是什么意思。佟济煦说这是袁金铠说的,不知这是怎么闹的。我又问商衍瀛,他对这件事怎么看,他也没说出个什么道理来,只抱怨这地方没有“乩坛”,否则的话,他一定可以得到神仙的解答。\\

这时我还不知道,日本人正在忙乱中。日本在国际上处势孤立,内部对于采取什么形式统治这块殖民地,意见还不统一,关东军自然还不便于立刻让我出场。我只感觉出日本人对我不像在天津那么尊敬了,这个上角也不是在天津驻屯军司令部里的那个上角了。我在不安的预感中,等待了一个星期,忽然接到了板垣的电话,请我搬到旅顺去。\\

为什么不去沈阳呢?上角利一笑嘻嘻地解释说,这还要等和板垣大佐谈过才能定。为什么要到旅顺等呢?据上角说,因为汤岗子这地方附近有“匪”,很不安全,不如住旅顺好,旅顺是个大地方,一切很方便。我听着有理,于是这天晚上搭上火车,第二天一早到了旅顺。\\

在旅顺住的是大和旅馆。又是在对翠阁的一套做法,楼上全部归我们这几个人占用,告诉我不要下楼,楼下的人也不准上来。上角和甘粕对我说的还是那几句:新国家问题还在讨论,不要着急,到时候就有人请我到沈阳去。在这里住了不多天,郑孝胥父子便获得了罗振玉一样的待遇,不但外出不受阻拦,而且还可以到大连去。这时郑孝胥脸上的郁郁不乐的神色没有了,说话的调子也和罗振玉一样了,说什么“皇上天威,不宜出头露面,一切宜由臣子们去办,待为臣子的办好,到时候皇上自然就会顺理成章地面南受贺”。又说在事成之前,不宜宣扬,因此也不要接见一切人员,关东军目前是这里的主人,我在“登极”之前,在这里暂时还算是客人,客随主便,也是理所当然。听了他们的话,我虽然心里着急,也只好捺下心等着。\\

事实上,这些口口声声叫我皇上的,这些绞着脑汁、不辞劳苦、为我奔波着的,他们心里的我,不过是纸牌上的皇帝,这种皇帝的作用不过是可以吃掉别人的牌,以赢得一笔赌注而已。日本人为了应付西方的磨擦和国内外的舆论压力,才准备下我这张牌,自然他们在需要打出去之前,要严密加以保藏。郑罗之流为了应付别的竞争者,独得日本人的犒赏,也都想独占我这张牌,都费尽心机把持我。于是就形成了对我的封锁,使我处于被隔离的状态中。在汤岗子,罗振玉想利用日本人规定的限制来断绝我和别人的来往,曾阻止我和郑孝胥与日本关东军的接触,以保障他的独家包办。到了旅顺,郑孝胥和日本人方面发生了关系,跟他唱上了对台戏,于是他只好亡羊补牢,设法再不要有第三个人插进来。在防范我这方面,罗和郑联合起来,这就出现了郑罗二人一方面联合垄断我,一方面又勾心斗角地在日本人方面争宠。\\

这些事实的内幕,我当时自然不明白。我只觉出了罗振玉和郑孝胥父子跟日本人沆瀣一气,要把我和别人隔离开。他们对于佟济煦和只知道算卦求神的商衍瀛,不怎么注意,对于从天津来的要见我的人,却防范得很厉害,甚至连对婉容都不客气。\\

我在离开静园以前,留下了一道手谕,叫一名随侍交给胡嗣瑗,命他随后来找我,命陈曾寿送婉容来。这三个人听说我在旅顺,就来到了大连。罗振玉派人去给他们找了地方住下,说关东军有命令,不许他们到旅顺来。婉容对这个命令起了疑心,以为我出了什么岔子,便大哭大闹,非来不可,这样才得到允许来旅顺看了我一次。过了大概一个月,关东军把我迁到善耆(这时已死)的儿子宪章家里去住,这才让婉容和后来赶到的二妹、三妹搬到我住的地方来。\\

我本来还想让胡嗣瑗、陈曾寿两人也搬到我身边,但郑孝胥说关东军规定,除了他父子加上罗振玉和万绳栻这几个人之外,任何人都不许见我。我请求他去和甘粕、上角商量,结果只准许胡嗣瑗见一面,条件是当天必须回大连。胡嗣瑗在这种情形下,一看见我就咧开大嘴哭起来了,说他真想不到在我身旁多年,今日落得连见一面都受人限制,说得我心里很不自在。一种孤立无援的恐惧在压迫着我,我只能安慰胡嗣瑗几句,告诉他等我到了可以说话的时候,一定“传谕”叫他和陈曾寿到我身边来。胡嗣瑗听了我的话,止住了哭泣,趁着室里没人,一五一十地向我叙说了郑罗二人对他们的多方刁难,攻击他们是“架空欺罔,挟上压下、排挤忠良”。\\

胡嗣瑗和陈曾寿住在大连,一有机会就托人带奏折和条陈来,在痛骂郑罗“虽秦桧、仇士良之所为,尚不敢公然无状、欺侮挟持一至于此”之外,总要酸劲十足和焦急万分地一再说些“当兹皇上广选才俊,登用贤良之时,如此掣肘,尚有何希望乎?”这类的话。胡嗣瑗曾劝我向日本人要求恢复天津的形势,身边应有亲信二三人,意思是他仍要当个代拆代行的大军机。陈曾寿则对我大谈“建国之道,内治莫先于纲纪,外交莫重于主权”,所谓“纲纪最要者,魁柄必操自上,主权最要者,政令必出自上”,总之一句话,我必须有权能用人,因为这样他才能做大官。这些人自然斗不过郑罗,在后来封官晋爵的时候,显贵角色里根本没有他们。后来经我要求,给了陈曾寿一个秘书职务,但他不干,请假走了,直到以后设立了内廷局叫他当局长,他才回来。胡嗣瑗曾和陈曾寿表示决不做官,“愿以白衣追随左右”,我给他弄上个秘书长的位置,他才不再提什么“白衣”。由于他恨极了当国务总理的郑孝胥,后来便和罗振玉联合起来攻郑。结果没有攻倒,自己反倒连秘书长也没有做成,这是后话,暂且不提。\\

我到旅顺的两个月后,陈宝琛也来了。郑孝胥这时成了关东军的红人,罗振玉眼看就要败在他手里,正当他接近全胜,他和关东军的交易接近成熟的时候,看见威望超过他的“帝师”出现在大连,立刻引起了他的警惕。他生怕他这位同乡会引起日本人更大的兴趣,急忙想撵陈回去。所以陈宝琛在旅顺一共住了两宿,只和我见了两面,就被郑孝胥借口日本人要在旅馆开会给送走了。\\

同时,天津和北京的一些想做官的遗老们借口服侍我,跑到旅顺来,也都被郑孝胥和甘粕正彦挡了驾。就连恭亲王溥伟想见我也遇到拦阻。我过生日的时候,他们再找不到借口,才无可奈何地让一部分人见了我,给我祝寿。其中有宝熙、商衍瀛、沈继贤、金卓、王季烈、陈曾寿、毓善等人,后来在伪满成立时都成了大小新贵。\\

当时互相倾轧、你争我夺的不但有遗老,在日本浪人和特务之间也不例外,得势的当然是板垣手下的上角和甘粕这一伙。当过我父亲家里家庭教师的远山猛雄本想到我身边沾沾光,由于不是军部系统的,最后都给上角和甘粕挤走了。\\

发生在郑与罗之间的斗争是最激烈的。这是这对冤家最后的殊死战,因此都使用出了全身的力气。罗振玉利用他和板垣。上角利一这些人的势力,对郑孝胥一到东北即行封锁,是他的头一“招”。他自恃有首倡“迎立”之功,相信只要能把我垄断在手,用我这张牌去和日本人谈判,一定可以达到位居首辅的目的。可是他在谈判中,一上来就坚持要大清复辟。日本方面对他这个意见不感兴趣。他跟我一样地不明白,复辟的做法和日本人宣传的“满洲民众要求独立自治”的说法,是配不上套的。这时日本人在国际上十分孤立,还不能把这场傀儡戏立刻搬上台去,因此关东军并不急于定案,暂时仍用什么“自治指导部”、“维持会”等名目支撑着。罗振玉认为郑孝胥被他封锁住,其他人更无法靠近我的身边,无从代表我和日本人去说话,他大可用独家经理的身分,不慌不忙地和日本人办交涉。复辟大清和另立国家之争在悬而未决,我和郑孝胥到了旅顺,出乎罗振玉的意外,他对郑孝胥的封锁失了效,关东军方面请郑孝胥去会谈。罗振玉既不知道郑孝胥和东京军部的关系,也想不到郑孝胥在离津之前就认识了上角利一。就像我出宫那年,罗振玉与日本竹本大住的关系变成了郑孝胥的关系一样,这回罗振玉带来的上角也很快变成了郑孝胥的朋友,成了郑与关东军之间的桥梁。郑氏父子到了营口、旅顺,和甘粕正彦谈了几次心,关东军因此了解到他父子远比罗振玉“灵活”,不像罗振玉那样非有蟒袍补褂、三跪九叩不过瘾,因此乐于以他为交易对手。郑孝胥被看中了之后,第一次和板垣会面(一九三二年一月二十八日在旅顺),听到板垣要叫我当“满蒙共和国大总统”,先很惊讶,后来明白了日本军方决不肯给我一顶皇帝帽子,便马上改了主意,由他儿子郑垂出面找军方选中的殖民地总管驹井德三,表示日本如果认为“帝国”称呼不适于这个新国家的话,只要同意他任未来的内阁首揆,一切没有问题,他可负责说服“宣统帝”接受其他的元首称号。顺便说一句,这时抢这个首揆椅子的,却大有人在。不但有罗振玉,还有张景惠、臧式毅、熙洽等人。熙洽几次派人送钱给我,共有十几万元,求我授他“总理”之职。郑孝胥自然很着急,所以忙不迭地叫郑垂从旁抢先递“价码”。驹井德三把这袖筒里来的价码告诉了本庄和板垣,于是郑孝胥便成了奉天关东军司令官的客人。就这样,关东军的第一交易对手由罗振玉变成了郑孝胥。\\

自然,这些真相是我在封锁中所看不透的。我所见到的是另外一样……\\