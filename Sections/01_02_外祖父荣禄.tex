\fancyhead[LO]{{\scriptsize 1859-1908: 我的家世 · 外祖父荣禄}} %奇數頁眉的左邊
\fancyhead[RO]{} %奇數頁眉的右邊
\fancyhead[LE]{} %偶數頁眉的左邊
\fancyhead[RE]{{\scriptsize 1859-1908: 我的家世 · 外祖父荣禄}} %偶數頁眉的右邊
\chapter*{外祖父荣禄}
\addcontentsline{toc}{chapter}{\hspace{1cm}外祖父荣禄}
\thispagestyle{empty}
醇贤亲王有四位“福晋”\footnote{即是满语妻子的意思,也含有贵妇的意义(一说即汉语“夫人”的音译),清朝制度对亲王、都王世子之妻室均要加封,正室封为“福晋”,侧室封为“侧福晋”。},生了七子三女。他去世时,遗下三子一女,最长的是第五子,即我的父亲载沣,那年八岁,承袭了王爵。我的两个叔父,五岁的载洵和三岁的载涛,同时晋封为公爵。我家从此又开始蒙受着新的“恩光福禄”。然而,醇王府这最后十几年的恩光福禄,比过去的几十年掺和着更多的中国人民的苦难与耻辱,也同样的和慈禧这个名字不能分开。\\

  一件大事是慈禧给我父亲母亲指婚。这次的“恩光”也可以说是戊戌政变和庚子事件的一件产物。首先,这是对于戊戌政变中给她立下大功的忠臣荣禄的恩典。我外祖父荣禄是瓜尔佳氏满洲正白旗人,咸丰年间做过户部银库员外郎,因为贪污几乎被肃顺杀了头。不知他用什么方法摆脱了这次厄运,又花钱买得候补道员的衔。这种做法就是清末广泛推行的“捐班”,是与“科举”同样合法的出身。同治初年,我祖父建立神机营(使用火器的皇家军队),荣禄被派去当差,做过翼长和总兵,经过一番累迁,由大学士文祥推荐授工部侍郎,以后又做过总管内务府大臣,光绪初年,升到工部尚书。后来因为被告发贪污受贿,革职降级调出北京。甲午战争这年,恭亲王出办军务,荣禄借进京为慈禧太后祝寿的机会,钻营到恭亲王身边,得到了恭亲王的信赖。甲午战后他推荐袁世凯练新军时,已经当上了兵部尚书。他这时已远比从前老练,善于看准关节,特别肯在总管太监李莲英跟前花银子,因此渐渐改变了慈禧太后对他的印象。他回到北京的第二年,得到了一件复查慈禧陵寝工程雨损的差使。这个工程先经一个大臣检查过,报称修缮费需银三十万。据说这位大臣因为工程原由醇亲王奕囗生前监工督办,不便低估原工程的质量,所以损毁情形也报得不太严重。但荣禄另是一个做法。他摸准了太后的心理,把损毁程度夸张了一番,修缮费报了一百五十万两。结果太后把那位大臣骂了一通,对已死的醇亲王的忠心也发生了疑问,而对荣禄却有了进一步的赏识。\\

  荣禄有了李莲英这个好朋友,加上他的妻子很会讨好太后,常被召进宫去陪伴太后聊天,所以他对慈禧的心理越摸越熟。他知道慈禧光绪母子不和的内情,也深知这场不和对自己前途的关系,当然他更愿意在这场内江中给慈禧出主意。在光绪皇帝发出变法维新的各种上谕时,那些被罢黜和担心被挤掉位置的人只知哭哭啼啼,而他早已给慈禧安排好计策。当时有人把皇帝太后身边这两派势力称为帝党和后党。荣禄是当权派后党的头脑,翁同和是没有实权的帝党的头脑。维新派之能够和皇帝接触上,是由于翁同和对康有为的推荐,慈禧按照事先安排好的计策,先强逼着光绪叫他的老师翁同和退休回了家。据说,翁同和行前荣禄还握着他的手挥泪问他:“您怎么把皇帝给得罪了?”翁同和离开北京不多天,荣禄就走马上任,做了文渊阁大学士兼直隶总督和北洋大臣,位居首辅,统辖近畿三军。荣禄得到了这个职位后,本想接着用六部九卿联名上疏的办法,废掉光绪,由太后恢复听政,但因甲午战败之后,当权派受到各方指责,有人很怕这一举动会引起民愤,不敢附议,只得作罢。但是荣禄的愿望终于在戊戌政变时乘机达到了。这件事的经过,据说是这样:先是荣禄定计要在太后和光绪在天津检阅新军时实行政变。光绪知道了这个消息,秘密通知维新派设法营救。维新派人士把希望寄托在统辖新军的直隶按察使袁世凯身上,结果反而断送了光绪。在举国以谈维新为时髦的时候,袁世凯曾参加过维新人士的团体“强学会”,翁同和革职返乡路过天津时,袁世凯还向他表示过同情,并且申述了对皇帝的无限忠诚。因此,维新派对他抱有很大幻想,建议光绪加以笼络。光绪召见了他,破格升他为兵部侍郎,专司练兵事务,然后维新派谭嗣同\footnote{谭嗣同(1865—1893),字复生,号壮飞,湖南浏阳人,是清末维新运动的思想家之一,忿中日战争失败,在测阳创“算学社”著“仁学”,后又组织“南学会”办“湘报”,成为维新运动的领袖之一。他被袁世凯出卖后遇害,一同遇害的还有维新派的林旭、杨锐、刘光第、杨深秀、康广仁等,旧史称为六君子。}又私下到他的寓所,说出了维新派的计划:在慈禧和光绪阅兵时,实行兵谏,诛杀荣禄,软禁慈禧,拥戴光绪。袁世凯听了,慷慨激昂,一口承担,说:“杀荣禄像杀一条狗似的那么容易!”谭嗣同有意试探地说:“你要不干也行,向西太后那边告发了,也有荣华富贵。”他立刻瞪了眼:“瞧你把我袁世凯看成了什么人!”可是他送走了谭嗣同,当天就奔回天津,向他的上司荣禄作了全盘报告。荣禄得讯,连忙乘火车北上,在丰台下车直奔颐和园,告诉了慈禧。结果,光绪被幽禁,谭嗣同等六位维新派人士被杀,康有为逃到日本,百日维新昙花一现,而我的外祖父,正如梁启超说的,是“身兼将相,权倾举朝”。《清史稿》里也说是“得太后信杖眷顾之隆,一时无比,事无细巨,常待一言决焉”。\\

  在庚子那年,慈禧利用义和团杀洋人,又利用洋人杀义和团的一场大灾难中,荣禄对慈禧太后的忠诚,有了进一步表现。慈禧在政变后曾散布过光绪病重消息,以便除掉光绪。这个阴谋不料被人发觉了,后来闹到洋人出面要给光绪看病,慈禧不敢惹洋人,只好让洋人看了病。此计不成,她又想出先为同治立嗣再除光绪的办法。她选的皇储是端王载漪的儿子溥儁,根据荣禄的主意,到元旦这天,请各国公使来道贺,以示对这件举动的支持。可是李鸿章的这次外交没办成功,公使们拒绝了。这件事情现在人们已经很清楚了,不是公使们对慈禧的为人有什么不满,而是英法美日各国公使不喜欢那些亲近帝俄的后党势力过分得势。当然,慈禧太后从上台那天起就没敢惹过洋人。洋人杀了中国百姓,抢了中国的财宝,这些问题对她还不大,但洋人保护了康有为,又反对废光绪和立皇储,直接表示反对她的统治,这是她最忍受不了的。荣禄劝告她,无论如何不能惹恼洋人,事情只能慢慢商量,关于溥儁的名分,不要弄得太明显。《清史稿》里有这样一段记载:“患外人为梗,用荣禄言,改称大阿哥。”慈禧听从了荣禄的意见,可是溥儁的父亲载漪因为想让儿子当上皇帝,伙同一批王公大臣如刚毅、徐桐等人给慈禧出了另一个主意,利用反对洋人的义和团,给洋人压力,以收两败俱伤之效。义和团的问题,这时是清廷最头痛的问题。在洋人教会的欺凌压榨之下,各地人民不但受不到朝廷的保护,反而受到洋人和朝廷的联合镇压,因此自发地爆发了武装斗争,各地都办起了义和团,提出灭洋口号。义和团经过不断的斗争,这时已形成一支强大的武装力量,朝廷里几次派去军队镇压,都被他们打得丢盔曳甲。对团民是“剿”是“抚”,成了慈禧举棋不定的问题。载漪和大学士刚毅为首的一批王公大臣主张“抚”,先利用它把干涉废立的洋人赶出去再说。兵部尚书徐用仪和户部尚书立山、内阁学士联元等人坚决反对这种办法,认为利用团民去反对洋人必定大祸临门,所以主张“剿”。两派意见正相持不下,一件未经甄别的紧急情报让慈禧下了决心。这个情报把洋人在各地的暴行解释为想逼慈禧归政于光绪。慈禧大怒,立刻下诏“宣抚”团民,下令进攻东交民巷使馆和兵营,发出内帑赏给团民,悬出赏格买洋人的脑袋。为了表示决心,她把主“剿”的徐用仪、立山、联元等人砍了头。后来,东交民巷没有攻下,大沽炮台和天津城却先后失守,联军打向北京来了。慈禧这时又拿出了另一手,暗中向洋人打招呼,在炮火连天中派人到东交民巷去联络。北京失陷,她逃到西安,为了进一步表示和洋人作对的原来不是她,她又下令把主“抚”的刚毅、徐桐等一批大臣杀了头。在这一场翻云覆雨中,荣禄尽可能不使自己卷入旋涡。他顺从地看慈禧的颜色行事,不忤逆慈禧的意思,同时,他也给慈禧准备着“后路”。他承旨调遣军队进攻东交民巷外国兵营,却又不给军队发炮弹,而且暗地还给外国兵营送水果,表示慰问。八国联军进入北京,慈禧出走,他授计负责议和的李鸿章和奕劻,在谈判中掌握一条原则:只要不追究慈禧的责任,不让慈禧归政,一切条件都可答应。就这样,签订了赔款连利息近十亿两亿、让外国军队驻兵京城的辛丑条约。荣禄办了这件事,到了西安,“宠礼有加,赏黄马褂\footnote{黄马褂是皇帝骑马时穿的黄色外衣,“赏穿黄马褂”是清朝皇帝赏给有功的臣工的特殊“恩典”之一。}。双眼花翎\footnote{花翎是清朝皇帝赏给有功的臣工的礼帽上的装饰品。皇族和高级官员赏孔雀翎,低级官员赏鹖翎(俗称老鸹翎,因是蓝色的又称蓝翎)。皇帝赏臣工戴的花翎又依据官阶高低有单眼、双眼、三眼之别。}紫貂,随扈还京,加太子太保\footnote{商代以来历朝一般都设太师、太傅、太保,少师、少傅、少保作为国君辅弼之官,设太子太师、太子太傅、太子太保、太子少师、太子少傅、太子少保作为辅导太子之官。但后来一般都是大官加衔,以示恩宠,而无实权。明清两季亦以朝臣兼任,纯属虚衔。},转文华殿大学士”——除了《清史稿》里这些记载外,另外值得一说的,就是西太后为荣禄的女儿“指婚”,嫁与醇亲王载沣为福晋。\\

  关于我父母亲这段姻缘,后来听到家里的老人们说起,西太后的用意是很深的。原来政变以后,西太后对醇王府颇为猜疑。据说在我祖父园寝(墓地)上有棵白果树,长得非常高大,不知是谁在太后面前说,醇王府出了皇帝,是由于醇王坟地上有棵白果树,“白”和“王”连起来不就是个“皇”字吗?慈禧听了,立即叫人到妙高峰把白果树砍掉了。引起她猜疑的其实不仅是白果树,更重要的是洋人对于光绪和光绪兄弟的兴趣。庚子事件前,她就觉得可怕的洋人有点倾心于光绪,对她却是不太客气。庚子后,联军统帅瓦德西提出,要皇帝的兄弟做代表,去德国为克林德公使被杀事道歉。父亲到德国后,受到了德国皇室的隆重礼遇,这也使慈禧大感不安,加深了她心里的疑忌:洋人对光绪兄弟的重视,这是比维新派康有为更叫她担心的一件事。为消除这个隐患,她终于想出了办法,就是把荣禄和醇王府撮合成为亲家。西太后就是这样一个人,凡是她感到对自己有一丝一毫不安全的地方,她都要仔细加以考虑和果断处理,她在庚子逃亡之前,还不忘叫人把珍妃推到井里淹死,又何尝不是怕留后患而下的毒手?维护自己的统治,才是她考虑一切的根据。就这样,我父亲于光绪二十七年在德国赔了礼回来,在开封迎上回京的銮驾,奏复了一番在德国受到的种种“礼遇”,十一月随驾走到保定,就奉到了“指婚”的懿旨。\\