\fancyhead[LO]{{\scriptsize 1924-1930: 天津的“行在” · 郑孝胥的理想}} %奇數頁眉的左邊
\fancyhead[RO]{} %奇數頁眉的右邊
\fancyhead[LE]{} %偶數頁眉的左邊
\fancyhead[RE]{{\scriptsize 1924-1930: 天津的“行在” · 郑孝胥的理想}} %偶數頁眉的右邊
\chapter*{郑孝胥的理想}
\addcontentsline{toc}{chapter}{\hspace{1cm} 郑孝胥的理想}
\thispagestyle{empty}
郑孝胥在北京被罗振玉气跑之后,转年春天回到了我的身边。这时罗振玉逐渐遭到怀疑和冷淡,敌对的人逐渐增多,而郑孝胥却受到了我的欢迎和日益增长的信赖。陈宝琛和胡嗣瑗跟他的关系也相当融洽。一九二五年,我派他总管总务处,一九二八年,又派他总管外务,派他的儿子郑垂承办外务,一同做我对外联络活动的代表。后来他与我之间的关系,可以说是到了荣禄与慈禧之间的那种程度。\\

他比陈宝琛更随和我。那次我会见张作霖,事前他和陈宝琛都表示反对,事后,陈宝琛鼓着嘴不说话,他却说:“张作霖有此诚意表示,见之亦善。”他和胡嗣瑗都是善于争辩的,但是胡嗣瑗出口或成文,只用些老古典,而他却能用一些洋知识,如墨索里尼创了什么法西斯主义,日本怎么有个明治维新,英国《泰晤士报》上如何评论了中国局势等等,这是胡嗣瑗望尘莫及的。陈宝琛是我认为最忠心的人,然而讲到我的未来,绝没有郑孝胥那种令我心醉的慷慨激昂,那种满腔热情,动辄声泪俱下。有一次他在给我讲《通鉴》时,话题忽然转到了我未来的“帝国”:\\

\begin{quote}
	“帝国的版图,将超越圣祖仁皇帝一朝的规模,那时京都将有三座,一在北京,一在南京,一在帕米尔高原之上……”\\
\end{quote}

他说话时是秃头摇晃,唾星四溅,终至四肢颤动,老泪横流。\\

有时,在同一件事上说的几句话,也让我觉出陈宝琛和郑孝胥的不同。在康有为赐谥问题上,他两人都是反对的,陈宝琛在反对之余,还表示以后少赐谥为妥,而他在发表反对意见之后,又添了这么一句:“戊戌之狱,将来自然要拿到朝议上去定。”好像我不久就可以回紫禁城似的。\\

郑孝胥和罗振玉都积极为复辟而奔走活动,但郑孝胥的主张更使我动心。虽然他也是屡次反对我出洋和移居旅顺、大连的计划的。\\

郑孝胥反对我离开天津到任何地方去,是七年来一贯的。甚至到“九一八”事变发生,罗振玉带着关东军的策划来找我的时候,他仍然不赞成我动身。这除了由于他和罗振玉的对立,不愿我被罗垄断居奇,以及他比罗略多一点慎重之外,还有一条被人们忽视了的原因,这就是:他当时并不把日本当做唯一的依靠;他所追求的东西,是“列强共管”。\\

在天津时代,郑孝胥有个著名的“三共论”。他常说:“大清亡于共和,共和将亡于共产,共产则必然亡于共管。”他把北伐战争是看做要实行“共产”的。这次革命战争失败后,他还是念不绝口。他说:“又闹罢工了,罢课了,外国人的商业受到了损失,怎能不出头来管?”他的“三共论”表面上看,好像是他的感慨,其实是他的理想,他的愿望。\\

如果考查一下郑、罗二人与日本人的结交历史,郑到日本做中国使馆的书记宫是一八九一年,罗卖古玩字画、办上海《农报》,由此结识了给《农报》译书的日人藤田剑峰是在一八九六年,郑结交日人比罗要早五年。但是罗振玉自从认识了日本方面的朋友,眼睛里就只有日本人,辛亥后,他把复辟希望全放到日本人的身上,而郑孝胥却在日本看见了“列强”,从那时起他就认为中国老百姓不用说,连做官的也都无能,没出息,中国这块地方理应让“列强”来开发,来经营。他比张之洞的“中学为体,西学为用”更发展了一步,不但要西洋技术,西洋资本,而且主张要西人来做官,连皇家的禁卫军也要由客卿训练、统领。不然的话,中国永远是乱得一团糟,中国的资源白白藏在地里,“我主江山”迟早被“乱党”、“乱民”抢走,以至毁灭。辛亥革命以后,他认为要想复辟成功,决不能没有列强的帮忙。这种帮忙如何才能实现呢?他把希望寄托在“共管”上。\\

那时关于“列强”共管中国的主张,经常可以从天津外文报纸上看到。郑孝胥对这类言论极为留意,曾认真地抄进他的日记、札记,同时还叫他的儿子郑垂译呈给我。这是一九二七年六月九日登在日文报纸《天津日日新闻》上的一篇:\\

\begin{quote}
	英人提倡共管中国\\

联合社英京特约通信。据政界某要人表示意见谓:中国现局,日形纷乱,旅华外国观察家曾留心考察,以为中国人民须候长久时期,方能解决内部纠纷,外国如欲作军事的或外交的干涉,以解决中国时局问题,乃不可能之事。其唯一方法,只有组织国际共管中国委员会,由英美法日德意六国各派代表一名为该会委员,以完全管理中国境内之军事。各委员之任期为三年,期内担任完全责任,首先由各国代筹二百五十兆元以为行政经费,外交家或政客不得充当委员,委员人才须与美国商(务)部长胡佛氏相仿佛。此外,又组织对该委员会负责之中外混合委员会,使中国人得在上述之会内受训练。\\
\end{quote}

郑孝胥认为,这类的计划如果能实现,我的复位的时机便到了。\\

那年夏天我听了罗振玉的劝说,打算到日本去,郑孝胥就根据那篇文章勾起的幻想,向我提出了“留津不动,静候共管”的劝告。这是他记在日记里的一段:\\

\begin{quote}
	五月戊子二十四日(六月二十三日)。诣行在。召见,询日领事约谈情形(即去日事)。因奏曰:今乘舆狩于天津,皇帝与天下犹未离也。中原士大夫与列国人士犹得常接,气脉未寒,若去津一步,则形势大变,是为去国亡命,自绝于天下。若寄居日本,则必为日本所留,兴复之望绝矣!自古中兴之主,必借兵力。今则海内大乱,日久莫能安戢,列国通不得已,乃遣兵自保其商业。他日非为中国置一贤主,则将启争端,其祸益大。故今日皇上欲图中兴,不必待兵力也,但使圣德令名彰于中外,必有人人欲以为君之日。\\
\end{quote}

他提出过不少使“圣德令名彰于中外”的办法,如用我的名义捐款助赈,用我的名义编纂《清朝历代政要》,用我的名义倡议召开世界各国弭兵会议等等。有的我照办了,有的无法办,我也表示了赞许和同意。\\

我委任奥国亡命贵族阿克第男爵到欧洲为我进行游说宣传,临行时,郑孝胥亲自向他说明,将来如蒙各国支持“复国”,立刻先实行这四条政策:“一、设责任内阁,阁员参用客卿;二、禁卫军以客将统帅教练;三、速办张家口——伊犁铁路,用借款包工之策;四、国内设立之官办商办事业,限五年内一体成立。”\\

郑孝胥的想法,以后日益体系化了。有一次,他说:“帝国铁路,将四通八达,矿山无处不开,学校教育以孔教为基础……。”我问他:“列强真的会投资吗?”他说:“他们要赚钱,一定争先恐后。臣当年承办瑷珲铁路,投资承包的就是如此,可惜朝廷给压下了,有些守旧大臣竟看不出这事大有便宜。”那时我还不知道,作为辛亥革命风暴导火线的铁路国有化政策,原来就是郑孝胥给盛宣怀做幕府时出的主意。假若我当时知道这事,就准不会再那样相信他。当时听他说起办铁路,只想到这样的问题:“可是辛亥国变,不就是川、湘各地路矿的事闹起来的吗?”他附和说:“是的,所以臣的方策中有官办有商办。不过中国人穷,钱少少办,外国人富,投资多多办,这很公平合理。”我又曾问过他:“那些外国人肯来当差吗?”他说:“待如上宾,许以优待,享以特权,绝无不来之理。”我又问他:“许多外国人都来投资,如果他们争起来怎么办?”他很有把握地说:“唯因如此,他们更非尊重皇上不可。”\\

这就是由共管论引申出来的日益体系化的郑孝胥的政策,也是我所赞许的政策。我和他共同认为,只有这样,才能取回我的宝座,继续大清的气脉,恢复宗室觉罗、文武臣僚、士大夫等等的旧日光景。\\

郑孝胥在我出宫后,曾向段祺瑞活动“复原还宫”,在我到天津后,曾支持我拉拢军阀、政客的活动,但是,在他心里始终没忘掉这个理想。特别是在其他活动屡不见效的情况下,他在这方面的愿望尤其显得热烈。这在使用谢米诺夫这位客卿的问题上,分外地可以看出来。\\

当我把接见谢米诺夫的问题提出来时,陈宝琛担心的是这件事会引起外界的责难,郑孝胥着急的却是怕我背着他和罗振玉进行这件事。他对陈宝琛说:“反对召见,反而使皇上避不咨询,不如为皇上筹一妥善谨密之策,召见一次。”结果,谢米诺夫这个关系便叫他拉到手上了。\\

使他对谢米诺夫最感到兴趣的,是谢和列强的关系。当谢米诺夫吹嘘列强如何支持他,而各国干涉中国的政局之声又甚嚣尘上的时候,郑孝胥认为时机来了,兴高采烈地给张宗昌和谢米诺夫撮合,让谢米诺夫的党羽多布端到蒙古举兵起事,并且亲自跑上海,跑青岛。他进行了些什么具体活动,我现在已记忆不清了,只记得他十分得意地写了不少诗。他的日记里有这样自我欣赏的描写:“晨起,忽念近事,此后剥极而复,乃乾旋坤转之会,非能创能改之才,不足以应之也。”“如袁世凯之谋篡,张勋之复辟,皆已成而旋败,何者?无改创之识则枘凿而不合矣!”(一九二五年十一月)“诸人本极畏事,固宜如此!”“夜与谢米诺夫。包文渊、毕瀚章、刘凤池同至国民饭店,……皆大欢畅,约为同志,而推余为大哥。”(一九二六年五月)\\

英国骗子罗斯,以办报纸助我复辟为名,骗了我一笔钱,后来又托郑孝胥介绍银行贷款,郑孝胥因罗是谢米诺夫和多布端的朋友,就用自己的存折作押,给他从银行借了四千元。郑垂觉得罗斯不可靠,来信请他父亲留心,他回信教训儿子说:“不能冒险,焉能举事?”后来果然不出他儿子所料,罗斯这笔钱到期不还,银行扣了郑的存款抵了账。尽管如此,当罗斯底下的人又来向郑借钱的时候,由于谢米诺夫的关系,经多布端的说情,他又掏出一千元给了那个骗子。当然,我的钱经他手送出去的,那就更多。被他讥笑为“本极畏事,固宜如此”的陈宝琛,后来在叹息“苏龛(郑宇),苏龛,真乃疏忽不堪!”之外更加了一句:“慷慨,慷慨,岂非慷他人之慨!”\\

后来,他由期待各国支持谢米诺夫,转而渴望日本多对谢米诺夫加点劲,他又由期待各国共管,转而渴望日本首先加速对中国的干涉。当他的路线转而步罗振玉后尘的时候,他的眼光远比罗振玉高得多,什么三野公馆以及天津日军司令部和领事馆,都不在他眼里;他活动的对象是直接找东京。不过他仍然没忘了共管,他不是把日本看做唯一的外援,而是第一个外援,是求得外援的起点,也可以说是为了吸引共管的第一步,为“开放门户”请的第一位“客人”。\\

他提出了到东京活动的建议,得到了我的赞许,得到了芳泽公使的同意。和他同去的,有一个在日本朝野间颇有“路子”的日本人太田外世雄。他经过这个浪人的安排,和军部以及黑龙会方面都发生了接触,后来,他很满意地告诉我,日本朝野大多数都对我的复辟表示了“关心”和“同情”,对我们的未来的开放政策感到了兴趣。总之,只要时机一到,我们就可以提出请求支援的要求来。\\

关于他在日本活动的详细情形,我已记不清了。我把他的日记摘录几段如后,也可以从中看出一些他在日本广泛活动的蛛丝马迹:\\

\begin{quote}
	八月乙丑初九日(阴历,下同)。八点抵神户。福田与其友来迎。每日新闻记者携具来摄影。偕太田、福田步至西村旅馆小憩,忽有岩日爱之助者,投刺云:兵库县得芳泽公使来电嘱招待,兵库县在东京未回,今备汽车唯公所用。遂同出至中华会馆。又至捕公庙,复归西村馆,即赴汽车站买票,至西京,入京都大旅馆。来访者有:大阪时事报社守田耕治、大田之友僧足利净回,岩田之友小山内大六,为国杂社干事。与岩田、福田、太田同至山东馆午饭。夜付本多吉来访,谈久之。去云:十点将复来,候至十二点,竟不至。\\

丙寅初十日。……将访竹本,遇于门外,遂同往。内藤虎来谈久之。\\

太田之友松尾八百藏来访,密谈奉天事。\\

丁卯十三日。福田以电话告:长尾昨日已归,即与太田、大七走访之。\\

长尾犹卧,告其夫人今日匆来,遂乘电车赴大限。……岩田爱之助与肃邸四子俱来访。宪立(定之)密语余奉天事,消息颇急,欲余至东京日往访藤田正实、宇垣一成。朝日、每日二社皆摄影,复与肃四子共摄一影,乃访住友经理小仓君。……\\

庚午十四日。长尾来谈,劝取奉天为恢复之基。……\\

壬申十六日。长尾雨山以电话约勿出,当即来访,遂以汽车同游天满官金阁寺而至岚山。高峰峭立,水色甚碧,密林到顶,若无路可入者。入酒家,亦在林中,隐约见岩岫压檐而已,饮酒食鱼,谈至三时乃去。\\

癸酉十七日。……长尾来赠画扇,送至圆山公园,左阿、囗家、狩野、内藤、近重、铃木皆至,顷之高濑亦至,唯荒木、内村在东京未归。……\\

丙子二十日。作字。雨。诣长尾辞行。……太田来云,东京备欢迎者甚众,将先往约期。\\

辛巳廿五日。十一时至东京下火车。至车站投刺者数十人。小田切、高田丰村、冈野皆来帝国旅馆。雨甚大。岩田、水野梅晓亦来。冈野自吴佩孚败后囗而为僧。夜宿于此。\\

壬午二十六日。……水野谈日政府近状颇详,谓如床次、后藤、细川侯、近卫公,皆可与谈。\\

癸未二十七日。……送过水野,复同访床次。床次脱离民主党而立昭和俱乐部,将为第三党之魁。岩田来。小田切来。大田、白井、水野、佃信夫来。山田来。汪荣宝来。……夜赴近卫公之约,坐客十余人,小田切、津田、水野、太田皆在坐。近卫询上近状,且极致殷勤。……\\

甲申二十九日。……川田瑞穗者称,长尾雨山之代理人,与松本洪同来约九月初八日会宴,坐客为:平沼骐一郎,枢密院副议长;桦山资英,前内阁秘书长;牧野谦次郎,能文,早稻田教授;松平康国,早稻田教授;国分青崖,诗人;田边碧堂,诗人;内田周平,能汉文。此外尚十余人。……岩田与肃邸第十八子宪开来访,今在士官学校。……津田静枝海军大佐邀至麻布区日本料理馆,为海军军令部公宴。主席者为米内少将,坐客为:有田八郎,水野梅晓,中岛少将,园田男爵(东乡之婿),久保田久晴海军中佐等。……\\

九月丙戌朔。太田来。参谋本部总长铃木,次长南,以电话约十时会晤。与大七、大田同往。铃木询上近状,且云:有恢复之志否?南次长云:如有所求,可以见语。对曰:正究将来开放全国之策,时机苟至,必将来求。吉田茂外务次官约午饭,座中有:清浦子爵奎吾,冈部长景子爵,高田中将,池田男爵,有田,岩村,水野,太田等。……\\

丁亥初二日。……岩田偕宪开、李宝琏、刘牧蟾来访。李刘皆在士官学校。……\\

庚寅初五日。……水野、太田来。与水野同访后藤新平,谈俄事良久。……\\

癸巳初八日。……工藤邀同至白井新太郎宅,晤高山中将,野中、多贺二少将,田锅、松平皆在座,颇询行在情形。\\

戊戌十三日。太田送至神户登长崎九,长尾雨山自西京来别。富冈、福田皆来。十一点半展轮。……\\
\end{quote}

他在日本,被当做我的代表,受到各种热心于恢复清朝的人物的接待。其中有不少原是我的旧交,例如高田丰村是前天津驻屯军司令官,有田八郎和吉田茂做过天津总领事,白井是副领事,竹本多吉是在北京时把我接进日本兵营的那位大任。岩田爱之助就是在我窗外放枪的那位黑龙会会员,佃信夫则是不肯在总领事有田面前谈“机密”的那位黑龙会重要人物。不管他们在中国时怎样不和,这时却彼此融洽无间地共同接待着“郑大臣”。除了这些过去曾直接出头露面的以外,那些原居于幕后的大人物,如后来做过首相、陆相等要职的近卫(文囗)、宇垣(一成)、米内(光政)、平沼(骐一郎)、铃木(贯太郎)、南(次郎),以及在第二次世界大战后上台的吉田茂等人,还有一些出名的政客、财阀,此时全都出了面。也许郑孝胥和这些人会谈时,他的“开放全国之策”引起的反应使他太高兴了,所以在伪满成立以后,第一批“客人”已经走进了打开的“门户”,他仍然没有忘记共管的理想,一有机会便向外面宣传“门户开放,机会均等”。这犹如给强盗做底线的仆人,打开了主人家的大门,放进了一帮强盗,当了一帮强盗的大管事,尤感不足,一定还要向所有各帮强盗发请帖,以广招徕。这自然就惹恼了已经进了门的强盗,一脚把他踢到一边。\\