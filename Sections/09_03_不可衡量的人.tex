\fancyhead[LO]{{\scriptsize 1955-1959: 接受改造 · 不可衡量的人}} %奇數頁眉的左邊
\fancyhead[RO]{} %奇數頁眉的右邊
\fancyhead[LE]{} %偶數頁眉的左邊
\fancyhead[RE]{{\scriptsize 1955-1959: 接受改造 · 不可衡量的人}} %偶數頁眉的右邊
\chapter*{不可衡量的人}
\addcontentsline{toc}{chapter}{\hspace{1cm} 不可衡量的人}
\thispagestyle{empty}
一九五六年春节后,有一天所长给我们讲完了国内建设情况,向我们宣布了一项决定:\\

“你们已经学完了关于第一个五年计划、农业合作化、手工业和私营工商业的社会主义改造这一系列的文件,你们从报上又看到了几个大城市私营企业实现了公私合营的新闻,你们得到的关于社会主义建设的知识还仅限于是书本上的。为了让理论学习与实际联系起来,你们需要看一看祖国社会上的实况,因此政府不久将要组织你们到外面去参观,先看看抚顺,然后再看看别的城市。”\\

这天管理所里出现了从来没有过的愉快气氛,许多人都感到兴奋,还有人把这件事看做是释放的预兆。而我却与他们不同,我想这对他们也许是可能的,对我则决无可能。我不但对于释放不敢奢望,就是对于抛头露面的参观,也感到惴惴不安。\\

这天下午,在花畦边上,我听到有人在议论我所担心的一个问题。\\

“你们说,老百姓看见咱们,会怎么样?”\\

“我看有政府人员带着,不会出什么岔子,不然政府不会让咱们出去的。”\\

“我看难说,老百姓万一激动起来呢?我可看见过,我是小职员出身的。”这是前伪满兴农部大臣老甫说的,他从前做过张作霖军队里的小粮袜官。“老百姓万一闹起来,政府该听谁的呢?”\\

“放心吧,政府有把握,否则是不会让我们去的。”\\

这时我们组新任的学习组长,前伪汪政权的外交官老初走了过来,插嘴道:“我想政府不会宣布我们的身份,对不对?”\\

“你以为不宣布,人家就不知道?”老元讥笑他,“你以为东北人不认识你就不要紧了?只要东北老百姓认出一个来,就全明白啦!想认出一个来可不难啊!”\\

老元的话正说到我心坎上。东北人民从前被迫向“御真影”行礼行了十来年,难道认出我来还费事吗?\\

东北人民那样恨我,政府怎么就敢相信他们见了我会不激动呢?如果激动起来,会不会向政府要求公审我?老甫问的也对,到那时候“政府听谁的呢?”\\

那时,在我心目中,老百姓是最无知的、最粗野的人。我认为尽管政府和共产党决定了宽大和改造政策,老百姓却是不管这一套的;他们怀着仇恨,发作起来,只会用最粗暴的手段对付仇人。政府那时是不是有办法应付,我很怀疑。我认为最大的可能,是“牺牲”掉我,以“收民心”。\\

许多人都以欢欣鼓舞的心情迎接这次参观,我却终日惴惴不安,好像面临着的是一场灾难。我竟没有料到,我在参观中所看到的人,所受到的待遇,完全与我想象的相反。\\

我在参观中看到了许多出乎意料的事,我将在下一节中再说,现在我要先说说那几个最出乎意料的、不可衡量的人物。\\

第一个是一位普通的青年妇女。她是当年平顶山惨案的幸存者,现在是抚顺露天矿托儿所的所长。我们首先参观的是抚顺露天矿。矿方人员介绍矿史时告诉了我们这个惨案。\\

抚顺露天矿大坑的东部,距市中心约四公里,有一座住着一千多户人家的村镇,地名叫平顶山。这里的居民大部分都是穷苦的矿工。日本强盗侵占了东北,抚顺地区和东北各地一样也出现了抗日义勇军,平顶山一带不断地有抗日军出没活动。一九三三年中秋节的夜里,南满抗日义勇军出击日寇。袭击抚顺矿的一路抗日义勇军在平顶山和日寇遭遇,击毙了日寇杨伯堡采炭所长渡边宽一和十几名日本守备队的队员,烧掉了日寇的仓库。在天亮以前,抗日义勇军转移到新宾一带去了。\\

抗日义勇军走后,日本强盗竟然决定用“通匪”的罪名,向手无寸铁的平顶山居民实行报复。第二天,日本守备队六个小队包围了平顶山,一百九十多名凶手和一些汉奸,端着上了刺刀的步枪,挨门挨户把人们赶出来,全村的男女老幼,一个不留全被赶到村外的山坡上。等全村三千多人全聚齐了,日寇汽车上蒙着黑布的六挺机枪全露了出来,向人群进行了扫射。三千多人,大人和孩子,男人和女人,生病的老人和怀孕的妇女,全倒在血泊里了。强盗凶手还不甘心,又重新挨个用刺刀扎了一遍,有的用皮鞋把没断气的人的肠子都踢出来,有的用刺刀划开孕妇的肚子,挑出未出生的婴儿举着喊:“这是小小的大刀匪!”\\

野兽们屠杀之后,害怕人民的报复,企图掩尸灭迹,用汽油将六七百栋房子全烧光,用大炮轰崩山土,压盖尸体,又用刺网封锁了四周,不准外村人通过。以后还向周围各村严厉宣布,谁收留从平顶山逃出去的人,谁全家就要替死。那天白天烟尘笼罩了平顶山,夜里火光映红了半边天。从此平顶山变成了一座尸骨堆积的荒山。以后,抚顺周围地区流传着一首悲痛的歌谣:\\

\begin{quote}
	当年平顶山人烟茂,一场血洗遍地生野草,拣起一块砖头,拾起一根人骨,日寇杀死我们的父母和同胞,血海深化永难消!\\
\end{quote}

但是日本强盗杀不绝英雄的平顶山人,也吓不倒英雄的抚顺工人。一个名叫方素荣的五岁小女孩,从血泊里逃出来,被一个残废的老矿工秘密收留下。她活下来了,今天她是血的历史见证人。\\

我们后来看完了矿场,轮到参观矿上福利事业的时候,便到方素荣工作的托儿所去访问。这天方所长有事到沈阳去了,所里的工作人员向我们谈了昨天日本战犯跟方素荣见面的情形。\\

日本战犯来参观托儿所,所里的工作人员说:“对不起,我们没让所长接待你们,因为她是平顶山人,我们不愿意让她受到刺激。”日本战犯差不多都知道平顶山事件,他们听了这话,一时面面相觑,不知如何是好。后来,他们商议了一下,认为应当向这位受到日本帝国主义者灾难的人表示谢罪,恳求她出来见一见他们。女工作人员很不愿意,但经他们再三恳求,终于把方所长请来了。\\

日本战犯们全体向她鞠躬表示谢罪之后,请求她把当时的经历讲一讲。方素荣答应了。\\

“我到现在还记得清清楚楚,”她说,“前前后后都是街坊,爷爷领着我,妈妈抱着我兄弟——他还不会说话。鬼子兵跟汉奸吆喝着说去照相。我问爷爷,照相是什么,爷爷给了我一个刚做好的风车,说别问了,别问了……”\\

五岁的方素荣就是这样随了全村的人,同做高粱秆风车的爷爷、守寡的妈妈和不会说话的兄弟,到刑场去的。机枪响了的时候,爷爷把她压在身子底下,她还没哭出声便昏了过去。等她醒过来,四周都是血腥,尘烟迷漫在上空,遮掩了天空的星斗。……\\

八处枪弹和刺刀的创伤使她疼痛难忍,但是更难忍的是恐怖。爷爷已经不说话了,妈妈和兄弟也不见了。她从尸体堆里爬出来,爬向自己的村子,那里只有余烬和烟尘。她连跑带爬,爬出一道刺网,在高粱茬地边用手蒙住脸趴在地上发抖。一个老爷爷把她抱起来,裹在破袄里,她又昏睡过去。\\

老爷爷是一个老矿工,在抚顺经历了“来到千金寨,就把铺盖卖,新的换旧的,旧的换麻袋”的生活,在矿里被鬼子压榨了一生,弄成残废,又被一脚踢出去,晚年只得靠卖烟卷混饭吃。他把方素荣悄悄地带到单身工人住的大房子,放在一个破麻袋里。这个大房子里二百多人睡在一起,老爷爷占着地头一个角落,麻袋就放在这里,白天扎着口,像所有的流浪汉的破烂包似的,没人察觉,到晚上人们都睡下的时候,他偷偷打开麻袋口,喂小姑娘吃喝。但这终不是长久之计,老爷爷问出她舅舅的地址,装出搬家的模样,挑起麻袋和烟卷箱子,混过鬼子的封锁口,把她送到不远一个屯子上的舅舅家里。舅舅不敢把她放在家里,只好藏在野外的草堆里,每天夜里给她送吃喝,给她调理伤口。这样熬到快要下雪的时候,才又把她送到更远的一个屯子的亲戚家里,改名换姓地活下来。\\

从心灵到肌肤,无处不是创伤的方素荣,怀着异常的仇恨盼到了日本鬼子投降,但是抚顺的日本守备队换上了国民党的保安团,日本豢养的汉奸换上了五子登科的劫收大员,大大小小的骑在人民头上的贪官污吏。流浪还是流浪,创伤还是创伤,仇恨还是仇恨。旧的血债未清,新的怨仇又写在抚顺人民的心上。为了对付人民的反抗,蒋介石军队在这个地区承继了日本强盗的“三光政策”,灾难重临了方素荣的家乡。方素荣又煎熬了四个年头,终于等到了这一天,她的家乡解放了,她的生命开始见到了阳光。党和人民政府找到了她,她得到了抚养,受到了教育,参加了工作,有了家庭,有了孩子。现在她是抚顺市的一名劳动模范。\\

今天,这个在仇恨和泪水中长大的,背后有个强大政权的人,面对着一群对中国人犯下滔天罪行的日本战犯,她是怎样对待他们的呢?\\

“凭我的冤仇,我今天见了你们这些罪犯,一口咬死也不解恨。可是,”她是这样说的,“我是一个共产党员,现在对我更重要的是我们的社会主义事业,是改造世界的伟大事业,不是我个人的恩仇利害。为了这个事业,我们党制定了各项政策,我相信它,我执行它,为了这个事业的利益,我可以永远不提我个人的冤仇。”\\

她表示的是宽恕!\\

这是使几百名日本战犯顿时变成目瞪口呆的宽恕,这是使他们流下羞愧悔恨眼泪的宽恕。他们激动地哭泣着,在她面前跪倒,要求中国政府给他们惩罚,因为这种宽恕不是一般的宽恕。\\

一个普通的青年妇女,能有如此巨大的气度,这实在是难以想象的。然而,我亲身遇到了还有更难以想像的事。假定说,方素荣由于是个共产党员、工作干部,她的职务让她必须如此(这本来就是够难于理解的),那么台山堡农村的普通农民,又是由于什么呢?\\

台山堡是抚顺郊区一个农业社的所在地。第二天早晨,在去这个农业社的路上,我心中一直七上八下,想着检举材料上那些农民的控诉,想象着怀着深仇的农民将如何对待我。我肯定方素荣对战犯所做到的事,“无知”而“粗野”的农民是决做不到的。昨天在抚顺矿区曾遇到一些工人和工人家属,对我们没有什么“粗野”的举动,甚至于当我们走进一幢大楼,参观工人宿舍时,还有一位老太太像待客人似地想把我让进地板擦得甑亮的屋子。我当时想,这是因为你不知道我们的身份,如果知道了的话,这些文明礼貌就全不会有了。昨天参观工人养老院时,所方让我们分头访问老人们。这都是当了一辈子矿工或者因工伤残废被日本人从矿里踢了出来的人,他们无依无靠,流浪街头,支持到抚顺解放时,只剩了一口气。人民政府一成立,就抢救了他们,用从前日本人的豪华旅店改做这个养老院,让他们安度晚年。他们每天下棋、养花、看报,按自己的兴趣进行各项文娱活动。我和几个伙伴访问的这位老人,向我们谈了他一生的遭遇,那等于一篇充满血泪和仇恨的控诉书。听他说的伪满政权下矿工们的苦难,我一面感到羞耻,一面感到害怕。我生怕他把我认了出来,因此一直躲在角落里,不敢出声。我当时曾注意到,老人的这间小屋的墙上,没有工人宿舍里的那些男女老少的照片,只有一张毛主席的像。显然,老人在世上没有一个亲人,即使有,也不会比这张相片上的人对他更亲。但是毛主席的改造罪犯的政策,在他心里能通得过吗?至少他不会同意宽大那些汉奸吧?\\

在第一天的参观中,每逢遇到人多的地方,我总是尽量低着头。我发现并非是我一人如此,整个的参观行列中,没有一名犯人是敢大声出气的。在抚顺曾督工修造日本神庙的大下巴,更是面如死灰,始终挤到行列中心,尽量藏在别人身后。我们到达台山堡的时候,简直没有一个人敢抬起脸来的。我们就是这样不安地听了农业社主任给我们讲的农业社的历史与现况,然后,又随着他看了新式农具、养鸡场、蔬菜暖房、牲口棚、仓库等处。我们一路上看到的人不多,许多社员都在田间劳动。在参观的几处地方遇到的人,态度都很和善,有的人还放下手中的活,站起来向我们打招呼。我庆幸着人们都没把我认出来,心里祝愿能永远如此。但是到最后,当我访问一家社员时,我就再也无法隐藏我自己了。\\

我同几个伙伴访问的这家姓刘,一共五口人,老夫妇俩参加农业劳动,大儿子是暖窖的记账员,二儿子读中学,女儿在水电站工作。我们去的时候只有刘大娘一个人在家。她正在做饭,看见社干部领着我们进来,忙着解下围裙,把我们让进了新洋灰顶的北房。她像对待真正的客人似的,按东北的风俗让我们进了里间,坐上炕头。我坐在炕边上,紧靠着西墙根一个躺柜,柜面上摆着带有玻璃罩的马蹄表,擦得晶亮的茶具,对称排列的瓷花瓶和茶叶缸。\\

陪我们来的一位社干部没有告诉刘大娘我们是什么人,只是对她说:“这几位是来参观的,看看咱们社员的生活,你给说说吧!”刘大娘不擅长词令,但是从她断续而零散的回忆中,我还是听出了这个早先种着七亩地的七口之家,在伪满过的原是像乞丐一样的生活。“种的是稻子,吃的却是橡子面,家里查出一粒大米就是‘经济犯’,稻子全出了荷。听说街上有个人,犯病吐出的东西里有大米,叫警察抓去了。……一家人穿的邋里邋遢。可还有不如咱家的,大姑娘披麻袋。有一年过年,孩子肚子里没食,冻的别提,老头子说,咱偷着吃一回大米饭吧,得,半夜警察进屯子啦,一家人吓得像啥似的。原来是抓差,叫去砍树、挖围子,说是防胡子,什么胡子,还不是怕咱们抗日联军!老头子抓去了。这屯子出劳工就没几个能活着回来的。……”\\

正说着,她的儿子回来了。他的个子很小,仔细一看,才知道他的腿很短,原是个先天残废的人。他回答了我们不少问题,谈到过去,这个青年在旧社会里,先天的残废使他就像一只狗似地活着,如今他却做了暖窖的记账员,像别人一样尊严地工作着。我从这不到三十岁的人的眼睛里,看到了对过去生活的仇恨和忿怒。但是,当话题一转到今天的生活,他和母亲一样,眼神和声调里充满了愉快和自信。他和母亲不同的地方,是谈家里的事比较少,而谈起了社里的暖房蔬菜的生产,则是如数家珍。这个社的蔬菜,主要是供应市区需要,不分四季,全年供应;蔬菜品种大部分是解放前没有的。当他历数着西红柿、大青椒等等品种的产量时,他母亲拦过了话头,说他们这一家从前不用说没见过西红柿,就连普通的大白菜也难得吃到。由蔬菜又谈到从前吃糠咽菜的生活,刘大娘顺手拉开屋角的一只瓮盖,让我们看看里面的大米。这时儿子不禁笑起来,说:“大米有什么可看的?”她立刻反驳道:“现在没什么可看的,可是你在康德那年头看见过几回?”\\

刘大娘的这句话,沉重地打在我的心上。\\

我刚走进这家人的房门时,还担心着是不是会有人问起我的姓名,而现在,我觉得如果在跨出这个房门之前再不说出自己的姓名,那简直是不可饶恕的欺骗。\\

我站立起来,向着刘大娘低头说:\\

“您说的那个康德,就是伪满的汉奸皇帝溥仪,就是我。我向您请罪。……”\\

我的话音未完,同来的几个伪大臣和伪将官都立起来了。\\

“我是那个抓劳工的伪勤劳部大臣……”\\

“我是搞粮谷出荷的兴农部大臣……”\\

“我是给鬼子抓国兵的伪军管区司令……”\\

……\\

那老大娘呆住了。显然这是出乎她意料的事。即使她知道来参观的是汉奸犯,也未必料到我们的姓名和具体的身份,即使她知道我的姓名、身份,也未必料到会向她请罪,请她发落。……\\

她怎样发落?痛骂吧?痛哭吧?或者走出去,把邻居们都叫来,把过去的死难者的家属都找来,共同地发泄一番怒气吧?\\

不。她叹息了一声。这是把凝结起来的空气和我的心脏融化开来的叹息:\\

“事情都过去了,不用再说了吧!”她擦擦眼泪,“只要你们肯学好,听毛主席的话,做个正经人就行了!”\\

原来我们是默默地垂泪,听了这句话,都放声哭出来了。\\

“我知道你们是什么人。”半晌没说话的儿子说,“毛主席说,大多数罪犯都能改造过来。他老人家的话是不会错的。你们好好改造认罪,老百姓可以原谅你们!”\\

这两个普普通通的农民,被我想象成“粗野的、无知的、容易激动地发泄仇恨而又根本不管什么改造和宽大”的农民,就是这样地宽恕了我们!\\

这是如此伟大的、不可能用我的标尺加以衡量的人。\\

我用最卑鄙、最可耻的坏心去揣度他们,而他们却用那么伟大、那么高贵的善心对待我们。\\

他们是今天当家做主的人,强大的政府和军队——共产党所领导的巨大力量全部站在他们身后,他们面前是对他们犯了滔天罪行的罪犯,而他们却给了宽恕!\\

他们为什么那样相信党和毛主席?他们怎么能把党的改造罪犯政策从心底上接受下来呢?而共产党和人民政府为什么那样相信人民,相信他们一定会接受它的政策?\\

这一次的参观也给了我答案。\\