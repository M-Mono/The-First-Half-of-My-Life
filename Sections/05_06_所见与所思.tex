\fancyhead[LO]{{\scriptsize 1931-1932: 到东北去 · 所见与所思}} %奇數頁眉的左邊
\fancyhead[RO]{} %奇數頁眉的右邊
\fancyhead[LE]{} %偶數頁眉的左邊
\fancyhead[RE]{{\scriptsize 1931-1932: 到东北去 · 所见与所思}} %偶數頁眉的右邊
\chapter*{所见与所思}
\addcontentsline{toc}{chapter}{\hspace{1cm} 所见与所思}
\thispagestyle{empty}
我到旅顺以后,感到最惶惑不安的,倒不是因为受到封锁。隔离,而是从上角这几个日本人口中听到,关东军似乎连新国家的国体问题还没定下来。\\

这对我说来,比没有人在码头上迎接我更堵心。没有人迎接,还可以用“筹备不及”、“尚未公布”的话来解释。“国体未定”又是怎么回事呢?国体既然未定,土肥原干么要请我到满洲来呢?\\

郑孝胥和上角向我解释说,土肥原没有说谎,关东军支持我复位和主持大计的话全不错,不过这是满洲的事,当然还要和满洲人商量,没有商量好以前,自然叫做“未定”。\\

我已经不像在汤岗子那样容易相信这些人了,但我又找不到任何别人商议事情。这还是我第一次离开我的师傅。在没师傅指点的情形下,我只好采取商衍瀛的办法,找神仙帮忙来解答问题。我拿出从天津带来的一本《未来预知术》,摇起了金钱神课。记得我摇出了一课“乾乾”卦,卦辞还算不坏。于是我就这样的在郑孝胥、罗振玉和诸葛亮\footnote{《未来预知术》是香港出版的一本迷信书,伪称是诸葛亮的著作,可是其中的封辞中有汉代以后的诗文典故。}的一致劝导下,捺着性子等待下去。\\

有一天,上角来问我,是不是认识马占山。我说在天津时,他到张园来过,算是认识吧。上角说,板垣希望我能写一封信,劝马占山归顺。我说在天津时已曾写过一封,如果需要,还可以再写。这第二封劝降书并没有用上,马占山就投降了。虽然我的信未发生作用,可是关东军请我写信这件事给了我一种安慰,我心里这样解释:这显然是日本人承认我的威信,承认这块江山必须由我统治才行。我是谁呢,不就是大清的皇帝吗?这样一想,我比较安心了些。\\

这样等了三个月,到我过生日的第二天,即一九三二年二月十九日,忽然来了一个消息,刚刚复会的“东北行政委员会”通过了一项决议,要在满洲建立一个“共和国”。所谓东北行政委员会是二月十八日复会的,这个委员会由投降的原哈尔滨特区长官张景惠、辽宁(这时被改称奉天)省主席臧式毅、黑龙江省代理主席马占山和被这委员会追认的吉林省主席熙洽组成,张景惠为委员长。二月十九日,这个委员会在板垣导演下通过了那项决议,接着又发表了一个“独立宣言”。这些消息传来之后,除了郑氏父子以外,我身边所有的人,包括罗振玉在内无不大起恐慌,人人愤慨。\\

这时占据着我全心的,不是东北老百姓死了多少人,不是日本人要用什么办法统治这块殖民地。它要驻多少兵,要采什么矿,我一概不管,我关心的只是要复辟,要他们承认我是个皇帝。如果我不为了这点,何必千里迢迢跑来这里呢?我如果不当皇帝,我存在于世上还有什么意义呢?陈宝琛老夫子以八十高龄的风烛残年之身来到旅顺时,曾再三对我说:“若非复位以正统系,何以对待列祖列宗在天之灵!”\\

我心中把土肥原、板垣恨得要死。那天我独自在前肃亲王的客厅里像发了疯似地转来转去,纸烟被我捏断了一根又一根,《未来预知术》被我扔到地毯上。我一下子想起了我的静园,想到假如我做不成皇帝,还不如去过舒适的寓公生活,因为那样我还可以卖掉一部分珍玩字画,到外国去享福。这样一想,我有了主意,我要向关东军表明态度,如果不接受我的要求,我就回天津去。我把这主意告诉了罗振玉和郑孝胥,他们都不反对。罗振玉建议我先送点礼物给板垣,我同意了,便从随身带的小件珍玩中挑了几样叫他去办。恰好这时板垣来电话请郑罗二人去会谈,于是我便叫陈曾寿为我写下必须“正统系”的理由,交给他们带给板垣,叫他们务必坚持,向板垣说清楚我的态度。\\

我写的那些理由共十二条(后四条是陈曾寿续上的):\\

\begin{quote}
	一、尊重东亚五千年道德,不得不正统系。\\

二、实行王道,首重伦常纲纪,不得不正统系。\\

三、统驭国家,必使人民信仰钦敬,不得不正统系。\\

四、中日两国为兄弟之邦,欲图共存共荣,必须尊崇固有之道德,使两国人民有同等之精神,此不得不正统系。\\

五、中国遭民主制度之害已二十余年,除少数自私自利者,其多数人民厌恶共和,思念本朝,故不得不正统系。\\

六、满蒙人民素来保存旧习惯,欲使之信服,不得不正统系。\\

七、共和制度日炽,加以失业人民日众,与日本帝国实有莫大之隐忧;若中国得以恢复帝制,于两国人民思想上。精神上保存至大,此不得不正统系。\\

八、大清在中华有二百余年之历史,(入关前)在满洲有一百余年之历史,从人民之习惯,安人民之心理,治地方之安靖,存东方之精神,行王政之复古,巩固贵国我国之皇统,不得不正统系。\\

九、贵国之兴隆,在明治大帝之王政。观其训谕群工,莫不推扬道德,教以忠义。科学兼采欧美,道德必本诸孔孟,保存东方固有之精神,挽回孺染欧风之弊习,故能万众人心亲上师长,保护国家,如手足之捍头目。此予之所敬佩者。为起步明治大帝,不能不正统系。\\

十、蒙古诸王公仍袭旧号,若行共和制度,欲取消其以前爵号,则因失望而人心涣散,更无由统制之,故不能不正统系。\\

十一、贵国扶助东三省,为三千万人民谋幸福,至可感佩。惟子之志愿,不仅在东三省之三千万人民,实欲以东三省为张本,而振兴全国之人心,以救民于水火,推至于东亚共存共荣,即贵国之九千万人民皆有息息相关之理,两国政体不得歧异。为振兴两国国势起见,不得不正统系。\\

十二、予自辛亥逊政,退处民间,今已二十年矣,毫无为一己尊崇之心,专以救民为宗旨。只要有人出而任天下之重,以正道挽回劫运,子虽为一平民,亦所欣愿。若必欲予承之,本个人之意见,非正名定分,实有用人行政之权,成一独立国家,不能挽回二十年来之弊政。否则有名无实,诸多牵制,毫无补救于民,如水益深,如火益热,徒负初心,更滋罪戾,此万万不敢承认者也。倘专为一己尊荣起见,则二十年来杜门削迹,一旦加之以土地人民,无论为总统,为王位,其所得已多,尚有何不足之念。\\
\end{quote}

实以所主张者纯为人民,纯为国家,纯为中日两国,纯为东亚大局起见,无一毫私利存乎其间,故不能不正统系。\\

郑孝胥知道,这次沈阳之行是决定自己命运的关键。因为关东军在叫东北行政委员会通过“国体”之前,要先排定一下“开国元勋”们的位置。因此,他在动身之前,对我尽量表示顺从,以免引起我对他发生戒心。但是等到他的目的已经达到,从沈阳返回来的时候,那情形就变了。他劝我不要和关东军争论,劝我接受共和制,出任“执政”。\\

“什么执政?叫我当共和国的执政?”我跳了起来。\\

“这事已成定局,臣再三向军方争论无效。军方表示,执政即元首……”\\

我不理他,转身问罗振玉,这是怎么回事。罗振玉说:“臣就见了板垣一面,是郑孝胥跟板垣谈的。”\\

后来据陈曾寿说,郑孝胥父子根本没把我的十二条“正统系”给板垣拿出来,而且还向板垣保证:“皇上的事,我全可以包下来,”“皇上如同一张白纸,你们军部怎么画都行,”等等。当时我还不知道这回事,只认为他们不会办事,都受了日本人的骗。\\

“你们都没用!”我大声喊道,“你们为什么不说,我的要求达不到,我就回天津!”\\

“皇上还是再三思考为好。”郑孝胥说,“复辟必须依赖日本,眼前与日本反目,将来的希望也完了。将来复辟不是没有希望呵!”\\

他又讲了一些历史故事,劝我答应,可是那些故事我早就听够了,再说无论是刘秀还是重耳,也都没有放弃君主称号的。最后他说:\\

“下午板垣就来觐见,请皇上对板垣说吧!”\\

“让他来!”我气呼呼地回答。\\